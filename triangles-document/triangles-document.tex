% article example for classicthesis.sty
\documentclass[10pt,a4paper]{article} % KOMA-Script article scrartcl
\usepackage{lipsum}
\usepackage{url}
\usepackage[nochapters]{classicthesis} % nochapters
\usepackage{graphicx}
\usepackage{float}
\usepackage{amsmath}
\usepackage{amsthm}
\usepackage{caption}

\newtheorem{theorem}{Theorem}[section]
\newtheorem{lemma}[theorem]{Lemma}
\newtheorem{conjecture}[theorem]{Conjecture}
\newtheorem{corollary}[theorem]{Corollary}
%\newtheorem{proof}[theorem]{Proof}

\begin{document}
    \title{\rmfamily\normalfont\spacedallcaps{Colouring Riordan arrays}}
    \author{\spacedlowsmallcaps{Donatella Merlini} \\ \spacedlowsmallcaps{Massimo Nocentini}}
    \date{\today} 
    
    \maketitle
    
    \begin{abstract}
        \noindent\lipsum[1] Just a test.\footnote{This is a footnote.}
    \end{abstract}
       
    \tableofcontents
    
    \section{Pascal}

    \subsection{Some modular proofs}
    
    In this section we present some proofs about Pascal array and its
    inverse, taken modulo some prime $p$. Colouring with different
    colours elements belonging to different remainder classes, we show
    formally that for $p$ even, ie. $p=2$, elements in the same
    position get the same colour (hence both triangle are coloured the
    same way, ignoring signs for elements in the inverse array), while
    for $p$ odd there is not such a simple correspondence and we
    attempt to observe some repeating pattern of mapping between
    elements.

    Let $\mathcal{P}$ be the Riordan array for the Pascal triangle,
    defined as:
    \begin{displaymath} 
        \mathcal{P} = \left(\frac{1}{1-t}, \frac{t}{1-t}  \right)
    \end{displaymath} 
    and let $\mathcal{P}^{-1}$ be its inverse Riordan array:
    \begin{displaymath} 
        \mathcal{P}^{-1} = \left(\frac{1}{1+t}, \frac{t}{1+t}  \right)
    \end{displaymath} 
    
    Let $d_{nk}$ and $\hat{d}_{nk}$ be the generic element of Pascal
    array and of its inverse, respectively. Since both arrays are
    Riordan, by definition:
    \begin{displaymath}
        \begin{split}
            d_{nk} &= [t^n]\frac{1}{1-t}\left(\frac{t}{1-t}\right)^k = [t^{n-k}](1-t)^{-(k+1)} \\
                &= {{-(k+1)} \choose {n-k}}(-1)^{n-k} = {{k+1 +n-k -1} \choose {n-k}} = {{n} \choose {n-k}} \\
        \end{split}
    \end{displaymath}
    and for the inverse:
    \begin{displaymath}
      \begin{split}
        \hat{d}_{nk} &= [t^n]\frac{1}{1+t}\left(\frac{t}{1+t}\right)^k = [t^{n-k}](1+t)^{-(k+1)} = 
        {{-(k+1)} \choose {n-k}} \\
        &= {{k+1 +n-k -1} \choose {n-k}} (-1)^{n-k} = {{n} \choose {n-k}} (-1)^{n-k}\\
      \end{split}
    \end{displaymath}
    Hence, equating binomial coefficients yields:
    \begin{displaymath}
      [t^n]\frac{1}{1-t}\left(\frac{t}{1-t}\right)^k = (-1)^{k-n}[t^n]\frac{1}{1+t}\left(\frac{t}{1+t}\right)^k 
    \end{displaymath}
    Choose a prime $p$ and take the modulo of both members:
    \begin{displaymath}
      [t^n]\frac{1}{1-t}\left(\frac{t}{1-t}\right)^k \equiv_{p} (-1)^{k-n}[t^n]\frac{1}{1+t}\left(\frac{t}{1+t}\right)^k 
    \end{displaymath}

    From now on we have to reason according modular arithmetic, hence
    multiply by a term $a$ both member of equations in order to
    simplify requests to show the existence of multiplicative inverse
    modulo $p$ of $a$, denoted by: $$a^{-1}\mod p$$

    First of all, observe that $-1 \equiv_{p} p-1$ and since $p$ is a
    prime by hp, it follows that $(p, p-1)=1$ (this result holds in
    general, not just for $p$ prime), which proofs the existence of
    both $-1$ and $p-1$ inverses, denoted by $(-1)^{-1}\mod p$ and
    $(p-1)^{-1}\mod p$ respectively.

    In order to find $(p-1)^{-1}\mod p$ we have to satisfy the
    congruence equation $(p-1) * (p-1)^{-1} \equiv_{p} 1$. Choose
    $(p-1)^{-1}\mod p = p-1$ and verify
    $(p-1) * (p-1) \equiv_{p} p^2 -2p +1 \equiv_{p} 1$ as required.

    Another useful observation concerns raising to negative powers,
    let $k \geq 0$:
    \begin{displaymath}
        (-1)^{-k} \equiv_{p} \left((-1)^{-1}\right)^{k} \equiv_{p} (p-1)^k
    \end{displaymath}
    and since $-1 \equiv_{p} p-1$ it follows that
    $(-1)^{-k} \equiv_{p} (-1)^k$, surprising to me.
    
    Now we can use previous observation to the main modular equation:
    \begin{displaymath}
        \begin{split}
            [t^n]\frac{1}{1-t}\left(\frac{t}{1-t}\right)^k 
                &\equiv_{p} (-1)^{k-n}[t^n]\frac{1}{1+t}\left(\frac{t}{1+t}\right)^k \\
                &\equiv_{p} (-1)^{k+n}[t^n]\frac{1}{1+t}\left(\frac{t}{1+t}\right)^k \\
                &\equiv_{p} (p-1)^{k+n}[t^n]\frac{1}{1+t}\left(\frac{t}{1+t}\right)^k \\
        \end{split}
    \end{displaymath}
    Hence, multiplying by $(p-1)^{-1}\mod p$ both members $k+n$ times:
    \begin{displaymath}
        (p-1)^{k+n}[t^n]\frac{1}{1-t}\left(\frac{t}{1-t}\right)^k \equiv_{p} [t^n]\frac{1}{1+t}\left(\frac{t}{1+t}\right)^k 
    \end{displaymath}
    which is the same as:
    \begin{displaymath}
        (-1)^{k+n}[t^n]\frac{1}{1-t}\left(\frac{t}{1-t}\right)^k \equiv_{p} [t^n]\frac{1}{1+t}\left(\frac{t}{1+t}\right)^k 
    \end{displaymath}
    and relating generic elements $d_{nk}$ with $\hat{d}_{nk}$:
    \begin{displaymath}
        (p-1)^{k+n}d_{nk}\equiv_{p}(-1)^{k+n}d_{nk} \equiv_{p} \hat{d}_{nk}
    \end{displaymath}

    \subsubsection{$p=2$}
    The case for $p$ even prime produce the colouring reported in
    REGENERATE AND PUT HERE THE RIGHT REFERENCE TO PASCAL: ignoring
    signs in the inverse array, we got the same colouring.  This is
    justified using the previous argument with $p=2$, for any choice
    of $n, k \geq 0$:
    \begin{displaymath} 
        d_{nk} \equiv_{2} \hat{d}_{nk} 
    \end{displaymath} 

    \subsubsection{$p=3$}
    Here there's a more interesting pattern to study, and in general,
    for any odd prime $p$, colourings of standard and inverse arrays
    mismatch. In this section we tackle the case for $p=3$,
    instantiating the modular equation:
    \begin{displaymath}
      2^{k+n}d_{nk}\equiv_{3}(-1)^{k+n}d_{nk} \equiv_{3} \hat{d}_{nk}
    \end{displaymath}
    For the sake of clarity, consider row 4th of both triangles:
    \begin{itemize}
    \item $\mathcal{P}[3,:] = (1 \quad 3 \quad 3 \quad 1) \equiv_{3}(1 \quad 0 \quad 0 \quad 1)$
    \item $\mathcal{P}^{-1}[3,:] = (-1 \quad 3 \quad -3 \quad 1) \equiv_{3}(2 \quad 0 \quad 0 \quad 1)$
    \end{itemize}
    Hence element $d_{30}$ gets a color $c$ while $\hat{d}_{30}$ gets
    a color $c'$ different from $c$.

    No general mapping among this relationship appears, so we
    carefully study two coloured triangles.  First fix a row which
    will be our reference row, choose index $3^4$. Repeatedly, move up
    one by one, and for each considered row move on the right over
    columns, here a modular equivalence going up one row:
    \begin{displaymath}
        \begin{split}
            d_{3^4 -1,0} &\equiv_{3} \hat{d}_{3^4 -1,0} \\
            d_{3^4 -1,1} &\equiv_{3} \hat{d}_{3^4 -2,0} \\
            d_{3^4 -1,2} &\equiv_{3} \hat{d}_{3^4 -3,0} \\
            d_{3^4 -1,3} &\equiv_{3} \hat{d}_{3^4 -4,0} \\
            &\vdots
        \end{split}
    \end{displaymath}
    next, a modular equivalence going up two rows:
    \begin{displaymath}
        \begin{split}
            d_{3^4 -2,0} &\equiv_{3} \hat{d}_{3^4 -1,1} \\
            d_{3^4 -2,1} &\equiv_{3} \hat{d}_{3^4 -2,1} \\
            d_{3^4 -2,2} &\equiv_{3} \hat{d}_{3^4 -3,1} \\
            d_{3^4 -2,3} &\equiv_{3} \hat{d}_{3^4 -4,1} \\
            &\vdots
        \end{split}
    \end{displaymath}
    next, a modular equivalence going up three rows:
    \begin{displaymath}
        \begin{split}
            d_{3^4 -3,0} &\equiv_{3} \hat{d}_{3^4 -1,2} \\
            d_{3^4 -3,1} &\equiv_{3} \hat{d}_{3^4 -2,2} \\
            d_{3^4 -3,2} &\equiv_{3} \hat{d}_{3^4 -3,2} \\
            d_{3^4 -3,3} &\equiv_{3} \hat{d}_{3^4 -4,2} \\
            &\vdots
        \end{split}
    \end{displaymath}
    Let introduce variable $b$, running over rows, and $a$, running
    over columns; a pattern appears, here it is:
    \begin{displaymath}
            d_{3^4 +b,a} \equiv_{3} \hat{d}_{3^4 -1-a,-1-b} 
    \end{displaymath}

    Let's say, assume the colouring for $\mathcal{P}^{-1}$ triangle is
    given, the colouring for row $26$ of $\mathcal{P}$ is desired. It
    is necessary to find $b$: by $3^4 +b=26$ get $b=-55$, so the
    required row satisfy the following modular equivalence:
    \begin{displaymath}
            d_{26,a} \equiv_{3} \hat{d}_{80-a,54} 
    \end{displaymath}
    This is pretty curious since in order to colour a row in triangle
    $\mathcal{P}$, from left to right since for proper arrays
    $a \geq 0$, column's colouring in triangle $\mathcal{P}^{-1}$ is
    used, from bottom to top\footnote{Some pictures highlighting the
      interested rows and columns should be very helpful}.
    \\\\
    A question is still open: why do we choose row $3^4$ as reference
    row?

    It is useful to recall a theorem due to Fine:
    \begin{theorem}
      A necessary and sufficient condition for a binomial coefficient
      ${{n} \choose {m}}$ to be divisible by a prime $p$ is that $n$
      be a power of $p$.
    \end{theorem}
    Consider the colouring for triangle $\mathcal{P}$, we can use the
    given theorem to point out ``interesting'' rows, namely those rows
    affected by the theorem, they correspond to powers
    $3^1, 3^2, 3^3, 3^4, \ldots$, each one of them can be easily
    recognized since dots lying on it have all the same colour. In the
    triangle $127$ former rows are drawn and in order to have ``more
    space'' to find a modular relationship between $d_{nk}$ and
    $\hat{d}_{nk}$ we choose as \emph{reference} row the one with
    index $3^4$.  From here we start moving backwards by rows toward
    the root: observe that the entire row $3^4 -1$, containing
    $3^4 -1$ remainders, of triangle $\mathcal{P}$ is the first
    segment of the first column of $\mathcal{P}^{-1}$, in other words
    $d_{3^4 -1,a} \equiv_{3} \hat{d}_{3^4 -1 -a, 0}$ for
    $a \in \lbrace 0, \ldots, 3^4 -1\rbrace$.

    It seems that coefficient $d_{3^4-1,0}$ acts as a pivot on which
    the triangle ``flips'': the root moves toward the reader while the
    bottom edge moves toward opposite the reader. This rigid motion is
    captured by the following modular relationships among three
    important points:
    \begin{displaymath}
        \begin{split}
            d_{3^4 -1,0} &\equiv_{3} \hat{d}_{3^4 -1,0} \\
            d_{3^4 -1,3^4 -1} &\equiv_{3} \hat{d}_{0,0} \\
            d_{3^4 -3^3,3^3-1} &\equiv_{3} \hat{d}_{3^4 -3^3,3^3-1} \\
        \end{split}
    \end{displaymath}

    \subsection{Sierpinski structure}

    In this section we show that in a Pascal array $\mathcal{P}$ it is
    possible to recognize a structure as Sierpinski observed using
    fractals. More precisely, we show that, choosen a prime $p$ and a
    height $n$, the upper triangle that starts at root and extends
    downward for $p^n$ rows, denoted by $\mathcal{P}_n$, repeats
    itself three times in a bigger triangle that extends from the root
    downward for $2p^{n}$ rows, surrounding an upside-down triangle of
    coefficients, say $c_{nk}$ is one of those, such that are divisible
    by $p$ all, formally $c_{nk} \equiv_p 0$. Note that for even prime
    $p$, triangle $\mathcal{P}_n$ repeats itself $p+1$ times in
    $\mathcal{P}_{n+1}$, exactly as it is.
    
    On the other hand, for an odd prime $p$ the structure loses
    symmetry. In this case triangle $\mathcal{P}_n$ repeats itself in
    $\mathcal{P}_{n+1}$ somewhere, without a plain regularity due to
    modulo $p$, ie. other coloured triangles appears in
    $\mathcal{P}_{n+1}$.  However if we study the repetition up to row
    $2p^n$ the proof still hold, while in general we can say that
    triangles of coefficients multiples of $p$ appear with regularity
    in $\mathcal{P}_{n+1}$ and the number of repetition is:
    \begin{displaymath}
        \sum_{i=1}^{p-1}{i} = \frac{(p-1)p}{2}
    \end{displaymath}
    in other words, considering triangles $\mathcal{P}_{n}$ and
    $\mathcal{P}_{n+1}$, there are $\frac{(p-1)p}{2}$ upside-down
    triangles, with all coefficients multiple of $p$, from row $p^n$
    to row $p^{n+1}-1$.

    \begin{proof}
      Let $\mathcal{P}_n$ be the triangle that starts at the root and
      extends downward $p^n$ rows, hence a row index $r$ for
      $\mathcal{P}_n$ satisfies
      $r \in \lbrace 0, \ldots, p^n -1 \rbrace$.
        
      Now consider a bigger triangle $\mathcal{P}_n^\prime$ that
      starts at the root and extends downward $2p^n$ rows, it is
      requested to prove that coefficients in \emph{equivalent
        positions} in the bottom left and bottom right triangles are
      congruent, modulo $p$, to the coefficient in equivalent position
      in $\mathcal{P}_n$. In order to formalize the concept of
      \emph{equivalent positions} think as follow: let
      $d_{rc}\in\mathcal{P}_n$, since $\mathcal{P}_{n}$ has $p^n$ rows
      and on the vary last row (ie, the one with index $p^n-1$) lay
      $p^n$ coefficients (ie, there are $p^n$ columns), the
      coefficient at \emph{equivalent position} in the bottom left
      triangle of $\mathcal{P}_{n}^\prime$ is denoted by
      $d_{p^n+r, c}^{\swarrow}$; on the other hand, the coefficient at
      \emph{equivalent position} in the bottom right triangle of
      $\mathcal{P}_{n}^\prime$ is denoted by
      $d_{p^n+r, p^n+c}^{\searrow}$.  Formal settings described, we've
      to prove:
      \begin{displaymath}
        d_{rc} \equiv_p d_{p^n+r,c}^{\swarrow} \equiv_p d_{p^n+r,p^n+c}^{\searrow} 
      \end{displaymath}
      or, in other words:
      \begin{displaymath}
        {{r} \choose {c}} \equiv_p {{p^n+r} \choose {c}} \equiv_p {{p^n+r} \choose {p^n+c}} 
      \end{displaymath}

      In order to prove such congruences we'll use Lucas theorem:
      first of all, observe that $c \leq r$ since $\mathcal{P}$ is a
      triangle, therefore $c \in \lbrace 0, \ldots, p^n -1 \rbrace$,
      as $r$ satisfies.  By basis representation theorem, there exists
      sequences $\lbrace r_i\rbrace$ and $\lbrace c_i\rbrace$, for
      $i \in \lbrace 0, \ldots, p^n -1 \rbrace$, with $r_i,c_i < p$,
      such that:
      \begin{displaymath}
        \begin{split}
          r &= r_0 + r_1 p + r_2 p^2 + \ldots + r_{n-1}p^{n-1} \\
          c &= c_0 + c_1 p + c_2 p^2 + \ldots + c_{n-1}p^{n-1} \\
        \end{split}
      \end{displaymath}
      Settings for Lucas theorem are ready, hence apply it:
      \begin{displaymath}
        \begin{split}
          {{p^n+r} \choose {c}} &\equiv_{p} {{r_0} \choose {c_0}} {{r_1} \choose {c_1}}{{r_2} \choose {c_2}} \ldots 
          {{r_{n-1}} \choose {c_{n-1}}}{{1} \choose {0}} \equiv_{p} {{r} \choose {c}}\\
          {{p^n+r} \choose {p^n+c}} &\equiv_{p} {{r_0} \choose {c_0}} {{r_1} \choose {c_1}}{{r_2} \choose {c_2}} \ldots 
          {{r_{n-1}} \choose {c_{n-1}}}{{1} \choose {1}} \equiv_{p} {{r} \choose {c}}\\
        \end{split}
      \end{displaymath}
      congruences holds, therefore coefficients located at
      \emph{equivalent positions} belong to the same remainder class,
      modulo $p$, as required.
      \\\\
      For the second part of the statement, we've to show that in
      $\mathcal{P}_{n}^\prime$ an upside-down triangle of
      coefficients, each one of them can by divided by $p$, is
      surrounded by the three ``congruent'' triangles discussed in the
      first part of this proof.  Observe that the very first
      coefficient $d_{p^n, 0}$ and very last $d_{p^n, p^n}$ of row
      with index $p^n$ are congruent to the unit, modulo $p$:
      \begin{displaymath}
        {{p^n} \choose {0}} \equiv_{p}{{p^n} \choose {p^n}} \equiv_{p} 1
      \end{displaymath}
      while $p$ divides every coefficient between them, let
      $c\in\lbrace1,\ldots, p^n-1 \rbrace$:
      \begin{displaymath}
        {{p^n} \choose {c}} \equiv_{p} {{0} \choose {c_0}} {{0} \choose {c_1}}{{0} \choose {c_2}} \ldots 
        {{0} \choose {c_{n-1}}}{{1} \choose {0}} \equiv_{p} 0
      \end{displaymath}
      By the recurrence rule $d_{n+1, k+1} = d_{n, k} + d_{n, k+1}$
      characterizing $\mathcal{P}$, observe that:
      \begin{displaymath}
        \begin{split}
          d_{p^n+1, 1} &\equiv_{p} d_{p^n, 0} + d_{p^n, 1}\equiv_{p} 1 \\
          d_{p^n+1, p^n} &\equiv_{p} d_{p^n, p^n-1} + d_{p^n, p^n}\equiv_{p} 1 \\
          d_{p^n+1, i} &\equiv_{p} d_{p^n, i-1} + d_{p^n, i}\equiv_{p} 0 \quad \forall i \in \lbrace 2, \ldots, p^n -1\rbrace \\
        \end{split}
      \end{displaymath}
      therefore row $p^n + 1$ has one coefficient multiple of $p$ less
      than row $p^n$, and since in $p^n$ there are $p^n-2$ such
      coefficients, after $p^n-2$ rows there are no such coefficients
      at all, such a row has index $2(p^n -1)+1$ ($1$ more because
      indexes are $0$-based). Formally, for any column index $c$:
      \begin{displaymath}
        \begin{split}
          {{2p^n - 1} \choose {c}} &\equiv_{p} {{p^n +(p^n- 1)} \choose {c}} \\
          &\equiv_{p} {{p-1} \choose {c_0}} {{p-1} \choose {c_1}}{{p-1} \choose {c_2}} \ldots 
          {{p-1} \choose {c_{n-1}}}{{1} \choose {0}} \\
          &\not\equiv_{p} 0
        \end{split}
      \end{displaymath}
      The last step holds by representation of $c$:
      $c_i \in \lbrace 0, \ldots, p-1 \rbrace$, for any $i$.

      The latter argument can be applied to every row with index of
      the form $p^k -1$, for some $k\geq n$:
      \begin{displaymath}
        \begin{split}
          {{p^k-1} \choose {c}} &\equiv_{p} {{p-1} \choose {c_0}} {{p-1} \choose {c_1}} \ldots 
          {{p-1} \choose {c_{n-1}}}{{p-1} \choose {0}}\ldots{{p-1} \choose {c_{k-1}=0}} \not\equiv_{p} 0
        \end{split}
      \end{displaymath}

      % the following derivation is simply a proof of coefficient extraction
      % using the definition of Riordan array for Pascal triangle, redundant.
      % Recall $\mathcal{P}$ is defined as the Riordan array :
      % \begin{displaymath}
      %   \mathcal{P} = \left(\frac{1}{1-t}, \frac{t}{1-t}  \right)
      % \end{displaymath}
      % hence the following derivation holds:
      % \begin{displaymath}
      %   \begin{split}
      %     {{p^n+r} \choose {c}} &\equiv_p [t^{p^n +r}]\frac{1}{1-t} \left(\frac{t}{1-t}\right)^c \\
      %     &\equiv_p [t^{p^n +r-c}](1-t)^{-(c+1)} \\
      %     &\equiv_p [t^{p^n +r-c}]\mathcal{G}\left\lbrace {{-(c+1)} \choose {k}}(-1)^k \right\rbrace_{k\in\mathbb{N}} \\
      %     &\equiv_p  {{-(c+1)} \choose {p^n +r-c}}(-1)^{p^n +r-c}  \\
      %     &\equiv_p  {{ p^n +r} \choose {p^n +r-c}} \left((-1)^{p^n +r-c}\right)^2  \\
      %     &\equiv_p  {{ p^n +r} \choose {c}}  \\
      %   \end{split}
      % \end{displaymath}

    \end{proof}

    We present a modular characterization under the theory of Riordan arrays, using the
    congruence modulo a prime $p$. Choose $n\in\mathbb{N}$ and let $\mathcal{P}_n$ denote the
    same portion of $\mathcal{P}$ as in the preceding proof:
      \begin{displaymath}
          \begin{split}
                {{r} \choose {c}} &\equiv_p {{p^n+r} \choose {c}} \\
                {{r} \choose {c}}t^r &\equiv_p {{p^n+r} \choose {c}}t^r \\
                \sum_{r\geq 0}{{{r} \choose {c}}t^r} &\equiv_p \sum_{r\geq 0}{{{p^n+r} \choose {c}}t^r} \\
                d(t)h(t)^{c} &\equiv_p \sum_{k\geq p^n}{{{k} \choose {c}}t^{k-p^n}} \\
                d(t)h(t)^{c} &\equiv_p t^{-p^n}\sum_{k\geq p^n}{{{k} \choose {c}}t^{k}} \\
                d(t)h(t)^{c} &\equiv_p t^{-p^n}\left(
                    \sum_{k\geq c}{{{k} \choose {c}}t^{k}}-\sum_{k=c}^{p^n -1}{{{k} \choose {c}}t^{k}}\right) \\
                d(t)h(t)^{c} &\equiv_p t^{-p^n}\left(
                    \sum_{k\geq 0}{{{k} \choose {c}}t^{k}}-\sum_{k=c}^{p^n -1}{{{k} \choose {c}}t^{k}}\right) \\
                d(t)h(t)^{c} &\equiv_p t^{-p^n}\left(d(t)h(t)^{c} -\sum_{k=c}^{p^n -1}{{{k} \choose {c}}t^{k}}\right) \\
          \end{split}
      \end{displaymath}
    Now assume $c \geq p^n$ in order to make the sum vanish:
      \begin{displaymath}
          \begin{split}
                d(t)h(t)^{c} &\equiv_p t^{-p^n} d(t)h(t)^{c} \\
                d(t)h(t)^{c} &\equiv_p d(t)h(t)^{c-p^n} (1-t)^{-p^n} \\
          \end{split}
      \end{displaymath}
    Since in general the inverse $\left((1-t)^{-p^n}\right)^{-1}\mod p$ is not $(1-t)^{p^n}$
    (we should prove it), let us proceed with algebraic manipulation:
      \begin{displaymath}
          \begin{split}
                d(t)h(t)^{c} &\equiv_p d(t)h(t)^{c-p^n}\left( (1-t)^{p^n}\right)^{-1} \\
                d(t)h(t)^{c} &\equiv_p d(t)h(t)^{c-p^n}\left(\sum_{k=0}^{p^n}{{{p^n} \choose {k}}(-t)^k } \right)^{-1}\\
                d(t)h(t)^{c} &\equiv_p d(t)h(t)^{c-p^n}\left(
                    1 + \sum_{k=1}^{p^n -1}{{{p^n} \choose {k}}(-t)^k +(-t)^{p^n} }\right)^{-1} \\
          \end{split}
      \end{displaymath}
    By Lucas theorem the inner sum vanish, since $k\in\lbrace 1, \ldots, p^n -1 \rbrace$:
      \begin{displaymath}
        \begin{split}
          {{p^n} \choose {k}} &\equiv_{p} {{0} \choose {k_0}} {{0} \choose {k_1}} \ldots 
          {{0} \choose {k_{n-1}}}{{1} \choose {0}}\equiv_{p} 0 
        \end{split}
      \end{displaymath}
    Therefore:
      \begin{displaymath}
          \begin{split}
                d(t)h(t)^{c} &\equiv_p d(t)h(t)^{c-p^n}\left(1 + (-t)^{p^n}\right)^{-1} \\
          \end{split}
      \end{displaymath}
    And assuming $p$ odd:
      \begin{displaymath}
          \begin{split}
                d(t)h(t)^{c} &\equiv_p \frac{d(t)h(t)^{c-p^n}}{1 - t^{p^n}} \\
          \end{split}
      \end{displaymath}

    \subsection{Using binomial symmetries}

    \begin{corollary}
    Let $\mathcal{P}$ be the Pascal array and let $p$ be a prime, than $k$-th column 
    is congruent to $k$-antidiagonal, modulo $p$.
    \end{corollary}
    \begin{proof}
        \begin{displaymath}
            \begin{split}
                { {n} \choose {k} } &\equiv_{p} { {n} \choose {n-k} } \\
                d_{nk} &\equiv_{p} d_{n,n-k}\\
                \sum_{n\geq 0}{d_{nk} t^n} &\equiv_{p}\sum_{n\geq 0}{d_{n,n-k} t^n} \\
            \end{split}
        \end{displaymath}
    \end{proof}

    \begin{lemma}
        Let $p$ be a prime and take any $m, \gamma \in \mathbb{N}$. Let $n$ index
        over rows of a Pascal array $\mathcal{P}$, such that $p^{m} \leq n < p^{m+1}$, then the following holds:
        \begin{displaymath}
            d_{n,n-p^{m}} \equiv_{p} d_{n+\gamma p^{m+1}, n-p^{m}}
        \end{displaymath}
    \end{lemma}
    \begin{proof} % $n+\gamma p^{m+1}$
        By the basis representation theorem, write $n$  in base $p$ upto $k$ such that $k > m+1$ as necessary:
        \begin{displaymath}
            n = n_{0} + n_{1}p + n_{2}p^2 + \ldots + n_{m}p^m + 0p^{m+1} + \ldots + 0p^k
        \end{displaymath}
        Now consider the following quantities:
        \begin{displaymath}
            \begin{split}
                n +\gamma p^{m+1} &= n_{0} + n_{1}p + n_{2}p^2 + \ldots + n_{m}p^m + \gamma p^{m+1} + 0p^{m+2} \ldots + 0p^k \\
                n - p^{m} &= n_{0} + n_{1}p + n_{2}p^2 + \ldots + (n_{m}-1)p^m + 0p^{m+1} + \ldots + 0p^k \\
            \end{split}
        \end{displaymath}
        By Lucas theorem both the congruence:
        \begin{displaymath}
            {{n+\gamma p^{m+1}} \choose { n - p^{m}}} \equiv_{p} 
                {{n_{0}} \choose {n_{0}}}  
                {{n_{1}} \choose {n_{1}}} 
                {{n_{2}} \choose {n_{2}}}
                \ldots
                {{n_{m}} \choose {n_{m}-1}} 
                {{\gamma} \choose {0}} 
        \end{displaymath}
        both the following one:
        \begin{displaymath}
            {{n} \choose { n - p^{m}}} \equiv_{p} 
                {{n_{0}} \choose {n_{0}}}  
                {{n_{1}} \choose {n_{1}}} 
                {{n_{2}} \choose {n_{2}}}
                \ldots
                {{n_{m}} \choose {n_{m}-1}} 
        \end{displaymath}
        hold. Assuming ${{0}\choose{0}} = {{k}\choose{0}} = 1$ for any $k\in\mathbb{N}$, 
        transitivity property of $\equiv_p$ relation over right sides of
        the previous two congruences, it allows to derive congruence:
        \begin{displaymath}
            {{n} \choose { n - p^{m}}} \equiv_{p} {{n+\gamma p^{m+1}} \choose { n - p^{m}}} 
        \end{displaymath}
        which is the requested relation.
    \end{proof}

    Previous lemma is important because it does hold for the inverse array $\mathcal{P}^{-1}$ too:
    \begin{displaymath}
        \begin{split}
            \hat{d}_{n,n-p^{m}} &\equiv_{p} \hat{d}_{n+\gamma p^{m+1}, n-p^{m}} \\
            (-1)^{n-(n-p^{m})}d_{n,n-p^{m}} &\equiv_{p} (-1)^{n+\gamma p^{m+1}-(n-p^{m})}d_{n+\gamma p^{m+1}, n-p^{m}} \\
            (-1)^{p^{m}}d_{n,n-p^{m}} &\equiv_{p} (-1)^{p^{m}(\gamma p+1)}d_{n+\gamma p^{m+1}, n-p^{m}} \\
            (-1)^{p m}d_{n,n-p^{m}} &\equiv_{p} (-1)^{p{m}(\gamma p+1)}d_{n+\gamma p^{m+1}, n-p^{m}} \\
            (-1)^{p}d_{n,n-p^{m}} &\equiv_{p} (-1)^{p(\gamma p+1)}d_{n+\gamma p^{m+1}, n-p^{m}} \\
            (-1)^{p}d_{n,n-p^{m}} &\equiv_{p} (-1)^{p^2\gamma}(-1)^p d_{n+\gamma p^{m+1}, n-p^{m}} \\
            (-1)^{p}d_{n,n-p^{m}} &\equiv_{p} (-1)^{2p \gamma}(-1)^p d_{n+\gamma p^{m+1}, n-p^{m}} \\
            (-1)^{p}d_{n,n-p^{m}} &\equiv_{p} (-1)^p d_{n+\gamma p^{m+1}, n-p^{m}} \\
        \end{split}
    \end{displaymath}
    An even module $p$ produces exactly the same remainder classes both for $\mathcal{P}$ 
    both for $\mathcal{P}^{-1}$. Suppose $p$ odd, in this case $(-1)^{p} = -1$ and since there
    exists $(-1)^{-1}\mod p$:
    \begin{displaymath}
        \begin{split}
            d_{n,n-p^{m}} &\equiv_{p} d_{n+\gamma p^{m+1}, n-p^{m}} \\
        \end{split}
    \end{displaymath}
    which holds by the former lemma. Pay attention: this doesn't relate coefficients
    of $\mathcal{P}$ and $\mathcal{P}^{-1}$ (from the colouring point of view 
    doesn't imply any relation about colours assignment: generally it is not the case,
    except for even $p$), it merely says that 
    congruent coefficients on the choosen antidiagonal, \emph{in the same triangle
    either $\mathcal{P}$ or $\mathcal{P}^{-1}$}, repeat with structure.
    \\\\
    The following theorem tackle what the above argument leaves out: it 
    shows another caracterization, we call it ``duality'', 
    over remainder classes for arrays $\mathcal{P}$ and its inverse, relating coefficients
    belonging to the two triangles.
    \begin{theorem}
        Let $d_{nk}, \hat{d}_{nk}$ be generic elements of arrays $\mathcal{P}$ and $\mathcal{P}^{-1}$,
        respectively. Choose an odd prime $p$, let $c\in \lbrace 0, \ldots, p-1 \rbrace$ a remainder class
        witness, then:
        \begin{displaymath}
            \begin{split}
                d_{n,n-p^{m}} \equiv_{p} c &\leftrightarrow \hat{d}_{n,n-p^{m}} \equiv_{p} p-c
            \end{split}
        \end{displaymath}
    \end{theorem}
    \begin{proof}
    We show both directions using a set of congruences: reading them from top to bottom provides a 
    proof for $\rightarrow$ direction, while reading them from bottom to top provides a proof for 
    $\leftarrow$ direction. Recall that $p^m$ is odd because $p$ is odd by hp, therefore $(-1)^{p^m} = -1$:
    \begin{displaymath}
        \begin{split}
            d_{n,n-p^{m}} &\equiv_{p} c \\
            (-1)^{p^m}\hat{d}_{n,n-p^{m}} &\equiv_{p} c \\
            (-1)^{p^m }(-1)\hat{d}_{n,n-p^{m}} &\equiv_{p} -c \\
            (-1)^{p^m }(-1)\hat{d}_{n,n-p^{m}} &\equiv_{p} p -c \\
            \hat{d}_{n,n-p^{m}} &\equiv_{p} p -c \\
        \end{split}
    \end{displaymath}
    \end{proof}

    \subsection{Expanded arrays and coloured triangles}

    % mod 2
    Expansion of Riordan array $\mathcal{P}$:
    

\begin{table}
    \begin{displaymath} 
        \hspace{-5cm}
        \mathcal{P}_{10}\left(\begin{array}{rrrrrrrrrr}
        1 & 0 & 0 & 0 & 0 & 0 & 0 & 0 & 0 & 0 \\
        1 & 1 & 0 & 0 & 0 & 0 & 0 & 0 & 0 & 0 \\
        1 & 2 & 1 & 0 & 0 & 0 & 0 & 0 & 0 & 0 \\
        1 & 3 & 3 & 1 & 0 & 0 & 0 & 0 & 0 & 0 \\
        1 & 4 & 6 & 4 & 1 & 0 & 0 & 0 & 0 & 0 \\
        1 & 5 & 10 & 10 & 5 & 1 & 0 & 0 & 0 & 0 \\
        1 & 6 & 15 & 20 & 15 & 6 & 1 & 0 & 0 & 0 \\
        1 & 7 & 21 & 35 & 35 & 21 & 7 & 1 & 0 & 0 \\
        1 & 8 & 28 & 56 & 70 & 56 & 28 & 8 & 1 & 0 \\
        1 & 9 & 36 & 84 & 126 & 126 & 84 & 36 & 9 & 1
        \end{array}\right) 
        \quad
        \mathcal{P}_{10}^{-1}\left(\begin{array}{rrrrrrrrrr}
        1 & 0 & 0 & 0 & 0 & 0 & 0 & 0 & 0 & 0 \\
        -1 & 1 & 0 & 0 & 0 & 0 & 0 & 0 & 0 & 0 \\
        1 & -2 & 1 & 0 & 0 & 0 & 0 & 0 & 0 & 0 \\
        -1 & 3 & -3 & 1 & 0 & 0 & 0 & 0 & 0 & 0 \\
        1 & -4 & 6 & -4 & 1 & 0 & 0 & 0 & 0 & 0 \\
        -1 & 5 & -10 & 10 & -5 & 1 & 0 & 0 & 0 & 0 \\
        1 & -6 & 15 & -20 & 15 & -6 & 1 & 0 & 0 & 0 \\
        -1 & 7 & -21 & 35 & -35 & 21 & -7 & 1 & 0 & 0 \\
        1 & -8 & 28 & -56 & 70 & -56 & 28 & -8 & 1 & 0 \\
        -1 & 9 & -36 & 84 & -126 & 126 & -84 & 36 & -9 & 1
        \end{array}\right) 
    \end{displaymath}

  \caption[$\mathcal{P}$ and $\mathcal{P}^{-1}$]{Two $10$-minors of
  $\mathcal{P}$ and $\mathcal{P}^{-1}$ matrix expansions, respectively}

  \label{tab:pascal:array} 
  
  \end{table}



    Expansion of Riordan array $\mathcal{P}^{-1}$:
    \input{../sympy/pascal/pascal-inverse-ignore-negatives-centered-colouring-127-rows-mod2-partitioning-include-matrix.tex}

    

\begin{figure}[p]
    
    \hspace{-1.5cm}
    \noindent\makebox[\textwidth]{
        \centering
        %\includegraphics[width=0.8\textwidth]{../../sympy/catalan/coloured.pdf}

        % using *angle* property to rotate it is difficult to properly align it
        % in order to have a "real" matrix representation.
        \includegraphics[width=20cm, height=20cm, keepaspectratio=true]{../sympy/pascal/pascal-standard-ignore-negatives-centered-colouring-127-rows-mod2-partitioning-triangle.pdf}
    }

    % this 'particular' line is necessary to use `displaymath' environment
    % into the caption environment, togheter with the inclusion of 
    % `caption' package. See here for more explanation:
    % http://stackoverflow.com/questions/2716227/adding-an-equation-or-formula-to-a-figure-caption-in-latex
    \captionsetup{singlelinecheck=off}
    \caption[$\mathcal{P}_{\equiv_{2}}$]{
        $\mathcal{P}_{\equiv_{2}}$
        \iffalse
        Pascal triangle, formally: 
        \begin{displaymath}
            \mathcal{P}=\left(\frac{1}{1-t}, \frac{t}{1-t}\right)
        \end{displaymath} % \newline % new line no more necessary
        standard, ignore negatives, centered colouring, 127 rows, mod2 partitioning
        \fi
        }

    \label{fig:pascal-standard-ignore-negatives-centered-colouring-127-rows-mod2-partitioning-triangle}

\end{figure}

    
\begin{figure}[p]

    \noindent\makebox[\textwidth]{
        \centering
        %\includegraphics[width=0.8\textwidth]{../../sympy/catalan/coloured.pdf}

        % using *angle* property to rotate it is difficult to properly align it
        % in order to have a "real" matrix representation.
        \includegraphics[width=20cm, height=20cm, keepaspectratio=true]{../sympy/pascal/pascal-inverse-ignore-negatives-centered-colouring-127-rows-mod2-partitioning-triangle.pdf}
    }

    % this 'particular' line is necessary to use `displaymath' environment
    % into the caption environment, togheter with the inclusion of 
    % `caption' package. See here for more explanation:
    % http://stackoverflow.com/questions/2716227/adding-an-equation-or-formula-to-a-figure-caption-in-latex
    \captionsetup{singlelinecheck=off}
    \caption[$\mathcal{P}_{\equiv_{2}}^{-1}$, ignore negative entries]{
        Pascal triangle, formally: 
        \begin{displaymath}
            \mathcal{P}^{-1}=\left(\frac{1}{t + 1}, \frac{t}{t + 1}\right)
        \end{displaymath} % \newline % new line no more necessary
        inverse, ignore negatives, centered colouring, 127 rows, mod2 partitioning}

    \label{fig:pascal-inverse-ignore-negatives-centered-colouring-127-rows-mod2-partitioning-triangle}

\end{figure}


    % mod 3
    % the following matrices are identical to the previous ones
    %\input{../sympy/pascal/pascal-standard-ignore-negatives-centered-colouring-127-rows-mod3-partitioning-include-matrix.tex}
    %\begin{displaymath} \left(\begin{array}{rrrrrrrrrr}
1 & 0 & 0 & 0 & 0 & 0 & 0 & 0 & 0 & 0 \\
-1 & 1 & 0 & 0 & 0 & 0 & 0 & 0 & 0 & 0 \\
1 & -2 & 1 & 0 & 0 & 0 & 0 & 0 & 0 & 0 \\
-1 & 3 & -3 & 1 & 0 & 0 & 0 & 0 & 0 & 0 \\
1 & -4 & 6 & -4 & 1 & 0 & 0 & 0 & 0 & 0 \\
-1 & 5 & -10 & 10 & -5 & 1 & 0 & 0 & 0 & 0 \\
1 & -6 & 15 & -20 & 15 & -6 & 1 & 0 & 0 & 0 \\
-1 & 7 & -21 & 35 & -35 & 21 & -7 & 1 & 0 & 0 \\
1 & -8 & 28 & -56 & 70 & -56 & 28 & -8 & 1 & 0 \\
-1 & 9 & -36 & 84 & -126 & 126 & -84 & 36 & -9 & 1
\end{array}\right) \end{displaymath}

    \input{../sympy/pascal/pascal-standard-ignore-negatives-centered-colouring-127-rows-mod3-partitioning-include-figure.tex}
    
\begin{figure}[p]

    \noindent\makebox[\textwidth]{
        \centering
        %\includegraphics[width=0.8\textwidth]{../../sympy/catalan/coloured.pdf}

        % using *angle* property to rotate it is difficult to properly align it
        % in order to have a "real" matrix representation.
        \includegraphics[width=20cm, height=20cm, keepaspectratio=true]{Chapters/mod-p-characterization/pascal/pascal-inverse-ignore-negatives-centered-colouring-127-rows-mod3-partitioning-triangle.pdf}
    }

    % this 'particular' line is necessary to use `displaymath' environment
    % into the caption environment, togheter with the inclusion of 
    % `caption' package. See here for more explanation:
    % http://stackoverflow.com/questions/2716227/adding-an-equation-or-formula-to-a-figure-caption-in-latex
    \captionsetup{singlelinecheck=off}
    \caption[.]{
        Pascal triangle, formally: 
        \begin{displaymath}
            \mathcal{P}^{-1}=\left(\frac{1}{t + 1}, \frac{t}{t + 1}\right)
        \end{displaymath} % \newline % new line no more necessary
        inverse, ignore negatives, centered colouring, 127 rows, mod3 partitioning}

    \label{fig:pascal-inverse-ignore-negatives-centered-colouring-127-rows-mod3-partitioning-triangle}

\end{figure}


    % mod 5
    % the following matrices are identical to the previous ones
    %\input{../sympy/pascal/pascal-standard-ignore-negatives-centered-colouring-127-rows-mod5-partitioning-include-matrix.tex}
    %\input{../sympy/pascal/pascal-inverse-ignore-negatives-centered-colouring-127-rows-mod5-partitioning-include-matrix.tex}

    \input{../sympy/pascal/pascal-standard-ignore-negatives-centered-colouring-127-rows-mod5-partitioning-include-figure.tex}
    \input{../sympy/pascal/pascal-inverse-ignore-negatives-centered-colouring-127-rows-mod5-partitioning-include-figure.tex}

    % the following two rows refer to old triangles, pay attention on the filename schema.
    \input{../sympy/pascal/Pascal-standard-handle-negatives-plain-colouring-127-rows-mod2-partitioning-include-figure.tex}
    \input{../sympy/pascal/Pascal-inverse-handle-negatives-plain-colouring-127-rows-mod2-partitioning-include-figure.tex}

    \section{Fibonacci}

    Expansion of Riordan array $\mathcal{F}$:
    \input{../sympy/fibonacci/Fibonacci-standard-handle-negatives-centered-colouring-127-rows-mod2-partitioning-include-matrix.tex}
    Expansion of Riordan array $\mathcal{F}^{-1}$:
    \input{../sympy/fibonacci/Fibonacci-inverse-handle-negatives-centered-colouring-127-rows-mod2-partitioning-include-matrix.tex}

    \input{../sympy/fibonacci/Fibonacci-standard-handle-negatives-centered-colouring-127-rows-mod2-partitioning-include-figure.tex}
    \input{../sympy/fibonacci/Fibonacci-inverse-handle-negatives-centered-colouring-127-rows-mod2-partitioning-include-figure.tex}

    \input{../sympy/fibonacci/Fibonacci-standard-handle-negatives-plain-colouring-127-rows-mod2-partitioning-include-figure.tex}
    \input{../sympy/fibonacci/Fibonacci-inverse-handle-negatives-plain-colouring-127-rows-mod2-partitioning-include-figure.tex}

    \section{Catalan}

    \subsection{Traditional version}

    Expansion of Riordan array $\mathcal{C}$:
    \input{../sympy/catalan/Catalan-traditional-standard-handle-negatives-centered-colouring-127-rows-mod2-partitioning-include-matrix.tex}
    Expansion of Riordan array $\mathcal{C}^{-1}$:
    \input{../sympy/catalan/Catalan-traditional-inverse-handle-negatives-centered-colouring-127-rows-mod2-partitioning-include-matrix.tex}

    \input{../sympy/catalan/Catalan-traditional-standard-handle-negatives-centered-colouring-127-rows-mod2-partitioning-include-figure.tex}
    \input{../sympy/catalan/Catalan-traditional-inverse-handle-negatives-centered-colouring-127-rows-mod2-partitioning-include-figure.tex}

    \subsection{Sprugnoli and He version}

    Expansion of Riordan array $\mathcal{C}$:
    \input{../sympy/catalan/catalan-sprugnoli-he-standard-ignore-negatives-centered-colouring-127-rows-mod2-partitioning-include-matrix.tex}
    Expansion of Riordan array $\mathcal{C}^{-1}$:
    \input{../sympy/catalan/catalan-sprugnoli-he-inverse-ignore-negatives-centered-colouring-127-rows-mod2-partitioning-include-matrix.tex}

    \input{../sympy/catalan/catalan-sprugnoli-he-standard-ignore-negatives-centered-colouring-127-rows-mod2-partitioning-include-figure.tex}
    \input{../sympy/catalan/catalan-sprugnoli-he-inverse-ignore-negatives-centered-colouring-127-rows-mod2-partitioning-include-figure.tex}
    \input{../sympy/catalan/Catalan-standard-handle-negatives-plain-colouring-127-rows-mod2-partitioning-include-figure.tex}
    \input{../sympy/catalan/Catalan-inverse-handle-negatives-plain-colouring-127-rows-mod2-partitioning-include-figure.tex}

    \subsection{Minor variants: triangles $\mathcal{S}, \mathcal{C}$ and $\mathcal{B}$}

    Expansion of Riordan array $\mathcal{S}$:
    \input{../sympy/catalan-like/catalan-variant-s-standard-ignore-negatives-centered-colouring-127-rows-mod2-partitioning-include-matrix}
    Expansion of Riordan array $\mathcal{S}^{-1}$:
    \input{../sympy/catalan-like/catalan-variant-s-inverse-ignore-negatives-centered-colouring-127-rows-mod2-partitioning-include-matrix}
    Expansion of Riordan array $\mathcal{C}$:
    \input{../sympy/catalan-like/catalan-variant-c-standard-ignore-negatives-centered-colouring-127-rows-mod2-partitioning-include-matrix}
    Expansion of Riordan array $\mathcal{C}^{-1}$:
    \input{../sympy/catalan-like/catalan-variant-c-inverse-ignore-negatives-centered-colouring-127-rows-mod2-partitioning-include-matrix}
    Expansion of Riordan array $\mathcal{B}$:
    \input{../sympy/catalan-like/catalan-variant-b-standard-ignore-negatives-centered-colouring-127-rows-mod2-partitioning-include-matrix}
    Expansion of Riordan array $\mathcal{B}^{\diamond}$:
    \input{../sympy/catalan-like/catalan-variant-b-diamond-standard-ignore-negatives-centered-colouring-127-rows-mod2-partitioning-include-matrix}

    \input{../sympy/catalan-like/catalan-variant-s-standard-ignore-negatives-centered-colouring-127-rows-mod2-partitioning-include-figure}
    \input{../sympy/catalan-like/catalan-variant-s-inverse-ignore-negatives-centered-colouring-127-rows-mod2-partitioning-include-figure}
    \input{../sympy/catalan-like/catalan-variant-c-standard-ignore-negatives-centered-colouring-127-rows-mod2-partitioning-include-figure}
    \input{../sympy/catalan-like/catalan-variant-c-inverse-ignore-negatives-centered-colouring-127-rows-mod2-partitioning-include-figure}
    \input{../sympy/catalan-like/catalan-variant-b-standard-ignore-negatives-centered-colouring-127-rows-mod2-partitioning-include-figure}
    \input{../sympy/catalan-like/catalan-variant-b-diamond-standard-ignore-negatives-centered-colouring-127-rows-mod2-partitioning-include-figure}

    \section{Motzkin}

    Expansion of Riordan array $\mathcal{M}$:
    \input{../sympy/motzkin/motzkin-standard-ignore-negatives-centered-colouring-127-rows-mod2-partitioning-include-matrix}
    Expansion of Riordan array $\mathcal{M}^{\diamond}$:
    \input{../sympy/motzkin/motzkin-diamond-standard-ignore-negatives-centered-colouring-127-rows-mod2-partitioning-include-matrix}
    Expansion of Riordan array $\mathcal{T}$:
    \input{../sympy/motzkin/motzkin-variant-t-standard-ignore-negatives-centered-colouring-127-rows-mod2-partitioning-include-matrix}
    Expansion of Riordan array $\mathcal{T}^{-1}$:
    \input{../sympy/motzkin/motzkin-variant-t-inverse-ignore-negatives-centered-colouring-127-rows-mod2-partitioning-include-matrix}
    Expansion of Riordan array $\mathcal{T}^{\perp}$:
    \input{../sympy/motzkin/motzkin-variant-t-perp-standard-ignore-negatives-centered-colouring-127-rows-mod2-partitioning-include-matrix}
    Expansion of Riordan array $\mathcal{T}^{\diamond}$:
    \input{../sympy/motzkin/motzkin-variant-t-diamond-standard-ignore-negatives-centered-colouring-127-rows-mod2-partitioning-include-matrix}

    
\begin{figure}[H]

    \noindent\makebox[\textwidth]{
        \centering
        %\includegraphics[width=0.8\textwidth]{../../sympy/catalan/coloured.pdf}

        % using *angle* property to rotate it is difficult to properly align it
        % in order to have a "real" matrix representation.
        \includegraphics[width=20cm, height=20cm, keepaspectratio=true]{../sympy/motzkin/motzkin-standard-ignore-negatives-centered-colouring-127-rows-mod2-partitioning-triangle.pdf}
    }

    % this 'particular' line is necessary to use `displaymath' environment
    % into the caption environment, togheter with the inclusion of 
    % `caption' package. See here for more explanation:
    % http://stackoverflow.com/questions/2716227/adding-an-equation-or-formula-to-a-figure-caption-in-latex
    \captionsetup{singlelinecheck=off}
    \caption[$\mathcal{M}_{\stackrel{\circ}{\equiv_{2}}}$]{ $\mathcal{M}_{\stackrel{\circ}{\equiv_{2}}}$ }

    \label{fig:motzkin-standard-ignore-negatives-centered-colouring-127-rows-mod2-partitioning-triangle}

\end{figure}

    \input{../sympy/motzkin/motzkin-diamond-standard-ignore-negatives-centered-colouring-127-rows-mod2-partitioning-include-figure}
    \input{../sympy/motzkin/motzkin-variant-t-standard-ignore-negatives-centered-colouring-127-rows-mod2-partitioning-include-figure}
    \input{../sympy/motzkin/motzkin-variant-t-inverse-ignore-negatives-centered-colouring-127-rows-mod2-partitioning-include-figure}
    \input{../sympy/motzkin/motzkin-variant-t-perp-standard-ignore-negatives-centered-colouring-127-rows-mod2-partitioning-include-figure}
    \input{../sympy/motzkin/motzkin-variant-t-diamond-standard-ignore-negatives-centered-colouring-127-rows-mod2-partitioning-include-figure}
    
    \input{../sympy/motzkin/Motzkin-standard-handle-negatives-plain-colouring-127-rows-mod2-partitioning-include-figure.tex}
    \input{../sympy/motzkin/Motzkin-inverse-handle-negatives-plain-colouring-127-rows-mod2-partitioning-include-figure.tex}

    \section{Delannoy}

    % mod 2
    Expansion of Riordan array $\mathcal{D}$:
    \input{../sympy/delannoy/delannoy-standard-ignore-negatives-centered-colouring-127-rows-mod2-partitioning-include-matrix.tex}
    Expansion of Riordan array $\mathcal{D}^{-1}$:
    \input{../sympy/delannoy/delannoy-inverse-ignore-negatives-centered-colouring-127-rows-mod2-partitioning-include-matrix.tex}

    \input{../sympy/delannoy/delannoy-standard-ignore-negatives-centered-colouring-127-rows-mod2-partitioning-include-figure.tex}
    \input{../sympy/delannoy/delannoy-inverse-ignore-negatives-centered-colouring-127-rows-mod2-partitioning-include-figure.tex}

    % mod 3
    % the following matrices are identical to the previous ones
    %\input{../sympy/delannoy/delannoy-standard-ignore-negatives-centered-colouring-127-rows-mod3-partitioning-include-matrix.tex}
    %\input{../sympy/delannoy/delannoy-inverse-ignore-negatives-centered-colouring-127-rows-mod3-partitioning-include-matrix.tex}

    \input{../sympy/delannoy/delannoy-standard-ignore-negatives-centered-colouring-127-rows-mod3-partitioning-include-figure.tex}
    \input{../sympy/delannoy/delannoy-inverse-ignore-negatives-centered-colouring-127-rows-mod3-partitioning-include-figure.tex}

    % bib stuff
    %\nocite{*}
    %\addtocontents{toc}{\protect\vspace{\beforebibskip}}
    %\addcontentsline{toc}{section}{\refname}    
    %\bibliographystyle{plain}
    %\bibliography{../Bibliography}
\end{document}
