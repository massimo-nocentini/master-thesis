% article example for classicthesis.sty
\documentclass[10pt,a4paper]{article} % KOMA-Script article scrartcl
\usepackage{lipsum}
\usepackage{url}
\usepackage[nochapters]{classicthesis} % nochapters
\usepackage{graphicx}
\usepackage{float}
\usepackage{amsmath}
\usepackage{caption}

\newtheorem{theorem}{Theorem}[section]
\newtheorem{lemma}[theorem]{Lemma}
\newtheorem{conjecture}[theorem]{Conjecture}
\newtheorem{corollary}[theorem]{Corollary}

\begin{document}
    \title{\rmfamily\normalfont\spacedallcaps{Colouring Riordan arrays}}
    \author{\spacedlowsmallcaps{Donatella Merlini} \\ \spacedlowsmallcaps{Massimo Nocentini}}
    \date{\today} 
    
    \maketitle
    
    \begin{abstract}
        \noindent\lipsum[1] Just a test.\footnote{This is a footnote.}
    \end{abstract}
       
    \tableofcontents
    
    \section{Pascal}

    \subsection{Some modular proofs}
    
    In this section we present some proofs about Pascal array and its
    inverse, taken modulo some prime $p$. Colouring with different
    colours elements belonging to different remainder classes, we show
    formally that for $p$ even, ie. $p=2$, elements in the same
    position get the same colour (hence both triangle are coloured the
    same way, ignoring signs for elements in the inverse array), while
    for $p$ odd there is not such a simple correspondence and we
    attempt to observe some repeating pattern of mapping between
    elements.

    Let $\mathcal{P}$ be the Riordan array for the Pascal triangle,
    defined as:
    \begin{displaymath} 
        \mathcal{P} = \left(\frac{1}{1-t}, \frac{t}{1-t}  \right)
    \end{displaymath} 
    and let $\mathcal{P}^{-1}$ be its inverse Riordan array:
    \begin{displaymath} 
        \mathcal{P}^{-1} = \left(\frac{1}{1+t}, \frac{t}{1+t}  \right)
    \end{displaymath} 
    
    Let $d_{nk}$ and $\hat{d}_{nk}$ be the generic element of Pascal
    array and of its inverse, respectively. Since both arrays are
    Riordan, by definition:
    \begin{displaymath}
        \begin{split}
            d_{nk} &= [t^n]\frac{1}{1-t}\left(\frac{t}{1-t}\right)^k = [t^{n-k}](1-t)^{-(k+1)} \\
                &= {{-(k+1)} \choose {n-k}}(-1)^{n-k} = {{k+1 +n-k -1} \choose {n-k}} = {{n} \choose {n-k}} \\
        \end{split}
    \end{displaymath}
    and for the inverse:
    \begin{displaymath}
        \begin{split}
            \hat{d}_{nk} &= [t^n]\frac{1}{1+t}\left(\frac{t}{1+t}\right)^k = [t^{n-k}](1+t)^{-(k+1)} = 
                {{-(k+1)} \choose {n-k}} \\
                &= {{k+1 +n-k -1} \choose {n-k}} (-1)^{n-k} = {{n} \choose {n-k}} (-1)^{n-k}\\
        \end{split}
    \end{displaymath}
    Hence, equating binomial coefficients yields:
    \begin{displaymath}
        [t^n]\frac{1}{1-t}\left(\frac{t}{1-t}\right)^k = (-1)^{k-n}[t^n]\frac{1}{1+t}\left(\frac{t}{1+t}\right)^k 
    \end{displaymath}
    Choose a prime $p$ and take the modulo of both members:
    \begin{displaymath}
        [t^n]\frac{1}{1-t}\left(\frac{t}{1-t}\right)^k \equiv_{p} (-1)^{k-n}[t^n]\frac{1}{1+t}\left(\frac{t}{1+t}\right)^k 
    \end{displaymath}

    From now on we have to reason according modular arithmetic, hence
    multiply by a term $a$ both member of equations in order to
    simplify requests to show the existence of multiplicative inverse
    modulo $p$ of $a$, denoted by: $$a^{-1}\mod p$$

    First of all, observe that $-1 \equiv_{p} p-1$ and since $p$ is a
    prime by hp, it follows that $(p, p-1)=1$ (this result holds in
    general, not just for $p$ prime), which proofs the existence of
    both $-1$ and $p-1$ inverses, denoted by $(-1)^{-1}\mod p$ and
    $(p-1)^{-1}\mod p$ respectively.

    In order to find $(p-1)^{-1}\mod p$ we have to satisfy the
    congruence equation $(p-1) * (p-1)^{-1} \equiv_{p} 1$. Choose
    $(p-1)^{-1}\mod p = p-1$ and verify
    $(p-1) * (p-1) \equiv_{p} p^2 -2p +1 \equiv_{p} 1$ as required.

    Another useful observation concerns raising to negative powers,
    let $k \geq 0$:
    \begin{displaymath}
        (-1)^{-k} \equiv_{p} \left((-1)^{-1}\right)^{k} \equiv_{p} (p-1)^k
    \end{displaymath}
    and since $-1 \equiv_{p} p-1$ it follows that
    $(-1)^{-k} \equiv_{p} (-1)^k$, surprising to me.
    
    Now we can use previous observation to the main modular equation:
    \begin{displaymath}
        \begin{split}
            [t^n]\frac{1}{1-t}\left(\frac{t}{1-t}\right)^k 
                &\equiv_{p} (-1)^{k-n}[t^n]\frac{1}{1+t}\left(\frac{t}{1+t}\right)^k \\
                &\equiv_{p} (-1)^{k+n}[t^n]\frac{1}{1+t}\left(\frac{t}{1+t}\right)^k \\
                &\equiv_{p} (p-1)^{k+n}[t^n]\frac{1}{1+t}\left(\frac{t}{1+t}\right)^k \\
        \end{split}
    \end{displaymath}
    Hence, multiplying by $(p-1)^{-1}\mod p$ both members $k+n$ times:
    \begin{displaymath}
        (p-1)^{k+n}[t^n]\frac{1}{1-t}\left(\frac{t}{1-t}\right)^k \equiv_{p} [t^n]\frac{1}{1+t}\left(\frac{t}{1+t}\right)^k 
    \end{displaymath}
    which is the same as:
    \begin{displaymath}
        (-1)^{k+n}[t^n]\frac{1}{1-t}\left(\frac{t}{1-t}\right)^k \equiv_{p} [t^n]\frac{1}{1+t}\left(\frac{t}{1+t}\right)^k 
    \end{displaymath}
    and relating generic elements $d_{nk}$ with $\hat{d}_{nk}$:
    \begin{displaymath}
        (p-1)^{k+n}d_{nk}\equiv_{p}(-1)^{k+n}d_{nk} \equiv_{p} \hat{d}_{nk}
    \end{displaymath}

    \subsubsection{$p=2$}
    The case for $p$ even prime produce the colouring reported in
    REGENERATE AND PUT HERE THE RIGHT REFERENCE TO PASCAL: ignoring
    signs in the inverse array, we got the same colouring.  This is
    justified using the previous argument with $p=2$, for any choice
    of $n, k \geq 0$:
    \begin{displaymath} 
        d_{nk} \equiv_{2} \hat{d}_{nk} 
    \end{displaymath} 

    \subsubsection{$p=3$}
    Here there's a more interesting pattern to study, and in general,
    for any odd prime $p$, colourings of standard and inverse arrays
    mismatch. In this section we tackle the case for $p=3$,
    instantiating the modular equation:
    \begin{displaymath}
        2^{k+n}d_{nk}\equiv_{3}(-1)^{k+n}d_{nk} \equiv_{3} \hat{d}_{nk}
    \end{displaymath}
    For the sake of clarity, consider row 4th of both triangles:
    \begin{itemize}
        \item $\mathcal{P}[3,:] = (1 \quad 3 \quad 3 \quad 1) \equiv_{3}(1 \quad 0 \quad 0 \quad 1)$
        \item $\mathcal{P}^{-1}[3,:] = (-1 \quad 3 \quad -3 \quad 1) \equiv_{3}(2 \quad 0 \quad 0 \quad 1)$
    \end{itemize}
    Hence element $d_{30}$ gets a color $c$ while $\hat{d}_{30}$ gets a color $c'$ different from $c$.

    No general mapping among this relationship appears, so we
    carefully study two coloured triangles.  First fix a row which
    will be our reference row, choose index $3^4$. Repeatedly, move up
    one by one, and for each considered row move on the right over
    columns, here a modular equivalence going up one row:
    \begin{displaymath}
        \begin{split}
            d_{3^4 -1,0} &\equiv_{3} \hat{d}_{3^4 -1,0} \\
            d_{3^4 -1,1} &\equiv_{3} \hat{d}_{3^4 -2,0} \\
            d_{3^4 -1,2} &\equiv_{3} \hat{d}_{3^4 -3,0} \\
            d_{3^4 -1,3} &\equiv_{3} \hat{d}_{3^4 -4,0} \\
            &\vdots
        \end{split}
    \end{displaymath}
    next, a modular equivalence going up two rows:
    \begin{displaymath}
        \begin{split}
            d_{3^4 -2,0} &\equiv_{3} \hat{d}_{3^4 -1,1} \\
            d_{3^4 -2,1} &\equiv_{3} \hat{d}_{3^4 -2,1} \\
            d_{3^4 -2,2} &\equiv_{3} \hat{d}_{3^4 -3,1} \\
            d_{3^4 -2,3} &\equiv_{3} \hat{d}_{3^4 -4,1} \\
            &\vdots
        \end{split}
    \end{displaymath}
    next, a modular equivalence going up three rows:
    \begin{displaymath}
        \begin{split}
            d_{3^4 -3,0} &\equiv_{3} \hat{d}_{3^4 -1,2} \\
            d_{3^4 -3,1} &\equiv_{3} \hat{d}_{3^4 -2,2} \\
            d_{3^4 -3,2} &\equiv_{3} \hat{d}_{3^4 -3,2} \\
            d_{3^4 -3,3} &\equiv_{3} \hat{d}_{3^4 -4,2} \\
            &\vdots
        \end{split}
    \end{displaymath}
    Let introduce variable $b$, running over rows, and $a$, running
    over columns; a pattern appears, here it is:
    \begin{displaymath}
            d_{3^4 +b,a} \equiv_{3} \hat{d}_{3^4 -1-a,-1-b} 
    \end{displaymath}

    Let's say, assume the colouring for $\mathcal{P}^{-1}$ triangle is
    given, the colouring for row $26$ of $\mathcal{P}$ is desired. It
    is necessary to find $b$: by $3^4 +b=26$ get $b=-55$, so the
    required row satisfy the following modular equivalence:
    \begin{displaymath}
            d_{26,a} \equiv_{3} \hat{d}_{80-a,54} 
    \end{displaymath}
    This is pretty curious since in order to colour a row in triangle
    $\mathcal{P}$, from left to right since for proper arrays
    $a \geq 0$, column's colouring in triangle $\mathcal{P}^{-1}$ is
    used, from bottom to top\footnote{Some pictures highlighting the
    interested rows and columns should be very helpful}.
    \\\\
    A question is still open: why do we choose row $3^4$ as reference row?

    It is useful to recall a theorem due to Fine:
    \begin{theorem}
        A necessary and sufficient condition for a binomial coefficient ${{n} \choose {m}}$
        to be divisible by a prime $p$ is that $n$ be a power of $p$.
    \end{theorem}
    Consider the colouring for triangle $\mathcal{P}$, we can use the given theorem
    to point out ``interesting'' rows, namely those rows affected by the theorem,
    they correspond to powers $3^1, 3^2, 3^3, 3^4, \ldots$, each one of them can be easily
    recognized since dots lying on it have all the same colour. In the triangle $127$ former
    rows are drawn and in order to have ``more space'' to find a modular relationship between
    $d_{nk}$ and $\hat{d}_{nk}$ we choose as \emph{reference} row the one with index $3^4$.
    From here we start moving backwards by rows toward the root: observe that the entire row
    $3^4 -1$, containing $3^4 -1$ remainders, of triangle $\mathcal{P}$ is the first segment of
    the first column of $\mathcal{P}^{-1}$, in other words $d_{3^4 -1,a} \equiv_{3} \hat{d}_{3^4 -1 -a, 0}$ 
    for $a \in \lbrace 0, \ldots, 3^4 -1\rbrace$. 

    It seems that coefficient $d_{3^4-1,0}$ acts as a pivot on which  the triangle ``flips'': the
    root moves toward the reader while the bottom edge moves toward opposite the reader. This rigid 
    motion is captured by the following modular relationships among three important points:
    \begin{displaymath}
        \begin{split}
            d_{3^4 -1,0} &\equiv_{3} \hat{d}_{3^4 -1,0} \\
            d_{3^4 -1,3^4 -1} &\equiv_{3} \hat{d}_{0,0} \\
            d_{3^4 -3^3,3^3-1} &\equiv_{3} \hat{d}_{3^4 -3^3,3^3-1} \\
        \end{split}
    \end{displaymath}

    \subsection{Expanded arrays and coloured triangles}

    % mod 2
    

\begin{table}
    \begin{displaymath} 
        \hspace{-5cm}
        \mathcal{P}_{10}\left(\begin{array}{rrrrrrrrrr}
        1 & 0 & 0 & 0 & 0 & 0 & 0 & 0 & 0 & 0 \\
        1 & 1 & 0 & 0 & 0 & 0 & 0 & 0 & 0 & 0 \\
        1 & 2 & 1 & 0 & 0 & 0 & 0 & 0 & 0 & 0 \\
        1 & 3 & 3 & 1 & 0 & 0 & 0 & 0 & 0 & 0 \\
        1 & 4 & 6 & 4 & 1 & 0 & 0 & 0 & 0 & 0 \\
        1 & 5 & 10 & 10 & 5 & 1 & 0 & 0 & 0 & 0 \\
        1 & 6 & 15 & 20 & 15 & 6 & 1 & 0 & 0 & 0 \\
        1 & 7 & 21 & 35 & 35 & 21 & 7 & 1 & 0 & 0 \\
        1 & 8 & 28 & 56 & 70 & 56 & 28 & 8 & 1 & 0 \\
        1 & 9 & 36 & 84 & 126 & 126 & 84 & 36 & 9 & 1
        \end{array}\right) 
        \quad
        \mathcal{P}_{10}^{-1}\left(\begin{array}{rrrrrrrrrr}
        1 & 0 & 0 & 0 & 0 & 0 & 0 & 0 & 0 & 0 \\
        -1 & 1 & 0 & 0 & 0 & 0 & 0 & 0 & 0 & 0 \\
        1 & -2 & 1 & 0 & 0 & 0 & 0 & 0 & 0 & 0 \\
        -1 & 3 & -3 & 1 & 0 & 0 & 0 & 0 & 0 & 0 \\
        1 & -4 & 6 & -4 & 1 & 0 & 0 & 0 & 0 & 0 \\
        -1 & 5 & -10 & 10 & -5 & 1 & 0 & 0 & 0 & 0 \\
        1 & -6 & 15 & -20 & 15 & -6 & 1 & 0 & 0 & 0 \\
        -1 & 7 & -21 & 35 & -35 & 21 & -7 & 1 & 0 & 0 \\
        1 & -8 & 28 & -56 & 70 & -56 & 28 & -8 & 1 & 0 \\
        -1 & 9 & -36 & 84 & -126 & 126 & -84 & 36 & -9 & 1
        \end{array}\right) 
    \end{displaymath}

  \caption[$\mathcal{P}$ and $\mathcal{P}^{-1}$]{Two $10$-minors of
  $\mathcal{P}$ and $\mathcal{P}^{-1}$ matrix expansions, respectively}

  \label{tab:pascal:array} 
  
  \end{table}



    \input{../sympy/pascal/pascal-inverse-ignore-negatives-centered-colouring-127-rows-mod2-partitioning-include-matrix.tex}

    

\begin{figure}[p]
    
    \hspace{-1.5cm}
    \noindent\makebox[\textwidth]{
        \centering
        %\includegraphics[width=0.8\textwidth]{../../sympy/catalan/coloured.pdf}

        % using *angle* property to rotate it is difficult to properly align it
        % in order to have a "real" matrix representation.
        \includegraphics[width=20cm, height=20cm, keepaspectratio=true]{../sympy/pascal/pascal-standard-ignore-negatives-centered-colouring-127-rows-mod2-partitioning-triangle.pdf}
    }

    % this 'particular' line is necessary to use `displaymath' environment
    % into the caption environment, togheter with the inclusion of 
    % `caption' package. See here for more explanation:
    % http://stackoverflow.com/questions/2716227/adding-an-equation-or-formula-to-a-figure-caption-in-latex
    \captionsetup{singlelinecheck=off}
    \caption[$\mathcal{P}_{\equiv_{2}}$]{
        $\mathcal{P}_{\equiv_{2}}$
        \iffalse
        Pascal triangle, formally: 
        \begin{displaymath}
            \mathcal{P}=\left(\frac{1}{1-t}, \frac{t}{1-t}\right)
        \end{displaymath} % \newline % new line no more necessary
        standard, ignore negatives, centered colouring, 127 rows, mod2 partitioning
        \fi
        }

    \label{fig:pascal-standard-ignore-negatives-centered-colouring-127-rows-mod2-partitioning-triangle}

\end{figure}

    
\begin{figure}[p]

    \noindent\makebox[\textwidth]{
        \centering
        %\includegraphics[width=0.8\textwidth]{../../sympy/catalan/coloured.pdf}

        % using *angle* property to rotate it is difficult to properly align it
        % in order to have a "real" matrix representation.
        \includegraphics[width=20cm, height=20cm, keepaspectratio=true]{../sympy/pascal/pascal-inverse-ignore-negatives-centered-colouring-127-rows-mod2-partitioning-triangle.pdf}
    }

    % this 'particular' line is necessary to use `displaymath' environment
    % into the caption environment, togheter with the inclusion of 
    % `caption' package. See here for more explanation:
    % http://stackoverflow.com/questions/2716227/adding-an-equation-or-formula-to-a-figure-caption-in-latex
    \captionsetup{singlelinecheck=off}
    \caption[$\mathcal{P}_{\equiv_{2}}^{-1}$, ignore negative entries]{
        Pascal triangle, formally: 
        \begin{displaymath}
            \mathcal{P}^{-1}=\left(\frac{1}{t + 1}, \frac{t}{t + 1}\right)
        \end{displaymath} % \newline % new line no more necessary
        inverse, ignore negatives, centered colouring, 127 rows, mod2 partitioning}

    \label{fig:pascal-inverse-ignore-negatives-centered-colouring-127-rows-mod2-partitioning-triangle}

\end{figure}


    % mod 3
    % the following matrices are identical to the previous ones
    %\input{../sympy/pascal/pascal-standard-ignore-negatives-centered-colouring-127-rows-mod3-partitioning-include-matrix.tex}
    %\begin{displaymath} \left(\begin{array}{rrrrrrrrrr}
1 & 0 & 0 & 0 & 0 & 0 & 0 & 0 & 0 & 0 \\
-1 & 1 & 0 & 0 & 0 & 0 & 0 & 0 & 0 & 0 \\
1 & -2 & 1 & 0 & 0 & 0 & 0 & 0 & 0 & 0 \\
-1 & 3 & -3 & 1 & 0 & 0 & 0 & 0 & 0 & 0 \\
1 & -4 & 6 & -4 & 1 & 0 & 0 & 0 & 0 & 0 \\
-1 & 5 & -10 & 10 & -5 & 1 & 0 & 0 & 0 & 0 \\
1 & -6 & 15 & -20 & 15 & -6 & 1 & 0 & 0 & 0 \\
-1 & 7 & -21 & 35 & -35 & 21 & -7 & 1 & 0 & 0 \\
1 & -8 & 28 & -56 & 70 & -56 & 28 & -8 & 1 & 0 \\
-1 & 9 & -36 & 84 & -126 & 126 & -84 & 36 & -9 & 1
\end{array}\right) \end{displaymath}

    \input{../sympy/pascal/pascal-standard-ignore-negatives-centered-colouring-127-rows-mod3-partitioning-include-figure.tex}
    
\begin{figure}[p]

    \noindent\makebox[\textwidth]{
        \centering
        %\includegraphics[width=0.8\textwidth]{../../sympy/catalan/coloured.pdf}

        % using *angle* property to rotate it is difficult to properly align it
        % in order to have a "real" matrix representation.
        \includegraphics[width=20cm, height=20cm, keepaspectratio=true]{Chapters/mod-p-characterization/pascal/pascal-inverse-ignore-negatives-centered-colouring-127-rows-mod3-partitioning-triangle.pdf}
    }

    % this 'particular' line is necessary to use `displaymath' environment
    % into the caption environment, togheter with the inclusion of 
    % `caption' package. See here for more explanation:
    % http://stackoverflow.com/questions/2716227/adding-an-equation-or-formula-to-a-figure-caption-in-latex
    \captionsetup{singlelinecheck=off}
    \caption[.]{
        Pascal triangle, formally: 
        \begin{displaymath}
            \mathcal{P}^{-1}=\left(\frac{1}{t + 1}, \frac{t}{t + 1}\right)
        \end{displaymath} % \newline % new line no more necessary
        inverse, ignore negatives, centered colouring, 127 rows, mod3 partitioning}

    \label{fig:pascal-inverse-ignore-negatives-centered-colouring-127-rows-mod3-partitioning-triangle}

\end{figure}


    % mod 5
    % the following matrices are identical to the previous ones
    %\input{../sympy/pascal/pascal-standard-ignore-negatives-centered-colouring-127-rows-mod5-partitioning-include-matrix.tex}
    %\input{../sympy/pascal/pascal-inverse-ignore-negatives-centered-colouring-127-rows-mod5-partitioning-include-matrix.tex}

    \input{../sympy/pascal/pascal-standard-ignore-negatives-centered-colouring-127-rows-mod5-partitioning-include-figure.tex}
    \input{../sympy/pascal/pascal-inverse-ignore-negatives-centered-colouring-127-rows-mod5-partitioning-include-figure.tex}

    % the following two rows refer to old triangles, pay attention on the filename schema.
    \input{../sympy/pascal/Pascal-standard-handle-negatives-plain-colouring-127-rows-mod2-partitioning-include-figure.tex}
    \input{../sympy/pascal/Pascal-inverse-handle-negatives-plain-colouring-127-rows-mod2-partitioning-include-figure.tex}

    \section{Fibonacci}

    \input{../sympy/fibonacci/Fibonacci-standard-handle-negatives-centered-colouring-127-rows-mod2-partitioning-include-matrix.tex}
    \input{../sympy/fibonacci/Fibonacci-inverse-handle-negatives-centered-colouring-127-rows-mod2-partitioning-include-matrix.tex}

    \input{../sympy/fibonacci/Fibonacci-standard-handle-negatives-centered-colouring-127-rows-mod2-partitioning-include-figure.tex}
    \input{../sympy/fibonacci/Fibonacci-inverse-handle-negatives-centered-colouring-127-rows-mod2-partitioning-include-figure.tex}

    \input{../sympy/fibonacci/Fibonacci-standard-handle-negatives-plain-colouring-127-rows-mod2-partitioning-include-figure.tex}
    \input{../sympy/fibonacci/Fibonacci-inverse-handle-negatives-plain-colouring-127-rows-mod2-partitioning-include-figure.tex}

    \section{Catalan}

    \subsection{Traditional version}

    \input{../sympy/catalan/Catalan-traditional-standard-handle-negatives-centered-colouring-127-rows-mod2-partitioning-include-figure.tex}
    \input{../sympy/catalan/Catalan-traditional-inverse-handle-negatives-centered-colouring-127-rows-mod2-partitioning-include-figure.tex}

    \input{../sympy/catalan/Catalan-traditional-standard-handle-negatives-centered-colouring-127-rows-mod2-partitioning-include-matrix.tex}
    \input{../sympy/catalan/Catalan-traditional-inverse-handle-negatives-centered-colouring-127-rows-mod2-partitioning-include-matrix.tex}

    \subsection{Sprugnoli and He version}

    \input{../sympy/catalan/Catalan-standard-handle-negatives-plain-colouring-127-rows-mod2-partitioning-include-figure.tex}
    \input{../sympy/catalan/catalan-sprugnoli-he-standard-ignore-negatives-centered-colouring-127-rows-mod2-partitioning-include-matrix.tex}

    \input{../sympy/catalan/Catalan-inverse-handle-negatives-plain-colouring-127-rows-mod2-partitioning-include-figure.tex}
    \input{../sympy/catalan/catalan-sprugnoli-he-inverse-ignore-negatives-centered-colouring-127-rows-mod2-partitioning-include-matrix.tex}

    \input{../sympy/catalan/catalan-sprugnoli-he-standard-ignore-negatives-centered-colouring-127-rows-mod2-partitioning-include-figure.tex}
    \input{../sympy/catalan/catalan-sprugnoli-he-inverse-ignore-negatives-centered-colouring-127-rows-mod2-partitioning-include-figure.tex}


    \section{Motzkin}

    \input{../sympy/motzkin/Motzkin-standard-ignore-negatives-centered-colouring-127-rows-mod2-partitioning-include-matrix.tex}
    \input{../sympy/motzkin/Motzkin-inverse-ignore-negatives-centered-colouring-127-rows-mod2-partitioning-include-matrix.tex}

    \input{../sympy/motzkin/Motzkin-standard-ignore-negatives-centered-colouring-127-rows-mod2-partitioning-include-figure.tex}
    \input{../sympy/motzkin/Motzkin-inverse-ignore-negatives-centered-colouring-127-rows-mod2-partitioning-include-figure.tex}
    
    \input{../sympy/motzkin/Motzkin-standard-handle-negatives-plain-colouring-127-rows-mod2-partitioning-include-figure.tex}
    \input{../sympy/motzkin/Motzkin-inverse-handle-negatives-plain-colouring-127-rows-mod2-partitioning-include-figure.tex}

    \section{Delannoy}

    % mod 2
    \input{../sympy/delannoy/delannoy-standard-ignore-negatives-centered-colouring-127-rows-mod2-partitioning-include-matrix.tex}
    \input{../sympy/delannoy/delannoy-inverse-ignore-negatives-centered-colouring-127-rows-mod2-partitioning-include-matrix.tex}

    \input{../sympy/delannoy/delannoy-standard-ignore-negatives-centered-colouring-127-rows-mod2-partitioning-include-figure.tex}
    \input{../sympy/delannoy/delannoy-inverse-ignore-negatives-centered-colouring-127-rows-mod2-partitioning-include-figure.tex}

    % mod 3
    % the following matrices are identical to the previous ones
    %\input{../sympy/delannoy/delannoy-standard-ignore-negatives-centered-colouring-127-rows-mod3-partitioning-include-matrix.tex}
    %\input{../sympy/delannoy/delannoy-inverse-ignore-negatives-centered-colouring-127-rows-mod3-partitioning-include-matrix.tex}

    \input{../sympy/delannoy/delannoy-standard-ignore-negatives-centered-colouring-127-rows-mod3-partitioning-include-figure.tex}
    \input{../sympy/delannoy/delannoy-inverse-ignore-negatives-centered-colouring-127-rows-mod3-partitioning-include-figure.tex}

    % bib stuff
    %\nocite{*}
    %\addtocontents{toc}{\protect\vspace{\beforebibskip}}
    %\addcontentsline{toc}{section}{\refname}    
    %\bibliographystyle{plain}
    %\bibliography{../Bibliography}
\end{document}
