
\noindent Catalan numbers were introduced with this name back in $1838$ by
Eugene Catalan, although they have been used even before by Ming Antu
in $1730$. Let $c_{n}$ denote the $n$-th Catalan number, then the following
table shows the first fifteen element of the infinite sequence 
$\lbrace c_{n}\rbrace_{n\in\mathbb{N}}$ ($A000108$\footnote{All integer sequences
defined in this paper are recorded in Sloane \cite{sloane:oeis}: keys of
the form $Annnnnn$ refer to entries in that database.}):
\begin{displaymath}
    \footnotesize
    \begin{array}{c|ccccccccccccccc}
        n & 0 & 1 & 2 & 3 & 4 & 5 & 6 & 7 & 8 & 9 & 10 & 11 & 12 & 13 & 14 \\
        \hline
        c_{n} & 1 & 1 & 2 & 5 & 14 & 42 & 132 & 429 & 1430 & 4862 & 16796 & 58786 & 208012 & 742900 & 2674440
    \end{array}
\end{displaymath}

A comprehensive manuscript entirely devoted to these numbers has been recently
published by Stanley \cite{stanley:2015}, where it is possible to find their history,
formal characterizations using generating functions and, finally, $214$ combinatorial 
interpretations.  For the sake of clarity, we report three classes of objects counted by 
$c_{n}$: first, binary trees with $n$ nodes; 
second, admissible bracketing of a string with length $n+1$ using $n-1$ pairs
of curly braces; at last, Dyck paths of length $2n$ using $\diagup$ and $\diagdown$ steps.

It is possible to take $c_{n}$ apart introducing an additional dimension,
obtaining coefficients $c_{n,k}$ defining a lower triangular infinite 
matrix $\mathcal{C}$ ($A033184$): in \autoref{tab:catalan:array} 
we report the upper chunk of that matrix where a coefficient at row $n$ 
and column $k$ can be interpreted as the number
of Dyck paths of length $2(n+1)$ with $k$ returns to the ground, start
and end points excluded -- conventionally, indexes are $0$-based.

% inclusion of matrix expansion for C and its inverse.

\begin{table}
    \begin{displaymath} 
        \hspace{-5.5cm}
        \mathcal{C}_{10} \left(\begin{array}{rrrrrrrrrr}
        1 & 0 & 0 & 0 & 0 & 0 & 0 & 0 & 0 & 0 \\
        1 & 1 & 0 & 0 & 0 & 0 & 0 & 0 & 0 & 0 \\
        2 & 2 & 1 & 0 & 0 & 0 & 0 & 0 & 0 & 0 \\
        5 & 5 & 3 & 1 & 0 & 0 & 0 & 0 & 0 & 0 \\
        14 & 14 & 9 & 4 & 1 & 0 & 0 & 0 & 0 & 0 \\
        42 & 42 & 28 & 14 & 5 & 1 & 0 & 0 & 0 & 0 \\
        132 & 132 & 90 & 48 & 20 & 6 & 1 & 0 & 0 & 0 \\
        429 & 429 & 297 & 165 & 75 & 27 & 7 & 1 & 0 & 0 \\
        1430 & 1430 & 1001 & 572 & 275 & 110 & 35 & 8 & 1 & 0 \\
        4862 & 4862 & 3432 & 2002 & 1001 & 429 & 154 & 44 & 9 & 1
        \end{array}\right) 
        \quad
        \mathcal{C}_{10}^{-1}\left(\begin{array}{rrrrrrrrrr}
        1 & 0 & 0 & 0 & 0 & 0 & 0 & 0 & 0 & 0 \\
        1 & 0 & 0 & 0 & 0 & 0 & 0 & 0 & 0 & 0 \\
        0 & 1 & 0 & 0 & 0 & 0 & 0 & 0 & 0 & 0 \\
        0 & -1 & 1 & 0 & 0 & 0 & 0 & 0 & 0 & 0 \\
        0 & 0 & -2 & 1 & 0 & 0 & 0 & 0 & 0 & 0 \\
        0 & 0 & 1 & -3 & 1 & 0 & 0 & 0 & 0 & 0 \\
        0 & 0 & 0 & 3 & -4 & 1 & 0 & 0 & 0 & 0 \\
        0 & 0 & 0 & -1 & 6 & -5 & 1 & 0 & 0 & 0 \\
        0 & 0 & 0 & 0 & -4 & 10 & -6 & 1 & 0 & 0 \\
        0 & 0 & 0 & 0 & 1 & -10 & 15 & -7 & 1 & 0
        \end{array}\right)
    \end{displaymath}

  \caption[$\mathcal{C}$ and $\mathcal{C}^{-1}$]{Two $10$-minors of
  $\mathcal{C}$ and $\mathcal{C}^{-1}$ matrix expansions, respectively}

  \label{tab:catalan:array} 

\end{table}


In the literature, matrix $\mathcal{C}$ is known as \emph{Catalan triangle} and
it turns out to be a Riordan array. Those arrays forms a family of matrices enjoying
nice properties and, looking at them from the abstract algebra perspective, 
form a group structure. 
Formally, a Riordan array is denoted by a pair of functions $d$ and $h$ which provide a unique 
characterization for the generic coefficient $d_{n,k}$, lying at row $n$ and column $k$, 
by extraction of the $n$-th coefficient from the following $k$-convolution:
\begin{displaymath}
    d_{n,k} = [t^{n}]d(t)h(t)^{k}
\end{displaymath}
Riordan arrays have been introduced by Shapiro et al. \cite{shapiro:1991} in $1991$;
furthermore, Sprugnoli \cite{sprugnoli:1991} in $1994$ pointed out their importance
to solve combinatorial sums.


We recall some basics of modular arithmetic. Let $a,b\in\mathbb{Z}$
and $n\in\mathbb{N}$, we say that $a$ is congruent to $b$ modulo $n$,
written $a \equiv_{n} b$, if and only if $a\mod\,n = b\mod\,n$,
which is the same to say that $a - b = nk$ for some $k\in\mathbb{Z}$.
