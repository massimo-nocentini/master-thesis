
\section{Introduction}

\noindent Catalan numbers were introduced with this name by
Eugene Catalan in $1838$, although they have been used even before by Ming Antu
in $1730$. Let $C_{n}$ be the $n$-th Catalan number, then the following
table shows the first fifteen element of the infinite sequence
$\left(C_{n}\right)_{n\in\mathbb{N}}$,
\begin{displaymath}
    \footnotesize
    \begin{array}{c|ccccccccccccccc}
        n & 0 & 1 & 2 & 3 & 4 & 5 & 6 & 7 & 8 & 9 & 10 & 11 & 12 & 13 & 14 \\
        \hline
        C_{n} & 1 & 1 & 2 & 5 & 14 & 42 & 132 & 429 & 1430 & 4862 & 16796 & 58786 & 208012 & 742900 & 2674440
    \end{array}
\end{displaymath} 
known as $A000108$ in the Online Encyclopedia of Integer Sequences
\cite{sloane:oeis}; moreover, all integer sequences we refer to are identified
by keys of the form $Annnnnn$ that refer to entries in the encyclopedia.


A comprehensive manuscript having origin in $1970$, entirely devoted to these
numbers, has been recently published by Stanley \cite{stanley:2015} where it is
possible to find their history, formal characterizations and $214$
combinatorial interpretations.  For the sake of clarity, we report three
classes of objects counted by $C_{n}$: first, binary trees with $n$ nodes;
second, admissible bracketing of a string with length $n+1$ using $n-1$ pairs
of curly braces; third, Dyck paths of length $2n$ using raising $\diagup$ and
falling $\diagdown$ steps.

It is possible to augment $C_{n}$ with an additional dimension, obtaining
coefficients $C_{n,k}$ defining a lower triangular, infinite matrix
$\mathcal{C}$ ($A033184$): in \autoref{tab:catalan:array} we report the upper
chunk of that matrix where a coefficient at row $n$ and column $k$ can be
interpreted as the number of Dyck paths of length $2(n+1)$ with $k$ returns to
the ground, start and end points excluded -- conventionally, indexes are
$0$-based.


\begin{table}
    \begin{displaymath} 
        \hspace{-5.5cm}
        \mathcal{C}_{10} \left(\begin{array}{rrrrrrrrrr}
        1 & 0 & 0 & 0 & 0 & 0 & 0 & 0 & 0 & 0 \\
        1 & 1 & 0 & 0 & 0 & 0 & 0 & 0 & 0 & 0 \\
        2 & 2 & 1 & 0 & 0 & 0 & 0 & 0 & 0 & 0 \\
        5 & 5 & 3 & 1 & 0 & 0 & 0 & 0 & 0 & 0 \\
        14 & 14 & 9 & 4 & 1 & 0 & 0 & 0 & 0 & 0 \\
        42 & 42 & 28 & 14 & 5 & 1 & 0 & 0 & 0 & 0 \\
        132 & 132 & 90 & 48 & 20 & 6 & 1 & 0 & 0 & 0 \\
        429 & 429 & 297 & 165 & 75 & 27 & 7 & 1 & 0 & 0 \\
        1430 & 1430 & 1001 & 572 & 275 & 110 & 35 & 8 & 1 & 0 \\
        4862 & 4862 & 3432 & 2002 & 1001 & 429 & 154 & 44 & 9 & 1
        \end{array}\right) 
        \quad
        \mathcal{C}_{10}^{-1}\left(\begin{array}{rrrrrrrrrr}
        1 & 0 & 0 & 0 & 0 & 0 & 0 & 0 & 0 & 0 \\
        1 & 0 & 0 & 0 & 0 & 0 & 0 & 0 & 0 & 0 \\
        0 & 1 & 0 & 0 & 0 & 0 & 0 & 0 & 0 & 0 \\
        0 & -1 & 1 & 0 & 0 & 0 & 0 & 0 & 0 & 0 \\
        0 & 0 & -2 & 1 & 0 & 0 & 0 & 0 & 0 & 0 \\
        0 & 0 & 1 & -3 & 1 & 0 & 0 & 0 & 0 & 0 \\
        0 & 0 & 0 & 3 & -4 & 1 & 0 & 0 & 0 & 0 \\
        0 & 0 & 0 & -1 & 6 & -5 & 1 & 0 & 0 & 0 \\
        0 & 0 & 0 & 0 & -4 & 10 & -6 & 1 & 0 & 0 \\
        0 & 0 & 0 & 0 & 1 & -10 & 15 & -7 & 1 & 0
        \end{array}\right)
    \end{displaymath}

  \caption[$\mathcal{C}$ and $\mathcal{C}^{-1}$]{Two $10$-minors of
  $\mathcal{C}$ and $\mathcal{C}^{-1}$ matrix expansions, respectively}

  \label{tab:catalan:array} 

\end{table}


In the literature, matrix $\mathcal{C}$ is known as \emph{Catalan triangle} and
it turns out to be a Riordan array. Those arrays form a family of matrices
enjoying nice properties and, looking at them from the perspective of abstract
algebra, they are a \textit{group}. Formally, a Riordan array $\mathcal{R}$ is
denoted by a pair of functions $d$ and $h$ which provides a unique
characterization for the generic coefficient $d_{n,k}\in\mathcal{R}$ lying at
row $n$ and column $k$, by means of the $n$-th coefficient extraction from
$\displaystyle d_{n,k} = [t^{n}]d(t)h(t)^{k}$.
Riordan arrays have been introduced by Shapiro et al. \cite{shapiro:1991} in $1991$;
thereafter, in $1994$, Sprugnoli \cite{sprugnoli:1991} pointed out their importance
to solve combinatorial sums.

In this paper we apply a modular transformation to $\mathcal{C}$ which maps
every coefficient to its remainder with respect to the congruence relation
$\equiv_{2}$: the resulting matrix is denoted by $\mathcal{C}_{\equiv_{2}}$ and
is graphically represented in
\autoref{fig:catalan-traditional-standard-ignore-negatives-centered-colouring-127-rows-mod2-partitioning-triangle}.
This is a parallel study of the same modular transformation applied to the
Pascal array $\mathcal{P}$, represented in
\autoref{fig:pascal-standard-handle-negatives-centered-colouring-127-rows-mod2-partitioning-triangle},
where a coefficient lying at row $n$ and column $k$ counts the number of
subsets with $k$ elements out of a set with $n$ elements, ${{n}\choose{k}}$ in
symbols.  In both pictures, we center the root of the triangle and represent
every coefficient with a colored dot, using colors blue and orange for
\textcolor{blue}{\emph{even}} and \textcolor{orange}{\emph{odd}} remainders,
respectively.

Although the array $\mathcal{P}_{\equiv_{2}}$ has been deeply studied already,
because of both its (i)~relation to congruences about binomial coefficients and
(ii)~connection with the Sierpinski gasket \cite{sokolov,
stewart:four:encounters:sierpinski}, to the best of our knowledge the array
$\mathcal{C}_{\equiv_{2}}$ seems to be fresh and our approach interleaves with
studies about modular transformations of Catalan numbers
\cite{alter:kubota:prime:power:catalan:divisibility,
egecioglu:parity:via:lattice:paths,
konvalinka:divisibility:generalized:catalan:numbers}, where their remainders 
have been characterized both arithmetically and combinatorially.  

The present study deepens our previous work \cite{merlini:nocentini:lecco} and
takes the long track toward to a formal characterization of
$\mathcal{C}_{\equiv_{2}}$; first, using the Riordan array definition of
$\mathcal{C}$ we find different closed formulae for the coefficient
$c_{n,k}\in\mathcal{C}$; second, we prove congruences over coefficients lying
in different regions of the matrix; third, we use such congruences to implement
an efficient procedure that builds $\mathcal{C}_{\equiv_{2}}$ inductively.
To support our work, a bunch of functions have been implemented using the
Python programming language on top of the Sage framework \cite{sage}, that
allow us (i)~to table a Riordan array the hard way by series expansion, (ii)~to
generate the \LaTeX\,code of graphical representations of triangles and, finally,
(iii)~to build $\mathcal{C}_{\equiv_{2}}$ in a cheaper way.


\begin{figure}[htb]

    %\hspace{-1.5cm}
    \noindent\makebox[\textwidth]{
        \centering
        %\includegraphics[width=0.8\textwidth]{../../sympy/catalan/coloured.pdf}

        % using *angle* property to rotate it is difficult to properly align it
        % in order to have a "real" matrix representation.
        \includegraphics[width=20cm, height=20cm, keepaspectratio=true]{../sympy/catalan/catalan-traditional-standard-ignore-negatives-centered-colouring-127-rows-mod2-partitioning-triangle.pdf}
    }

    % this 'particular' line is necessary to use `displaymath' environment
    % into the caption environment, togheter with the inclusion of 
    % `caption' package. See here for more explanation:
    % http://stackoverflow.com/questions/2716227/adding-an-equation-or-formula-to-a-figure-caption-in-latex
    \captionsetup{singlelinecheck=off}
    \caption[$\mathcal{C}_{\equiv_{2}}$]{
        Modular Catalan triangle $\mathcal{C}_{\equiv_{2}}$
        \iffalse
        Catalan traditional triangle, formally: 
        \begin{displaymath}
            \mathcal{C}=\left(\frac{1-\sqrt{1-4 \, t}}{2 \, t}, \frac{1-\sqrt{1-4 \, t}}{2}\right)
        \end{displaymath} % \newline % new line no more necessary
        standard, ignore negatives, centered colouring, 127 rows, mod2 partitioning
        \fi
        }

    \label{fig:catalan-traditional-standard-ignore-negatives-centered-colouring-127-rows-mod2-partitioning-triangle}

\end{figure}


\begin{figure}[p]

    \noindent\makebox[\textwidth]{
        \centering
        %\includegraphics[width=0.8\textwidth]{../../sympy/catalan/coloured.pdf}

        % using *angle* property to rotate it is difficult to properly align it
        % in order to have a "real" matrix representation.
        \includegraphics[width=20cm, height=20cm, keepaspectratio=true]{pascal/pascal-standard-handle-negatives-centered-colouring-127-rows-mod2-partitioning-triangle.pdf}
    }

    % this 'particular' line is necessary to use `displaymath' environment
    % into the caption environment, togheter with the inclusion of 
    % `caption' package. See here for more explanation:
    % http://stackoverflow.com/questions/2716227/adding-an-equation-or-formula-to-a-figure-caption-in-latex
    \captionsetup{singlelinecheck=off}
    \caption[.]{$\mathcal{P}_{\equiv_{2}}$}

    \label{fig:pascal-standard-handle-negatives-centered-colouring-127-rows-mod2-partitioning-triangle}

\end{figure}


Our work is organized as follows: the current section introduces Catalan
numbers and the concept of Riordan arrays together with the modular
transformation under study; \Cref{sec:previous:work} recalls important results
from the existing literature about the application of modular arithmetic to
infinite, lower triangular matrices; \Cref{sec:C:precisely} formally defines
the Catalan array $\mathcal{C}$ and presents some facts about it;
\Cref{sec:catalan:characterization} takes the matrix $\mathcal{C}_{\equiv_{2}}$
apart in order to reason on it piece-wise; finally, \Cref{sec:conclusions}
concludes with final comments and leaves open questions.

 %In particular, we report a
%fundamental theorem about binomial coefficients, proved by Lucas
%in \cite{lucas:theorie:des:nombres}, and
%how it has been used by various authors to establish new and curious results.
