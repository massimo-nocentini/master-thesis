
\section{Introduction}

\noindent Catalan numbers were introduced with this name back in $1838$ by
Eugene Catalan, although they have been used even before by Ming Antu
in $1730$. Let $c_{n}$ be the $n$-th Catalan number, then the following
table shows the first fifteen element of the infinite sequence 
$\lbrace c_{n}\rbrace_{n\in\mathbb{N}}$ ($A000108$\footnote{All integer sequences
defined in this paper are recorded in Sloane \cite{sloane:oeis}: keys of
the form $Annnnnn$ refer to entries in that database.}):
\begin{displaymath}
    \footnotesize
    \begin{array}{c|ccccccccccccccc}
        n & 0 & 1 & 2 & 3 & 4 & 5 & 6 & 7 & 8 & 9 & 10 & 11 & 12 & 13 & 14 \\
        \hline
        c_{n} & 1 & 1 & 2 & 5 & 14 & 42 & 132 & 429 & 1430 & 4862 & 16796 & 58786 & 208012 & 742900 & 2674440
    \end{array}
\end{displaymath}

A comprehensive manuscript entirely devoted to these numbers has been recently
published by Stanley \cite{stanley:2015}, where it is possible to find their history,
formal characterizations using generating functions and, finally, $214$ combinatorial 
interpretations.  For the sake of clarity, we report three classes of objects counted by 
$c_{n}$: first, binary trees with $n$ nodes; 
second, admissible bracketing of a string with length $n+1$ using $n-1$ pairs
of curly braces; at last, Dyck paths of length $2n$ using $\diagup$ and $\diagdown$ steps.

It is possible to augment $c_{n}$ with an additional dimension,
obtaining coefficients $c_{n,k}$ defining a lower triangular infinite 
matrix $\mathcal{C}$ ($A033184$): in \autoref{tab:catalan:array} 
we report the upper chunk of that matrix where a coefficient at row $n$ 
and column $k$ can be interpreted as the number
of Dyck paths of length $2(n+1)$ with $k$ returns to the ground, start
and end points excluded -- conventionally, indexes are $0$-based.

% inclusion of matrix expansion for C and its inverse.

\begin{table}
    \begin{displaymath} 
        \hspace{-5.5cm}
        \mathcal{C}_{10} \left(\begin{array}{rrrrrrrrrr}
        1 & 0 & 0 & 0 & 0 & 0 & 0 & 0 & 0 & 0 \\
        1 & 1 & 0 & 0 & 0 & 0 & 0 & 0 & 0 & 0 \\
        2 & 2 & 1 & 0 & 0 & 0 & 0 & 0 & 0 & 0 \\
        5 & 5 & 3 & 1 & 0 & 0 & 0 & 0 & 0 & 0 \\
        14 & 14 & 9 & 4 & 1 & 0 & 0 & 0 & 0 & 0 \\
        42 & 42 & 28 & 14 & 5 & 1 & 0 & 0 & 0 & 0 \\
        132 & 132 & 90 & 48 & 20 & 6 & 1 & 0 & 0 & 0 \\
        429 & 429 & 297 & 165 & 75 & 27 & 7 & 1 & 0 & 0 \\
        1430 & 1430 & 1001 & 572 & 275 & 110 & 35 & 8 & 1 & 0 \\
        4862 & 4862 & 3432 & 2002 & 1001 & 429 & 154 & 44 & 9 & 1
        \end{array}\right) 
        \quad
        \mathcal{C}_{10}^{-1}\left(\begin{array}{rrrrrrrrrr}
        1 & 0 & 0 & 0 & 0 & 0 & 0 & 0 & 0 & 0 \\
        1 & 0 & 0 & 0 & 0 & 0 & 0 & 0 & 0 & 0 \\
        0 & 1 & 0 & 0 & 0 & 0 & 0 & 0 & 0 & 0 \\
        0 & -1 & 1 & 0 & 0 & 0 & 0 & 0 & 0 & 0 \\
        0 & 0 & -2 & 1 & 0 & 0 & 0 & 0 & 0 & 0 \\
        0 & 0 & 1 & -3 & 1 & 0 & 0 & 0 & 0 & 0 \\
        0 & 0 & 0 & 3 & -4 & 1 & 0 & 0 & 0 & 0 \\
        0 & 0 & 0 & -1 & 6 & -5 & 1 & 0 & 0 & 0 \\
        0 & 0 & 0 & 0 & -4 & 10 & -6 & 1 & 0 & 0 \\
        0 & 0 & 0 & 0 & 1 & -10 & 15 & -7 & 1 & 0
        \end{array}\right)
    \end{displaymath}

  \caption[$\mathcal{C}$ and $\mathcal{C}^{-1}$]{Two $10$-minors of
  $\mathcal{C}$ and $\mathcal{C}^{-1}$ matrix expansions, respectively}

  \label{tab:catalan:array} 

\end{table}


In the literature, matrix $\mathcal{C}$ is known as \emph{Catalan triangle} and
it turns out to be a Riordan array. Those arrays forms a family of matrices enjoying
nice properties and, looking at them from the abstract algebra perspective, 
form a group structure. 
Formally, a Riordan array is denoted by a pair of functions $d$ and $h$ which provide a unique 
characterization for the generic coefficient $d_{n,k}$, lying at row $n$ and column $k$, 
by extraction of the $n$-th coefficient from the following $k$-convolution:
\begin{equation}
    d_{n,k} = [t^{n}]d(t)h(t)^{k}
    \label{eq:Riordan:array:coefficient}
\end{equation}
Riordan arrays have been introduced by Shapiro et al. \cite{shapiro:1991} in $1991$;
furthermore, Sprugnoli \cite{sprugnoli:1991} in $1994$ pointed out their importance
to solve combinatorial sums.

In this paper we apply a modular transformation to $\mathcal{C}$ where every coefficient 
is taken modulo $2$, denoting the resulting matrix with $\mathcal{C}_{\equiv_{2}}$ and
depicted in \autoref{fig:catalan-traditional-standard-ignore-negatives-centered-colouring-127-rows-mod2-partitioning-triangle}.
This is a parallel study of the same modular transformation applied to the Pascal array $\mathcal{P}$,
depicted in \autoref{fig:pascal-standard-handle-negatives-centered-colouring-127-rows-mod2-partitioning-triangle},
where a coefficient lying at row $n$ and column $k$ counts the number of subsets
with $k$ elements of a set with $n$ elements, in other words ${{n}\choose{k}}$.
In both pictures, we center the root of the triangle and represent every coefficient 
with a coloured dot, using \textcolor{blue}{blue} for \emph{evens} and \textcolor{orange}{orange} 
for \emph{odds} remainders, respectively. 

Although array $\mathcal{P}_{\equiv_{2}}$ has been studied in depth both because its 
relation to congruences about binomial coefficients and its connection with
the Sierpinski gasket \cite{stewart:four:encounters:sierpinski} \cite{sokolov},
to the best of our knowledge array $\mathcal{C}_{\equiv_{2}}$ seems to be fresh.  
Our study takes the long track to formally characterize $\mathcal{C}_{\equiv_{2}}$,
getting back an efficient procedure to build it as a reward: first,
using the Riordan array definition of $\mathcal{C}$, we find different closed form for
the generic coefficient $c_{n,k}\in\mathcal{C}$; second, we prove congruences over coefficients
lying in different regions of the matrix; lastly, we use such congruences to implement a procedure
that builds $\mathcal{C}_{\equiv_{2}}$ inductively. 

To support our work, a bunch of functions using the Python 
programming language have been implemented on top of the Sage framework 
\cite{sage}: these allow us to table expansion of a Riordan array
resorting to Taylor series -- the hard way --, to generate \TeX\,code
for graphical triangles representations and, finally, to build $\mathcal{C}_{\equiv_{2}}$ 
in a cheaper way.


\begin{figure}[p]

    \noindent\makebox[\textwidth]{
        \centering
        %\includegraphics[width=0.8\textwidth]{../../sympy/catalan/coloured.pdf}

        % using *angle* property to rotate it is difficult to properly align it
        % in order to have a "real" matrix representation.
        \includegraphics[width=20cm, height=20cm, keepaspectratio=true]{pascal/pascal-standard-handle-negatives-centered-colouring-127-rows-mod2-partitioning-triangle.pdf}
    }

    % this 'particular' line is necessary to use `displaymath' environment
    % into the caption environment, togheter with the inclusion of 
    % `caption' package. See here for more explanation:
    % http://stackoverflow.com/questions/2716227/adding-an-equation-or-formula-to-a-figure-caption-in-latex
    \captionsetup{singlelinecheck=off}
    \caption[.]{$\mathcal{P}_{\equiv_{2}}$}

    \label{fig:pascal-standard-handle-negatives-centered-colouring-127-rows-mod2-partitioning-triangle}

\end{figure}


This paper is structured as follows: in this section 
Catalan numbers and the concept of Riordan arrays have been introduced, moreover we stated the modular
transformation subject of our study; in \autoref{sec:previous:work} important
results from existing literature about the application of modular
arithmetic to infinite lower triangular matrices are refreshed;
in \autoref{sec:C:precisely} Catalan array $\mathcal{C}$ is formally defined
and some facts about it are derived; 
in \autoref{sec:catalan:characterization} $\mathcal{C}_{\equiv_{2}}$ is taken apart 
in order to reason on it piecewise; finally, \autoref{sec:conclusions} concludes
with final comments and open questions.

 %In particular, we report a
%fundamental theorem about binomial coefficients, proved by Lucas
%in \cite{lucas:theorie:des:nombres}, and
%how it has been used by various authors to establish new and curious results.
