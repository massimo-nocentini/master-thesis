

\section{Previous work}
\label{sec:previous:work}

First of all, we recall some modular arithmetic fundamentals. Let
$a,b\in\mathbb{Z}$ and $n\in\mathbb{N}$, \textit{$a$ is congruent to $b$ modulo
$n$}, written as $a \equiv_{n} b$, if $a \mod n = b \mod n$; from this, follows
that $n \left| (a-b)\right.$ and $a - b = nk$, for some $k\in\mathbb{Z}$, are
equivalent definitions.

%\subsection{Lucas theorem exposed}

In \cite{fine:1947}, Fine exposes the following theorem which is a fundamental
result for our work, 

\begin{theorem}[Lucas] Let $p$ be a prime and $m,n\in\mathbb{N}$, then
    \begin{displaymath}
        {{m}\choose{n}} \equiv_{p} 
            {{m_{0}}\choose{n_{0}}} 
            {{m_{1}}\choose{n_{1}}} 
            \cdots 
            {{m_{k}}\choose{n_{k}}} 
    \end{displaymath}
    representing $m=\left(m_{0},m_{1},\ldots,m_{k}\right)_{p}$ and
    $n=\left(n_{0},n_{1},\ldots,n_{k}\right)_{p}$ in base $p$, for some $k\in\mathbb{N}$.
    \iffalse
    So both $m_{j}$ and $n_{j}$ are in $\lbrace 0,\ldots, p-1 \rbrace$, for
    $j\in\lbrace 0,\ldots,k\rbrace$.
    \fi
    \label{thm:lucas:theorem}
\end{theorem}

\iffalse
\begin{proof}
    Consider the sequence
    $\vect{\alpha}_{m}=\left({{m}\choose{n}}\right)_{n\in\mathbb{N}}$ of
    binomial coefficients which count the number of $n$-elements subsets out of
    a set composed by $m$ elements. We use $\vect{\alpha}_{m}$ to
    define a formal power series on the indeterminate variable $t$ and its generating
    function
    \begin{displaymath}
        %\begin{split}
            \sum_{n=0}^{m}{{{m}\choose{n}}\,t^{n}} %&
                = \left(1+t\right)^{m}
                = \left(1+t\right)^{m_{0}+m_{1}p+\cdots+m_{k}p^{k}}%\\
                %&
                = \prod_{r=0}^{k}{{\left(1+t\right)^{m_{r}p^{r}}}}
                = \prod_{r=0}^{k}{\left(\left(1+t\right)^{p^{r}}\right)^{m_{r}}}
        %\end{split}
    \end{displaymath} 
    where $m=\left(m_{0},m_{1},\ldots,m_{k}\right)_{p}$ is represented in 
    base $p$. The expansion of $\left(1+t\right)^{p^{r}}$ yields
    \begin{displaymath}
            \left(1+t\right)^{p^{r}} = \sum_{s=0}^{p^{r}}{{{p^{r}}\choose{s}}\,t^{s}}
                = 1+\sum_{s=1}^{p^{r}-1}{{{p^{r}}\choose{s}}\,t^{s}}+t^{p^{r}}
    \end{displaymath}
    and the application of the congruence relation $\equiv_{p}$ makes the last sum
    vanish because $\displaystyle p\left|{{p^{r}}\choose{s}}\right.$, therefore
    $\left(1+t\right)^{p^{r}} \equiv_{p} 1+t^{p^{r}}$ holds and it allows us to rewrite
    the products as
    \begin{displaymath}
            \prod_{r=0}^{k}{\left(\left(1+t\right)^{p^{r}}\right)^{m_{r}}}
                \equiv_{p} \prod_{r=0}^{k}{\left(1+t^{p^{r}}\right)^{m_{r}}}
                \equiv_{p} \prod_{r=0}^{k}{\sum_{s_{r}=0}^{m_{r}}{{{m_{r}}\choose{s_{r}}}t^{s_{r}p^{r}}}}.
    \end{displaymath}
    there is a product of $k+1$ formal power series, properly shifted according to 
    $p_{0},p_{1},\ldots,p_{k}$: such product yields a new formal power series which satisfies
    \begin{displaymath}
        \sum_{n=0}^{m}{{{m}\choose{n}}\,t^{n}} 
        \equiv_{p}
        \sum_{n=0}^{m}{\left(\sum_{\vect{\omega}\in\Omega_{n}}{\prod_{i=0}^{k}{{{m_{i}}\choose{\omega_{i}}}}}\right)\,t^{n}}
    \end{displaymath}
    where $\Omega_{n}=\left\lbrace\vect{\omega}=\left(\omega_{0},\ldots,\omega_{k}\right):
        \sum_{j=0}^{k}{\omega_{j}p^{j}}=n\right\rbrace$.
    There exists a \emph{unique} set of coefficients
    $\lbrace n_{0},n_{1},\ldots,n_{k}\rbrace$ such that $n=\sum_{j=0}^{k}{n_{j}p^{j}}$, therefore
    each set $\Omega_{n}$ contains only one element, namely $\vect{\omega}=(n_{0},\ldots,n_{k})$, so
    \begin{displaymath}
        \sum_{n=0}^{m}{{{m}\choose{n}}\,t^{n}} 
        \equiv_{p} \sum_{n=0}^{m}{\left({\prod_{i=0}^{k}{{{m_{i}}\choose{n_{i}}}}}\right)\,t^{n}}
    \end{displaymath}
    holds and the argument follows by equating coefficients attached to $t^{n}$, 
    %for $n\in\lbrace0,\ldots,m\rbrace$, 
    as required.

\end{proof}
\fi

%\subsection{Divisibility -- with visibility}

In \cite{sved:1988}, Marta Sved gives solid bases to understand
\textit{divisibility properties} of some \emph{counting numbers}, namely
coefficients occurring in combinatorics such as binomials and Stirling numbers,
that shows divisibility structures of remarkable design. 

She starts working on the Pascal triangle $\mathcal{P}$, whose coefficients
are defined by the recurrence
\begin{equation}
    {{n}\choose{k}}={{n-1}\choose{k-1}}+{{n-1}\choose{k}},
    \label{eq:binomial:recurrence}
\end{equation}
and studies a new triangle $\mathcal{P}_{\equiv_{p}}$ obtained from
$\mathcal{P}$ where each coefficient is taken modulo $p$, for some small prime
number $p$. Besides, she introduces a little language that allows her to
characterize these "modular" triangles and we readly adopt it as well.

Let $\mathcal{M}=\left(m_{n,k}\right)_{n,k\in\mathbb{N}}$ be an infinite, lower
triangular array so that $m_{n,k}=0$ if $n<k$. Choose a prime $p$, then the
array $\displaystyle\mathcal{M}_{\equiv_{p}} = \left(m_{n,k}\mod p :
m_{n,k} \in \mathcal{M}\right)_{n,k\in\mathbb{N}}$ contains the following regions
\begin{itemize}

\item a \emph{principal cell} $\mathcal{M}_{\equiv_{p}}^{\bigtriangleup} = 
        \left(m_{n,k}\in \mathcal{M}_{\equiv_{p}}\right)_{n,k\in\lbrace 0,\ldots,p-1\rbrace}$;

\item a \emph{cell} $c_{p,\beta}=\left( \beta\,m_{n,k}\mod p:m_{n,k}\in
\mathcal{M}_{\equiv_{p}}^{\bigtriangleup}\right)_{n,k\in\mathbb{N}}$ which is a
copy of the \emph{principal cell} with each entry multiplied by the same
constant $\beta\in\mathbb{N}$;

\item a \emph{principal cluster} $\mathcal{M}_{\equiv_{p}}^{(\alpha)} =
\left(m_{n,k}\in \mathcal{M}_{\equiv_{p}}\right)_{n,k\in\lbrace
0,\ldots,p^{\alpha}-1\rbrace}$ of order $\alpha\in\mathbb{N}$, so that for
$\alpha=1$ it denotes the \emph{principal cell};

\item a \emph{cluster} $\mathcal{C}_{p,\beta}^{(\alpha)}=\left(
\beta\,m_{n,k}\mod p:m_{n,k}\in
\mathcal{M}_{\equiv_{p}}^{(\alpha)}\right)_{n,k\in\mathbb{N}}$ of order
$\alpha\in\mathbb{N}$ which is a copy of the \emph{principal cluster} of order
$\alpha$ with each entry multiplied by the same constant $\beta\in\mathbb{N}$;

\item finally, a \emph{zero-hole} of order $\alpha\in\mathbb{N}$ which is an
\emph{upside-down} triangular array of coefficients where each one is a
multiple of $p$, from here the name, distributed according to the scheme
(i)~$p^{\alpha}-1$ coefficients lie on the first row, (ii)~$p^{\alpha}-2$
coefficients lie on the second row down to (iii)~a single coefficient that lies
on the $(p^{\alpha}-1)$-th row, eventually.  Observe that a \emph{zero-hole} of
order $\alpha$ has $p^{\alpha}-1$ rows, one less than a \emph{principal
cluster} of the same order.

\end{itemize}


Although a formal proof can be stated by induction using
\autoref{eq:binomial:recurrence}, Marta shows that the following  theorem  has
an easier formulation using the proposed language.

\begin{theorem}[Sved]
    Let $p$ be a prime and $\mathcal{P}_{\equiv_{p}}$ be the 
    modular Pascal array. The principal cluster $\mathcal{P}_{\equiv_{p}}^{(\alpha+1)}$
    of order $\alpha + 1$ consists of $p$ layers of clusters of order $\alpha$ alternating
    with zero-holes of order $\alpha$, while the first layer contains the principal
    cluster of order $\alpha$ only.
\end{theorem}
