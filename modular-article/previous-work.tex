

\section{Previous work}
\label{sec:previous:work}


First of all, we recall some modular arithmetic fundamentals. Let $a,b\in\mathbb{Z}$
and $n\in\mathbb{N}$, to say that $a$ is congruent to $b$ modulo $n$,
written as $a \equiv_{n} b$, is the same to say:
\begin{displaymath}
    a\mod n = b\mod n\quad \leftrightarrow\quad n \left| (a-b)\right. \quad\leftrightarrow \quad a - b = nk
\end{displaymath}
for some $k\in\mathbb{Z}$.

\subsection{Lucas theorem exposed}

In \cite{fine:1947}, Fine exposes Lucas theorem: we state it and report a proof since
it is a fundamental result for our work. 

\begin{theorem}[Lucas theorem]
    Let $p$ be a prime and $m,n\in\mathbb{N}$. Then:
    \begin{displaymath}
        {{m}\choose{n}} \equiv_{p} 
            {{m_{0}}\choose{n_{0}}} 
            {{m_{1}}\choose{n_{1}}} 
            \cdots 
            {{m_{k}}\choose{n_{k}}} 
    \end{displaymath}
    representing $m=\left(m_{0},m_{1},\ldots,m_{k}\right)_{p}$ and
    $n=\left(n_{0},n_{1},\ldots,n_{k}\right)_{p}$ in base $p$, for some $k\in\mathbb{N}$.
    So both $m_{j}$ and $n_{j}$ are in $\lbrace 0,\ldots, p-1 \rbrace$, for
    $j\in\lbrace 0,\ldots,k\rbrace$.
    \label{thm:lucas:theorem}
\end{theorem}

\begin{proof}
    Consider the sequence $\vect{\alpha}_{m}=\left\lbrace{{m}\choose{n}}\right\rbrace_{n\in\mathbb{N}}$ 
    of coefficients which counts the number of subset
    of a set composed by $m$ elements. We can use sequence $\vect{\alpha}_{m}$ to define a 
    formal power series on the indeterminate variable $t$ and manipulate:
    \begin{displaymath}
        %\begin{split}
            \sum_{n=0}^{m}{{{m}\choose{n}}\,t^{n}} %&
                = \left(1+t\right)^{m}
                = \left(1+t\right)^{m_{0}+m_{1}p+\cdots+m_{k}p^{k}}%\\
                %&
                = \prod_{r=0}^{k}{{\left(1+t\right)^{m_{r}p^{r}}}}
                = \prod_{r=0}^{k}{\left(\left(1+t\right)^{p^{r}}\right)^{m_{r}}}
        %\end{split}
    \end{displaymath} where $m=\left(m_{0},m_{1},\ldots,m_{k}\right)_{p}$ represented in 
    base $p$. The expansion of $\left(1+t\right)^{p^{r}}$ yield
    \begin{displaymath}
            \left(1+t\right)^{p^{r}} = \sum_{s=0}^{p^{r}}{{{p^{r}}\choose{s}}\,t^{s}}
                = 1+\sum_{s=1}^{p^{r}-1}{{{p^{r}}\choose{s}}\,t^{s}}+t^{p^{r}}
    \end{displaymath}
    and the application of congruence relation modulo $p$ makes the sum vanish since ${{p^{r}}\choose{s}}=p\,u_{s}$,
    for some $u_{s}\in\mathbb{N}$ and $s\in\lbrace 1,\ldots,p^{r}-1\rbrace$, 
    therefore $\left(1+t\right)^{p^{r}} \equiv_{p} 1+t^{p^{r}}$. On the very right hand side of 
    the rewriting
    \begin{displaymath}
            \prod_{r=0}^{k}{\left(\left(1+t\right)^{p^{r}}\right)^{m_{r}}}
                \equiv_{p} \prod_{r=0}^{k}{\left(1+t^{p^{r}}\right)^{m_{r}}}
                \equiv_{p} \prod_{r=0}^{k}{\sum_{s_{r}=0}^{m_{r}}{{{m_{r}}\choose{s_{r}}}t^{s_{r}p^{r}}}}
    \end{displaymath}
    there is a product of $k+1$ formal power series, properly shifted according to 
    $p_{0},p_{1},\ldots,p_{k}$: such product yields a new formal power series which satisfies
    \begin{displaymath}
        \sum_{n=0}^{m}{{{m}\choose{n}}\,t^{n}} 
        \equiv_{p}
        \sum_{n=0}^{m}{\left(\sum_{\vect{\omega}\in\Omega_{n}}{\prod_{i=0}^{k}{{{m_{i}}\choose{\omega_{i}}}}}\right)\,t^{n}}
    \end{displaymath}
    where $\Omega_{n}=\left\lbrace\vect{\omega}=\left(\omega_{0},\ldots,\omega_{k}\right):
        \sum_{j=0}^{k}{\omega_{j}p^{j}}=n\right\rbrace$.
    There exists a \emph{unique} set of coefficients
    $\lbrace n_{0},n_{1},\ldots,n_{k}\rbrace$ such that $n=\sum_{j=0}^{k}{n_{j}p^{j}}$, therefore
    each set $\Omega_{n}$ contains only one element, namely $\vect{\omega}=(n_{0},\ldots,n_{k})$, so
    \begin{displaymath}
        \sum_{n=0}^{m}{{{m}\choose{n}}\,t^{n}} 
        \equiv_{p} \sum_{n=0}^{m}{\left({\prod_{i=0}^{k}{{{m_{i}}\choose{n_{i}}}}}\right)\,t^{n}}
    \end{displaymath}
    holds and the argument follows by equating coefficients attached to $t^{n}$, 
    %for $n\in\lbrace0,\ldots,m\rbrace$, 
    as required.

\end{proof}


\subsection{Divisibility -- with visibility}

In \cite{sved:1988}, Marta Sved gives solid bases to understand
divisibility properties of some \emph{counting numbers}, namely, coefficients
occurring in combinatorics, such as binomials and Stirling numbers. Many of
such coefficients shows divisibility structures of remarkable design. 

She starts working on the Pascal triangle $\mathcal{P}$, whose coefficients
can be defined inductively using the recurrence
\begin{equation}
    {{n}\choose{k}}={{n-1}\choose{k-1}}+{{n-1}\choose{k}}
    \label{eq:binomial:recurrence}
\end{equation}
and studies the Pascal array where each coefficient is taken modulo $p$, 
denoted by $\mathcal{P}_{\equiv_{p}}$, for some small prime $p$. Besides, 
she introduces a little language, described in the next paragraph, 
that allows her to characterize these arrays and we adopt it as well.

Let $\mathcal{M}=\lbrace m_{n,k}\rbrace_{n,k\in\mathbb{N}}$ 
be an infinite lower triangular array, where $m_{n,k}=0$ if 
$n<k$. Choose a prime $p$, then build a modular array
$\mathcal{M}_{\equiv_{p}}$, defined as 
$\lbrace m_{n,k}\mod p : m_{n,k} \in \mathcal{M}\rbrace$, containing
the following regions:
\begin{itemize}

\item a \emph{principal cell} is % a chunk of $\mathcal{M}_{\equiv_{p}}$ 
    defined as
    $\mathcal{M}_{\equiv_{p}}^{\bigtriangleup} = 
        \left\lbrace m_{n,k}\in \mathcal{M}_{\equiv_{p}}: n\in\lbrace 0,\ldots,p-1\rbrace\right\rbrace$;

\item a \emph{cell} $c_{\beta}$ is a copy of the \emph{principal cell} with each entry multiplied 
by the same constant $\beta\in\mathbb{N}$, formally
    $c_{\beta}=\left\lbrace \beta\,m_{n,k}\mod p:m_{n,k}\in \mathcal{M}_{\equiv_{p}}^{\bigtriangleup}\right\rbrace$;

\item a \emph{principal cluster} of order $\alpha$ is %a chunk of $\mathcal{M}_{\equiv_{p}}$ 
    defined as
    $\mathcal{M}_{\equiv_{p}}^{(\alpha)} = 
        \left\lbrace m_{n,k}\in \mathcal{M}_{\equiv_{p}}: n\in\lbrace 0,\ldots,p^{\alpha}-1\rbrace\right\rbrace$. 
        The special case of $\alpha=1$ denotes the \emph{principal cell} and this concept of \emph{cluster}
        is the main one used in our proofs;

\item a \emph{cluster} $\mathcal{C}_{\beta}^{(\alpha)}$ of order $\alpha$ is a copy of the \emph{principal cluster}
of order $\alpha$ with each entry multiplied 
by the same constant $\beta\in\mathbb{N}$, formally 
    $\mathcal{C}_{\beta}^{(\alpha)}=\left\lbrace \beta\,m_{n,k}\mod p:m_{n,k}\in \mathcal{M}_{\equiv_{p}}^{(\alpha)}\right\rbrace$.

\end{itemize}

Finally, a \emph{zero-hole} of order $\alpha$ is an \emph{upside-down} triangular array of
coefficients multiples of $p$, from here the name: it consists of $p^{\alpha}-1$
coefficients lying on the first row, $p^{\alpha}-2$ coefficients lying on the second row
and a single coefficient on the $(p^{\alpha}-1)$-th row, eventually. 
Observe that a \emph{zero-hole} of order $\alpha$ has $p^{\alpha}-1$ rows, one less
than a \emph{principal cluster} of the same order.

Although a formal proof can be stated by induction using
\autoref{eq:binomial:recurrence}, Sved shows that the following fact 
about the modular Pascal array has an easier formulation using the proposed language.

\begin{theorem}
    Let $p$ be a prime and $\mathcal{P}_{\equiv_{p}}$ be the 
    modular Pascal array. The principal cluster $\mathcal{P}_{\equiv_{p}}^{(\alpha+1)}$,
    of order $\alpha + 1$, consists of $p$ layers of clusters of order $\alpha$, alternating
    with zero-holes of order $\alpha$ too. The first layer containing only the principal
    cluster of order $\alpha$.
\end{theorem}
