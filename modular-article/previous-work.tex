

\section{Previous work}

In this section we recall existing literature about the application of modular
arithmetic to infinite lower triangular matrices. In particular, we report a
fundamental theorem about binomial coefficients, proved by Lucas
in \cite{lucas:theorie:des:nombres}, and
how it has been used by various authors to establish new and curious results.

\subsection{Lucas theorem exposed}

In \cite{fine:1947}, Fine exposes Lucas theorem; we state it and report a proof since
it is a fundamental result for our work. 

\begin{theorem}[Lucas theorem]
    Let $p$ be a prime and $m,n\in\mathbb{N}$. Then:
    \begin{displaymath}
        {{m}\choose{n}} \equiv_{p} 
            {{m_{0}}\choose{n_{0}}} 
            {{m_{1}}\choose{n_{1}}} 
            \cdots 
            {{m_{k}}\choose{n_{k}}} 
    \end{displaymath}
    representing $m=\left(m_{0},m_{1},\ldots,m_{k}\right)_{p}$ and
    $n=\left(n_{0},n_{1},\ldots,n_{k}\right)_{p}$ in base $p$, for some $k\in\mathbb{N}$.
    So both $m_{j}$ and $n_{j}$ are in $\lbrace 0,\ldots, p-1 \rbrace$, for
    $j\in\lbrace 0,\ldots,k\rbrace$.
    \label{thm:lucas:theorem}
\end{theorem}

\begin{proof}
    Consider the sequence $\vect{\alpha}_{m}=\left\lbrace{{m}\choose{n}}\right\rbrace_{n\in\mathbb{N}}$ 
    of coefficients which counts the number of subset
    of a set composed by $m$ elements. Use sequence $\vect{\alpha}$ to define a formal power series:
    \begin{displaymath}
        \begin{split}
            \sum_{n=0}^{m}{{{m}\choose{n}}\,t^{n}} &= \left(1+t\right)^{m}
                = \left(1+t\right)^{m_{0}+m_{1}p+\ldots+m_{k}p^{k}}\\
                &= \prod_{r=0}^{k}{{\left(1+t\right)^{m_{r}p^{r}}}}
                = \prod_{r=0}^{k}{\left(\left(1+t\right)^{p^{r}}\right)^{m_{r}}}
        \end{split}
    \end{displaymath}
    expansion of $\left(1+t\right)^{p^{r}}$ yield: 
    \begin{displaymath}
            \left(1+t\right)^{p^{r}} = \sum_{s=0}^{p^{r}}{{{p^{r}}\choose{s}}\,t^{s}}
                = 1+\sum_{s=1}^{p^{r}-1}{{{p^{r}}\choose{s}}\,t^{s}}+t^{p^{r}}
    \end{displaymath}
    application of congruence relation modulo $p$ makes the sum vanish since ${{p^{r}}\choose{s}}=p\,u_{s}$,
    for some $u_{s}\in\mathbb{N}$ and $s\in\lbrace 1,\ldots,p^{r}-1\rbrace$, therefore:
    \begin{displaymath}
        \left(1+t\right)^{p^{r}} = 1+t^{p^{r}}
    \end{displaymath}
    we can rewrite as:
    \begin{displaymath}
        \begin{split}
            \prod_{r=0}^{k}{\left(\left(1+t\right)^{p^{r}}\right)^{m_{r}}}
                &\equiv_{p} \prod_{r=0}^{k}{\left(1+t^{p^{r}}\right)^{m_{r}}}\\
                &\equiv_{p} \prod_{r=0}^{k}{\sum_{s_{r}=0}^{m_{r}}{{{m_{r}}\choose{s_{r}}}t^{s_{r}p^{r}}}}
        \end{split}
    \end{displaymath}
    \marginpar{abstracting a convolution introducing set $\Omega_{n}$}
    on the very right hand side there is a product of $k+1$ formal power series, properly shifted according to coefficients
    $m_{0},m_{1},\ldots,m_{k}$. Such product yields a new formal power series which satisfies:
    \begin{displaymath}
        \sum_{n=0}^{m}{{{m}\choose{n}}\,t^{n}} 
        \equiv_{p}
        \sum_{n=0}^{m}{\left(\sum_{\vect{\omega}\in\Omega_{n}}{\prod_{i=0}^{k}{{{m_{i}}\choose{\omega_{i}}}}}\right)\,t^{n}}
    \end{displaymath}
    where $\Omega_{n}=\left\lbrace\vect{\omega}=\left(\omega_{0},\ldots,\omega_{k}\right):
        \sum_{j=0}^{k}{\omega_{j}p^{j}}=n\right\rbrace$.
    By the representation theorem, there exists a \emph{unique} set of coefficients
    $\lbrace n_{0},n_{1},\ldots,n_{k}\rbrace$ such that $n=\sum_{j=0}^{k}{n_{j}p^{j}}$, therefore
    each set $\Omega_{n}$ contains only one element, namely $\vect{\omega}=(n_{0},\ldots,n_{k})$, so:
    \begin{displaymath}
        \equiv_{p} \sum_{n=0}^{m}{\left({\prod_{i=0}^{k}{{{m_{i}}\choose{n_{i}}}}}\right)\,t^{n}}
    \end{displaymath}
    equating coefficients the argument follows, as required.

\end{proof}

Fine used such powerful theorem to prove additional facts
about binomial coefficients modulo a prime $p$; moreover

\subsection{Divisibility -- with visibility}

In \cite{sved:1988}, Marta Sved gives solid bases to understand
divisibility properties of some \emph{counting numbers}, that is coefficients
occurring in combinatorics, such as binomials and Stirling numbers. Many of
such coefficients shows divisibility structures of remarkable design. 

She starts working on the Pascal triangle, namely the lower triangular infinite
matrix defined inductively by the recurrence: 
\begin{equation}
    {{n}\choose{k}}={{n-1}\choose{k-1}}+{{n-1}\choose{k}}
    \label{eq:binomial:recurrence}
\end{equation}
and applies a modification to build the Pascal array modulo a prime $p$.
%reporting some interesting tables for some values of the module.

We believe that one of the core concepts she introduces is a little language
that allows her to characterize these arrays and we adopt it 
as well. In the following paragraphs we describe this language. 
\\\\
Let $\mathcal{M}=\lbrace m_{nk}\rbrace_{n,k\in\mathbb{N}}$ 
be an infinite lower triangular array, where $m_{nk}=0$ if 
$n<k$. Choose a prime $p$, then build a modular array
$\mathcal{M}_{\equiv_{p}}$ defined as 
$\lbrace m_{nk}\mod p : m_{nk} \in \mathcal{M}\rbrace$.


A \emph{principal cell} is a chunk of $\mathcal{M}_{\equiv_{p}}$ denoted by:
\begin{displaymath}
    \mathcal{M}_{\equiv_{p}}^{\bigtriangleup} = 
        \left\lbrace m_{nk}\in \mathcal{M}_{\equiv_{p}}: n\in\lbrace 0,\ldots,p-1\rbrace\right\rbrace
\end{displaymath}

A \emph{cell} $c_{\beta}$ is a copy of the \emph{principal cell} with each entry multiplied 
by the same constant $\beta\in\mathbb{N}$, formally:
\begin{displaymath}
    c_{\beta}=\left\lbrace \beta\,m_{nk}\mod p:m_{nk}\in \mathcal{M}_{\equiv_{p}}^{\bigtriangleup}\right\rbrace
\end{displaymath}

A \emph{principal cluster} of order $\alpha$ is a chunk of $\mathcal{M}_{\equiv_{p}}$ denoted by:
\begin{displaymath}
    \mathcal{M}_{\equiv_{p}}^{(\alpha)} = 
        \left\lbrace m_{nk}\in \mathcal{M}_{\equiv_{p}}: n\in\lbrace 0,\ldots,p^{\alpha}-1\rbrace\right\rbrace
\end{displaymath}

A \emph{cluster} $\mathcal{C}_{\beta}^{(\alpha)}$ of order $\alpha$ is a copy of the \emph{principal cluster}
of order $\alpha$ with each entry multiplied 
by the same constant $\beta\in\mathbb{N}$, formally:
\begin{displaymath}
    \mathcal{C}_{\beta}^{(\alpha)}=\left\lbrace \beta\,m_{nk}\mod p:m_{nk}\in \mathcal{M}_{\equiv_{p}}^{(\alpha)}\right\rbrace
\end{displaymath}

Finally, a \emph{zero-hole} of order $\alpha$ is an inverted triangular array of
coefficients multiples of $p$, from here the name, consisting of $p^{\alpha}-1$
coefficients lying on the first row, $p^{\alpha}-2$ coefficients on the second row
and a single coefficient on the $(p^{\alpha}-1)$-th row. 
Observe that a \emph{zero-hole} of order $\alpha$ has $p^{\alpha}-1$ rows, one less
than a \emph{principal cluster} of the same order.

Although a formal proof can be stated by induction using
\autoref{eq:binomial:recurrence}, Sved shows that the following fact 
about the modular Pascal array has an easier formulation using the proposed language.

\begin{theorem}
    Let $p$ be a prime and $\mathcal{P}_{\equiv_{p}}$ be the 
    modular Pascal array. The principal cluster $\mathcal{P}_{\equiv_{p}}^{(\alpha+1)}$,
    of order $\alpha + 1$, consists of $p$ layers of clusters of order $\alpha$, alternating
    with zero-holes of order $\alpha$ too. The first layer containing only the principal
    cluster of order $\alpha$.
\end{theorem}
