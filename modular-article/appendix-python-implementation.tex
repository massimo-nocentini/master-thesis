
\appendix

\section{A Python prototype}

Observation in last paragraph allows us to code a bunch of Python functions, 
lying on top of SymPy module, that produce 
$\mathcal{C}_{\equiv 2}^{(\alpha+1)}$ consuming $\mathcal{C}_{\equiv 2}^{(\alpha)}$:
the construction is strictly \emph{blockwise}, according theorems stated
in previous section. Here's the implementation:

\begin{adjustwidth}{-1cm}{0cm}
    \inputminted{python}{../../PhD/projects/recurrences-unfolding/sympy-notebook/colouring.py}
\end{adjustwidth}

The following is a session that builds $\mathcal{C}_{\equiv 2}^{(7)}$:
\begin{minted}{python}
    catalan_matrix = Matrix([
        [1,0,0,0,0,0], 
        [1,1,0,0,0,0], 
        [2,2,1,0,0,0], 
        [5,5,3,1,0,0], 
        [14,14,9,4,1,0], 
        [42,42,28,14,5,1]])
    alpha = 2
    bound = 2**alpha
    pc = catalan_matrix[:bound, :bound]
    pc = pc.applyfunc(lambda c: c % 2)
    # _ = colour_matrix(pc)
    pc = build_modular_catalan(pc)
    # _ = colour_matrix(pc)
    pc = build_modular_catalan(pc)
    # _ = colour_matrix(pc)
    pc = build_modular_catalan(pc)
    # _ = colour_matrix(pc)
    pc = build_modular_catalan(pc)
    # _ = colour_matrix(pc)
    pc = build_modular_catalan(pc)
    # _ = colour_matrix(pc)
\end{minted}

Function \mintinline{python}|build_modular_catalan| implements given theorems to avoid
computing convolutions and series expansion: the starting building block in the 
above session is $\mathcal{C}_{\equiv 2}^{(2)}$, hence it is very quick. 
If desired, it is possible to display coloured triangles as svg images:
just uncomment \mintinline{python}|# _ = colour_matrix(pc)| lines.
