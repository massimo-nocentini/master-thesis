
\section{Catalan array $\mathcal{C}$, precisely}
\label{sec:C:precisely}

% big comment: skip it, maybe the content is interesting, but stated very ugly.
\iffalse
Let $\mathcal{C}_{h}$ a principal $h$-cluster of
the Catalan array, in its traditional definition: We attempt to point out
some properties about $\mathcal{C}_{h+1}$:
\begin{itemize}
    \item a more general pattern, for any $h\in\mathcal{N}$, 
        a coefficient $d_{p^{h}-1,k}$, for $k\in\lbrace0,\ldots,p^{h}-1 \rbrace$, 
        satisfies:
        \begin{displaymath}
            d_{p^{h}-1,k} \equiv_{p} 0
        \end{displaymath}
    \item consider the subcluster $\mathcal{C}_{h}^{(0,p^{h}-1)}$ of 
        $\mathcal{C}_{h}$ obtained by removing antidiagonal $0$ (namely,
        the boundary one, composed by $1$s only) and by removing row $p^{h}-1$.
        By construction, $\mathcal{C}_{h}^{(0,p^{h}-1)}$ is a triangle too, 
        with $p^{h}-2$ coefficients per side. Therefore, $\mathcal{C}_{h+1}$
        is composed on the very top, starting at the root, by a copy of $\mathcal{C}_{h}$,
        while on the bottom there are three triangles:
        \begin{itemize}                
            \item an equilateral triangle $T_{\equiv_{p} 0}^{(h)}$, 
                with $p^{h}-1$ coefficients on each side, such that:
                \begin{displaymath}
                    d_{nk} \in T_{\equiv_{p} 0}^{(h)} \rightarrow d_{nk} \equiv_{p} 0
                \end{displaymath}
                where $n\in\lbrace p^{h},\ldots,p^{h+1}-2\rbrace$ and 
                $k\in\lbrace 0,\ldots, n-p^{h}\rbrace$, in other words
                $k\in\lbrace 0,\ldots, p^{h}(p-1)-2\rbrace$;
            \item a mirror copy of $\mathcal{C}_{h}^{(0,p^{h}-1)}$ respect to column orientation,
                defining coefficients $d_{nk}$  
                where $n\in\lbrace p^{h},\ldots,p^{h+1}-2\rbrace$ and 
                $k\in\lbrace 1,\ldots, p^{h}-2\rbrace$;
            \item a segment of coefficients $d_{n, p^{h}-1} \equiv_{p} 0$ 
                where $n\in\lbrace p^{h},\ldots,p^{h+1}-2\rbrace$;
            \item a copy of $\mathcal{C}_{h}^{(0,p^{h}-1)}$,
                defining coefficients $d_{nk}$  
                where $n\in\lbrace p^{h}+1,\ldots,p^{h+1}-2\rbrace$ and 
                $k\in\lbrace p^{h},\ldots, n-1\rbrace$;
        \end{itemize}                
\end{itemize}
\fi

In this section we would like to give a quick description of
array $\mathcal{C}$ and introduce some mathematical objects
that will be used later. This array is defined as follow:
\begin{displaymath}
    \mathcal{C}=\left(\frac{1-\sqrt{1-4\,t}}{2\,t},
        \frac{1-\sqrt{1-4\,t}}{2}\right)
\end{displaymath}
since $h_{\mathcal{C}}(t)=t\,d_{\mathcal{C}}(t)$, then $\mathcal{C}$
is in the \emph{renewal} subgroup.

Denote with $c_{j}$ the $j$-th Catalan coefficient and, 
by definition of array $\mathcal{C}$, observe that:
\begin{displaymath}
    c_{j} = [t^{j}]d_{\mathcal{C}}(t)= [t^{j}]\frac{1-\sqrt{1-4\,t}}{2\,t}
\end{displaymath}
and since $\mathcal{C}$'s $A$-sequence is:
\begin{displaymath}
    A_{\mathcal{C}}(t)=\frac{1}{1-t}=1+t+t^{2}+t^{3}+t^{4}+t^{5}+t^{6}+t^{7}+t^{8}+
        \mathcal{O}(t^{9})
\end{displaymath}
each coefficient $c_{j}$ can be written as the following combination:
\begin{displaymath}
    c_{j} = \sum_{k=0}^{j-1}{d_{j-1,k}}
\end{displaymath}
in other words, $c_{j}$ equals the $(j-1)$-row sum of $\mathcal{C}$, 
therefore by the \emph{fundamental theorem} of the Riordan group:

\begin{displaymath}
    c_{j} = [t^{j-1}]\mathcal{C}\,A_{\mathcal{C}}(t)
          = [t^{j-1}]d_{\mathcal{C}}(t)\,A_{\mathcal{C}}(h_{\mathcal{C}}(t))
          = [t^{j}]h_{\mathcal{C}}(t)\,A_{\mathcal{C}}(h_{\mathcal{C}}(t))
\end{displaymath}
yielding:
\begin{displaymath}
    c_{j} = [t^{j}]\frac{1-\sqrt{1-4 \, t}}{1+\sqrt{1-4 \, t}}
\end{displaymath}
with boundary condition $c_{0}=1$, which cannot be read from the above
series expansion. Moreover, from the above equation, it is possible to derive
another generating function for the sequence of Catalan numbers:

\begin{displaymath}
    \frac{1-\sqrt{1-4 \, t}}{1+\sqrt{1-4 \, t}}=
    \frac{\left(1-\sqrt{1-4 \, t}\right)^{2}}{4 \, t}=
    \frac{1-2\,t-\sqrt{1-4 \, t}}{2 \, t}
\end{displaymath}

% no more necessary?
\iffalse
Go back to identity:
\begin{displaymath}
    d_{nk} = \sum_{i_{1}+ i_{2}+ \ldots+ i_{k+1}=n+1}{
            c_{i_{1}-1}\,c_{i_{2}-1}\,\ldots\,c_{i_{k+1}-1} }
\end{displaymath}
\fi

Coefficient extraction on previous generating function allow us to write 
a Catalan coefficient $c_{j}$ in a closed form:
\begin{equation}
    c_{j} = \frac{1}{j+1}{{2j}\choose{j}} 
        = {{2j}\choose{j}} - {{2j}\choose{j+1}}
    \label{eq:catalan:coeff:rewriting}
\end{equation}

On the other hand, the generic element
$d_{nk}\in\mathcal{C}$ admits closed formulae too. According to
\cite{luzon:2012631}, two of them are of
particular interest in order to complete our proofs:
\begin{equation}
    d_{nk}=\frac{k+1}{n+1}{{2n-k}\choose{n-k}}
    \label{eq:catalan:array:first:identity}
\end{equation}
and:
\begin{equation}
    d_{nk}={{2n-k}\choose{n-k}} - {{2n-k}\choose{n-k-1}}
    \label{eq:catalan:array:second:identity}
\end{equation}


The last object we want to
introduce is a pair that allow to talk about a subset of elements lying on a
column $\vect{c}_{k}$ of $\mathcal{C}$. Let $S\subseteq\mathbb{N}$  
and let $k\in\mathbb{N}$, then a \emph{segment} of column $\vect{c}_{k}$ is
denoted by the following relation:
\begin{equation}
    \diagup_{\lbrace\alpha_{i}\rbrace_{i\in S}}^{k}
        = \left(\vect{c}_{k},\lbrace k+\alpha_{i}\rbrace_{i\in S}\right)
        = \lbrace d_{k+\alpha_{i},k}\rbrace_{i\in S}
    \label{eq:column:segment}
\end{equation}
where $\lbrace\alpha_{i}\rbrace_{i\in S}\subseteq\mathbb{N}$. 
In order to fetch the set of 
row indices belonging to a segment we introduce the $rows$ function, defined as follow:
\begin{displaymath}
    rows\left(\diagup_{\lbrace\alpha_{i}\rbrace_{i\in S}}^{k}\right)
        = rows\left(\left(\vect{c}_{k},\lbrace k+\alpha_{i}\rbrace_{i\in S}\right)\right)
        = \lbrace k+\alpha_{i}\rbrace_{i\in S}
    \label{eq:column:segment:rows:function}
\end{displaymath}


In order to formally proof the modular characterization $\mathcal{C}_{\equiv_{2}}$, 
write the generic element $d_{nk}\in\mathcal{C}$ using the definition:
\begin{displaymath}
    d_{nk} = [t^n] \frac{1-\sqrt{1-4\,t}}{2\,t}\,
        \left(\frac{1-\sqrt{1-4\,t}}{2}\right)^{k}
           = [t^{n+1}] \left(\frac{1-\sqrt{1-4\,t}}{2}\right)^{k+1} 
\end{displaymath}
the right most term requires to extract coefficient $n+1$ from the $(k+1)$-fold
convolution of Catalan numbers' generating function with itself, \emph{shifted
by one place}. Before going on, careful
understanding of what coefficients are introduced is necessary. 
The following expansion defines a sequence of coefficients 
$\lbrace \hat{c}_{i}\rbrace_{i\in\mathbb{N}}$:
\begin{displaymath}
    \frac{1-\sqrt{1-4\,t}}{2} = t + t^{2} + 2 t^{3} + 5 t^{4} 
        + 14 t^{5} + 42 t^{6} + 132 t^{7} %+ 429 t^{8} + 1430 t^{9} 
        + \mathcal{O}\left(t^{8}\right)
\end{displaymath}
where $\hat{c}_{0}=0$. On the other hand the following one, defines a sequence 
of coefficients $\lbrace c_{i}\rbrace_{i\in\mathbb{N}}$, 
the traditional Catalan numbers:
\begin{displaymath}
    \frac{1-\sqrt{1-4\,t}}{2\,t} = 1 + t + 2 t^{2} + 5 t^{3} + 14 t^{4} 
        + 42 t^{5} + 132 t^{6} %+ 429 t^{7} + 1430 t^{8} + 4862 t^{9} 
        + \mathcal{O}\left(t^{7}\right)
\end{displaymath}
so the relationship $\hat{c}_{j} = c_{j-1}$ holds for any $j\in\mathbb{N}$, 
with boundary condition $c_{-1}=0$.  Looking at the previous expression for $d_{nk}$,
there exists indices $i_{1}, i_{2}, \ldots, i_{k+1}$,
where $i_{j}\in\lbrace0,\ldots,n+1\rbrace$ for each $j\in\lbrace1,\ldots,k+1\rbrace$, 
such that:
\begin{displaymath}
    [t^{n+1}] \left(\frac{1-\sqrt{1-4\,t}}{2}\right)^{k+1} 
        = \sum_{i_{1}+ i_{2}+ \ldots+ i_{k+1}=n+1}{
            \hat{c}_{i_{1}}\,\hat{c}_{i_{2}}\,\ldots\,\hat{c}_{i_{k+1}} }
\end{displaymath}
which is the same as:
\begin{equation}
    d_{nk} = \sum_{i_{1}+ i_{2}+ \ldots+ i_{k+1}=n+1}{
            c_{i_{1}-1}\,c_{i_{2}-1}\,\ldots\,c_{i_{k+1}-1} }
    \label{eq:convolution:expansion:for:generic:element:in:catalan:array}
\end{equation}
This last equation will be used in the first proofs that follows, while
in the last ones, \autoref{eq:catalan:array:second:identity} will be used.

