
\section{Catalan array, precisely}
\label{sec:C:precisely}

% big comment: skip it, maybe the content is interesting, but stated very ugly.
\iffalse
Let $\mathcal{C}_{h}$ a principal $h$-cluster of
the Catalan array, in its traditional definition: We attempt to point out
some properties about $\mathcal{C}_{h+1}$:
\begin{itemize}
    \item a more general pattern, for any $h\in\mathcal{N}$,
        a coefficient $d_{p^{h}-1,k}$, for $k\in\lbrace0,\ldots,p^{h}-1 \rbrace$,
        satisfies:
        \begin{displaymath}
            d_{p^{h}-1,k} \equiv_{p} 0
        \end{displaymath}
    \item consider the subcluster $\mathcal{C}_{h}^{(0,p^{h}-1)}$ of
        $\mathcal{C}_{h}$ obtained by removing antidiagonal $0$ (namely,
        the boundary one, composed by $1$s only) and by removing row $p^{h}-1$.
        By construction, $\mathcal{C}_{h}^{(0,p^{h}-1)}$ is a triangle too,
        with $p^{h}-2$ coefficients per side. Therefore, $\mathcal{C}_{h+1}$
        is composed on the very top, starting at the root, by a copy of $\mathcal{C}_{h}$,
        while on the bottom there are three triangles:
        \begin{itemize}
            \item an equilateral triangle $T_{\equiv_{p} 0}^{(h)}$,
                with $p^{h}-1$ coefficients on each side, such that:
                \begin{displaymath}
                    d_{nk} \in T_{\equiv_{p} 0}^{(h)} \rightarrow d_{nk} \equiv_{p} 0
                \end{displaymath}
                where $n\in\lbrace p^{h},\ldots,p^{h+1}-2\rbrace$ and
                $k\in\lbrace 0,\ldots, n-p^{h}\rbrace$, in other words
                $k\in\lbrace 0,\ldots, p^{h}(p-1)-2\rbrace$;
            \item a mirror copy of $\mathcal{C}_{h}^{(0,p^{h}-1)}$ respect to column orientation,
                defining coefficients $d_{nk}$
                where $n\in\lbrace p^{h},\ldots,p^{h+1}-2\rbrace$ and
                $k\in\lbrace 1,\ldots, p^{h}-2\rbrace$;
            \item a segment of coefficients $d_{n, p^{h}-1} \equiv_{p} 0$
                where $n\in\lbrace p^{h},\ldots,p^{h+1}-2\rbrace$;
            \item a copy of $\mathcal{C}_{h}^{(0,p^{h}-1)}$,
                defining coefficients $d_{nk}$
                where $n\in\lbrace p^{h}+1,\ldots,p^{h+1}-2\rbrace$ and
                $k\in\lbrace p^{h},\ldots, n-1\rbrace$;
        \end{itemize}
\end{itemize}
\fi

We are now in the position to formally define the Catalan
array $\mathcal{C}$ and to introduce some mathematical objects
that will be used later; formally, $\mathcal{C}$ is defined as
\begin{displaymath}
    \mathcal{C}=\left(\frac{1-\sqrt{1-4\,t}}{2\,t},
        \frac{1-\sqrt{1-4\,t}}{2}\right)
\end{displaymath}
and it is a \emph{renewal} array \cite{rogers:1977} because
$h(t)=t\,d(t)$. Application of the substitution $k=0$ in the definition of the
coefficient $\displaystyle d_{n,k} = [t^{n}]d(t)h(t)^{k}$ entails that the
$n$-th Catalan number $C_{n}$ satisfies
\begin{displaymath}
    C_{n} = [t^{n}]d(t)= [t^{n}]\frac{1-\sqrt{1-4\,t}}{2\,t}
\end{displaymath}
where the operator $[t^{n}]$ extracts the coefficient of $t^{n}$ in the series expansion
of the function $g(t)$ to which it is applied.

An equivalent characterization of a each matrix $\mathcal{R}(d(t), h(t))$ in
the Riordan group is given by two sequences of coefficients
$\left(a_{n}\right)_{n\in\mathbb{N}}$  and
$\left(z_{n}\right)_{n\in\mathbb{N}}$ called $A$-sequence and $Z$-sequence,
respectively. The former one can be used to define every
coefficient $d_{n,k}$ with $k>0$,
\begin{equation}
    d_{n+1, k+1} = a_{0}d_{n,k} + a_{1}d_{n,k+1} + a_{2}d_{n,k+2} + \ldots + a_{j}d_{n,k+j} + \ldots %+ a_{j+1}d_{n,k+j+1} + \cdots
\end{equation}
where the sum is finite because exists $j\in\mathbb{N}$ such that $n=k+j$. On
the other hand, the latter one can be used to define every coefficient
$d_{n,0}$ lying on the first column,
\begin{equation}
    d_{n+1, 0} = z_{0}d_{n,0} + z_{1}d_{n,1} + z_{2}d_{n,2} + \ldots + z_{n}d_{n,n} + \ldots %+ z_{n+1}d_{n,n+1} + \cdots
\end{equation}
where the sum is finite because $d_{n,k+j}=0$ for $j>n-k$.

Moreover, let $A(t)$ and $Z(t)$ be the generating functions of the $A$-sequence
and $Z$-sequence, respectively, then relations
\begin{equation}
    h(t) = tA(h(t)) \quad\text{and}\quad d(t)=\frac{d_{0,0}}{1-tZ(h(t))}
\end{equation}
connect them with functions $d(t)$ and $h(t)$,
where $d_{0,0}$ is the very first element %in the top left corner
of $\mathcal{R}$, see \cite{merlini:some:alternative:characterizations:1997}.

A last fact to be aware of is a fundamental theorem that allows us to perform
matrix-vector product of an array $\mathcal{R}$ and an infinite column vector
$\vect{\omega}=(\omega_{0},\omega_{1},\omega_{2},\ldots)$
as the convolution
\begin{equation}
    \mathcal{R}\cdot\vect{\omega} = d(t)\Omega(h(t))
\end{equation}
where $\Omega$ is the $\vect{\omega}$'s generating function which admits the
series expansion $\Omega(t)=\sum_{k\in\mathbb{N}}{\omega_{k}t^{k}}$.

In the rest of the paper we use the notation $c_{n,k}\in\mathcal{C}$ to denote
a coefficient belonging to the Catalan array at row $n$ and column $k$.
We instantiate previous properties for the Catalan array $\mathcal{C}$,
starting with its $A$-sequence
\begin{displaymath}
    A_{\mathcal{C}}(t)=\frac{1}{1-t}=1+t+t^{2}+t^{3}+t^{4}+t^{5}+t^{6}+t^{7}+t^{8}+
        \mathcal{O}(t^{9})
\end{displaymath}
and, since $\mathcal{C}$ is a renewal array, it is not difficult to see that
its $Z$-sequence is the same as the $A$-sequence. Consequently, function $Z(t)$
defines each Catalan number $C_{n}$ as the sum $\displaystyle C_{n} = c_{n,0} =
\sum_{k=0}^{n-1}{c_{n-1,k}} $ of coefficients lying on row $n-1$ and
the application of $[t^{n}]$ to the generating function $d(t)$ 
allows us to write a well known closed formula for the $n$-th Catalan number
\begin{equation}
    C_{n} = \frac{1}{n+1}{{2n}\choose{n}} = {{2n}\choose{n}} - {{2n}\choose{n+1}}.
    \label{eq:catalan:coeff:rewriting}
\end{equation}

On the other hand, an arbitrary element $c_{n,k}\in\mathcal{C}$
admits closed formulae too and, according to \cite{luzon:2012631}
\begin{align}
    & c_{n,k}=\frac{k+1}{n+1}{{2n-k}\choose{n-k}}\quad\text{and}
    \label{eq:catalan:array:first:identity}\\
    & c_{n,k}={{2n-k}\choose{n-k}} - {{2n-k}\choose{n-k-1}}
    \label{eq:catalan:array:second:identity}
\end{align}
are of particular interest for our work.  In order to formally characterize
$\mathcal{C}_{\equiv_{2}}$ we use the definition to write the generic coefficient
\begin{displaymath}
    c_{n,k} = [t^n] \frac{1-\sqrt{1-4\,t}}{2\,t}\,
        \left(\frac{1-\sqrt{1-4\,t}}{2}\right)^{k}
           = [t^{n+1}] \left(\frac{1-\sqrt{1-4\,t}}{2}\right)^{k+1}
\end{displaymath}
where the right most term requires to extract coefficient $n+1$ from the $(k+1)$-fold
convolution of Catalan numbers' generating function with itself, \emph{shifted
by one place}
\begin{equation}
    c_{n,k} = [t^{n+1}] \left(\frac{1-\sqrt{1-4\,t}}{2}\right)^{k+1}
            %= \sum_{i_{1}+ i_{2}+ \ldots+ i_{k+1}=n+1}{
                %\hat{c}_{i_{1}}\,\hat{c}_{i_{2}}\,\ldots\,\hat{c}_{i_{k+1}} }
            = \sum_{i_{1}+ i_{2}+ \ldots+ i_{k+1}=n+1}{
                C_{i_{1}-1}\,C_{i_{2}-1}\,\ldots\,C_{i_{k+1}-1} }.
    \label{eq:convolution:expansion:for:generic:element:in:catalan:array}
\end{equation}

