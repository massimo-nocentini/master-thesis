
\section{Catalan characterization}


\begin{figure}[htb]

    %\hspace{-1.5cm}
    \noindent\makebox[\textwidth]{
        \centering
        %\includegraphics[width=0.8\textwidth]{../../sympy/catalan/coloured.pdf}

        % using *angle* property to rotate it is difficult to properly align it
        % in order to have a "real" matrix representation.
        \includegraphics[width=20cm, height=20cm, keepaspectratio=true]{../sympy/catalan/catalan-traditional-standard-ignore-negatives-centered-colouring-127-rows-mod2-partitioning-triangle.pdf}
    }

    % this 'particular' line is necessary to use `displaymath' environment
    % into the caption environment, togheter with the inclusion of 
    % `caption' package. See here for more explanation:
    % http://stackoverflow.com/questions/2716227/adding-an-equation-or-formula-to-a-figure-caption-in-latex
    \captionsetup{singlelinecheck=off}
    \caption[$\mathcal{C}_{\equiv_{2}}$]{
        Modular Catalan triangle $\mathcal{C}_{\equiv_{2}}$
        \iffalse
        Catalan traditional triangle, formally: 
        \begin{displaymath}
            \mathcal{C}=\left(\frac{1-\sqrt{1-4 \, t}}{2 \, t}, \frac{1-\sqrt{1-4 \, t}}{2}\right)
        \end{displaymath} % \newline % new line no more necessary
        standard, ignore negatives, centered colouring, 127 rows, mod2 partitioning
        \fi
        }

    \label{fig:catalan-traditional-standard-ignore-negatives-centered-colouring-127-rows-mod2-partitioning-triangle}

\end{figure}

In this section we tackle a modular characterization, using congruence
$\equiv_{2}$, for the Catalan array $\mathcal{C}$ from a formal point of view.
We aim at a proof driven from
\autoref{fig:catalan-traditional-standard-ignore-negatives-centered-colouring-127-rows-mod2-partitioning-triangle},
and in order to complete it, we proceed \emph{piecewise}, breaking such
characterization into the following sections, where an independent chunk of
$\mathcal{C}_{\equiv_{2}}$ is described in each one of them.

% inclusion of matrix expansion for C and its inverse.

\begin{table}
    \begin{displaymath} 
        \hspace{-4cm}
        \mathcal{C}_{10} \left(\begin{array}{rrrrrrrrrr}
        1 &  &  &  &  &  &  &  &  &  \\
        1 & 1 &  &  &  &  &  &  &  &  \\
        2 & 2 & 1 &  &  &  &  &  &  &  \\
        5 & 5 & 3 & 1 &  &  &  &  &  &  \\
        14 & 14 & 9 & 4 & 1 &  &  &  &  &  \\
        42 & 42 & 28 & 14 & 5 & 1 &  &  &  &  \\
        132 & 132 & 90 & 48 & 20 & 6 & 1 &  &  &  \\
        429 & 429 & 297 & 165 & 75 & 27 & 7 & 1 &  &  \\
        1430 & 1430 & 1001 & 572 & 275 & 110 & 35 & 8 & 1 &  \\
        4862 & 4862 & 3432 & 2002 & 1001 & 429 & 154 & 44 & 9 & 1
        \end{array}\right) 
        \quad
        \mathcal{C}_{10}^{-1}\left(\begin{array}{rrrrrrrrrr}
        1 &  &  &  &  &  &  &  &  &  \\
        1 &  &  &  &  &  &  &  &  &  \\
         & 1 &  &  &  &  &  &  &  &  \\
         & -1 & 1 &  &  &  &  &  &  &  \\
         &  & -2 & 1 &  &  &  &  &  &  \\
         &  & 1 & -3 & 1 &  &  &  &  &  \\
         &  &  & 3 & -4 & 1 &  &  &  &  \\
         &  &  & -1 & 6 & -5 & 1 &  &  &  \\
         &  &  &  & -4 & 10 & -6 & 1 &  &  \\
         &  &  &  & 1 & -10 & 15 & -7 & 1 & 
        \end{array}\right)
    \end{displaymath}

  \caption[$\mathcal{C}$ and $\mathcal{C}^{-1}$]{Two $10$-minors of
  $\mathcal{C}$ and $\mathcal{C}^{-1}$ matrix expansions, respectively}

  \label{tab:catalan:array} 

\end{table}


% big comment: skip it, maybe the content is interesting, but stated very ugly.
\iffalse
Let $\mathcal{C}_{h}$ a principal $h$-cluster of
the Catalan array, in its traditional definition: We attempt to point out
some properties about $\mathcal{C}_{h+1}$:
\begin{itemize}
    \item a more general pattern, for any $h\in\mathcal{N}$, 
        a coefficient $d_{p^{h}-1,k}$, for $k\in\lbrace0,\ldots,p^{h}-1 \rbrace$, 
        satisfies:
        \begin{displaymath}
            d_{p^{h}-1,k} \equiv_{p} 0
        \end{displaymath}
    \item consider the subcluster $\mathcal{C}_{h}^{(0,p^{h}-1)}$ of 
        $\mathcal{C}_{h}$ obtained by removing antidiagonal $0$ (namely,
        the boundary one, composed by $1$s only) and by removing row $p^{h}-1$.
        By construction, $\mathcal{C}_{h}^{(0,p^{h}-1)}$ is a triangle too, 
        with $p^{h}-2$ coefficients per side. Therefore, $\mathcal{C}_{h+1}$
        is composed on the very top, starting at the root, by a copy of $\mathcal{C}_{h}$,
        while on the bottom there are three triangles:
        \begin{itemize}                
            \item an equilateral triangle $T_{\equiv_{p} 0}^{(h)}$, 
                with $p^{h}-1$ coefficients on each side, such that:
                \begin{displaymath}
                    d_{nk} \in T_{\equiv_{p} 0}^{(h)} \rightarrow d_{nk} \equiv_{p} 0
                \end{displaymath}
                where $n\in\lbrace p^{h},\ldots,p^{h+1}-2\rbrace$ and 
                $k\in\lbrace 0,\ldots, n-p^{h}\rbrace$, in other words
                $k\in\lbrace 0,\ldots, p^{h}(p-1)-2\rbrace$;
            \item a mirror copy of $\mathcal{C}_{h}^{(0,p^{h}-1)}$ respect to column orientation,
                defining coefficients $d_{nk}$  
                where $n\in\lbrace p^{h},\ldots,p^{h+1}-2\rbrace$ and 
                $k\in\lbrace 1,\ldots, p^{h}-2\rbrace$;
            \item a segment of coefficients $d_{n, p^{h}-1} \equiv_{p} 0$ 
                where $n\in\lbrace p^{h},\ldots,p^{h+1}-2\rbrace$;
            \item a copy of $\mathcal{C}_{h}^{(0,p^{h}-1)}$,
                defining coefficients $d_{nk}$  
                where $n\in\lbrace p^{h}+1,\ldots,p^{h+1}-2\rbrace$ and 
                $k\in\lbrace p^{h},\ldots, n-1\rbrace$;
        \end{itemize}                
\end{itemize}
\fi

\subsection{Introducing array $\mathcal{C}$}

In this section we would like to give a quick description of
array $\mathcal{C}$ and introduce some mathematical objects
that will be used later. This array is defined as follow:
\begin{displaymath}
    \mathcal{C}=\left(\frac{1-\sqrt{1-4\,t}}{2\,t},
        \frac{1-\sqrt{1-4\,t}}{2}\right)
\end{displaymath}
since $h_{\mathcal{C}}(t)=t\,d_{\mathcal{C}}(t)$, then $\mathcal{C}$
is in the \emph{renewal} subgroup.

Denote with $c_{j}$ the $j$-th Catalan coefficient and, 
by definition of array $\mathcal{C}$, observe that:
\begin{displaymath}
    c_{j} = [t^{j}]d_{\mathcal{C}}(t)= [t^{j}]\frac{1-\sqrt{1-4\,t}}{2\,t}
\end{displaymath}
and since $\mathcal{C}$'s $A$-sequence is:
\begin{displaymath}
    A_{\mathcal{C}}(t)=\frac{1}{1-t}=1+t+t^{2}+t^{3}+t^{4}+t^{5}+t^{6}+t^{7}+t^{8}+
        \mathcal{O}(t^{9})
\end{displaymath}
each coefficient $c_{j}$ can be written as the following combination:
\begin{displaymath}
    c_{j} = \sum_{k=0}^{j-1}{d_{j-1,k}}
\end{displaymath}
in other words, $c_{j}$ equals the $(j-1)$-row sum of $\mathcal{C}$, 
therefore by the \emph{fundamental theorem} of the Riordan group:

\marginpar{another \ac{gf} for the Catalan numbers}
\begin{displaymath}
    c_{j} = [t^{j-1}]\mathcal{C}\,A_{\mathcal{C}}(t)
          = [t^{j-1}]d_{\mathcal{C}}(t)\,A_{\mathcal{C}}(h_{\mathcal{C}}(t))
          = [t^{j}]h_{\mathcal{C}}(t)\,A_{\mathcal{C}}(h_{\mathcal{C}}(t))
\end{displaymath}
yielding:
\begin{displaymath}
    c_{j} = [t^{j}]\frac{1-\sqrt{1-4 \, t}}{1+\sqrt{1-4 \, t}}
\end{displaymath}
with boundary condition $c_{0}=1$, which cannot be read from the above
series expansion. Moreover, from the above equation, it is possible to derive
another \ac{gf} for the sequence of Catalan numbers:

\marginpar{yet another \ac{gf} for the Catalan numbers}
\begin{displaymath}
    \frac{1-\sqrt{1-4 \, t}}{1+\sqrt{1-4 \, t}}=
    \frac{\left(1-\sqrt{1-4 \, t}\right)^{2}}{4 \, t}=
    \frac{1-2\,t-\sqrt{1-4 \, t}}{2 \, t}
\end{displaymath}

% no more necessary?
\iffalse
Go back to identity:
\begin{displaymath}
    d_{nk} = \sum_{i_{1}+ i_{2}+ \ldots+ i_{k+1}=n+1}{
            c_{i_{1}-1}\,c_{i_{2}-1}\,\ldots\,c_{i_{k+1}-1} }
\end{displaymath}
\fi

Coefficient extraction on previous \ac{gf} allow us to write 
a Catalan coefficient $c_{j}$ in a closed form:
\begin{equation}
    c_{j} = \frac{1}{j+1}{{2j}\choose{j}} 
        = {{2j}\choose{j}} - {{2j}\choose{j+1}}
    \label{eq:catalan:coeff:rewriting}
\end{equation}

\marginpar{closed formulae for Catalan coefficient $c_{j}$ and
$d_{nk}\in\mathcal{C}$} On the other hand, the generic element
$d_{nk}\in\mathcal{C}$ admits closed formulae too. According to
\cite{luzon:2012631}, two of them are of
particular interest in order to complete our proofs:
\begin{equation}
    d_{nk}=\frac{k+1}{n+1}{{2n-k}\choose{n-k}}
    \label{eq:catalan:array:first:identity}
\end{equation}
and:
\begin{equation}
    d_{nk}={{2n-k}\choose{n-k}} - {{2n-k}\choose{n-k-1}}
    \label{eq:catalan:array:second:identity}
\end{equation}


The
\marginpar{$\diagup_{\lbrace\alpha_{0},\alpha_{1},\alpha_{2},\ldots\rbrace}^{k}$
is a kind of macro, it is a set builder, eventually}last object we want to
introduce is a pair that allow to talk about a subset of elements lying on a
column $\vect{c}_{k}$ of $\mathcal{C}$. Let $S\subseteq\mathbb{N}$  
and let $k\in\mathbb{N}$, then a \emph{segment} of column $\vect{c}_{k}$ is
denoted by the following relation:
\begin{equation}
    \diagup_{\lbrace\alpha_{i}\rbrace_{i\in S}}^{k}
        = \left(\vect{c}_{k},\lbrace k+\alpha_{i}\rbrace_{i\in S}\right)
        = \lbrace d_{k+\alpha_{i},k}\rbrace_{i\in S}
    \label{eq:column:segment}
\end{equation}
where $\lbrace\alpha_{i}\rbrace_{i\in S}\subseteq\mathbb{N}$. 
In order\marginpar{$rows$ helper function} to fetch the set of 
row indices belonging to a segment we introduce the $rows$ function, defined as follow:
\begin{displaymath}
    rows\left(\diagup_{\lbrace\alpha_{i}\rbrace_{i\in S}}^{k}\right)
        = rows\left(\left(\vect{c}_{k},\lbrace k+\alpha_{i}\rbrace_{i\in S}\right)\right)
        = \lbrace k+\alpha_{i}\rbrace_{i\in S}
    \label{eq:column:segment:rows:function}
\end{displaymath}

\subsection{Some initial attempts}

In order to formally proof the modular characterization $\mathcal{C}_{\equiv_{2}}$, 
we begin with an attempt which applies the \ac{LIF}. 
Write the generic element $d_{nk}\in\mathcal{C}$ using the definition:
\begin{displaymath}
    d_{nk} = [t^n] \frac{1-\sqrt{1-4\,t}}{2\,t}\,
        \left(\frac{1-\sqrt{1-4\,t}}{2}\right)^{k}
           = [t^{n+1}] \left(\frac{1-\sqrt{1-4\,t}}{2}\right)^{k+1} 
\end{displaymath}
\marginpar{For explanation of when and how
    about \ac{LIF} see \cite{merlini:lagrange:2006}}
The version of the \ac{LIF} we want to apply is the following one:
\begin{displaymath}
    \alpha\,\left[z^\alpha\right] A(z)^{\beta} = 
        \beta\,\left[w^{\alpha-\beta}\right]\Phi(w)^{\alpha}
\end{displaymath}
where $\alpha,\beta\in\mathbb{Z}$ and function $A$ relates to function
$\Phi$ according to: 
\begin{displaymath} A(z)=z\,\Phi(A(z)) \end{displaymath}
under the constraint $\Phi(0)\neq0$.
% the following explanation for the constraing over 
% function $\Phi$ should help only *our* understanding.
\iffalse
, otherwise a contraddiction arises
(critical relation is emphasized with $\circeq$ symbol):
\begin{displaymath} 
    z\,\Phi(0) \circeq 0 \rightarrow z\circeq\frac{0}{\Phi(0)} 
\end{displaymath}
\fi
\ac{LIF}'s ``from right to left'' direction allow us to make 
the following definitions:
\begin{displaymath}
    \Phi(t)=\frac{1-\sqrt{1-4\,t}}{2}\qquad \alpha=k+1\qquad \beta=k-n
\end{displaymath}
Since $\Phi(0)=0$ we cannot apply \ac{LIF} directly, so start over again 
and observe that:
\begin{displaymath}
    d_{nk}=[t^{n+1}] \left(\frac{1-\sqrt{1-4\,t}}{2}\right)^{k+1} 
\end{displaymath}
requires to extract coefficient $n+1$ from the $(k+1)$-fold
convolution of Catalan numbers' \ac{gf} with itself, \emph{shifted
by one place}. Before going on, careful
understanding of what coefficients are introduced is necessary. 
The following expansion defines a sequence of coefficients 
$\lbrace \hat{c}_{i}\rbrace_{i\in\mathbb{N}}$:
\begin{displaymath}
    \frac{1-\sqrt{1-4\,t}}{2} = t + t^{2} + 2 t^{3} + 5 t^{4} 
        + 14 t^{5} + 42 t^{6} + 132 t^{7} %+ 429 t^{8} + 1430 t^{9} 
        + \mathcal{O}\left(t^{8}\right)
\end{displaymath}
where $\hat{c}_{0}=0$. On the other hand the following one, defines a sequence 
of coefficients $\lbrace c_{i}\rbrace_{i\in\mathbb{N}}$, 
the traditional Catalan numbers:
\begin{displaymath}
    \frac{1-\sqrt{1-4\,t}}{2\,t} = 1 + t + 2 t^{2} + 5 t^{3} + 14 t^{4} 
        + 42 t^{5} + 132 t^{6} %+ 429 t^{7} + 1430 t^{8} + 4862 t^{9} 
        + \mathcal{O}\left(t^{7}\right)
\end{displaymath}
so the relationship $\hat{c}_{j} = c_{j-1}$ holds for any $j\in\mathbb{N}$, 
with boundary condition $c_{-1}=0$.  Looking at the previous expression for $d_{nk}$,
there exists indices $i_{1}, i_{2}, \ldots, i_{k+1}$,
where $i_{j}\in\lbrace0,\ldots,n+1\rbrace$ for each $j\in\lbrace1,\ldots,k+1\rbrace$, 
such that:
\begin{displaymath}
    [t^{n+1}] \left(\frac{1-\sqrt{1-4\,t}}{2}\right)^{k+1} 
        = \sum_{i_{1}+ i_{2}+ \ldots+ i_{k+1}=n+1}{
            \hat{c}_{i_{1}}\,\hat{c}_{i_{2}}\,\ldots\,\hat{c}_{i_{k+1}} }
\end{displaymath}
which is the same as:
\begin{equation}
    d_{nk} = \sum_{i_{1}+ i_{2}+ \ldots+ i_{k+1}=n+1}{
            c_{i_{1}-1}\,c_{i_{2}-1}\,\ldots\,c_{i_{k+1}-1} }
    \label{eq:convolution:expansion:for:generic:element:in:catalan:array}
\end{equation}
This last equation will be used in the first proofs that follows, while
in the last ones, \autoref{eq:catalan:array:second:identity} will be used.


\subsection{On the very first column of $\mathcal{C}_{\equiv_{2}}$}

In this section we want to characterize the very first column $\vect{c}_{0}$
of $\mathcal{C}$, namely the column composed of Catalan numbers only,
formally $c_{j}=d_{j0}\in\vect{c}_{0}$, for any $j\in\mathbb{N}$.  
Use \autoref{eq:catalan:coeff:rewriting}
and set a congruence relation, modulo $2$: write $j$ in base $2$, 
so there exists $k\in\mathbb{N}$ such that 
$j=j_{0} + j_{1}\,2 + j_{2}\,2^{2} + \ldots + j_{k}\,2^{k}$, where
$j_{r}\in\lbrace0,1\rbrace$, for each $r\in\lbrace0,\ldots,k\rbrace$. 
% now start distinguishing $j$ by cases: if even..., if odd...
In order to apply Lucas theorem, we proceed by cases on $j$'s parity:
\begin{itemize}
    \item let $j=2\alpha$, for some $\alpha\in\mathbb{N}$, so $j_{0}=0$,
        which make vanish the minuend:
        \begin{displaymath}
            {{2j}\choose{j+1}}
            \equiv_{2} {{0}\choose{1}}{{0}\choose{j_{1}}}{{j_{1}}\choose{j_{2}}}
                \ldots{{j_{k-1}}\choose{j_{k}}}{{j_{k}}\choose{0}}\equiv_{2}0
        \end{displaymath}
        so:
        \begin{displaymath}
            c_{j}\equiv_{2}{{2j}\choose{j}}
            \equiv_{2} {{0}\choose{0}}{{0}\choose{j_{1}}}{{j_{1}}\choose{j_{2}}}
                \ldots{{j_{k-1}}\choose{j_{k}}}{{j_{k}}\choose{0}}\equiv_{2}0
        \end{displaymath}
        \marginpar{$c_{0}$ is the only odd Catalan number among those 
            $\lbrace c_{j}\rbrace_{j\in\mathbb{N}}$ for which $j$ is even}
        A boundary case pops out when $\alpha=0$ where $j$ has to be written as 
        $j=0 + 0\cdot2 + 0\cdot2^{2} + \ldots + 0\cdot2^{k}$, therefore:
        \begin{displaymath}
            c_{0}\equiv_{2}{{0}\choose{0}}
            \equiv_{2} {{0}\choose{0}}{{0}\choose{0}}{{0}\choose{0}}
                \ldots{{0}\choose{0}}{{0}\choose{0}}\equiv_{2}1
        \end{displaymath}
        this is indeed the only case where $c_{2\alpha} \equiv_{2}1$:
        \begin{proof}
            Assume not, hence there exists $\hat{\alpha}\in\mathbb{N}$, greater than $0$,
            such that $c_{2\hat{\alpha}} \equiv_{2}1$. So for ${{0}\choose{j_{1}}}$
            be not zero then $j_{1}=0$; in turn, this constraint requires a new one, namely
            for ${{0}\choose{j_{2}}}$ be not zero then $j_{2}=0$; 
            in turn, this constraint requires a new one, namely
            for ${{0}\choose{j_{3}}}$ be not zero then $j_{3}=0$; 
            in turn, this constraint requires a new one, namely\ldots
            for ${{0}\choose{j_{k}}}$ be not zero then $j_{k}=0$.
            Therefore $j=0$, which contradicts the hypothesis $\alpha>0$. So,
            for even $j > 0$, coefficient $c_{j}\equiv_{2}0$, as required.
        \end{proof}

    \item let $j=2\alpha+1$, for some $\alpha\in\mathbb{N}$, so $j_{0}=1$,
        which make vanish the augend:
        \begin{displaymath}
            {{2j}\choose{j}}
            \equiv_{2} {{0}\choose{1}}{{1}\choose{j_{1}}}{{j_{1}}\choose{j_{2}}}
                \ldots{{j_{k-1}}\choose{j_{k}}}{{j_{k}}\choose{0}}\equiv_{2}0
        \end{displaymath}
        so:
        \begin{displaymath}
            c_{j}\equiv_{2}-{{2j}\choose{j+1}}
            %\equiv_{2} {{0}\choose{0}}{{0}\choose{j_{1}}}{{j_{1}}\choose{j_{2}}}
                %\ldots{{j_{k-1}}\choose{j_{k}}}{{j_{k}}\choose{0}}\equiv_{2}0
        \end{displaymath}
        observe that $(-1)^{-1}\mod2$, the multiplicative inverse of $-1$, equals $1$
        since $(-1)\cdot 1 \equiv_{2}1$, therefore we can multiply both members by
        $(-1)^{-1}\mod2$ to get:
        \begin{displaymath}
            c_{j}\equiv_{2}{{2j}\choose{j+1}}
            %\equiv_{2} {{0}\choose{0}}{{0}\choose{j_{1}}}{{j_{1}}\choose{j_{2}}}
                %\ldots{{j_{k-1}}\choose{j_{k}}}{{j_{k}}\choose{0}}\equiv_{2}0
        \end{displaymath}

        Careful handling is necessary for the term $j+1$: 
        since $j$ is odd by hypothesis, increasing it could yield a chain of carries.
        So, handling $j+1$ in a generic way, we write
        $j$ in base $2$ in its most general form, putting evidence on coefficients:
        \begin{displaymath}
            j=\left(\underbrace{1,1,\ldots,1}_{\beta},0,j_{\beta+1},\ldots,j_{\beta+\gamma}\right)_{2}
        \end{displaymath}
        where $\beta,\gamma\in\mathbb{N}$ such that $\beta>0$ and $\beta+\gamma=k$. 
        Increment $j$ by $1$:
        \begin{displaymath}
            j+1=\left(\underbrace{0,0,\ldots,0}_{\beta},1,j_{\beta+1},\ldots,j_{\beta+\gamma}%,j_{\beta+\gamma+1}
                \right)_{2}
        \end{displaymath}
        so the congruence relation gets the following shape:
        \begin{displaymath}
            c_{j}\equiv_{2}{{2j}\choose{j+1}}
                \equiv_{2} \underbrace{{{0}\choose{0}}{{1}\choose{0}}
                {{1}\choose{0}}\ldots{{1}\choose{0}}}_{\beta} 
                    {{1}\choose{1}}{{0}\choose{j_{\beta+1}}}{{j_{\beta+1}}\choose{j_{\beta+2}}}
                    \ldots{{j_{\beta+\gamma-1}}\choose{j_{\beta+\gamma}}}{{j_{\beta+\gamma}}\choose{0}}%{\beta+\gamma+1}}
        \end{displaymath}
        a simplification yield:
        \begin{displaymath}
            c_{j}\equiv_{2} {{0}\choose{j_{\beta+1}}}
                {{j_{\beta+1}}\choose{j_{\beta+2}}}
                    \ldots{{j_{\beta+\gamma-1}}\choose{j_{\beta+\gamma}}}
        \end{displaymath}
        so $c_{j}\equiv_{2} 1$ if coefficients 
            $j_{\beta+1}, \ldots, j_{\beta+\gamma-1},j_{\beta+\gamma}$
        are $0$ them all, which implies that:
        \begin{displaymath}
            j=\left(\underbrace{1,1,\ldots,1}_{\beta},\underbrace{0,0,\ldots,0}_{k-\beta+1}\right)_{2}
        \end{displaymath}

        \marginpar{$c_{2^{\alpha}-1} \equiv_{2} 1$\\Otherwise, for any $j$ odd, 
            $c_{j} \equiv_{2} 0$} 
        in other words $j = 2^{\beta+1}-1$. As boundary case, to handle $c_{1}$ correctly 
        (the above result doesn't cover it because $\beta>0$) observe:
        \begin{displaymath}
            c_{1}\equiv_{2} {{2}\choose{2}}\equiv_{2} {{0}\choose{0}}{{1}\choose{1}}\equiv_{2}1
        \end{displaymath}
\end{itemize}

In \autoref{fig:catalan-first-column}, segment
$\diagup_{\lbrace 0,1,2,\ldots,2^{4}\rbrace}^{0}$ of $\mathcal{C}_{\equiv_{2}}$
is highlighted.
\\\\
\marginpar{using the ``classic'' closed formula on $c_{2\alpha+1}$ doesn't
    close the above proof}
It is interesting to observe that if we use the ``traditional'' closed formula
for a Catalan coefficient $c_{j}$:
\begin{displaymath}
    (j+1)\,c_{j} = {{2j}\choose{j}} 
\end{displaymath}
where $j=2\alpha+1$, it would have been hard to handle, because:
\begin{displaymath}
    2(\alpha+1)\,c_{j}\equiv_{2} {{0}\choose{1}}{{1}\choose{j_{1}}}
            \ldots{{j_{k-1}}\choose{j_{k}}} \equiv_{2} 0
\end{displaymath}
reduces to $0\equiv_{2}0$, giving no opportunity to derive some
properties about coefficient $c_{j}$.


\begin{figure}[p]

    \noindent\makebox[\textwidth]{
        \centering
        %\includegraphics[width=0.8\textwidth]{../../sympy/catalan/coloured.pdf}

        % using *angle* property to rotate it is difficult to properly align it
        % in order to have a "real" matrix representation.
        \includegraphics[
            width=7cm, 
            height=6cm, 
            keepaspectratio=true]{catalan-tikz/first-column/first-column.pdf}
    }

    % this 'particular' line is necessary to use `displaymath' environment
    % into the caption environment, togheter with the inclusion of 
    % `caption' package. See here for more explanation:
    % http://stackoverflow.com/questions/2716227/adding-an-equation-or-formula-to-a-figure-caption-in-latex
    \captionsetup{singlelinecheck=off}
    \caption[.]{ \textcolor{blue}{$d_{nk}$} if even,
        \textcolor{orange}{$d_{nk}$} otherwise  }

    \label{fig:catalan-first-column}

\end{figure}


\subsection{On rows composed of \emph{odd} coeffients only}

\begin{theorem}
    Every row $\vect{r}_{2^{\alpha}-1}$ of $\mathcal{C}_{\equiv_{2}}$, 
    for $\alpha\in\mathbb{N}$, is composed of \emph{odd} coefficients only.
\end{theorem}

\begin{proof}
    Choose any $\alpha\in\mathbb{N}$. For the very first coefficient $d_{2^{\alpha}-1,0}$ 
    lying on row $\vect{r}_{2^{\alpha}-1}$ we have seen in the
    previous section that $d_{2^{\alpha}-1,0}\equiv_{2}1$. What about $d_{2^{\alpha}-1,1}$?
    Recall we can write it, according to 
    \autoref{eq:convolution:expansion:for:generic:element:in:catalan:array}, as:
    \begin{displaymath}
        d_{2^{\alpha}-1,1} = \sum_{i_{1}+ i_{2}=2^{\alpha}}{ c_{i_{1}-1}\,c_{i_{2}-1} }
    \end{displaymath}
    Summation constraints indices $i_{1}$ and $i_{2}$ such that 
    $2^{\alpha}$ divides $i_{1}+i_{2}$ \emph{exactly}, so:
    \begin{displaymath}
        1 = \frac{i_{1}}{2^{\alpha}}+\frac{i_{2}}{2^{\alpha}} \qquad 
            i_{1},i_{2}\in\lbrace 0,\ldots,2^{\alpha}\rbrace
    \end{displaymath}
    %, which is the same to say that 
    %$\frac{i_{1}}{2^{\alpha}}$ and $\frac{i_{2}}{2^{\alpha}}$ are both integers.
    This implies that there exists $\beta,\gamma\in\mathbb{N}$, where $\beta,\gamma\leq\alpha$,
    such that $i_{1}=2^{\beta}$ and $i_{2}=2^{\gamma}$, respectively.

    By this fact follows that $c_{i_{1}-1}=c_{2^{\beta}-1}\equiv_{2}1$ and 
    $c_{i_{2}-1}=c_{2^{\gamma}-1}\equiv_{2}1$, therefore $c_{i_{1}-1}\,c_{i_{2}-1}\equiv_{2}1$.
    By construction, fixing one index in  
    $i_{1}+ i_{2}=2^{\alpha}$ fixes the other as well: there are $2^{\alpha}+1$ available choices
    for the first index and only $1$ for the second.
    We're almost finished:
    \begin{displaymath}
        d_{2^{\alpha}-1,1} = \sum_{i_{1}+ i_{2}=2^{\alpha}}{ c_{i_{1}-1}\,c_{i_{2}-1} }
            \equiv_{2} \sum_{k=1}^{2^{\alpha}+1}{1}\equiv_{2} 2^{\alpha}+1\equiv_{2} 1
    \end{displaymath}

    What about for an arbitrary coefficient $d_{2^{\alpha}-1,s}$ where
    $s\in\lbrace{2,\ldots,2^{\alpha}-1}\rbrace$?  Again, rewrite it as:
    \begin{displaymath}
        d_{2^{\alpha}-1,s} = \sum_{i_{1}+i_{2}+\ldots+i_{s+1}=2^{\alpha}}
            {c_{i_{1}-1}\,c_{i_{2}-1}\ldots\,c_{i_{s+1}-1}}
    \end{displaymath}
    Using an argument similar to the previous one, from:
    \begin{displaymath}
        1 = \frac{i_{1}}{2^{\alpha}}+\ldots+\frac{i_{s+1}}{2^{\alpha}} \qquad 
            i_{j}\in\lbrace 0,\ldots,2^{\alpha}\rbrace
    \end{displaymath} 
    the generic index $i_{j}$ has to satisfy $i_{j}=2^{\alpha_{j}}$, for some
    $\alpha_{j}\leq\alpha$. Therefore $c_{i_{j}-1}\equiv_{2}1$,
    for $j\in\lbrace1,\ldots,s+1\rbrace$, so:
    \begin{displaymath}
        d_{2^{\alpha}-1,s} \equiv_{2} \sum_{i_{1}+i_{2}+\ldots+i_{s+1}=2^{\alpha}}{1}
            \equiv_{2} {{s+2^{\alpha}}\choose{2^{\alpha}}}
    \end{displaymath}
    because \marginpar{See \emph{Introduction to Combinatorial Analysis} by
    Riordan}the summation over indices $i_{1},\ldots,i_{s+1}$ asks to count the
    number of $2^{\alpha}$-combinations of $s+1$ distinct objects, each of
    which may appear indefinitely often, that is, $0$ to $2^{\alpha}$ times:
    the saught number is ${{(s+1)+2^{\alpha}-1}\choose{2^{\alpha}}}$.

    Again, we're interested on the parity of such coefficient, therefore
    write $s=s_{0}+s_{1}\,2+\ldots+s_{\alpha-1}\,2^{\alpha-1}$, because $s$ can equal 
    $2^{\alpha}-1$ at most, and apply Lucas theorem one more time:
    \begin{displaymath}
        {{s+2^{\alpha}}\choose{2^{\alpha}}}\equiv_{2} 
            {{s_{0}}\choose{0}}{{s_{1}}\choose{0}} \ldots
                {{s_{\alpha-1}}\choose{0}}{{1}\choose{1}}\equiv_{2}1 
    \end{displaymath}
    the general case holds too, therefore each coefficient lying on
    a row $\vect{r}_{2^{\alpha}-1}$ is odd, as required.
\end{proof}


\begin{figure}[p]

    \noindent\makebox[\textwidth]{
        \centering
        %\includegraphics[width=0.8\textwidth]{../../sympy/catalan/coloured.pdf}

        % using *angle* property to rotate it is difficult to properly align it
        % in order to have a "real" matrix representation.
        \includegraphics[width=10cm, height=3cm, keepaspectratio=true]{catalan-tikz/odd-row/odd-row.pdf}
    }

    % this 'particular' line is necessary to use `displaymath' environment
    % into the caption environment, togheter with the inclusion of 
    % `caption' package. See here for more explanation:
    % http://stackoverflow.com/questions/2716227/adding-an-equation-or-formula-to-a-figure-caption-in-latex
    \captionsetup{singlelinecheck=off}
    \caption[.]{Row of coefficients $\textcolor{orange}{d_{2^{4}-1,s}} \equiv_{2} 1$ 
        for $s\in\lbrace1,\ldots,2^{4}-1 \rbrace$ }

    \label{fig:catalan-odd-row}

\end{figure}

In \autoref{fig:catalan-odd-row} row $\vect{r}_{2^{4}-1}$ is highlighted.

\subsection{On rows composed of odd and even coefficients}

\begin{theorem}
    Let $\vect{r}_{2^{\alpha}}$ be a row of $\mathcal{C}_{\equiv_{2}}$, 
    for some $\alpha\in\mathbb{N}$. Then, exclused the very first coefficient 
        $d_{2^{\alpha},0}$ which is even, $\vect{r}_{2^{\alpha}}$ is composed of
        alternating even and odd coefficients.
\end{theorem}

\begin{proof}
    Let $d_{2^{\alpha},j}$ be a coefficient lying on row $\vect{r}_{2^{\alpha}}$,
    for some $j\in\lbrace1,\ldots,2^{\alpha}\rbrace$.

    Since $\mathcal{C}$'s $A$-sequence is:
    \begin{displaymath}
        A_{\mathcal{C}}(t)=\frac{1}{1-t}=1+t+t^{2}+t^{3}+t^{4}+t^{5}+t^{6}+t^{7}+t^{8}+
            \mathcal{O}(t^{9})
    \end{displaymath}
    it follows that $d_{2^{\alpha},j}$ can be written as the combination of $r+2$
    coefficients lying on the previous row, namely $\vect{r}_{2^{\alpha}-1}$:
    \begin{displaymath}
        d_{2^{\alpha},j} = d_{2^{\alpha}-1,j-1} +d_{2^{\alpha}-1,j} +\ldots+d_{2^{\alpha}-1,j+r} 
    \end{displaymath}
    where $r$ satisfies $r=2^{\alpha}-1-j$, so $2^{\alpha}-j+1$ coefficients are 
    combined.  By \marginpar{$d_{2^{\alpha},j}\equiv_{2}0 \leftrightarrow j$ odd}
    the theorem above, row $\vect{r}_{2^{\alpha}-1}$ is composed by \emph{odd}
    coefficients only, therefore proceed by cases on the parity of $j$:
    \begin{itemize}
        \item if $j$ is \emph{odd}, assume $j=2k+1$ for some $k\in\mathbb{N}$, then
            $2^{\alpha}-2k$ coefficients are combined, which is an \emph{even} number. 
            Adding an \emph{even} number of \emph{odd} numbers yield an \emph{even} number;
        \item if $j$ is \emph{even}, assume $j=2k$ for some $k\in\mathbb{N}$, then
            $2^{\alpha}-2k+1$ coefficients are combined, which is an \emph{odd} number. 
            Adding an \emph{odd} number of \emph{odd} numbers yield an \emph{odd} number.
    \end{itemize}
\end{proof}


\begin{figure}[htb]

    \noindent\makebox[\textwidth]{
        \centering
        %\includegraphics[width=0.8\textwidth]{../../sympy/catalan/coloured.pdf}

        % using *angle* property to rotate it is difficult to properly align it
        % in order to have a "real" matrix representation.
        \includegraphics[width=10cm, height=3cm, keepaspectratio=true]
            {../RART2015/catalan-tikz/alternating-row/alternating-row.pdf}
    }

    % this 'particular' line is necessary to use `displaymath' environment
    % into the caption environment, togheter with the inclusion of 
    % `caption' package. See here for more explanation:
    % http://stackoverflow.com/questions/2716227/adding-an-equation-or-formula-to-a-figure-caption-in-latex
    \captionsetup{singlelinecheck=off}
    \caption[Row $\vect{r}_{2^{4}}$ of $\mathcal{C}_{\equiv_{2}}$]
        {Row $\vect{r}_{2^{4}}$ composed of alternating even and odd coefficients:
            $d_{2^{4},j}\equiv_{2}0 \leftrightarrow j$ odd,
                for $j\in\lbrace1,\ldots,2^{4} \rbrace$ }

    \label{fig:catalan-alternating-row}

\end{figure}

In \autoref{fig:catalan-alternating-row} row $\vect{r}_{2^{4}}$ is highlighted.

\subsection{On the \flqq mirror\frqq\, segment}

\begin{theorem}
    Choose any $\alpha\in\mathbb{N}$. \marginpar{we call $\Phi^{(\alpha)}$ the \flqq mirror\frqq\,segment}
    The segment $\Phi^{(\alpha)}=\diagup_{\lbrace 1,2,\ldots,2^{\alpha}-1\rbrace}^{2^{\alpha}-1}$, contained 
    in the principal cluster $\mathcal{C}^{(\alpha+1)}$, satisfies: 
    \begin{displaymath}
        d_{s,2^{\alpha}-1}\in\Phi^{(\alpha)} \rightarrow d_{s,2^{\alpha}-1}\equiv_{2}0
    \end{displaymath}
    where $s\in rows\left(\Phi^{(\alpha)}\right)=\lbrace2^{\alpha},\ldots,2^{\alpha+1}-2\rbrace$.
\end{theorem}

\begin{proof}
Let $d_{s,2^{\alpha}-1}$ a coefficient in the segment $\Phi^{(\alpha)}$, 
according to \autoref{eq:convolution:expansion:for:generic:element:in:catalan:array} 
write it as:
\begin{displaymath}
    d_{s, 2^{\alpha}-1} = \sum_{i_{1}+i_{2}+\ldots+i_{2^{\alpha}}=s+1}
        {c_{i_{1}-1}\,c_{i_{2}-1}\ldots\,c_{i_{2^{\alpha}}-1}}
\end{displaymath}
Observe that, for $s$ varing in the allowed range of $\Phi^{(\alpha)}$,
the number of Catalan coefficients
multiplied together in each summand is the same, namely $2^{\alpha}$. 
What changes respect to $s$ is the set of values 
each index $i_{j}$ can take. Look at the following table, where $j$ and $k$
are dummy variables:
\begin{displaymath}
    \begin{array}{c|c|c}
        s = 2^{\alpha} 
            & i_{j}\in\lbrace0,\ldots,2^{\alpha}+1\rbrace 
            & c_{k}\in\lbrace c_{-1},\ldots,c_{2^{\alpha}}\rbrace\\
        s = 2^{\alpha} +1
            & i_{j}\in\lbrace0,\ldots,2^{\alpha}+2\rbrace 
            & c_{k}\in\lbrace c_{-1},\ldots,c_{2^{\alpha}+1}\rbrace\\
        \vdots & \vdots&\vdots \\
        s = 2^{\alpha+1} -2
            & i_{j}\in\lbrace0,\ldots,2^{\alpha+1}-1\rbrace 
            & c_{k}\in\lbrace c_{-1},\ldots,c_{2^{\alpha+1}-2}\rbrace\\
    \end{array}
\end{displaymath}
observe that set $\lbrace c_{-1},\ldots,c_{2^{\alpha}}\rbrace$, the smaller one, and set
$\lbrace c_{-1},\ldots,c_{2^{\alpha+1}-2}\rbrace$, the bigger one, have the same subset
$\Omega^{(\alpha)}$ of \emph{odd} coefficients: 
\begin{displaymath}
    \Omega^{(\alpha)}=\lbrace c_{-1}, c_{2^{\alpha-(\alpha-1)}-1},c_{2^{\alpha-(\alpha-2)}-1},\ldots, 
        c_{2^{\alpha-1}-1},c_{2^{\alpha}-1}\rbrace
\end{displaymath}
where $\left|\Omega^{(\alpha)}\right|=\alpha+1$.
Since each summand term $c_{i_{1}-1}\,c_{i_{2}-1}\ldots\,c_{i_{2^{\alpha}}-1}$ 
has $2^{\alpha}$ coefficients, no matter if it contains each coefficient in $\Omega^{(\alpha)}$ and,
more importantly, one of them cannot belong to $\Omega^{(\alpha)}$:
the remaining ones make it vanish and $d_{s, 2^{\alpha}-1} \equiv_{2} 0$, for any suitable $s$,
as required.

\end{proof}


\begin{figure}[p]

    \noindent\makebox[\textwidth]{
        \centering
        %\includegraphics[width=0.8\textwidth]{../../sympy/catalan/coloured.pdf}

        % using *angle* property to rotate it is difficult to properly align it
        % in order to have a "real" matrix representation.
        \includegraphics[width=10cm, height=10cm, keepaspectratio=true]{catalan-tikz/mirror-segment/mirror-segment.pdf}
    }

    % this 'particular' line is necessary to use `displaymath' environment
    % into the caption environment, togheter with the inclusion of 
    % `caption' package. See here for more explanation:
    % http://stackoverflow.com/questions/2716227/adding-an-equation-or-formula-to-a-figure-caption-in-latex
    \captionsetup{singlelinecheck=off}
    \caption[.]{for $s\in S=\lbrace2^{\alpha},\ldots,2^{\alpha+1}-2\rbrace$, segment of 
        $\textcolor{blue}{d_{s,2^{\alpha}-1}}\equiv_{2}0$ lying on column $2^{\alpha}-1$,
        where $\alpha=4$}



    \label{fig:mirror-segment}

\end{figure}

In \autoref{fig:mirror-segment} the \flqq mirror\frqq\,segment $\Phi^{(4)}$ is highlighted.

\subsection{On the \flqq dual\frqq\,segment of the \flqq mirror\frqq\, segment}

\begin{theorem}
    Let \marginpar{we call set $\left\lbrace d_{s,s-(2^{\alpha}-1)}\right\rbrace$,
        for $s\in rows\left(\Phi^{(\alpha)}\right)$, the \flqq dual\frqq\, segment of 
    the \flqq mirror\frqq\,segment} 
    $\Phi^{(\alpha)}=\diagup_{\lbrace 1,2,\ldots,2^{\alpha}-1\rbrace}^{2^{\alpha}-1}$, 
    be a \flqq mirror\frqq\,segment in $\mathcal{C}^{(\alpha+1)}$, then: 
    \begin{equation}
        d_{s,2^{\alpha}-1}\in\Phi^{(\alpha)}\rightarrow d_{s,s-(2^{\alpha}-1)}\equiv_{2}d_{s,2^{\alpha}-1}
    \end{equation}
    where $s\in rows\left(\Phi^{(\alpha)}\right)=\lbrace2^{\alpha},\ldots,2^{\alpha+1}-2\rbrace$.
\end{theorem}

\begin{proof}
    Use \autoref{eq:catalan:array:second:identity} on both members:
    \begin{displaymath}
        {{s+2^{\alpha}-1}\choose{2^{\alpha}-1}}- {{s+2^{\alpha}-1}\choose{2^{\alpha}-2}} \equiv_{2}
        {{2s-2^{\alpha}+1}\choose{s-2^{\alpha}+1}}- {{2s-2^{\alpha}+1}\choose{s-2^{\alpha}}}
    \end{displaymath}
    by symmetry property of binomial coefficients:
    \begin{displaymath}
        {{s+2^{\alpha}-1}\choose{s}}- {{s+2^{\alpha}-1}\choose{s+1}} \equiv_{2}
        {{2s-2^{\alpha}+1}\choose{s}}- {{2s-2^{\alpha}+1}\choose{s+1}}
    \end{displaymath}
    by simplification using $(-1)^{-1}\mod 2=1$: 
    \begin{displaymath}
        {{s+2^{\alpha}-1}\choose{s}}+ {{s+2^{\alpha}-1}\choose{s+1}} \equiv_{2}
        {{2s-2^{\alpha}+1}\choose{s}}+ {{2s-2^{\alpha}+1}\choose{s+1}}
    \end{displaymath}
    by classic recurrence rule of binomial coefficients:
    \begin{displaymath}
        {{s+2^{\alpha}}\choose{s+1}} \equiv_{2} {{2s-2^{\alpha}+2}\choose{s+1}}
    \end{displaymath}
    since $s$ can assume $2^{\alpha}$ at least and $2^{\alpha+1}-2$ at most, 
    $s$ can be written in base $2$ as follows:
    \begin{displaymath}
        s=s_{0} + s_{1}2 + s_{2}2^{2}+\ldots+s_{\alpha-1}2^{\alpha-1}+2^{\alpha}
    \end{displaymath}
    and applying Lucas theorem we get:

    \begin{displaymath}
        \hspace{-2cm}
        {{s_{0}}\choose{s_{0}+1}}
        {{s_{1}}\choose{s_{1}}}
        \ldots
        {{s_{\alpha-1}}\choose{s_{\alpha-1}}}
        {{0}\choose{1}}
        {{1}\choose{0}}
        \equiv_{2}
        {{0}\choose{s_{0}+1}}
        {{s_{0}+1}\choose{s_{1}}}
        {{s_{1}}\choose{s_{2}}}
        \ldots
        {{s_{\alpha-2}}\choose{s_{\alpha-1}}}
        {{s_{\alpha-1}-1}\choose{1}}
        {{1}\choose{0}}
    \end{displaymath}
    simple algebra:
    \begin{displaymath}
        0
        \equiv_{2}
        {{0}\choose{s_{0}+1}}
        {{s_{0}+1}\choose{s_{1}}}
        {{s_{1}}\choose{s_{2}}}
        \ldots
        {{s_{\alpha-2}}\choose{s_{\alpha-1}}}
        {{s_{\alpha-1}-1}\choose{1}}
    \end{displaymath}
    \marginpar{in order to finish this proof we have to introduce a lemma about
        the row of alternating odd and even coefficient, which can be proved using
        the $A$-sequence of $\mathcal{C}$}
    By cases on the parity of $s$:
    \begin{itemize}
        \item assume $s$ is \emph{even}, therefore $s_{0}=0$ and the \ac{rhs} vanishes due to ${{0}\choose{s_{0}+1}}=0$;
        \item assume $s$ is \emph{odd}, therefore $s_{0}=1$, so apply Lucas theorem to ${{0}\choose{2}}$ again,
            yielding ${{0}\choose{2}}\equiv_{2}{{0}\choose{0}}{{0}\choose{1}}\equiv_{2}0$.
    \end{itemize}
    both cases shows that \ac{rhs} is a multiple of $p$, as required.
\end{proof}


\begin{figure}[p]

    \noindent\makebox[\textwidth]{
        \centering
        %\includegraphics[width=0.8\textwidth]{../../sympy/catalan/coloured.pdf}

        % using *angle* property to rotate it is difficult to properly align it
        % in order to have a "real" matrix representation.
        \includegraphics[width=10cm, height=10cm, keepaspectratio=true]
            {../RART2015/catalan-tikz/dual-of-mirror-segment/dual-of-mirror-segment.pdf}
    }

    % this 'particular' line is necessary to use `displaymath' environment
    % into the caption environment, togheter with the inclusion of 
    % `caption' package. See here for more explanation:
    % http://stackoverflow.com/questions/2716227/adding-an-equation-or-formula-to-a-figure-caption-in-latex
    \captionsetup{singlelinecheck=off}
    \caption[\emph{Dual} segment $\left\lbrace d_{s,s-(2^{4}-1)}\right\rbrace$,
        for $s\in rows\left(\Phi^{(4)}\right)$, in $\mathcal{C}_{\equiv_{2}}^{(5)}$]
        { \emph{Dual} segment $\left\lbrace d_{s,s-(2^{4}-1)}\right\rbrace$,
                for $s\in rows\left(\Phi^{(4)}\right)$, of \emph{mirror} segment $\Phi^{(4)}$ }



    \label{fig:dual-of-mirror-segment}

\end{figure}

In \autoref{fig:dual-of-mirror-segment} the \flqq dual\frqq\,segment 
    $\left\lbrace d_{s,s-(2^{4}-1)}\right\rbrace$,
    for $s\in rows\left(\Phi^{(4)}\right)$, of \flqq mirror\frqq\,segment 
    $\Phi^{(4)}$ in $\mathcal{C}_{\equiv_{2}}^{(5)}$ is highlighted.

\subsection{On the \emph{upside-down} zero-hole}

\begin{theorem}
    Let \marginpar{$H_{\bigtriangleup}^{({\alpha})}\subset\mathcal{C}_{\equiv_{2}}^{(\alpha+1)}$ }
    $\mathcal{C}_{\equiv_{2}}^{(\alpha+1)}$ be a principal cluster 
    of order $\alpha+1$ of the Catalan array $\mathcal{C}$. Then, 
    $H_{\bigtriangleup}^{({\alpha})}$ denotes an \emph{upside-down} zero-hole of order $\alpha$,
    such that coefficient $d_{nk}\in H_{\bigtriangleup}^{({\alpha})}$ if 
    $n\in\lbrace 2^{{\alpha}},\ldots,2^{{\alpha}+1}-2\rbrace$ and 
    $k\in\lbrace 0,\ldots, n-2^{{\alpha}}\rbrace$. 
\end{theorem}


\begin{proof}
We repeatedly use the approach
of the proof about the \flqq mirror\frqq\,segment, considering the set of columns 
$\Xi=\lbrace \vect{c}_{0},\ldots, \vect{c}_{2^{{\alpha}}-2}\rbrace$ and
for each column $\vect{c}_{k}\in\Xi$, the segment
    $\diagup_{\lbrace 2^{\alpha},2^{\alpha}+1,\ldots,2^{\alpha+1}-2-k\rbrace}^{k}$.

Everything is set, so start from column on the very left, column $\vect{c}_{0}$.
So the corresponding set of row indices  is a segment of Catalan numbers:
\begin{displaymath}
    S_{0}=\diagup_{\lbrace 2^{\alpha},2^{\alpha}+1,\ldots,2^{\alpha+1}-2\rbrace}^{0}
        = \lbrace c_{2^{\alpha}},c_{2^{\alpha}+1},\ldots,c_{2^{\alpha+1}-2}\rbrace
\end{displaymath}
since no coefficient $c_{j}\in S_{0}$ has the shape $c_{2^{\alpha}-1}$, 
all coefficients in $S_{0}$ are even.

Go ahead with column $\vect{c}_{1}$, so the corresponding segment is :
\begin{displaymath}
    S_{1}=\diagup_{\lbrace 2^{\alpha},2^{\alpha}+1,\ldots,2^{\alpha+1}-3\rbrace}^{1}
    %S_{1}=\lbrace d_{2^{{\alpha}}+1,1},\ldots,d_{2^{{\alpha}+1}-2,1} \rbrace
\end{displaymath}
and coefficients in it are defined according to:
\begin{displaymath}
    d_{s, 1} = \sum_{i_{1}+i_{2}=s+1} {c_{i_{1}-1}\,c_{i_{2}-1}}
\end{displaymath}
where $s\in rows(S_{1})= \left\lbrace 2^{\alpha}+1,2^{\alpha}+2,\ldots,2^{\alpha+1}-2\right\rbrace$. 
This is quite similar to the proof developed for the \flqq mirror\frqq\,segment,
with the difference that summand term is composed of two coefficients, namely
$c_{i_{1}-1}\,c_{i_{2}-1}$ instead of $2^{{\alpha}}$ coefficients as in the previous proof, 
therefore the same argument applies,
since if multiplying $2^{{\alpha}}$ coefficients fails to make \emph{not} vanish
the summand term, modulo $2$, the same failure is reached if multiplying only $2$ coefficients.

The same reasoning holds for remaining columns in $\Xi$: the last one of them is 
$\vect{c}_{2^{\alpha}-2}$ (with only \emph{one} coefficient, namely $d_{2^{\alpha}-2,2^{\alpha}-2}$), 
hence $H_{\bigtriangleup}^{({\alpha})}$ is an \emph{upside-down} zero-holes of order $\alpha$,
positioned at the very left in the bottom half of $\mathcal{C}_{\equiv_{2}}^{(\alpha+1)}$, as required.

\end{proof}


\begin{figure}[htb]

    \noindent\makebox[\textwidth]{
        \centering
        %\includegraphics[width=0.8\textwidth]{../../sympy/catalan/coloured.pdf}

        % using *angle* property to rotate it is difficult to properly align it
        % in order to have a "real" matrix representation.
        \includegraphics[width=10cm, height=10cm, keepaspectratio=true]
            {../RART2015/catalan-tikz/zero-hole/zero-hole.pdf}
    }

    % this 'particular' line is necessary to use `displaymath' environment
    % into the caption environment, togheter with the inclusion of 
    % `caption' package. See here for more explanation:
    % http://stackoverflow.com/questions/2716227/adding-an-equation-or-formula-to-a-figure-caption-in-latex
    \captionsetup{singlelinecheck=off}
    \caption[Upside-down zero-hole $H_{\bigtriangleup}^{(4)}$ within
        $\mathcal{C}_{\equiv_{2}}^{(\alpha+1)}$]{Zero hole $H_{\bigtriangleup}^{(4)} \subset \mathcal{C}_{\equiv_{2}}^{(5)}$}

    \label{fig:catalan-zero-hole}

\end{figure}

In \autoref{fig:catalan-zero-hole} is reported $H_{\bigtriangleup}^{(4)}$.

\subsection{On two \flqq mirrored\frqq\,clusters}

\begin{theorem}
    Let $\Phi^{(\alpha)}=\diagup_{\lbrace 2,\ldots,2^{\alpha}-1\rbrace}^{2^{\alpha}-1}$
    be a \flqq mirror\frqq\,segment and $\hat{d}_{s,2^{{\alpha}}-1}$ 
    be a coefficient in $\Phi^{(\alpha)}$, for some $s\in rows\left(\Phi^{(\alpha)}\right)$. Then:
    \begin{displaymath}
        d_{s-e,2^{{\alpha}}-1-e} \equiv_{2} d_{s,2^{{\alpha}}-1+e}
    \end{displaymath}
    for $e\in\lbrace1,\ldots,s-2^{{\alpha}}\rbrace$.
\end{theorem}

Before approaching a structured proof, we look for some insights.
According to \autoref{eq:convolution:expansion:for:generic:element:in:catalan:array},
rewrite both members:
\begin{displaymath}
    \sum_{i_{1}+i_{2}+\ldots+i_{2^{\alpha}-e}=s-e+1}
        {c_{i_{1}-1}\,c_{i_{2}-1}\ldots\,c_{i_{2^{\alpha}-e}-1}}
    \equiv_{2}
    \sum_{i_{1}+i_{2}+\ldots+i_{2^{\alpha}+e}=s+1}
        {c_{i_{1}-1}\,c_{i_{2}-1}\ldots\,c_{i_{2^{\alpha}+e}-1}}
\end{displaymath}
panic! No idea to tackle the general setting as a whole. So, start small and
let $s=2^{{\alpha}}+1$, therefore no choices for $e$, $e=1$ is mandatory:
\begin{displaymath}
    \sum_{i_{1}+i_{2}+\ldots+i_{2^{\alpha}-1}=2^{{\alpha}}+1}
        {c_{i_{1}-1}\,c_{i_{2}-1}\ldots\,c_{i_{2^{\alpha}-1}-1}}
    \equiv_{2}
    \sum_{i_{1}+i_{2}+\ldots+i_{2^{\alpha}+1}=2^{{\alpha}}+2}
        {c_{i_{1}-1}\,c_{i_{2}-1}\ldots\,c_{i_{2^{\alpha}+1}-1}}
\end{displaymath}

too difficult to proceed on this path, too. 

We attempt another \emph{false start} using closed formulae for
the generic element $d_{nk}\in\mathcal{C}$.
\begin{proof}
Assume not, therefore we need to yield a contraddiction if:
\begin{displaymath}
    d_{s-e,2^{{\alpha}}-1-e} \not\equiv_{2} d_{s,2^{{\alpha}}-1+e}
\end{displaymath}
holds. Rewriting the congruence according to \autoref{eq:catalan:array:first:identity}
 for $d_{nk}$:
\begin{displaymath}
    \frac{2^{{\alpha}}-e}{s-e+1}{{2(s-e)-(2^{{\alpha}}-1-e)}\choose{s-e-(2^{{\alpha}}-1-e)}}
    \not\equiv_{2}
    \frac{2^{{\alpha}}+e}{s+1}{{2s-(2^{{\alpha}}-1+e)}\choose{s-(2^{{\alpha}}-1+e)}}
\end{displaymath}
simplify it:
\begin{displaymath}
    \frac{2^{{\alpha}}-e}{s-e+1}{{2s-e-2^{{\alpha}}+1}\choose{s-2^{{\alpha}}+1}}
    \not\equiv_{2}
    \frac{2^{{\alpha}}+e}{s+1}{{2s-2^{{\alpha}}+1-e}\choose{s-2^{{\alpha}}+1-e}}
\end{displaymath}
manipulate using ${{n}\choose{k}}={{n}\choose{n-k}}$:
\begin{displaymath}
    \frac{2^{{\alpha}}-e}{s-e+1}{{2s-e-2^{{\alpha}}+1}\choose{s-e}}
    \not\equiv_{2}
    \frac{2^{{\alpha}}+e}{s+1}{{2s-2^{{\alpha}}+1-e}\choose{s}}
\end{displaymath}
dead end: nonetheless binomial coeffients allow some grunging,
fractions on both side are difficult to handle since it's hard
to find their multiplicative inverses, modulo $2$. A contraddiction
is difficult to be reached using this approach.
\end{proof} 

Trace back and use \autoref{eq:catalan:array:second:identity}. 
\begin{proof}
Recall we would like to prove the following statement:
\begin{displaymath}
    d_{s-e,2^{{\alpha}}-1-e} \equiv_{2} d_{s,2^{{\alpha}}-1+e}
\end{displaymath}
Rewrite the \ac{lhs}:
\begin{displaymath}
    d_{s-e,2^{{\alpha}}-1-e}= {{2(s-e)-(2^{{\alpha}}-1-e)}\choose{(s-e)-(2^{{\alpha}}-1-e)}}
        - {{2(s-e)-(2^{{\alpha}}-1-e)}\choose{(s-e)-(2^{{\alpha}}-1-e)-1}}
\end{displaymath}
in the same spirit, rewrite the \ac{rhs}:
\begin{displaymath}
    d_{s,2^{{\alpha}}-1+e}={{2s-(2^{{\alpha}}-1+e)}\choose{s-(2^{{\alpha}}-1+e)}}
        - {{2s-(2^{{\alpha}}-1+e)}\choose{s-(2^{{\alpha}}-1+e)-1}}
\end{displaymath}
therefore:
\begin{displaymath}
    \begin{split}
        {{2s-e-2^{{\alpha}}+1}\choose{s-2^{{\alpha}}+1}}
            - {{2s-e-2^{{\alpha}}+1}\choose{s-2^{{\alpha}}}}
        &\equiv_{2}
        {{2s-2^{{\alpha}}+1-e}\choose{s-2^{{\alpha}}+1-e}}
            - {{2s-2^{{\alpha}}+1-e}\choose{s-2^{{\alpha}}-e}}\\
        {{2s-e-2^{{\alpha}}+1}\choose{s-e}}
            - {{2s-e-2^{{\alpha}}+1}\choose{s-e+1}}
        &\equiv_{2}
        {{2s-2^{{\alpha}}+1-e}\choose{s}}
            - {{2s-2^{{\alpha}}+1-e}\choose{s+1}}\\
    \end{split}
\end{displaymath}

Since $e\in\lbrace1,\ldots,s-2^{{\alpha}}\rbrace$, proceed by complete induction on $e$:
\begin{itemize}
    \item base case $e=1$ yield the following congruence:
        \begin{displaymath}
                {{2s-2^{{\alpha}}}\choose{s-1}}-{{2s-2^{{\alpha}}}\choose{s}}
                \equiv_{2}
                {{2s-2^{{\alpha}}}\choose{s}}-{{2s-2^{{\alpha}}}\choose{s+1}}\\
        \end{displaymath}
        which is the same to say:
        \begin{displaymath}
                {{2s-2^{{\alpha}}}\choose{s-1}}+{{2s-2^{{\alpha}}}\choose{s+1}} \equiv_{2} 0
        \end{displaymath}
        let $s=s_{0}+s_{1}\,2+s_{2}\,2^{2}+\ldots+s_{{\alpha}-1}\,2^{{\alpha}-1} + 2^{{\alpha}}$
        be the generic representation of $s$ in base $2$, since 
        $s\in\lbrace 2^{{\alpha}}+1,\ldots,2^{{\alpha}+1}-2 \rbrace$; also  
        let $2s-2^{{\alpha}}=s_{0}\,2+s_{1}\,2^{2}+s_{2}\,2^{3}+\ldots+s_{{\alpha}-1}^{*}\,2^{{\alpha}} + s_{{\alpha}}^{*}\,2^{{\alpha}+1}$,
        where $(s_{{\alpha}-1}^{*},s_{{\alpha}}^{*})$ equals $(0,1)$ if $s_{{\alpha}-1}=1$, otherwise equals $(1,0)$.
        By cases on the parity of $s$:
        \begin{itemize}
            \item assume $s$ even, therefore both $s-1$ both $s+1$ are odd, 
                let $\hat{s}=1+\hat{s}_{1}\,2+\hat{s}_{2}\,2^{2}+\ldots+
                    \hat{s}_{{\alpha}-1}\,2^{{\alpha}-1}+2^{{\alpha}}$ be one of them, hence:
                \begin{displaymath}
                        {{2s-2^{{\alpha}}}\choose{\hat{s}}}  
                        \equiv_{2}
                        {{0}\choose{1}} 
                        {{0}\choose{\hat{s}_{1}}}
                        {{s_{1}}\choose{\hat{s}_{2}}}
                        \ldots
                        {{s_{{\alpha}-2}}\choose{\hat{s}_{{\alpha}-1}}}
                        {{s_{{\alpha}-1}^{*}}\choose{1}}
                        {{s_{{\alpha}}^{*}}\choose{0}} = 0
                \end{displaymath}
                Observe $\hat{s}_{{\alpha}}=1$ against boundary cases:
                if $s=2^{{\alpha}}+1$ then $\hat{s}=s-1=2^{{\alpha}}$, on the other
                hand if $s=2^{{\alpha}+1}-2$ then $\hat{s}=s+1=2^{{\alpha}+1}-1$,
                therefore in both cases the coefficient of $2^{{\alpha}}$ is $1$.
                Eventually we get $0+0 \equiv_{2}0$, which holds;

            \item assume $s$ odd, therefore both $s-1$ both $s+1$ are even, 
                let's study the former:
                \begin{displaymath}
                        {{2s-2^{{\alpha}}}\choose{s-1}}  
                        \equiv_{2}
                        {{0}\choose{0}} 
                        {{1}\choose{s_{1}}}
                        {{s_{1}}\choose{s_{2}}}
                        \ldots
                        {{s_{{\alpha}-2}}\choose{s_{{\alpha}-1}}}
                        {{s_{{\alpha}-1}^{*}}\choose{1}}
                        {{s_{{\alpha}}^{*}}\choose{0}}
                \end{displaymath}
                in order for the right hand side to not vanish, modulo $2$,
                it is mandatory for coefficients $\lbrace s_{i}\rbrace_{i\in\lbrace1,\ldots,{\alpha}-2\rbrace}$
                to satisfy $s_{i}\geq s_{i+1}$: if any one of them is $0$, say $s_{j}$, then
                $s_{j+1},\ldots,s_{j+k}$, with $j+k={\alpha}-1$,
                have to be all $0$ too. In particular, if $s_{{\alpha}-1}=0$ then 
                $(s_{{\alpha}-1}^{*},s_{{\alpha}}^{*})=(1,0)$
                therefore the right hand side reduces to $1$, modulo $2$.
                Observe that coefficients $\lbrace s_{i}\rbrace_{i\in\lbrace1,\ldots,{\alpha}-2\rbrace}$
                cannot be all $1$ otherwise 
                $s=(\underbrace{1,1,1,\ldots,1}_{{\alpha}+1})_{2}=2^{{\alpha}+1}-1$ raises a contraddiction, because
                $s$ can assume $2^{{\alpha}+1}-2$ at most. 
                
                For the latter, namely $s+1$, assume $s$ can be represented as 
                $(\underbrace{1,1,\ldots,1}_{r},0,s_{r+1},s_{r+2},\ldots,s_{{\alpha}-1},1)_{2}$, for
                $r\in\lbrace1,\ldots,{\alpha}-2\rbrace$, since a $0$ must occur otherwise $s=2^{{\alpha}+1}-1$
                which cannot be the case, as we've already seen.  Adding $1$ yield the representation
                $(\underbrace{0,0,\ldots,0}_{r},1,s_{r+1},s_{r+2},\ldots,s_{{\alpha}-1},1)_{2}$, therefore:
                \begin{displaymath}
                        {{2s-2^{{\alpha}}}\choose{s+1}} 
                        \equiv_{2} 
                        \underbrace{
                            {{0}\choose{0}} 
                            {{1}\choose{0}}
                            {{1}\choose{0}}
                            \ldots
                            {{1}\choose{0}}
                        }_{r}
                        {{1}\choose{1}}
                        {{0}\choose{s_{r+1}}}
                        {{s_{r+1}}\choose{s_{r+2}}}
                        \ldots
                        {{s_{{\alpha}-2}}\choose{s_{{\alpha}-1}}}
                        {{s_{{\alpha}-1}^{*}}\choose{1}}
                        {{s_{{\alpha}}^{*}}\choose{0}}
                \end{displaymath}
                In order to not vanish, modulo $2$, 
                $s_{r+1}=0$, this propagates in turn that $s_{r+2}=0$, \ldots, 
                this propagates in turn that $s_{{\alpha}-2}=0$,
                this propagates in turn that $s_{{\alpha}-1}=0$. But if $s_{{\alpha}-1}=0$
                then $(s_{{\alpha}-1}^{*},s_{{\alpha}}^{*})=(1,0)$, therefore the right
                hand side reduces to $1$.

                Combining the above cases for $s$ odd we reach:
                \begin{displaymath}
                        {{2s-2^{{\alpha}}}\choose{s-1}}+{{2s-2^{{\alpha}}}\choose{s+1}} \equiv_{2} 1+1\equiv_{2} 0
                \end{displaymath}
        \end{itemize}

        \item assume the argument holds for $k\leq e$ and prove for $k=e+1$, so we need to show:

            \begin{displaymath}
                \begin{split}
                    {{2s-(e+1)-2^{{\alpha}}+1}\choose{s-2^{{\alpha}}+1}}
                        - {{2s-(e+1)-2^{{\alpha}}+1}\choose{s-2^{{\alpha}}}}
                    &\equiv_{2}
                    {{2s-2^{{\alpha}}+1-(e+1)}\choose{s-2^{{\alpha}}+1-(e+1)}}
                        - {{2s-2^{{\alpha}}+1-(e+1)}\choose{s-2^{{\alpha}}-(e+1)}}\\
                    {{2s-(e+1)-2^{{\alpha}}+1}\choose{s-(e+1)}}
                        - {{2s-(e+1)-2^{{\alpha}}+1}\choose{s-(e+1)+1}}
                    &\equiv_{2}
                    {{2s-2^{{\alpha}}+1-(e+1)}\choose{s}}
                        - {{2s-2^{{\alpha}}+1-(e+1)}\choose{s+1}}\\
                    {{2s-e-2^{{\alpha}}}\choose{s-e-1}}
                        - {{2s-e-2^{{\alpha}}}\choose{s-e}}
                    &\equiv_{2}
                    {{2s-2^{{\alpha}}-e}\choose{s}}
                        - {{2s-2^{{\alpha}}-e}\choose{s+1}}\\
                \end{split}
            \end{displaymath}
            which follows directly by complete induction hypothesis.
\end{itemize}

\end{proof}


\begin{figure}[p]

    \noindent\makebox[\textwidth]{
        \centering
        %\includegraphics[width=0.8\textwidth]{../../sympy/catalan/coloured.pdf}

        % using *angle* property to rotate it is difficult to properly align it
        % in order to have a "real" matrix representation.
        \includegraphics[width=10cm, height=10cm, keepaspectratio=true]
            {../RART2015/catalan-tikz/mirrored-clusters/mirrored-clusters.pdf}
    }

    % this 'particular' line is necessary to use `displaymath' environment
    % into the caption environment, togheter with the inclusion of 
    % `caption' package. See here for more explanation:
    % http://stackoverflow.com/questions/2716227/adding-an-equation-or-formula-to-a-figure-caption-in-latex
    \captionsetup{singlelinecheck=off}
    \caption[\emph{Mirrored} principal clusters $\mathcal{C}_{\equiv_{2}}^{(4)}$]
        {Some coefficients in \emph{mirrored} cluster $\mathcal{C}_{\equiv_{2}}^{(4)}$ 
        over $\hat{d}_{20,2^{4}-1}$: congruences $d_{20-e,2^{4}-1-e} \equiv_{2} d_{20,2^{4}-1+e}$ 
        for $e\in\lbrace1,\ldots,4\rbrace$ }

    \label{fig:catalan-mirrored-clusters}

\end{figure}

In \autoref{fig:catalan-mirrored-clusters} some coefficients in \flqq mirrored\frqq\,
cluster $\mathcal{C}_{\equiv_{2}}^{(4)}$ are highlighted.

\subsection{$\mathcal{C}_{\equiv_{2}}^{(\alpha+1)}$ 
    contains a copy of $\mathcal{C}_{\equiv_{2}}^{(\alpha)}$}

In order to fully characterize $\mathcal{C}$ in a modular context we are
left with a last theorem.

\begin{theorem}
    $\mathcal{C}_{\equiv_{2}}^{(\alpha+1)}$ 
    contains a copy of $\mathcal{C}_{\equiv_{2}}^{(\alpha)}$,
    located at the very right of its bottom half. Formally, 
    let $\Phi^{(\alpha)}=\diagup_{\lbrace 2,\ldots,2^{\alpha}-1\rbrace}^{2^{\alpha}-1}$
    be a \flqq mirror\frqq\,segment and $\hat{d}_{s,2^{{\alpha}}-1}$ 
    be a coefficient in $\Phi^{(\alpha)}$, for some $s\in rows\left(\Phi^{(\alpha)}\right)$. Then:
    \begin{displaymath}
        d_{s,2^{{\alpha}}-1+e} \equiv_{2} d_{s-2^{{\alpha}},e-1}
    \end{displaymath}
    for $e\in\lbrace1,\ldots,s-2^{{\alpha}}\rbrace$.
\end{theorem}

\begin{proof}
We tackle the proof using \autoref{eq:catalan:array:second:identity}:

\begin{displaymath}
    \begin{split}
        {{2s-(2^{{\alpha}}-1+e)}\choose{s-(2^{{\alpha}}-1+e)}} - {{2s-(2^{{\alpha}}-1+e)}\choose{s-(2^{{\alpha}}-1+e)-1}}
        &\equiv_{2}
        {{2(s-2^{{\alpha}})-(e-1)}\choose{(s-2^{{\alpha}})-(e-1)}} - {{2(s-2^{{\alpha}})-(e-1)}\choose{(s-2^{{\alpha}})-(e-1)-1}}\\
        {{2s-2^{{\alpha}}+1-e}\choose{s-2^{{\alpha}}+1-e}} - {{2s-2^{{\alpha}}+1-e}\choose{s-2^{{\alpha}}-e}}
        &\equiv_{2}
        {{2s-2^{{\alpha}+1}-e+1}\choose{s-2^{{\alpha}}-e+1}} - {{2s-2^{{\alpha}+1}-e+1}\choose{s-2^{{\alpha}}-e}}\\
        {{2s-2^{{\alpha}}+1-e}\choose{s}} - {{2s-2^{{\alpha}}+1-e}\choose{s+1}}
        &\equiv_{2}
        {{2s-2^{{\alpha}+1}-e+1}\choose{s-2^{{\alpha}}}} - {{2s-2^{{\alpha}+1}-e+1}\choose{s-2^{{\alpha}}+1}}\\
    \end{split}
\end{displaymath}
Since $e\in\lbrace1,\ldots,s-2^{{\alpha}}\rbrace$, proceed by complete induction on $e$:
    \begin{itemize}
        \item base case $e=1$ therefore:

            \begin{displaymath}
                \begin{split}
                    {{2s-2^{{\alpha}}}\choose{s}} - {{2s-2^{{\alpha}}}\choose{s+1}}
                    &\equiv_{2}
                    {{2s-2^{{\alpha}+1}}\choose{s-2^{{\alpha}}}} - {{2s-2^{{\alpha}+1}}\choose{s-2^{{\alpha}}+1}}\\
                \end{split}
            \end{displaymath}
            In the previous proof a detailed (and boring) derivation has been performed,
            here we observe that upper terms of each binomial coefficient, $2s-2^{{\alpha}}$
            and $2s-2^{{\alpha}+1}$ respectively, have $2$ in their prime factorization, therefore
            expanding each binomial, $2$ can be factored out in turn, making congruent $0$
            each one of them. However an approach similar to the previous one can be
            taken as well.

        \item assume the argument holds for $k\leq e$ and prove for $k=e+1$, so we need to show:
            \begin{displaymath}
                \hspace{-4cm}
                \begin{split}
                    {{2s-2^{{\alpha}}+1-(e+1)}\choose{s}} - {{2s-2^{{\alpha}}+1-(e+1)}\choose{s+1}}
                    &\equiv_{2}
                    {{2s-2^{{\alpha}+1}-(e+1)+1}\choose{s-2^{{\alpha}}}} - {{2s-2^{{\alpha}+1}-(e+1)+1}\choose{s-2^{{\alpha}}+1}}\\
                    {{2s-2^{{\alpha}}-e}\choose{s}} - {{2s-2^{{\alpha}}-e}\choose{s+1}}
                    &\equiv_{2}
                    {{2s-2^{{\alpha}+1}-e}\choose{s-2^{{\alpha}}}} - {{2s-2^{{\alpha}+1}-e}\choose{s-2^{{\alpha}}+1}}\\
                \end{split}
            \end{displaymath}
            which follows directly by complete induction hypothesis.
    \end{itemize}
\end{proof}


\begin{figure}[p]

    \noindent\makebox[\textwidth]{
        \centering
        %\includegraphics[width=0.8\textwidth]{../../sympy/catalan/coloured.pdf}

        % using *angle* property to rotate it is difficult to properly align it
        % in order to have a "real" matrix representation.
        \includegraphics[width=6cm, height=6cm, keepaspectratio=true]{../RART2015/catalan-tikz/principal-cluster/principal-cluster.pdf}
    }

    % this 'particular' line is necessary to use `displaymath' environment
    % into the caption environment, togheter with the inclusion of 
    % `caption' package. See here for more explanation:
    % http://stackoverflow.com/questions/2716227/adding-an-equation-or-formula-to-a-figure-caption-in-latex
    \captionsetup{singlelinecheck=off}
    \caption[$\mathcal{C}_{\equiv_{2}}^{(4)}$ 
    contains a copy of $\mathcal{C}_{\equiv_{2}}^{(3)}$]{$\hat{d}_{s,2^{3}-1}$ for $s\in S_{2^{3}-1}$,
    $d_{s,2^{3}-1+e} \equiv_{2} d_{s-2^{3},e-1}$ with $e\in\lbrace1,\ldots,s-2^{3}\rbrace$ }

    \label{fig:catalan-principal-cluster}

\end{figure}


Before concluding the modular characterization of $\mathcal{C}$, 
we would like to observe that the last two
theorems do not say anything about the \emph{value} of remainder for 
a coefficient belonging to subtriangles of interest: 
we have only shown that a \emph{complete} subtriangle is repeated 
when coefficients are taken modulo $2$. 


\subsection{Some open questions}

\input{catalan/catalan-traditional-inverse-ignore-negatives-centered-colouring-127-rows-mod2-partitioning-include-figure.tex}
We finish this section leaving some problems we are currently working on:
\begin{itemize}
    \item find a modular characterization of $\mathcal{C}_{\equiv_{2}}^{-1}$, reported in 
        \autoref{fig:catalan-traditional-inverse-ignore-negatives-centered-colouring-127-rows-mod2-partitioning-triangle}
    \item how the characterization described in this section can be generalized to congruences modulo an arbitrary prime $p$?
\end{itemize}
