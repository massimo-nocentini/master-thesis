
\noindent This project aims to study formal methods and their application to
the analysis of algorithms and data structures.

\marginpar{analytic combinatorics and discrete objects}A solid base for such
methods comes from the field of analytic combinatorics, which comprises tools
such as generating functions, Riordan arrays and the symbolic method. Many
interesting books by Flajolet and Sedgewick\footnote{P. Flajolet and R.
Sedgewick, \emph{Analytic Combinatorics}, Cambridge University Press, 2009.},
Knuth\footnote{D. Knuth, \emph{The Art of Computer Programming}, vol.  1-3,
Addison-Wesley, 1973.} and Graham et al. \footnote{Graham, Knuth and Patashnik,
\emph{Concrete Mathematics: A Foundation for Computer Science}, Addison-Wesley,
1994} exist on those topics, where skillful methods to handle sequences of
\emph{counting numbers}, combinatorial sums and classes of combinatorial
discrete objects, such as graphs, words, lattice paths and trees, are
presented. 

Actually, the goal of this project is twofold: 
\begin{itemize}

    \item \marginpar{to find combinatorial meanings and\ldots} to study how
    such powerful techniques provide a combinatorial \emph{interpretation} of
    analytic results about a class $\mathcal{G}$ of objects, usually stated
    using \emph{formal power series} $g(t)=\sum_{k\geq0}{g_{k}\,t^{k}}$ in the
    ring $\mathbb{Z}\llbracket t \rrbracket$: our final interest is to find
    \emph{combinatorial meanings} for each coefficient in $\lbrace
    g_{k}\rbrace_{k\in\mathbb{N}}$ and, possibly, a characterization in terms
    of lattice paths, urn models, bracelet configurations, domino and square
    tiling and so on, in the spirit of Benjamin and Quinn \footnote{Benjamin
    and Quinn, \emph{Proofs that really counts}, Mathematical Association of
    America, 2003} and Stanley\footnote{Richard Stanley, \emph{Enumerative
    combinatorics. {V}ol. 1\&2}, Cambridge University Press}. Moreover, we want
    to apply such techniques and interpretations to the analysis of algorithms
    as far as the analytic aspect is concerned, and to data structures for the
    combinatorial one. % here France advice to substitute `one' with `aspect'. 

    \item to give an implementation of the
        \emph{symbolic method}, different from the one supplied by \emph{Maple}
        \footnote{\texttt{combstruct} Maple package, proprietary implementation and
        closed-source}. We aim at: 
            
        \marginpar{\ldots to implement a mathematical language}    
        \begin{itemize} 

            \item an \emph{open-source} code base with an \emph{on-line} interactive
            session, in the spirit of \emph{IPython
            notebooks}\footnote{\url{http://ipython.org}, in particular the
            \emph{Notebook} feature}, to provide an open web interface to our
            implementations; 

            \item a tight integration with \emph{Eclipse IDE}, developing a set of
            plug-ins using the \emph{Xtext}\footnote{\url{https://eclipse.org/Xtext/}}
            framework, with the idea of defining a mathematical language that allow us to
            handle combinatorial and discrete objects in a simple and flexible way. Such
            a language should have constructs to define classes of objects by the
            introduction of \emph{datatypes}; constructs to attach measures of interest
            to each one of them; and, finally, some computational rules in order to
            combine matrices, generating functions and arrays in uniform and
            polymorphic ways; 

            \item a set of \emph{LISP macros} to tackle enumerative problems,
            where the generation of discrete objects according to a given
            distribution is difficult for arbitrary classes: using macros, much
            of the complexity is moved from \emph{run-time} to
            \emph{compile-time}, as a kind of \emph{tabling} technique.
            Moreover, with the introduction of \emph{macros}, we are in a
            position to extend the \emph{LISP} language, in order to provide a
            \emph{domain specific language} which, possibly, allows us
            embedding existing computer algebra systems, in particular
            \emph{Maxima}\footnote{\emph{Maxima}, a Computer Algebra System,
            \url{http://maxima.sourceforge.net/}}, which is itself written in
            \emph{LISP}.

        \end{itemize}
\end{itemize}

\marginpar{Riordan arrays: $A$-sequence, $A$-matrix and some generalizations}
At the same time, we want to continue to work on the subjects studied in my
master thesis, in particular the topic of \emph{Riordan arrays}. We are
interested in finding new characterizations to spot some properties of their
structure. One example is the $h$-characterization $\mathcal{R}_{h(t)}$ of a
Riordan array $\mathcal{R}$ found by us and explored within the thesis.
Another path to follow is the study of two particular objects, $A$-sequence
$\lbrace a_{n}\rbrace_{n\in\mathbb{N}}$ and $A$-matrix $\lbrace
a_{ij}\rbrace_{i,j\in\mathbb{N}}$ respectively, generalizing them in order to
discover new combinatorial identities.%, possibly involving row sums.

Although \emph{Riordan group}\marginpar{Riordan group theory: from the
basics\ldots} theory has been studied intensively in the recent past, we would
like to rethink about the original and introductory papers of this theory, in
particular those by \emph{Shapiro}, who introduces the \emph{Riordan group} and
builds triangles from lattice paths under some constraints; by \emph{Rogers},
who introduces the concept of \emph{renewal arrays} and finds the important
concept of their $A$-sequences; by \emph{Eplett}, who provides an identity
involving determinants and \emph{Catalan} numbers; and, finally, by
\emph{Sprugnoli}, who uses Riordan arrays in order to find generating
functions of combinatorial sums in a \emph{constructive} way, not just proving
that a sum equals a \emph{given} value (usually denoted by a closed formula). 

\marginpar{\ldots toward a modular arithmetic point of view} The other topic of
my thesis shows our interest in the description and formalization of Riordan
arrays under the light of modular arithmetic. We have shown some congruences
about \emph{Pascal} array $\mathcal{P}$ and its inverse $\mathcal{P}^{-1}$. We
have also proved a formal characterization for the \emph{Catalan} array
$\mathcal{C}$. These results were presented in a talk contributed at a recent
conference held in Lecco\footnote{Second International Symposium on Riordan
Arrays and Related Topics, RART$2015$}. All major researchers involved in Riordan group
theory were present and some of them threw some important ideas, such as
\emph{Mandelbrot} and \emph{Julia} sets and how those concepts could be used to
build proofs without looking for a closed formula of the generic coefficient
$d_{nk}$ in an array $\mathcal{R}$.  

\marginpar{Mandelbrot and Julia sets.  Orthogonal polynomials and continued
fractions}At the same conference, the topics of \emph{orthogonal polynomials} and
\emph{continued fraction expansion} arose in many talks and we would like to
study them, starting from Hennessy's Ph.D. thesis\footnote{Aoife Hennessy, \emph{A Study of
Riordan Arrays with Applications to Continued Fractions, Orthogonal Polynomials
and Lattice Paths}, Ph.  D.  thesis}.

