
\noindent This project aims to study formal methods and their application to
the analysis of algorithms and data structures used by them.

\marginpar{analytic combinatorics and discrete objects}A solid base for such
methods comes from the field of analytic combinatorics, which comprises tools
such as generating functions, Riordan arrays and the symbolic method. Many
interesting books \footnote{P. Flajolet and R. Sedgewick, \emph{Analytic
Combinatorics}, Cambridge University Press, 2009.}, \footnote{D. Knuth,
\emph{The Art of Computer Programming}, vol.  1-3, Addison-Wesley, 1973.} and
\footnote{Graham, Knuth and Patashnik, \emph{Concrete Mathematics: A Foundation
for Computer Science}, Addison-Wesley, 1994} exist on those subjects, where
    skillful methods to handle sequences of \emph{counting numbers},
    combinatorial sums and classes of discrete objects, such as graphs, words
    and trees, are presented. 

Actually, the goal of this project is twofold:\marginpar{to find combinatorial
meanings and\ldots} we're interested in studying how such powerful techniques
provide a combinatorial \emph{interpretation} of analytic results about a class
$\mathcal{G}$ of objects, usually stated using a \emph{formal power series}
$g(t)=\sum_{k\geq0}{g_{k}\,t^{k}}$ in the ring $\mathbb{Z}\llbracket t
\rrbracket$: our final interest is to find a \emph{combinatorial meanings} for
each coefficient in $\lbrace g_{k}\rbrace_{k\in\mathbb{N}}$ and, possibly, a
characterization in terms of lattice paths, urn models, bracelets
configurations, domino and square tiling and so on, in the spirit of
\footnote{Benjamin and Quinn, \emph{Proofs that really counts}, Mathematical
Association of America, 2003}.  Moreover, we would apply such techniques and
interpretations to the analysis of algorithms for what concerns the analytic
aspect, and to data structures for what concerns the combinatorial one. 

The second aspect of our goal is to give an implementation of the
\emph{symbolic method}, departing from \footnote{\texttt{combstruct} Maple
package, proprietary implementation and closed-source} toward: 
    
\marginpar{\ldots to implement a mathematical language}    
\begin{itemize} 

    \item an \emph{open-source} code base with an \emph{on-line} interactive
    session, in the spirit of \footnote{\url{http://ipython.org}, in particular
    the \emph{Notebook} feature}, to provide an open web interface to our
    implementations; 

    \item a tight integration with \emph{Eclipse IDE}, developing a set of
    plug-ins using the \emph{Xtext}\footnote{\url{https://eclipse.org/Xtext/}}
    framework, with the idea to define a mathematical language that allow to
    handle combinatorial and discrete objects in a simple a flexible way. Such
    a language should have constructs to define classes of objects by the
    introduction of \emph{datatypes}; constructs to attach measures of interest
    to each one of them; and, finally, some computational rules in order to
    combine matrices, generating functions and arrays in uniform and
    polymorphic ways; 

    \item a set of \emph{LISP macros} to tackle enumerative problems, where the
    generation of discrete objects according a distribution over them is
    difficult for arbitrary classes: using macros, much of the complexity is
    moved from \emph{run-time} to \emph{compile-time}, as a kind of
    \emph{tabling} technique.

\end{itemize}

\marginpar{Riordan arrays: $A$-sequence, $A$-matrix and some generalizations}
At the same time, we would continue to work on the subjects studied in our
master thesis, in particular the topic of \emph{Riordan arrays}. We are
interested in finding new characterizations to spot some properties, one
example is the $h$-characterization $\mathcal{R}_{h(t)}$ of a Riordan array
$\mathcal{R}$ found by us and explored within the thesis.  Another track to
follow is the study of two particular objects, $A$-sequence $\lbrace
a_{n}\rbrace_{n\in\mathbb{N}}$ and $A$-matrix $\lbrace
a_{ij}\rbrace_{i,j\in\mathbb{N}}$ respectively, generalizing them in order to
discover new combinatorial identities, possibly involving row sums.

Nonetheless \emph{Riordan group}\marginpar{Riordan group theory: from the
basics\ldots} theory is studied intensively in the recent past, we would like
to rethink about original and introductory papers, in particular those by
\emph{Shapiro}, which introduces the \emph{Riordan group} and builds triangles
from lattice paths under some constraints; by \emph{Rogers}, which introduces
the concept of \emph{renewal arrays} and finds the important concept of their
$A$-sequences; by \emph{Eplett}, which gives a nice identity involving
determinants and \emph{Catalan} numbers; and, finally, by \emph{Sprugnoli},
which uses Riordan arrays in order to find generating functions of
combinatorial sums in a \emph{constructive} way, not just proving that a sum
equals a \emph{given} value (usually denoted by a closed formula). 

\marginpar{\ldots toward a modular arithmetic point of view} The other topic of
our thesis shows our interest in the description and formalization of Riordan
arrays under the light of modular arithmetic: we've shown some congruences
about \emph{Pascal} array $\mathcal{P}$ and its inverse $\mathcal{P}^{-1}$; a
formal characterization for the \emph{Catalan} array $\mathcal{C}$ has been
proved also. These results were presented in a contributed talk
\footnote{Second International Symposium on Riordan Arrays and Related Topics,
held in Lecco last July}: at the conference all major researchers involved in
Riordan group theory were present and some of them threw some important ideas,
such as \emph{Mandelbrot} and \emph{Julia} sets and how those concepts could be
used to build proofs without looking for a closed formula of the generic
coefficient $d_{nk}$ in an array $\mathcal{R}$.  

\marginpar{Mandelbrot and Julia sets.  Orthogonal polynomials and continued
fractions}Finally, the topics of \emph{orthogonal polynomials} and
\emph{continued fraction expansion} arise in many talks and we would like to
study them, starting from \footnote{Aoife Hennessy, \emph{A Study of
Riordan Arrays with Applications to Continued Fractions, Orthogonal Polynomials
and Lattice Paths}, Ph.  D.  thesis}.

