
\noindent This project aims to study formal methods to be applied to the
analysis of algorithms and to data structures used by them.
\marginpar{analytic combinatorics and discrete objects}A solid base for
the former ones comes from analytic combinatorics, which comprises tools
such as generating functions, Riordan arrays and the symbolic method.
For the latter ones, on the other hand, results about permutations,
lattice path and combinatorial objects' classes provide many insights.
Many interesting books  [1]\footnote{[1] P. Flajolet and R. Sedgewick,
\emph{Analytic Combinatorics}, Cambridge University Press, 2009.},
[2]\footnote{[2] D. Knuth, \emph{The Art of Computer Programming}, vol.
1-3, Addison-Wesley, 1973.} and [3]\footnote{[3] Graham, Knuth and
Patashnik, \emph{Concrete Mathematics: A Foundation for Computer
Science}, Addison-Wesley, 1994} exists on those subjects, where skillful
methods to handle sequences of \emph{counting numbers}, combinatorial
sums and classes of discrete objects, such as graphs, words and trees,
are contained within. Actually, the goal of this project is
twofold:\marginpar{combinatorial meaning} we're interested to study how
such powerful techniques can help in providing a combinatorial
\emph{interpretation} of analytic results about a class $\mathcal{G}$ of
objects, usually stated using the ring of \emph{formal power series}
$\llbracket g(t)=g_{0}+g_{1}\,t+
g_{2}\,t^{2}+\ldots+g_{k}\,t^{k}+\ldots\rrbracket_{k\in\mathbb{N}}$, so
the final interest is to find the \emph{combinatorial meaning} of each
coefficient $g_{k}$ and possibly an explanation for it in terms of
lattice paths, urn models, bracelets configurations, domino and square
tiling and so on, in the spirit of [4]\footnote{[4] Benjamin and Quinn,
\emph{Proofs that really counts}, Mathematical Association of America,
2003}. Moreover, we would apply such techniques and interpretations to
the analysis of algorithms for what concerns the analytic aspect, and to
data structures for what concerns the combinatorial one, in particular
we would build a language\marginpar{apply formal methods introducing a
language}, and later giving it an implementation, in order to play with
classes of combinatorial objects, providing a formal way to define what
objects can be manipulated and how they can be combined, in a
\emph{objective} style; only after, integration with a symbolic engine
would be set in order to derive actual results, so we give importance to
the modeling phase too. In order to provide some idea, such a language
could be implemented using the \emph{Xtext} [5]\footnote{[5]
\url{https://eclipse.org/Xtext/}} framework and all the related tools as
a set of \emph{Eclipse plug-ins}; another possible implementation is an
``old-style'' one, toward writing a \emph{Vim plugin} that allows a deep
integration with other UNIX tools, in order to have an interactive Vim
session. \marginpar{Riordan arrays: $A$-sequence, $A$-matrix and some
generalizations} At the same time, we would continue to work on the
subjects studied in our master thesis, in particular the topic of
\emph{Riordan arrays}. We are interested in finding new
characterizations that allow to spot some properties, one example is the
$h$-characterization $\mathcal{R}_{h(t)}$ of a Riordan array
$\mathcal{R}$ found by us and explored within the thesis. Another track
to follow is the study of two particular objects, $A$-sequence $\lbrace
a_{n}\rbrace_{n\in\mathbb{N}}$ and $A$-matrix $\lbrace
a_{ij}\rbrace_{i,j\in\mathbb{N}}$ respectively, generalizing them in
order to discover new combinatorial identities, possibly involving row
sums. \marginpar{Riordan group theory: back to basics} This lead us
\emph{backwards}, since we believe that introductory papers by
\emph{Shapiro}, which introduces the \emph{Riordan group} and builds
triangles from lattice paths under some constraints; by \emph{Rogers},
which introduces the concept of \emph{renewal arrays} and finds the
important concept of their $A$-sequences; by \emph{Eplett}, which gives
a nice identity involving determinants and \emph{Catalan} numbers;
finally, by \emph{Sprugnoli}, which uses Riordan arrays in order to find
generating functions of combinatorial sum in a \emph{constructive} way,
not just proving that sums equals a \emph{given} value (usually denoted
by a closed formula); in our opinion, all their works contain some ideas
that allow further developments and insights, nonetheless they have been
studied extensively till now. \marginpar{Riordan arrays from modular
arithmetic point of view} Another track is to continue the description
and formalization of a Riordan array under the light of modular
arithmetic: we've shown some congruences about \emph{Pascal} array
$\mathcal{P}$ and its inverse $\mathcal{P}^{-1}$; a formal description
for the \emph{Catalan} array $\mathcal{C}$ has been proved also. All
    this joint work with Prof. \emph{Donatella Merlini} has been shown
    as contributed talk at [6]\footnote{[6] Second International
    Symposium on Riordan Arrays and Related Topics}, held in Lecco last
    July. At the conference all major researchers involved in Riordan
    group theory was present and some of them advice us with some
    important ideas: we would like to study if \emph{Mandelbrot} and
    \emph{Julia} sets can help in proofs of modular formalization by
    algebraic manipulation, avoiding looking for a closed formula for
    the generic coefficient $d_{nk}$ in an array $\mathcal{R}$.
    \marginpar{Mandelbrot and Julia sets. Orthogonal polynomials and
    continued fractions} Moreover, the topics of \emph{orthogonal
    polynomials} and \emph{continued fraction expansion} arise in many
    talks and we would like to study them starting from [7]\footnote{[7]
    Aoife Hennessy, \emph{A Study of Riordan Arrays with Applications to
    Continued Fractions, Orthogonal Polynomials and Lattice Paths}, Ph.
    D.  thesis}.

