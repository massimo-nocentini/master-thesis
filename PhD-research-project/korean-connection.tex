
\section{Korean connection} The analysis of algorithms and data
structures studies the efficiency of algorithms and properties of data
structures according to some characteristic parameters. It has become
popular after the publication of the books [2] by D. Knuth; in the last
decades, it was developed in a large number of papers and books, among
which [1,3]. The aim is to give precise information on the efficiency of
any kind of algorithm looking for its worst, best and average behavior
and to precisely define the data structures used by it. By using the
results so obtained, it is then possible to choose the best among
different algorithms performing the same task or between different data
structures to store the data of interests. These goals can be reached by
using some formal tools belonging to the field of analytic
combinatorics, such as generating functions, Riordan arrays and the
symbolic method. The purpose of this proposal is twofold: on the one
hand we want to delve the study of these methods from a theoretical
point of view. On the other hand, we are interested to apply these
techniques in combinatorial and algorithmic contexts arising from areas
such as multidimensional lattice paths and queueing theory.


The partners of this proposal are expert in formal methods for the
analysis of algorithms and combinatorial data structures; these
methods belong to the field of analytic combinatorics. In this
community, the groups in Italy and in Korea are related by the common
interest in generating functions and Riordan arrays, for what concerns
the algebraic aspects, and in lattice paths, words, trees,
permutations, graphs, and so on, for what concerns the
combinatorics. Therefore, the aim of the present proposal is to deepen
the study of formal methods for the analysis of algorithms from an
algebraic and combinatorial point of view. In general, data structures
used in computer science can be considered as combinatorial objects
and the study of these structures is fundamental for the correct
analysis of the corresponding algorithms. For example, the random and
exhaustive generation of these structures can be used both to examine
the behaviour of the same structures and to evaluate the performance
of the algorithms that use them. Analytic combinatorics is a branch of
combinatorics that describes combinatorial classes using generating
functions and aims at predicting precisely the properties of large
structured combinatorial configurations. This approach starts from an
exact enumerative description of combinatorial structures by means of
generating functions; the next step consists in extracting their
coefficients in exact or asymptotic form and thus finding results for
the corresponding counting sequences. When a bivariate generating
function describes a sequence which depends on two indexes and has a
particular form, we call it a Riordan Array. Riordan arrays strongly
simplify computations about this kind of sequences in addition to
having an interesting algebraic structure and to provide a powerful
and simple method to obtain the solution of combinatorial sums and
lattice path counting problems. At international level, the partners
of this proposal are among the main experts


The general effort to precisely predict the performance of algorithms
has come to involve research in analytic combinatorics, the analysis
of random discrete structures, asymptotic analysis, exact and limiting
distributions, and other fields of inquiry in computer science,
probability theory, and enumerative combinatorics. In this project we
will study these aspects by using an approach based on (probability)
generating functions and the method of coefficients, Riordan arrays,
combinatorial sums and the symbolic method. The relevance of these
concepts in the context of the analysis of algorithms and data
structures has been illustrated in the books [1,2,3] listed below and
in many others books and papers. Generating functions are used in
combinatorics since 300 years and in the last two decades they have
been one of the mostly used devices in the analysis of algorithms. In
fact, they serve both as a combinatorial tool to facilitate Counting
and as an analytic tool to develop precise estimates for quantities of
interest. The method of coefficients represents a completion of the
previous concept while the symbolic method allows to develop
relationships between combinatorial structures and the associated
generating functions. Riordan arrays have been introduced in [4] with
the aim of defining a class of infinite lower triangular arrays with
properties analogous to those of the Pascal triangle. Few years later,
it was pointed out the importance of these matrices in the computation
of combinatorial sums. Since then, Riordan arrays have attracted, and
continue to attract, much attention in the literature. In particular,
the algebraic structure of these matrices, their relationship with the
computation of combinatorial sums and many combinatorial applications
have been studied over the years. Combinatorial identities have
received attention since very old times, and part of this literature
has a strong connection with the analysis of algorithms. In this
project, the role of the Italian group will focus on the theoretical
aspect related to the concept of Riordan arrays, with particular
emphasis on their characterization in terms of A-sequence and
A-matrix, and on their application to various combinatorial contexts,
such as lattice paths and words avoiding a specific pattern, as well
as on their use to analyse algorithms related to such combinatorial
structures. Another field of research will be the study of Riordan
arrays in modular arithmetic with the aim of detecting regularities in
the combinatorial structures which can be enumerated in terms of this
kind of matrices. All these studies will be accompained by simulation
with a system for symbolic computation. The role of the Korean group
will focus on the algebraic and combinatorial properties of the
multidimensional and multivariate Riordan arrays, with particular
emphasis on the analysis of algorithms and data structures in
analysing large-scale and high dimensional data. Riordan arrays can be
interpreted as another way to organize the data of Riordan bases using
matrices. Schauder bases provide a flexible way to represent power
series. A special type of Schauder bases, Riordan bases, is convenient
for computations. The idea carries over to the multivariate Riordan
arrays. The group will also study on their applications to queuing
theory together with high dimensional lattice path counting problems.

