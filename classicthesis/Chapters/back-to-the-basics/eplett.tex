
\subsection{Eplett identity, by Eplett}

\citeauthor{riordan:intro:combinatorial:analysis},
in his book \cite{riordan:intro:combinatorial:analysis},
asks for a solution of the following exercise:

\begin{quote}
    let $\lbrace a_{n} \rbrace_{n\in\mathbb{N}}$ be a sequence, with $a_{0}\neq0$, 
    and $\lbrace a_{n}^{\prime} \rbrace_{n\in\mathbb{N}}$ be its \emph{inverse}:
    if $A(t)$ and $A^{\prime}(t)$ are two \ac{fps} over them, respectively, 
    then $A(t)A^{\prime}(t)=1$. Show that:
    \begin{displaymath}                
        a_{n}^{\prime} = \frac{(-1)^{n}}{a_{0}^{n+1}}
            \left|
            \begin{array}{ccccc}
                a_1 & a_2 & \ldots & a_{n-1} & a_{n}\\
                a_0 & a_1 & \ldots & a_{n-2} & a_{n-1}\\
                0   & a_0 & \ldots & a_{n-3} & a_{n-2}\\
                \vdots & \vdots & \vdots & \vdots & \vdots\\
                0 & 0 & \ldots & a_{1} & a_{2}\\
                0 & 0 & \ldots & a_{0} & a_{1}\\
            \end{array}
            \right|
    \end{displaymath}                
    for $n \geq 2$.
    \marginpar{another way to find the inverse of a given sequence}
\end{quote}

\begin{proof}
Although not explicitly requested, %to show for $n\geq2$, 
it's useful to recall formula about $a_{0}^{\prime}$ and $a_{1}^{\prime}$, 
since they'll occur in the following. 

First of all we interpret the product
$A(t)A^{\prime}(t)=1$ as the Cauchy product over the ring of \ac{fps}, 
namely as a convolution, therefore: 
\begin{displaymath}
    \left[t^{0}\right]A(t)A^{\prime}(t)=a_{0}a_{0}^{\prime}=1
\end{displaymath}
is the relation which defines $a_{0}^{\prime}=\frac{1}{a_0}$.  On the other hand: 
\begin{displaymath}
    \left[t^{1}\right]A(t)A^{\prime}(t)=a_{0}a_{1}^{\prime}+a_{1}a_{0}^{\prime}=0
\end{displaymath}
is the relation which defines $a_{1}^{\prime}=-\frac{a_1}{a_{0}^{2}}$.

The proof proceeds by induction on $n$:
\begin{itemize}
    \item base $n=2$:
        \begin{displaymath}                
            a_{2}^{\prime} = \frac{1}{a_{0}^{3}}
                \left|
                \begin{array}{cc}
                    a_1 & a_2 \\
                    a_0 & a_1 \\
                \end{array}
                \right| 
                = \frac{a_{1}^{2}-a_{0}a_{2}}{a_{0}^{3}}
        \end{displaymath}                
        since $\big[t^{2}\big]A(t)A^{\prime}(t)=a_{0}a_{2}^{\prime}
            +a_{1}a_{1}^{\prime}+a_{2}a_{0}^{\prime}=0$, we have:
            \begin{displaymath}
                \begin{split}
                    a_{0}a_{2}^{\prime} &= -\left(a_{1}a_{1}^{\prime}+a_{2}a_{0}^{\prime}\right)
                        = -\left(-\left(\frac{a_{1}}{a_{0}}\right)^{2}+\frac{a_{2}}{a_{0}}\right)\\
                    a_{2}^{\prime} &= \frac{a_{1}^{2}}{a_{0}^{3}}-\frac{a_{2}}{a_{0}^{2}}
                \end{split}
            \end{displaymath}
        which proves the base case;

    \item assume the statement holds for $n$ and show it still holds for $n+1$, therefore
        we're left to prove the following:
    \begin{displaymath}                
        a_{n+1}^{\prime} = 
            -\frac{1}{a_{0}}\frac{(-1)^{n}}{a_{0}^{n+1}}
            \left|
            \begin{array}{ccccc}
                a_1 & a_2 & \ldots & a_{n} & a_{n+1}\\
                a_0 & a_1 & \ldots & a_{n-1} & a_{n}\\
                0   & a_0 & \ldots & a_{n-2} & a_{n-1}\\
                \vdots & \vdots & \vdots & \vdots & \vdots\\
                0 & 0 & \ldots & a_{1} & a_{2}\\
                0 & 0 & \ldots & a_{0} & a_{1}\\
            \end{array}
            \right|
    \end{displaymath}                
    expanding the determinant:
    \begin{displaymath}                
        a_{1}
            \left|
            \begin{array}{ccccc}
                a_1 & a_2 & \ldots & a_{n-1} & a_{n}\\
                a_0 & a_1 & \ldots & a_{n-2} & a_{n-1}\\
                0   & a_0 & \ldots & a_{n-3} & a_{n-2}\\
                \vdots & \vdots & \vdots & \vdots & \vdots\\
                0 & 0 & \ldots & a_{1} & a_{2}\\
                0 & 0 & \ldots & a_{0} & a_{1}\\
            \end{array}
            \right|
        - a_{0}
            \left|
            \begin{array}{ccccc}
                a_2 & a_3 & \ldots & a_{n} & a_{n+1}\\
                a_0 & a_1 & \ldots & a_{n-2} & a_{n-1}\\
                0   & a_0 & \ldots & a_{n-3} & a_{n-2}\\
                \vdots & \vdots & \vdots & \vdots & \vdots\\
                0 & 0 & \ldots & a_{1} & a_{2}\\
                0 & 0 & \ldots & a_{0} & a_{1}\\
            \end{array}
            \right|
    \end{displaymath}                
    therefore:
    \begin{displaymath}                
        a_{n+1}^{\prime} = 
            -\frac{a_{1}}{a_{0}}a_{n}^{\prime}
        - 
            \frac{1}{a_{0}}\frac{(-1)^{n-1}}{a_{0}^{n}}
            \left|
            \begin{array}{ccccc}
                a_2 & a_3 & \ldots & a_{n} & a_{n+1}\\
                a_0 & a_1 & \ldots & a_{n-2} & a_{n-1}\\
                0   & a_0 & \ldots & a_{n-3} & a_{n-2}\\
                \vdots & \vdots & \vdots & \vdots & \vdots\\
                0 & 0 & \ldots & a_{1} & a_{2}\\
                0 & 0 & \ldots & a_{0} & a_{1}\\
            \end{array}
            \right|
    \end{displaymath}                
    expanding the determinant a second time:
    \marginpar{tedious manipulations, skip them at a first reading}
    \begin{displaymath}                
        a_{2}
            \left|
            \begin{array}{ccccc}
                a_1 & a_2 & \ldots & a_{n-2} & a_{n-1}\\
                a_0 & a_1 & \ldots & a_{n-3} & a_{n-2}\\
                \vdots & \vdots & \vdots & \vdots & \vdots\\
                0 & 0 & \ldots & a_{1} & a_{2}\\
                0 & 0 & \ldots & a_{0} & a_{1}\\
            \end{array}
            \right|
        - a_{0}
            \left|
            \begin{array}{ccccc}
                a_3 & a_4 & \ldots & a_{n} & a_{n+1}\\
                a_0 & a_1 & \ldots & a_{n-3} & a_{n-2}\\
                0   & a_0 & \ldots & a_{n-4} & a_{n-3}\\
                \vdots & \vdots & \vdots & \vdots & \vdots\\
                0 & 0 & \ldots & a_{1} & a_{2}\\
                0 & 0 & \ldots & a_{0} & a_{1}\\
            \end{array}
            \right|
    \end{displaymath}                
    therefore:
    \begin{displaymath}                
        a_{n+1}^{\prime} = 
            -\frac{a_{1}}{a_{0}}a_{n}^{\prime}
            -\frac{a_{2}}{a_{0}}a_{n-1}^{\prime}
            -\frac{1}{a_{0}}\frac{(-1)^{n-2}}{a_{0}^{n-1}}
            \left|
            \begin{array}{ccccc}
                a_3 & a_4 & \ldots & a_{n} & a_{n+1}\\
                a_0 & a_1 & \ldots & a_{n-3} & a_{n-2}\\
                0   & a_0 & \ldots & a_{n-4} & a_{n-3}\\
                \vdots & \vdots & \vdots & \vdots & \vdots\\
                0 & 0 & \ldots & a_{1} & a_{2}\\
                0 & 0 & \ldots & a_{0} & a_{1}\\
            \end{array}
            \right|
    \end{displaymath}                
    keep expanding determinants $n$ times yield:
    \begin{displaymath}                
        a_{n+1}^{\prime} = 
            -\frac{a_{1}}{a_{0}}a_{n}^{\prime}
            -\frac{a_{2}}{a_{0}}a_{n-1}^{\prime}
            \ldots
            -\frac{a_{n}}{a_{0}}a_{1}^{\prime}
            -\frac{a_{n+1}}{a_{0}}a_{0}^{\prime}
    \end{displaymath}                
    which is the same to say:
    \begin{displaymath}                
        a_{0}a_{n+1}^{\prime}  
            +a_{1}a_{n}^{\prime}
            +a_{2}a_{n-1}^{\prime}
            \ldots
            +a_{n}a_{1}^{\prime}
            +a_{n+1}a_{0}^{\prime}
            = 0
    \end{displaymath}                
    which is the relation we get from $\big[t^{n+1}\big]A(t)A^{\prime}(t)=0$
    so the induction step holds and the argument is proved as required.

\end{itemize}

\end{proof}

Previous exercise is actually a lemma for an identity which
\citeauthor{eplett:1979} finds and discusses in \cite{eplett:1979}.  Let
$\lbrace d_{nk}\rbrace_{n,k\in\mathbb{N}}$ be the Catalan triangle as defined
in \autoref{eq:shapiro:catalan:triangle:expanded}, so \citeauthor{eplett:1979}'s
result reads as follow.

\begin{theorem}
    \marginpar{\ldots this combination can be also expressed using a $A$-matrix, as
        we will see later}
    \begin{displaymath}                
        c_{j-1}=\sum_{k=1}^{j}{(-1)^{k-1}d_{jk}}
    \end{displaymath}                
    In other words, coefficient $c_{j-1}$, the first one lying on row $j-1$,
    combines coefficients lying on row $j$, summing with alternating signs.
    \label{thm:eplett:identity}
\end{theorem}
\begin{proof}
    Let $\vect{a}=\lbrace a_{n}\rbrace_{n\in\mathbb{N}}$ be a sequence such
    that $a_{0}=1$ and consider its \emph{inverse} sequence, call it $\vect{b}$,
    where $b_{0}=1$ by inverse relation on the very first coefficient.
    By previous exercise observe that:
    \begin{equation}                
        \hspace{-2cm}
        a_{n} = b_{n}+a_{1}b_{n-1}+a_{2}b_{n-2}+\ldots+a_{n-1}b_{1}
        \leftrightarrow
        b_{n} = (-1)^{n-1}
            \left|
            \begin{array}{ccccc}
                a_1 & a_2 & \ldots & a_{n-1} & a_{n}\\
                1 & a_1 & \ldots & a_{n-2} & a_{n-1}\\
                0   & 1 & \ldots & a_{n-3} & a_{n-2}\\
                \vdots & \vdots & \vdots & \vdots & \vdots\\
                0 & 0 & \ldots & a_{1} & a_{2}\\
                0 & 0 & \ldots & 1 & a_{1}\\
            \end{array}
            \right|
        \label{eq:eplett:det:inverse:identity:lemma}
    \end{equation}                
    As we have seen in
    \autoref{eq:shapiro:catalan:triangle:convolution:for:generic:coefficient}, 
    we can rewrite theorem's \ac{rhs}, calling the new term $b_{j}$:
    \begin{displaymath}                
        c_{j-1}=\sum_{k=1}^{j}{(-1)^{k-1}d_{jk}}
            = \sum_{k=1}^{j}{(-1)^{k-1}\sum_{i_{1}+\ldots+i_{k}=j}{c_{i_{1}}\cdots c_{i_{k}}}}=b_{j}
    \end{displaymath}                
    % TODO in order to be truly precise, we should prove that the term abstracted out by `b_{j}'
    % actually equals the determinant of Catalan matrix.
    Applying \autoref{eq:eplett:det:inverse:identity:lemma} \emph{from right to left}, 
    it is necessary to find a sequence $\vect{a}$ such that:
    \begin{displaymath}                
        a_{n} = c_{n-1}+a_{1}c_{n-2}+a_{2}c_{n-3}+\ldots+a_{n-1}c_{0}
    \end{displaymath}                
    according to \citeauthor{feller:intro:combinatorial:analysis}, page $94$ of
    \cite{feller:intro:combinatorial:analysis}, the following combination over
    Catalan numbers holds, for $n>0$:
    \begin{displaymath}                
        c_{n}=\sum_{k=1}^{n}{c_{k-1}c_{n-k}}\quad\wedge\quad c_{0}=1
    \end{displaymath}                
    therefore setting $a_{n}=c_{n}$ sequence $\vect{a}$ is found and theorem
    is proved.

\end{proof}
\quad
\\\\
The above proof shows another interesting identity which relates Catalan
numbers with matrix determinants\marginpar{yet another characterization for
coefficient $c_{j}$ using matrix theory}:
\begin{displaymath}                
    c_{n-1} = (-1)^{n-1}
        \left|
        \begin{array}{ccccc}
            c_1 & c_2 & \ldots & c_{n-1} & c_{n}\\
            1 & c_1 & \ldots & c_{n-2} & c_{n-1}\\
            0   & 1 & \ldots & c_{n-3} & c_{n-2}\\
            \vdots & \vdots & \vdots & \vdots & \vdots\\
            0 & 0 & \ldots & c_{1} & c_{2}\\
            0 & 0 & \ldots & 1 & c_{1}\\
        \end{array}
        \right|
\end{displaymath}                

Taking apart the sum in theorem's \ac{rhs}, we can restate it as:
\begin{displaymath}                
    c_{j-1} + \sum_{{k \text{ is even}}}{d_{jk}}
        = \sum_{{k \text{ is odd}}}{d_{jk}}\qquad\text{where } k\in\mathbb{N}\setminus \lbrace0\rbrace
\end{displaymath}
Since $d_{j0}$ has no meaning by definition of the triangle, we can
attach to it a combinatorial meaning: let $d_{j0}$ denotes the
number of pairs of paths that intersect for the first time at
step $j$ and possibly after again. Therefore $d_{j0}=c_{j-1}$, so 
we can rewrite one more time:
\begin{displaymath}                
    \sum_{{k \text{ is even}}}{d_{jk}}
        = \sum_{{k \text{ is odd}}}{d_{jk}} \qquad\text{where } k\in\mathbb{N}
\end{displaymath}
This last version supplies a combinatorial 
interpretation in terms of \emph{non-intersecting} pairs of paths:
\begin{quote}
    \marginpar{a combinatorial interpretation of \citeauthor{eplett:1979}'s result}
    consider the set $S$ of pairs of paths with distance $j$. Then,
    the subset of $S$ composed of \emph{non-intersecting} pairs of paths 
    with \emph{even} distance augmented by the subset of $S$ of 
    pairs of paths that \emph{intersect} for the first time at step $j$
    equals, in number, the subset of $S$ is composed of pairs of paths 
    with \emph{odd} distance
\end{quote}
\begin{proof}
    Consider a pair of paths $(\vect{\rho},\vect{\sigma})$ of length $j$ with
    \emph{even} distance $k$.  If a pair of steps, such that they \emph{do not
    point to the same direction}, is applied to $(\vect{\rho},\vect{\sigma})$
    then a new pair of paths $(\vect{\rho}^{\prime},\vect{\sigma}^{\prime})$ of
    length $j+1$ with \emph{odd} distance $k+1$ is built. This production is a
    bijection, therefore the number of pairs of paths with \emph{even} distance
    equal the one with \emph{odd} distance, under the constraint that paths
    have the same length them all, as required.
\end{proof}
\quad\\\\
Although identity stated in \autoref{thm:eplett:identity} can be consider
\marginpar{\autoref{thm:eplett:identity} can be applied to Renewal arrays too}
useless for our work, it is precious for two reasons: first, it characterize
Catalan numbers using matrix theory and, second, allows
\citeauthor{rogers:eplett:identity} to find further results about the theory of
\emph{Renewal arrays}, as described in \cite{rogers:eplett:identity}.


















