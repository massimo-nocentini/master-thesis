
\subsection{The Riordan group}

In \cite{shapiro:1991}, \citeauthor{shapiro:1991} introduces a
generalization of the concepts developed so far: a group called 
\emph{Riordan group}, in honor of professor John Riordan
\marginpar{Thank you John Riordan, $1903-1988$}.
This group has been defined in order to unify many themes in 
enumeration and combinatoric problems; to solve binomial, 
inverse identities and, more recently, combinatorial
sums and binary words counting. 
\\\\
Let $\mathcal{M}=\lbrace m_{nk}\rbrace_{n,k\in\mathbb{N}}$ be an \emph{infinite
lower triangular matrix}, namely:
\begin{displaymath}
     m_{nk}\in\mathbb{Z} \quad\wedge\quad n < k \rightarrow m_{nk} = 0
\end{displaymath}
Call $\mathcal{M}_{(t)}$ the vector produced by the following product, 
where $t$ is an \emph{dummy variable}\marginpar{$\mathcal{M}$ is a matrix of integers, 
$\mathcal{M}_{(t)}$ is a vector of \ac{fps} in $\mathbb{Z}[\![t]\!]$}:
\begin{displaymath}
    \left[
        \begin{array}{cccccc}
            1 & t & t^{2} & t^{3} & t^{4} &\ldots
        \end{array}
    \right]
    \left[
        \begin{array}{cccccc}
            m_{00} & & & &  &\\
            m_{10} & m_{11} & & &  &\\
            m_{20} & m_{21}& m_{22}& &  &\\
            m_{30} & m_{31}& m_{32}& m_{33}&  &\\
            m_{40} & m_{41}& m_{42}& m_{43}& m_{44} &\\
            \vdots & \vdots& \vdots& \vdots& \vdots & \ddots\\
        \end{array}
    \right]
\end{displaymath}
therefore $\mathcal{M}_{(t)} =
    \left[
        \begin{array}{cccccc}
            m_{0}(t) & m_{1}(t) & m_{2}(t) & m_{3}(t) &m_{4}(t) & \ldots
        \end{array}
    \right]$ 
with function $m_{j}$, for $j\in\mathbb{N}$, is a \ac{fps} in the ring 
$\mathbb{Z}[\![t]\!]$. If there exists two analytic functions $d$
and $h$, such that $d(0)\neq0$ and $h(0)=0 \wedge h^{\prime}(0)\neq0$, which
satisfy \marginpar{matrix $\mathcal{M}$ will be denoted by the pair $(d(t),h(t))$}:
\begin{displaymath}
    m_{j}(t)=d(t)\,h(t)^{j} \quad \forall j\in\mathbb{N}
\end{displaymath}
then matrix $\mathcal{M}$ is called a \emph{Riordan array}. Such an array is
directly related to coefficients matrix $\lbrace m_{nk}\rbrace_{n,k\in\mathbb{N}}$ as follows:
\begin{displaymath}
    [t^{n}]m_{j}(t)=m_{nj}
\end{displaymath}


\begin{theorem}[Fundamental theorem]
    Let $\mathcal{M}=(d(t),h(t))$ be a Riordan matrix and $\vect{b}=
    \left[\begin{array}{cccc}b_{0}&b_{1}&b_{2}&\ldots\end{array}\right]^{T}$ 
    be an infinite column vector with $b_{j}\in\mathbb{Z}$, for $j\in\mathbb{N}$. Then:
    \begin{displaymath}
        \mathcal{M}_{(t)}\cdot\vect{b}=d(t)b(h(t))
    \end{displaymath}
    where function $b$ is a \ac{fps} over coefficients in $\vect{b}$.
    \label{thm:riordan:group:fundamental:theorem}
\end{theorem}
\begin{proof}
    By definition of vector $\mathcal{M}_{(t)}$:
    \begin{displaymath}
        \mathcal{M}_{(t)}\cdot\vect{b}= b_{0}\,m_{0}(t) + b_{1}\,m_{1}(t) + b_{2}\,m_{2}(t) 
            + b_{3}\,m_{3}(t) + b_{4}\,m_{4}(t) + \ldots
    \end{displaymath}
    by definition of function $m_{j}$, for any $j\in\mathbb{N}$:
    \begin{displaymath}
        = b_{0}\,d(t)h(t)^{0} + b_{1}\,d(t)h(t) + b_{2}\,d(t)h(t)^{2} 
            + b_{3}\,d(t)h(t)^{3} + b_{4}\,d(t)h(t)^{4} + \ldots
    \end{displaymath}
    factor out function $d$:
    \begin{displaymath}
        = d(t)\left(b_{0}\,h(t)^{0} + b_{1}\,h(t) + b_{2}\,h(t)^{2} 
            + b_{3}\,h(t)^{3} + b_{4}\,h(t)^{4} + \ldots\right)
    \end{displaymath}
    finally, rewrite composing function $h$ with function $b$:
    \begin{displaymath}
        = d(t)b(h(t))  
    \end{displaymath}
    as required.

\end{proof}

In the introductory paragraph we said that the current generalization
we are studying is a \emph{group}. More formally, we are going to define a 
\emph{group} over a \emph{set of Riordan matrices}, 
denoted by $(\lbrace \mathcal{M}_{i}\rbrace_{i\in\mathbb{N}},\cdot)$, 
where $\cdot$ is the \emph{group
operator}: additionally it is required to show the \emph{identity element} and,
finally, to prove that there exists an \emph{inverse} for every element in the group.
\\\\
Let $\mathcal{M}=\left(d_{\mathcal{M}}(t),h_{\mathcal{M}}(t)\right)$ and
$\mathcal{N}=\left(d_{\mathcal{N}}(t),h_{\mathcal{N}}(t)\right)$ be two Riordan matrices.
Define the \emph{group operator} as their product
\marginpar{matrix multiplication as group operator}, formally:
\begin{displaymath}
    \left(d_{\mathcal{M}}(t),h_{\mathcal{M}}(t)\right)\cdot
        \left(d_{\mathcal{N}}(t),h_{\mathcal{N}}(t)\right) = 
        \left(d_{\mathcal{M}}(t)d_{\mathcal{N}}(h_{\mathcal{M}}(t)),
            h_{\mathcal{N}}(h_{\mathcal{M}}(t))\right)
\end{displaymath}
\begin{proof}
    Consider a column $j$ of matrix $\mathcal{N}$,
    for some $j\in\mathbb{N}$. Denoting by $\vect{e}_{j}$ the $j$-th versor and
    by $n_{j}(t)=d_{\mathcal{N}}(t)h_{\mathcal{N}}(t)^{j}$ the \ac{fps} over such column, 
    by \autoref{thm:riordan:group:fundamental:theorem} the following holds:
    \begin{displaymath}
        \mathcal{M}_{(t)}\cdot\left(\mathcal{N}\cdot\vect{e}_{j}\right)
            = d_{\mathcal{M}}(t) d_{\mathcal{N}}(h_{\mathcal{M}}(t))
                h_{\mathcal{N}}(h_{\mathcal{M}}(t))^{j}
    \end{displaymath}
    Let $d_{\mathcal{M}\mathcal{N}}$ be a function such that 
    $d_{\mathcal{M}\mathcal{N}}(t)=d_{\mathcal{M}}(t) d_{\mathcal{N}}( h_{\mathcal{M}}(t))$,
    and let $h_{\mathcal{M}\mathcal{N}}$ be a function such that 
    $h_{\mathcal{M}\mathcal{N}}(t)=h_{\mathcal{N}}(h_{\mathcal{M}}(t))$, then the \ac{rhs}
    has the shape
    $d_{\mathcal{M}\mathcal{N}}(t)h_{\mathcal{M}\mathcal{N}}(t)^{j}$, which denotes the
    Riordan matrix $\mathcal{M}\mathcal{N}$, as required.

\end{proof}
\quad
\\\\
The \marginpar{$(1,t)$ is the identity element} identity element of the group
is the Riordan matrix $\mathcal{I}=(1,t)$.
\begin{proof}
    Let $\mathcal{M}=\left(d_{\mathcal{M}}(t),h_{\mathcal{M}}(t)\right)$ be a Riordan matrix.
    By definition of group operator:
    \begin{displaymath}
        \left(d_{\mathcal{M}}(t),h_{\mathcal{M}}(t)\right)\cdot \left(1,t\right) = 
            \left(d_{\mathcal{M}}(t), h_{\mathcal{M}}(t)\right)
    \end{displaymath}
    since group operator $\cdot$ is no commutative in general, check the following also:
    \begin{displaymath}
        \left(1,t\right) \cdot \left(d_{\mathcal{M}}(t),h_{\mathcal{M}}(t)\right)= 
            \left(d_{\mathcal{M}}(t), h_{\mathcal{M}}(t)\right)
    \end{displaymath}
    as required.
\end{proof}
\quad
\\\\
In order to find the \emph{inverse} of a Riordan matrix, we have to
introduce the concept of \marginpar{compositional inverse of a function}
\emph{compositional inverse} of an analytic function $h$.
Let $h$ be a function such that $h(0)=0 \wedge h^{\prime}(0)\neq0$, then
$\hat{h}$ is the \emph{compositional inverse} of function $h$ if and only if 
$\hat{h}(h(t))=h(\hat{h}(t))=t$. 
\\\\
Let $\mathcal{M}=\left(d_{\mathcal{M}}(t),h_{\mathcal{M}}(t)\right)$ be a Riordan matrix.
The inverse matrix $\mathcal{M}^{-1}$ of $\mathcal{M}$ satisfies the following relation:
\begin{displaymath}
    \mathcal{M}^{-1}=\left(
        \frac{1}{d_{\mathcal{M}}(\hat{h}_{\mathcal{M}}(t))}, \hat{h}_{\mathcal{M}}(t)
    \right)
\end{displaymath}

\begin{proof}
    By definition of inverse element\marginpar{inverse of an element in 
    $\left(\lbrace \mathcal{M}_{i}\rbrace,\cdot\right)$}, 
    we have to find a matrix $\mathcal{M}^{-1}$ which satisfies the following relation:
    \begin{displaymath}
        \mathcal{M}\cdot\mathcal{M}^{-1}
            =\left(d_{\mathcal{M}}(t)d_{\mathcal{M}^{-1}}(h_{\mathcal{M}}(t)),
                    h_{\mathcal{M}^{-1}}(h_{\mathcal{M}}(t))\right)
            =\left(1,t\right)
    \end{displaymath}

    Looking at the first component we get:
    \begin{displaymath}
        d_{\mathcal{M}}(t)d_{\mathcal{M}^{-1}}(h_{\mathcal{M}}(t))=1    
    \end{displaymath}
    which can be rewritten as\marginpar{abstracting function $g$ in the
        composition $f(g(t))$ yield $\left.\left[f(y)\right|y=g(t)\right]$,
        a trick also called ``changing variable''}:
    \begin{displaymath}
        d_{\mathcal{M}^{-1}}(h_{\mathcal{M}}(t))=\frac{1}{d_{\mathcal{M}}(t)}    
            = \left.\left[
                d_{\mathcal{M}^{-1}}(y)=\frac{1}{d_{\mathcal{M}}(\hat{h}_{\mathcal{M}}(y))}
                    \right|y=h_{\mathcal{M}}(t)
                \right]
    \end{displaymath}

    On the other hand, looking at the second component we get:
    \begin{displaymath}
        h_{\mathcal{M}^{-1}}(h_{\mathcal{M}}(t))=t
    \end{displaymath}
    therefore $h_{\mathcal{M}^{-1}}=\hat{h}_{\mathcal{M}}$, as required.
\end{proof}
\quad
\\\\
It \marginpar{subgroups of the Riordan group}
is interesting to dig a little on the \emph{Riordan group}, looking at
common patterns that arises in Riordan matrices definitions. Denoting with
$\cdot$ the group operator defined as before, we report
these patterns by introducing the following \emph{subgroups}:
\begin{itemize}
    \item $\left(\lbrace \mathcal{M}_{i}=(d_{\mathcal{M}_{i}}(t),t)
        \rbrace_{i\in\mathbb{N}},\cdot\right)$, the \emph{Appell subgroup};
    \item $\left(\lbrace \mathcal{M}_{i}=(1,h_{\mathcal{M}_{i}}(t))
        \rbrace_{i\in\mathbb{N}},\cdot\right)$, the \emph{Associated subgroup};
    \item $\left(\lbrace \mathcal{M}_{i}=(d_{\mathcal{M}_{i}}(t),td_{\mathcal{M}_{i}}(t))
        \rbrace_{i\in\mathbb{N}},\cdot\right)$, the \emph{Renewal subgroup};
    \item $\left(\lbrace \mathcal{M}_{i}=(d_{\mathcal{M}_{i}}(t),h_{\mathcal{M}_{i}}(t))
        \rbrace_{i\in\mathbb{N}},\cdot\right)$, where function $d$ is \emph{even} and 
        function $h$ is \emph{odd}, the \emph{Checkerboard subgroup};
    \item $\left(\left\lbrace \mathcal{M}_{i}=\left(\frac{t\,h_{\mathcal{M}_{i}}^{\prime}(t)}
            {h_{\mathcal{M}_{i}}(t)},h_{\mathcal{M}_{i}}(t)\right)
        \right\rbrace_{i\in\mathbb{N}},\cdot\right)$, the \emph{Hitting-time subgroup};
\end{itemize}
\quad
\\\\
Finally, \marginpar{a characterization using a bivariate function}
we would like to discuss another interesting definition which characterize
a Riordan array $\mathcal{M}=(d_{\mathcal{M}}(t),h_{\mathcal{M}}(t))$. 
The main point is to associate a bivariate function $m$ 
to the infinite lower triangular matrix $\lbrace m_{nk}\rbrace_{n,k\in\mathbb{N}}$
of coefficients, used to build $\mathcal{M}$. Function $m$ is defined as follows:
\begin{displaymath}
    \begin{split}
        m(t,u) &= \sum_{k\in\mathbb{N}}{\sum_{n\geq k}{m_{nk} t^{n} u^{k}}}
            = \sum_{k\in\mathbb{N}}{u^{k}\sum_{n\geq k}{m_{nk} t^{n}}}
            = \sum_{k\in\mathbb{N}}{m_{k}(t)u^{k}}\\
            &= \sum_{k\in\mathbb{N}}{d_{\mathcal{M}}(t)\,h_{\mathcal{M}}(t)^{k}\,u^{k}}
            = \frac{d_{\mathcal{M}}(t)}{1-h_{\mathcal{M}}(t)\,u}
    \end{split}
\end{displaymath}
This characterization is more general and allows us to have \ac{fps}
$m_{j}^{(t)}(u)$ for a column $j$, if it is expanded respect to variable $t$;
and, by symmetry, to have \ac{fps} $m_{l}^{(u)}(t)$ for a row $l$, if it is
expanded respect to variable $u$.









