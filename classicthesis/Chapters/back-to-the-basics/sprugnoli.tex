

\subsection{Applications to combinatorial sums}

\citeauthor{sprugnoli:1991}, in \cite{sprugnoli:1991}, shows how the
\emph{Riordan group} and its elements \emph{Riordan arrays} provide an elegant
methodology to solve combinatorial sums. This method is quite different from
ones proposed in \cite{egorychev:1990}, \cite{wilf:zeilberger:1990} and
\cite{wilf:1989}, since it offer a \emph{constructive} proof for a sum identity
$\sum_{k}f_{k}=v$: \marginpar{not only a $certification$ for $\sum_{k}f_{k}=v$}
in addition to be a \emph{certification} for the identity, it shows how to find
such value $v$. 

The following result is another way to state
\autoref{thm:riordan:group:fundamental:theorem} and it allows us to show in the
proof an interesting trick about coefficient extractor $[t^{n}]$. 

\begin{theorem}
    Let $\mathcal{M}=(d(t),h(t))$ be a Riordan matrix and $\vect{\omega}$ 
    be a \emph{given} sequence $\lbrace\omega_{i}\rbrace_{i\in\mathbb{N}}$. Choose
    a row index $n\in\mathbb{N}$, then:
    \begin{displaymath}
        \sum_{k=0}^{n}{m_{nk}\,\omega_{k}}=[t^{n}]d(t)\Omega(h(t))
    \end{displaymath}
    where function $\Omega$ is a \ac{gf} of sequence $\vect{\omega}$.
    \label{thm:sprugnoli:riordan:combinatorial:sums}
\end{theorem}

\begin{proof}
    By relation among Riordan array $\mathcal{M}$ and matrix 
    $\lbrace m_{nk}\rbrace_{n,k\in\mathbb{N}}$:
    \begin{displaymath}
        \sum_{k=0}^{n}{m_{nk}\omega_{k}}
            =\sum_{k=0}^{n}{[t^{n}]m_{k}(t)\,\omega_{k}}
            =d(t)\sum_{k=0}^{n}{[t^{n}]h(t)^{k}\,\omega_{k}}
    \end{displaymath}
    since summing $n+1$ coefficients extracted from $n+1$
    \marginpar{a trick on $[t^{n}]$ within a sum: moving it from inside to outside}
    convolutions of function $h$ with itself, is the same as
    do $n+1$ convolutions of function $h$ with itself, sum them
    and, lately, do \emph{one} coefficient extraction. Therefore:
    \begin{displaymath}
        =[t^{n}]d(t)\sum_{k=0}^{n}{\omega_{k}\,h(t)^{k}}
        =[t^{n}]d(t)\Omega(h(t))
    \end{displaymath}
    as required.
\end{proof}

For the sake of clarity, we would like to apply previous result to
Pascal array $\mathcal{P}$ in order to find diagonal sums. Define
a new array $\hat{\mathcal{P}}$ as:
\begin{equation}
    \hat{\mathcal{P}}=\left(\frac{1}{1-t},\frac{t^{2}}{1-t}\right)
    \label{eq:pascal:array:for:diagonals}
\end{equation}
in order to have coefficients lying on diagonals in \autoref{eq:pascal:array:derived:for:example}
to lie on rows. The following is the matrix denoted by $\hat{\mathcal{P}}$:
\marginpar{an example: computing diagonal sums of array $\mathcal{P}$}
\begin{displaymath} 
    \left[
    \begin{array}{rrrrr}
    1 &  &  &  &   \\
    1 &  &  &  &   \\
    1 & 1 &  &  &   \\
    1 & 2 &  &  &   \\
    1 & 3 & 1 &  &   \\
    1 & 4 & 3 &  &   \\
    1 & 5 & 6 & 1 &   \\
    1 & 6 & 10 & 4 &  \\
    1 & 7 & 15 & 10 & 1 \\
    1 & 8 & 21 & 20 & 5 
    \end{array}
    \right] 
\end{displaymath}
where $d_{nk}\in\mathcal{P} \rightarrow d_{n-k,k}\in\hat{\mathcal{P}}$, 
therefore array $\hat{\mathcal{P}}$ is defined over matrix 
$\left\lbrace{{n-k}\choose{k}}\right\rbrace_{n,k\in\mathbb{N}}$.
Consider sequence $\vect{1}=\lbrace1\rbrace_{n\in\mathbb{N}}$, 
\autoref{thm:sprugnoli:riordan:combinatorial:sums} allows us to write:
\marginpar{sum of the $n$-th $\mathcal{P}$'s diagonal yield the $n$-th Fibonacci number}
\begin{displaymath}
    \sum_{k=0}^{n}{{{n-k}\choose{k}}}=[t^{n}]
        \frac{d_{\hat{\mathcal{P}}}(t)}{1-h_{\hat{\mathcal{P}}}(t)}
        =[t^{n}]\frac{1}{1-t-t^{2}}
        =F_{n}
\end{displaymath}
where $F_{n}$ is the $n$-th \emph{Fibonacci number}.
\\\\
It is very interesting the following result that allows us to solve
yet another combinatorial sum using both a Riordan array $\mathcal{M}$
and its inverse.

\begin{theorem}
    Let $\mathcal{M}=(d(t),h(t))$ be a Riordan array and $\vect{\omega}$ 
    be a sequence $\lbrace\omega_{i}\rbrace_{i\in\mathbb{N}}$ with an \emph{unknown} 
    \ac{gf} $\Omega$. Choose a row index $n\in\mathbb{N}$, if $\vect{\theta}$ is a  
    sequence $\lbrace\theta_{i}\rbrace_{i\in\mathbb{N}}$ with a \emph{given} \ac{gf} $\Theta$, 
    then:
    \begin{displaymath}
        \sum_{k=0}^{n}{\tilde{m}_{nk}\,\theta_{k}}=\omega_{n}
        \quad\rightarrow\quad 
        \sum_{k=0}^{n}{m_{nk}\,\omega_{k}}=\theta_{n}
    \end{displaymath}
    where $\tilde{m}_{nk}\in\mathcal{M}^{-1}$. Moreover, 
    $\Omega(t)=\frac{1}{d(\hat{h}(t))}\,\Theta(\hat{h}(t))$,
    where function $\hat{h}$ is the compositional inverse of 
    function $h$.
    \label{thm:sprugnoli:sequences:inversion:by:combinatorial:sums}
\end{theorem}

\begin{proof}
    By \autoref{thm:sprugnoli:riordan:combinatorial:sums} we can rewrite:
    \begin{displaymath}
        \sum_{k=0}^{n}{\tilde{m}_{nk}\,\theta_{k}}
            = [t^{n}]\frac{1}{d(\hat{h}(t))}\,\Theta(\hat{h}(t)) 
            = [t^{n}]\Omega(t) = \omega_{n}
    \end{displaymath}
    which sets unknown function $\Omega$, namely the \ac{gf} for 
    sequence $\vect{\omega}$, such that $\Omega(t)=\frac{1}{d(\hat{h}(t))}\,\Theta(\hat{h}(t))$. 
    Then:
    \begin{displaymath}
        \sum_{k=0}^{n}{m_{nk}\,\omega_{k}}
            = [u^{n}]d(u)\Omega(h(u))
            = [u^{n}]\Theta(u)
            = \theta_{n}
    \end{displaymath}
    as required.
\end{proof}

Why is \autoref{thm:sprugnoli:sequences:inversion:by:combinatorial:sums} important?
\marginpar{another way to find \ac{gf} for a row in a Riordan array}
Assume that you have a Riordan array $\mathcal{M}=(d(t),h(t))$ over a matrix of coefficients
$\lbrace m_{nk}\rbrace_{n,k\in\mathbb{N}}$. You are interested to find the
\ac{gf} $\Omega$ of a row $r\in\mathbb{N}$ over the sequence:
\begin{displaymath}
    \vect{\omega} 
        =(w_{0},w_{1},w_{2},\ldots,w_{r},0,\ldots)
        =(m_{r0},m_{r1},m_{r2},\ldots,m_{rr},0,\ldots)
\end{displaymath}
when you know the \ac{gf} $\Theta$ over a sequence $\vect{\theta}$ such that:
\begin{displaymath}
    \sum_{k=0}^{n}{m_{nk}\,\omega_{k}}
        =\sum_{k=0}^{n}{m_{nk}\,m_{rk}}
        =\theta_{n}
\end{displaymath}
applying \autoref{thm:sprugnoli:sequences:inversion:by:combinatorial:sums},
it is necessary to find $\mathcal{M}^{-1}$ and compute function $\Omega$ as:
\begin{displaymath}
    \Omega(t)=\frac{1}{d(\hat{h}(t))}\,\Theta(\hat{h}(t))
\end{displaymath}
such function $\Omega$ is the desired one, namely the \ac{gf} over
sequence $\vect{\omega}$, itself over coefficients lying on row $r$, as required.
