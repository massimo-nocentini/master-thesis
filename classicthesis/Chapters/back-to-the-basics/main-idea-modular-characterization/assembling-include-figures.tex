
\begin{figure}[p]

    \noindent\makebox[\textwidth]{
        \centering
        %\includegraphics[width=0.8\textwidth]{../../sympy/catalan/coloured.pdf}

        % using *angle* property to rotate it is difficult to properly align it
        % in order to have a "real" matrix representation.
        \includegraphics[width=9cm, height=9cm, keepaspectratio=true]
            {../RART2015/pascal-tikz/main-idea-four-splitted/plain-numbers.pdf}

        \includegraphics[width=9cm, height=9cm, keepaspectratio=true]
            {../RART2015/pascal-tikz/main-idea-four-splitted/centered-numbers.pdf}

    }

    \vskip1cm

    \noindent\makebox[\textwidth]{
        \centering
        %\includegraphics[width=0.8\textwidth]{../../sympy/catalan/coloured.pdf}

        % using *angle* property to rotate it is difficult to properly align it
        % in order to have a "real" matrix representation.
        \includegraphics[width=9cm, height=9cm, keepaspectratio=true]
            {../RART2015/pascal-tikz/main-idea-four-splitted/remainders.pdf}

        \includegraphics[width=9cm, height=9cm, keepaspectratio=true]
            {../RART2015/pascal-tikz/main-idea-four-splitted/dots.pdf}
    }

    % this 'particular' line is necessary to use `displaymath' environment
    % into the caption environment, togheter with the inclusion of 
    % `caption' package. See here for more explanation:
    % http://stackoverflow.com/questions/2716227/adding-an-equation-or-formula-to-a-figure-caption-in-latex
    \captionsetup{singlelinecheck=off}
    \caption[$\mathcal{P}_{\equiv_{2}}^{(3)}$: from plain matrix to coloured triangle]{
        \emph{Top left} corner: chunk of raw expansion of Pascal array $\mathcal{P}$

        \emph{Top right} corner: the same chunk but using a centered layout, a-l\`a Sierpinski gasket

        \emph{Bottom left} corner: apply modular transformation on the same chunk of coefficient, where $p=2$.
            Associate colour \textcolor{blue}{blue} to $[0]_{2}$ 
            remainder class and colour \textcolor{orange}{orange} to $[1]_{2}$ remainder class

        \emph{Bottom right} corner: abstract each coefficient $m_{nk}$ with a coloured dot 
            according to $m_{nk}$'s remainder class membership}

    \label{fig:main:idea:modular:characterization}

\end{figure}
