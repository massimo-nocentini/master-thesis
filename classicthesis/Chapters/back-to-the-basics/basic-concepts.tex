
In this section we recall some fundamental concepts that are essential building
blocks for the treatment that follows. First of all we denote a \emph{countable}
sequence $\vect{a}$ of integers with:
\begin{displaymath}
    \vect{a}=\lbrace a_{n} \rbrace_{n\in\mathbb{N}}
\end{displaymath}
where term $a_{n}$ is called the \emph{generic coefficient} of
sequence $\vect{a}$, which can be either simply a costant $a_{n}\in\mathbb{Z}$
or a function in the variable $n$, denoting a rule for building each
coefficient belonging to sequence $\vect{a}$.

When a sequence depends on \emph{two} indices, we denote it with:
\begin{displaymath}
    \lbrace a_{nk} \rbrace_{n,k\in\mathbb{N}}
\end{displaymath}
which is usually called \emph{matrix}.

We will be interested to handle coefficients in a sequence $\vect{\omega}$,
using them to build \ac{fps} over the ring $\mathbb{Z}[\![t]\!]$, as follows:
\begin{equation}
    \sum_{n\geq0}{\omega_{n}\,t^{n}}
    \label{eq:fps:formally}
\end{equation}
where $t$ is an undeterminate variable, it usually does \emph{not} take
any value in order to \emph{evaluate} the sum at a given point. 
Moreover, such \ac{fps} is the Taylor expansion of the \ac{gf}
$\Omega$ for sequence $\vect{\omega}$, formally:
\begin{equation}
    \sum_{n\geq0}{\omega_{n}\,t^{n}} = \sum_{i\geq0}{\frac{1}{i!}\frac{
        \partial^{i}\Omega(t)}{\partial\,t^{i}}t^{i}}
    \label{eq:fps:gf:relation}
\end{equation}
where function $\Omega$ satisfies:
\begin{displaymath}
    \Omega(t) = \mathcal{G}\left(\lbrace\omega_{n}\rbrace_{n\in\mathbb{N}}\right)
\end{displaymath}
denoting with $\mathcal{G}$ the operator which consumes a sequence $\vect{\omega}$
and produces the function $\Omega$ satisfying \autoref{eq:fps:gf:relation}.

We don't care about questions of convergence in infinite series, such as in 
\autoref{eq:fps:formally}, since a systematic theory of \ac{fps} has been proposed
by \citeauthor{niven:1969} in \cite{niven:1969} from the theoretical point of view,
and by \citeauthor{riordan:1964}, in \cite{riordan:1964}, for the combinatorial aspect.
Those theories do not involve any question of convergence or divergence, characterizing
solutions of functional equations, how coefficients attached to the same power of an underterminate
variable can be equated, how \ac{fps} can be multiplied or derived and so on.
