
In this section we recall some fundamental concepts that are essential building
blocks for the treatment that follows. First of all we denote a sequence
$\vect{\omega}$ of \emph{countable} integers with:
\marginpar{sequences and matrices}
\begin{displaymath}
    \vect{\omega}=\lbrace \omega_{n} \rbrace_{n\in\mathbb{N}}
\end{displaymath}
where term $\omega_{n}$ is called the \emph{generic coefficient} of
sequence $\vect{\omega}$, which can be either simply a costant $\omega_{n}\in\mathbb{Z}$
or a function in the variable $n$, denoting a rule for building each
coefficient belonging to sequence $\vect{\omega}$.

When a sequence depends on \emph{two} indices, we denote it with:
\begin{displaymath}
    \lbrace \omega_{nk} \rbrace_{n,k\in\mathbb{N}}
\end{displaymath}
which is usually called an \emph{infinite matrix}.

We will be interested to handle coefficients in a sequence $\vect{\omega}$,
using them to build \ac{fps} over the ring $\mathbb{Z}[\![t]\!]$, as follows:
\begin{equation}
    \sum_{n\geq0}{\omega_{n}\,t^{n}}
    \label{eq:fps:formally}
\end{equation}
where $t$ is an undeterminate variable, it usually does \emph{not} take
any value in order to \emph{evaluate} the sum at a given point. 
Moreover, such \ac{fps} is the Taylor expansion of the \ac{gf}
$\Omega$ for sequence $\vect{\omega}$, formally:
\marginpar{\ac{fps} $\Omega$ over a sequence $\vect{\omega}$}
\begin{equation}
    \sum_{n\geq0}{\omega_{n}\,t^{n}} = \sum_{i\geq0}{\frac{1}{i!}\frac{
        \partial^{i}\Omega(0)}{\partial\,t^{i}}t^{i}}
    \label{eq:fps:gf:relation}
\end{equation}
where function $\Omega$ satisfies:
\begin{displaymath}
    \Omega(t) = \mathcal{G}_{t}\left(\lbrace\omega_{n}\rbrace_{n\in\mathbb{N}}\right)
\end{displaymath}
denoting with $\mathcal{G}_{t}$ the operator which consumes a sequence
$\vect{\omega}$ and produces the function $\Omega$, in the undeterminate
variable $t$, satisfying \autoref{eq:fps:gf:relation}. From now on, we
omit the subscript $_{n\in\mathbb{N}}$ when ambiguities do not arise 
in the context where it should appear.

We don't care about questions of convergence in infinite series, such as in
\marginpar{\ac{fps} convergence doesn't matter}
\autoref{eq:fps:formally}, since a systematic theory of \ac{fps} has been
proposed by \citeauthor{niven:1969} in \cite{niven:1969} from the theoretical
point of view, and by \citeauthor{riordan:1964}, in \cite{riordan:1964}, for
the combinatorial aspect.  Those theories do not involve any question of
convergence or divergence; on the other hand they provide some
characterizations of solutions of functional equations, how coefficients
attached to the same power of an underterminate variable can be equated, how
\ac{fps} can be multiplied or derived and so on. Additional treatment of \ac{gf}
using this kind of approach can be found in \cite{Wilf:2006:GEN:1204575}.
\\\\
We report a simple example involving two basic sequences, first the one
composed of all $1$s, namely $\vect{1}=\lbrace1\rbrace_{n\in\mathbb{N}}$; second, the
one composed of natural numbers, namely $\vect{n}=\lbrace n\rbrace_{n\in\mathbb{N}}$.
We start recalling an important result about sequences.
\begin{theorem}[Identity Principle]
    Let $\vect{\omega}=\lbrace\omega_{n}\rbrace_{n\in\mathbb{N}}$ and 
        $\vect{\theta}=\lbrace\theta_{n}\rbrace_{n\in\mathbb{N}}$ be two sequences.
        Then:
    \begin{displaymath}
        \omega_{n}=\theta_{n}\leftrightarrow\mathcal{G}_{t}\left(\vect{\omega}\right)
            =\mathcal{G}_{t}\left(\vect{\theta}\right)
    \end{displaymath}
    \label{thm:identity:principle}
    for each $n\in\mathbb{N}$.
\end{theorem}
In order to find the \ac{gf} $\Omega$ of sequence
\marginpar{an example: computing \ac{gf} of sequence $\vect{1}=\lbrace1\rbrace_{n\in\mathbb{N}}$}
$\vect{\omega}=\lbrace\omega_{n}\rbrace_{n\in\mathbb{N}}
=\lbrace1\rbrace_{n\in\mathbb{N}}=\vect{1}$ we reason as follows.  Define a sequence
$\vect{\theta}=\lbrace\theta_{n}\rbrace_{n\in\mathbb{N}}=\lbrace\omega_{n+1}\rbrace_{n\in\mathbb{N}}$.
Since $\omega_{n}=\theta_{n}$ because $\omega_{n}=\omega_{n+1}=1$, for each
$n\in\mathbb{N}$, then it is possible to apply
\autoref{thm:identity:principle}, so:
\begin{displaymath}
    \mathcal{G}_{t}\left(\lbrace\omega_{n}\rbrace\right) 
        =\mathcal{G}_{t}\left(\lbrace\omega_{n+1}\rbrace\right)
        =\frac{\mathcal{G}_{t}\left(\lbrace\omega_{n}\rbrace\right)-\omega_{0}}{t}
        =\frac{\mathcal{G}_{t}\left(\lbrace\omega_{n}\rbrace\right)-1}{t}
\end{displaymath}
where the middle step hold because $\lbrace\omega_{n+1}\rbrace_{n\in\mathbb{N}}$ is
just sequence $\vect{\omega}$ shifted by one position. A simple manipulation yield:
\begin{equation}
    \Omega(t)=\mathcal{G}_{t}\left(\vect{1}\right) =\frac{1}{1-t}
    \label{eq:1:gf}
\end{equation}
which we are going to expand as \ac{fps}. Taylor expansion of function $\Omega$
yields:
\begin{displaymath}
    \Omega(t)=1+t+t^{2}+t^{3}+t^{4}+t^{5}+\mathcal{O}(t^{6})
\end{displaymath}
as required.

In order to find the \ac{gf} $N$ of sequence $\vect{n}=\lbrace
n\rbrace_{n\in\mathbb{N}}$, namely the sequence of natural numbers, observe
\marginpar{an example: computing \ac{gf} of sequence $\vect{n}=\lbrace n\rbrace_{n\in\mathbb{N}}$}
that sequence $\vect{n}$ can be written, more verbosly, as $\vect{n}=\lbrace
n\cdot1\rbrace_{n\in\mathbb{N}}$, which has shape $\lbrace
n\cdot\omega_{n}\rbrace_{n\in\mathbb{N}}$, where $\omega_{n}=1$, for each
$n\in\mathbb{N}$, so $\vect{\omega}=\vect{1}$. Therefore:
\begin{equation}
    N(t)=\mathcal{G}_{t}\left(\vect{n}\right) =\frac{t}{(1-t)^{2}}
    \label{eq:n:gf}
\end{equation}
because $\mathcal{G}_{t}\left(\lbrace n\omega_{n}\rbrace\right) =t\,
\frac{\partial\,\mathcal{G}_{t}\left(\lbrace
\omega_{n}\rbrace\right)}{\partial\,t}$, in our case
$\mathcal{G}_{t}\left(\lbrace n\cdot1\rbrace\right) =
t\,\frac{\partial\,\mathcal{G}_{t}\left(\vect{1}\right)}{\partial\,t}$.
Again, do a little check, expanding function $N$ as a \ac{fps}:
\begin{displaymath}
    N(t)=t+2t^{2}+3t^{3}+4t^{4}+5t^{5}+\mathcal{O}(t^{6})
\end{displaymath}
as required.
\\\\
Another important concept that will be used in the definition of
\marginpar{operator $\left[t^{n}\right]$: the coefficient extractor}
\emph{Riordan arrays} is the $\left[t^{n}\right]$ operator, where
$t$ is an undeterminate function and $n\in\mathbb{N}$, also called
\emph{coefficient extractor}. This operator is defined by the following
relation:
\begin{equation}
    \left[t^{n}\right]\mathcal{G}_{t}\left(\lbrace\omega_{n}\rbrace\right) = \omega_{n}
    \label{eq:coefficient:extractor:def}
\end{equation}
for some sequence $\vect{\omega}=\lbrace\omega_{n}\rbrace_{n\in\mathbb{N}}$. More properties
of $\left[t^{n}\right]$ operator can be found in \cite{merlini:method:of:coefficient}.



% put here something about the method of coefficient
%\cite{merlini:method:of:coefficient}

