
\subsection{Renewal arrays, by Rogers}
\label{sec:back:to:the:basics:rogers}

\citeauthor{rogers:1977}, in \cite{rogers:1977}, proposes a theory
as a solution to the question left open by \citeauthor{shapiro:1976}, 
introducing the concept of \emph{renewal arrays}.

In order to understand his work, we formalize first the concept of $k$-fold
convolution of a sequence $\lbrace s_{i}\rbrace_{i\in\mathbb{N}}$ 
with itself and, second, the concept of \emph{renewal arrays}.
\\\\
Let $\lbrace \alpha_{i}\rbrace_{i\in\mathbb{N}}$ and 
$\lbrace \beta_{i}\rbrace_{i\in\mathbb{N}}$ be two sequences. Denote
with $\cdot$ the \emph{convolution} operator which consumes a pair
of sequences and produces a new sequence, written in \emph{infix} 
notation:
\begin{displaymath}
    \lbrace \alpha_{i}\rbrace_{i\in\mathbb{N}}\cdot
    \lbrace \beta_{i}\rbrace_{i\in\mathbb{N}} =
    \lbrace \theta_{i}\rbrace_{i\in\mathbb{N}}
\end{displaymath}
where sequence  $\lbrace \theta_{i}\rbrace_{i\in\mathbb{N}}$ satisfies:
\begin{displaymath}
    \theta_{n}=\sum_{j=0}^{n}{\alpha_{j}\beta_{n-j}}
\end{displaymath}

The $k$-fold convolution of a sequence $\lbrace \theta_{i}\rbrace_{i\in\mathbb{N}}$
\marginpar{$k$-fold convolution, precisely}
with itself, denoted with  $\lbrace \theta_{i}^{(k)}\rbrace_{i\in\mathbb{N}}$, is defined
recursively as:
\begin{displaymath}
    \lbrace\theta_{n}^{(k+1)}\rbrace= \lbrace\theta_{n}^{(k)}\rbrace\cdot
        \lbrace\theta_{n}\rbrace
\end{displaymath}
with boundary condition $\lbrace\theta_{n}^{(0)}\rbrace=(1,0,0,\ldots)$ so as:
\begin{displaymath}
    \lbrace\theta_{n}^{(1)}\rbrace= \lbrace\theta_{n}^{(0)}\rbrace\cdot
        \lbrace\theta_{n}\rbrace=
        \lbrace\theta_{n}+0\theta_{n-1}+\ldots+0\theta_{0}\rbrace=
        \lbrace\theta_{n}\rbrace
\end{displaymath}

