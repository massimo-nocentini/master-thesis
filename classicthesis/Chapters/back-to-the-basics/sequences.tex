
\subsection{$A$-sequence and $A$-matrix}
\label{sec:back:to:the:basics:sequences}

In this section we rework the concept of $A$-sequence introduced by
\citeauthor{rogers:1977}, subject of our discussion in
\autoref{sec:back:to:the:basics:rogers}, using the \emph{Riordan group}
to characterize it for \emph{Riordan arrays}. Furthermore, we will see
a generalization of this concept, introducing $A$-matrices and some
applications are reported for the sake of clarity.

\begin{theorem}
    $\mathcal{M}$ is a Riordan array over a matrix of coefficients
    $\lbrace m_{n,k}\rbrace_{n,k\in\mathbb{N}}$ if and only if
    there exists a sequence $\vect{a}=\lbrace a_{n}\rbrace_{n\in\mathbb{N}}$,
    with $a_{0}\neq0$, such that:
    \begin{displaymath}
        m_{n+1,k+1}=a_{0}m_{n,k}+a_{1}m_{n,k+1}+a_{2}m_{n,k+2}+\ldots+a_{j}m_{n,k+j}
    \end{displaymath}
    where $n,k,j\in\mathbb{N}$ and $k+j=n$. Sequence $\vect{a}$ is called the $A$-sequence
    of Riordan array $\mathcal{M}$. 
    \label{thm:merlini:A:sequence:characterization} 
\end{theorem}
\marginpar{The idea behind this proof comes from \cite{he:sprugnoli:2009}}
\begin{proof}[Proof of ($\rightarrow$) direction]
    Let $\mathcal{M}=(d_{\mathcal{M}}(t),h_{\mathcal{M}}(t))$ be a Riordan array. Since it is requested
    to prove that such a sequence $\vect{a}$ \emph{exists}, we proceed to
    build a Riordan array $\mathcal{A}=(d_{\mathcal{A}}(t),h_{\mathcal{A}}(t))$ such
    that function $d_{\mathcal{A}}$ is the \ac{gf} for sequence $\vect{a}$, proving this direction.
    Riordan array $\mathcal{A}$ is the defined as the solution of the following relation:
    \begin{displaymath}
        (d_{\mathcal{M}}(t),h_{\mathcal{M}}(t))\cdot(d_{\mathcal{A}}(t),h_{\mathcal{A}}(t))
            = \left(d_{\mathcal{M}}(t)\frac{h_{\mathcal{M}}(t)}{t},h_{\mathcal{M}}(t)\right)
    \end{displaymath}
    by definition of group operator $\cdot$, the \ac{lhs} can be rewritten as:
    \begin{displaymath}
        (d_{\mathcal{M}}(t)d_{\mathcal{A}}(h_{\mathcal{M}}(t)),h_{\mathcal{A}}(h_{\mathcal{M}}(t)))
            = \left(d_{\mathcal{M}}(t)\frac{h_{\mathcal{M}}(t)}{t},h_{\mathcal{M}}(t)\right)
    \end{displaymath}
    
    Looking at second components, the following relation defines function $h_{\mathcal{A}}$:
    \begin{displaymath}
         \left[h_{\mathcal{A}}(y)=y \mid y = h_{\mathcal{M}}(t)\right]
    \end{displaymath}
    while looking at the first component, the following relation defines function $d_{\mathcal{A}}$:
    \marginpar{a \ac{gf} over $A$-sequence $\vect{a}$ for Riordan arrays}
    \begin{displaymath}
         \left[d_{\mathcal{A}}(y)=\frac{y}{\hat{h}_{\mathcal{M}}(y)} \mid y = h_{\mathcal{M}}(t)\right]
    \end{displaymath}
    therefore it is possible to write the complete definition for array $\mathcal{A}$:
    \begin{displaymath}
         \mathcal{A}=(d_{\mathcal{A}}(y),h_{\mathcal{A}}(y))
            =\left(\frac{y}{\hat{h}_{\mathcal{M}}(y)}, y \right)
    \end{displaymath}
    
    \marginpar{although we are in the Riordan group, here we go back to 
        matrix theory}
    Assume function $d_{\mathcal{A}}$ can be expanded as \ac{fps} over a sequence
    $\vect{a}=\lbrace a_{n}\rbrace_{n\in\mathbb{N}}$, therefore expanding array $\mathcal{A}$
    as a matrix $\lbrace a_{nk}\rbrace_{n,k\in\mathbb{N}}$ yield:
    \begin{displaymath}
        \mathcal{A} =
        \left[
            \begin{array}{cccccc}
                a_{0} & & & &  &\\
                a_{1} & a_{0} & & &  &\\
                a_{2} & a_{1}& a_{0}& &  &\\
                a_{3} & a_{2}& a_{1}& a_{0}&  &\\
                a_{4} & a_{3}& a_{2}& a_{1}& a_{0} &\\
                \vdots & \vdots& \vdots& \vdots& \vdots & \ddots\\
            \end{array}
        \right]
    \end{displaymath}
    so doing the \emph{matrix} product of array $\mathcal{M}$ with array $\mathcal{A}$ yield:
    \begin{displaymath}
        \mathcal{M}\mathcal{A}
        = 
        \left[
            \begin{array}{cccccc}
                m_{00} & & & &  &\\
                m_{10} & m_{11} & & &  &\\
                m_{20} & m_{21}& m_{22}& &  &\\
                m_{30} & m_{31}& m_{32}& m_{33}&  &\\
                m_{40} & m_{41}& m_{42}& m_{43}& m_{44} &\\
                \vdots & \vdots& \vdots& \vdots& \vdots & \ddots\\
            \end{array}
        \right]
        \left[
            \begin{array}{cccccc}
                a_{0} & & & &  &\\
                a_{1} & a_{0} & & &  &\\
                a_{2} & a_{1}& a_{0}& &  &\\
                a_{3} & a_{2}& a_{1}& a_{0}&  &\\
                a_{4} & a_{3}& a_{2}& a_{1}& a_{0} &\\
                \vdots & \vdots& \vdots& \vdots& \vdots & \ddots\\
            \end{array}
        \right]
    \end{displaymath}
    Consider coefficient at row $n$ and column $k$ of $\mathcal{M}\mathcal{A}$, by definition
    of \emph{matrix} product:
    \begin{displaymath}
        \left(\mathcal{M}\mathcal{A}\right)_{nk}=
        \sum_{j=0}^{n}{m_{nj}{a_{jk}}}
            =\sum_{j=0}^{n-k}{m_{n,k+j}{a_{k+j,k}}}
            =\sum_{j=0}^{n-k}{m_{n,k+j}{a_{j}}}
    \end{displaymath}
    Do the same for coefficient at row $n$ and column $k$ of
    $\left(d_{\mathcal{M}}(t)\frac{h_{\mathcal{M}}(t)}{t},h_{\mathcal{M}}(t)\right)$:
    \begin{displaymath}
        [t^{n+1}]d_{\mathcal{M}}(t)\,h_{\mathcal{M}}(t)^{k+1} = m_{n+1,k+1}
    \end{displaymath}
    so equating this and the previous result yield:
    \begin{displaymath}
        m_{n+1,k+1}=\sum_{j=0}^{n}{m_{n,k+j}{a_{j}}}
    \end{displaymath}
    as required.
\end{proof}

The above proof leave space for another result. Start again from the following equation:
\begin{displaymath}
    (d_{\mathcal{M}}(t)d_{\mathcal{A}}(h_{\mathcal{M}}(t)),h_{\mathcal{M}}(t))
        = \left(d_{\mathcal{M}}(t)\frac{h_{\mathcal{M}}(t)}{t},h_{\mathcal{M}}(t)\right)
\end{displaymath}
and consider coefficient at row $n$ and column $k$ in both arrays, as done before:
\begin{displaymath}
    [t^{n}]d_{\mathcal{A}}(h_{\mathcal{M}}(t))\left(d_{\mathcal{M}}(t)\,h_{\mathcal{M}}(t)^{k}\right)
        = [t^{n+1}]d_{\mathcal{M}}(t)\,h_{\mathcal{M}}(t)^{k+1}
\end{displaymath}
\marginpar{sequence $\vect{\hat{a}}$, with \ac{gf} $\hat{A}(t)$, 
is yet another characterization}
on the \ac{lhs} appears a convolution of function $\hat{A}(t)=d_{\mathcal{A}}(h_{\mathcal{M}}(t))$,
which can be written as \ac{fps} over sequence $\vect{\hat{a}}=\lbrace\hat{a}_{i}\rbrace_{i\in\mathbb{N}}$,
with function $m_{k}(t)=d_{\mathcal{M}}(t)\,h_{\mathcal{M}}(t)^{k}$. Therefore the following identity
holds as well:
\begin{displaymath}
    \sum_{j=0}^{n}{\hat{a}_{j}\,m_{n-j,k}}
        = m_{n+1,k+1}
\end{displaymath}
combining this result with the one stated in the above theorem, the following holds:
\begin{displaymath}
    \sum_{j=0}^{n}{\hat{a}_{j}\,m_{n-j,k}}
        =\sum_{j=0}^{n}{m_{n,k+j}{a_{j}}}
\end{displaymath}

Yet another characterization for a coefficient $m_{nk}$ in a Riordan array $\mathcal{M}$
can be stated and is closely related to $\mathcal{M}$'s $A$-sequence.
\begin{theorem}
    Let $\mathcal{M}$ be a Riordan array over a matrix of coefficients
    $\lbrace m_{n,k}\rbrace_{n,k\in\mathbb{N}}$, then there exists a sequence
    $\vect{\omega}=\lbrace \omega_{n}\rbrace_{n\in\mathbb{N}}$ such that:
    \begin{displaymath}
        m_{nk}=\sum_{j=0}^{n-k}{\omega_{j}\,m_{n+1,k+1+j}}
    \end{displaymath}
    moreover, denoting with $\Omega$ a \ac{gf} over sequence $\vect{\omega}$ and
    with $A$ the \ac{gf} for $\mathcal{M}$'s $A$-sequence, then:
    \begin{displaymath}
        A(t)\,\Omega(t)=1
    \end{displaymath}
    \label{thm:characterization:next:row}
\end{theorem}
\begin{proof}
    Assume not, therefore for \emph{each} sequence $\vect{\omega}=\lbrace \omega_{n}\rbrace_{n\in\mathbb{N}}$,
    from:
    \begin{displaymath}
        m_{nk}\neq\sum_{j=0}^{n-k}{\omega_{j}\,m_{n+1,k+1+j}}
    \end{displaymath}
    we have to show a contraddiction. By $A$-sequence characterization, 
    there exists a sequence $\vect{a}=\lbrace a_{n}\rbrace_{n\in\mathbb{N}}$
    such that allows us to rewrite terms $m_{n+1,k+1+j}$
    within the sum in the \ac{rhs}:
    \begin{displaymath}
        \begin{split}
            m_{nk} &\neq \omega_{0}\left(a_{0}m_{n,k}+a_{1}m_{n,k+1}+a_{2}m_{n,k+2}+a_{3}m_{n,k+3}+\ldots+a_{j_{0}}m_{n,k+j_{0}}\right)\\
                   &+ \omega_{1}\left(a_{0}m_{n,k+1}+a_{1}m_{n,k+2}+a_{2}m_{n,k+3}+a_{3}m_{n,k+4}+\ldots+a_{j_{1}}m_{n,k+j_{1}}\right)\\
                   &+ \omega_{2}\left(a_{0}m_{n,k+2}+a_{1}m_{n,k+3}+a_{2}m_{n,k+4}+a_{3}m_{n,k+5}+\ldots+a_{j_{2}}m_{n,k+j_{2}}\right)\\
                   &+ \omega_{3}\left(a_{0}m_{n,k+3}+a_{1}m_{n,k+4}+a_{2}m_{n,k+5}+a_{3}m_{n,k+6}+\ldots+a_{j_{3}}m_{n,k+j_{3}}\right)\\
                   &\ldots\\
                   &+ \omega_{n-k-1}\left(a_{0}m_{n,n-1}+a_{1}m_{n,n}\right)\\
                   &+ \omega_{n-k}\left(a_{0}m_{n,n}\right)\\
        \end{split}
    \end{displaymath}
    factoring coefficients $m_{nk}$, the \ac{rhs} can be rewritten as:
    \begin{displaymath}
        \begin{split}
            m_{nk}&\neq \omega_{0}a_{0}m_{n,k} + \left(\sum_{i_{1}+i_{2}=1}{\omega_{i_{1}}a_{i_{2}}}\right)m_{n,k+1}
                + \left(\sum_{i_{1}+i_{2}=2}{\omega_{i_{1}}a_{i_{2}}}\right)m_{n,k+2}\\
                &+ \left(\sum_{i_{1}+i_{2}=3}{\omega_{i_{1}}a_{i_{2}}}\right)m_{n,k+3}
                + \ldots 
                + \left(\sum_{i_{1}+i_{2}=n-k}{\omega_{i_{1}}a_{i_{2}}}\right)m_{n,n}
        \end{split}
    \end{displaymath}
    which can be written more compactly as:
    \begin{displaymath}
        m_{nk} \neq \omega_{0}a_{0}m_{n,k} +
            \sum_{j=1}^{n-k}{\left(\sum_{i_{1}+i_{2}=j}{\omega_{i_{1}}a_{i_{2}}}\right)m_{n,k+j}}
    \end{displaymath}
    but if we take sequence $\vect{\omega}$ to be the \emph{inverse} sequence of 
    $\mathcal{M}$'s $A$-sequence $\vect{a}$, then the previous equation is false,
    since $\omega_{0}a_{0}=1$ and $\sum_{i_{1}+i_{2}=j}{\omega_{i_{1}}a_{i_{2}}}=0$,
    for each $j\in\lbrace1,\ldots,n-k\rbrace$, equality \emph{holds}. 
    A contraddiction occurs, as required.        
\end{proof}

\subsubsection{$A$-matrix}

We are now ready to tackle the most general characterization for a coefficient
$m_{nk}$ in a Riordan array $\mathcal{M}$, the concept of $A$-matrix. Actually,
we will see only one formulation but there exist another two of them, yet more
general indeed. The following theorem and the proof idea comes from 
\cite{merlini:some:alternative:characterizations:1997}, where additional
formulation of the $A$-matrix concept can be found.

\begin{theorem}
    $\mathcal{M}$ is a Riordan array, over a matrix of coefficients
    $\lbrace m_{n,k}\rbrace_{n,k\in\mathbb{N}}$, if there exists a matrix
    $\lbrace \sigma_{nk}\rbrace_{n,k\in\mathbb{N}}$ such that:
    \begin{displaymath}
        m_{n+1,k+1}=\sum_{i\in\mathbb{N}}{\sum_{j\in\mathbb{N}}{\sigma_{ij}m_{n-i,k+j}}}
    \end{displaymath}
\end{theorem}

\begin{proof}
    The idea underlying this proof is to show that a sequence 
    $\vect{a}=\lbrace a_{n}\rbrace_{n\in\mathbb{N}}$ can be built from
    matrix $\lbrace \sigma_{nk}\rbrace_{n,k\in\mathbb{N}}$,
    such that $\vect{a}$ can be used to combine elements in matrix
    $\lbrace m_{n,k}\rbrace_{n,k\in\mathbb{N}}$ according to: 
    \begin{displaymath}
        m_{n+1,k+1}=a_{0}m_{nk}+a_{1}m_{n,k+1}+\ldots+a_{j}m_{n,k+j}
    \end{displaymath}
    where $k+j=n$, as usual.  If we success, then sequence $\vect{a}$ is the $A$-sequence for 
    $\lbrace m_{n,k}\rbrace_{n,k\in\mathbb{N}}$, therefore $\mathcal{M}$ is a Riordan array.
    \\\\
    Choose $n\in\mathbb{N}$, then proceed by natural induction on $n-k$:
    \begin{itemize}
        \item base case, $n-k=0$. By assumption the following holds:
            \begin{displaymath}
                m_{n+1,n+1}=\sigma_{00}\,m_{nn}
            \end{displaymath}
            which suggests to set $a_{0}=\sigma_{00}$;
        \item although not required, consider case $n-k=1$. By assumption the following holds:
            \begin{displaymath}
                m_{n+1,n}=\sigma_{00}m_{n,n-1}+\sigma_{01}m_{n,n}+\sigma_{10}m_{n-1,n-1}
            \end{displaymath}
            by \autoref{thm:characterization:next:row}, there exist a sequence $\vect{\omega}=
            \lbrace \omega_{n}\rbrace_{n\in\mathbb{N}}$ which allow us to rewrite $m_{n-1,n-1}$
            as a combination of $m_{nn}$:
            \begin{displaymath}
                m_{n+1,n}=\sigma_{00}m_{n,n-1}+\left(\sigma_{01}+\sigma_{10}\omega_{0}\right)m_{nn}
            \end{displaymath}
            which reinforces $a_{0}=\sigma_{00}$ and suggests to set 
            $a_{1}=\sigma_{01}+\sigma_{10}\omega_{0}$; moreover, using sequence $\vect{\omega}$
            to rewrite terms in the \ac{rhs}:
            \begin{displaymath}
                m_{n+1,n}=\sigma_{00}\left(\omega_{0}m_{n+1,n}+\omega_{1}m_{n+1,n+1}\right)+
                    \left(\frac{\sigma_{01}+\sigma_{10}\omega_{0}}{\sigma_{00}}\right)m_{n+1,n+1}
            \end{displaymath}
            manipulating:
            \begin{displaymath}
                m_{n+1,n}=
                    \left(\frac{\sigma_{00}^{2}\omega_{1}+\sigma_{01}+\sigma_{10}\omega_{0}}
                    {(1-\sigma_{00}\omega_{0})\sigma_{00}}\right)m_{n+1,n+1}
            \end{displaymath}
        \item \emph{induction hp} assume that, for $n-k=n-1$, ie. $k=1$, if:
            \begin{displaymath}
                m_{n+1,2}=\sum_{i\in\mathbb{N}}{\sum_{j\in\mathbb{N}}{\sigma_{ij}m_{n-i,1+j}}}
            \end{displaymath}
            then there exists a sequence $\vect{\hat{a}}$ such that:
            \begin{displaymath}
                m_{n+1,2}=\hat{a}_{0}m_{n1}+\hat{a}_{1}m_{n,2}+\ldots+\hat{a}_{j}m_{n,1+j}
            \end{displaymath}
            where $1+j=n$;
        \item \emph{induction step} show that, for $n-k=n$, ie. $k=0$, if:
            \begin{displaymath}
                m_{n+1,1}=\sum_{i\in\mathbb{N}}{\sum_{j\in\mathbb{N}}{\sigma_{ij}m_{n-i,j}}}
            \end{displaymath}
            then there exists a sequence $\vect{a}$ such that:
            \begin{displaymath}
                m_{n+1,1}=a_{0}m_{n0}+a_{1}m_{n,1}+\ldots+a_{n}m_{n,n}
            \end{displaymath}

            Begin by expanding the assumption:
            \begin{displaymath}
                \hspace{-2cm}
                \begin{split}
                    m_{n+1,1}&=\sigma_{00}m_{n0}+\sigma_{01}m_{n1}+\sigma_{02}m_{n2}+\ldots
                                +\sigma_{0,n-2}m_{n,n-2}+\sigma_{0,n-1}m_{n,n-1}+\sigma_{0n}m_{nn}\\
                             &+\sigma_{10}m_{n-1,0}+\sigma_{11}m_{n-1,1}+\sigma_{12}m_{n-1,2}+\ldots
                                +\sigma_{1,n-2}m_{n-1,n-2}+\sigma_{1,n-1}m_{n-1,n-1}\\
                             &+\sigma_{20}m_{n-2,0}+\sigma_{21}m_{n-2,1}+\sigma_{22}m_{n-2,2}+\ldots+\sigma_{2,n-2}m_{n-2,n-2}\\
                             &\ldots\\
                             &+\sigma_{n-1,0}m_{10}+\sigma_{n-1,1}m_{11}\\
                             &+\sigma_{n0}m_{00}\\
                \end{split}
            \end{displaymath}
            from the bottom line of the previous sum expansion, keep applying
            \autoref{thm:characterization:next:row} to every coefficient $m_{rc}$, for
            $r\in\lbrace0,\ldots,n-1\rbrace$ and, consequently, $c\in\lbrace0,\ldots,r\rbrace$.
            When every coefficient $m_{n-1,c}$, for $c\in\lbrace0,\ldots,n-1\rbrace$, has been
            expanded, coefficient $m_{n+1,1}$ is a combination of coefficients 
            $\lbrace m_{n0}\rbrace\cup\lbrace m_{n,1},m_{n,2}\ldots,m_{n,n}\rbrace$ using
            a sequence $\vect{\beta}$, namely:
            \begin{displaymath}
                m_{n+1,1}=\beta_{0}m_{n0}+\beta_{1}m_{n1}+\ldots+\beta_{n}m_{nn}
            \end{displaymath}
            By induction hypothesis, there exists a sequence $\vect{\hat{a}}$ such that:
            \begin{displaymath}
                m_{n+1,2}=\hat{a}_{0}m_{n1}+\hat{a}_{1}m_{n2}+\ldots+\hat{a}_{n-1}m_{nn}
            \end{displaymath}
            so, we can build a sequence $\vect{a}$ as follows:
            \begin{displaymath}
                \vect{a}=\left(\hat{a}_{0},\hat{a}_{1},\hat{a}_{2},\ldots,\hat{a}_{n-1},\beta_{n}\right)
            \end{displaymath}
            where $\beta_{n}$ can be possibly different from $\hat{a}_{n-1}$. 
            Sequence $\vect{a}$ is the $A$-sequence
            for matrix $\lbrace m_{nk}\rbrace_{n,k\in\mathbb{N}}$, therefore $\mathcal{M}$ is a Riordan
            array, as required.
    \end{itemize}
\end{proof}


















