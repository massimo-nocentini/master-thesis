%********************************************************************
% Appendix
%*******************************************************
% If problems with the headers: get headings in appendix etc. right
%\markboth{\spacedlowsmallcaps{Appendix}}{\spacedlowsmallcaps{Appendix}}
\chapter{Appendix: Python implementation}
\label{ch:appendix:python:implementation}

\section{Engineering Python's modules}

In this appendix we talk about how we implemented a subset of the \emph{Riordan
group} theory using the Python programming language.  Although our code is pure
Python code, we rest on the \emph{Sage} \cite{sage} mathematical framework for
symbolic computations\marginpar{pure Python classes, delegating to \emph{Sage}}.

We start from the abstract world of mathematics and, gradually, introduce the
architecture, principles, design choices and, finally, hierarchies of classes
that allow us to fullfil our implementation.

\subsection{From mathematics to code}
\label{subsection:python:appendix:from:math:to:code}

Mathematically, a function $colouring$ has been implemented. Let $\mathcal{M}$
be a Riordan array and $m_{nk} \in \mathcal{M}$ its generic element, moreover
define a set of $k$ colours $\lbrace c_0, \ldots, c_{k-1} \rbrace$. So,
function $colouring$ has the following type: 
\begin{displaymath} 
    colouring : \mathbb{N} \times\mathbb{N} \rightarrow \lbrace c_0, \ldots, c_{k-1} \rbrace
\end{displaymath}
Let $p\in\mathbb{N}$, usually a prime number, the following definitions have been implemented:
\begin{itemize}
    \item colour $\mathcal{M}$ associating to each remainder class 
        $r \in \lbrace[0],\ldots,[p-1]\rbrace$ a different colour $c_r$:
        \begin{displaymath}
            colouring_{p}(n,k) = c_{r} \leftrightarrow m_{nk} \equiv_{p} r
        \end{displaymath}
    \item \emph{bi}-colour $\mathcal{M}$ with $c_0$ if $m_{nk}$ 
        is a multiple of $p$, otherwise use a colour $c_1$:
        \begin{displaymath}
            colouring_{p}(n,k) = c_{0} \leftrightarrow p | m_{nk}
        \end{displaymath}
    \item a less used one, \emph{bi}-colour $\mathcal{M}$ with $c_0$ 
        if $m_{nk}$ is a prime, otherwise use a colour $c_1$
            \marginpar{``prime'' definition of $colouring$ 
                seems to produce triangles coloured at random\ldots}:
        \begin{displaymath}
            colouring(n,k) = c_{0} \leftrightarrow 
                \nexists s\in\lbrace 2,\ldots,m_{nk}-1\rbrace.s|m_{nk} 
        \end{displaymath}
\end{itemize}

We choose the \emph{Python} language to implement such a function and we rest
on the \emph{Sage} \cite{sage} framework to do hard computational stuff, like
Taylor expansion of a function or solving equations \emph{symbolically}.

Our implementation aims at a subset of the \emph{Riordan group} theory,
focusing on simplicity and sound design principles. The interest of the author
for the \emph{Smalltalk} programming language \cite{Goldberg:1983:SLI:273} has
    influenced the design of the code base, with a strong focus on the concept
    of \emph{messaging} between objects as the core programming paradigm. We
    believe that keeping such an approach, classes are structured to be really
    extendible, keeping the whole implementation limited to few core concept.
    In addition to this orientation, we have integrated some particular
    features supplied by the Python programming language in order to ease the
    experience of playing with our objects.

We \marginpar{all Python code and \LaTeX files of this work is under \emph{Git}
version control} wont describe all the ``nitty-gritty'' details of the
implementation, on the other hand we focus on the architecture, pointing out
the concepts, and some design principles and patterns used to implement it.
Finally, we review a set of core classes, pointing the interested reader to the
complete implementation, which is freely available on line, in the \emph{Git}
repository \cite{nocentini:git:repository}.

\subsection{Architecture}

Talking about \emph{an architecture} to describe the implementation of a single
\emph{functionality}, namely the $colouring$ function, seems to be wasteful.
However, \emph{Smalltalkers} like to introduce a little language for each
problem they tackle, which is the same to say they introduce \emph{an
architecture}, eventually. What is important is the spirit and the principle
behind their approach: \emph{messaging}. As \citeauthor{kay:on:messaging}
states in \cite{kay:on:messaging}, what is really important is the relation and
the net of messages that a set of objects send to each other. Pairing this
principle with the one shown in
\cite{friedman:felleisen:few:java:few:patterns}, about looking at behavior as
an object, we have a solid track to follow.

Our objects are instances of a few class hierarchies, many of which can be seen
\marginpar{dispatching and action objects} as \emph{dispatching objects}, while
actual behavior is performed by a very restricted set of them, what we call
\emph{action objects}. So, the whole $colouring$ functionality is taken apart
into a set of small hierarchies, such that each one of them catches a
\emph{context} that can happen in a study session; example contexts are the
desired type of partitioning or if the representation should have a plain
layout or a centered one\ldots Each object which is an instance of a little
hierarchy, doesn't know how to do hard symbolic computation or how to generate
a result \TeX\,file. Its only responsibility is to inform that \emph{the
requested functionality should be performed under the context it represents}.
This is pretty similar to the approach advised by \emph{functional
programming}, where \emph{pattern matching} over type constructors is a compact
way to express the same computation as we do with a net of \emph{messages}.

Only a small set of \emph{action objects}\marginpar{an action object is the
recipient of a chain of dispatching messages}, aka \emph{visitors} if we look
at them from the design patterns \cite{Gamma:1995:DPE:186897} point of view,
code the actual implementation because only them know the \emph{entire}
context.  With this approach, if we would like to implement a new
\emph{functionality}, we have to \emph{create} a new \emph{action object}; on
the other hand, if we would like to augment the context with new informations,
we are required to \emph{refine} methods already implemented in \emph{action
objects}, without updating the remaining classes.

Finally, this approach leads quite naturally to\marginpar{favor
\emph{composition} over \emph{inheritance}} favor \emph{composition} over
\emph{inheritance}. Due to the \emph{dynamic} typing of Python, without the
constraint to assign \emph{explicitly} a type to every expression, using the
proposed methodology we have built a code base that is highly polymorphic,
without constraining objects in rigid inheritance structures.

\subsection{Core classes and hierarchies}

\subsubsection{\emph{RiordanArray} class and \emph{Characterization} hierarchy}

The \emph{RiordanArray} class is the first of a set of \emph{interface} classes
that the client will work with.  Its responsibility is to denote the
\emph{mathematical} concept of Riordan array, but kept alone, it couldn't do
many things. It can be \emph{indexed} to supply a coefficient $m_{nk}$ it
contains but asking for raw matrix expansion or doing group operations, such as
multiplication or inversion, depends on how the Riordan array has been
\emph{characterized}, and this context information is caught by the
\emph{Characterization} hierarchy\marginpar{instances built from a class in the
Characterization hierarchy are dispatching objects}. We provide the following
characterizations: 
\begin{itemize}
    \item by \emph{sequences}, that is using $A$ and $Z$ sequences, together with
        coefficient $m_{00}$;
    \item by \emph{matrices}, that is using $A$-matrices, where each $A$-matrix can 
        be expressed as a table of coefficients $\lbrace a_{nk}\rbrace_{n,k\in\mathbb{N}}$ 
        or as a sequence of \ac{gf}, one for each matrix's row;
    \item by \emph{subgroup}, which essentially allows the client to use \emph{analytic
        functions} $d$ and $h$ to define the desired array. Up to now, only one
        subgroup is implemented, the \emph{vanilla} subgroup which resembles the
        formal way to define a Riordan array $\mathcal{M}$ as a pair of functions $(d(t),h(t))$.
        We will extend the set of \emph{subgroups} toward the one studied in the literature:
        this will allow the client to play with our object having a \emph{mathematical}
        experience, for example if an array in the \emph{Renewal} subgroup is desired,
        the client should only define function $d$, since function $h$ is implicitly
        defined as $h(t)=t\,d(t)$.
\end{itemize}

Therefore, with the introduction of \emph{characterization} context, denoted by
an object instance of the previous classes, it is possible for a
\emph{RiordanArray} instance to \emph{dispatch} messages on such
characterization that it cannot fullfil by itself, for example messages about
doing group operations, raw matrix expansion, building a formal
\emph{mathematical} name, formatted as a pair of functions $(d(t),h(t))$.

\subsubsection{\emph{TriangleColouring} class and auxiliary hierarchies}

The \emph{TriangleColouring} class is the second \emph{interface} class, which
denotes the desired colouring, namely a representatation of a Riordan array
under a congruence relation $\equiv_{p}$, for some modulo $p$. An object,
instance of this class, builds context objects about the desired layout of the
triangle (\emph{centered} or \emph{left-aligned}) or if negative coefficient in
an inverse array should be associated with \emph{lighter} colour variant
respect positive coefficients, when both belong to the same remainder class of
$\equiv_{p}$ relation. Such additional context objects are instances of
\emph{auxiliary hierarchies}, such as \emph{TriangleShape} and
\emph{NegativeChoices}.

Python language allows to \emph{override} some \emph{special} methods that the
virtual machine calls on an object when it is used in a particular
\emph{syntatic context}. \marginpar{\mintinline{python}|__getindex__(index)| is
another message that the vm sends to an object $obj$ when it is indexed, as in
$obj[n,k]$} One of such method is \mintinline{python}|__call__|, which is sent
to an object \mintinline{python}|obj| when it is used like a function, for
example: \mint{python}|obj(array=anArray, partitioning=aPartitioning)| for some
objects \mintinline{python}|anArray| and \mintinline{python}|aPartitioning|. We
have \emph{overridden} such a method in order to provide a more \emph{fluent}
interface: recall that, eventually, our interest is to implement the function
$colouring$, so it is nice to be able to use the object that denotes such
functionality \emph{like} a function, as in \emph{mathematics} we are able to
do. 

\subsubsection{\emph{Partitioning} hierarchy}

\emph{Partitioning} hierarchy denotes how the \emph{modular transformation} is
applied to an array $\mathcal{M}$. It is composed of three classes, one for
each implementation of \emph{math} function $colouring$ described in
\autoref{subsection:python:appendix:from:math:to:code}.

\subsubsection{\emph{ActionUsingSubgroupCharacterization} hierarchy}

Previous classes, hierarchies and auxiliaries build instances that belong to a
set of objects that we call \emph{dispatching objects}, since their responsibilities
are to dispatch messages, where each such object augment the message with the
context information it is responsible for. 

Classes in the \emph{ActionUsingSubgroupCharacterization} hierarchy, on the other hand,
build instances that belong to a set of objects that we call \emph{action objects},
since they \emph{can} perform a functionality since they \emph{know} the entire context,
having collected all context informations \emph{dispatched} from remaining objects.
There are the following classes in this hierarchy, each one denoting a \emph{major}
functionality that is necessary to fullfil the implementation of the $colouring$ 
function\marginpar{action objects}:
\begin{itemize}
    \item \emph{ExpansionActionUsingSubgroupCharacterization} class denotes
        the operation of doing raw matrix expansion of an array $\mathcal{M}$; 
    \item \emph{FormalDefLatexCodeUsingSubgroupCharacterization} class denotes
        the operation of producing a formal description of the array, 
        formatted as $\mathcal{M}=(d(t),h(t))$;
    \item \emph{BuildInverseActionUsingSubgroupCharacterization} class denotes
        the operation of \emph{inverting} an array $\mathcal{M}$ to produce 
        the array $\mathcal{M}^{-1}$.
\end{itemize}

All the above classes stricly depend on how the Riordan array has been
characterized, therefore they code a different implementation for each such
characterization. Looking at them from this point of view, it is easy to
recognize that the set of \emph{action objects} is actually a set of
\emph{visitors}. It is interesting how the\marginpar{Visitor pattern}
\emph{Visitor pattern} can be generalized to a net of \emph{dispatching
messages}, without limiting its application to hierarchies denoting a recursive
domain.


\section{A \emph{Sage} study file}

In this section we show a typical session where we study the \emph{Motzkin}
array $\mathcal{M}$.  First we report the complete script that we have used to
generate matrix raw expansion and coloured representations of
$\mathcal{M}_{\equiv_{p}}$, for some prime $p$; after we take it apart
describing the meaning of each chunk of code.

\subsection{The complete \emph{study} script for Motzkin array $\mathcal{M}$...}

\inputminted[
    mathescape,
    linenos,
    %numbersep=5pt,
    %gobble=2,
    frame=lines,
    framesep=2mm,
    breaklines
    ]{python}{../sympy/motzkin/motzkin-for-document-inclusion.sage}

\subsection{...and taking it apart, bite after bite}

Having seen the complete session script for array $\mathcal{M}$, in the
following paragraphs we take it apart, describing each little chunk in greater
detail.

\inputminted[
    mathescape,
    linenos,
    %numbersep=5pt,
    %gobble=2,
    frame=lines,
    framesep=2mm,
    breaklines,
    firstline=1,
    lastline=16
    ]{python}{../sympy/motzkin/motzkin-for-document-inclusion.sage}
    
In this chunk we do some boring \emph{imports} and prepare a variable
\mintinline{python}|tex_parent_prefix| in order to localize the destination
path for files that will be generated. Finally, variable \mintinline{python}|t|
is defined in order to denote a \emph{math} undeterminate variable
$t$\marginpar{\mintinline{python}|t| denotes a \emph{Python object}, $t$
denotes a \emph{math variable}}.

\inputminted[
    mathescape,
    linenos,
    %numbersep=5pt,
    %gobble=2,
    frame=lines,
    framesep=2mm,
    breaklines,
    firstline=19,
    lastline=25
    ]{python}{../sympy/motzkin/motzkin-for-document-inclusion.sage}

Here we define the desired partitioning, in this case we want
$\mathcal{M}_{\stackrel{\circ}{\equiv_{7}}}$, and the colouring object. It
keeps track of the number of rows, namely $127$; if the representation should
be centered or aligned to the left, as a raw matrix expansion looks like; and,
finally, if negative coefficients in the inverse array $\mathcal{M}^{-1}$
should get lighter colour variants respect ones that positive coefficient in
$\mathcal{M}$ get, under congruence relation $\equiv_{7}$.

\inputminted[
    mathescape,
    linenos,
    %numbersep=5pt,
    %gobble=2,
    frame=lines,
    framesep=2mm,
    breaklines,
    firstline=32,
    lastline=45
    ]{python}{../sympy/motzkin/motzkin-for-document-inclusion.sage}

Here we define function $d$ and function $h$ of $\mathcal{M}$ and build an
object that denotes the \emph{mathematical} concept of Riordan array, composed
of the following slots\marginpar{we provide strong integration with \LaTeX}:
\begin{itemize}
    \item a \emph{subgroup}, in this case we use the \emph{plain vanilla} definition by
        providing functions $d$ and $h$ directly;
    \item a \emph{name}, which will be used to build unique filenames for 
        the generated \TeX\,files;
    \item a \emph{mathematical name}, as the ones used in this document, such as $\mathcal{M}$ and so on,
        used within \emph{captions} environments, for example;
    \item an \emph{additional text} in order to augment array's description, it will be attached
        to \emph{captions} environments again. In this text \emph{any} \LaTeX\, code can be used.
\end{itemize}

\inputminted[
    mathescape,
    linenos,
    %numbersep=5pt,
    %gobble=2,
    frame=lines,
    framesep=2mm,
    breaklines,
    firstline=48,
    lastline=56
    ]{python}{../sympy/motzkin/motzkin-for-document-inclusion.sage}

Here some computation happen, actually: application of
\mintinline{python}|colouring| object expands $\mathcal{M}$ as a matrix
$\lbrace m_{nk}\rbrace_{n,k\in\mathbb{N}}$, up to row $126$ included, starting
from row $0$. Finally, it writes a bunch of files containing results, splitted
according their content.

\inputminted[
    mathescape,
    linenos,
    %numbersep=5pt,
    %gobble=2,
    frame=lines,
    framesep=2mm,
    breaklines,
    firstline=58,
    lastline=62
    ]{python}{../sympy/motzkin/motzkin-for-document-inclusion.sage}

Here we tackle the settings for computing
$\mathcal{M}^{-1}$\marginpar{computing $\mathcal{M}^{-1}$ providing an helper
lambda expression}. Mathematically, it is necessary to compute the
\emph{computational inverse} $\hat{h}$ of function $h$ and this is the main
difficult point to solve, for \emph{Sage} \cite{sage} too.  \emph{Sage}
framework is pretty powerful, for instance it can compute function
$\hat{h}_{\mathcal{P}}$ for the Pascal array $\mathcal{P}$, but for Motzkin
array things get complicated because a radical appears in function $h$. 

For this reason, \emph{message} \mintinline{python}|inverse| sent to the object
denoting $\mathcal{M}$ is a curious one, because it asks for an \emph{help}
function which will allow \emph{Sage} to solve an equation, yielding function
$\hat{h}$ eventually. Formally, to find $\hat{h}$ we could solve a
\emph{functional} equation $h(\hat{h}(t))=t$ because $h$ is known.  Otherwise,
we could use the \emph{change of variable} trick yielding $\hat{h}(y)$ if we
can solve $h(t)=y$: 
\begin{displaymath}
    \left.\left[\hat{h}(y)=t\right|y=h(t)\right]
\end{displaymath}
so, at the end, we have to solve another equation respect 
variable $t$, obtaining it as a function of variable $y$.

This is exactly the aim of the helper function. It receive three arguments, 
namely variable $y$, variable $t$ and an equation over function $h$ such that:
\begin{displaymath}
    y=h(t)
\end{displaymath}
which in the case of array $\mathcal{M}$ rewrites as:
\begin{displaymath}
    y=\frac{1-t-\sqrt{1-2t-3t^{2}}}{2t}
\end{displaymath}
our help is to manipulate the equation in order to remove the radical. Therefore
we can do the following steps, in the given order:
\begin{displaymath}
    \begin{split}
        y&=\frac{1-t-\sqrt{1-2t-3t^{2}}}{2t} &\quad\text{given}\\
        2ty &=1-t-\sqrt{1-2t-3t^{2}} &\quad\text{multiply by } 2t\\
        -2ty &=-1+t+\sqrt{1-2t-3t^{2}} &\quad\text{multiply by } -1\\
        1-t-2ty &=\sqrt{1-2t-3t^{2}} &\quad\text{add } -(t-1)\\
        \left(1-t-2ty\right)^{2} &=1-2t-3t^{2} &\quad\text{square}\\
    \end{split}
\end{displaymath}
which is exactly the meaning of the supplied lambda: \mint{python}|lambda y, t,
equation: (equation*2*t*(-1) -(t-1))**2| 

\emph{Sage} is now able to solve the last equation respect variable $t$ and it
can find the desired function $\hat{h}$ as:
\begin{displaymath}
    \left.\left[\hat{h}(y)=\frac{y}{1+y+y^{2}}\right|
        y=\frac{1-t-\sqrt{1-2t-3t^{2}}}{2t} \right]
\end{displaymath}

To be truly precise,\marginpar{the supplied helper works as a certificate as
well} the above \emph{helper} lambda expression is both an help to \emph{Sage}
to build the inverse of an array  and it is also a \emph{certificate} for the
computed compositional inverse function $\hat{h}$.  Under the hood it is used
also at the end of the search of $\hat{h}$ to check if it really satisfies
$\hat{h}(h(t))=t$.

\inputminted[
    mathescape,
    linenos,
    %numbersep=5pt,
    %gobble=2,
    frame=lines,
    framesep=2mm,
    breaklines,
    firstline=65,
    lastline=74
    ]{python}{../sympy/motzkin/motzkin-for-document-inclusion.sage}

Finally, as before, we actually do the computation that will expand
$\mathcal{M}^{-1}$ as a matrix of coefficients and will produce desired
\LaTeX\,files composed of code necessary to build the modular representation
$\left(\mathcal{M}^{-1}\right)_{\equiv_{7}}$.
\\\\
This is the most complete example our simple implementation allows us to do and,
although simple, it generates \emph{objects of beauty}.
