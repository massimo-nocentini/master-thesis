
This thesis has the following structure:
\begin{itemize}
    \item in \autoref{ch:introduction}, namely the present chapter,
        a brief introduction of motivations and targets is given;
    \item in \autoref{ch:chronologial:development}, first we recall
        some basic concepts about sequences and \ac{gf}; second,
        we provide a \emph{chronological} development that yield
        the \emph{Riordan array} concept, reviewing basic theorems
        and reworking their proofs;
    \item in \autoref{ch:modular:characterization}, we apply a modular
        transformation to \emph{Pascal} and \emph{Catalan} arrays and
        we provide a formal characterizations for them;
    \item in \autoref{ch:h:characterization}, we show a new characterization
        for a Riordan array, called \emph{$h$-characterization}, which
        allows us to denote the same array using a combination of its
        original defining functions;
    \item in \autoref{ch:conclusions}, we briefly review the content of this
        work and we point out a list of further interesting developments;
    \item in \autoref{ch:appendix:python:implementation}, we describe
        the Python implementation supporting our work, showing design principles
        of the code base and commenting, chunk by chunk, an usual \emph{script}
        that allows us to study the \emph{Motzkin} Riordan array;
    \item in \autoref{ch:appendix:bonus:triangles}, we collect some pictures
        reporting coloured triangles, which are representation of arrays
        after a modular transformation has been applied to them.
\end{itemize}
