
This work aims to study\marginpar{analytic combinatorics and discrete objects}
a subset of objects belonging to the field of analytic combinatorics, which
comprises tools such as generating functions, Riordan arrays and the symbolic
method. Many interesting books exist on those topics, such as
\cite{Flajolet:2009:AC:1506267}, \cite{Knuth:1998:ACP:521463} and
\cite{Graham:1994:CMF:562056} by \citeauthor{Flajolet:2009:AC:1506267},
\citeauthor{Knuth:1998:ACP:521463} and \citeauthor{Graham:1994:CMF:562056},
respectively: they describe skillful methods to handle sequences of
\emph{counting numbers}, combinatorial sums and classes of combinatorial
discrete objects, such as graphs, words, lattice paths and trees. 

In particular, the concept of \emph{Riordan array} is the core of the present
work.  We are interested to show new characterizations to spot some properties
of their structure. One example is the $h$-characterization
$\mathcal{R}_{h(t)}$ of a Riordan array $\mathcal{R}$ and a second one is the
generalization of \textit{$A$-sequence} %$\lbrace a_{n}\rbrace_{n\in\mathbb{N}}$ 
and
\textit{$A$-matrix} %$\lbrace a_{ij}\rbrace_{i,j\in\mathbb{N}}$ 
concepts.

Although \emph{Riordan group}\marginpar{Riordan group theory: from the
basics\ldots} theory has been studied intensively in the recent past, we would
like to give an introduction using our words, rethink about the original and
introductory papers of this theory, in particular those by \emph{Shapiro}, who
introduces the \emph{Riordan group} and builds a combinatorial triangle
counting \emph{non-intersecting} paths; by \emph{Rogers}, who introduces the
concept of \emph{renewal arrays} and finds the important concept of their
$A$-sequences; by \emph{Eplett}, who provides an identity involving
determinants and \emph{Catalan} numbers; and, finally, by \emph{Sprugnoli}, who
uses Riordan arrays in order to find generating functions of combinatorial sums
in a \emph{constructive} way, not just proving that a sum equals a \emph{given}
value.% (usually denoted by a closed formula). 

\marginpar{\ldots toward a modular arithmetic point of view} The other topic of
this work is the description and formalization of Riordan arrays under the
light of modular arithmetic. We have shown some congruences about \emph{Pascal}
array $\mathcal{P}$ and its inverse $\mathcal{P}^{-1}$. We have also proved a
formal characterization for the \emph{Catalan} array $\mathcal{C}$. These
results were presented in a talk contributed at a recent conference held in
Lecco\footnote{Second International Symposium on Riordan Arrays and Related
Topics, RART$2015$}. All major researchers involved in Riordan group theory
were present and some of them threw some important ideas about possible
enhancement of our results.  

Finally, we have implemented a subset of \emph{Riordan group theory} using the
Python programming language, on top of \emph{Sage} mathematical framework.  Our
implementation is written in \emph{pure object-oriented} style and allows us to
do raw matrix expansion, computing inverse arrays, applying modular
    transformation, using a set of partition functions, and building \LaTeX\,code
    for representing such modular arrays as coloured triangles.
