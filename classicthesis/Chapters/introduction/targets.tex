

\subsection{Looking from a \emph{modular} perspective}

Given a Riodan array $\mathcal{M}$, we are interested to look at it
doing a transformation of the matrix $\lbrace m_{nk}\rbrace_{n,k\in\mathbb{N}}$,
building a new one defined by $\lbrace m_{nk}\mod\,p\rbrace_{n,k\in\mathbb{N}}$,
where $p\in\mathbb{N}$ is a module, usually a \emph{prime} number.

This transformation allows us to represent any array $\mathcal{M}$ pictorially,
associating a colour to each \emph{remainder class} of $\equiv_{p}$ relation:
inspecting the produced coloured triangle it is possible to recognize some
patterns in the colouring and we will see that for same classes of Riordan
arrays, these colourings are pretty interesting, while for other classes they
seems to be quite random.

We will attempt to formally prove what can be see in the pictures, in
particular we will discuss the Pascal array $\mathcal{P}$ and the Catalan array
$\mathcal{C}$.  We believe that this kind of study is important in order to
understand and to make conjectures about the combinatorial structure underlying
those arrays and, why not, any array $\mathcal{M}$ in general.


\begin{figure}[p]

    \noindent\makebox[\textwidth]{
        \centering
        %\includegraphics[width=0.8\textwidth]{../../sympy/catalan/coloured.pdf}

        % using *angle* property to rotate it is difficult to properly align it
        % in order to have a "real" matrix representation.
        \includegraphics[width=9cm, height=9cm, keepaspectratio=true]
            {../RART2015/pascal-tikz/main-idea-four-splitted/plain-numbers.pdf}

        \includegraphics[width=9cm, height=9cm, keepaspectratio=true]
            {../RART2015/pascal-tikz/main-idea-four-splitted/centered-numbers.pdf}

    }

    \vskip1cm

    \noindent\makebox[\textwidth]{
        \centering
        %\includegraphics[width=0.8\textwidth]{../../sympy/catalan/coloured.pdf}

        % using *angle* property to rotate it is difficult to properly align it
        % in order to have a "real" matrix representation.
        \includegraphics[width=9cm, height=9cm, keepaspectratio=true]
            {../RART2015/pascal-tikz/main-idea-four-splitted/remainders.pdf}

        \includegraphics[width=9cm, height=9cm, keepaspectratio=true]
            {../RART2015/pascal-tikz/main-idea-four-splitted/dots.pdf}
    }

    % this 'particular' line is necessary to use `displaymath' environment
    % into the caption environment, togheter with the inclusion of 
    % `caption' package. See here for more explanation:
    % http://stackoverflow.com/questions/2716227/adding-an-equation-or-formula-to-a-figure-caption-in-latex
    \captionsetup{singlelinecheck=off}
    \caption[$\mathcal{P}_{\equiv_{2}}^{(3)}$: from plain matrix to coloured triangle]{
        \emph{Top left} corner: chunk of raw expansion of Pascal array $\mathcal{P}$

        \emph{Top right} corner: the same chunk but using a centered layout, a-l\`a Sierpinski gasket

        \emph{Bottom left} corner: apply modular transformation on the same chunk of coefficient, where $p=2$.
            Associate colour \textcolor{blue}{blue} to $[0]$ 
            remainder class and colour \textcolor{orange}{orange} to $[1]$ remainder class

        \emph{Bottom right} corner: abstract each coefficient $m_{nk}$ with a coloured dot 
            according to $m_{nk}$'s remainder class membership}

    \label{fig:main:idea:modular:characterization}

\end{figure}


In \autoref{fig:main:idea:modular:characterization} is shown this kind of study,
and it will be addressed in \autoref{ch:modular:characterization}.

\subsection{Looking for characterizations, again}

Altough pretty interesting and fairly general characterizations exists for
a coefficient $m_{nk}$ in a Riordan array $\mathcal{M}$, we would like to
rework some of them, looking from different points of view.

In particular we provide a new characterization, called $h$-\emph{characterization},
which allows us to rewrite a Riordan array $\mathcal{M}=(d(t),h(t))$ only
using function $h$, namely $\mathcal{M}_{h(t)}(\alpha(h(t)),h(t))$, for some
function $\alpha$ in the ``variable'' $h(t)$. If the given array $\mathcal{M}$
is in the \emph{Renewal} subgroup, then the $h$-characterization shows the $A$-sequence's
\ac{gf} in the first component of array $\mathcal{M}_{h(t)}$, automagically.

Moreover, we provide another derivation of a concept similar to $A$-matrix characterization,
found by reasoning from a \emph{column}'s \ac{gf} point of view, instead of
focusing on the combination for a generic element $m_{n+1,k+1}$. This study will be
addressed in \autoref{ch:h:characterization}.

\subsection{Python implementation}

In order to support the present work, we have coded a bunch of classes in the
\emph{Python} programming language, that implement a subset of \emph{Riordan
group} theory. We rest on \emph{Sage} \cite{sage} mathematical framework in
order to do hard math computation and our implementation aims to be truly
object oriented and very minimal, in order to be extended easily. 

Nonetheless, it is possible to play with Riordan arrays, inverting them, doing
raw matrix expansions and building a modular representation producing
\LaTeX\,ready code.  All pictures in this document are generated by compilation
of \emph{TikZ} statements, produced by our implementation.

