

\section{Sequences, sequences and sequences}

\subsection{A generalization of the $A$-sequence concept}

Every Riordan array $\mathcal{R}$ has a particular sequence 
$\lbrace a_i\rbrace_{i\in\mathbb{N}}$,
called $A$-sequence (and denote with $A(t)$ the fps over it),
such that uniquely characterize it (to be precise another sequence
$\lbrace z_i\rbrace_{i\in\mathbb{N}}$, called $Z$-sequence, is needed 
to fulfill the very first column, together with $d_{00}$, the root element), 
capturing the way every element $d_{n+1,k+1}$ can be written as a linear combination
of elements lying on the previous row, formally:
\begin{displaymath}
    d_{n+1,k+1} = a_{0}d_{n,k} +a_{1}d_{n,k+1} +a_{2}d_{n,k+2} + 
        \ldots + a_{j}d_{n,k+j}
\end{displaymath}
where $j = n-k$. In this section we would like to offer a generalization
of this ``combinatorial device'', providing a machinery to build sequences 
that combine coefficients, possibly lying on arbitrary rows, as desired.
Finally, a connection with $A$-matrix concept is explored, leaving an open 
question.


What follows doesn't use the alternative characterization developed in 
previous section, we put it here because use a \emph{column oriented}
approach, as used in the characterization. For the sake of simplicity,
however, we start from an array $\mathcal{R}(d(t),h(t))$ and its 
factorization $\mathcal{R}_{h(t)}(g(h(t)), h(t))$, for some function $g$.

\subsubsection{Localizing $A$-sequences}

Consider the following question: there exists a sequence 
$\lbrace \gamma_{i} \rbrace_{i\in\mathbb{N}}$ of coefficients 
such that the generic element $d_{n+1,k+1}\in\mathcal{R}$ can be written
as a linear combination of \emph{all} other elements $d_{n,j}$ 
lying on the previous row, namely for all $j\in\lbrace0,1,2,\ldots,n\rbrace$? 
Later we ask a more general question, for now tackle the current one.

The previous statement can be written in a compact way, or \emph{column-wise}, as:
\begin{displaymath}
    g(h(t))h(t)^k = \sum_{i\geq0}{\gamma_i\,t\,g(h(t)) h(t)^i}\\
\end{displaymath}
To see why, recall that generating function of a generic column 
$k$ is $g(h(t))h(t)^k$ and imagine that series written vertically, 
with increasing degree of $t$ from top to bottom.
The required constraint on coefficient of $t^n$, namely $d_{nk}$, 
to be a linear combination of elements lying on the previous row 
can be satisfied if we \emph{shift downward} every column,
namely multiplying each one by $t$, and combining them using coefficients
from $\lbrace \gamma_{i} \rbrace_{i\in\mathbb{N}}$. 
We learned this trick from Shapiro introductory article \emph{The Riordan group}.

Simplifying $g(h(t))$ shows that using the factored representation or
the natural one doesn't make any difference, therefore:
\begin{displaymath}
    h(t)^k = t \sum_{i\geq 0}{\gamma_i\,h(t)^i} = t \Gamma(h(t))
\end{displaymath}
where function $\Gamma$ is a fps over sequence 
$\lbrace \gamma_{i} \rbrace_{i\in\mathbb{N}}$.
Doing a change of variable to abstract over $h(t)$, its possible to
structure the basic for a generic schema (or \emph{device} if you please):
\begin{displaymath}
    \left[y^{k} = \hat{h}(y) \Gamma(y) \big| y = h(t) \right]
\end{displaymath}

Before going on, just use the previous device against known arrays. Take 
array $\mathcal{C}$, so $\hat{h}_{\mathcal{C}}(y) = y-y^2$, hence:
\begin{displaymath}
    \left[\frac{y^{k-1}}{1-y} =  \Gamma(y) \big| y = h(t) \right]\\
\end{displaymath}
so sequence $\lbrace \gamma_{i} \rbrace_{i\in\mathbb{N}}$ satisfies:
\begin{displaymath}
    \lbrace \gamma_{i} \rbrace_{i\in\mathbb{N}} = 
        \left(\underbrace{\gamma_{0}=0,\ldots,\gamma_{k-2}=0}_{k-1 \text{ zeros}},
            \underbrace{\gamma_{k-1}=1, \ldots}_{\text{infinitely many ones}} \right)
\end{displaymath}
what it says is: ``in order to get $d_{nk}\in\mathcal{C}$ take the sum 
$\sum_{i=k-1}^{n-1}{d_{n-1,i}}$'', as known. Motzkin's turn, 
$\hat{h}_{\mathcal{M}}(y) = \frac{y}{1+y+y^2}$, hence:
\begin{displaymath}
        \left[\Gamma(y)=y^{k-1}+y^{k}+y^{k+1}\big| y = h(t) \right]
\end{displaymath}
what it says is: ``in order to get $d_{nk}\in\mathcal{M}$ take the sum 
$d_{n-1,k-1}+d_{n-1,k}+d_{n-1,k+1}$'', as known.

What we've done is nothing more nothing less than writing $A$-sequences 
in a more generic format, putting evidence on the \emph{local} meaning 
of the combination: it is explicitly written the dependency of 
elements belonging to a column $k$. On the other hand, natural $A$-sequences 
are stated with a fixed start index in mind, namely $k-1$ if combining 
for elements in a column $k$.

It is possible to get the natural $A$-sequence back by requiring that 
combination of elements lying on the previous row starts at index $k-1$, formally:
\begin{displaymath}
    \left[y^{k} = y^{k-1}\,\hat{h}(y)\,\Gamma(y) \big| y = h(t) \right] = 
    \left[y = \hat{h}(y)\,\Gamma(y) \big| y = h(t) \right]
\end{displaymath}
since $h(t) = t\,A(h(t))$, then formal power series 
$\Gamma(y)$ and $A(t)$ are defined over the same sequence, as desired.
\\\\
Here there are a couple of applications. 

What if we ask: find sequences $_{k}\lbrace \gamma_{i} \rbrace_{i\in\mathbb{N}}$ 
such that an element $d_{nk}\in\mathcal{C}$ combines elements lying on 
row three lines above, starting from column index $2$, 
namely $d_{n-3,k}$ for all $k\in\lbrace 2,\ldots,n-3\rbrace$. 
Set the device:
\begin{displaymath}
    \left[y^{k} = y^{2}\,\hat{h}(y)^3\,\Gamma(y) \big| y = h(t) \right] =
        \left[\frac{y^{k-5}}{(1-y)^3} = \Gamma(y) \big| y = h(t) \right]
\end{displaymath}

Two quick checks: in order to find $d_{7,5}$, chosen at random, 
expand function $\Gamma$ with $k=5$:
\begin{displaymath}
    \left.\left[\Gamma(y)=1 + 3y + 6y^2 + 10y^3 + 15y^4 + \mathcal{O}(y^5) 
        \big| y = h(t) \right]\right|_{k=5}
\end{displaymath}
therefore $d_{7,5}=d_{4,2} + 3\,d_{4,3} + 6\,d_{4,4}$, 
instantiating $27 = 9 + 3\cdot4 + 6\cdot1$, which holds.
Just another element, say $d_{9,7}$, so expand function $\Gamma$ with $k=7$:
\begin{displaymath}
    \left.\left[\Gamma(y)=y^2 + 3y^3 + 6y^4 + 10y^5 +  \mathcal{O}(y^6) 
        \big| y = h(t) \right]\right|_{k=7}
\end{displaymath}
therefore $d_{9,7}=d_{6,4} + 3\,d_{6,5} + 6\,d_{6,6}$, 
instantiating $44 = 20 + 3\cdot6 + 6\cdot1$, which holds.
It's interesting to observe the following fact: while a natural $A$-sequence
says how to combine starting always from the previous column index, with
this generalization we got the same coefficients for combining, while
an additional information from what column index the combination should start.

The previous two checks shows exactly this aspect: for column index $k=5$
combination starts at column index $2$, while for column index $k=7$ combination
starts at column index $4$.
\\\\
One last application, before the interlude: 
find a sequence $\lbrace \gamma_{i} \rbrace_{i\in\mathbb{N}}$ such that 
an element $d_{nk}\in\mathcal{M}$ combines \emph{all} elements lying on 
the next row, namely $d_{n+1,k}$ for all $k\in\lbrace0,\ldots,n+1\rbrace$.
The first step is always setting the device:
\begin{displaymath}
    \left[y^{k} = \hat{h}(y)^{-1}\,\Gamma(y) \big| y = h(t) \right]=
        \left[ \frac{y^{k + 1}}{y^2 + y + 1} = \Gamma(y) \big| y = h(t) \right]
\end{displaymath}
in order to find $d_{6,3}$, chosen at random, expand function $\Gamma$ with $k=3$:
\begin{displaymath}
    \left.\left[\Gamma(y)=y^4 -y^5 + y^7 -y^8 +y^{10} + \mathcal{O}(y^{11}) 
        \big| y = h(t) \right]\right|_{k=3}
\end{displaymath}
therefore $d_{6,3}=d_{7,4} - d_{7,5} + d_{7,7}$, instantiating $44 = 70 -27 +1$, 
which holds. Just another element, say $d_{8,1}$, so expand function 
$\Gamma$ with $k=1$:
\begin{displaymath}
    \left.\left[\Gamma(y)=y^2 -y^3 + y^5 -y^6 + y^8 -y^9 + y^{11} + 
        \mathcal{O}(y^{12}) \big| y = h(t) \right]\right|_{k=1}
\end{displaymath}
therefore $d_{8,1}=d_{9,2} - d_{9,3} + d_{9,5}- d_{9,6}+ d_{9,8}- d_{9,9}$, 
instantiating $512 = 1422 -1140 +369 -147 +9 -1$, which holds.
\\\\
As a final remark, under the insights of two solved exercises, is that a
sequence found using this approach, knows which elements to combine and which
ones to discard, we're required to supply the set of available elements only.


\subsubsection{Interlude: let's generalize}

Previous applications shows a pattern that can be pointed out looking at 
the generic device for a Riordan array $\mathcal{R}$ where 
function $h$ is its second component and function $\hat{h}$ its compositional
inverse:
\begin{displaymath}
    \left[y^{k} = \hat{h}(y) \Gamma(y) \big| y = h(t) \right]
\end{displaymath}
on the left hand side of the variable substitution block there's \emph{an
equation}, where function $\Gamma$ is the unknown. In this format, 
a constraint is stated: let $d_{nk}\in\mathcal{R}$ be a generic element, 
then a \emph{localized} sequence 
$\lbrace \gamma_{i} \rbrace_{i\in\mathbb{N}}$ respect column $k$ exists, which
combines all elements lying on the previous row $n-1$.

Since we're dealing with an equation, we can augment it as desired in order to
constrain over additional facts. For instance, consider the following question:
find a sequence $\lbrace \gamma_{i} \rbrace_{i\in\mathbb{N}}$ such that 
an element $d_{nk}\in\mathcal{R}$ combines \emph{all} elements lying on 
\emph{some} row $r_\alpha$, with the additional property to add the combination of 
elements, defined by a sequence $\lbrace \theta_{i} \rbrace_{i\in\mathbb{N}}$, 
lying on \emph{some different} row $r_\beta$. Formally, $\alpha,\beta\in\mathbb{Z}$ 
and $\alpha \not=\beta$, so set the device as usual:
\begin{displaymath}
    \left[y^{k} = \hat{h}(y)^{\alpha} \Gamma(y) + \hat{h}(y)^{\beta} \Theta(y) \big| y = h(t) \right]
\end{displaymath}
Nonetheless its generality, its not the best one: to be truly general,
we should introduce two new parameters $c_\mu$ and $c_\nu$, which allow
to fix the column index from where the combinations denoted by functions
$\Gamma$ and $\Theta$ start, respectively. Here is the most general device:
\begin{displaymath}
    \left[y^{k} = y^{c_\mu}\hat{h}(y)^{\alpha} \Gamma(y) + 
        y^{c_\nu}\hat{h}(y)^{\beta} \Theta(y) \big| y = h(t) \right]
\end{displaymath}

Try to make the generalization at work:
find a sequence $\lbrace \gamma_{i} \rbrace_{i\in\mathbb{N}}$ such that 
an element $d_{nk}\in\mathcal{D}$, in the Delannoy array, 
combines \emph{all} elements lying on 
the next row, namely $n+1$, in addition to the
combination of elements lying two row above, namely $n-2$, 
defined as their sum, simply. 

In order to set the device, we need $\hat{h}_{\mathcal{D}}$:
\begin{displaymath} 
    \hat{h}_{\mathcal{D}}(y) = \frac{\sqrt{1+6y+y^2}-y-1}{2}
\end{displaymath} 
and build function $\Theta$ from the additional requirement:
\begin{displaymath} 
    \Theta(y) = \frac{1}{1-y}
\end{displaymath} 
now we're ready:
\begin{displaymath}
\begin{split}
    &\left[y^{k} = \hat{h}_{\mathcal{D}}(y)^{-1} \Gamma(y) + 
        \hat{h}_{\mathcal{D}}(y)^{2}\frac{1}{1-y} \big| y = h(t) \right]\\
    &= \frac{y^{3} + {\left(y^{2} - 1\right)} y^{k} + 6 \, y^{2} - {\left({\left(y - 1\right)} y^{k} + y^{2} + 3 \, y + 1\right)} \sqrt{y^{2} + 6 \, y + 1} + 6 \, y + 1}{2 \, {\left(1-y\right)}}\\
\end{split}
\end{displaymath}
in order to find $d_{8,1}$, chosen at random, expand function $\Gamma$ with $k=1$:
\begin{displaymath}
    \left.\left[\Gamma(y)=y^2 -3y^3 + 11y^4  -47y^5 + 211y^6 -987y^7 + 4747y^8 
        -23335y^9 + \mathcal{O}(y^{10}) \big| y = h(t) \right]\right|_{k=1}
\end{displaymath}
therefore $d_{8,1}=d_{9,2} -3\,d_{9,3} +11\,d_{9,4}-47\,d_{9,5} 
    +211\,d_{9,6} -987\,d_{9,7} +4747\,d_{9,8}-23335\,d_{9,9}+\epsilon$,
    where $\epsilon = d_{6,0}+d_{6,1}+d_{6,2}+d_{6,3}+d_{6,4}+d_{6,5}+d_{6,6} = 
            2(d_{6,0}+d_{6,1}+d_{6,2})+d_{6,3} = 2(1 + 11 + 41) + 63 = 169$: 
    instantiating $15 = 113 -3\cdot377 +11\cdot681 -47\cdot681 +211\cdot377
        -987\cdot113 +4747\cdot17 -23335 + \epsilon = -154 + 169$, which holds.

It's interesting to observe that function $\Gamma$ above, an algebraic one,
produces sequences that use different coefficients for different values of $k$,
and we observe a curious pattern for increasing values of 
$k\in\lbrace 0,\ldots,10 \rbrace$:
\begin{lenghtydisplaymath}
    \begin{split}
        &\left.\left[\Gamma(y)=
        1 y + {(-2)} y^{2} + 5 y^{3} + {(-17)} y^{4} + 65 y^{5} + {(-273)} y^{6} + 1213 y^{7} + {(-5617)} y^{8} + 26809 y^{9} + \mathcal{O}\left(y^{10}\right)
            \big| y = h(t) \right]\right|_{k=0}\\
        &\left.\left[\Gamma(y)=
        1 y^{2} + {(-3)} y^{3} + 11 y^{4} + {(-47)} y^{5} + 211 y^{6} + {(-987)} y^{7} + 4747 y^{8} + {(-23335)} y^{9} + \mathcal{O}\left(y^{10}\right)
            \big| y = h(t) \right]\right|_{k=1}\\
        &\left.\left[\Gamma(y)=
        3 y^{4} + {(-19)} y^{5} + 99 y^{6} + {(-503)} y^{7} + 2547 y^{8} + {(-12971)} y^{9} + \mathcal{O}\left(y^{10}\right)
            \big| y = h(t) \right]\right|_{k=2}\\
        &\left.\left[\Gamma(y)=
        {(-1)} y^{3} + 6 y^{4} + {(-27)} y^{5} + 127 y^{6} + {(-615)} y^{7} + 3031 y^{8} + {(-15171)} y^{9} + \mathcal{O}\left(y^{10}\right)
            \big| y = h(t) \right]\right|_{k=3}\\
        &\left.\left[\Gamma(y)=
        {(-1)} y^{3} + 5 y^{4} + {(-24)} y^{5} + 119 y^{6} + {(-587)} y^{7} + 2919 y^{8} + {(-14687)} y^{9} + \mathcal{O}\left(y^{10}\right)
            \big| y = h(t) \right]\right|_{k=4}\\
        &\left.\left[\Gamma(y)=
        {(-1)} y^{3} + 5 y^{4} + {(-25)} y^{5} + 122 y^{6} + {(-595)} y^{7} + 2947 y^{8} + {(-14799)} y^{9} + \mathcal{O}\left(y^{10}\right)
            \big| y = h(t) \right]\right|_{k=5}\\
        &\left.\left[\Gamma(y)=
        {(-1)} y^{3} + 5 y^{4} + {(-25)} y^{5} + 121 y^{6} + {(-592)} y^{7} + 2939 y^{8} + {(-14771)} y^{9} + \mathcal{O}\left(y^{10}\right)
            \big| y = h(t) \right]\right|_{k=6}\\
        &\left.\left[\Gamma(y)=
        {(-1)} y^{3} + 5 y^{4} + {(-25)} y^{5} + 121 y^{6} + {(-593)} y^{7} + 2942 y^{8} + {(-14779)} y^{9} + \mathcal{O}\left(y^{10}\right)
            \big| y = h(t) \right]\right|_{k=7}\\
        &\left.\left[\Gamma(y)=
        {(-1)} y^{3} + 5 y^{4} + {(-25)} y^{5} + 121 y^{6} + {(-593)} y^{7} + 2941 y^{8} + {(-14776)} y^{9} + \mathcal{O}\left(y^{10}\right)
            \big| y = h(t) \right]\right|_{k=8}\\
        &\left.\left[\Gamma(y)=
        {(-1)} y^{3} + 5 y^{4} + {(-25)} y^{5} + 121 y^{6} + {(-593)} y^{7} + 2941 y^{8} + {(-14777)} y^{9} + \mathcal{O}\left(y^{10}\right)
            \big| y = h(t) \right]\right|_{k=9}\\
        &\left.\left[\Gamma(y)=
        {(-1)} y^{3} + 5 y^{4} + {(-25)} y^{5} + 121 y^{6} + {(-593)} y^{7} + 2941 y^{8} + {(-14777)} y^{9} + \mathcal{O}\left(y^{10}\right)
            \big| y = h(t) \right]\right|_{k=10}\\
    \end{split}
\end{lenghtydisplaymath}

From $k=3$, it seems that a coefficient stabilizes in the expansion, pretty nice.
To finish this section involving Delannoy triangle, we report its natural
$A$-sequence:
\begin{lenghtydisplaymath}
    \left.\left[\Gamma(y)=
    1 + 2 y + {(-2)} y^{2} + 6 y^{3} + {(-22)} y^{4} + 90 y^{5} + {(-394)} y^{6} + 1806 y^{7}  + \mathcal{O}\left(y^{8}\right)
        \big| y = h(t) \right]\right|_{k=1}\\
\end{lenghtydisplaymath}

\subsubsection{$A$-matrix connection}

Is the previous device the most general one? Really is it? 
We lie. In this section we enhance the last version to 
find an interesting connection with 
the $A$-matrix concept of a Riordan array $\mathcal{R}$, 
introduced and developed by Merlini, Rogers, Sprugnoli and Verri.
\\\\
Let $\lbrace\Omega_{i}\rbrace_{i\in\mathbb{N}}$ be a collection of formal
power series and assume we would like to \emph{not} localize them (a-l\`a natural 
$A$-sequence) in order to combine elements lying on the previous row, the previous
but one and so on \ldots hence, respect an element $d_{nk}\in\mathcal{R}$,
set the following device:
\begin{displaymath}
    \left.\left[
            \begin{split}
                y^{k} &= y^{k-1}\hat{h}(y) \Omega_{0}(y) + 
                y^{k-1}\hat{h}(y)^{2} \Omega_{1}(y) \\
                &+ y^{k-1}\hat{h}(y)^{3} \Omega_{2}(y) +
                \ldots +
                y^{k-1}\hat{h}(y)^{i+1} \Omega_{i}(y) + \ldots
            \end{split}
        \right| y = h(t) \right]
\end{displaymath}
a simplification rewrites:
\begin{displaymath}
    \left.\left[
        y = \hat{h}(y) \Omega_{0}(y) + 
        \hat{h}(y)^{2} \Omega_{1}(y) + \hat{h}(y)^{3} \Omega_{2}(y) +
        \ldots +
        \hat{h}(y)^{i+1} \Omega_{i}(y) + \ldots
        \right| y = h(t) \right]
\end{displaymath}
Here are our thoughts:
\begin{itemize}
    \item it can be seen as an additional generalization of the last device,
        where some new functions are thrown in, in order to augment the
        combination including sets of coefficients lying on more rows, 
        each of them combined according a new introduced function $\Omega_{j}$,
        which are \emph{explicitly given}.  As before,
        you are interested to find one such function, say $\Omega_{0}$,
        and you can really do it since it is the same to solve a system
        with one equation in one unknown;
    \item it provides a ``factorization'' of function $A(t)$ with coefficients
        over $\mathcal{R}$'s $A$-sequence, respect function $\hat{h}$, 
        the compositional inverse of function $h$.  
        Recognizing $h(t)=tA(h(t))$ yield:
        \begin{displaymath}
            \left.\left[
                A(y) =  \Omega_{0}(y) + 
                \hat{h}(y)\,\Omega_{1}(y) + \hat{h}(y)^{2}\,\Omega_{2}(y) + \ldots +
                \hat{h}(y)^{i}\,\Omega_{i}(y) + \ldots
                \right| y = h(t) \right]
        \end{displaymath}
        This is quite interesting from the theoretical point of 
        view but can have a practical application
        when every term is in the polynomial ring: if this is the case,
        can apply the \emph{division theorem} among polynomial and 
        proceed to factor polynomial $A$ dividing it by polynomial $\hat{h}$,
        repeatedly.

        For the sake of clarity, \added{consider the following cases:}
        \begin{itemize}
        \item \replaced{let $\mathcal{P}$ be the}{consider} 
            Pascal array $\mathcal{P}$ \replaced{and recall the following facts:}{
            : recall that $\hat{h}(y)=\frac{y}{1+y}$ and $A(y)=1+y$}
        \begin{displaymath} 
            \added{
            \hat{h}_{\mathcal{P}}(y)=\frac{y}{1+y} \quad\quad A_{\mathcal{P}}(y)=1+y
            }
        \end{displaymath} 
        therefore, by \emph{division theorem}, 
        there exist polynomials $\Omega_{0}$ and \replaced{$\Delta_{0}$}{
            $\Omega_{1}$} such that:
        \begin{displaymath}
            \left.\left[
                (1+y)^2 =  (1+y)\Omega_{0}(y) + y\,\replaced{\Delta_{0}}{
            \Omega_{1}}(y) \right| y = \replaced{h_{\mathcal{P}}}{h}(t) \right]
        \end{displaymath}
        dividing $(1+y)^2$ by $y$ yield \replaced{$\Delta_{0}$}{
            $\Omega_{1}$}$(y)=2+y$ and $(1+y)\Omega_{0}(y)=1$ \added{, 
            this define polynomial $\Omega_{0}$}.  
        \added{We need to 
        keep applying \emph{division theorem}, so there exist the following
        sequences of polynomials such that:}
        \begin{lenghtydisplaymath}
            \begin{split} 
                &
                \added{
                \left.\left[
                    \frac{\Delta_{0}(y)}{\hat{h}_{\mathcal{P}}(y)} = 
                        \left(y+3, 2\right)\triangleq
                        \left(\Delta_{1}(y), (1+y)\Omega_{1}(y) \right)
                     \right| y = h_{\mathcal{P}}(t) \right]
                }\\
                &
                \added{
                \left.\left[
                    \frac{\Delta_{1}(y)}{\hat{h}_{\mathcal{P}}(y)} = 
                        \left(y+4, 3\right)\triangleq
                        \left(\Delta_{2}(y), (1+y)\Omega_{2}(y) \right)
                     \right| y = h_{\mathcal{P}}(t) \right]
                }\\
                &
                \added{
                \left.\left[
                    \frac{\Delta_{2}(y)}{\hat{h}_{\mathcal{P}}(y)} = 
                        \left(y+5, 4\right)\triangleq
                        \left(\Delta_{3}(y), (1+y)\Omega_{3}(y) \right)
                     \right| y = h_{\mathcal{P}}(t) \right]
                }\\
                &
                \added{
                \left.\left[
                    \frac{\Delta_{3}(y)}{\hat{h}_{\mathcal{P}}(y)} = 
                        \left(y+6, 5 \right)\triangleq
                        \left(\Delta_{4}(y), (1+y)\Omega_{4}(y) \right)
                     \right| y = h_{\mathcal{P}}(t) \right]
                }\\
                &\vdots
            \end{split} 
        \end{lenghtydisplaymath}
        \added{A pattern seems to emerge and can be captured with the 
        following rule in order to define polynomial $\Omega_{i}$:}
        \begin{displaymath} 
            \added{
                \left.\left[
                    \Delta_{i-1}(y) \triangleq q_{i-1}(y) + r_{i-1}
                    \rightarrow (1+y)\,\Omega_{i}(y)=r_{i-1}
                     \right| y = h_{\mathcal{P}}(t) \right]
            }
        \end{displaymath} 
        \added{where each polynomial $q_{j}$ satisfies $q_{j}(0)=0$ and
        each $r_{j}\in\mathbb{N}$ is a remainder, with boundary initial 
        condition $\Delta_{-1}(y)=1$. Therefore it is possible to state
        a closed formula for polynomial $\Omega_{i}$:} 
        \begin{displaymath} 
            \added{\Omega_{i}(y)=\frac{1+i}{1+y}}
        \end{displaymath} 

        Putting it all together, the factorization of polynomial 
        \replaced{$A_{\mathcal{P}}$}{$A$} respect polynomial 
        $\replaced{\hat{h}_{\mathcal{P}}}{\hat{h}}$ is:
        \begin{displaymath}
            \replaced{
                \left.\left[
                    A_{\mathcal{P}}(y) = \sum_{i \geq0}{\left(\frac{1+i}{1+y}\right)
                        \hat{h}_{\mathcal{P}}(y)^{i}} \right| y = h_{\mathcal{P}}(t) \right]
            }{
                \left.\left[
                    A(y) =  \frac{1}{1+y} + \hat{h}(y)(2+y) \right| y = h(t) \right]
            }
        \end{displaymath}

        \item
        \added{for Catalan array $\mathcal{C}$ things are quite interesting,
        first of all recall the following facts:}
        \begin{displaymath} 
            \added{
            \hat{h}_{\mathcal{C}}(y)=y-y^2 \quad\quad 
                A_{\mathcal{C}}(y)=\frac{1}{1-y}
            }
        \end{displaymath} 
        \added{therefore, by \emph{division theorem}, 
        there exist polynomials $\Omega_{0}$ and $\Delta_{0}$ such that:}
        \begin{displaymath}
            \added{
            \left.\left[
                1 = (1-y)\Omega_{0}(y) + y(1-y)^{2}\,\Delta_{0}(y) 
                    \right| y = h_{\mathcal{C}}(t) \right]
            }
        \end{displaymath}
        \added{dividing $1$ by $y(1-y)^2$ yield $\Delta_{0}(y)=0$ 
        and $(1-y)\Omega_{0}(y)=1$, this define polynomial $\Omega_{0}$. 
        This means that ``polynomial'' $A_{\mathcal{C}}$ is already
        the factorization of itself, which is the same to say that
        there exists a \emph{unique} $A$-matrix for $\mathcal{C}$, pretty curious;}

        \item
        \added{for Motzkin array $\mathcal{M}$ things are quite interesting,
        first of all recall the following facts:}
        \begin{displaymath} 
            \added{
            \hat{h}_{\mathcal{M}}(y)=\frac{y}{1+y+y^2} \quad\quad 
                A_{\mathcal{M}}(y)=1+y+y^2
            }
        \end{displaymath} 
        \added{therefore, by \emph{division theorem}, 
        there exist polynomials $\Omega_{0}$ and $\Delta_{0}$ such that:}
        \begin{displaymath}
            \added{
            \left.\left[
                (1+y+y^2)^2 = (1+y+y^2)\Omega_{0}(y) + y\,\Delta_{0}(y) 
                    \right| y = h_{\mathcal{M}}(t) \right]
            }
        \end{displaymath}
        \added{dividing $(1+y+y^2)^2$ by $y$ yield $\Delta_{0}(y)=2+3y+2y^2+y^3$ 
        and $(1+y+y^2)\Omega_{0}(y)=1$, this define polynomial $\Omega_{0}$,
        which expanded as fps equals: }
        \begin{displaymath}
            \added{
            \left.\left[
                \Omega_{0}(y) = 1 -y +y^{3} -y^{4} + y^{6} 
                    -y^{7} + y^{9} -y^{10} 
                    + y^{12} + \mathcal{O}\left(y^{13}\right)
                    \right| y = h_{\mathcal{M}}(t) \right]
            }
        \end{displaymath}
        \added{We need to 
        keep applying \emph{division theorem}, so there exist the following
        sequences of polynomials such that:}
        \begin{lenghtydisplaymath}
            \begin{split} 
                &
                \added{
                \left.\left[
                    \frac{\Delta_{0}(y)}{\hat{h}_{\mathcal{M}}(y)} = 
                        \left(y^4 + 3y^3 + 6y^2 + 7y + 5, 2\right)\triangleq
                        \left(\Delta_{1}(y), (1+y+y^2)\Omega_{1}(y) \right)
                     \right| y = h_{\mathcal{M}}(t) \right]
                }\\
                &
                \added{
                \left.\left[
                    \frac{\Delta_{1}(y)}{\hat{h}_{\mathcal{M}}(y)} = 
                        \left(y^5 + 4y^4+10y^3+16y^2 + 18y + 12, 5\right)\triangleq
                        \left(\Delta_{2}(y), (1+y+y^2)\Omega_{2}(y) \right)
                     \right| y = h_{\mathcal{M}}(t) \right]
                }\\
                &
                \added{
                \left.\left[
                    \frac{\Delta_{2}(y)}{\hat{h}_{\mathcal{M}}(y)} = 
                        \left(y^6 + 5y^5 + 15y^4 + 30y^3 + 44y^2+46y+30, 
                            12\right)\triangleq
                        \left(\Delta_{3}(y), (1+y+y^2)\Omega_{3}(y) \right)
                     \right| y = h_{\mathcal{M}}(t) \right]
                }\\
                &
                \added{
                \left.\left[
                    \frac{\Delta_{3}(y)}{\hat{h}_{\mathcal{M}}(y)} = 
                        \left(y^7+6y^6+21y^5+50y^4+89y^3+120y^2+120y+76, 30
                            \right)\triangleq
                        \left(\Delta_{4}(y), (1+y+y^2)\Omega_{4}(y) \right)
                     \right| y = h_{\mathcal{M}}(t) \right]
                }\\
                &\vdots
            \end{split} 
        \end{lenghtydisplaymath}
        \added{A pattern seems to emerge and can be captured with the 
        following rule in order to define polynomial $\Omega_{i}$:}
        \begin{displaymath} 
            \added{
                \left.\left[
                    \Delta_{i-1}(y) \triangleq q_{i-1}(y) + r_{i-1}
                    \rightarrow (1+y+y^2)\,\Omega_{i}(y)=r_{i-1}
                     \right| y = h_{\mathcal{M}}(t) \right]
            }
        \end{displaymath} 
        \added{where each polynomial $q_{j}$ satisfies $q_{j}(0)=0$ and
        each $r_{j}\in\mathbb{N}$ is a remainder, with boundary initial 
        condition $\Delta_{-1}(y)=1$.}

        \added{It is quite interesting that, as the case for array $\mathcal{P}$,
        the sequence $\lbrace r_{j} \rbrace_{j\in\mathbb{N}}$ of remainders
        is exactly the sequence of coefficients of the function defining
        the second column of array $\mathcal{M}$, namely function 
        $d_{\mathcal{M}}$ times function $h_{\mathcal{M}}$, shifted by \emph{one}
        position. Putting it all together, the factorization of polynomial 
        $A_{\mathcal{M}}$ respect polynomial $\hat{h}_{\mathcal{M}}$ is:}
        \begin{displaymath}
            \added{
                \left.\left[
                    A_{\mathcal{M}}(y) = \sum_{i \geq0}{
                        \left(\frac{[t^{1+i}]d_{\mathcal{M}}(t)h_{\mathcal{M}}(t)}
                            {1+y+y^2}\right)
                        \hat{h}_{\mathcal{M}}(y)^{i}} 
                        \right| y = h_{\mathcal{M}}(t) \right]
            }
        \end{displaymath}

        \item if terms are not in a polynomial ring, we've difficulty to show a
        factorization: \replaced{for example Delannoy array $\mathcal{D}$ is}
        {both Catalan array $\mathcal{C}$ both Delannoy array
        $\mathcal{D}$ are} affected by this difficulty.

        \end{itemize}
        
    \item looking at the formulation under study, we observe that functions
        $\Omega_{0}, \Omega_{1}, \Omega_{2}, \ldots$ are exactly the same
        as those introduced by Merlini et al., therefore we've expressed
        the concept of $A$-matrix from a different point of view. A problem
        is still open: what if we would like to find all such functions? A reply
        to this question seems interesting but we've no idea how to satisfy it.
        The major difficulty can be spotted looking at the device as a system:
        there's \emph{one} equation in possibly $k$ unknowns, no
        idea about the shape of the remaining $k-1$ equations.

        Have a little check about Pascal array $\mathcal{P}$ again:
        let $d_{nk}\in\mathcal{P}$, from the point above we've, firstly,
        $\Omega_{0}(y)=\frac{1}{1+y}$, which says to sum elements
        \added{lying} on row $n-1$ using alternating sings
        \replaced{; then add the doubled sum of
        elements lying on row $n-2$ using alternating signs; 
        then add the tripled sum of elements lying on row $n-3$ 
        using alternating signs; and so on \ldots}
        { and, secondly, $\Omega_{1}(y)=2+y$, 
        which says to double the ``first'' element and sum to the ``second'' element
        on row $n-2$}, all respect column index $k$. 
        
        Element $d_{7,4}$, chosen at random, is the combination 
        \replaced{$(d_{6,3}-d_{6,4}+d_{6,5}-d_{6,6})+
            2(d_{5,3}-d_{5,4}+d_{5,5}) + 3(d_{4,3}-d_{4,4}) + 4d_{4,4}$, 
            instantiating $35 = (20-15+6-1)+2(10-5+1)+3(4-1)+4\cdot1$
            }{
        $d_{6,3}-d_{6,4}+d_{6,5}-d_{6,6}+2d_{5,3}+d_{5,4}$, instantiating
        $35 = 20-15+6-1+2\cdot10+5$
        } which holds.

        \deleted[remark=boring and tedious the combination for $d_{8,1}$]{
            Element $d_{8,1}$, chosen at random, is the combination
            $d_{7,0}-d_{7,1}+d_{7,2}-d_{7,3}+d_{7,4}-d_{7,5}+d_{7,6}-d_{7,7}
                +2d_{7,0}+d_{7,1}$, instantiating 
                $8 = 1-7+21-35+35-21+7-1+2\cdot1+6$ which holds.}
        

        \added{On the other hand, let $d_{6,2}\in\mathcal{M}$ be the
        combination $(d_{5,1}-d_{5,2}+d_{5,4}-d_{5,5})
            +2(d_{4,1}-d_{4,2}+d_{4,4})
            +5(d_{3,1}-d_{3,2})
            +12(d_{2,1} -d_{2,2})
            +30\,d_{1,1}$, instantiating: 
            $69=(30-25+5-1)
            +2(12-9+1)
            +5(5-3)
            +12(2 -1)
            +30\cdot1$, which holds.}


\end{itemize}


