
\section{Main idea}

Consider a \emph{Riordan array} $\mathcal{R}\left(d(t),h(t)\right)$ 
defined over generating functions $d$ and $h$. 
By definition, coefficients lying on 
column $k$ are the coefficients of the following combination:
\begin{displaymath}
    d(t)h(t)^k
\end{displaymath}
our idea is to characterize $\mathcal{R}$ doing a variable change, as the
following manipulation catches:
\begin{displaymath}
    d(t)h(t)^k = d(t)(1 + (h(t)-1))^k = \left[ \left. d(\hat{h}(1+y))(1+y)^k \right|y = h(t)-1  \right]
\end{displaymath}
where function $\hat{h}$ is the compositional inverse of $h$, the one that
satisfies $\hat{h}(h(t)) = t$: in order to get $t$ back, apply $\hat{h}$ to
both members of $y+1 = h(t)$.
Inside the square brackets there's a shape of a Riordan array, therefore:
\begin{displaymath}
    \begin{split}
        \mathcal{R}\left(d(t),h(t)\right) &= \left[ \mathcal{R}\left(d(\hat{h}(1+y)), 1+y\right) \left| y = h(t)-1 \right. \right]\\
        &= \mathcal{R}_{y=h(t)-1}\left( f(y), 1+y \right) =  \mathcal{R}_{h(t)}\left( g(h(t)), h(t) \right) 
    \end{split}
\end{displaymath}
$\mathcal{R}_{y=h(t)-1}$ is interesting since its first component $f(y)$ allows to 
develop a new array $\mathcal{R}_{h(t)}$ where it's first component
is a function $g$ in the ``variable'' $h(t)$, eventually the moral is:
\begin{quote}
    \graffito{$h$-characterization}
    \emph{function composition $d$ with $\hat{h}$ yields array $\mathcal{R}_{h(t)}$,\\
        the $h$-characterization of $\mathcal{R}$, which depends on function $h$ only }
\end{quote}
It's important to understand the meaning of those new symbols:
\begin{itemize}
    \item $\mathcal{R}_{\gamma=\Omega(t)}$ denotes an array where a variable
        substitution has to take place, namely $\Omega(t)$ is the substitution 
        for variable $\gamma$;
        \marginpar{$\mathcal{R}_{t}(d(t),h(t))$ denotes verbosly, but correctly, the same
            array $\mathcal{R}(d(t),h(t))$ denotes}
    \item $\mathcal{R}_{\Omega(t)}$ denotes an array where $\Omega(t)$ should 
        be seen as a variable, like a \emph{datum}, and should not be expanded,
        even if $\Omega$'s definition is known. 
        % the same concept can be expressed as follow:
            %(a point very important to understand: here $h(t)$ shouldn't be
            %interpreted as a \emph{function}, the one in $\mathcal{R}$'s definition, instead
            %abstract over it and consider it as a \emph{variable}).
        % this has been cut from a paragraph within `Group operations, revisited' section
\end{itemize}

\label{par:h:characterization:is:an:array:polymorphism} Moreover, the
$h$-characterization $\mathcal{R}_{h(t)}\left(g(h(t)),h(t)\right)$,
\marginpar{$\mathcal{R}_{h(t)}$ is polymorphic} for some function $g$, can be
used as a classic Riordan array, %\emph{trasparently}, 
where traditional function $d$ satisfies $d(t) = g(h(t))$. 
On the other hand, function $h$ is interpreted as usual.


\subsection{Applying it to known triangles}

In the following sections we characterize some well known 
arrays under this development.

\subsubsection{Pascal}
Let $\mathcal{P}$ be the Riordan array for the Pascal triangle,
defined as:
\begin{displaymath} 
    \mathcal{P} = \left(\frac{1}{1-t}, \frac{t}{1-t}  \right)
\end{displaymath} 
computing the compositional inverse of $h$ yields:
\begin{displaymath} 
    \hat{h}(y) = \frac{y}{1+y}
\end{displaymath} 
so $f(y)=d(\hat{h}(1+y))=2+y$ therefore:
\begin{displaymath} 
    \mathcal{P}_{y=h(t)-1}\left( 2+y, 1+y \right)= \mathcal{P}_{h(t)}\left( 1+h(t), h(t) \right)
\end{displaymath} 
Now study the generating function which carries coefficients lying on column $k$
of $\mathcal{P}_{h(t)}$:
\marginpar{this recall the Rogers' work about \emph{renewal arrays}, 
which we described in \autoref{sec:back:to:the:basics:rogers}} 
\begin{displaymath} 
    h(t)^k + h(t)^{k+1}
\end{displaymath} 
hence column $k$ is the sum of $k$-fold and $(k+1)$-fold convolutions 
of function $h$ with itself.

\subsubsection{Fibonacci}
Let $\mathcal{F}$ be the Riordan array for the Fibonacci triangle,
defined as:
\begin{displaymath} 
    \mathcal{F} = \left(\frac{1}{1-t-t^2}, \frac{1-\sqrt{1-4t}}{2}  \right)
\end{displaymath} 
computing the compositional inverse of $h$ yields:
\begin{displaymath} 
    \hat{h}(y) = y - y^2
\end{displaymath} 
so $f(y)=d(\hat{h}(1+y))=\frac{1}{1+y-2y^3-y^4}$ therefore:
\begin{displaymath} 
    \begin{split} 
        & \mathcal{F}_{y=h(t)-1}\left( \frac{1}{1+y-2y^3-y^4}, 1+y \right) = \mathcal{F}_{h(t)}\left( \frac{1}{1-h(t)+2h(t)^3-h(t)^4}, h(t) \right)\\
    \end{split} 
\end{displaymath} 
Now study the generating function which carries coefficients lying on column $k$
of $\mathcal{F}_{h(t)}$:
\begin{displaymath} 
    \frac{h(t)^k}{1-h(t)+2h(t)^3-h(t)^4}
\end{displaymath} 

\subsubsection{Catalan}
Let $\mathcal{C}$ be the Riordan array for the Catalan triangle,
defined as:
\begin{displaymath} 
    \mathcal{C} = \left(\frac{1-\sqrt{1-4t}}{2t}, \frac{1-\sqrt{1-4t}}{2}  \right)
\end{displaymath} 
computing the compositional inverse of $h$ yields:
\begin{displaymath} 
    \hat{h}(y) = y - y^2
\end{displaymath} 
so $f(y)=d(\hat{h}(1+y))=\frac{\sqrt{4 \, y^{2} + 4 \, y + 1} - 1}{2 \, {\left(y + 1\right)} y}$ therefore:
\marginpar{$h_{\mathcal{F}}=h_{\mathcal{C}}$, here function $d$ makes the difference}
\begin{displaymath} 
    \begin{split} 
        &\mathcal{C}_{y=h(t)-1}\left(\frac{\sqrt{4 \, y^{2} + 4 \, y + 1} - 1}{2 \, {\left(y + 1\right)} y}, 1+y \right) \\
        &= \mathcal{C}_{h(t)}\left(\frac{\sqrt{4 \, h\left(t\right)^{2} - 4 \, h\left(t\right) + 1} - 1}{2 \, {\left(h\left(t\right) - 1\right)} h\left(t\right)}, h(t) \right)\\
        &= \mathcal{C}_{h(t)}\left(\frac{\sqrt{(2 h\left(t\right) -  1)^2} - 1}{2 \, {\left(h\left(t\right) - 1\right)} h\left(t\right)}, h(t) \right)\\
    \end{split} 
\end{displaymath} 
By cases on $\sqrt{(2 h\left(t\right) -  1)^2}$:
\marginpar{$\sqrt{x^{2}}=\pm x$ trick}
\begin{itemize}
    \item $\sqrt{(2 h\left(t\right) -  1)^2}=2 h\left(t\right) -  1$, hence:
        \begin{displaymath} 
            \mathcal{C}_{h(t)}\left(\frac{2(h\left(t\right) -  1)}{2 \, {\left(h\left(t\right) - 1\right)} h\left(t\right)}, h(t) \right)=
            \mathcal{C}_{h(t)}\left(\frac{1}{h\left(t\right)}, h(t) \right)
        \end{displaymath} 
        since $h(0)=0$ the previous result has no meaning, therefore discard this case;
    \item $\sqrt{(2 h\left(t\right) -  1)^2}=1 -2 h\left(t\right)$, hence:
        \begin{displaymath} 
            \mathcal{C}_{h(t)}\left(\frac{h\left(t\right)}{ {\left(1-h\left(t\right) \right)} h\left(t\right)}, h(t) \right)=
            \mathcal{C}_{h(t)}\left(\frac{1}{1-h\left(t\right)}, h(t) \right)
        \end{displaymath} 
        checking against $h(0)=0$ the previous result has meaning, therefore accept this case.
\end{itemize}
Now study the generating function which carries coefficients lying on column $k$
of $\mathcal{C}_{h(t)}$:
\begin{displaymath} 
    \frac{h(t)^{k}}{1-h\left(t\right)} 
\end{displaymath} 


\subsubsection{Motzkin, classic version}

Let $\mathcal{M}$ be the Riordan array for the Motzkin triangle, defined as:
\begin{displaymath} 
    \mathcal{M} =\left( \frac{1-t-\sqrt{1-2t-3t^2}}{2t^2}, \frac{1-t-\sqrt{1-2t-3t^2}}{2t}  \right)
\end{displaymath} 
computing the compositional inverse of $h$ yields:
\begin{displaymath} 
    \hat{h}(y) = \frac{y}{1+y+y^2}
\end{displaymath} 
so $f(y)=d(\hat{h}(1+y))$ equals:
\begin{lenghtydisplaymath} 
    \begin{split} 
        & -\frac{{\left(y^{2} \sqrt{\frac{{\left(y^{2} + 4 \, y + 4\right)} y^{2}}{{\left(y^{2} + 3 \, y + 3\right)}^{2}}} - y^{2} + 3 \, y \sqrt{\frac{{\left(y^{2} + 4 \, y + 4\right)} y^{2}}{{\left(y^{2} + 3 \, y + 3\right)}^{2}}} - 2 \, y + 3 \, \sqrt{\frac{{\left(y^{2} + 4 \, y + 4\right)} y^{2}}{{\left(y^{2} + 3 \, y + 3\right)}^{2}}} - 2\right)} {\left(y^{2} + 3 \, y + 3\right)}}{2 \, {\left(y + 1\right)}^{2}} \\
        & = -\frac{{ \left(\sqrt{\frac{{\left(y^{2} + 4 \, y + 4\right)} y^{2}}{{\left(y^{2} + 3 \, y + 3\right)}^{2}}} \left( y^{2}+3y+3 \right) -\left(y^{2}+2y+2\right)\right){\left(y^{2} + 3 \, y + 3\right)}}}{2 \, {\left(y + 1\right)}^{2}} \\
        & = -\frac{ \left(\sqrt{\left(y^{2} + 4 y + 4\right) y^{2}} -\left(y^{2}+2y+2\right)\right)\left(y^{2} + 3y + 3\right)}{2 {\left(y + 1\right)}^{2}} \\
    \end{split} 
\end{lenghtydisplaymath} 
Recognize a perfect square $y^{2} + 4 y + 4 = (y+2)^{2}$:
\marginpar{we could also stop here and reason by cases about $\pm\left(\left(y + 2\right) y\right)$
    without going on substituting variable $y$ too soon\ldots}
\begin{lenghtydisplaymath} 
        -\frac{ \left(\sqrt{\left(\left(y + 2\right) y\right)^{2}} -\left(y^{2}+2y+2\right)\right)\left(y^{2} + 3y + 3\right)}{2 {\left(y + 1\right)}^{2}} 
\end{lenghtydisplaymath} 
therefore:
\begin{lenghtydisplaymath} 
    \begin{split} 
        &\mathcal{M}_{y=h(t)-1}\left(
            -\frac{ \left(\sqrt{\left(\left(y + 2\right) y\right)^{2}} -\left(y^{2}+2y+2\right)\right)\left(y^{2} + 3y + 3\right)}{2 {\left(y + 1\right)}^{2}} , 1+y \right) \\
        &= \mathcal{M}_{h(t)}\left(
        -\frac{{\left(\sqrt{\frac{{\left(h(t)^{2} + 2 \, h(t) + 1\right)} {\left(h(t) - 1\right)}^{2}}{{\left(h(t)^{2} + h(t) + 1\right)}^{2}}}\left( h(t)^{2} + h(t) + 1\right) 
            - (h(t)^{2} + 1) \right)} {\left(h(t)^{2} + h(t) + 1\right)}}{2 \, h(t)^{2}} , h(t) \right)\\
        &= \mathcal{M}_{h(t)}\left(
        -\frac{{\left(\sqrt{\left(h(t)^2 - 1\right)^{2}} - (h(t)^{2} + 1) \right)} {\left(h(t)^{2} + h(t) + 1\right)}}{2 \, h(t)^{2}} , h(t) \right)\\
    \end{split} 
\end{lenghtydisplaymath} 
By cases on $\sqrt{\left(h(t)^2 - 1\right)^{2}}$:
\begin{itemize}
    \item $\sqrt{\left(h(t)^2 - 1\right)^{2}}= h(t)^2 - 1$, hence:
        \begin{displaymath} 
            \mathcal{M}_{h(t)}\left(\frac{ h(t)^{2} + h(t) + 1}{h(t)^{2}} , h(t) \right)\\
        \end{displaymath} 
        since $h(0)=0$ the previous result has no meaning, therefore discard this case;
    \item $\sqrt{\left(h(t)^2 - 1\right)^{2}}= 1-h(t)^2$, hence:
        \begin{displaymath} 
            \mathcal{M}_{h(t)}\left( h(t)^{2} + h(t) + 1, h(t) \right)\\
        \end{displaymath} 
        checking against $h(0)=0$ the previous result has meaning, therefore accept this case.
\end{itemize}

It is interesting to note that the tedious derivation developed above can have
a simpler handling if we defer substituting $h(t)-1$ for $y$, consider the following \ldots
by cases on $\sqrt{\left(\left(y + 2\right) y\right)^{2}}$:
\begin{itemize}
    \item $\sqrt{\left(\left(y + 2\right) y\right)^{2}}=\left(y + 2\right) y$, hence:
        \begin{displaymath} 
            \mathcal{M}_{y=h(t)-1}\left(\frac{y^{2} + 3y + 3}{{\left(y + 1\right)}^{2}} , 1+y \right) = 
                \mathcal{M}_{h(t)}\left( \frac{1+h(t)+h(t)^2}{h(t)^2}, h(t) \right) 
        \end{displaymath} 
        since $h(0)=0$ the previous result has no meaning, therefore discard this case;
    \item $\sqrt{\left(\left(y + 2\right) y\right)^{2}}=-\left(y + 2\right) y$, hence:
        \begin{displaymath} 
            \mathcal{M}_{y=h(t)-1}\left(y^{2} + 3y + 3 , h(t) \right) = 
                \mathcal{M}_{h(t)}\left( 1+h(t)+h(t)^2, h(t) \right) 
        \end{displaymath} 
        checking against $h(0)=0$ the previous result has meaning, therefore accept this case.
\end{itemize}

In both branches same definitions for $\mathcal{M}_{h(t)}$ are reached applying
same checks, while at the same time substituting variable $y$ is easier. 
\\\\
However, here it is the generating function which carries coefficients 
lying on column $k$ of $\mathcal{M}_{h(t)}$:
\begin{displaymath} 
    h(t)^{k}+h(t)^{k+1}+h(t)^{k+2}
\end{displaymath} 


\subsubsection{Motzkin, $\mathcal{T}$ variant}

Let $\mathcal{T}$ be the Riordan array for a variant
of the Motzkin array defined as:
\begin{displaymath} 
    \mathcal{T} = \left(\frac{1}{\sqrt{1-2t-3t^2}}, 
       \frac{1-t-\sqrt{1-2t-3t^2}}{2t}  \right)
\end{displaymath} 
computing the compositional inverse of $h$ yields:
\begin{displaymath} 
    \hat{h}(y) = \frac{y}{1+y+y^2}
\end{displaymath} 
so $f(y)=d(\hat{h}(1+y))=\sqrt{\left(\frac{y^2+3y+3}{(y+2)y}\right)^{2}}$ therefore:
\begin{displaymath} 
    \mathcal{T}_{y=h(t)-1}\left( \sqrt{\left(\frac{y^2+3y+3}{(y+2)y}\right)^{2}}, 1+y \right) = 
        \mathcal{T}_{h(t)}\left( \sqrt{\left(\frac{h(t)^2+h(t)+1}{h(t)^2-1}\right)^{2}}, h(t) \right) 
\end{displaymath} 
By cases on $\sqrt{\left(\frac{h(t)^2+h(t)+1}{h(t)^2-1}\right)^{2}}$:
\begin{itemize}
    \item $\sqrt{\left(\frac{h(t)^2+h(t)+1}{h(t)^2-1}\right)^{2}}=\frac{h(t)^2+h(t)+1}{h(t)^2-1}$, hence:
        \begin{displaymath} 
            \mathcal{T}_{h(t)}\left(\frac{h(t)^2+h(t)+1}{h(t)^2-1}, h(t) \right)
        \end{displaymath} 
        requirement $h(0)=0$ doesn't raise a non-sense, so use the 
        constraint $d(0)=1$, mandatory in order to have a \emph{proper} array, although
        condition can be relaxed to $d(0)\neq0$:
        \begin{displaymath}
            \left. \frac{h(t)^2+h(t)+1}{h(t)^2-1} \right|_{t=0} = -1 \not= 1 
        \end{displaymath} 
        so discard this case;
    \item $\sqrt{\left(\frac{h(t)^2+h(t)+1}{h(t)^2-1}\right)^{2}}=\frac{h(t)^2+h(t)+1}{1-h(t)^2}$, hence:
        \begin{displaymath}
            \mathcal{T}_{h(t)}\left(\frac{h(t)^2+h(t)+1}{1-h(t)^2}, h(t) \right)
        \end{displaymath} 
        requirement $h(0)=0$ doesn't raise a non-sense, so check $d(0)=1$ again:
        \begin{displaymath}
            \left. \frac{h(t)^2+h(t)+1}{1-h(t)^2} \right|_{t=0} = 1 
        \end{displaymath} 
        which fulfill the constraint, so accept this case.
\end{itemize}
Now study the generating function which carries coefficients lying on column $k$
of $\mathcal{T}_{h(t)}$:
\marginpar{same numerator as in classic $\mathcal{M}$ array, what's
    the relation among function $d$ and multiplier $\frac{1}{1-h(t)^{2}}$?}
\begin{displaymath} 
    \frac{h(t)^{k}+h(t)^{k+1}+h(t)^{k+2}}{1-h(t)^2 }
\end{displaymath} 


\subsubsection{Delannoy}

Let $\mathcal{D}$ be the Riordan array for the Delannoy triangle, defined as:
\begin{displaymath} 
    \mathcal{D} =\left( \frac{1}{1-t}, \frac{t(1+t)}{1-t}  \right)
\end{displaymath} 
computing the compositional inverse of $h$ yields:
\begin{displaymath} 
    \hat{h}(y) = \frac{\sqrt{1+6y+y^2}-y-1}{2}
\end{displaymath} 
so $f(y)=d(\hat{h}(1+y))=\frac{2}{4 + y - \sqrt{y^2 + 8y + 8} }$ therefore:
\begin{displaymath} 
    \begin{split}
        & \mathcal{D}_{y=h(t)-1}\left( \frac{2}{4+y-\sqrt{y^2+8y+8}}, 1+y \right)\\
        &= \mathcal{D}_{h(t)}\left( \frac{2}{3+h(t)-\sqrt{h(t)^2+6h(t)+1}}, h(t) \right) \\
    \end{split}
\end{displaymath} 
Since $h(t)^2+6h(t)+1$ isn't a perfect square, no problem arises with the previous
derivation; now study the generating function which carries coefficients lying on column $k$
of $\mathcal{D}_{h(t)}$:
\begin{displaymath} 
    \frac{2\,h(t)^k}{3+h(t)-\sqrt{h(t)^2+6h(t)+1}}
\end{displaymath} 

\subsection{Points of view}

Let $\mathcal{R}\left(d(t),h(t)\right)$ be a Riordan array and $\mathcal{R}_{h(t)}$ be
its $h$-characterization.

The first point of view we would like to introduce is to see 
$\mathcal{R}_{h(t)}$ as a \emph{factorization} \marginpar{$\mathcal{R}_{h(t)}$ factorizes $\mathcal{R}$}
of $\mathcal{R}$ in terms of $h(t)$. Namely, the structure of the characterization
depends both on functions $d$ and $\hat{h}$, but the building block is function $h$
alone. Moreover, second component of $\mathcal{R}_{h(t)}$ is \emph{always} function $h$ itself,
therefore $k$-fold convolution of function $h$ with itself in the generic
$k$ column expansion is common both to $\mathcal{R}$ both to $\mathcal{R}_{h(t)}$.

The second point of view is to see $\mathcal{R}_{h(t)}$ as a \emph{schema} array 
\marginpar{$\mathcal{R}_{h(t)}$ is a schema}.
To understand this concept, abstract over $h(t)$ and think about it as a ``plugin'', 
a ``context'' that can be filled with any function $g(t)$ you like, to get a new array
$\mathcal{R}^{\stackrel{g(t)}{\rightarrow}}$ (this notation is just a reminder that the array is obtained
by plugging in $g(t)$ into the schema $\mathcal{R}_{h(t)}$). It's pretty easy, and sound, to check:
\marginpar{plugging in $h(t)$ itself within schema $\mathcal{R}_{h(t)}$ safely gets back
    to the original array $\mathcal{R}$}
\begin{displaymath}
    \mathcal{R}^{\stackrel{h(t)}{\rightarrow}} = \mathcal{R}_{y=h(t)-1}\left( d(\hat{h}(1+y)), 1+y \right) = \mathcal{R}
\end{displaymath}
while considering an arbitrary function $g(t)$:
\begin{displaymath}
    \mathcal{R}^{\stackrel{g(t)}{\rightarrow}} = \mathcal{R}_{y=g(t)-1}\left( d(\hat{h}(1+y)), 1+y \right) = 
    \left( d(\hat{h}(g(t))), g(t) \right) 
\end{displaymath}
observe how the first component of $\mathcal{R}^{\stackrel{g(t)}{\rightarrow}}$ depends on
$\mathcal{R}$ by composition of functions $d$ and $\hat{h}$. So, every array seems to have
a ``nested schema'' that allows to build new arrays.
\\\\
We finish with a theorem about arrays in the \emph{renewal} subgroup.

\begin{theorem}
    Let $\mathcal{R}\left(d(t), h(t)\right)$ be a Riordan array belonging
    to the \emph{renewal} subgroup, so $h(t)=td(t)$. Then:
    \marginpar{it can be written as $\left[\left.\mathcal{R}\left(A(y), y)\right)\right|y=h(t)\right]$
        or as $\mathcal{R}_{y=h(t)}\left(A(y), y)\right)$ too}
    \begin{displaymath}
        \mathcal{R}_{h(t)}\left(A(h(t)), h(t))\right)
    \end{displaymath}
    where $A$ is a \ac{fps} over $\mathcal{R}$'s $A$-sequence 
    $\lbrace a_i \rbrace_{i\in\mathbb{N}}$. 
\end{theorem}

\begin{proof}
    Recall that $A$-sequence of a Riordan array $\mathcal{R}\left(d(t), h(t)\right)$
    satisfies:
    \begin{displaymath}
        h(t)=tA(h(t))
    \end{displaymath}
    By hypothesis, assume $\mathcal{R}$ belongs to the \emph{renewal} subgroup, therefore
    \begin{displaymath}
        \mathcal{R}\left(A(h(t)), h(t)\right)
    \end{displaymath}
    Let $\mathcal{R}_{h(t)}\left(g(h(t)), h(t)\right)$ be the $h$-characterization of $\mathcal{R}$,
    for some function $g$. Now plug $h(t)$ into $\mathcal{R}_{h(t)}$:
    \begin{displaymath}
        \mathcal{R}^{\stackrel{h(t)}{\rightarrow}} =
            \mathcal{R}\left(g(h(t)), h(t)\right)
    \end{displaymath}
    But $\mathcal{R}^{\stackrel{h(t)}{\rightarrow}} = \mathcal{R}$, therefore 
    $[g(y)=A(y)|y=h(t)]$ follows, as required.

\end{proof}

%\marginpar{$\mathcal{R}_{h(t)}\left(A(h(t)), h(t))\right)\rightarrow
    %\mathcal{R}\left(d(t), t\,d(t)\right)$ holds too}
\marginpar{the converse is true indeed, by $A$-sequence uniqueness}
Looking at it deeply, it is possible to have a stronger formulation, namely the 
converse holds indeed: the reason for this is the unique existence of an $A$-sequence
for each Riordan array, the argument follows using $h(t)=tA(h(t))$ again .
\\\\
This little theorem shows a possible application of the factorization point of view.
Consider the Motzkin array defined as:
\begin{displaymath} 
    \mathcal{M} =\left( \frac{1-t-\sqrt{1-2t-3t^2}}{2t^2},
       \frac{1-t-\sqrt{1-2t-3t^2}}{2t}  \right)
\end{displaymath} 
Surely $\mathcal{M}$ is a renewal array and, being in the Riordan group, 
has an $A$-sequence, which satisfies $A(t)=1+t+t^2$: 
this information \emph{can't} be read from the above definition.
\\\\
\marginpar{finding $\mathcal{M}$'s $A$-sequence}
Let $h(t)=\frac{1-t-\sqrt{1-2t-3t^2}}{2t}$ be the second
component of Motzkin array $\mathcal{M}$, which factor as follows:
\begin{displaymath} 
    \mathcal{M}_{h(t)}\left( 1+h(t)+h(t)^2, h(t) \right) 
\end{displaymath} 
using this $h$-characterization, $A$-sequence $[A(y)=1+y+y^2|y=h(t)]$ can
be read from $\mathcal{M}_{h(t)}$'s first component directly, without
solving the equation as usual:
\begin{displaymath} 
    \frac{1-t-\sqrt{1-2t-3t^2}}{2t} = t\,A\left(\frac{1-t-\sqrt{1-2t-3t^2}}{2t}\right)
\end{displaymath} 

