

\section{Open questions}

Under previous insights, we ask respectively:
\begin{enumerate}
    \item let $\mathcal{R}_{h(t)}=\left(g(h(t)), h(t)\right)$ a $h$-characterization 
        of $\mathcal{R}$, and consider the series expansion of funtion $g$:
        \begin{displaymath}
            \left.\left[g(y)=g_0 + g_1 y + g_2 y^2 + \ldots\right|y=h(t)\right] 
        \end{displaymath}
        what's the interpretation of coefficients in the sequence
        $\lbrace g_i\rbrace_{i\in\mathbb{N}}$? 
        
        We've seen that if $\mathcal{R}=\left(d(t), td(t)\right)$, then 
        $[g(y)=A(y)|y=h(t)]$ where $A(t)$ is $\mathcal{R}$'s A-sequence: 
        does exist a deep relation among those sequences for \emph{arbitrary} arrays?   
        For instance, what about Fibonacci array $\mathcal{F}$: 
        \begin{displaymath}
            \begin{split}
                \big[ g_{\mathcal{F}}(y) &= 1 + y + y^{2} - y^{3} -2y^{4} 
                -3 y^{5}  +3 y^{7} \\
                &+7 y^{8} +4 y^{9} + \mathcal{O}\left(y^{10}\right) | y=h_{\mathcal{F}}(t) \big]
            \end{split}
        \end{displaymath}
        or about Delannoy array $\mathcal{D}$: 
        \begin{displaymath}
            \begin{split}
                \big[ g_{\mathcal{D}}(y) &= 1 + y - y^{2} + 3 y^{3} -11y^{4} + 45 y^{5} -197y^{6}\\ 
                &+ 903 y^{7} -4279y^{8} + \mathcal{O}\left(y^{9}\right)| y=h_{\mathcal{D}}(t) \big]
            \end{split}
        \end{displaymath}
    \item let $\mathcal{P}$ and $\mathcal{C}$ be Pascal and Catalan arrays, respectively: 
        what does $\mathcal{C}^{\stackrel{\frac{t}{1-t}}{\rightarrow}}$ count? 
        More generally, chosen two Riordan arrays $\mathcal{A}$ and $\mathcal{B}$, 
        $\mathcal{A}^{\stackrel{h_{\mathcal{B}}(t)}{\rightarrow}}$ also has a combinatorial meaning? 
        So does $\mathcal{B}^{\stackrel{h_{\mathcal{A}}(t)}{\rightarrow}}$? What about considering
        combinations involving  $\mathcal{A}^{-1}$ and $\mathcal{B}^{-1}$ too?


    \item let $\mathcal{R}(d(t),h(t))$ be a Riordan array in 
        natural notation. If function $\hat{h}$, the compositional inverse
        of function $h$, has the following structure:
        \begin{displaymath}
            \hat{h}(y) = \frac{y}{\Pi(y)}
        \end{displaymath}
        where $\Pi$ is a polynomial such that $\Pi(0)\not=0$, the question is:
        is it always the case that there exists a sequence of polynomials
        $\lbrace \Omega_{i} \rbrace_{i\in\mathbb{N}}$ such
        that $\mathcal{R}$'s $A$-sequence can be be factored respect
        function $\hat{h}$ as:
        \begin{displaymath}
                \left.\left[
                    A_{\mathcal{R}}(y) = \sum_{i \geq0}{
                        \left(\frac{[t^{1+i}]d(t)h(t)}
                            {\Pi(y)}\right)
                        \hat{h}(y)^{i}} 
                        \right| y = h(t) \right]
        \end{displaymath}
        in other words, the sequence of coefficients of the function defining
        the second column of array $\mathcal{R}$, namely the convolution of functions 
        $d$ and $h$, shifted by \emph{one} position, does occur in
        the given factorization?

\end{enumerate}

\subsection{An hint\ldots}

In this section we offer an hint to tackle question number $1$
asked in previous enumeration. Let $\mathcal{R}_{h(t)}(\gamma(h(t)), h(t))$
be a Riordan array written in $h$-characterization form, for some function $\gamma$. 
Now, with abuse of notation, do a \emph{standard} matrix-vector product, namely
consider $\mathcal{R}_{h(t)}$ as a matrix: multiply it by a vector $\vect{\omega}$
whose coefficients are defined by a sequence $\lbrace \omega_i \rbrace_{i\in\mathbb{N}}$
and set equal to a vector $\vect{a}$ whose coefficients are defined by $\mathcal{R}$'s
$A$-sequence. Formally:

\begin{displaymath}
    \mathcal{R}_{h(t)}\left[\begin{array}{c} \omega_0 \\ \omega_1 \\ \omega_2 \\ \vdots \end{array}\right] =
        \left[\begin{array}{c} a_0 \\ a_1 \\ a_2 \\ \vdots \end{array}\right]
\end{displaymath}
Interpreting columns of $\mathcal{R}_{h(t)}$ and vector $\vect{a}$ via
corresponding generating functions, rewrite as follow:
\begin{displaymath}
    \begin{split}
            \omega_0\,\gamma(h(t))\,h(t)^{0} + 
            \omega_1\,\gamma(h(t))\,h(t)^{1} + 
            \omega_2\,\gamma(h(t))\,h(t)^{2} + 
            \ldots &= A(t) \\
            \gamma(h(t))\left(\omega_0\,h(t)^{0} + 
            \omega_1\,h(t)^{1} + 
            \omega_2\,h(t)^{2} + 
            \ldots \right) &= A(t) \\
            \gamma(h(t))\,\Omega(h(t)) &= A(t) \\
    \end{split}
\end{displaymath}
where functions $A$ and $\Omega$ are \ac{fps} over sequences
$\lbrace a_i \rbrace_{i\in\mathbb{N}}$ and
$\lbrace \omega_i \rbrace_{i\in\mathbb{N}}$, respectively. Note that abstracting
over function $h$ yield:
\begin{displaymath}
    \left.\left[
        \gamma(y)\,\Omega(y) = A(\hat{h}(y)) \right| y = h(t) \right]
\end{displaymath}
but this relationship is quite difficult since introduces function $\hat{h}$,
the compositional inverse of function $h$. Therefore, function
$\gamma$, parameterized by function $h$, is related to $\mathcal{R}$'s
$A$-sequence, parameterized over variable $t$, by the existence of 
a function $\Omega$, parameterized over function $h$, such that:
\begin{displaymath}
    \gamma(h(t))\,\Omega(h(t)) = A(t) 
\end{displaymath}
for the sake of clarity, let $\mathcal{M}$ be the Motzkin array, so:
\begin{displaymath}
        \left.\left[
            \Omega_{\mathcal{M}}(y) = \frac{y^{4} + 3 \, y^{3} + 5 \, y^{2} + 3 \, y + 1}{{\left(y^{2} + y + 1\right)}^{3}}
                \right| y = h(t) \right]
\end{displaymath}
and the first terms of fps expansion:
\begin{displaymath}
        \left.\left[
            \Omega_{\mathcal{M}}(y) = 1 -y^{2} -y^{3} + 4 y^{4} + -2\,y^{5} 
                + -6\,y^{6} + 11 y^{7}+\mathcal{O}\left(y^{8}\right)
                \right| y = h(t) \right]
\end{displaymath}
an hint has been provided, in turn another question arises: \emph{there
exists a smart way to compute function $\Omega$}? If there exists such a method, then
$\mathcal{R}$'s $A$-sequence could be computed easily, since function $\gamma$ 
is read directly from the factorization, for \emph{any} Riordan array\ldots
