%%%%%%%%%%%%%%%%%%%%%%%%%%%
% Author : Paul Gaborit (2009)
% under Creative Commons attribution license.
% Title : Pascal's triangle and Sierpinski triangle
% Note : 17 lines maximum
\documentclass[landscape]{article}
\usepackage[landscape,margin=1cm]{geometry}
\pagestyle{empty}
\usepackage[T1]{fontenc}
\usepackage{lmodern}

\usepackage{tikz}
\usetikzlibrary{positioning,shadows,backgrounds}
\usetikzlibrary{external}

%%%<
\usepackage{verbatim}
\usepackage[active,tightpage]{preview}
\PreviewEnvironment{tikzpicture}
\setlength\PreviewBorder{5pt}%
%%%>

%\tikzexternalize % activate!
\begin{comment}
:Title:  Pascal's triangle and Sierpinski triangle

\end{comment}


\begin{document}
\centering

\begin{tikzpicture}[x=13mm,y=9mm]
  % some colors
  
%  \colorlet{even}{cyan!60!black}
%  \colorlet{odd}{orange!100!black}
%  \colorlet{links}{red!70!black}
%  \colorlet{back}{yellow!20!white}

%  \colorlet{zero}{cyan!80!black}
%  \colorlet{one}{orange!80!black}
%  \colorlet{two}{blue!80!black}
%  \colorlet{three}{red!80!black}

  \colorlet{0}{blue!100!black}
  \colorlet{1}{orange!100!black}
  \colorlet{2}{red!100!black}
  \colorlet{3}{cyan!100!black}
  \colorlet{4}{magenta!100!black}
  \colorlet{5}{violet!100!black}
  \colorlet{6}{black!100!black}
  \colorlet{7}{gray!100!black}
  \colorlet{8}{darkgray!100!black}
  \colorlet{9}{lightgray!100!black}
  \colorlet{10}{brown!100!black}
  \colorlet{11}{lime!100!black}
  \colorlet{12}{olive!100!black}
  \colorlet{13}{green!100!black}
  \colorlet{14}{pink!100!black}
  \colorlet{15}{purple!100!black}
  \colorlet{16}{teal!100!black}
  \colorlet{17}{yellow!100!black}
  \colorlet{18}{white!100!black}

  \colorlet{0-for-negatives}{blue!40!white}
  \colorlet{1-for-negatives}{orange!40!white}
  \colorlet{2-for-negatives}{red!40!white}
  \colorlet{3-for-negatives}{cyan!40!white}
  \colorlet{4-for-negatives}{magenta!40!white}
  \colorlet{5-for-negatives}{violet!40!white}
  \colorlet{6-for-negatives}{white!40!white}
  \colorlet{7-for-negatives}{gray!40!white}
  \colorlet{8-for-negatives}{darkgray!40!white}
  \colorlet{9-for-negatives}{lightgray!40!white}
  \colorlet{10-for-negatives}{brown!40!white}
  \colorlet{11-for-negatives}{lime!40!white}
  \colorlet{12-for-negatives}{olive!40!white}
  \colorlet{13-for-negatives}{green!40!white}
  \colorlet{14-for-negatives}{pink!40!white}
  \colorlet{15-for-negatives}{purple!40!white}
  \colorlet{16-for-negatives}{teal!40!white}
  \colorlet{17-for-negatives}{yellow!40!white}
  \colorlet{18-for-negatives}{white!40!white}

  \colorlet{blue}{blue!100!black}
  \colorlet{orange}{orange!100!black}
  \colorlet{red}{red!100!black}
  \colorlet{cyan}{cyan!100!black}
  \colorlet{magenta}{magenta!100!black}
  \colorlet{violet}{violet!100!black}
  \colorlet{black}{black!100!black}
  \colorlet{gray}{gray!100!black}
  \colorlet{darkgray}{darkgray!100!black}
  \colorlet{lightgray}{lightgray!100!black}
  \colorlet{brown}{brown!100!black}
  \colorlet{lime}{lime!100!black}
  \colorlet{olive}{olive!100!black}
  \colorlet{green}{green!100!black}
  \colorlet{pink}{pink!100!black}
  \colorlet{purple}{purple!100!black}
  \colorlet{teal}{teal!100!black}
  \colorlet{yellow}{yellow!100!black}
  \colorlet{white}{white!100!black}

  \colorlet{blue-for-negatives}{blue!40!white}
  \colorlet{orange-for-negatives}{orange!40!white}
  \colorlet{red-for-negatives}{red!40!white}
  \colorlet{cyan-for-negatives}{cyan!40!white}
  \colorlet{magenta-for-negatives}{magenta!40!white}
  \colorlet{violet-for-negatives}{violet!40!white}
  \colorlet{white-for-negatives}{white!40!white}
  \colorlet{gray-for-negatives}{gray!40!white}
  \colorlet{darkgray-for-negatives}{darkgray!40!white}
  \colorlet{lightgray-for-negatives}{lightgray!40!white}
  \colorlet{brown-for-negatives}{brown!40!white}
  \colorlet{lime-for-negatives}{lime!40!white}
  \colorlet{olive-for-negatives}{olive!40!white}
  \colorlet{green-for-negatives}{green!40!white}
  \colorlet{pink-for-negatives}{pink!40!white}
  \colorlet{purple-for-negatives}{purple!40!white}
  \colorlet{teal-for-negatives}{teal!40!white}
  \colorlet{yellow-for-negatives}{yellow!40!white}
  \colorlet{white-for-negatives}{white!40!white}


  % some styles
  \tikzset{
    box/.style={
        circle,
      minimum height=5mm,
      inner sep=.7mm,
      outer sep=0mm,
      text width=10mm,
      text centered,
      font=\small\bfseries\sffamily,
      text=#1!50!black,
      draw=#1,
      line width=.25mm,
      top color=#1!5,
      bottom color=#1!40,
      shading angle=0,
      rounded corners=2.3mm,
      drop shadow={fill=#1!40!gray,fill opacity=.8},
      rotate=0,
    },
%    link/.style={-latex,links,line width=.3mm},
%    plus/.style={text=links,font=\footnotesize\bfseries\sffamily},
  }
  % the following key `pascal-inverse-ignore-negatives-centered-colouring-127-rows-mod3-partitioning-tikz-nodes.tex' is used as a template key
  % to allow Python to do string substitution
  \node[box=1] (p-0-0) at (-0/2+0,-0) {};
\node[box=2] (p-1-0) at (-1/2+0,-1) {};
\node[box=1] (p-1-1) at (-1/2+1,-1) {};
\node[box=1] (p-2-0) at (-2/2+0,-2) {};
\node[box=1] (p-2-1) at (-2/2+1,-2) {};
\node[box=1] (p-2-2) at (-2/2+2,-2) {};
\node[box=2] (p-3-0) at (-3/2+0,-3) {};
\node[box=0] (p-3-1) at (-3/2+1,-3) {};
\node[box=0] (p-3-2) at (-3/2+2,-3) {};
\node[box=1] (p-3-3) at (-3/2+3,-3) {};
\node[box=1-for-negatives] (p-4-0) at (-4/2+0,-4) {};
\node[box=2-for-negatives] (p-4-1) at (-4/2+1,-4) {};
\node[box=lightgray-for-negatives] (p-4-2) at (-4/2+2,-4) {};
\node[box=lightgray-for-negatives] (p-4-3) at (-4/2+3,-4) {};
\node[box=lightgray-for-negatives] (p-4-4) at (-4/2+4,-4) {};
\node[box=2-for-negatives] (p-5-0) at (-5/2+0,-5) {};
\node[box=2-for-negatives] (p-5-1) at (-5/2+1,-5) {};
\node[box=2-for-negatives] (p-5-2) at (-5/2+2,-5) {};
\node[box=lightgray-for-negatives] (p-5-3) at (-5/2+3,-5) {};
\node[box=lightgray-for-negatives] (p-5-4) at (-5/2+4,-5) {};
\node[box=lightgray-for-negatives] (p-5-5) at (-5/2+5,-5) {};
\node[box=1-for-negatives] (p-6-0) at (-6/2+0,-6) {};
\node[box=0] (p-6-1) at (-6/2+1,-6) {};
\node[box=0-for-negatives] (p-6-2) at (-6/2+2,-6) {};
\node[box=1-for-negatives] (p-6-3) at (-6/2+3,-6) {};
\node[box=lightgray-for-negatives] (p-6-4) at (-6/2+4,-6) {};
\node[box=lightgray-for-negatives] (p-6-5) at (-6/2+5,-6) {};
\node[box=lightgray-for-negatives] (p-6-6) at (-6/2+6,-6) {};
\node[box=2-for-negatives] (p-7-0) at (-7/2+0,-7) {};
\node[box=1-for-negatives] (p-7-1) at (-7/2+1,-7) {};
\node[box=0-for-negatives] (p-7-2) at (-7/2+2,-7) {};
\node[box=2-for-negatives] (p-7-3) at (-7/2+3,-7) {};
\node[box=1-for-negatives] (p-7-4) at (-7/2+4,-7) {};
\node[box=lightgray-for-negatives] (p-7-5) at (-7/2+5,-7) {};
\node[box=lightgray-for-negatives] (p-7-6) at (-7/2+6,-7) {};
\node[box=lightgray-for-negatives] (p-7-7) at (-7/2+7,-7) {};
\node[box=1] (p-8-0) at (-8/2+0,-8) {};
\node[box=1-for-negatives] (p-8-1) at (-8/2+1,-8) {};
\node[box=1-for-negatives] (p-8-2) at (-8/2+2,-8) {};
\node[box=1] (p-8-3) at (-8/2+3,-8) {};
\node[box=1-for-negatives] (p-8-4) at (-8/2+4,-8) {};
\node[box=1-for-negatives] (p-8-5) at (-8/2+5,-8) {};
\node[box=lightgray-for-negatives] (p-8-6) at (-8/2+6,-8) {};
\node[box=lightgray-for-negatives] (p-8-7) at (-8/2+7,-8) {};
\node[box=lightgray-for-negatives] (p-8-8) at (-8/2+8,-8) {};
\node[box=lightgray-for-negatives] (p-9-0) at (-9/2+0,-9) {};
\node[box=0-for-negatives] (p-9-1) at (-9/2+1,-9) {};
\node[box=0-for-negatives] (p-9-2) at (-9/2+2,-9) {};
\node[box=0-for-negatives] (p-9-3) at (-9/2+3,-9) {};
\node[box=0-for-negatives] (p-9-4) at (-9/2+4,-9) {};
\node[box=0-for-negatives] (p-9-5) at (-9/2+5,-9) {};
\node[box=0-for-negatives] (p-9-6) at (-9/2+6,-9) {};
\node[box=lightgray-for-negatives] (p-9-7) at (-9/2+7,-9) {};
\node[box=lightgray-for-negatives] (p-9-8) at (-9/2+8,-9) {};
\node[box=lightgray-for-negatives] (p-9-9) at (-9/2+9,-9) {};
\node[box=lightgray-for-negatives] (p-10-0) at (-10/2+0,-10) {};
\node[box=lightgray-for-negatives] (p-10-1) at (-10/2+1,-10) {};
\node[box=0-for-negatives] (p-10-2) at (-10/2+2,-10) {};
\node[box=0-for-negatives] (p-10-3) at (-10/2+3,-10) {};
\node[box=0-for-negatives] (p-10-4) at (-10/2+4,-10) {};
\node[box=0-for-negatives] (p-10-5) at (-10/2+5,-10) {};
\node[box=0-for-negatives] (p-10-6) at (-10/2+6,-10) {};
\node[box=0-for-negatives] (p-10-7) at (-10/2+7,-10) {};
\node[box=lightgray-for-negatives] (p-10-8) at (-10/2+8,-10) {};
\node[box=lightgray-for-negatives] (p-10-9) at (-10/2+9,-10) {};
\node[box=lightgray-for-negatives] (p-10-10) at (-10/2+10,-10) {};
\node[box=lightgray-for-negatives] (p-11-0) at (-11/2+0,-11) {};
\node[box=lightgray-for-negatives] (p-11-1) at (-11/2+1,-11) {};
\node[box=lightgray-for-negatives] (p-11-2) at (-11/2+2,-11) {};
\node[box=0-for-negatives] (p-11-3) at (-11/2+3,-11) {};
\node[box=0-for-negatives] (p-11-4) at (-11/2+4,-11) {};
\node[box=0-for-negatives] (p-11-5) at (-11/2+5,-11) {};
\node[box=0-for-negatives] (p-11-6) at (-11/2+6,-11) {};
\node[box=0-for-negatives] (p-11-7) at (-11/2+7,-11) {};
\node[box=0-for-negatives] (p-11-8) at (-11/2+8,-11) {};
\node[box=lightgray-for-negatives] (p-11-9) at (-11/2+9,-11) {};
\node[box=lightgray-for-negatives] (p-11-10) at (-11/2+10,-11) {};
\node[box=lightgray-for-negatives] (p-11-11) at (-11/2+11,-11) {};
\node[box=lightgray-for-negatives] (p-12-0) at (-12/2+0,-12) {};
\node[box=lightgray-for-negatives] (p-12-1) at (-12/2+1,-12) {};
\node[box=lightgray-for-negatives] (p-12-2) at (-12/2+2,-12) {};
\node[box=lightgray-for-negatives] (p-12-3) at (-12/2+3,-12) {};
\node[box=0-for-negatives] (p-12-4) at (-12/2+4,-12) {};
\node[box=0-for-negatives] (p-12-5) at (-12/2+5,-12) {};
\node[box=0-for-negatives] (p-12-6) at (-12/2+6,-12) {};
\node[box=0-for-negatives] (p-12-7) at (-12/2+7,-12) {};
\node[box=0-for-negatives] (p-12-8) at (-12/2+8,-12) {};
\node[box=2-for-negatives] (p-12-9) at (-12/2+9,-12) {};
\node[box=lightgray-for-negatives] (p-12-10) at (-12/2+10,-12) {};
\node[box=lightgray-for-negatives] (p-12-11) at (-12/2+11,-12) {};
\node[box=lightgray-for-negatives] (p-12-12) at (-12/2+12,-12) {};
\node[box=lightgray-for-negatives] (p-13-0) at (-13/2+0,-13) {};
\node[box=lightgray-for-negatives] (p-13-1) at (-13/2+1,-13) {};
\node[box=lightgray-for-negatives] (p-13-2) at (-13/2+2,-13) {};
\node[box=lightgray-for-negatives] (p-13-3) at (-13/2+3,-13) {};
\node[box=lightgray-for-negatives] (p-13-4) at (-13/2+4,-13) {};
\node[box=0-for-negatives] (p-13-5) at (-13/2+5,-13) {};
\node[box=0-for-negatives] (p-13-6) at (-13/2+6,-13) {};
\node[box=0-for-negatives] (p-13-7) at (-13/2+7,-13) {};
\node[box=0-for-negatives] (p-13-8) at (-13/2+8,-13) {};
\node[box=1-for-negatives] (p-13-9) at (-13/2+9,-13) {};
\node[box=2-for-negatives] (p-13-10) at (-13/2+10,-13) {};
\node[box=lightgray-for-negatives] (p-13-11) at (-13/2+11,-13) {};
\node[box=lightgray-for-negatives] (p-13-12) at (-13/2+12,-13) {};
\node[box=lightgray-for-negatives] (p-13-13) at (-13/2+13,-13) {};
\node[box=lightgray-for-negatives] (p-14-0) at (-14/2+0,-14) {};
\node[box=lightgray-for-negatives] (p-14-1) at (-14/2+1,-14) {};
\node[box=lightgray-for-negatives] (p-14-2) at (-14/2+2,-14) {};
\node[box=lightgray-for-negatives] (p-14-3) at (-14/2+3,-14) {};
\node[box=lightgray-for-negatives] (p-14-4) at (-14/2+4,-14) {};
\node[box=lightgray-for-negatives] (p-14-5) at (-14/2+5,-14) {};
\node[box=0-for-negatives] (p-14-6) at (-14/2+6,-14) {};
\node[box=0-for-negatives] (p-14-7) at (-14/2+7,-14) {};
\node[box=0-for-negatives] (p-14-8) at (-14/2+8,-14) {};
\node[box=2-for-negatives] (p-14-9) at (-14/2+9,-14) {};
\node[box=2-for-negatives] (p-14-10) at (-14/2+10,-14) {};
\node[box=2-for-negatives] (p-14-11) at (-14/2+11,-14) {};
\node[box=lightgray-for-negatives] (p-14-12) at (-14/2+12,-14) {};
\node[box=lightgray-for-negatives] (p-14-13) at (-14/2+13,-14) {};
\node[box=lightgray-for-negatives] (p-14-14) at (-14/2+14,-14) {};
\node[box=lightgray-for-negatives] (p-15-0) at (-15/2+0,-15) {};
\node[box=lightgray-for-negatives] (p-15-1) at (-15/2+1,-15) {};
\node[box=lightgray-for-negatives] (p-15-2) at (-15/2+2,-15) {};
\node[box=lightgray-for-negatives] (p-15-3) at (-15/2+3,-15) {};
\node[box=lightgray-for-negatives] (p-15-4) at (-15/2+4,-15) {};
\node[box=lightgray-for-negatives] (p-15-5) at (-15/2+5,-15) {};
\node[box=lightgray-for-negatives] (p-15-6) at (-15/2+6,-15) {};
\node[box=0-for-negatives] (p-15-7) at (-15/2+7,-15) {};
\node[box=0-for-negatives] (p-15-8) at (-15/2+8,-15) {};
\node[box=1-for-negatives] (p-15-9) at (-15/2+9,-15) {};
\node[box=0] (p-15-10) at (-15/2+10,-15) {};
\node[box=0-for-negatives] (p-15-11) at (-15/2+11,-15) {};
\node[box=1-for-negatives] (p-15-12) at (-15/2+12,-15) {};
\node[box=lightgray-for-negatives] (p-15-13) at (-15/2+13,-15) {};
\node[box=lightgray-for-negatives] (p-15-14) at (-15/2+14,-15) {};
\node[box=lightgray-for-negatives] (p-15-15) at (-15/2+15,-15) {};
\node[box=lightgray-for-negatives] (p-16-0) at (-16/2+0,-16) {};
\node[box=lightgray-for-negatives] (p-16-1) at (-16/2+1,-16) {};
\node[box=lightgray-for-negatives] (p-16-2) at (-16/2+2,-16) {};
\node[box=lightgray-for-negatives] (p-16-3) at (-16/2+3,-16) {};
\node[box=lightgray-for-negatives] (p-16-4) at (-16/2+4,-16) {};
\node[box=lightgray-for-negatives] (p-16-5) at (-16/2+5,-16) {};
\node[box=lightgray-for-negatives] (p-16-6) at (-16/2+6,-16) {};
\node[box=lightgray-for-negatives] (p-16-7) at (-16/2+7,-16) {};
\node[box=0-for-negatives] (p-16-8) at (-16/2+8,-16) {};
\node[box=2-for-negatives] (p-16-9) at (-16/2+9,-16) {};
\node[box=1-for-negatives] (p-16-10) at (-16/2+10,-16) {};
\node[box=0-for-negatives] (p-16-11) at (-16/2+11,-16) {};
\node[box=2-for-negatives] (p-16-12) at (-16/2+12,-16) {};
\node[box=1-for-negatives] (p-16-13) at (-16/2+13,-16) {};
\node[box=lightgray-for-negatives] (p-16-14) at (-16/2+14,-16) {};
\node[box=lightgray-for-negatives] (p-16-15) at (-16/2+15,-16) {};
\node[box=lightgray-for-negatives] (p-16-16) at (-16/2+16,-16) {};
\node[box=lightgray-for-negatives] (p-17-0) at (-17/2+0,-17) {};
\node[box=lightgray-for-negatives] (p-17-1) at (-17/2+1,-17) {};
\node[box=lightgray-for-negatives] (p-17-2) at (-17/2+2,-17) {};
\node[box=lightgray-for-negatives] (p-17-3) at (-17/2+3,-17) {};
\node[box=lightgray-for-negatives] (p-17-4) at (-17/2+4,-17) {};
\node[box=lightgray-for-negatives] (p-17-5) at (-17/2+5,-17) {};
\node[box=lightgray-for-negatives] (p-17-6) at (-17/2+6,-17) {};
\node[box=lightgray-for-negatives] (p-17-7) at (-17/2+7,-17) {};
\node[box=lightgray-for-negatives] (p-17-8) at (-17/2+8,-17) {};
\node[box=1] (p-17-9) at (-17/2+9,-17) {};
\node[box=1-for-negatives] (p-17-10) at (-17/2+10,-17) {};
\node[box=1-for-negatives] (p-17-11) at (-17/2+11,-17) {};
\node[box=1] (p-17-12) at (-17/2+12,-17) {};
\node[box=1-for-negatives] (p-17-13) at (-17/2+13,-17) {};
\node[box=1-for-negatives] (p-17-14) at (-17/2+14,-17) {};
\node[box=lightgray-for-negatives] (p-17-15) at (-17/2+15,-17) {};
\node[box=lightgray-for-negatives] (p-17-16) at (-17/2+16,-17) {};
\node[box=lightgray-for-negatives] (p-17-17) at (-17/2+17,-17) {};
\node[box=lightgray-for-negatives] (p-18-0) at (-18/2+0,-18) {};
\node[box=lightgray-for-negatives] (p-18-1) at (-18/2+1,-18) {};
\node[box=lightgray-for-negatives] (p-18-2) at (-18/2+2,-18) {};
\node[box=lightgray-for-negatives] (p-18-3) at (-18/2+3,-18) {};
\node[box=lightgray-for-negatives] (p-18-4) at (-18/2+4,-18) {};
\node[box=lightgray-for-negatives] (p-18-5) at (-18/2+5,-18) {};
\node[box=lightgray-for-negatives] (p-18-6) at (-18/2+6,-18) {};
\node[box=lightgray-for-negatives] (p-18-7) at (-18/2+7,-18) {};
\node[box=lightgray-for-negatives] (p-18-8) at (-18/2+8,-18) {};
\node[box=lightgray-for-negatives] (p-18-9) at (-18/2+9,-18) {};
\node[box=0-for-negatives] (p-18-10) at (-18/2+10,-18) {};
\node[box=0-for-negatives] (p-18-11) at (-18/2+11,-18) {};
\node[box=0-for-negatives] (p-18-12) at (-18/2+12,-18) {};
\node[box=0-for-negatives] (p-18-13) at (-18/2+13,-18) {};
\node[box=0-for-negatives] (p-18-14) at (-18/2+14,-18) {};
\node[box=0-for-negatives] (p-18-15) at (-18/2+15,-18) {};
\node[box=lightgray-for-negatives] (p-18-16) at (-18/2+16,-18) {};
\node[box=lightgray-for-negatives] (p-18-17) at (-18/2+17,-18) {};
\node[box=lightgray-for-negatives] (p-18-18) at (-18/2+18,-18) {};
\node[box=lightgray-for-negatives] (p-19-0) at (-19/2+0,-19) {};
\node[box=lightgray-for-negatives] (p-19-1) at (-19/2+1,-19) {};
\node[box=lightgray-for-negatives] (p-19-2) at (-19/2+2,-19) {};
\node[box=lightgray-for-negatives] (p-19-3) at (-19/2+3,-19) {};
\node[box=lightgray-for-negatives] (p-19-4) at (-19/2+4,-19) {};
\node[box=lightgray-for-negatives] (p-19-5) at (-19/2+5,-19) {};
\node[box=lightgray-for-negatives] (p-19-6) at (-19/2+6,-19) {};
\node[box=lightgray-for-negatives] (p-19-7) at (-19/2+7,-19) {};
\node[box=lightgray-for-negatives] (p-19-8) at (-19/2+8,-19) {};
\node[box=lightgray-for-negatives] (p-19-9) at (-19/2+9,-19) {};
\node[box=lightgray-for-negatives] (p-19-10) at (-19/2+10,-19) {};
\node[box=0-for-negatives] (p-19-11) at (-19/2+11,-19) {};
\node[box=0-for-negatives] (p-19-12) at (-19/2+12,-19) {};
\node[box=0-for-negatives] (p-19-13) at (-19/2+13,-19) {};
\node[box=0-for-negatives] (p-19-14) at (-19/2+14,-19) {};
\node[box=0-for-negatives] (p-19-15) at (-19/2+15,-19) {};
\node[box=0-for-negatives] (p-19-16) at (-19/2+16,-19) {};
\node[box=lightgray-for-negatives] (p-19-17) at (-19/2+17,-19) {};
\node[box=lightgray-for-negatives] (p-19-18) at (-19/2+18,-19) {};
\node[box=lightgray-for-negatives] (p-19-19) at (-19/2+19,-19) {};
\node[box=lightgray-for-negatives] (p-20-0) at (-20/2+0,-20) {};
\node[box=lightgray-for-negatives] (p-20-1) at (-20/2+1,-20) {};
\node[box=lightgray-for-negatives] (p-20-2) at (-20/2+2,-20) {};
\node[box=lightgray-for-negatives] (p-20-3) at (-20/2+3,-20) {};
\node[box=lightgray-for-negatives] (p-20-4) at (-20/2+4,-20) {};
\node[box=lightgray-for-negatives] (p-20-5) at (-20/2+5,-20) {};
\node[box=lightgray-for-negatives] (p-20-6) at (-20/2+6,-20) {};
\node[box=lightgray-for-negatives] (p-20-7) at (-20/2+7,-20) {};
\node[box=lightgray-for-negatives] (p-20-8) at (-20/2+8,-20) {};
\node[box=lightgray-for-negatives] (p-20-9) at (-20/2+9,-20) {};
\node[box=lightgray-for-negatives] (p-20-10) at (-20/2+10,-20) {};
\node[box=lightgray-for-negatives] (p-20-11) at (-20/2+11,-20) {};
\node[box=0-for-negatives] (p-20-12) at (-20/2+12,-20) {};
\node[box=0-for-negatives] (p-20-13) at (-20/2+13,-20) {};
\node[box=0-for-negatives] (p-20-14) at (-20/2+14,-20) {};
\node[box=0-for-negatives] (p-20-15) at (-20/2+15,-20) {};
\node[box=0-for-negatives] (p-20-16) at (-20/2+16,-20) {};
\node[box=0-for-negatives] (p-20-17) at (-20/2+17,-20) {};
\node[box=lightgray-for-negatives] (p-20-18) at (-20/2+18,-20) {};
\node[box=lightgray-for-negatives] (p-20-19) at (-20/2+19,-20) {};
\node[box=lightgray-for-negatives] (p-20-20) at (-20/2+20,-20) {};
\node[box=lightgray-for-negatives] (p-21-0) at (-21/2+0,-21) {};
\node[box=lightgray-for-negatives] (p-21-1) at (-21/2+1,-21) {};
\node[box=lightgray-for-negatives] (p-21-2) at (-21/2+2,-21) {};
\node[box=lightgray-for-negatives] (p-21-3) at (-21/2+3,-21) {};
\node[box=lightgray-for-negatives] (p-21-4) at (-21/2+4,-21) {};
\node[box=lightgray-for-negatives] (p-21-5) at (-21/2+5,-21) {};
\node[box=lightgray-for-negatives] (p-21-6) at (-21/2+6,-21) {};
\node[box=lightgray-for-negatives] (p-21-7) at (-21/2+7,-21) {};
\node[box=lightgray-for-negatives] (p-21-8) at (-21/2+8,-21) {};
\node[box=lightgray-for-negatives] (p-21-9) at (-21/2+9,-21) {};
\node[box=lightgray-for-negatives] (p-21-10) at (-21/2+10,-21) {};
\node[box=lightgray-for-negatives] (p-21-11) at (-21/2+11,-21) {};
\node[box=lightgray-for-negatives] (p-21-12) at (-21/2+12,-21) {};
\node[box=0-for-negatives] (p-21-13) at (-21/2+13,-21) {};
\node[box=0-for-negatives] (p-21-14) at (-21/2+14,-21) {};
\node[box=0-for-negatives] (p-21-15) at (-21/2+15,-21) {};
\node[box=0-for-negatives] (p-21-16) at (-21/2+16,-21) {};
\node[box=0-for-negatives] (p-21-17) at (-21/2+17,-21) {};
\node[box=2-for-negatives] (p-21-18) at (-21/2+18,-21) {};
\node[box=lightgray-for-negatives] (p-21-19) at (-21/2+19,-21) {};
\node[box=lightgray-for-negatives] (p-21-20) at (-21/2+20,-21) {};
\node[box=lightgray-for-negatives] (p-21-21) at (-21/2+21,-21) {};
\node[box=lightgray-for-negatives] (p-22-0) at (-22/2+0,-22) {};
\node[box=lightgray-for-negatives] (p-22-1) at (-22/2+1,-22) {};
\node[box=lightgray-for-negatives] (p-22-2) at (-22/2+2,-22) {};
\node[box=lightgray-for-negatives] (p-22-3) at (-22/2+3,-22) {};
\node[box=lightgray-for-negatives] (p-22-4) at (-22/2+4,-22) {};
\node[box=lightgray-for-negatives] (p-22-5) at (-22/2+5,-22) {};
\node[box=lightgray-for-negatives] (p-22-6) at (-22/2+6,-22) {};
\node[box=lightgray-for-negatives] (p-22-7) at (-22/2+7,-22) {};
\node[box=lightgray-for-negatives] (p-22-8) at (-22/2+8,-22) {};
\node[box=lightgray-for-negatives] (p-22-9) at (-22/2+9,-22) {};
\node[box=lightgray-for-negatives] (p-22-10) at (-22/2+10,-22) {};
\node[box=lightgray-for-negatives] (p-22-11) at (-22/2+11,-22) {};
\node[box=lightgray-for-negatives] (p-22-12) at (-22/2+12,-22) {};
\node[box=lightgray-for-negatives] (p-22-13) at (-22/2+13,-22) {};
\node[box=0-for-negatives] (p-22-14) at (-22/2+14,-22) {};
\node[box=0-for-negatives] (p-22-15) at (-22/2+15,-22) {};
\node[box=0-for-negatives] (p-22-16) at (-22/2+16,-22) {};
\node[box=0-for-negatives] (p-22-17) at (-22/2+17,-22) {};
\node[box=1-for-negatives] (p-22-18) at (-22/2+18,-22) {};
\node[box=2-for-negatives] (p-22-19) at (-22/2+19,-22) {};
\node[box=lightgray-for-negatives] (p-22-20) at (-22/2+20,-22) {};
\node[box=lightgray-for-negatives] (p-22-21) at (-22/2+21,-22) {};
\node[box=lightgray-for-negatives] (p-22-22) at (-22/2+22,-22) {};
\node[box=lightgray-for-negatives] (p-23-0) at (-23/2+0,-23) {};
\node[box=lightgray-for-negatives] (p-23-1) at (-23/2+1,-23) {};
\node[box=lightgray-for-negatives] (p-23-2) at (-23/2+2,-23) {};
\node[box=lightgray-for-negatives] (p-23-3) at (-23/2+3,-23) {};
\node[box=lightgray-for-negatives] (p-23-4) at (-23/2+4,-23) {};
\node[box=lightgray-for-negatives] (p-23-5) at (-23/2+5,-23) {};
\node[box=lightgray-for-negatives] (p-23-6) at (-23/2+6,-23) {};
\node[box=lightgray-for-negatives] (p-23-7) at (-23/2+7,-23) {};
\node[box=lightgray-for-negatives] (p-23-8) at (-23/2+8,-23) {};
\node[box=lightgray-for-negatives] (p-23-9) at (-23/2+9,-23) {};
\node[box=lightgray-for-negatives] (p-23-10) at (-23/2+10,-23) {};
\node[box=lightgray-for-negatives] (p-23-11) at (-23/2+11,-23) {};
\node[box=lightgray-for-negatives] (p-23-12) at (-23/2+12,-23) {};
\node[box=lightgray-for-negatives] (p-23-13) at (-23/2+13,-23) {};
\node[box=lightgray-for-negatives] (p-23-14) at (-23/2+14,-23) {};
\node[box=0-for-negatives] (p-23-15) at (-23/2+15,-23) {};
\node[box=0-for-negatives] (p-23-16) at (-23/2+16,-23) {};
\node[box=0-for-negatives] (p-23-17) at (-23/2+17,-23) {};
\node[box=2-for-negatives] (p-23-18) at (-23/2+18,-23) {};
\node[box=2-for-negatives] (p-23-19) at (-23/2+19,-23) {};
\node[box=2-for-negatives] (p-23-20) at (-23/2+20,-23) {};
\node[box=lightgray-for-negatives] (p-23-21) at (-23/2+21,-23) {};
\node[box=lightgray-for-negatives] (p-23-22) at (-23/2+22,-23) {};
\node[box=lightgray-for-negatives] (p-23-23) at (-23/2+23,-23) {};
\node[box=lightgray-for-negatives] (p-24-0) at (-24/2+0,-24) {};
\node[box=lightgray-for-negatives] (p-24-1) at (-24/2+1,-24) {};
\node[box=lightgray-for-negatives] (p-24-2) at (-24/2+2,-24) {};
\node[box=lightgray-for-negatives] (p-24-3) at (-24/2+3,-24) {};
\node[box=lightgray-for-negatives] (p-24-4) at (-24/2+4,-24) {};
\node[box=lightgray-for-negatives] (p-24-5) at (-24/2+5,-24) {};
\node[box=lightgray-for-negatives] (p-24-6) at (-24/2+6,-24) {};
\node[box=lightgray-for-negatives] (p-24-7) at (-24/2+7,-24) {};
\node[box=lightgray-for-negatives] (p-24-8) at (-24/2+8,-24) {};
\node[box=lightgray-for-negatives] (p-24-9) at (-24/2+9,-24) {};
\node[box=lightgray-for-negatives] (p-24-10) at (-24/2+10,-24) {};
\node[box=lightgray-for-negatives] (p-24-11) at (-24/2+11,-24) {};
\node[box=lightgray-for-negatives] (p-24-12) at (-24/2+12,-24) {};
\node[box=lightgray-for-negatives] (p-24-13) at (-24/2+13,-24) {};
\node[box=lightgray-for-negatives] (p-24-14) at (-24/2+14,-24) {};
\node[box=lightgray-for-negatives] (p-24-15) at (-24/2+15,-24) {};
\node[box=0-for-negatives] (p-24-16) at (-24/2+16,-24) {};
\node[box=0-for-negatives] (p-24-17) at (-24/2+17,-24) {};
\node[box=1-for-negatives] (p-24-18) at (-24/2+18,-24) {};
\node[box=0] (p-24-19) at (-24/2+19,-24) {};
\node[box=0-for-negatives] (p-24-20) at (-24/2+20,-24) {};
\node[box=1-for-negatives] (p-24-21) at (-24/2+21,-24) {};
\node[box=lightgray-for-negatives] (p-24-22) at (-24/2+22,-24) {};
\node[box=lightgray-for-negatives] (p-24-23) at (-24/2+23,-24) {};
\node[box=lightgray-for-negatives] (p-24-24) at (-24/2+24,-24) {};
\node[box=lightgray-for-negatives] (p-25-0) at (-25/2+0,-25) {};
\node[box=lightgray-for-negatives] (p-25-1) at (-25/2+1,-25) {};
\node[box=lightgray-for-negatives] (p-25-2) at (-25/2+2,-25) {};
\node[box=lightgray-for-negatives] (p-25-3) at (-25/2+3,-25) {};
\node[box=lightgray-for-negatives] (p-25-4) at (-25/2+4,-25) {};
\node[box=lightgray-for-negatives] (p-25-5) at (-25/2+5,-25) {};
\node[box=lightgray-for-negatives] (p-25-6) at (-25/2+6,-25) {};
\node[box=lightgray-for-negatives] (p-25-7) at (-25/2+7,-25) {};
\node[box=lightgray-for-negatives] (p-25-8) at (-25/2+8,-25) {};
\node[box=lightgray-for-negatives] (p-25-9) at (-25/2+9,-25) {};
\node[box=lightgray-for-negatives] (p-25-10) at (-25/2+10,-25) {};
\node[box=lightgray-for-negatives] (p-25-11) at (-25/2+11,-25) {};
\node[box=lightgray-for-negatives] (p-25-12) at (-25/2+12,-25) {};
\node[box=lightgray-for-negatives] (p-25-13) at (-25/2+13,-25) {};
\node[box=lightgray-for-negatives] (p-25-14) at (-25/2+14,-25) {};
\node[box=lightgray-for-negatives] (p-25-15) at (-25/2+15,-25) {};
\node[box=lightgray-for-negatives] (p-25-16) at (-25/2+16,-25) {};
\node[box=0-for-negatives] (p-25-17) at (-25/2+17,-25) {};
\node[box=2-for-negatives] (p-25-18) at (-25/2+18,-25) {};
\node[box=1-for-negatives] (p-25-19) at (-25/2+19,-25) {};
\node[box=0-for-negatives] (p-25-20) at (-25/2+20,-25) {};
\node[box=2-for-negatives] (p-25-21) at (-25/2+21,-25) {};
\node[box=1-for-negatives] (p-25-22) at (-25/2+22,-25) {};
\node[box=lightgray-for-negatives] (p-25-23) at (-25/2+23,-25) {};
\node[box=lightgray-for-negatives] (p-25-24) at (-25/2+24,-25) {};
\node[box=lightgray-for-negatives] (p-25-25) at (-25/2+25,-25) {};
\node[box=lightgray-for-negatives] (p-26-0) at (-26/2+0,-26) {};
\node[box=lightgray-for-negatives] (p-26-1) at (-26/2+1,-26) {};
\node[box=lightgray-for-negatives] (p-26-2) at (-26/2+2,-26) {};
\node[box=lightgray-for-negatives] (p-26-3) at (-26/2+3,-26) {};
\node[box=lightgray-for-negatives] (p-26-4) at (-26/2+4,-26) {};
\node[box=lightgray-for-negatives] (p-26-5) at (-26/2+5,-26) {};
\node[box=lightgray-for-negatives] (p-26-6) at (-26/2+6,-26) {};
\node[box=lightgray-for-negatives] (p-26-7) at (-26/2+7,-26) {};
\node[box=lightgray-for-negatives] (p-26-8) at (-26/2+8,-26) {};
\node[box=lightgray-for-negatives] (p-26-9) at (-26/2+9,-26) {};
\node[box=lightgray-for-negatives] (p-26-10) at (-26/2+10,-26) {};
\node[box=lightgray-for-negatives] (p-26-11) at (-26/2+11,-26) {};
\node[box=lightgray-for-negatives] (p-26-12) at (-26/2+12,-26) {};
\node[box=lightgray-for-negatives] (p-26-13) at (-26/2+13,-26) {};
\node[box=lightgray-for-negatives] (p-26-14) at (-26/2+14,-26) {};
\node[box=lightgray-for-negatives] (p-26-15) at (-26/2+15,-26) {};
\node[box=lightgray-for-negatives] (p-26-16) at (-26/2+16,-26) {};
\node[box=lightgray-for-negatives] (p-26-17) at (-26/2+17,-26) {};
\node[box=1] (p-26-18) at (-26/2+18,-26) {};
\node[box=1-for-negatives] (p-26-19) at (-26/2+19,-26) {};
\node[box=1-for-negatives] (p-26-20) at (-26/2+20,-26) {};
\node[box=1] (p-26-21) at (-26/2+21,-26) {};
\node[box=1-for-negatives] (p-26-22) at (-26/2+22,-26) {};
\node[box=1-for-negatives] (p-26-23) at (-26/2+23,-26) {};
\node[box=lightgray-for-negatives] (p-26-24) at (-26/2+24,-26) {};
\node[box=lightgray-for-negatives] (p-26-25) at (-26/2+25,-26) {};
\node[box=lightgray-for-negatives] (p-26-26) at (-26/2+26,-26) {};

\end{tikzpicture}

\end{document}
%%%%%%%%%%%%%%%%%%%%%%%%%%%%%%%%%
