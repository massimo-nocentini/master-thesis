%%%%%%%%%%%%%%%%%%%%%%%%%%%
% Author : Paul Gaborit (2009)
% under Creative Commons attribution license.
% Title : Pascal's triangle and Sierpinski triangle
% Note : 17 lines maximum
\documentclass[landscape]{article}
\usepackage[landscape,margin=1cm]{geometry}
\pagestyle{empty}
\usepackage[T1]{fontenc}
\usepackage{lmodern}

\usepackage{tikz}
\usetikzlibrary{positioning,shadows,backgrounds}
\usetikzlibrary{external}

%%%<
\usepackage{verbatim}
\usepackage[active,tightpage]{preview}
\PreviewEnvironment{tikzpicture}
\setlength\PreviewBorder{5pt}%
%%%>

%\tikzexternalize % activate!
\begin{comment}
:Title:  Pascal's triangle and Sierpinski triangle

\end{comment}


\begin{document}
\centering

\begin{tikzpicture}[x=13mm,y=9mm]
  % some colors
  
%  \colorlet{even}{cyan!60!black}
%  \colorlet{odd}{orange!100!black}
%  \colorlet{links}{red!70!black}
%  \colorlet{back}{yellow!20!white}

%  \colorlet{zero}{cyan!80!black}
%  \colorlet{one}{orange!80!black}
%  \colorlet{two}{blue!80!black}
%  \colorlet{three}{red!80!black}

  \colorlet{0}{blue!100!black}
  \colorlet{1}{orange!100!black}
  \colorlet{2}{red!100!black}
  \colorlet{3}{cyan!100!black}
  \colorlet{4}{magenta!100!black}
  \colorlet{5}{violet!100!black}
  \colorlet{6}{black!100!black}
  \colorlet{7}{gray!100!black}
  \colorlet{8}{darkgray!100!black}
  \colorlet{9}{lightgray!100!black}
  \colorlet{10}{brown!100!black}
  \colorlet{11}{lime!100!black}
  \colorlet{12}{olive!100!black}
  \colorlet{13}{green!100!black}
  \colorlet{14}{pink!100!black}
  \colorlet{15}{purple!100!black}
  \colorlet{16}{teal!100!black}
  \colorlet{17}{yellow!100!black}
  \colorlet{18}{white!100!black}

  \colorlet{0-for-negatives}{blue!40!white}
  \colorlet{1-for-negatives}{orange!40!white}
  \colorlet{2-for-negatives}{red!40!white}
  \colorlet{3-for-negatives}{cyan!40!white}
  \colorlet{4-for-negatives}{magenta!40!white}
  \colorlet{5-for-negatives}{violet!40!white}
  \colorlet{6-for-negatives}{white!40!white}
  \colorlet{7-for-negatives}{gray!40!white}
  \colorlet{8-for-negatives}{darkgray!40!white}
  \colorlet{9-for-negatives}{lightgray!40!white}
  \colorlet{10-for-negatives}{brown!40!white}
  \colorlet{11-for-negatives}{lime!40!white}
  \colorlet{12-for-negatives}{olive!40!white}
  \colorlet{13-for-negatives}{green!40!white}
  \colorlet{14-for-negatives}{pink!40!white}
  \colorlet{15-for-negatives}{purple!40!white}
  \colorlet{16-for-negatives}{teal!40!white}
  \colorlet{17-for-negatives}{yellow!40!white}
  \colorlet{18-for-negatives}{white!40!white}

  \colorlet{blue}{blue!100!black}
  \colorlet{orange}{orange!100!black}
  \colorlet{red}{red!100!black}
  \colorlet{cyan}{cyan!100!black}
  \colorlet{magenta}{magenta!100!black}
  \colorlet{violet}{violet!100!black}
  \colorlet{black}{black!100!black}
  \colorlet{gray}{gray!100!black}
  \colorlet{darkgray}{darkgray!100!black}
  \colorlet{lightgray}{lightgray!100!black}
  \colorlet{brown}{brown!100!black}
  \colorlet{lime}{lime!100!black}
  \colorlet{olive}{olive!100!black}
  \colorlet{green}{green!100!black}
  \colorlet{pink}{pink!100!black}
  \colorlet{purple}{purple!100!black}
  \colorlet{teal}{teal!100!black}
  \colorlet{yellow}{yellow!100!black}
  \colorlet{white}{white!100!black}

  \colorlet{blue-for-negatives}{blue!40!white}
  \colorlet{orange-for-negatives}{orange!40!white}
  \colorlet{red-for-negatives}{red!40!white}
  \colorlet{cyan-for-negatives}{cyan!40!white}
  \colorlet{magenta-for-negatives}{magenta!40!white}
  \colorlet{violet-for-negatives}{violet!40!white}
  \colorlet{white-for-negatives}{white!40!white}
  \colorlet{gray-for-negatives}{gray!40!white}
  \colorlet{darkgray-for-negatives}{darkgray!40!white}
  \colorlet{lightgray-for-negatives}{lightgray!40!white}
  \colorlet{brown-for-negatives}{brown!40!white}
  \colorlet{lime-for-negatives}{lime!40!white}
  \colorlet{olive-for-negatives}{olive!40!white}
  \colorlet{green-for-negatives}{green!40!white}
  \colorlet{pink-for-negatives}{pink!40!white}
  \colorlet{purple-for-negatives}{purple!40!white}
  \colorlet{teal-for-negatives}{teal!40!white}
  \colorlet{yellow-for-negatives}{yellow!40!white}
  \colorlet{white-for-negatives}{white!40!white}


  % some styles
  \tikzset{
    box/.style={
        circle,
      minimum height=5mm,
      inner sep=.7mm,
      outer sep=0mm,
      text width=10mm,
      text centered,
      font=\small\bfseries\sffamily,
      text=#1!50!black,
      draw=#1,
      line width=.25mm,
      top color=#1!5,
      bottom color=#1!40,
      shading angle=0,
      rounded corners=2.3mm,
      drop shadow={fill=#1!40!gray,fill opacity=.8},
      rotate=0,
    },
%    link/.style={-latex,links,line width=.3mm},
%    plus/.style={text=links,font=\footnotesize\bfseries\sffamily},
  }
  % the following key `pascal-inverse-ignore-negatives-centered-colouring-127-rows-mod3-partitioning-tikz-nodes.tex' is used as a template key
  % to allow Python to do string substitution
  \node[box=1] (p-0-0) at (-0/2+0,-0) {};
\node[box=2] (p-1-0) at (-1/2+0,-1) {};
\node[box=1-for-negatives] (p-1-1) at (-1/2+1,-1) {};
\node[box=1] (p-2-0) at (-2/2+0,-2) {};
\node[box=1-for-negatives] (p-2-1) at (-2/2+1,-2) {};
\node[box=1-for-negatives] (p-2-2) at (-2/2+2,-2) {};
\node[box=2] (p-3-0) at (-3/2+0,-3) {};
\node[box=0-for-negatives] (p-3-1) at (-3/2+1,-3) {};
\node[box=0-for-negatives] (p-3-2) at (-3/2+2,-3) {};
\node[box=1-for-negatives] (p-3-3) at (-3/2+3,-3) {};
\node[box=1] (p-4-0) at (-4/2+0,-4) {};
\node[box=2-for-negatives] (p-4-1) at (-4/2+1,-4) {};
\node[box=0-for-negatives] (p-4-2) at (-4/2+2,-4) {};
\node[box=2-for-negatives] (p-4-3) at (-4/2+3,-4) {};
\node[box=1-for-negatives] (p-4-4) at (-4/2+4,-4) {};
\node[box=2] (p-5-0) at (-5/2+0,-5) {};
\node[box=2-for-negatives] (p-5-1) at (-5/2+1,-5) {};
\node[box=2-for-negatives] (p-5-2) at (-5/2+2,-5) {};
\node[box=1-for-negatives] (p-5-3) at (-5/2+3,-5) {};
\node[box=1-for-negatives] (p-5-4) at (-5/2+4,-5) {};
\node[box=1-for-negatives] (p-5-5) at (-5/2+5,-5) {};
\node[box=1] (p-6-0) at (-6/2+0,-6) {};
\node[box=0-for-negatives] (p-6-1) at (-6/2+1,-6) {};
\node[box=0-for-negatives] (p-6-2) at (-6/2+2,-6) {};
\node[box=1-for-negatives] (p-6-3) at (-6/2+3,-6) {};
\node[box=0-for-negatives] (p-6-4) at (-6/2+4,-6) {};
\node[box=0-for-negatives] (p-6-5) at (-6/2+5,-6) {};
\node[box=1-for-negatives] (p-6-6) at (-6/2+6,-6) {};
\node[box=2] (p-7-0) at (-7/2+0,-7) {};
\node[box=1-for-negatives] (p-7-1) at (-7/2+1,-7) {};
\node[box=0-for-negatives] (p-7-2) at (-7/2+2,-7) {};
\node[box=2-for-negatives] (p-7-3) at (-7/2+3,-7) {};
\node[box=1-for-negatives] (p-7-4) at (-7/2+4,-7) {};
\node[box=0-for-negatives] (p-7-5) at (-7/2+5,-7) {};
\node[box=2-for-negatives] (p-7-6) at (-7/2+6,-7) {};
\node[box=1-for-negatives] (p-7-7) at (-7/2+7,-7) {};
\node[box=1] (p-8-0) at (-8/2+0,-8) {};
\node[box=1-for-negatives] (p-8-1) at (-8/2+1,-8) {};
\node[box=1-for-negatives] (p-8-2) at (-8/2+2,-8) {};
\node[box=1-for-negatives] (p-8-3) at (-8/2+3,-8) {};
\node[box=1-for-negatives] (p-8-4) at (-8/2+4,-8) {};
\node[box=1-for-negatives] (p-8-5) at (-8/2+5,-8) {};
\node[box=1-for-negatives] (p-8-6) at (-8/2+6,-8) {};
\node[box=1-for-negatives] (p-8-7) at (-8/2+7,-8) {};
\node[box=1-for-negatives] (p-8-8) at (-8/2+8,-8) {};
\node[box=2] (p-9-0) at (-9/2+0,-9) {};
\node[box=0-for-negatives] (p-9-1) at (-9/2+1,-9) {};
\node[box=0-for-negatives] (p-9-2) at (-9/2+2,-9) {};
\node[box=0-for-negatives] (p-9-3) at (-9/2+3,-9) {};
\node[box=0-for-negatives] (p-9-4) at (-9/2+4,-9) {};
\node[box=0-for-negatives] (p-9-5) at (-9/2+5,-9) {};
\node[box=0-for-negatives] (p-9-6) at (-9/2+6,-9) {};
\node[box=0-for-negatives] (p-9-7) at (-9/2+7,-9) {};
\node[box=0-for-negatives] (p-9-8) at (-9/2+8,-9) {};
\node[box=1-for-negatives] (p-9-9) at (-9/2+9,-9) {};
\node[box=1] (p-10-0) at (-10/2+0,-10) {};
\node[box=2-for-negatives] (p-10-1) at (-10/2+1,-10) {};
\node[box=0-for-negatives] (p-10-2) at (-10/2+2,-10) {};
\node[box=0-for-negatives] (p-10-3) at (-10/2+3,-10) {};
\node[box=0-for-negatives] (p-10-4) at (-10/2+4,-10) {};
\node[box=0-for-negatives] (p-10-5) at (-10/2+5,-10) {};
\node[box=0-for-negatives] (p-10-6) at (-10/2+6,-10) {};
\node[box=0-for-negatives] (p-10-7) at (-10/2+7,-10) {};
\node[box=0-for-negatives] (p-10-8) at (-10/2+8,-10) {};
\node[box=2-for-negatives] (p-10-9) at (-10/2+9,-10) {};
\node[box=1-for-negatives] (p-10-10) at (-10/2+10,-10) {};
\node[box=2] (p-11-0) at (-11/2+0,-11) {};
\node[box=2-for-negatives] (p-11-1) at (-11/2+1,-11) {};
\node[box=2-for-negatives] (p-11-2) at (-11/2+2,-11) {};
\node[box=0-for-negatives] (p-11-3) at (-11/2+3,-11) {};
\node[box=0-for-negatives] (p-11-4) at (-11/2+4,-11) {};
\node[box=0-for-negatives] (p-11-5) at (-11/2+5,-11) {};
\node[box=0-for-negatives] (p-11-6) at (-11/2+6,-11) {};
\node[box=0-for-negatives] (p-11-7) at (-11/2+7,-11) {};
\node[box=0-for-negatives] (p-11-8) at (-11/2+8,-11) {};
\node[box=1-for-negatives] (p-11-9) at (-11/2+9,-11) {};
\node[box=1-for-negatives] (p-11-10) at (-11/2+10,-11) {};
\node[box=1-for-negatives] (p-11-11) at (-11/2+11,-11) {};
\node[box=1] (p-12-0) at (-12/2+0,-12) {};
\node[box=0-for-negatives] (p-12-1) at (-12/2+1,-12) {};
\node[box=0-for-negatives] (p-12-2) at (-12/2+2,-12) {};
\node[box=2-for-negatives] (p-12-3) at (-12/2+3,-12) {};
\node[box=0-for-negatives] (p-12-4) at (-12/2+4,-12) {};
\node[box=0-for-negatives] (p-12-5) at (-12/2+5,-12) {};
\node[box=0-for-negatives] (p-12-6) at (-12/2+6,-12) {};
\node[box=0-for-negatives] (p-12-7) at (-12/2+7,-12) {};
\node[box=0-for-negatives] (p-12-8) at (-12/2+8,-12) {};
\node[box=2-for-negatives] (p-12-9) at (-12/2+9,-12) {};
\node[box=0-for-negatives] (p-12-10) at (-12/2+10,-12) {};
\node[box=0-for-negatives] (p-12-11) at (-12/2+11,-12) {};
\node[box=1-for-negatives] (p-12-12) at (-12/2+12,-12) {};
\node[box=2] (p-13-0) at (-13/2+0,-13) {};
\node[box=1-for-negatives] (p-13-1) at (-13/2+1,-13) {};
\node[box=0-for-negatives] (p-13-2) at (-13/2+2,-13) {};
\node[box=1-for-negatives] (p-13-3) at (-13/2+3,-13) {};
\node[box=2-for-negatives] (p-13-4) at (-13/2+4,-13) {};
\node[box=0-for-negatives] (p-13-5) at (-13/2+5,-13) {};
\node[box=0-for-negatives] (p-13-6) at (-13/2+6,-13) {};
\node[box=0-for-negatives] (p-13-7) at (-13/2+7,-13) {};
\node[box=0-for-negatives] (p-13-8) at (-13/2+8,-13) {};
\node[box=1-for-negatives] (p-13-9) at (-13/2+9,-13) {};
\node[box=2-for-negatives] (p-13-10) at (-13/2+10,-13) {};
\node[box=0-for-negatives] (p-13-11) at (-13/2+11,-13) {};
\node[box=2-for-negatives] (p-13-12) at (-13/2+12,-13) {};
\node[box=1-for-negatives] (p-13-13) at (-13/2+13,-13) {};
\node[box=1] (p-14-0) at (-14/2+0,-14) {};
\node[box=1-for-negatives] (p-14-1) at (-14/2+1,-14) {};
\node[box=1-for-negatives] (p-14-2) at (-14/2+2,-14) {};
\node[box=2-for-negatives] (p-14-3) at (-14/2+3,-14) {};
\node[box=2-for-negatives] (p-14-4) at (-14/2+4,-14) {};
\node[box=2-for-negatives] (p-14-5) at (-14/2+5,-14) {};
\node[box=0-for-negatives] (p-14-6) at (-14/2+6,-14) {};
\node[box=0-for-negatives] (p-14-7) at (-14/2+7,-14) {};
\node[box=0-for-negatives] (p-14-8) at (-14/2+8,-14) {};
\node[box=2-for-negatives] (p-14-9) at (-14/2+9,-14) {};
\node[box=2-for-negatives] (p-14-10) at (-14/2+10,-14) {};
\node[box=2-for-negatives] (p-14-11) at (-14/2+11,-14) {};
\node[box=1-for-negatives] (p-14-12) at (-14/2+12,-14) {};
\node[box=1-for-negatives] (p-14-13) at (-14/2+13,-14) {};
\node[box=1-for-negatives] (p-14-14) at (-14/2+14,-14) {};
\node[box=2] (p-15-0) at (-15/2+0,-15) {};
\node[box=0-for-negatives] (p-15-1) at (-15/2+1,-15) {};
\node[box=0-for-negatives] (p-15-2) at (-15/2+2,-15) {};
\node[box=2-for-negatives] (p-15-3) at (-15/2+3,-15) {};
\node[box=0-for-negatives] (p-15-4) at (-15/2+4,-15) {};
\node[box=0-for-negatives] (p-15-5) at (-15/2+5,-15) {};
\node[box=2-for-negatives] (p-15-6) at (-15/2+6,-15) {};
\node[box=0-for-negatives] (p-15-7) at (-15/2+7,-15) {};
\node[box=0-for-negatives] (p-15-8) at (-15/2+8,-15) {};
\node[box=1-for-negatives] (p-15-9) at (-15/2+9,-15) {};
\node[box=0-for-negatives] (p-15-10) at (-15/2+10,-15) {};
\node[box=0-for-negatives] (p-15-11) at (-15/2+11,-15) {};
\node[box=1-for-negatives] (p-15-12) at (-15/2+12,-15) {};
\node[box=0-for-negatives] (p-15-13) at (-15/2+13,-15) {};
\node[box=0-for-negatives] (p-15-14) at (-15/2+14,-15) {};
\node[box=1-for-negatives] (p-15-15) at (-15/2+15,-15) {};
\node[box=1] (p-16-0) at (-16/2+0,-16) {};
\node[box=2-for-negatives] (p-16-1) at (-16/2+1,-16) {};
\node[box=0-for-negatives] (p-16-2) at (-16/2+2,-16) {};
\node[box=1-for-negatives] (p-16-3) at (-16/2+3,-16) {};
\node[box=2-for-negatives] (p-16-4) at (-16/2+4,-16) {};
\node[box=0-for-negatives] (p-16-5) at (-16/2+5,-16) {};
\node[box=1-for-negatives] (p-16-6) at (-16/2+6,-16) {};
\node[box=2-for-negatives] (p-16-7) at (-16/2+7,-16) {};
\node[box=0-for-negatives] (p-16-8) at (-16/2+8,-16) {};
\node[box=2-for-negatives] (p-16-9) at (-16/2+9,-16) {};
\node[box=1-for-negatives] (p-16-10) at (-16/2+10,-16) {};
\node[box=0-for-negatives] (p-16-11) at (-16/2+11,-16) {};
\node[box=2-for-negatives] (p-16-12) at (-16/2+12,-16) {};
\node[box=1-for-negatives] (p-16-13) at (-16/2+13,-16) {};
\node[box=0-for-negatives] (p-16-14) at (-16/2+14,-16) {};
\node[box=2-for-negatives] (p-16-15) at (-16/2+15,-16) {};
\node[box=1-for-negatives] (p-16-16) at (-16/2+16,-16) {};
\node[box=2] (p-17-0) at (-17/2+0,-17) {};
\node[box=2-for-negatives] (p-17-1) at (-17/2+1,-17) {};
\node[box=2-for-negatives] (p-17-2) at (-17/2+2,-17) {};
\node[box=2-for-negatives] (p-17-3) at (-17/2+3,-17) {};
\node[box=2-for-negatives] (p-17-4) at (-17/2+4,-17) {};
\node[box=2-for-negatives] (p-17-5) at (-17/2+5,-17) {};
\node[box=2-for-negatives] (p-17-6) at (-17/2+6,-17) {};
\node[box=2-for-negatives] (p-17-7) at (-17/2+7,-17) {};
\node[box=2-for-negatives] (p-17-8) at (-17/2+8,-17) {};
\node[box=1-for-negatives] (p-17-9) at (-17/2+9,-17) {};
\node[box=1-for-negatives] (p-17-10) at (-17/2+10,-17) {};
\node[box=1-for-negatives] (p-17-11) at (-17/2+11,-17) {};
\node[box=1-for-negatives] (p-17-12) at (-17/2+12,-17) {};
\node[box=1-for-negatives] (p-17-13) at (-17/2+13,-17) {};
\node[box=1-for-negatives] (p-17-14) at (-17/2+14,-17) {};
\node[box=1-for-negatives] (p-17-15) at (-17/2+15,-17) {};
\node[box=1-for-negatives] (p-17-16) at (-17/2+16,-17) {};
\node[box=1-for-negatives] (p-17-17) at (-17/2+17,-17) {};
\node[box=1] (p-18-0) at (-18/2+0,-18) {};
\node[box=0-for-negatives] (p-18-1) at (-18/2+1,-18) {};
\node[box=0-for-negatives] (p-18-2) at (-18/2+2,-18) {};
\node[box=0-for-negatives] (p-18-3) at (-18/2+3,-18) {};
\node[box=0-for-negatives] (p-18-4) at (-18/2+4,-18) {};
\node[box=0-for-negatives] (p-18-5) at (-18/2+5,-18) {};
\node[box=0-for-negatives] (p-18-6) at (-18/2+6,-18) {};
\node[box=0-for-negatives] (p-18-7) at (-18/2+7,-18) {};
\node[box=0-for-negatives] (p-18-8) at (-18/2+8,-18) {};
\node[box=1-for-negatives] (p-18-9) at (-18/2+9,-18) {};
\node[box=0-for-negatives] (p-18-10) at (-18/2+10,-18) {};
\node[box=0-for-negatives] (p-18-11) at (-18/2+11,-18) {};
\node[box=0-for-negatives] (p-18-12) at (-18/2+12,-18) {};
\node[box=0-for-negatives] (p-18-13) at (-18/2+13,-18) {};
\node[box=0-for-negatives] (p-18-14) at (-18/2+14,-18) {};
\node[box=0-for-negatives] (p-18-15) at (-18/2+15,-18) {};
\node[box=0-for-negatives] (p-18-16) at (-18/2+16,-18) {};
\node[box=0-for-negatives] (p-18-17) at (-18/2+17,-18) {};
\node[box=1-for-negatives] (p-18-18) at (-18/2+18,-18) {};
\node[box=2] (p-19-0) at (-19/2+0,-19) {};
\node[box=1-for-negatives] (p-19-1) at (-19/2+1,-19) {};
\node[box=0-for-negatives] (p-19-2) at (-19/2+2,-19) {};
\node[box=0-for-negatives] (p-19-3) at (-19/2+3,-19) {};
\node[box=0-for-negatives] (p-19-4) at (-19/2+4,-19) {};
\node[box=0-for-negatives] (p-19-5) at (-19/2+5,-19) {};
\node[box=0-for-negatives] (p-19-6) at (-19/2+6,-19) {};
\node[box=0-for-negatives] (p-19-7) at (-19/2+7,-19) {};
\node[box=0-for-negatives] (p-19-8) at (-19/2+8,-19) {};
\node[box=2-for-negatives] (p-19-9) at (-19/2+9,-19) {};
\node[box=1-for-negatives] (p-19-10) at (-19/2+10,-19) {};
\node[box=0-for-negatives] (p-19-11) at (-19/2+11,-19) {};
\node[box=0-for-negatives] (p-19-12) at (-19/2+12,-19) {};
\node[box=0-for-negatives] (p-19-13) at (-19/2+13,-19) {};
\node[box=0-for-negatives] (p-19-14) at (-19/2+14,-19) {};
\node[box=0-for-negatives] (p-19-15) at (-19/2+15,-19) {};
\node[box=0-for-negatives] (p-19-16) at (-19/2+16,-19) {};
\node[box=0-for-negatives] (p-19-17) at (-19/2+17,-19) {};
\node[box=2-for-negatives] (p-19-18) at (-19/2+18,-19) {};
\node[box=1-for-negatives] (p-19-19) at (-19/2+19,-19) {};
\node[box=1] (p-20-0) at (-20/2+0,-20) {};
\node[box=1-for-negatives] (p-20-1) at (-20/2+1,-20) {};
\node[box=1-for-negatives] (p-20-2) at (-20/2+2,-20) {};
\node[box=0-for-negatives] (p-20-3) at (-20/2+3,-20) {};
\node[box=0-for-negatives] (p-20-4) at (-20/2+4,-20) {};
\node[box=0-for-negatives] (p-20-5) at (-20/2+5,-20) {};
\node[box=0-for-negatives] (p-20-6) at (-20/2+6,-20) {};
\node[box=0-for-negatives] (p-20-7) at (-20/2+7,-20) {};
\node[box=0-for-negatives] (p-20-8) at (-20/2+8,-20) {};
\node[box=1-for-negatives] (p-20-9) at (-20/2+9,-20) {};
\node[box=1-for-negatives] (p-20-10) at (-20/2+10,-20) {};
\node[box=1-for-negatives] (p-20-11) at (-20/2+11,-20) {};
\node[box=0-for-negatives] (p-20-12) at (-20/2+12,-20) {};
\node[box=0-for-negatives] (p-20-13) at (-20/2+13,-20) {};
\node[box=0-for-negatives] (p-20-14) at (-20/2+14,-20) {};
\node[box=0-for-negatives] (p-20-15) at (-20/2+15,-20) {};
\node[box=0-for-negatives] (p-20-16) at (-20/2+16,-20) {};
\node[box=0-for-negatives] (p-20-17) at (-20/2+17,-20) {};
\node[box=1-for-negatives] (p-20-18) at (-20/2+18,-20) {};
\node[box=1-for-negatives] (p-20-19) at (-20/2+19,-20) {};
\node[box=1-for-negatives] (p-20-20) at (-20/2+20,-20) {};
\node[box=2] (p-21-0) at (-21/2+0,-21) {};
\node[box=0-for-negatives] (p-21-1) at (-21/2+1,-21) {};
\node[box=0-for-negatives] (p-21-2) at (-21/2+2,-21) {};
\node[box=1-for-negatives] (p-21-3) at (-21/2+3,-21) {};
\node[box=0-for-negatives] (p-21-4) at (-21/2+4,-21) {};
\node[box=0-for-negatives] (p-21-5) at (-21/2+5,-21) {};
\node[box=0-for-negatives] (p-21-6) at (-21/2+6,-21) {};
\node[box=0-for-negatives] (p-21-7) at (-21/2+7,-21) {};
\node[box=0-for-negatives] (p-21-8) at (-21/2+8,-21) {};
\node[box=2-for-negatives] (p-21-9) at (-21/2+9,-21) {};
\node[box=0-for-negatives] (p-21-10) at (-21/2+10,-21) {};
\node[box=0-for-negatives] (p-21-11) at (-21/2+11,-21) {};
\node[box=1-for-negatives] (p-21-12) at (-21/2+12,-21) {};
\node[box=0-for-negatives] (p-21-13) at (-21/2+13,-21) {};
\node[box=0-for-negatives] (p-21-14) at (-21/2+14,-21) {};
\node[box=0-for-negatives] (p-21-15) at (-21/2+15,-21) {};
\node[box=0-for-negatives] (p-21-16) at (-21/2+16,-21) {};
\node[box=0-for-negatives] (p-21-17) at (-21/2+17,-21) {};
\node[box=2-for-negatives] (p-21-18) at (-21/2+18,-21) {};
\node[box=0-for-negatives] (p-21-19) at (-21/2+19,-21) {};
\node[box=0-for-negatives] (p-21-20) at (-21/2+20,-21) {};
\node[box=1-for-negatives] (p-21-21) at (-21/2+21,-21) {};
\node[box=1] (p-22-0) at (-22/2+0,-22) {};
\node[box=2-for-negatives] (p-22-1) at (-22/2+1,-22) {};
\node[box=0-for-negatives] (p-22-2) at (-22/2+2,-22) {};
\node[box=2-for-negatives] (p-22-3) at (-22/2+3,-22) {};
\node[box=1-for-negatives] (p-22-4) at (-22/2+4,-22) {};
\node[box=0-for-negatives] (p-22-5) at (-22/2+5,-22) {};
\node[box=0-for-negatives] (p-22-6) at (-22/2+6,-22) {};
\node[box=0-for-negatives] (p-22-7) at (-22/2+7,-22) {};
\node[box=0-for-negatives] (p-22-8) at (-22/2+8,-22) {};
\node[box=1-for-negatives] (p-22-9) at (-22/2+9,-22) {};
\node[box=2-for-negatives] (p-22-10) at (-22/2+10,-22) {};
\node[box=0-for-negatives] (p-22-11) at (-22/2+11,-22) {};
\node[box=2-for-negatives] (p-22-12) at (-22/2+12,-22) {};
\node[box=1-for-negatives] (p-22-13) at (-22/2+13,-22) {};
\node[box=0-for-negatives] (p-22-14) at (-22/2+14,-22) {};
\node[box=0-for-negatives] (p-22-15) at (-22/2+15,-22) {};
\node[box=0-for-negatives] (p-22-16) at (-22/2+16,-22) {};
\node[box=0-for-negatives] (p-22-17) at (-22/2+17,-22) {};
\node[box=1-for-negatives] (p-22-18) at (-22/2+18,-22) {};
\node[box=2-for-negatives] (p-22-19) at (-22/2+19,-22) {};
\node[box=0-for-negatives] (p-22-20) at (-22/2+20,-22) {};
\node[box=2-for-negatives] (p-22-21) at (-22/2+21,-22) {};
\node[box=1-for-negatives] (p-22-22) at (-22/2+22,-22) {};
\node[box=2] (p-23-0) at (-23/2+0,-23) {};
\node[box=2-for-negatives] (p-23-1) at (-23/2+1,-23) {};
\node[box=2-for-negatives] (p-23-2) at (-23/2+2,-23) {};
\node[box=1-for-negatives] (p-23-3) at (-23/2+3,-23) {};
\node[box=1-for-negatives] (p-23-4) at (-23/2+4,-23) {};
\node[box=1-for-negatives] (p-23-5) at (-23/2+5,-23) {};
\node[box=0-for-negatives] (p-23-6) at (-23/2+6,-23) {};
\node[box=0-for-negatives] (p-23-7) at (-23/2+7,-23) {};
\node[box=0-for-negatives] (p-23-8) at (-23/2+8,-23) {};
\node[box=2-for-negatives] (p-23-9) at (-23/2+9,-23) {};
\node[box=2-for-negatives] (p-23-10) at (-23/2+10,-23) {};
\node[box=2-for-negatives] (p-23-11) at (-23/2+11,-23) {};
\node[box=1-for-negatives] (p-23-12) at (-23/2+12,-23) {};
\node[box=1-for-negatives] (p-23-13) at (-23/2+13,-23) {};
\node[box=1-for-negatives] (p-23-14) at (-23/2+14,-23) {};
\node[box=0-for-negatives] (p-23-15) at (-23/2+15,-23) {};
\node[box=0-for-negatives] (p-23-16) at (-23/2+16,-23) {};
\node[box=0-for-negatives] (p-23-17) at (-23/2+17,-23) {};
\node[box=2-for-negatives] (p-23-18) at (-23/2+18,-23) {};
\node[box=2-for-negatives] (p-23-19) at (-23/2+19,-23) {};
\node[box=2-for-negatives] (p-23-20) at (-23/2+20,-23) {};
\node[box=1-for-negatives] (p-23-21) at (-23/2+21,-23) {};
\node[box=1-for-negatives] (p-23-22) at (-23/2+22,-23) {};
\node[box=1-for-negatives] (p-23-23) at (-23/2+23,-23) {};
\node[box=1] (p-24-0) at (-24/2+0,-24) {};
\node[box=0-for-negatives] (p-24-1) at (-24/2+1,-24) {};
\node[box=0-for-negatives] (p-24-2) at (-24/2+2,-24) {};
\node[box=1-for-negatives] (p-24-3) at (-24/2+3,-24) {};
\node[box=0-for-negatives] (p-24-4) at (-24/2+4,-24) {};
\node[box=0-for-negatives] (p-24-5) at (-24/2+5,-24) {};
\node[box=1-for-negatives] (p-24-6) at (-24/2+6,-24) {};
\node[box=0-for-negatives] (p-24-7) at (-24/2+7,-24) {};
\node[box=0-for-negatives] (p-24-8) at (-24/2+8,-24) {};
\node[box=1-for-negatives] (p-24-9) at (-24/2+9,-24) {};
\node[box=0-for-negatives] (p-24-10) at (-24/2+10,-24) {};
\node[box=0-for-negatives] (p-24-11) at (-24/2+11,-24) {};
\node[box=1-for-negatives] (p-24-12) at (-24/2+12,-24) {};
\node[box=0-for-negatives] (p-24-13) at (-24/2+13,-24) {};
\node[box=0-for-negatives] (p-24-14) at (-24/2+14,-24) {};
\node[box=1-for-negatives] (p-24-15) at (-24/2+15,-24) {};
\node[box=0-for-negatives] (p-24-16) at (-24/2+16,-24) {};
\node[box=0-for-negatives] (p-24-17) at (-24/2+17,-24) {};
\node[box=1-for-negatives] (p-24-18) at (-24/2+18,-24) {};
\node[box=0-for-negatives] (p-24-19) at (-24/2+19,-24) {};
\node[box=0-for-negatives] (p-24-20) at (-24/2+20,-24) {};
\node[box=1-for-negatives] (p-24-21) at (-24/2+21,-24) {};
\node[box=0-for-negatives] (p-24-22) at (-24/2+22,-24) {};
\node[box=0-for-negatives] (p-24-23) at (-24/2+23,-24) {};
\node[box=1-for-negatives] (p-24-24) at (-24/2+24,-24) {};
\node[box=2] (p-25-0) at (-25/2+0,-25) {};
\node[box=1-for-negatives] (p-25-1) at (-25/2+1,-25) {};
\node[box=0-for-negatives] (p-25-2) at (-25/2+2,-25) {};
\node[box=2-for-negatives] (p-25-3) at (-25/2+3,-25) {};
\node[box=1-for-negatives] (p-25-4) at (-25/2+4,-25) {};
\node[box=0-for-negatives] (p-25-5) at (-25/2+5,-25) {};
\node[box=2-for-negatives] (p-25-6) at (-25/2+6,-25) {};
\node[box=1-for-negatives] (p-25-7) at (-25/2+7,-25) {};
\node[box=0-for-negatives] (p-25-8) at (-25/2+8,-25) {};
\node[box=2-for-negatives] (p-25-9) at (-25/2+9,-25) {};
\node[box=1-for-negatives] (p-25-10) at (-25/2+10,-25) {};
\node[box=0-for-negatives] (p-25-11) at (-25/2+11,-25) {};
\node[box=2-for-negatives] (p-25-12) at (-25/2+12,-25) {};
\node[box=1-for-negatives] (p-25-13) at (-25/2+13,-25) {};
\node[box=0-for-negatives] (p-25-14) at (-25/2+14,-25) {};
\node[box=2-for-negatives] (p-25-15) at (-25/2+15,-25) {};
\node[box=1-for-negatives] (p-25-16) at (-25/2+16,-25) {};
\node[box=0-for-negatives] (p-25-17) at (-25/2+17,-25) {};
\node[box=2-for-negatives] (p-25-18) at (-25/2+18,-25) {};
\node[box=1-for-negatives] (p-25-19) at (-25/2+19,-25) {};
\node[box=0-for-negatives] (p-25-20) at (-25/2+20,-25) {};
\node[box=2-for-negatives] (p-25-21) at (-25/2+21,-25) {};
\node[box=1-for-negatives] (p-25-22) at (-25/2+22,-25) {};
\node[box=0-for-negatives] (p-25-23) at (-25/2+23,-25) {};
\node[box=2-for-negatives] (p-25-24) at (-25/2+24,-25) {};
\node[box=1-for-negatives] (p-25-25) at (-25/2+25,-25) {};
\node[box=1] (p-26-0) at (-26/2+0,-26) {};
\node[box=1-for-negatives] (p-26-1) at (-26/2+1,-26) {};
\node[box=1-for-negatives] (p-26-2) at (-26/2+2,-26) {};
\node[box=1-for-negatives] (p-26-3) at (-26/2+3,-26) {};
\node[box=1-for-negatives] (p-26-4) at (-26/2+4,-26) {};
\node[box=1-for-negatives] (p-26-5) at (-26/2+5,-26) {};
\node[box=1-for-negatives] (p-26-6) at (-26/2+6,-26) {};
\node[box=1-for-negatives] (p-26-7) at (-26/2+7,-26) {};
\node[box=1-for-negatives] (p-26-8) at (-26/2+8,-26) {};
\node[box=1-for-negatives] (p-26-9) at (-26/2+9,-26) {};
\node[box=1-for-negatives] (p-26-10) at (-26/2+10,-26) {};
\node[box=1-for-negatives] (p-26-11) at (-26/2+11,-26) {};
\node[box=1-for-negatives] (p-26-12) at (-26/2+12,-26) {};
\node[box=1-for-negatives] (p-26-13) at (-26/2+13,-26) {};
\node[box=1-for-negatives] (p-26-14) at (-26/2+14,-26) {};
\node[box=1-for-negatives] (p-26-15) at (-26/2+15,-26) {};
\node[box=1-for-negatives] (p-26-16) at (-26/2+16,-26) {};
\node[box=1-for-negatives] (p-26-17) at (-26/2+17,-26) {};
\node[box=1-for-negatives] (p-26-18) at (-26/2+18,-26) {};
\node[box=1-for-negatives] (p-26-19) at (-26/2+19,-26) {};
\node[box=1-for-negatives] (p-26-20) at (-26/2+20,-26) {};
\node[box=1-for-negatives] (p-26-21) at (-26/2+21,-26) {};
\node[box=1-for-negatives] (p-26-22) at (-26/2+22,-26) {};
\node[box=1-for-negatives] (p-26-23) at (-26/2+23,-26) {};
\node[box=1-for-negatives] (p-26-24) at (-26/2+24,-26) {};
\node[box=1-for-negatives] (p-26-25) at (-26/2+25,-26) {};
\node[box=1-for-negatives] (p-26-26) at (-26/2+26,-26) {};
\node[box=2] (p-27-0) at (-27/2+0,-27) {};
\node[box=0-for-negatives] (p-27-1) at (-27/2+1,-27) {};
\node[box=0-for-negatives] (p-27-2) at (-27/2+2,-27) {};
\node[box=0-for-negatives] (p-27-3) at (-27/2+3,-27) {};
\node[box=0-for-negatives] (p-27-4) at (-27/2+4,-27) {};
\node[box=0-for-negatives] (p-27-5) at (-27/2+5,-27) {};
\node[box=0-for-negatives] (p-27-6) at (-27/2+6,-27) {};
\node[box=0-for-negatives] (p-27-7) at (-27/2+7,-27) {};
\node[box=0-for-negatives] (p-27-8) at (-27/2+8,-27) {};
\node[box=0-for-negatives] (p-27-9) at (-27/2+9,-27) {};
\node[box=0-for-negatives] (p-27-10) at (-27/2+10,-27) {};
\node[box=0-for-negatives] (p-27-11) at (-27/2+11,-27) {};
\node[box=0-for-negatives] (p-27-12) at (-27/2+12,-27) {};
\node[box=0-for-negatives] (p-27-13) at (-27/2+13,-27) {};
\node[box=0-for-negatives] (p-27-14) at (-27/2+14,-27) {};
\node[box=0-for-negatives] (p-27-15) at (-27/2+15,-27) {};
\node[box=0-for-negatives] (p-27-16) at (-27/2+16,-27) {};
\node[box=0-for-negatives] (p-27-17) at (-27/2+17,-27) {};
\node[box=0-for-negatives] (p-27-18) at (-27/2+18,-27) {};
\node[box=0-for-negatives] (p-27-19) at (-27/2+19,-27) {};
\node[box=0-for-negatives] (p-27-20) at (-27/2+20,-27) {};
\node[box=0-for-negatives] (p-27-21) at (-27/2+21,-27) {};
\node[box=0-for-negatives] (p-27-22) at (-27/2+22,-27) {};
\node[box=0-for-negatives] (p-27-23) at (-27/2+23,-27) {};
\node[box=0-for-negatives] (p-27-24) at (-27/2+24,-27) {};
\node[box=0-for-negatives] (p-27-25) at (-27/2+25,-27) {};
\node[box=0-for-negatives] (p-27-26) at (-27/2+26,-27) {};
\node[box=1-for-negatives] (p-27-27) at (-27/2+27,-27) {};
\node[box=1] (p-28-0) at (-28/2+0,-28) {};
\node[box=2-for-negatives] (p-28-1) at (-28/2+1,-28) {};
\node[box=0-for-negatives] (p-28-2) at (-28/2+2,-28) {};
\node[box=0-for-negatives] (p-28-3) at (-28/2+3,-28) {};
\node[box=0-for-negatives] (p-28-4) at (-28/2+4,-28) {};
\node[box=0-for-negatives] (p-28-5) at (-28/2+5,-28) {};
\node[box=0-for-negatives] (p-28-6) at (-28/2+6,-28) {};
\node[box=0-for-negatives] (p-28-7) at (-28/2+7,-28) {};
\node[box=0-for-negatives] (p-28-8) at (-28/2+8,-28) {};
\node[box=0-for-negatives] (p-28-9) at (-28/2+9,-28) {};
\node[box=0-for-negatives] (p-28-10) at (-28/2+10,-28) {};
\node[box=0-for-negatives] (p-28-11) at (-28/2+11,-28) {};
\node[box=0-for-negatives] (p-28-12) at (-28/2+12,-28) {};
\node[box=0-for-negatives] (p-28-13) at (-28/2+13,-28) {};
\node[box=0-for-negatives] (p-28-14) at (-28/2+14,-28) {};
\node[box=0-for-negatives] (p-28-15) at (-28/2+15,-28) {};
\node[box=0-for-negatives] (p-28-16) at (-28/2+16,-28) {};
\node[box=0-for-negatives] (p-28-17) at (-28/2+17,-28) {};
\node[box=0-for-negatives] (p-28-18) at (-28/2+18,-28) {};
\node[box=0-for-negatives] (p-28-19) at (-28/2+19,-28) {};
\node[box=0-for-negatives] (p-28-20) at (-28/2+20,-28) {};
\node[box=0-for-negatives] (p-28-21) at (-28/2+21,-28) {};
\node[box=0-for-negatives] (p-28-22) at (-28/2+22,-28) {};
\node[box=0-for-negatives] (p-28-23) at (-28/2+23,-28) {};
\node[box=0-for-negatives] (p-28-24) at (-28/2+24,-28) {};
\node[box=0-for-negatives] (p-28-25) at (-28/2+25,-28) {};
\node[box=0-for-negatives] (p-28-26) at (-28/2+26,-28) {};
\node[box=2-for-negatives] (p-28-27) at (-28/2+27,-28) {};
\node[box=1-for-negatives] (p-28-28) at (-28/2+28,-28) {};
\node[box=2] (p-29-0) at (-29/2+0,-29) {};
\node[box=2-for-negatives] (p-29-1) at (-29/2+1,-29) {};
\node[box=2-for-negatives] (p-29-2) at (-29/2+2,-29) {};
\node[box=0-for-negatives] (p-29-3) at (-29/2+3,-29) {};
\node[box=0-for-negatives] (p-29-4) at (-29/2+4,-29) {};
\node[box=0-for-negatives] (p-29-5) at (-29/2+5,-29) {};
\node[box=0-for-negatives] (p-29-6) at (-29/2+6,-29) {};
\node[box=0-for-negatives] (p-29-7) at (-29/2+7,-29) {};
\node[box=0-for-negatives] (p-29-8) at (-29/2+8,-29) {};
\node[box=0-for-negatives] (p-29-9) at (-29/2+9,-29) {};
\node[box=0-for-negatives] (p-29-10) at (-29/2+10,-29) {};
\node[box=0-for-negatives] (p-29-11) at (-29/2+11,-29) {};
\node[box=0-for-negatives] (p-29-12) at (-29/2+12,-29) {};
\node[box=0-for-negatives] (p-29-13) at (-29/2+13,-29) {};
\node[box=0-for-negatives] (p-29-14) at (-29/2+14,-29) {};
\node[box=0-for-negatives] (p-29-15) at (-29/2+15,-29) {};
\node[box=0-for-negatives] (p-29-16) at (-29/2+16,-29) {};
\node[box=0-for-negatives] (p-29-17) at (-29/2+17,-29) {};
\node[box=0-for-negatives] (p-29-18) at (-29/2+18,-29) {};
\node[box=0-for-negatives] (p-29-19) at (-29/2+19,-29) {};
\node[box=0-for-negatives] (p-29-20) at (-29/2+20,-29) {};
\node[box=0-for-negatives] (p-29-21) at (-29/2+21,-29) {};
\node[box=0-for-negatives] (p-29-22) at (-29/2+22,-29) {};
\node[box=0-for-negatives] (p-29-23) at (-29/2+23,-29) {};
\node[box=0-for-negatives] (p-29-24) at (-29/2+24,-29) {};
\node[box=0-for-negatives] (p-29-25) at (-29/2+25,-29) {};
\node[box=0-for-negatives] (p-29-26) at (-29/2+26,-29) {};
\node[box=1-for-negatives] (p-29-27) at (-29/2+27,-29) {};
\node[box=1-for-negatives] (p-29-28) at (-29/2+28,-29) {};
\node[box=1-for-negatives] (p-29-29) at (-29/2+29,-29) {};
\node[box=1] (p-30-0) at (-30/2+0,-30) {};
\node[box=0-for-negatives] (p-30-1) at (-30/2+1,-30) {};
\node[box=0-for-negatives] (p-30-2) at (-30/2+2,-30) {};
\node[box=2-for-negatives] (p-30-3) at (-30/2+3,-30) {};
\node[box=0-for-negatives] (p-30-4) at (-30/2+4,-30) {};
\node[box=0-for-negatives] (p-30-5) at (-30/2+5,-30) {};
\node[box=0-for-negatives] (p-30-6) at (-30/2+6,-30) {};
\node[box=0-for-negatives] (p-30-7) at (-30/2+7,-30) {};
\node[box=0-for-negatives] (p-30-8) at (-30/2+8,-30) {};
\node[box=0-for-negatives] (p-30-9) at (-30/2+9,-30) {};
\node[box=0-for-negatives] (p-30-10) at (-30/2+10,-30) {};
\node[box=0-for-negatives] (p-30-11) at (-30/2+11,-30) {};
\node[box=0-for-negatives] (p-30-12) at (-30/2+12,-30) {};
\node[box=0-for-negatives] (p-30-13) at (-30/2+13,-30) {};
\node[box=0-for-negatives] (p-30-14) at (-30/2+14,-30) {};
\node[box=0-for-negatives] (p-30-15) at (-30/2+15,-30) {};
\node[box=0-for-negatives] (p-30-16) at (-30/2+16,-30) {};
\node[box=0-for-negatives] (p-30-17) at (-30/2+17,-30) {};
\node[box=0-for-negatives] (p-30-18) at (-30/2+18,-30) {};
\node[box=0-for-negatives] (p-30-19) at (-30/2+19,-30) {};
\node[box=0-for-negatives] (p-30-20) at (-30/2+20,-30) {};
\node[box=0-for-negatives] (p-30-21) at (-30/2+21,-30) {};
\node[box=0-for-negatives] (p-30-22) at (-30/2+22,-30) {};
\node[box=0-for-negatives] (p-30-23) at (-30/2+23,-30) {};
\node[box=0-for-negatives] (p-30-24) at (-30/2+24,-30) {};
\node[box=0-for-negatives] (p-30-25) at (-30/2+25,-30) {};
\node[box=0-for-negatives] (p-30-26) at (-30/2+26,-30) {};
\node[box=2-for-negatives] (p-30-27) at (-30/2+27,-30) {};
\node[box=0-for-negatives] (p-30-28) at (-30/2+28,-30) {};
\node[box=0-for-negatives] (p-30-29) at (-30/2+29,-30) {};
\node[box=1-for-negatives] (p-30-30) at (-30/2+30,-30) {};
\node[box=2] (p-31-0) at (-31/2+0,-31) {};
\node[box=1-for-negatives] (p-31-1) at (-31/2+1,-31) {};
\node[box=0-for-negatives] (p-31-2) at (-31/2+2,-31) {};
\node[box=1-for-negatives] (p-31-3) at (-31/2+3,-31) {};
\node[box=2-for-negatives] (p-31-4) at (-31/2+4,-31) {};
\node[box=0-for-negatives] (p-31-5) at (-31/2+5,-31) {};
\node[box=0-for-negatives] (p-31-6) at (-31/2+6,-31) {};
\node[box=0-for-negatives] (p-31-7) at (-31/2+7,-31) {};
\node[box=0-for-negatives] (p-31-8) at (-31/2+8,-31) {};
\node[box=0-for-negatives] (p-31-9) at (-31/2+9,-31) {};
\node[box=0-for-negatives] (p-31-10) at (-31/2+10,-31) {};
\node[box=0-for-negatives] (p-31-11) at (-31/2+11,-31) {};
\node[box=0-for-negatives] (p-31-12) at (-31/2+12,-31) {};
\node[box=0-for-negatives] (p-31-13) at (-31/2+13,-31) {};
\node[box=0-for-negatives] (p-31-14) at (-31/2+14,-31) {};
\node[box=0-for-negatives] (p-31-15) at (-31/2+15,-31) {};
\node[box=0-for-negatives] (p-31-16) at (-31/2+16,-31) {};
\node[box=0-for-negatives] (p-31-17) at (-31/2+17,-31) {};
\node[box=0-for-negatives] (p-31-18) at (-31/2+18,-31) {};
\node[box=0-for-negatives] (p-31-19) at (-31/2+19,-31) {};
\node[box=0-for-negatives] (p-31-20) at (-31/2+20,-31) {};
\node[box=0-for-negatives] (p-31-21) at (-31/2+21,-31) {};
\node[box=0-for-negatives] (p-31-22) at (-31/2+22,-31) {};
\node[box=0-for-negatives] (p-31-23) at (-31/2+23,-31) {};
\node[box=0-for-negatives] (p-31-24) at (-31/2+24,-31) {};
\node[box=0-for-negatives] (p-31-25) at (-31/2+25,-31) {};
\node[box=0-for-negatives] (p-31-26) at (-31/2+26,-31) {};
\node[box=1-for-negatives] (p-31-27) at (-31/2+27,-31) {};
\node[box=2-for-negatives] (p-31-28) at (-31/2+28,-31) {};
\node[box=0-for-negatives] (p-31-29) at (-31/2+29,-31) {};
\node[box=2-for-negatives] (p-31-30) at (-31/2+30,-31) {};
\node[box=1-for-negatives] (p-31-31) at (-31/2+31,-31) {};
\node[box=1] (p-32-0) at (-32/2+0,-32) {};
\node[box=1-for-negatives] (p-32-1) at (-32/2+1,-32) {};
\node[box=1-for-negatives] (p-32-2) at (-32/2+2,-32) {};
\node[box=2-for-negatives] (p-32-3) at (-32/2+3,-32) {};
\node[box=2-for-negatives] (p-32-4) at (-32/2+4,-32) {};
\node[box=2-for-negatives] (p-32-5) at (-32/2+5,-32) {};
\node[box=0-for-negatives] (p-32-6) at (-32/2+6,-32) {};
\node[box=0-for-negatives] (p-32-7) at (-32/2+7,-32) {};
\node[box=0-for-negatives] (p-32-8) at (-32/2+8,-32) {};
\node[box=0-for-negatives] (p-32-9) at (-32/2+9,-32) {};
\node[box=0-for-negatives] (p-32-10) at (-32/2+10,-32) {};
\node[box=0-for-negatives] (p-32-11) at (-32/2+11,-32) {};
\node[box=0-for-negatives] (p-32-12) at (-32/2+12,-32) {};
\node[box=0-for-negatives] (p-32-13) at (-32/2+13,-32) {};
\node[box=0-for-negatives] (p-32-14) at (-32/2+14,-32) {};
\node[box=0-for-negatives] (p-32-15) at (-32/2+15,-32) {};
\node[box=0-for-negatives] (p-32-16) at (-32/2+16,-32) {};
\node[box=0-for-negatives] (p-32-17) at (-32/2+17,-32) {};
\node[box=0-for-negatives] (p-32-18) at (-32/2+18,-32) {};
\node[box=0-for-negatives] (p-32-19) at (-32/2+19,-32) {};
\node[box=0-for-negatives] (p-32-20) at (-32/2+20,-32) {};
\node[box=0-for-negatives] (p-32-21) at (-32/2+21,-32) {};
\node[box=0-for-negatives] (p-32-22) at (-32/2+22,-32) {};
\node[box=0-for-negatives] (p-32-23) at (-32/2+23,-32) {};
\node[box=0-for-negatives] (p-32-24) at (-32/2+24,-32) {};
\node[box=0-for-negatives] (p-32-25) at (-32/2+25,-32) {};
\node[box=0-for-negatives] (p-32-26) at (-32/2+26,-32) {};
\node[box=2-for-negatives] (p-32-27) at (-32/2+27,-32) {};
\node[box=2-for-negatives] (p-32-28) at (-32/2+28,-32) {};
\node[box=2-for-negatives] (p-32-29) at (-32/2+29,-32) {};
\node[box=1-for-negatives] (p-32-30) at (-32/2+30,-32) {};
\node[box=1-for-negatives] (p-32-31) at (-32/2+31,-32) {};
\node[box=1-for-negatives] (p-32-32) at (-32/2+32,-32) {};
\node[box=2] (p-33-0) at (-33/2+0,-33) {};
\node[box=0-for-negatives] (p-33-1) at (-33/2+1,-33) {};
\node[box=0-for-negatives] (p-33-2) at (-33/2+2,-33) {};
\node[box=2-for-negatives] (p-33-3) at (-33/2+3,-33) {};
\node[box=0-for-negatives] (p-33-4) at (-33/2+4,-33) {};
\node[box=0-for-negatives] (p-33-5) at (-33/2+5,-33) {};
\node[box=2-for-negatives] (p-33-6) at (-33/2+6,-33) {};
\node[box=0-for-negatives] (p-33-7) at (-33/2+7,-33) {};
\node[box=0-for-negatives] (p-33-8) at (-33/2+8,-33) {};
\node[box=0-for-negatives] (p-33-9) at (-33/2+9,-33) {};
\node[box=0-for-negatives] (p-33-10) at (-33/2+10,-33) {};
\node[box=0-for-negatives] (p-33-11) at (-33/2+11,-33) {};
\node[box=0-for-negatives] (p-33-12) at (-33/2+12,-33) {};
\node[box=0-for-negatives] (p-33-13) at (-33/2+13,-33) {};
\node[box=0-for-negatives] (p-33-14) at (-33/2+14,-33) {};
\node[box=0-for-negatives] (p-33-15) at (-33/2+15,-33) {};
\node[box=0-for-negatives] (p-33-16) at (-33/2+16,-33) {};
\node[box=0-for-negatives] (p-33-17) at (-33/2+17,-33) {};
\node[box=0-for-negatives] (p-33-18) at (-33/2+18,-33) {};
\node[box=0-for-negatives] (p-33-19) at (-33/2+19,-33) {};
\node[box=0-for-negatives] (p-33-20) at (-33/2+20,-33) {};
\node[box=0-for-negatives] (p-33-21) at (-33/2+21,-33) {};
\node[box=0-for-negatives] (p-33-22) at (-33/2+22,-33) {};
\node[box=0-for-negatives] (p-33-23) at (-33/2+23,-33) {};
\node[box=0-for-negatives] (p-33-24) at (-33/2+24,-33) {};
\node[box=0-for-negatives] (p-33-25) at (-33/2+25,-33) {};
\node[box=0-for-negatives] (p-33-26) at (-33/2+26,-33) {};
\node[box=1-for-negatives] (p-33-27) at (-33/2+27,-33) {};
\node[box=0-for-negatives] (p-33-28) at (-33/2+28,-33) {};
\node[box=0-for-negatives] (p-33-29) at (-33/2+29,-33) {};
\node[box=1-for-negatives] (p-33-30) at (-33/2+30,-33) {};
\node[box=0-for-negatives] (p-33-31) at (-33/2+31,-33) {};
\node[box=0-for-negatives] (p-33-32) at (-33/2+32,-33) {};
\node[box=1-for-negatives] (p-33-33) at (-33/2+33,-33) {};
\node[box=1] (p-34-0) at (-34/2+0,-34) {};
\node[box=2-for-negatives] (p-34-1) at (-34/2+1,-34) {};
\node[box=0-for-negatives] (p-34-2) at (-34/2+2,-34) {};
\node[box=1-for-negatives] (p-34-3) at (-34/2+3,-34) {};
\node[box=2-for-negatives] (p-34-4) at (-34/2+4,-34) {};
\node[box=0-for-negatives] (p-34-5) at (-34/2+5,-34) {};
\node[box=1-for-negatives] (p-34-6) at (-34/2+6,-34) {};
\node[box=2-for-negatives] (p-34-7) at (-34/2+7,-34) {};
\node[box=0-for-negatives] (p-34-8) at (-34/2+8,-34) {};
\node[box=0-for-negatives] (p-34-9) at (-34/2+9,-34) {};
\node[box=0-for-negatives] (p-34-10) at (-34/2+10,-34) {};
\node[box=0-for-negatives] (p-34-11) at (-34/2+11,-34) {};
\node[box=0-for-negatives] (p-34-12) at (-34/2+12,-34) {};
\node[box=0-for-negatives] (p-34-13) at (-34/2+13,-34) {};
\node[box=0-for-negatives] (p-34-14) at (-34/2+14,-34) {};
\node[box=0-for-negatives] (p-34-15) at (-34/2+15,-34) {};
\node[box=0-for-negatives] (p-34-16) at (-34/2+16,-34) {};
\node[box=0-for-negatives] (p-34-17) at (-34/2+17,-34) {};
\node[box=0-for-negatives] (p-34-18) at (-34/2+18,-34) {};
\node[box=0-for-negatives] (p-34-19) at (-34/2+19,-34) {};
\node[box=0-for-negatives] (p-34-20) at (-34/2+20,-34) {};
\node[box=0-for-negatives] (p-34-21) at (-34/2+21,-34) {};
\node[box=0-for-negatives] (p-34-22) at (-34/2+22,-34) {};
\node[box=0-for-negatives] (p-34-23) at (-34/2+23,-34) {};
\node[box=0-for-negatives] (p-34-24) at (-34/2+24,-34) {};
\node[box=0-for-negatives] (p-34-25) at (-34/2+25,-34) {};
\node[box=0-for-negatives] (p-34-26) at (-34/2+26,-34) {};
\node[box=2-for-negatives] (p-34-27) at (-34/2+27,-34) {};
\node[box=1-for-negatives] (p-34-28) at (-34/2+28,-34) {};
\node[box=0-for-negatives] (p-34-29) at (-34/2+29,-34) {};
\node[box=2-for-negatives] (p-34-30) at (-34/2+30,-34) {};
\node[box=1-for-negatives] (p-34-31) at (-34/2+31,-34) {};
\node[box=0-for-negatives] (p-34-32) at (-34/2+32,-34) {};
\node[box=2-for-negatives] (p-34-33) at (-34/2+33,-34) {};
\node[box=1-for-negatives] (p-34-34) at (-34/2+34,-34) {};
\node[box=2] (p-35-0) at (-35/2+0,-35) {};
\node[box=2-for-negatives] (p-35-1) at (-35/2+1,-35) {};
\node[box=2-for-negatives] (p-35-2) at (-35/2+2,-35) {};
\node[box=2-for-negatives] (p-35-3) at (-35/2+3,-35) {};
\node[box=2-for-negatives] (p-35-4) at (-35/2+4,-35) {};
\node[box=2-for-negatives] (p-35-5) at (-35/2+5,-35) {};
\node[box=2-for-negatives] (p-35-6) at (-35/2+6,-35) {};
\node[box=2-for-negatives] (p-35-7) at (-35/2+7,-35) {};
\node[box=2-for-negatives] (p-35-8) at (-35/2+8,-35) {};
\node[box=0-for-negatives] (p-35-9) at (-35/2+9,-35) {};
\node[box=0-for-negatives] (p-35-10) at (-35/2+10,-35) {};
\node[box=0-for-negatives] (p-35-11) at (-35/2+11,-35) {};
\node[box=0-for-negatives] (p-35-12) at (-35/2+12,-35) {};
\node[box=0-for-negatives] (p-35-13) at (-35/2+13,-35) {};
\node[box=0-for-negatives] (p-35-14) at (-35/2+14,-35) {};
\node[box=0-for-negatives] (p-35-15) at (-35/2+15,-35) {};
\node[box=0-for-negatives] (p-35-16) at (-35/2+16,-35) {};
\node[box=0-for-negatives] (p-35-17) at (-35/2+17,-35) {};
\node[box=0-for-negatives] (p-35-18) at (-35/2+18,-35) {};
\node[box=0-for-negatives] (p-35-19) at (-35/2+19,-35) {};
\node[box=0-for-negatives] (p-35-20) at (-35/2+20,-35) {};
\node[box=0-for-negatives] (p-35-21) at (-35/2+21,-35) {};
\node[box=0-for-negatives] (p-35-22) at (-35/2+22,-35) {};
\node[box=0-for-negatives] (p-35-23) at (-35/2+23,-35) {};
\node[box=0-for-negatives] (p-35-24) at (-35/2+24,-35) {};
\node[box=0-for-negatives] (p-35-25) at (-35/2+25,-35) {};
\node[box=0-for-negatives] (p-35-26) at (-35/2+26,-35) {};
\node[box=1-for-negatives] (p-35-27) at (-35/2+27,-35) {};
\node[box=1-for-negatives] (p-35-28) at (-35/2+28,-35) {};
\node[box=1-for-negatives] (p-35-29) at (-35/2+29,-35) {};
\node[box=1-for-negatives] (p-35-30) at (-35/2+30,-35) {};
\node[box=1-for-negatives] (p-35-31) at (-35/2+31,-35) {};
\node[box=1-for-negatives] (p-35-32) at (-35/2+32,-35) {};
\node[box=1-for-negatives] (p-35-33) at (-35/2+33,-35) {};
\node[box=1-for-negatives] (p-35-34) at (-35/2+34,-35) {};
\node[box=1-for-negatives] (p-35-35) at (-35/2+35,-35) {};
\node[box=1] (p-36-0) at (-36/2+0,-36) {};
\node[box=0-for-negatives] (p-36-1) at (-36/2+1,-36) {};
\node[box=0-for-negatives] (p-36-2) at (-36/2+2,-36) {};
\node[box=0-for-negatives] (p-36-3) at (-36/2+3,-36) {};
\node[box=0-for-negatives] (p-36-4) at (-36/2+4,-36) {};
\node[box=0-for-negatives] (p-36-5) at (-36/2+5,-36) {};
\node[box=0-for-negatives] (p-36-6) at (-36/2+6,-36) {};
\node[box=0-for-negatives] (p-36-7) at (-36/2+7,-36) {};
\node[box=0-for-negatives] (p-36-8) at (-36/2+8,-36) {};
\node[box=2-for-negatives] (p-36-9) at (-36/2+9,-36) {};
\node[box=0-for-negatives] (p-36-10) at (-36/2+10,-36) {};
\node[box=0-for-negatives] (p-36-11) at (-36/2+11,-36) {};
\node[box=0-for-negatives] (p-36-12) at (-36/2+12,-36) {};
\node[box=0-for-negatives] (p-36-13) at (-36/2+13,-36) {};
\node[box=0-for-negatives] (p-36-14) at (-36/2+14,-36) {};
\node[box=0-for-negatives] (p-36-15) at (-36/2+15,-36) {};
\node[box=0-for-negatives] (p-36-16) at (-36/2+16,-36) {};
\node[box=0-for-negatives] (p-36-17) at (-36/2+17,-36) {};
\node[box=0-for-negatives] (p-36-18) at (-36/2+18,-36) {};
\node[box=0-for-negatives] (p-36-19) at (-36/2+19,-36) {};
\node[box=0-for-negatives] (p-36-20) at (-36/2+20,-36) {};
\node[box=0-for-negatives] (p-36-21) at (-36/2+21,-36) {};
\node[box=0-for-negatives] (p-36-22) at (-36/2+22,-36) {};
\node[box=0-for-negatives] (p-36-23) at (-36/2+23,-36) {};
\node[box=0-for-negatives] (p-36-24) at (-36/2+24,-36) {};
\node[box=0-for-negatives] (p-36-25) at (-36/2+25,-36) {};
\node[box=0-for-negatives] (p-36-26) at (-36/2+26,-36) {};
\node[box=2-for-negatives] (p-36-27) at (-36/2+27,-36) {};
\node[box=0-for-negatives] (p-36-28) at (-36/2+28,-36) {};
\node[box=0-for-negatives] (p-36-29) at (-36/2+29,-36) {};
\node[box=0-for-negatives] (p-36-30) at (-36/2+30,-36) {};
\node[box=0-for-negatives] (p-36-31) at (-36/2+31,-36) {};
\node[box=0-for-negatives] (p-36-32) at (-36/2+32,-36) {};
\node[box=0-for-negatives] (p-36-33) at (-36/2+33,-36) {};
\node[box=0-for-negatives] (p-36-34) at (-36/2+34,-36) {};
\node[box=0-for-negatives] (p-36-35) at (-36/2+35,-36) {};
\node[box=1-for-negatives] (p-36-36) at (-36/2+36,-36) {};
\node[box=2] (p-37-0) at (-37/2+0,-37) {};
\node[box=1-for-negatives] (p-37-1) at (-37/2+1,-37) {};
\node[box=0-for-negatives] (p-37-2) at (-37/2+2,-37) {};
\node[box=0-for-negatives] (p-37-3) at (-37/2+3,-37) {};
\node[box=0-for-negatives] (p-37-4) at (-37/2+4,-37) {};
\node[box=0-for-negatives] (p-37-5) at (-37/2+5,-37) {};
\node[box=0-for-negatives] (p-37-6) at (-37/2+6,-37) {};
\node[box=0-for-negatives] (p-37-7) at (-37/2+7,-37) {};
\node[box=0-for-negatives] (p-37-8) at (-37/2+8,-37) {};
\node[box=1-for-negatives] (p-37-9) at (-37/2+9,-37) {};
\node[box=2-for-negatives] (p-37-10) at (-37/2+10,-37) {};
\node[box=0-for-negatives] (p-37-11) at (-37/2+11,-37) {};
\node[box=0-for-negatives] (p-37-12) at (-37/2+12,-37) {};
\node[box=0-for-negatives] (p-37-13) at (-37/2+13,-37) {};
\node[box=0-for-negatives] (p-37-14) at (-37/2+14,-37) {};
\node[box=0-for-negatives] (p-37-15) at (-37/2+15,-37) {};
\node[box=0-for-negatives] (p-37-16) at (-37/2+16,-37) {};
\node[box=0-for-negatives] (p-37-17) at (-37/2+17,-37) {};
\node[box=0-for-negatives] (p-37-18) at (-37/2+18,-37) {};
\node[box=0-for-negatives] (p-37-19) at (-37/2+19,-37) {};
\node[box=0-for-negatives] (p-37-20) at (-37/2+20,-37) {};
\node[box=0-for-negatives] (p-37-21) at (-37/2+21,-37) {};
\node[box=0-for-negatives] (p-37-22) at (-37/2+22,-37) {};
\node[box=0-for-negatives] (p-37-23) at (-37/2+23,-37) {};
\node[box=0-for-negatives] (p-37-24) at (-37/2+24,-37) {};
\node[box=0-for-negatives] (p-37-25) at (-37/2+25,-37) {};
\node[box=0-for-negatives] (p-37-26) at (-37/2+26,-37) {};
\node[box=1-for-negatives] (p-37-27) at (-37/2+27,-37) {};
\node[box=2-for-negatives] (p-37-28) at (-37/2+28,-37) {};
\node[box=0-for-negatives] (p-37-29) at (-37/2+29,-37) {};
\node[box=0-for-negatives] (p-37-30) at (-37/2+30,-37) {};
\node[box=0-for-negatives] (p-37-31) at (-37/2+31,-37) {};
\node[box=0-for-negatives] (p-37-32) at (-37/2+32,-37) {};
\node[box=0-for-negatives] (p-37-33) at (-37/2+33,-37) {};
\node[box=0-for-negatives] (p-37-34) at (-37/2+34,-37) {};
\node[box=0-for-negatives] (p-37-35) at (-37/2+35,-37) {};
\node[box=2-for-negatives] (p-37-36) at (-37/2+36,-37) {};
\node[box=1-for-negatives] (p-37-37) at (-37/2+37,-37) {};
\node[box=1] (p-38-0) at (-38/2+0,-38) {};
\node[box=1-for-negatives] (p-38-1) at (-38/2+1,-38) {};
\node[box=1-for-negatives] (p-38-2) at (-38/2+2,-38) {};
\node[box=0-for-negatives] (p-38-3) at (-38/2+3,-38) {};
\node[box=0-for-negatives] (p-38-4) at (-38/2+4,-38) {};
\node[box=0-for-negatives] (p-38-5) at (-38/2+5,-38) {};
\node[box=0-for-negatives] (p-38-6) at (-38/2+6,-38) {};
\node[box=0-for-negatives] (p-38-7) at (-38/2+7,-38) {};
\node[box=0-for-negatives] (p-38-8) at (-38/2+8,-38) {};
\node[box=2-for-negatives] (p-38-9) at (-38/2+9,-38) {};
\node[box=2-for-negatives] (p-38-10) at (-38/2+10,-38) {};
\node[box=2-for-negatives] (p-38-11) at (-38/2+11,-38) {};
\node[box=0-for-negatives] (p-38-12) at (-38/2+12,-38) {};
\node[box=0-for-negatives] (p-38-13) at (-38/2+13,-38) {};
\node[box=0-for-negatives] (p-38-14) at (-38/2+14,-38) {};
\node[box=0-for-negatives] (p-38-15) at (-38/2+15,-38) {};
\node[box=0-for-negatives] (p-38-16) at (-38/2+16,-38) {};
\node[box=0-for-negatives] (p-38-17) at (-38/2+17,-38) {};
\node[box=0-for-negatives] (p-38-18) at (-38/2+18,-38) {};
\node[box=0-for-negatives] (p-38-19) at (-38/2+19,-38) {};
\node[box=0-for-negatives] (p-38-20) at (-38/2+20,-38) {};
\node[box=0-for-negatives] (p-38-21) at (-38/2+21,-38) {};
\node[box=0-for-negatives] (p-38-22) at (-38/2+22,-38) {};
\node[box=0-for-negatives] (p-38-23) at (-38/2+23,-38) {};
\node[box=0-for-negatives] (p-38-24) at (-38/2+24,-38) {};
\node[box=0-for-negatives] (p-38-25) at (-38/2+25,-38) {};
\node[box=0-for-negatives] (p-38-26) at (-38/2+26,-38) {};
\node[box=2-for-negatives] (p-38-27) at (-38/2+27,-38) {};
\node[box=2-for-negatives] (p-38-28) at (-38/2+28,-38) {};
\node[box=2-for-negatives] (p-38-29) at (-38/2+29,-38) {};
\node[box=0-for-negatives] (p-38-30) at (-38/2+30,-38) {};
\node[box=0-for-negatives] (p-38-31) at (-38/2+31,-38) {};
\node[box=0-for-negatives] (p-38-32) at (-38/2+32,-38) {};
\node[box=0-for-negatives] (p-38-33) at (-38/2+33,-38) {};
\node[box=0-for-negatives] (p-38-34) at (-38/2+34,-38) {};
\node[box=0-for-negatives] (p-38-35) at (-38/2+35,-38) {};
\node[box=1-for-negatives] (p-38-36) at (-38/2+36,-38) {};
\node[box=1-for-negatives] (p-38-37) at (-38/2+37,-38) {};
\node[box=1-for-negatives] (p-38-38) at (-38/2+38,-38) {};
\node[box=2] (p-39-0) at (-39/2+0,-39) {};
\node[box=0-for-negatives] (p-39-1) at (-39/2+1,-39) {};
\node[box=0-for-negatives] (p-39-2) at (-39/2+2,-39) {};
\node[box=1-for-negatives] (p-39-3) at (-39/2+3,-39) {};
\node[box=0-for-negatives] (p-39-4) at (-39/2+4,-39) {};
\node[box=0-for-negatives] (p-39-5) at (-39/2+5,-39) {};
\node[box=0-for-negatives] (p-39-6) at (-39/2+6,-39) {};
\node[box=0-for-negatives] (p-39-7) at (-39/2+7,-39) {};
\node[box=0-for-negatives] (p-39-8) at (-39/2+8,-39) {};
\node[box=1-for-negatives] (p-39-9) at (-39/2+9,-39) {};
\node[box=0-for-negatives] (p-39-10) at (-39/2+10,-39) {};
\node[box=0-for-negatives] (p-39-11) at (-39/2+11,-39) {};
\node[box=2-for-negatives] (p-39-12) at (-39/2+12,-39) {};
\node[box=0-for-negatives] (p-39-13) at (-39/2+13,-39) {};
\node[box=0-for-negatives] (p-39-14) at (-39/2+14,-39) {};
\node[box=0-for-negatives] (p-39-15) at (-39/2+15,-39) {};
\node[box=0-for-negatives] (p-39-16) at (-39/2+16,-39) {};
\node[box=0-for-negatives] (p-39-17) at (-39/2+17,-39) {};
\node[box=0-for-negatives] (p-39-18) at (-39/2+18,-39) {};
\node[box=0-for-negatives] (p-39-19) at (-39/2+19,-39) {};
\node[box=0-for-negatives] (p-39-20) at (-39/2+20,-39) {};
\node[box=0-for-negatives] (p-39-21) at (-39/2+21,-39) {};
\node[box=0-for-negatives] (p-39-22) at (-39/2+22,-39) {};
\node[box=0-for-negatives] (p-39-23) at (-39/2+23,-39) {};
\node[box=0-for-negatives] (p-39-24) at (-39/2+24,-39) {};
\node[box=0-for-negatives] (p-39-25) at (-39/2+25,-39) {};
\node[box=0-for-negatives] (p-39-26) at (-39/2+26,-39) {};
\node[box=1-for-negatives] (p-39-27) at (-39/2+27,-39) {};
\node[box=0-for-negatives] (p-39-28) at (-39/2+28,-39) {};
\node[box=0-for-negatives] (p-39-29) at (-39/2+29,-39) {};
\node[box=2-for-negatives] (p-39-30) at (-39/2+30,-39) {};
\node[box=0-for-negatives] (p-39-31) at (-39/2+31,-39) {};
\node[box=0-for-negatives] (p-39-32) at (-39/2+32,-39) {};
\node[box=0-for-negatives] (p-39-33) at (-39/2+33,-39) {};
\node[box=0-for-negatives] (p-39-34) at (-39/2+34,-39) {};
\node[box=0-for-negatives] (p-39-35) at (-39/2+35,-39) {};
\node[box=2-for-negatives] (p-39-36) at (-39/2+36,-39) {};
\node[box=0-for-negatives] (p-39-37) at (-39/2+37,-39) {};
\node[box=0-for-negatives] (p-39-38) at (-39/2+38,-39) {};
\node[box=1-for-negatives] (p-39-39) at (-39/2+39,-39) {};
\node[box=1] (p-40-0) at (-40/2+0,-40) {};
\node[box=2-for-negatives] (p-40-1) at (-40/2+1,-40) {};
\node[box=0-for-negatives] (p-40-2) at (-40/2+2,-40) {};
\node[box=2-for-negatives] (p-40-3) at (-40/2+3,-40) {};
\node[box=1-for-negatives] (p-40-4) at (-40/2+4,-40) {};
\node[box=0-for-negatives] (p-40-5) at (-40/2+5,-40) {};
\node[box=0-for-negatives] (p-40-6) at (-40/2+6,-40) {};
\node[box=0-for-negatives] (p-40-7) at (-40/2+7,-40) {};
\node[box=0-for-negatives] (p-40-8) at (-40/2+8,-40) {};
\node[box=2-for-negatives] (p-40-9) at (-40/2+9,-40) {};
\node[box=1-for-negatives] (p-40-10) at (-40/2+10,-40) {};
\node[box=0-for-negatives] (p-40-11) at (-40/2+11,-40) {};
\node[box=1-for-negatives] (p-40-12) at (-40/2+12,-40) {};
\node[box=2-for-negatives] (p-40-13) at (-40/2+13,-40) {};
\node[box=0-for-negatives] (p-40-14) at (-40/2+14,-40) {};
\node[box=0-for-negatives] (p-40-15) at (-40/2+15,-40) {};
\node[box=0-for-negatives] (p-40-16) at (-40/2+16,-40) {};
\node[box=0-for-negatives] (p-40-17) at (-40/2+17,-40) {};
\node[box=0-for-negatives] (p-40-18) at (-40/2+18,-40) {};
\node[box=0-for-negatives] (p-40-19) at (-40/2+19,-40) {};
\node[box=0-for-negatives] (p-40-20) at (-40/2+20,-40) {};
\node[box=0-for-negatives] (p-40-21) at (-40/2+21,-40) {};
\node[box=0-for-negatives] (p-40-22) at (-40/2+22,-40) {};
\node[box=0-for-negatives] (p-40-23) at (-40/2+23,-40) {};
\node[box=0-for-negatives] (p-40-24) at (-40/2+24,-40) {};
\node[box=0-for-negatives] (p-40-25) at (-40/2+25,-40) {};
\node[box=0-for-negatives] (p-40-26) at (-40/2+26,-40) {};
\node[box=2-for-negatives] (p-40-27) at (-40/2+27,-40) {};
\node[box=1-for-negatives] (p-40-28) at (-40/2+28,-40) {};
\node[box=0-for-negatives] (p-40-29) at (-40/2+29,-40) {};
\node[box=1-for-negatives] (p-40-30) at (-40/2+30,-40) {};
\node[box=2-for-negatives] (p-40-31) at (-40/2+31,-40) {};
\node[box=0-for-negatives] (p-40-32) at (-40/2+32,-40) {};
\node[box=0-for-negatives] (p-40-33) at (-40/2+33,-40) {};
\node[box=0-for-negatives] (p-40-34) at (-40/2+34,-40) {};
\node[box=0-for-negatives] (p-40-35) at (-40/2+35,-40) {};
\node[box=1-for-negatives] (p-40-36) at (-40/2+36,-40) {};
\node[box=2-for-negatives] (p-40-37) at (-40/2+37,-40) {};
\node[box=0-for-negatives] (p-40-38) at (-40/2+38,-40) {};
\node[box=2-for-negatives] (p-40-39) at (-40/2+39,-40) {};
\node[box=1-for-negatives] (p-40-40) at (-40/2+40,-40) {};
\node[box=2] (p-41-0) at (-41/2+0,-41) {};
\node[box=2-for-negatives] (p-41-1) at (-41/2+1,-41) {};
\node[box=2-for-negatives] (p-41-2) at (-41/2+2,-41) {};
\node[box=1-for-negatives] (p-41-3) at (-41/2+3,-41) {};
\node[box=1-for-negatives] (p-41-4) at (-41/2+4,-41) {};
\node[box=1-for-negatives] (p-41-5) at (-41/2+5,-41) {};
\node[box=0-for-negatives] (p-41-6) at (-41/2+6,-41) {};
\node[box=0-for-negatives] (p-41-7) at (-41/2+7,-41) {};
\node[box=0-for-negatives] (p-41-8) at (-41/2+8,-41) {};
\node[box=1-for-negatives] (p-41-9) at (-41/2+9,-41) {};
\node[box=1-for-negatives] (p-41-10) at (-41/2+10,-41) {};
\node[box=1-for-negatives] (p-41-11) at (-41/2+11,-41) {};
\node[box=2-for-negatives] (p-41-12) at (-41/2+12,-41) {};
\node[box=2-for-negatives] (p-41-13) at (-41/2+13,-41) {};
\node[box=2-for-negatives] (p-41-14) at (-41/2+14,-41) {};
\node[box=0-for-negatives] (p-41-15) at (-41/2+15,-41) {};
\node[box=0-for-negatives] (p-41-16) at (-41/2+16,-41) {};
\node[box=0-for-negatives] (p-41-17) at (-41/2+17,-41) {};
\node[box=0-for-negatives] (p-41-18) at (-41/2+18,-41) {};
\node[box=0-for-negatives] (p-41-19) at (-41/2+19,-41) {};
\node[box=0-for-negatives] (p-41-20) at (-41/2+20,-41) {};
\node[box=0-for-negatives] (p-41-21) at (-41/2+21,-41) {};
\node[box=0-for-negatives] (p-41-22) at (-41/2+22,-41) {};
\node[box=0-for-negatives] (p-41-23) at (-41/2+23,-41) {};
\node[box=0-for-negatives] (p-41-24) at (-41/2+24,-41) {};
\node[box=0-for-negatives] (p-41-25) at (-41/2+25,-41) {};
\node[box=0-for-negatives] (p-41-26) at (-41/2+26,-41) {};
\node[box=1-for-negatives] (p-41-27) at (-41/2+27,-41) {};
\node[box=1-for-negatives] (p-41-28) at (-41/2+28,-41) {};
\node[box=1-for-negatives] (p-41-29) at (-41/2+29,-41) {};
\node[box=2-for-negatives] (p-41-30) at (-41/2+30,-41) {};
\node[box=2-for-negatives] (p-41-31) at (-41/2+31,-41) {};
\node[box=2-for-negatives] (p-41-32) at (-41/2+32,-41) {};
\node[box=0-for-negatives] (p-41-33) at (-41/2+33,-41) {};
\node[box=0-for-negatives] (p-41-34) at (-41/2+34,-41) {};
\node[box=0-for-negatives] (p-41-35) at (-41/2+35,-41) {};
\node[box=2-for-negatives] (p-41-36) at (-41/2+36,-41) {};
\node[box=2-for-negatives] (p-41-37) at (-41/2+37,-41) {};
\node[box=2-for-negatives] (p-41-38) at (-41/2+38,-41) {};
\node[box=1-for-negatives] (p-41-39) at (-41/2+39,-41) {};
\node[box=1-for-negatives] (p-41-40) at (-41/2+40,-41) {};
\node[box=1-for-negatives] (p-41-41) at (-41/2+41,-41) {};
\node[box=1] (p-42-0) at (-42/2+0,-42) {};
\node[box=0-for-negatives] (p-42-1) at (-42/2+1,-42) {};
\node[box=0-for-negatives] (p-42-2) at (-42/2+2,-42) {};
\node[box=1-for-negatives] (p-42-3) at (-42/2+3,-42) {};
\node[box=0-for-negatives] (p-42-4) at (-42/2+4,-42) {};
\node[box=0-for-negatives] (p-42-5) at (-42/2+5,-42) {};
\node[box=1-for-negatives] (p-42-6) at (-42/2+6,-42) {};
\node[box=0-for-negatives] (p-42-7) at (-42/2+7,-42) {};
\node[box=0-for-negatives] (p-42-8) at (-42/2+8,-42) {};
\node[box=2-for-negatives] (p-42-9) at (-42/2+9,-42) {};
\node[box=0-for-negatives] (p-42-10) at (-42/2+10,-42) {};
\node[box=0-for-negatives] (p-42-11) at (-42/2+11,-42) {};
\node[box=2-for-negatives] (p-42-12) at (-42/2+12,-42) {};
\node[box=0-for-negatives] (p-42-13) at (-42/2+13,-42) {};
\node[box=0-for-negatives] (p-42-14) at (-42/2+14,-42) {};
\node[box=2-for-negatives] (p-42-15) at (-42/2+15,-42) {};
\node[box=0-for-negatives] (p-42-16) at (-42/2+16,-42) {};
\node[box=0-for-negatives] (p-42-17) at (-42/2+17,-42) {};
\node[box=0-for-negatives] (p-42-18) at (-42/2+18,-42) {};
\node[box=0-for-negatives] (p-42-19) at (-42/2+19,-42) {};
\node[box=0-for-negatives] (p-42-20) at (-42/2+20,-42) {};
\node[box=0-for-negatives] (p-42-21) at (-42/2+21,-42) {};
\node[box=0-for-negatives] (p-42-22) at (-42/2+22,-42) {};
\node[box=0-for-negatives] (p-42-23) at (-42/2+23,-42) {};
\node[box=0-for-negatives] (p-42-24) at (-42/2+24,-42) {};
\node[box=0-for-negatives] (p-42-25) at (-42/2+25,-42) {};
\node[box=0-for-negatives] (p-42-26) at (-42/2+26,-42) {};
\node[box=2-for-negatives] (p-42-27) at (-42/2+27,-42) {};
\node[box=0-for-negatives] (p-42-28) at (-42/2+28,-42) {};
\node[box=0-for-negatives] (p-42-29) at (-42/2+29,-42) {};
\node[box=2-for-negatives] (p-42-30) at (-42/2+30,-42) {};
\node[box=0-for-negatives] (p-42-31) at (-42/2+31,-42) {};
\node[box=0-for-negatives] (p-42-32) at (-42/2+32,-42) {};
\node[box=2-for-negatives] (p-42-33) at (-42/2+33,-42) {};
\node[box=0-for-negatives] (p-42-34) at (-42/2+34,-42) {};
\node[box=0-for-negatives] (p-42-35) at (-42/2+35,-42) {};
\node[box=1-for-negatives] (p-42-36) at (-42/2+36,-42) {};
\node[box=0-for-negatives] (p-42-37) at (-42/2+37,-42) {};
\node[box=0-for-negatives] (p-42-38) at (-42/2+38,-42) {};
\node[box=1-for-negatives] (p-42-39) at (-42/2+39,-42) {};
\node[box=0-for-negatives] (p-42-40) at (-42/2+40,-42) {};
\node[box=0-for-negatives] (p-42-41) at (-42/2+41,-42) {};
\node[box=1-for-negatives] (p-42-42) at (-42/2+42,-42) {};
\node[box=2] (p-43-0) at (-43/2+0,-43) {};
\node[box=1-for-negatives] (p-43-1) at (-43/2+1,-43) {};
\node[box=0-for-negatives] (p-43-2) at (-43/2+2,-43) {};
\node[box=2-for-negatives] (p-43-3) at (-43/2+3,-43) {};
\node[box=1-for-negatives] (p-43-4) at (-43/2+4,-43) {};
\node[box=0-for-negatives] (p-43-5) at (-43/2+5,-43) {};
\node[box=2-for-negatives] (p-43-6) at (-43/2+6,-43) {};
\node[box=1-for-negatives] (p-43-7) at (-43/2+7,-43) {};
\node[box=0-for-negatives] (p-43-8) at (-43/2+8,-43) {};
\node[box=1-for-negatives] (p-43-9) at (-43/2+9,-43) {};
\node[box=2-for-negatives] (p-43-10) at (-43/2+10,-43) {};
\node[box=0-for-negatives] (p-43-11) at (-43/2+11,-43) {};
\node[box=1-for-negatives] (p-43-12) at (-43/2+12,-43) {};
\node[box=2-for-negatives] (p-43-13) at (-43/2+13,-43) {};
\node[box=0-for-negatives] (p-43-14) at (-43/2+14,-43) {};
\node[box=1-for-negatives] (p-43-15) at (-43/2+15,-43) {};
\node[box=2-for-negatives] (p-43-16) at (-43/2+16,-43) {};
\node[box=0-for-negatives] (p-43-17) at (-43/2+17,-43) {};
\node[box=0-for-negatives] (p-43-18) at (-43/2+18,-43) {};
\node[box=0-for-negatives] (p-43-19) at (-43/2+19,-43) {};
\node[box=0-for-negatives] (p-43-20) at (-43/2+20,-43) {};
\node[box=0-for-negatives] (p-43-21) at (-43/2+21,-43) {};
\node[box=0-for-negatives] (p-43-22) at (-43/2+22,-43) {};
\node[box=0-for-negatives] (p-43-23) at (-43/2+23,-43) {};
\node[box=0-for-negatives] (p-43-24) at (-43/2+24,-43) {};
\node[box=0-for-negatives] (p-43-25) at (-43/2+25,-43) {};
\node[box=0-for-negatives] (p-43-26) at (-43/2+26,-43) {};
\node[box=1-for-negatives] (p-43-27) at (-43/2+27,-43) {};
\node[box=2-for-negatives] (p-43-28) at (-43/2+28,-43) {};
\node[box=0-for-negatives] (p-43-29) at (-43/2+29,-43) {};
\node[box=1-for-negatives] (p-43-30) at (-43/2+30,-43) {};
\node[box=2-for-negatives] (p-43-31) at (-43/2+31,-43) {};
\node[box=0-for-negatives] (p-43-32) at (-43/2+32,-43) {};
\node[box=1-for-negatives] (p-43-33) at (-43/2+33,-43) {};
\node[box=2-for-negatives] (p-43-34) at (-43/2+34,-43) {};
\node[box=0-for-negatives] (p-43-35) at (-43/2+35,-43) {};
\node[box=2-for-negatives] (p-43-36) at (-43/2+36,-43) {};
\node[box=1-for-negatives] (p-43-37) at (-43/2+37,-43) {};
\node[box=0-for-negatives] (p-43-38) at (-43/2+38,-43) {};
\node[box=2-for-negatives] (p-43-39) at (-43/2+39,-43) {};
\node[box=1-for-negatives] (p-43-40) at (-43/2+40,-43) {};
\node[box=0-for-negatives] (p-43-41) at (-43/2+41,-43) {};
\node[box=2-for-negatives] (p-43-42) at (-43/2+42,-43) {};
\node[box=1-for-negatives] (p-43-43) at (-43/2+43,-43) {};
\node[box=1] (p-44-0) at (-44/2+0,-44) {};
\node[box=1-for-negatives] (p-44-1) at (-44/2+1,-44) {};
\node[box=1-for-negatives] (p-44-2) at (-44/2+2,-44) {};
\node[box=1-for-negatives] (p-44-3) at (-44/2+3,-44) {};
\node[box=1-for-negatives] (p-44-4) at (-44/2+4,-44) {};
\node[box=1-for-negatives] (p-44-5) at (-44/2+5,-44) {};
\node[box=1-for-negatives] (p-44-6) at (-44/2+6,-44) {};
\node[box=1-for-negatives] (p-44-7) at (-44/2+7,-44) {};
\node[box=1-for-negatives] (p-44-8) at (-44/2+8,-44) {};
\node[box=2-for-negatives] (p-44-9) at (-44/2+9,-44) {};
\node[box=2-for-negatives] (p-44-10) at (-44/2+10,-44) {};
\node[box=2-for-negatives] (p-44-11) at (-44/2+11,-44) {};
\node[box=2-for-negatives] (p-44-12) at (-44/2+12,-44) {};
\node[box=2-for-negatives] (p-44-13) at (-44/2+13,-44) {};
\node[box=2-for-negatives] (p-44-14) at (-44/2+14,-44) {};
\node[box=2-for-negatives] (p-44-15) at (-44/2+15,-44) {};
\node[box=2-for-negatives] (p-44-16) at (-44/2+16,-44) {};
\node[box=2-for-negatives] (p-44-17) at (-44/2+17,-44) {};
\node[box=0-for-negatives] (p-44-18) at (-44/2+18,-44) {};
\node[box=0-for-negatives] (p-44-19) at (-44/2+19,-44) {};
\node[box=0-for-negatives] (p-44-20) at (-44/2+20,-44) {};
\node[box=0-for-negatives] (p-44-21) at (-44/2+21,-44) {};
\node[box=0-for-negatives] (p-44-22) at (-44/2+22,-44) {};
\node[box=0-for-negatives] (p-44-23) at (-44/2+23,-44) {};
\node[box=0-for-negatives] (p-44-24) at (-44/2+24,-44) {};
\node[box=0-for-negatives] (p-44-25) at (-44/2+25,-44) {};
\node[box=0-for-negatives] (p-44-26) at (-44/2+26,-44) {};
\node[box=2-for-negatives] (p-44-27) at (-44/2+27,-44) {};
\node[box=2-for-negatives] (p-44-28) at (-44/2+28,-44) {};
\node[box=2-for-negatives] (p-44-29) at (-44/2+29,-44) {};
\node[box=2-for-negatives] (p-44-30) at (-44/2+30,-44) {};
\node[box=2-for-negatives] (p-44-31) at (-44/2+31,-44) {};
\node[box=2-for-negatives] (p-44-32) at (-44/2+32,-44) {};
\node[box=2-for-negatives] (p-44-33) at (-44/2+33,-44) {};
\node[box=2-for-negatives] (p-44-34) at (-44/2+34,-44) {};
\node[box=2-for-negatives] (p-44-35) at (-44/2+35,-44) {};
\node[box=1-for-negatives] (p-44-36) at (-44/2+36,-44) {};
\node[box=1-for-negatives] (p-44-37) at (-44/2+37,-44) {};
\node[box=1-for-negatives] (p-44-38) at (-44/2+38,-44) {};
\node[box=1-for-negatives] (p-44-39) at (-44/2+39,-44) {};
\node[box=1-for-negatives] (p-44-40) at (-44/2+40,-44) {};
\node[box=1-for-negatives] (p-44-41) at (-44/2+41,-44) {};
\node[box=1-for-negatives] (p-44-42) at (-44/2+42,-44) {};
\node[box=1-for-negatives] (p-44-43) at (-44/2+43,-44) {};
\node[box=1-for-negatives] (p-44-44) at (-44/2+44,-44) {};
\node[box=2] (p-45-0) at (-45/2+0,-45) {};
\node[box=0-for-negatives] (p-45-1) at (-45/2+1,-45) {};
\node[box=0-for-negatives] (p-45-2) at (-45/2+2,-45) {};
\node[box=0-for-negatives] (p-45-3) at (-45/2+3,-45) {};
\node[box=0-for-negatives] (p-45-4) at (-45/2+4,-45) {};
\node[box=0-for-negatives] (p-45-5) at (-45/2+5,-45) {};
\node[box=0-for-negatives] (p-45-6) at (-45/2+6,-45) {};
\node[box=0-for-negatives] (p-45-7) at (-45/2+7,-45) {};
\node[box=0-for-negatives] (p-45-8) at (-45/2+8,-45) {};
\node[box=2-for-negatives] (p-45-9) at (-45/2+9,-45) {};
\node[box=0-for-negatives] (p-45-10) at (-45/2+10,-45) {};
\node[box=0-for-negatives] (p-45-11) at (-45/2+11,-45) {};
\node[box=0-for-negatives] (p-45-12) at (-45/2+12,-45) {};
\node[box=0-for-negatives] (p-45-13) at (-45/2+13,-45) {};
\node[box=0-for-negatives] (p-45-14) at (-45/2+14,-45) {};
\node[box=0-for-negatives] (p-45-15) at (-45/2+15,-45) {};
\node[box=0-for-negatives] (p-45-16) at (-45/2+16,-45) {};
\node[box=0-for-negatives] (p-45-17) at (-45/2+17,-45) {};
\node[box=2-for-negatives] (p-45-18) at (-45/2+18,-45) {};
\node[box=0-for-negatives] (p-45-19) at (-45/2+19,-45) {};
\node[box=0-for-negatives] (p-45-20) at (-45/2+20,-45) {};
\node[box=0-for-negatives] (p-45-21) at (-45/2+21,-45) {};
\node[box=0-for-negatives] (p-45-22) at (-45/2+22,-45) {};
\node[box=0-for-negatives] (p-45-23) at (-45/2+23,-45) {};
\node[box=0-for-negatives] (p-45-24) at (-45/2+24,-45) {};
\node[box=0-for-negatives] (p-45-25) at (-45/2+25,-45) {};
\node[box=0-for-negatives] (p-45-26) at (-45/2+26,-45) {};
\node[box=1-for-negatives] (p-45-27) at (-45/2+27,-45) {};
\node[box=0-for-negatives] (p-45-28) at (-45/2+28,-45) {};
\node[box=0-for-negatives] (p-45-29) at (-45/2+29,-45) {};
\node[box=0-for-negatives] (p-45-30) at (-45/2+30,-45) {};
\node[box=0-for-negatives] (p-45-31) at (-45/2+31,-45) {};
\node[box=0-for-negatives] (p-45-32) at (-45/2+32,-45) {};
\node[box=0-for-negatives] (p-45-33) at (-45/2+33,-45) {};
\node[box=0-for-negatives] (p-45-34) at (-45/2+34,-45) {};
\node[box=0-for-negatives] (p-45-35) at (-45/2+35,-45) {};
\node[box=1-for-negatives] (p-45-36) at (-45/2+36,-45) {};
\node[box=0-for-negatives] (p-45-37) at (-45/2+37,-45) {};
\node[box=0-for-negatives] (p-45-38) at (-45/2+38,-45) {};
\node[box=0-for-negatives] (p-45-39) at (-45/2+39,-45) {};
\node[box=0-for-negatives] (p-45-40) at (-45/2+40,-45) {};
\node[box=0-for-negatives] (p-45-41) at (-45/2+41,-45) {};
\node[box=0-for-negatives] (p-45-42) at (-45/2+42,-45) {};
\node[box=0-for-negatives] (p-45-43) at (-45/2+43,-45) {};
\node[box=0-for-negatives] (p-45-44) at (-45/2+44,-45) {};
\node[box=1-for-negatives] (p-45-45) at (-45/2+45,-45) {};
\node[box=1] (p-46-0) at (-46/2+0,-46) {};
\node[box=2-for-negatives] (p-46-1) at (-46/2+1,-46) {};
\node[box=0-for-negatives] (p-46-2) at (-46/2+2,-46) {};
\node[box=0-for-negatives] (p-46-3) at (-46/2+3,-46) {};
\node[box=0-for-negatives] (p-46-4) at (-46/2+4,-46) {};
\node[box=0-for-negatives] (p-46-5) at (-46/2+5,-46) {};
\node[box=0-for-negatives] (p-46-6) at (-46/2+6,-46) {};
\node[box=0-for-negatives] (p-46-7) at (-46/2+7,-46) {};
\node[box=0-for-negatives] (p-46-8) at (-46/2+8,-46) {};
\node[box=1-for-negatives] (p-46-9) at (-46/2+9,-46) {};
\node[box=2-for-negatives] (p-46-10) at (-46/2+10,-46) {};
\node[box=0-for-negatives] (p-46-11) at (-46/2+11,-46) {};
\node[box=0-for-negatives] (p-46-12) at (-46/2+12,-46) {};
\node[box=0-for-negatives] (p-46-13) at (-46/2+13,-46) {};
\node[box=0-for-negatives] (p-46-14) at (-46/2+14,-46) {};
\node[box=0-for-negatives] (p-46-15) at (-46/2+15,-46) {};
\node[box=0-for-negatives] (p-46-16) at (-46/2+16,-46) {};
\node[box=0-for-negatives] (p-46-17) at (-46/2+17,-46) {};
\node[box=1-for-negatives] (p-46-18) at (-46/2+18,-46) {};
\node[box=2-for-negatives] (p-46-19) at (-46/2+19,-46) {};
\node[box=0-for-negatives] (p-46-20) at (-46/2+20,-46) {};
\node[box=0-for-negatives] (p-46-21) at (-46/2+21,-46) {};
\node[box=0-for-negatives] (p-46-22) at (-46/2+22,-46) {};
\node[box=0-for-negatives] (p-46-23) at (-46/2+23,-46) {};
\node[box=0-for-negatives] (p-46-24) at (-46/2+24,-46) {};
\node[box=0-for-negatives] (p-46-25) at (-46/2+25,-46) {};
\node[box=0-for-negatives] (p-46-26) at (-46/2+26,-46) {};
\node[box=2-for-negatives] (p-46-27) at (-46/2+27,-46) {};
\node[box=1-for-negatives] (p-46-28) at (-46/2+28,-46) {};
\node[box=0-for-negatives] (p-46-29) at (-46/2+29,-46) {};
\node[box=0-for-negatives] (p-46-30) at (-46/2+30,-46) {};
\node[box=0-for-negatives] (p-46-31) at (-46/2+31,-46) {};
\node[box=0-for-negatives] (p-46-32) at (-46/2+32,-46) {};
\node[box=0-for-negatives] (p-46-33) at (-46/2+33,-46) {};
\node[box=0-for-negatives] (p-46-34) at (-46/2+34,-46) {};
\node[box=0-for-negatives] (p-46-35) at (-46/2+35,-46) {};
\node[box=2-for-negatives] (p-46-36) at (-46/2+36,-46) {};
\node[box=1-for-negatives] (p-46-37) at (-46/2+37,-46) {};
\node[box=0-for-negatives] (p-46-38) at (-46/2+38,-46) {};
\node[box=0-for-negatives] (p-46-39) at (-46/2+39,-46) {};
\node[box=0-for-negatives] (p-46-40) at (-46/2+40,-46) {};
\node[box=0-for-negatives] (p-46-41) at (-46/2+41,-46) {};
\node[box=0-for-negatives] (p-46-42) at (-46/2+42,-46) {};
\node[box=0-for-negatives] (p-46-43) at (-46/2+43,-46) {};
\node[box=0-for-negatives] (p-46-44) at (-46/2+44,-46) {};
\node[box=2-for-negatives] (p-46-45) at (-46/2+45,-46) {};
\node[box=1-for-negatives] (p-46-46) at (-46/2+46,-46) {};
\node[box=2] (p-47-0) at (-47/2+0,-47) {};
\node[box=2-for-negatives] (p-47-1) at (-47/2+1,-47) {};
\node[box=2-for-negatives] (p-47-2) at (-47/2+2,-47) {};
\node[box=0-for-negatives] (p-47-3) at (-47/2+3,-47) {};
\node[box=0-for-negatives] (p-47-4) at (-47/2+4,-47) {};
\node[box=0-for-negatives] (p-47-5) at (-47/2+5,-47) {};
\node[box=0-for-negatives] (p-47-6) at (-47/2+6,-47) {};
\node[box=0-for-negatives] (p-47-7) at (-47/2+7,-47) {};
\node[box=0-for-negatives] (p-47-8) at (-47/2+8,-47) {};
\node[box=2-for-negatives] (p-47-9) at (-47/2+9,-47) {};
\node[box=2-for-negatives] (p-47-10) at (-47/2+10,-47) {};
\node[box=2-for-negatives] (p-47-11) at (-47/2+11,-47) {};
\node[box=0-for-negatives] (p-47-12) at (-47/2+12,-47) {};
\node[box=0-for-negatives] (p-47-13) at (-47/2+13,-47) {};
\node[box=0-for-negatives] (p-47-14) at (-47/2+14,-47) {};
\node[box=0-for-negatives] (p-47-15) at (-47/2+15,-47) {};
\node[box=0-for-negatives] (p-47-16) at (-47/2+16,-47) {};
\node[box=0-for-negatives] (p-47-17) at (-47/2+17,-47) {};
\node[box=2-for-negatives] (p-47-18) at (-47/2+18,-47) {};
\node[box=2-for-negatives] (p-47-19) at (-47/2+19,-47) {};
\node[box=2-for-negatives] (p-47-20) at (-47/2+20,-47) {};
\node[box=0-for-negatives] (p-47-21) at (-47/2+21,-47) {};
\node[box=0-for-negatives] (p-47-22) at (-47/2+22,-47) {};
\node[box=0-for-negatives] (p-47-23) at (-47/2+23,-47) {};
\node[box=0-for-negatives] (p-47-24) at (-47/2+24,-47) {};
\node[box=0-for-negatives] (p-47-25) at (-47/2+25,-47) {};
\node[box=0-for-negatives] (p-47-26) at (-47/2+26,-47) {};
\node[box=1-for-negatives] (p-47-27) at (-47/2+27,-47) {};
\node[box=1-for-negatives] (p-47-28) at (-47/2+28,-47) {};
\node[box=1-for-negatives] (p-47-29) at (-47/2+29,-47) {};
\node[box=0-for-negatives] (p-47-30) at (-47/2+30,-47) {};
\node[box=0-for-negatives] (p-47-31) at (-47/2+31,-47) {};
\node[box=0-for-negatives] (p-47-32) at (-47/2+32,-47) {};
\node[box=0-for-negatives] (p-47-33) at (-47/2+33,-47) {};
\node[box=0-for-negatives] (p-47-34) at (-47/2+34,-47) {};
\node[box=0-for-negatives] (p-47-35) at (-47/2+35,-47) {};
\node[box=1-for-negatives] (p-47-36) at (-47/2+36,-47) {};
\node[box=1-for-negatives] (p-47-37) at (-47/2+37,-47) {};
\node[box=1-for-negatives] (p-47-38) at (-47/2+38,-47) {};
\node[box=0-for-negatives] (p-47-39) at (-47/2+39,-47) {};
\node[box=0-for-negatives] (p-47-40) at (-47/2+40,-47) {};
\node[box=0-for-negatives] (p-47-41) at (-47/2+41,-47) {};
\node[box=0-for-negatives] (p-47-42) at (-47/2+42,-47) {};
\node[box=0-for-negatives] (p-47-43) at (-47/2+43,-47) {};
\node[box=0-for-negatives] (p-47-44) at (-47/2+44,-47) {};
\node[box=1-for-negatives] (p-47-45) at (-47/2+45,-47) {};
\node[box=1-for-negatives] (p-47-46) at (-47/2+46,-47) {};
\node[box=1-for-negatives] (p-47-47) at (-47/2+47,-47) {};
\node[box=1] (p-48-0) at (-48/2+0,-48) {};
\node[box=0-for-negatives] (p-48-1) at (-48/2+1,-48) {};
\node[box=0-for-negatives] (p-48-2) at (-48/2+2,-48) {};
\node[box=2-for-negatives] (p-48-3) at (-48/2+3,-48) {};
\node[box=0-for-negatives] (p-48-4) at (-48/2+4,-48) {};
\node[box=0-for-negatives] (p-48-5) at (-48/2+5,-48) {};
\node[box=0-for-negatives] (p-48-6) at (-48/2+6,-48) {};
\node[box=0-for-negatives] (p-48-7) at (-48/2+7,-48) {};
\node[box=0-for-negatives] (p-48-8) at (-48/2+8,-48) {};
\node[box=1-for-negatives] (p-48-9) at (-48/2+9,-48) {};
\node[box=0-for-negatives] (p-48-10) at (-48/2+10,-48) {};
\node[box=0-for-negatives] (p-48-11) at (-48/2+11,-48) {};
\node[box=2-for-negatives] (p-48-12) at (-48/2+12,-48) {};
\node[box=0-for-negatives] (p-48-13) at (-48/2+13,-48) {};
\node[box=0-for-negatives] (p-48-14) at (-48/2+14,-48) {};
\node[box=0-for-negatives] (p-48-15) at (-48/2+15,-48) {};
\node[box=0-for-negatives] (p-48-16) at (-48/2+16,-48) {};
\node[box=0-for-negatives] (p-48-17) at (-48/2+17,-48) {};
\node[box=1-for-negatives] (p-48-18) at (-48/2+18,-48) {};
\node[box=0-for-negatives] (p-48-19) at (-48/2+19,-48) {};
\node[box=0-for-negatives] (p-48-20) at (-48/2+20,-48) {};
\node[box=2-for-negatives] (p-48-21) at (-48/2+21,-48) {};
\node[box=0-for-negatives] (p-48-22) at (-48/2+22,-48) {};
\node[box=0-for-negatives] (p-48-23) at (-48/2+23,-48) {};
\node[box=0-for-negatives] (p-48-24) at (-48/2+24,-48) {};
\node[box=0-for-negatives] (p-48-25) at (-48/2+25,-48) {};
\node[box=0-for-negatives] (p-48-26) at (-48/2+26,-48) {};
\node[box=2-for-negatives] (p-48-27) at (-48/2+27,-48) {};
\node[box=0-for-negatives] (p-48-28) at (-48/2+28,-48) {};
\node[box=0-for-negatives] (p-48-29) at (-48/2+29,-48) {};
\node[box=1-for-negatives] (p-48-30) at (-48/2+30,-48) {};
\node[box=0-for-negatives] (p-48-31) at (-48/2+31,-48) {};
\node[box=0-for-negatives] (p-48-32) at (-48/2+32,-48) {};
\node[box=0-for-negatives] (p-48-33) at (-48/2+33,-48) {};
\node[box=0-for-negatives] (p-48-34) at (-48/2+34,-48) {};
\node[box=0-for-negatives] (p-48-35) at (-48/2+35,-48) {};
\node[box=2-for-negatives] (p-48-36) at (-48/2+36,-48) {};
\node[box=0-for-negatives] (p-48-37) at (-48/2+37,-48) {};
\node[box=0-for-negatives] (p-48-38) at (-48/2+38,-48) {};
\node[box=1-for-negatives] (p-48-39) at (-48/2+39,-48) {};
\node[box=0-for-negatives] (p-48-40) at (-48/2+40,-48) {};
\node[box=0-for-negatives] (p-48-41) at (-48/2+41,-48) {};
\node[box=0-for-negatives] (p-48-42) at (-48/2+42,-48) {};
\node[box=0-for-negatives] (p-48-43) at (-48/2+43,-48) {};
\node[box=0-for-negatives] (p-48-44) at (-48/2+44,-48) {};
\node[box=2-for-negatives] (p-48-45) at (-48/2+45,-48) {};
\node[box=0-for-negatives] (p-48-46) at (-48/2+46,-48) {};
\node[box=0-for-negatives] (p-48-47) at (-48/2+47,-48) {};
\node[box=1-for-negatives] (p-48-48) at (-48/2+48,-48) {};
\node[box=2] (p-49-0) at (-49/2+0,-49) {};
\node[box=1-for-negatives] (p-49-1) at (-49/2+1,-49) {};
\node[box=0-for-negatives] (p-49-2) at (-49/2+2,-49) {};
\node[box=1-for-negatives] (p-49-3) at (-49/2+3,-49) {};
\node[box=2-for-negatives] (p-49-4) at (-49/2+4,-49) {};
\node[box=0-for-negatives] (p-49-5) at (-49/2+5,-49) {};
\node[box=0-for-negatives] (p-49-6) at (-49/2+6,-49) {};
\node[box=0-for-negatives] (p-49-7) at (-49/2+7,-49) {};
\node[box=0-for-negatives] (p-49-8) at (-49/2+8,-49) {};
\node[box=2-for-negatives] (p-49-9) at (-49/2+9,-49) {};
\node[box=1-for-negatives] (p-49-10) at (-49/2+10,-49) {};
\node[box=0-for-negatives] (p-49-11) at (-49/2+11,-49) {};
\node[box=1-for-negatives] (p-49-12) at (-49/2+12,-49) {};
\node[box=2-for-negatives] (p-49-13) at (-49/2+13,-49) {};
\node[box=0-for-negatives] (p-49-14) at (-49/2+14,-49) {};
\node[box=0-for-negatives] (p-49-15) at (-49/2+15,-49) {};
\node[box=0-for-negatives] (p-49-16) at (-49/2+16,-49) {};
\node[box=0-for-negatives] (p-49-17) at (-49/2+17,-49) {};
\node[box=2-for-negatives] (p-49-18) at (-49/2+18,-49) {};
\node[box=1-for-negatives] (p-49-19) at (-49/2+19,-49) {};
\node[box=0-for-negatives] (p-49-20) at (-49/2+20,-49) {};
\node[box=1-for-negatives] (p-49-21) at (-49/2+21,-49) {};
\node[box=2-for-negatives] (p-49-22) at (-49/2+22,-49) {};
\node[box=0-for-negatives] (p-49-23) at (-49/2+23,-49) {};
\node[box=0-for-negatives] (p-49-24) at (-49/2+24,-49) {};
\node[box=0-for-negatives] (p-49-25) at (-49/2+25,-49) {};
\node[box=0-for-negatives] (p-49-26) at (-49/2+26,-49) {};
\node[box=1-for-negatives] (p-49-27) at (-49/2+27,-49) {};
\node[box=2-for-negatives] (p-49-28) at (-49/2+28,-49) {};
\node[box=0-for-negatives] (p-49-29) at (-49/2+29,-49) {};
\node[box=2-for-negatives] (p-49-30) at (-49/2+30,-49) {};
\node[box=1-for-negatives] (p-49-31) at (-49/2+31,-49) {};
\node[box=0-for-negatives] (p-49-32) at (-49/2+32,-49) {};
\node[box=0-for-negatives] (p-49-33) at (-49/2+33,-49) {};
\node[box=0-for-negatives] (p-49-34) at (-49/2+34,-49) {};
\node[box=0-for-negatives] (p-49-35) at (-49/2+35,-49) {};
\node[box=1-for-negatives] (p-49-36) at (-49/2+36,-49) {};
\node[box=2-for-negatives] (p-49-37) at (-49/2+37,-49) {};
\node[box=0-for-negatives] (p-49-38) at (-49/2+38,-49) {};
\node[box=2-for-negatives] (p-49-39) at (-49/2+39,-49) {};
\node[box=1-for-negatives] (p-49-40) at (-49/2+40,-49) {};
\node[box=0-for-negatives] (p-49-41) at (-49/2+41,-49) {};
\node[box=0-for-negatives] (p-49-42) at (-49/2+42,-49) {};
\node[box=0-for-negatives] (p-49-43) at (-49/2+43,-49) {};
\node[box=0-for-negatives] (p-49-44) at (-49/2+44,-49) {};
\node[box=1-for-negatives] (p-49-45) at (-49/2+45,-49) {};
\node[box=2-for-negatives] (p-49-46) at (-49/2+46,-49) {};
\node[box=0-for-negatives] (p-49-47) at (-49/2+47,-49) {};
\node[box=2-for-negatives] (p-49-48) at (-49/2+48,-49) {};
\node[box=1-for-negatives] (p-49-49) at (-49/2+49,-49) {};
\node[box=1] (p-50-0) at (-50/2+0,-50) {};
\node[box=1-for-negatives] (p-50-1) at (-50/2+1,-50) {};
\node[box=1-for-negatives] (p-50-2) at (-50/2+2,-50) {};
\node[box=2-for-negatives] (p-50-3) at (-50/2+3,-50) {};
\node[box=2-for-negatives] (p-50-4) at (-50/2+4,-50) {};
\node[box=2-for-negatives] (p-50-5) at (-50/2+5,-50) {};
\node[box=0-for-negatives] (p-50-6) at (-50/2+6,-50) {};
\node[box=0-for-negatives] (p-50-7) at (-50/2+7,-50) {};
\node[box=0-for-negatives] (p-50-8) at (-50/2+8,-50) {};
\node[box=1-for-negatives] (p-50-9) at (-50/2+9,-50) {};
\node[box=1-for-negatives] (p-50-10) at (-50/2+10,-50) {};
\node[box=1-for-negatives] (p-50-11) at (-50/2+11,-50) {};
\node[box=2-for-negatives] (p-50-12) at (-50/2+12,-50) {};
\node[box=2-for-negatives] (p-50-13) at (-50/2+13,-50) {};
\node[box=2-for-negatives] (p-50-14) at (-50/2+14,-50) {};
\node[box=0-for-negatives] (p-50-15) at (-50/2+15,-50) {};
\node[box=0-for-negatives] (p-50-16) at (-50/2+16,-50) {};
\node[box=0-for-negatives] (p-50-17) at (-50/2+17,-50) {};
\node[box=1-for-negatives] (p-50-18) at (-50/2+18,-50) {};
\node[box=1-for-negatives] (p-50-19) at (-50/2+19,-50) {};
\node[box=1-for-negatives] (p-50-20) at (-50/2+20,-50) {};
\node[box=2-for-negatives] (p-50-21) at (-50/2+21,-50) {};
\node[box=2-for-negatives] (p-50-22) at (-50/2+22,-50) {};
\node[box=2-for-negatives] (p-50-23) at (-50/2+23,-50) {};
\node[box=0-for-negatives] (p-50-24) at (-50/2+24,-50) {};
\node[box=0-for-negatives] (p-50-25) at (-50/2+25,-50) {};
\node[box=0-for-negatives] (p-50-26) at (-50/2+26,-50) {};
\node[box=2-for-negatives] (p-50-27) at (-50/2+27,-50) {};
\node[box=2-for-negatives] (p-50-28) at (-50/2+28,-50) {};
\node[box=2-for-negatives] (p-50-29) at (-50/2+29,-50) {};
\node[box=1-for-negatives] (p-50-30) at (-50/2+30,-50) {};
\node[box=1-for-negatives] (p-50-31) at (-50/2+31,-50) {};
\node[box=1-for-negatives] (p-50-32) at (-50/2+32,-50) {};
\node[box=0-for-negatives] (p-50-33) at (-50/2+33,-50) {};
\node[box=0-for-negatives] (p-50-34) at (-50/2+34,-50) {};
\node[box=0-for-negatives] (p-50-35) at (-50/2+35,-50) {};
\node[box=2-for-negatives] (p-50-36) at (-50/2+36,-50) {};
\node[box=2-for-negatives] (p-50-37) at (-50/2+37,-50) {};
\node[box=2-for-negatives] (p-50-38) at (-50/2+38,-50) {};
\node[box=1-for-negatives] (p-50-39) at (-50/2+39,-50) {};
\node[box=1-for-negatives] (p-50-40) at (-50/2+40,-50) {};
\node[box=1-for-negatives] (p-50-41) at (-50/2+41,-50) {};
\node[box=0-for-negatives] (p-50-42) at (-50/2+42,-50) {};
\node[box=0-for-negatives] (p-50-43) at (-50/2+43,-50) {};
\node[box=0-for-negatives] (p-50-44) at (-50/2+44,-50) {};
\node[box=2-for-negatives] (p-50-45) at (-50/2+45,-50) {};
\node[box=2-for-negatives] (p-50-46) at (-50/2+46,-50) {};
\node[box=2-for-negatives] (p-50-47) at (-50/2+47,-50) {};
\node[box=1-for-negatives] (p-50-48) at (-50/2+48,-50) {};
\node[box=1-for-negatives] (p-50-49) at (-50/2+49,-50) {};
\node[box=1-for-negatives] (p-50-50) at (-50/2+50,-50) {};
\node[box=2] (p-51-0) at (-51/2+0,-51) {};
\node[box=0-for-negatives] (p-51-1) at (-51/2+1,-51) {};
\node[box=0-for-negatives] (p-51-2) at (-51/2+2,-51) {};
\node[box=2-for-negatives] (p-51-3) at (-51/2+3,-51) {};
\node[box=0-for-negatives] (p-51-4) at (-51/2+4,-51) {};
\node[box=0-for-negatives] (p-51-5) at (-51/2+5,-51) {};
\node[box=2-for-negatives] (p-51-6) at (-51/2+6,-51) {};
\node[box=0-for-negatives] (p-51-7) at (-51/2+7,-51) {};
\node[box=0-for-negatives] (p-51-8) at (-51/2+8,-51) {};
\node[box=2-for-negatives] (p-51-9) at (-51/2+9,-51) {};
\node[box=0-for-negatives] (p-51-10) at (-51/2+10,-51) {};
\node[box=0-for-negatives] (p-51-11) at (-51/2+11,-51) {};
\node[box=2-for-negatives] (p-51-12) at (-51/2+12,-51) {};
\node[box=0-for-negatives] (p-51-13) at (-51/2+13,-51) {};
\node[box=0-for-negatives] (p-51-14) at (-51/2+14,-51) {};
\node[box=2-for-negatives] (p-51-15) at (-51/2+15,-51) {};
\node[box=0-for-negatives] (p-51-16) at (-51/2+16,-51) {};
\node[box=0-for-negatives] (p-51-17) at (-51/2+17,-51) {};
\node[box=2-for-negatives] (p-51-18) at (-51/2+18,-51) {};
\node[box=0-for-negatives] (p-51-19) at (-51/2+19,-51) {};
\node[box=0-for-negatives] (p-51-20) at (-51/2+20,-51) {};
\node[box=2-for-negatives] (p-51-21) at (-51/2+21,-51) {};
\node[box=0-for-negatives] (p-51-22) at (-51/2+22,-51) {};
\node[box=0-for-negatives] (p-51-23) at (-51/2+23,-51) {};
\node[box=2-for-negatives] (p-51-24) at (-51/2+24,-51) {};
\node[box=0-for-negatives] (p-51-25) at (-51/2+25,-51) {};
\node[box=0-for-negatives] (p-51-26) at (-51/2+26,-51) {};
\node[box=1-for-negatives] (p-51-27) at (-51/2+27,-51) {};
\node[box=0-for-negatives] (p-51-28) at (-51/2+28,-51) {};
\node[box=0-for-negatives] (p-51-29) at (-51/2+29,-51) {};
\node[box=1-for-negatives] (p-51-30) at (-51/2+30,-51) {};
\node[box=0-for-negatives] (p-51-31) at (-51/2+31,-51) {};
\node[box=0-for-negatives] (p-51-32) at (-51/2+32,-51) {};
\node[box=1-for-negatives] (p-51-33) at (-51/2+33,-51) {};
\node[box=0-for-negatives] (p-51-34) at (-51/2+34,-51) {};
\node[box=0-for-negatives] (p-51-35) at (-51/2+35,-51) {};
\node[box=1-for-negatives] (p-51-36) at (-51/2+36,-51) {};
\node[box=0-for-negatives] (p-51-37) at (-51/2+37,-51) {};
\node[box=0-for-negatives] (p-51-38) at (-51/2+38,-51) {};
\node[box=1-for-negatives] (p-51-39) at (-51/2+39,-51) {};
\node[box=0-for-negatives] (p-51-40) at (-51/2+40,-51) {};
\node[box=0-for-negatives] (p-51-41) at (-51/2+41,-51) {};
\node[box=1-for-negatives] (p-51-42) at (-51/2+42,-51) {};
\node[box=0-for-negatives] (p-51-43) at (-51/2+43,-51) {};
\node[box=0-for-negatives] (p-51-44) at (-51/2+44,-51) {};
\node[box=1-for-negatives] (p-51-45) at (-51/2+45,-51) {};
\node[box=0-for-negatives] (p-51-46) at (-51/2+46,-51) {};
\node[box=0-for-negatives] (p-51-47) at (-51/2+47,-51) {};
\node[box=1-for-negatives] (p-51-48) at (-51/2+48,-51) {};
\node[box=0-for-negatives] (p-51-49) at (-51/2+49,-51) {};
\node[box=0-for-negatives] (p-51-50) at (-51/2+50,-51) {};
\node[box=1-for-negatives] (p-51-51) at (-51/2+51,-51) {};
\node[box=1] (p-52-0) at (-52/2+0,-52) {};
\node[box=2-for-negatives] (p-52-1) at (-52/2+1,-52) {};
\node[box=0-for-negatives] (p-52-2) at (-52/2+2,-52) {};
\node[box=1-for-negatives] (p-52-3) at (-52/2+3,-52) {};
\node[box=2-for-negatives] (p-52-4) at (-52/2+4,-52) {};
\node[box=0-for-negatives] (p-52-5) at (-52/2+5,-52) {};
\node[box=1-for-negatives] (p-52-6) at (-52/2+6,-52) {};
\node[box=2-for-negatives] (p-52-7) at (-52/2+7,-52) {};
\node[box=0-for-negatives] (p-52-8) at (-52/2+8,-52) {};
\node[box=1-for-negatives] (p-52-9) at (-52/2+9,-52) {};
\node[box=2-for-negatives] (p-52-10) at (-52/2+10,-52) {};
\node[box=0-for-negatives] (p-52-11) at (-52/2+11,-52) {};
\node[box=1-for-negatives] (p-52-12) at (-52/2+12,-52) {};
\node[box=2-for-negatives] (p-52-13) at (-52/2+13,-52) {};
\node[box=0-for-negatives] (p-52-14) at (-52/2+14,-52) {};
\node[box=1-for-negatives] (p-52-15) at (-52/2+15,-52) {};
\node[box=2-for-negatives] (p-52-16) at (-52/2+16,-52) {};
\node[box=0-for-negatives] (p-52-17) at (-52/2+17,-52) {};
\node[box=1-for-negatives] (p-52-18) at (-52/2+18,-52) {};
\node[box=2-for-negatives] (p-52-19) at (-52/2+19,-52) {};
\node[box=0-for-negatives] (p-52-20) at (-52/2+20,-52) {};
\node[box=1-for-negatives] (p-52-21) at (-52/2+21,-52) {};
\node[box=2-for-negatives] (p-52-22) at (-52/2+22,-52) {};
\node[box=0-for-negatives] (p-52-23) at (-52/2+23,-52) {};
\node[box=1-for-negatives] (p-52-24) at (-52/2+24,-52) {};
\node[box=2-for-negatives] (p-52-25) at (-52/2+25,-52) {};
\node[box=0-for-negatives] (p-52-26) at (-52/2+26,-52) {};
\node[box=2-for-negatives] (p-52-27) at (-52/2+27,-52) {};
\node[box=1-for-negatives] (p-52-28) at (-52/2+28,-52) {};
\node[box=0-for-negatives] (p-52-29) at (-52/2+29,-52) {};
\node[box=2-for-negatives] (p-52-30) at (-52/2+30,-52) {};
\node[box=1-for-negatives] (p-52-31) at (-52/2+31,-52) {};
\node[box=0-for-negatives] (p-52-32) at (-52/2+32,-52) {};
\node[box=2-for-negatives] (p-52-33) at (-52/2+33,-52) {};
\node[box=1-for-negatives] (p-52-34) at (-52/2+34,-52) {};
\node[box=0-for-negatives] (p-52-35) at (-52/2+35,-52) {};
\node[box=2-for-negatives] (p-52-36) at (-52/2+36,-52) {};
\node[box=1-for-negatives] (p-52-37) at (-52/2+37,-52) {};
\node[box=0-for-negatives] (p-52-38) at (-52/2+38,-52) {};
\node[box=2-for-negatives] (p-52-39) at (-52/2+39,-52) {};
\node[box=1-for-negatives] (p-52-40) at (-52/2+40,-52) {};
\node[box=0-for-negatives] (p-52-41) at (-52/2+41,-52) {};
\node[box=2-for-negatives] (p-52-42) at (-52/2+42,-52) {};
\node[box=1-for-negatives] (p-52-43) at (-52/2+43,-52) {};
\node[box=0-for-negatives] (p-52-44) at (-52/2+44,-52) {};
\node[box=2-for-negatives] (p-52-45) at (-52/2+45,-52) {};
\node[box=1-for-negatives] (p-52-46) at (-52/2+46,-52) {};
\node[box=0-for-negatives] (p-52-47) at (-52/2+47,-52) {};
\node[box=2-for-negatives] (p-52-48) at (-52/2+48,-52) {};
\node[box=1-for-negatives] (p-52-49) at (-52/2+49,-52) {};
\node[box=0-for-negatives] (p-52-50) at (-52/2+50,-52) {};
\node[box=2-for-negatives] (p-52-51) at (-52/2+51,-52) {};
\node[box=1-for-negatives] (p-52-52) at (-52/2+52,-52) {};
\node[box=2] (p-53-0) at (-53/2+0,-53) {};
\node[box=2-for-negatives] (p-53-1) at (-53/2+1,-53) {};
\node[box=2-for-negatives] (p-53-2) at (-53/2+2,-53) {};
\node[box=2-for-negatives] (p-53-3) at (-53/2+3,-53) {};
\node[box=2-for-negatives] (p-53-4) at (-53/2+4,-53) {};
\node[box=2-for-negatives] (p-53-5) at (-53/2+5,-53) {};
\node[box=2-for-negatives] (p-53-6) at (-53/2+6,-53) {};
\node[box=2-for-negatives] (p-53-7) at (-53/2+7,-53) {};
\node[box=2-for-negatives] (p-53-8) at (-53/2+8,-53) {};
\node[box=2-for-negatives] (p-53-9) at (-53/2+9,-53) {};
\node[box=2-for-negatives] (p-53-10) at (-53/2+10,-53) {};
\node[box=2-for-negatives] (p-53-11) at (-53/2+11,-53) {};
\node[box=2-for-negatives] (p-53-12) at (-53/2+12,-53) {};
\node[box=2-for-negatives] (p-53-13) at (-53/2+13,-53) {};
\node[box=2-for-negatives] (p-53-14) at (-53/2+14,-53) {};
\node[box=2-for-negatives] (p-53-15) at (-53/2+15,-53) {};
\node[box=2-for-negatives] (p-53-16) at (-53/2+16,-53) {};
\node[box=2-for-negatives] (p-53-17) at (-53/2+17,-53) {};
\node[box=2-for-negatives] (p-53-18) at (-53/2+18,-53) {};
\node[box=2-for-negatives] (p-53-19) at (-53/2+19,-53) {};
\node[box=2-for-negatives] (p-53-20) at (-53/2+20,-53) {};
\node[box=2-for-negatives] (p-53-21) at (-53/2+21,-53) {};
\node[box=2-for-negatives] (p-53-22) at (-53/2+22,-53) {};
\node[box=2-for-negatives] (p-53-23) at (-53/2+23,-53) {};
\node[box=2-for-negatives] (p-53-24) at (-53/2+24,-53) {};
\node[box=2-for-negatives] (p-53-25) at (-53/2+25,-53) {};
\node[box=2-for-negatives] (p-53-26) at (-53/2+26,-53) {};
\node[box=1-for-negatives] (p-53-27) at (-53/2+27,-53) {};
\node[box=1-for-negatives] (p-53-28) at (-53/2+28,-53) {};
\node[box=1-for-negatives] (p-53-29) at (-53/2+29,-53) {};
\node[box=1-for-negatives] (p-53-30) at (-53/2+30,-53) {};
\node[box=1-for-negatives] (p-53-31) at (-53/2+31,-53) {};
\node[box=1-for-negatives] (p-53-32) at (-53/2+32,-53) {};
\node[box=1-for-negatives] (p-53-33) at (-53/2+33,-53) {};
\node[box=1-for-negatives] (p-53-34) at (-53/2+34,-53) {};
\node[box=1-for-negatives] (p-53-35) at (-53/2+35,-53) {};
\node[box=1-for-negatives] (p-53-36) at (-53/2+36,-53) {};
\node[box=1-for-negatives] (p-53-37) at (-53/2+37,-53) {};
\node[box=1-for-negatives] (p-53-38) at (-53/2+38,-53) {};
\node[box=1-for-negatives] (p-53-39) at (-53/2+39,-53) {};
\node[box=1-for-negatives] (p-53-40) at (-53/2+40,-53) {};
\node[box=1-for-negatives] (p-53-41) at (-53/2+41,-53) {};
\node[box=1-for-negatives] (p-53-42) at (-53/2+42,-53) {};
\node[box=1-for-negatives] (p-53-43) at (-53/2+43,-53) {};
\node[box=1-for-negatives] (p-53-44) at (-53/2+44,-53) {};
\node[box=1-for-negatives] (p-53-45) at (-53/2+45,-53) {};
\node[box=1-for-negatives] (p-53-46) at (-53/2+46,-53) {};
\node[box=1-for-negatives] (p-53-47) at (-53/2+47,-53) {};
\node[box=1-for-negatives] (p-53-48) at (-53/2+48,-53) {};
\node[box=1-for-negatives] (p-53-49) at (-53/2+49,-53) {};
\node[box=1-for-negatives] (p-53-50) at (-53/2+50,-53) {};
\node[box=1-for-negatives] (p-53-51) at (-53/2+51,-53) {};
\node[box=1-for-negatives] (p-53-52) at (-53/2+52,-53) {};
\node[box=1-for-negatives] (p-53-53) at (-53/2+53,-53) {};
\node[box=1] (p-54-0) at (-54/2+0,-54) {};
\node[box=0-for-negatives] (p-54-1) at (-54/2+1,-54) {};
\node[box=0-for-negatives] (p-54-2) at (-54/2+2,-54) {};
\node[box=0-for-negatives] (p-54-3) at (-54/2+3,-54) {};
\node[box=0-for-negatives] (p-54-4) at (-54/2+4,-54) {};
\node[box=0-for-negatives] (p-54-5) at (-54/2+5,-54) {};
\node[box=0-for-negatives] (p-54-6) at (-54/2+6,-54) {};
\node[box=0-for-negatives] (p-54-7) at (-54/2+7,-54) {};
\node[box=0-for-negatives] (p-54-8) at (-54/2+8,-54) {};
\node[box=0-for-negatives] (p-54-9) at (-54/2+9,-54) {};
\node[box=0-for-negatives] (p-54-10) at (-54/2+10,-54) {};
\node[box=0-for-negatives] (p-54-11) at (-54/2+11,-54) {};
\node[box=0-for-negatives] (p-54-12) at (-54/2+12,-54) {};
\node[box=0-for-negatives] (p-54-13) at (-54/2+13,-54) {};
\node[box=0-for-negatives] (p-54-14) at (-54/2+14,-54) {};
\node[box=0-for-negatives] (p-54-15) at (-54/2+15,-54) {};
\node[box=0-for-negatives] (p-54-16) at (-54/2+16,-54) {};
\node[box=0-for-negatives] (p-54-17) at (-54/2+17,-54) {};
\node[box=0-for-negatives] (p-54-18) at (-54/2+18,-54) {};
\node[box=0-for-negatives] (p-54-19) at (-54/2+19,-54) {};
\node[box=0-for-negatives] (p-54-20) at (-54/2+20,-54) {};
\node[box=0-for-negatives] (p-54-21) at (-54/2+21,-54) {};
\node[box=0-for-negatives] (p-54-22) at (-54/2+22,-54) {};
\node[box=0-for-negatives] (p-54-23) at (-54/2+23,-54) {};
\node[box=0-for-negatives] (p-54-24) at (-54/2+24,-54) {};
\node[box=0-for-negatives] (p-54-25) at (-54/2+25,-54) {};
\node[box=0-for-negatives] (p-54-26) at (-54/2+26,-54) {};
\node[box=1-for-negatives] (p-54-27) at (-54/2+27,-54) {};
\node[box=0-for-negatives] (p-54-28) at (-54/2+28,-54) {};
\node[box=0-for-negatives] (p-54-29) at (-54/2+29,-54) {};
\node[box=0-for-negatives] (p-54-30) at (-54/2+30,-54) {};
\node[box=0-for-negatives] (p-54-31) at (-54/2+31,-54) {};
\node[box=0-for-negatives] (p-54-32) at (-54/2+32,-54) {};
\node[box=0-for-negatives] (p-54-33) at (-54/2+33,-54) {};
\node[box=0-for-negatives] (p-54-34) at (-54/2+34,-54) {};
\node[box=0-for-negatives] (p-54-35) at (-54/2+35,-54) {};
\node[box=0-for-negatives] (p-54-36) at (-54/2+36,-54) {};
\node[box=0-for-negatives] (p-54-37) at (-54/2+37,-54) {};
\node[box=0-for-negatives] (p-54-38) at (-54/2+38,-54) {};
\node[box=0-for-negatives] (p-54-39) at (-54/2+39,-54) {};
\node[box=0-for-negatives] (p-54-40) at (-54/2+40,-54) {};
\node[box=0-for-negatives] (p-54-41) at (-54/2+41,-54) {};
\node[box=0-for-negatives] (p-54-42) at (-54/2+42,-54) {};
\node[box=0-for-negatives] (p-54-43) at (-54/2+43,-54) {};
\node[box=0-for-negatives] (p-54-44) at (-54/2+44,-54) {};
\node[box=0-for-negatives] (p-54-45) at (-54/2+45,-54) {};
\node[box=0-for-negatives] (p-54-46) at (-54/2+46,-54) {};
\node[box=0-for-negatives] (p-54-47) at (-54/2+47,-54) {};
\node[box=0-for-negatives] (p-54-48) at (-54/2+48,-54) {};
\node[box=0-for-negatives] (p-54-49) at (-54/2+49,-54) {};
\node[box=0-for-negatives] (p-54-50) at (-54/2+50,-54) {};
\node[box=0-for-negatives] (p-54-51) at (-54/2+51,-54) {};
\node[box=0-for-negatives] (p-54-52) at (-54/2+52,-54) {};
\node[box=0-for-negatives] (p-54-53) at (-54/2+53,-54) {};
\node[box=1-for-negatives] (p-54-54) at (-54/2+54,-54) {};
\node[box=2] (p-55-0) at (-55/2+0,-55) {};
\node[box=1-for-negatives] (p-55-1) at (-55/2+1,-55) {};
\node[box=0-for-negatives] (p-55-2) at (-55/2+2,-55) {};
\node[box=0-for-negatives] (p-55-3) at (-55/2+3,-55) {};
\node[box=0-for-negatives] (p-55-4) at (-55/2+4,-55) {};
\node[box=0-for-negatives] (p-55-5) at (-55/2+5,-55) {};
\node[box=0-for-negatives] (p-55-6) at (-55/2+6,-55) {};
\node[box=0-for-negatives] (p-55-7) at (-55/2+7,-55) {};
\node[box=0-for-negatives] (p-55-8) at (-55/2+8,-55) {};
\node[box=0-for-negatives] (p-55-9) at (-55/2+9,-55) {};
\node[box=0-for-negatives] (p-55-10) at (-55/2+10,-55) {};
\node[box=0-for-negatives] (p-55-11) at (-55/2+11,-55) {};
\node[box=0-for-negatives] (p-55-12) at (-55/2+12,-55) {};
\node[box=0-for-negatives] (p-55-13) at (-55/2+13,-55) {};
\node[box=0-for-negatives] (p-55-14) at (-55/2+14,-55) {};
\node[box=0-for-negatives] (p-55-15) at (-55/2+15,-55) {};
\node[box=0-for-negatives] (p-55-16) at (-55/2+16,-55) {};
\node[box=0-for-negatives] (p-55-17) at (-55/2+17,-55) {};
\node[box=0-for-negatives] (p-55-18) at (-55/2+18,-55) {};
\node[box=0-for-negatives] (p-55-19) at (-55/2+19,-55) {};
\node[box=0-for-negatives] (p-55-20) at (-55/2+20,-55) {};
\node[box=0-for-negatives] (p-55-21) at (-55/2+21,-55) {};
\node[box=0-for-negatives] (p-55-22) at (-55/2+22,-55) {};
\node[box=0-for-negatives] (p-55-23) at (-55/2+23,-55) {};
\node[box=0-for-negatives] (p-55-24) at (-55/2+24,-55) {};
\node[box=0-for-negatives] (p-55-25) at (-55/2+25,-55) {};
\node[box=0-for-negatives] (p-55-26) at (-55/2+26,-55) {};
\node[box=2-for-negatives] (p-55-27) at (-55/2+27,-55) {};
\node[box=1-for-negatives] (p-55-28) at (-55/2+28,-55) {};
\node[box=0-for-negatives] (p-55-29) at (-55/2+29,-55) {};
\node[box=0-for-negatives] (p-55-30) at (-55/2+30,-55) {};
\node[box=0-for-negatives] (p-55-31) at (-55/2+31,-55) {};
\node[box=0-for-negatives] (p-55-32) at (-55/2+32,-55) {};
\node[box=0-for-negatives] (p-55-33) at (-55/2+33,-55) {};
\node[box=0-for-negatives] (p-55-34) at (-55/2+34,-55) {};
\node[box=0-for-negatives] (p-55-35) at (-55/2+35,-55) {};
\node[box=0-for-negatives] (p-55-36) at (-55/2+36,-55) {};
\node[box=0-for-negatives] (p-55-37) at (-55/2+37,-55) {};
\node[box=0-for-negatives] (p-55-38) at (-55/2+38,-55) {};
\node[box=0-for-negatives] (p-55-39) at (-55/2+39,-55) {};
\node[box=0-for-negatives] (p-55-40) at (-55/2+40,-55) {};
\node[box=0-for-negatives] (p-55-41) at (-55/2+41,-55) {};
\node[box=0-for-negatives] (p-55-42) at (-55/2+42,-55) {};
\node[box=0-for-negatives] (p-55-43) at (-55/2+43,-55) {};
\node[box=0-for-negatives] (p-55-44) at (-55/2+44,-55) {};
\node[box=0-for-negatives] (p-55-45) at (-55/2+45,-55) {};
\node[box=0-for-negatives] (p-55-46) at (-55/2+46,-55) {};
\node[box=0-for-negatives] (p-55-47) at (-55/2+47,-55) {};
\node[box=0-for-negatives] (p-55-48) at (-55/2+48,-55) {};
\node[box=0-for-negatives] (p-55-49) at (-55/2+49,-55) {};
\node[box=0-for-negatives] (p-55-50) at (-55/2+50,-55) {};
\node[box=0-for-negatives] (p-55-51) at (-55/2+51,-55) {};
\node[box=0-for-negatives] (p-55-52) at (-55/2+52,-55) {};
\node[box=0-for-negatives] (p-55-53) at (-55/2+53,-55) {};
\node[box=2-for-negatives] (p-55-54) at (-55/2+54,-55) {};
\node[box=1-for-negatives] (p-55-55) at (-55/2+55,-55) {};
\node[box=1] (p-56-0) at (-56/2+0,-56) {};
\node[box=1-for-negatives] (p-56-1) at (-56/2+1,-56) {};
\node[box=1-for-negatives] (p-56-2) at (-56/2+2,-56) {};
\node[box=0-for-negatives] (p-56-3) at (-56/2+3,-56) {};
\node[box=0-for-negatives] (p-56-4) at (-56/2+4,-56) {};
\node[box=0-for-negatives] (p-56-5) at (-56/2+5,-56) {};
\node[box=0-for-negatives] (p-56-6) at (-56/2+6,-56) {};
\node[box=0-for-negatives] (p-56-7) at (-56/2+7,-56) {};
\node[box=0-for-negatives] (p-56-8) at (-56/2+8,-56) {};
\node[box=0-for-negatives] (p-56-9) at (-56/2+9,-56) {};
\node[box=0-for-negatives] (p-56-10) at (-56/2+10,-56) {};
\node[box=0-for-negatives] (p-56-11) at (-56/2+11,-56) {};
\node[box=0-for-negatives] (p-56-12) at (-56/2+12,-56) {};
\node[box=0-for-negatives] (p-56-13) at (-56/2+13,-56) {};
\node[box=0-for-negatives] (p-56-14) at (-56/2+14,-56) {};
\node[box=0-for-negatives] (p-56-15) at (-56/2+15,-56) {};
\node[box=0-for-negatives] (p-56-16) at (-56/2+16,-56) {};
\node[box=0-for-negatives] (p-56-17) at (-56/2+17,-56) {};
\node[box=0-for-negatives] (p-56-18) at (-56/2+18,-56) {};
\node[box=0-for-negatives] (p-56-19) at (-56/2+19,-56) {};
\node[box=0-for-negatives] (p-56-20) at (-56/2+20,-56) {};
\node[box=0-for-negatives] (p-56-21) at (-56/2+21,-56) {};
\node[box=0-for-negatives] (p-56-22) at (-56/2+22,-56) {};
\node[box=0-for-negatives] (p-56-23) at (-56/2+23,-56) {};
\node[box=0-for-negatives] (p-56-24) at (-56/2+24,-56) {};
\node[box=0-for-negatives] (p-56-25) at (-56/2+25,-56) {};
\node[box=0-for-negatives] (p-56-26) at (-56/2+26,-56) {};
\node[box=1-for-negatives] (p-56-27) at (-56/2+27,-56) {};
\node[box=1-for-negatives] (p-56-28) at (-56/2+28,-56) {};
\node[box=1-for-negatives] (p-56-29) at (-56/2+29,-56) {};
\node[box=0-for-negatives] (p-56-30) at (-56/2+30,-56) {};
\node[box=0-for-negatives] (p-56-31) at (-56/2+31,-56) {};
\node[box=0-for-negatives] (p-56-32) at (-56/2+32,-56) {};
\node[box=0-for-negatives] (p-56-33) at (-56/2+33,-56) {};
\node[box=0-for-negatives] (p-56-34) at (-56/2+34,-56) {};
\node[box=0-for-negatives] (p-56-35) at (-56/2+35,-56) {};
\node[box=0-for-negatives] (p-56-36) at (-56/2+36,-56) {};
\node[box=0-for-negatives] (p-56-37) at (-56/2+37,-56) {};
\node[box=0-for-negatives] (p-56-38) at (-56/2+38,-56) {};
\node[box=0-for-negatives] (p-56-39) at (-56/2+39,-56) {};
\node[box=0-for-negatives] (p-56-40) at (-56/2+40,-56) {};
\node[box=0-for-negatives] (p-56-41) at (-56/2+41,-56) {};
\node[box=0-for-negatives] (p-56-42) at (-56/2+42,-56) {};
\node[box=0-for-negatives] (p-56-43) at (-56/2+43,-56) {};
\node[box=0-for-negatives] (p-56-44) at (-56/2+44,-56) {};
\node[box=0-for-negatives] (p-56-45) at (-56/2+45,-56) {};
\node[box=0-for-negatives] (p-56-46) at (-56/2+46,-56) {};
\node[box=0-for-negatives] (p-56-47) at (-56/2+47,-56) {};
\node[box=0-for-negatives] (p-56-48) at (-56/2+48,-56) {};
\node[box=0-for-negatives] (p-56-49) at (-56/2+49,-56) {};
\node[box=0-for-negatives] (p-56-50) at (-56/2+50,-56) {};
\node[box=0-for-negatives] (p-56-51) at (-56/2+51,-56) {};
\node[box=0-for-negatives] (p-56-52) at (-56/2+52,-56) {};
\node[box=0-for-negatives] (p-56-53) at (-56/2+53,-56) {};
\node[box=1-for-negatives] (p-56-54) at (-56/2+54,-56) {};
\node[box=1-for-negatives] (p-56-55) at (-56/2+55,-56) {};
\node[box=1-for-negatives] (p-56-56) at (-56/2+56,-56) {};
\node[box=2] (p-57-0) at (-57/2+0,-57) {};
\node[box=0-for-negatives] (p-57-1) at (-57/2+1,-57) {};
\node[box=0-for-negatives] (p-57-2) at (-57/2+2,-57) {};
\node[box=1-for-negatives] (p-57-3) at (-57/2+3,-57) {};
\node[box=0-for-negatives] (p-57-4) at (-57/2+4,-57) {};
\node[box=0-for-negatives] (p-57-5) at (-57/2+5,-57) {};
\node[box=0-for-negatives] (p-57-6) at (-57/2+6,-57) {};
\node[box=0-for-negatives] (p-57-7) at (-57/2+7,-57) {};
\node[box=0-for-negatives] (p-57-8) at (-57/2+8,-57) {};
\node[box=0-for-negatives] (p-57-9) at (-57/2+9,-57) {};
\node[box=0-for-negatives] (p-57-10) at (-57/2+10,-57) {};
\node[box=0-for-negatives] (p-57-11) at (-57/2+11,-57) {};
\node[box=0-for-negatives] (p-57-12) at (-57/2+12,-57) {};
\node[box=0-for-negatives] (p-57-13) at (-57/2+13,-57) {};
\node[box=0-for-negatives] (p-57-14) at (-57/2+14,-57) {};
\node[box=0-for-negatives] (p-57-15) at (-57/2+15,-57) {};
\node[box=0-for-negatives] (p-57-16) at (-57/2+16,-57) {};
\node[box=0-for-negatives] (p-57-17) at (-57/2+17,-57) {};
\node[box=0-for-negatives] (p-57-18) at (-57/2+18,-57) {};
\node[box=0-for-negatives] (p-57-19) at (-57/2+19,-57) {};
\node[box=0-for-negatives] (p-57-20) at (-57/2+20,-57) {};
\node[box=0-for-negatives] (p-57-21) at (-57/2+21,-57) {};
\node[box=0-for-negatives] (p-57-22) at (-57/2+22,-57) {};
\node[box=0-for-negatives] (p-57-23) at (-57/2+23,-57) {};
\node[box=0-for-negatives] (p-57-24) at (-57/2+24,-57) {};
\node[box=0-for-negatives] (p-57-25) at (-57/2+25,-57) {};
\node[box=0-for-negatives] (p-57-26) at (-57/2+26,-57) {};
\node[box=2-for-negatives] (p-57-27) at (-57/2+27,-57) {};
\node[box=0-for-negatives] (p-57-28) at (-57/2+28,-57) {};
\node[box=0-for-negatives] (p-57-29) at (-57/2+29,-57) {};
\node[box=1-for-negatives] (p-57-30) at (-57/2+30,-57) {};
\node[box=0-for-negatives] (p-57-31) at (-57/2+31,-57) {};
\node[box=0-for-negatives] (p-57-32) at (-57/2+32,-57) {};
\node[box=0-for-negatives] (p-57-33) at (-57/2+33,-57) {};
\node[box=0-for-negatives] (p-57-34) at (-57/2+34,-57) {};
\node[box=0-for-negatives] (p-57-35) at (-57/2+35,-57) {};
\node[box=0-for-negatives] (p-57-36) at (-57/2+36,-57) {};
\node[box=0-for-negatives] (p-57-37) at (-57/2+37,-57) {};
\node[box=0-for-negatives] (p-57-38) at (-57/2+38,-57) {};
\node[box=0-for-negatives] (p-57-39) at (-57/2+39,-57) {};
\node[box=0-for-negatives] (p-57-40) at (-57/2+40,-57) {};
\node[box=0-for-negatives] (p-57-41) at (-57/2+41,-57) {};
\node[box=0-for-negatives] (p-57-42) at (-57/2+42,-57) {};
\node[box=0-for-negatives] (p-57-43) at (-57/2+43,-57) {};
\node[box=0-for-negatives] (p-57-44) at (-57/2+44,-57) {};
\node[box=0-for-negatives] (p-57-45) at (-57/2+45,-57) {};
\node[box=0-for-negatives] (p-57-46) at (-57/2+46,-57) {};
\node[box=0-for-negatives] (p-57-47) at (-57/2+47,-57) {};
\node[box=0-for-negatives] (p-57-48) at (-57/2+48,-57) {};
\node[box=0-for-negatives] (p-57-49) at (-57/2+49,-57) {};
\node[box=0-for-negatives] (p-57-50) at (-57/2+50,-57) {};
\node[box=0-for-negatives] (p-57-51) at (-57/2+51,-57) {};
\node[box=0-for-negatives] (p-57-52) at (-57/2+52,-57) {};
\node[box=0-for-negatives] (p-57-53) at (-57/2+53,-57) {};
\node[box=2-for-negatives] (p-57-54) at (-57/2+54,-57) {};
\node[box=0-for-negatives] (p-57-55) at (-57/2+55,-57) {};
\node[box=0-for-negatives] (p-57-56) at (-57/2+56,-57) {};
\node[box=1-for-negatives] (p-57-57) at (-57/2+57,-57) {};
\node[box=1] (p-58-0) at (-58/2+0,-58) {};
\node[box=2-for-negatives] (p-58-1) at (-58/2+1,-58) {};
\node[box=0-for-negatives] (p-58-2) at (-58/2+2,-58) {};
\node[box=2-for-negatives] (p-58-3) at (-58/2+3,-58) {};
\node[box=1-for-negatives] (p-58-4) at (-58/2+4,-58) {};
\node[box=0-for-negatives] (p-58-5) at (-58/2+5,-58) {};
\node[box=0-for-negatives] (p-58-6) at (-58/2+6,-58) {};
\node[box=0-for-negatives] (p-58-7) at (-58/2+7,-58) {};
\node[box=0-for-negatives] (p-58-8) at (-58/2+8,-58) {};
\node[box=0-for-negatives] (p-58-9) at (-58/2+9,-58) {};
\node[box=0-for-negatives] (p-58-10) at (-58/2+10,-58) {};
\node[box=0-for-negatives] (p-58-11) at (-58/2+11,-58) {};
\node[box=0-for-negatives] (p-58-12) at (-58/2+12,-58) {};
\node[box=0-for-negatives] (p-58-13) at (-58/2+13,-58) {};
\node[box=0-for-negatives] (p-58-14) at (-58/2+14,-58) {};
\node[box=0-for-negatives] (p-58-15) at (-58/2+15,-58) {};
\node[box=0-for-negatives] (p-58-16) at (-58/2+16,-58) {};
\node[box=0-for-negatives] (p-58-17) at (-58/2+17,-58) {};
\node[box=0-for-negatives] (p-58-18) at (-58/2+18,-58) {};
\node[box=0-for-negatives] (p-58-19) at (-58/2+19,-58) {};
\node[box=0-for-negatives] (p-58-20) at (-58/2+20,-58) {};
\node[box=0-for-negatives] (p-58-21) at (-58/2+21,-58) {};
\node[box=0-for-negatives] (p-58-22) at (-58/2+22,-58) {};
\node[box=0-for-negatives] (p-58-23) at (-58/2+23,-58) {};
\node[box=0-for-negatives] (p-58-24) at (-58/2+24,-58) {};
\node[box=0-for-negatives] (p-58-25) at (-58/2+25,-58) {};
\node[box=0-for-negatives] (p-58-26) at (-58/2+26,-58) {};
\node[box=1-for-negatives] (p-58-27) at (-58/2+27,-58) {};
\node[box=2-for-negatives] (p-58-28) at (-58/2+28,-58) {};
\node[box=0-for-negatives] (p-58-29) at (-58/2+29,-58) {};
\node[box=2-for-negatives] (p-58-30) at (-58/2+30,-58) {};
\node[box=1-for-negatives] (p-58-31) at (-58/2+31,-58) {};
\node[box=0-for-negatives] (p-58-32) at (-58/2+32,-58) {};
\node[box=0-for-negatives] (p-58-33) at (-58/2+33,-58) {};
\node[box=0-for-negatives] (p-58-34) at (-58/2+34,-58) {};
\node[box=0-for-negatives] (p-58-35) at (-58/2+35,-58) {};
\node[box=0-for-negatives] (p-58-36) at (-58/2+36,-58) {};
\node[box=0-for-negatives] (p-58-37) at (-58/2+37,-58) {};
\node[box=0-for-negatives] (p-58-38) at (-58/2+38,-58) {};
\node[box=0-for-negatives] (p-58-39) at (-58/2+39,-58) {};
\node[box=0-for-negatives] (p-58-40) at (-58/2+40,-58) {};
\node[box=0-for-negatives] (p-58-41) at (-58/2+41,-58) {};
\node[box=0-for-negatives] (p-58-42) at (-58/2+42,-58) {};
\node[box=0-for-negatives] (p-58-43) at (-58/2+43,-58) {};
\node[box=0-for-negatives] (p-58-44) at (-58/2+44,-58) {};
\node[box=0-for-negatives] (p-58-45) at (-58/2+45,-58) {};
\node[box=0-for-negatives] (p-58-46) at (-58/2+46,-58) {};
\node[box=0-for-negatives] (p-58-47) at (-58/2+47,-58) {};
\node[box=0-for-negatives] (p-58-48) at (-58/2+48,-58) {};
\node[box=0-for-negatives] (p-58-49) at (-58/2+49,-58) {};
\node[box=0-for-negatives] (p-58-50) at (-58/2+50,-58) {};
\node[box=0-for-negatives] (p-58-51) at (-58/2+51,-58) {};
\node[box=0-for-negatives] (p-58-52) at (-58/2+52,-58) {};
\node[box=0-for-negatives] (p-58-53) at (-58/2+53,-58) {};
\node[box=1-for-negatives] (p-58-54) at (-58/2+54,-58) {};
\node[box=2-for-negatives] (p-58-55) at (-58/2+55,-58) {};
\node[box=0-for-negatives] (p-58-56) at (-58/2+56,-58) {};
\node[box=2-for-negatives] (p-58-57) at (-58/2+57,-58) {};
\node[box=1-for-negatives] (p-58-58) at (-58/2+58,-58) {};
\node[box=2] (p-59-0) at (-59/2+0,-59) {};
\node[box=2-for-negatives] (p-59-1) at (-59/2+1,-59) {};
\node[box=2-for-negatives] (p-59-2) at (-59/2+2,-59) {};
\node[box=1-for-negatives] (p-59-3) at (-59/2+3,-59) {};
\node[box=1-for-negatives] (p-59-4) at (-59/2+4,-59) {};
\node[box=1-for-negatives] (p-59-5) at (-59/2+5,-59) {};
\node[box=0-for-negatives] (p-59-6) at (-59/2+6,-59) {};
\node[box=0-for-negatives] (p-59-7) at (-59/2+7,-59) {};
\node[box=0-for-negatives] (p-59-8) at (-59/2+8,-59) {};
\node[box=0-for-negatives] (p-59-9) at (-59/2+9,-59) {};
\node[box=0-for-negatives] (p-59-10) at (-59/2+10,-59) {};
\node[box=0-for-negatives] (p-59-11) at (-59/2+11,-59) {};
\node[box=0-for-negatives] (p-59-12) at (-59/2+12,-59) {};
\node[box=0-for-negatives] (p-59-13) at (-59/2+13,-59) {};
\node[box=0-for-negatives] (p-59-14) at (-59/2+14,-59) {};
\node[box=0-for-negatives] (p-59-15) at (-59/2+15,-59) {};
\node[box=0-for-negatives] (p-59-16) at (-59/2+16,-59) {};
\node[box=0-for-negatives] (p-59-17) at (-59/2+17,-59) {};
\node[box=0-for-negatives] (p-59-18) at (-59/2+18,-59) {};
\node[box=0-for-negatives] (p-59-19) at (-59/2+19,-59) {};
\node[box=0-for-negatives] (p-59-20) at (-59/2+20,-59) {};
\node[box=0-for-negatives] (p-59-21) at (-59/2+21,-59) {};
\node[box=0-for-negatives] (p-59-22) at (-59/2+22,-59) {};
\node[box=0-for-negatives] (p-59-23) at (-59/2+23,-59) {};
\node[box=0-for-negatives] (p-59-24) at (-59/2+24,-59) {};
\node[box=0-for-negatives] (p-59-25) at (-59/2+25,-59) {};
\node[box=0-for-negatives] (p-59-26) at (-59/2+26,-59) {};
\node[box=2-for-negatives] (p-59-27) at (-59/2+27,-59) {};
\node[box=2-for-negatives] (p-59-28) at (-59/2+28,-59) {};
\node[box=2-for-negatives] (p-59-29) at (-59/2+29,-59) {};
\node[box=1-for-negatives] (p-59-30) at (-59/2+30,-59) {};
\node[box=1-for-negatives] (p-59-31) at (-59/2+31,-59) {};
\node[box=1-for-negatives] (p-59-32) at (-59/2+32,-59) {};
\node[box=0-for-negatives] (p-59-33) at (-59/2+33,-59) {};
\node[box=0-for-negatives] (p-59-34) at (-59/2+34,-59) {};
\node[box=0-for-negatives] (p-59-35) at (-59/2+35,-59) {};
\node[box=0-for-negatives] (p-59-36) at (-59/2+36,-59) {};
\node[box=0-for-negatives] (p-59-37) at (-59/2+37,-59) {};
\node[box=0-for-negatives] (p-59-38) at (-59/2+38,-59) {};
\node[box=0-for-negatives] (p-59-39) at (-59/2+39,-59) {};
\node[box=0-for-negatives] (p-59-40) at (-59/2+40,-59) {};
\node[box=0-for-negatives] (p-59-41) at (-59/2+41,-59) {};
\node[box=0-for-negatives] (p-59-42) at (-59/2+42,-59) {};
\node[box=0-for-negatives] (p-59-43) at (-59/2+43,-59) {};
\node[box=0-for-negatives] (p-59-44) at (-59/2+44,-59) {};
\node[box=0-for-negatives] (p-59-45) at (-59/2+45,-59) {};
\node[box=0-for-negatives] (p-59-46) at (-59/2+46,-59) {};
\node[box=0-for-negatives] (p-59-47) at (-59/2+47,-59) {};
\node[box=0-for-negatives] (p-59-48) at (-59/2+48,-59) {};
\node[box=0-for-negatives] (p-59-49) at (-59/2+49,-59) {};
\node[box=0-for-negatives] (p-59-50) at (-59/2+50,-59) {};
\node[box=0-for-negatives] (p-59-51) at (-59/2+51,-59) {};
\node[box=0-for-negatives] (p-59-52) at (-59/2+52,-59) {};
\node[box=0-for-negatives] (p-59-53) at (-59/2+53,-59) {};
\node[box=2-for-negatives] (p-59-54) at (-59/2+54,-59) {};
\node[box=2-for-negatives] (p-59-55) at (-59/2+55,-59) {};
\node[box=2-for-negatives] (p-59-56) at (-59/2+56,-59) {};
\node[box=1-for-negatives] (p-59-57) at (-59/2+57,-59) {};
\node[box=1-for-negatives] (p-59-58) at (-59/2+58,-59) {};
\node[box=1-for-negatives] (p-59-59) at (-59/2+59,-59) {};
\node[box=1] (p-60-0) at (-60/2+0,-60) {};
\node[box=0-for-negatives] (p-60-1) at (-60/2+1,-60) {};
\node[box=0-for-negatives] (p-60-2) at (-60/2+2,-60) {};
\node[box=1-for-negatives] (p-60-3) at (-60/2+3,-60) {};
\node[box=0-for-negatives] (p-60-4) at (-60/2+4,-60) {};
\node[box=0-for-negatives] (p-60-5) at (-60/2+5,-60) {};
\node[box=1-for-negatives] (p-60-6) at (-60/2+6,-60) {};
\node[box=0-for-negatives] (p-60-7) at (-60/2+7,-60) {};
\node[box=0-for-negatives] (p-60-8) at (-60/2+8,-60) {};
\node[box=0-for-negatives] (p-60-9) at (-60/2+9,-60) {};
\node[box=0-for-negatives] (p-60-10) at (-60/2+10,-60) {};
\node[box=0-for-negatives] (p-60-11) at (-60/2+11,-60) {};
\node[box=0-for-negatives] (p-60-12) at (-60/2+12,-60) {};
\node[box=0-for-negatives] (p-60-13) at (-60/2+13,-60) {};
\node[box=0-for-negatives] (p-60-14) at (-60/2+14,-60) {};
\node[box=0-for-negatives] (p-60-15) at (-60/2+15,-60) {};
\node[box=0-for-negatives] (p-60-16) at (-60/2+16,-60) {};
\node[box=0-for-negatives] (p-60-17) at (-60/2+17,-60) {};
\node[box=0-for-negatives] (p-60-18) at (-60/2+18,-60) {};
\node[box=0-for-negatives] (p-60-19) at (-60/2+19,-60) {};
\node[box=0-for-negatives] (p-60-20) at (-60/2+20,-60) {};
\node[box=0-for-negatives] (p-60-21) at (-60/2+21,-60) {};
\node[box=0-for-negatives] (p-60-22) at (-60/2+22,-60) {};
\node[box=0-for-negatives] (p-60-23) at (-60/2+23,-60) {};
\node[box=0-for-negatives] (p-60-24) at (-60/2+24,-60) {};
\node[box=0-for-negatives] (p-60-25) at (-60/2+25,-60) {};
\node[box=0-for-negatives] (p-60-26) at (-60/2+26,-60) {};
\node[box=1-for-negatives] (p-60-27) at (-60/2+27,-60) {};
\node[box=0-for-negatives] (p-60-28) at (-60/2+28,-60) {};
\node[box=0-for-negatives] (p-60-29) at (-60/2+29,-60) {};
\node[box=1-for-negatives] (p-60-30) at (-60/2+30,-60) {};
\node[box=0-for-negatives] (p-60-31) at (-60/2+31,-60) {};
\node[box=0-for-negatives] (p-60-32) at (-60/2+32,-60) {};
\node[box=1-for-negatives] (p-60-33) at (-60/2+33,-60) {};
\node[box=0-for-negatives] (p-60-34) at (-60/2+34,-60) {};
\node[box=0-for-negatives] (p-60-35) at (-60/2+35,-60) {};
\node[box=0-for-negatives] (p-60-36) at (-60/2+36,-60) {};
\node[box=0-for-negatives] (p-60-37) at (-60/2+37,-60) {};
\node[box=0-for-negatives] (p-60-38) at (-60/2+38,-60) {};
\node[box=0-for-negatives] (p-60-39) at (-60/2+39,-60) {};
\node[box=0-for-negatives] (p-60-40) at (-60/2+40,-60) {};
\node[box=0-for-negatives] (p-60-41) at (-60/2+41,-60) {};
\node[box=0-for-negatives] (p-60-42) at (-60/2+42,-60) {};
\node[box=0-for-negatives] (p-60-43) at (-60/2+43,-60) {};
\node[box=0-for-negatives] (p-60-44) at (-60/2+44,-60) {};
\node[box=0-for-negatives] (p-60-45) at (-60/2+45,-60) {};
\node[box=0-for-negatives] (p-60-46) at (-60/2+46,-60) {};
\node[box=0-for-negatives] (p-60-47) at (-60/2+47,-60) {};
\node[box=0-for-negatives] (p-60-48) at (-60/2+48,-60) {};
\node[box=0-for-negatives] (p-60-49) at (-60/2+49,-60) {};
\node[box=0-for-negatives] (p-60-50) at (-60/2+50,-60) {};
\node[box=0-for-negatives] (p-60-51) at (-60/2+51,-60) {};
\node[box=0-for-negatives] (p-60-52) at (-60/2+52,-60) {};
\node[box=0-for-negatives] (p-60-53) at (-60/2+53,-60) {};
\node[box=1-for-negatives] (p-60-54) at (-60/2+54,-60) {};
\node[box=0-for-negatives] (p-60-55) at (-60/2+55,-60) {};
\node[box=0-for-negatives] (p-60-56) at (-60/2+56,-60) {};
\node[box=1-for-negatives] (p-60-57) at (-60/2+57,-60) {};
\node[box=0-for-negatives] (p-60-58) at (-60/2+58,-60) {};
\node[box=0-for-negatives] (p-60-59) at (-60/2+59,-60) {};
\node[box=1-for-negatives] (p-60-60) at (-60/2+60,-60) {};
\node[box=2] (p-61-0) at (-61/2+0,-61) {};
\node[box=1-for-negatives] (p-61-1) at (-61/2+1,-61) {};
\node[box=0-for-negatives] (p-61-2) at (-61/2+2,-61) {};
\node[box=2-for-negatives] (p-61-3) at (-61/2+3,-61) {};
\node[box=1-for-negatives] (p-61-4) at (-61/2+4,-61) {};
\node[box=0-for-negatives] (p-61-5) at (-61/2+5,-61) {};
\node[box=2-for-negatives] (p-61-6) at (-61/2+6,-61) {};
\node[box=1-for-negatives] (p-61-7) at (-61/2+7,-61) {};
\node[box=0-for-negatives] (p-61-8) at (-61/2+8,-61) {};
\node[box=0-for-negatives] (p-61-9) at (-61/2+9,-61) {};
\node[box=0-for-negatives] (p-61-10) at (-61/2+10,-61) {};
\node[box=0-for-negatives] (p-61-11) at (-61/2+11,-61) {};
\node[box=0-for-negatives] (p-61-12) at (-61/2+12,-61) {};
\node[box=0-for-negatives] (p-61-13) at (-61/2+13,-61) {};
\node[box=0-for-negatives] (p-61-14) at (-61/2+14,-61) {};
\node[box=0-for-negatives] (p-61-15) at (-61/2+15,-61) {};
\node[box=0-for-negatives] (p-61-16) at (-61/2+16,-61) {};
\node[box=0-for-negatives] (p-61-17) at (-61/2+17,-61) {};
\node[box=0-for-negatives] (p-61-18) at (-61/2+18,-61) {};
\node[box=0-for-negatives] (p-61-19) at (-61/2+19,-61) {};
\node[box=0-for-negatives] (p-61-20) at (-61/2+20,-61) {};
\node[box=0-for-negatives] (p-61-21) at (-61/2+21,-61) {};
\node[box=0-for-negatives] (p-61-22) at (-61/2+22,-61) {};
\node[box=0-for-negatives] (p-61-23) at (-61/2+23,-61) {};
\node[box=0-for-negatives] (p-61-24) at (-61/2+24,-61) {};
\node[box=0-for-negatives] (p-61-25) at (-61/2+25,-61) {};
\node[box=0-for-negatives] (p-61-26) at (-61/2+26,-61) {};
\node[box=2-for-negatives] (p-61-27) at (-61/2+27,-61) {};
\node[box=1-for-negatives] (p-61-28) at (-61/2+28,-61) {};
\node[box=0-for-negatives] (p-61-29) at (-61/2+29,-61) {};
\node[box=2-for-negatives] (p-61-30) at (-61/2+30,-61) {};
\node[box=1-for-negatives] (p-61-31) at (-61/2+31,-61) {};
\node[box=0-for-negatives] (p-61-32) at (-61/2+32,-61) {};
\node[box=2-for-negatives] (p-61-33) at (-61/2+33,-61) {};
\node[box=1-for-negatives] (p-61-34) at (-61/2+34,-61) {};
\node[box=0-for-negatives] (p-61-35) at (-61/2+35,-61) {};
\node[box=0-for-negatives] (p-61-36) at (-61/2+36,-61) {};
\node[box=0-for-negatives] (p-61-37) at (-61/2+37,-61) {};
\node[box=0-for-negatives] (p-61-38) at (-61/2+38,-61) {};
\node[box=0-for-negatives] (p-61-39) at (-61/2+39,-61) {};
\node[box=0-for-negatives] (p-61-40) at (-61/2+40,-61) {};
\node[box=0-for-negatives] (p-61-41) at (-61/2+41,-61) {};
\node[box=0-for-negatives] (p-61-42) at (-61/2+42,-61) {};
\node[box=0-for-negatives] (p-61-43) at (-61/2+43,-61) {};
\node[box=0-for-negatives] (p-61-44) at (-61/2+44,-61) {};
\node[box=0-for-negatives] (p-61-45) at (-61/2+45,-61) {};
\node[box=0-for-negatives] (p-61-46) at (-61/2+46,-61) {};
\node[box=0-for-negatives] (p-61-47) at (-61/2+47,-61) {};
\node[box=0-for-negatives] (p-61-48) at (-61/2+48,-61) {};
\node[box=0-for-negatives] (p-61-49) at (-61/2+49,-61) {};
\node[box=0-for-negatives] (p-61-50) at (-61/2+50,-61) {};
\node[box=0-for-negatives] (p-61-51) at (-61/2+51,-61) {};
\node[box=0-for-negatives] (p-61-52) at (-61/2+52,-61) {};
\node[box=0-for-negatives] (p-61-53) at (-61/2+53,-61) {};
\node[box=2-for-negatives] (p-61-54) at (-61/2+54,-61) {};
\node[box=1-for-negatives] (p-61-55) at (-61/2+55,-61) {};
\node[box=0-for-negatives] (p-61-56) at (-61/2+56,-61) {};
\node[box=2-for-negatives] (p-61-57) at (-61/2+57,-61) {};
\node[box=1-for-negatives] (p-61-58) at (-61/2+58,-61) {};
\node[box=0-for-negatives] (p-61-59) at (-61/2+59,-61) {};
\node[box=2-for-negatives] (p-61-60) at (-61/2+60,-61) {};
\node[box=1-for-negatives] (p-61-61) at (-61/2+61,-61) {};
\node[box=1] (p-62-0) at (-62/2+0,-62) {};
\node[box=1-for-negatives] (p-62-1) at (-62/2+1,-62) {};
\node[box=1-for-negatives] (p-62-2) at (-62/2+2,-62) {};
\node[box=1-for-negatives] (p-62-3) at (-62/2+3,-62) {};
\node[box=1-for-negatives] (p-62-4) at (-62/2+4,-62) {};
\node[box=1-for-negatives] (p-62-5) at (-62/2+5,-62) {};
\node[box=1-for-negatives] (p-62-6) at (-62/2+6,-62) {};
\node[box=1-for-negatives] (p-62-7) at (-62/2+7,-62) {};
\node[box=1-for-negatives] (p-62-8) at (-62/2+8,-62) {};
\node[box=0-for-negatives] (p-62-9) at (-62/2+9,-62) {};
\node[box=0-for-negatives] (p-62-10) at (-62/2+10,-62) {};
\node[box=0-for-negatives] (p-62-11) at (-62/2+11,-62) {};
\node[box=0-for-negatives] (p-62-12) at (-62/2+12,-62) {};
\node[box=0-for-negatives] (p-62-13) at (-62/2+13,-62) {};
\node[box=0-for-negatives] (p-62-14) at (-62/2+14,-62) {};
\node[box=0-for-negatives] (p-62-15) at (-62/2+15,-62) {};
\node[box=0-for-negatives] (p-62-16) at (-62/2+16,-62) {};
\node[box=0-for-negatives] (p-62-17) at (-62/2+17,-62) {};
\node[box=0-for-negatives] (p-62-18) at (-62/2+18,-62) {};
\node[box=0-for-negatives] (p-62-19) at (-62/2+19,-62) {};
\node[box=0-for-negatives] (p-62-20) at (-62/2+20,-62) {};
\node[box=0-for-negatives] (p-62-21) at (-62/2+21,-62) {};
\node[box=0-for-negatives] (p-62-22) at (-62/2+22,-62) {};
\node[box=0-for-negatives] (p-62-23) at (-62/2+23,-62) {};
\node[box=0-for-negatives] (p-62-24) at (-62/2+24,-62) {};
\node[box=0-for-negatives] (p-62-25) at (-62/2+25,-62) {};
\node[box=0-for-negatives] (p-62-26) at (-62/2+26,-62) {};
\node[box=1-for-negatives] (p-62-27) at (-62/2+27,-62) {};
\node[box=1-for-negatives] (p-62-28) at (-62/2+28,-62) {};
\node[box=1-for-negatives] (p-62-29) at (-62/2+29,-62) {};
\node[box=1-for-negatives] (p-62-30) at (-62/2+30,-62) {};
\node[box=1-for-negatives] (p-62-31) at (-62/2+31,-62) {};
\node[box=1-for-negatives] (p-62-32) at (-62/2+32,-62) {};
\node[box=1-for-negatives] (p-62-33) at (-62/2+33,-62) {};
\node[box=1-for-negatives] (p-62-34) at (-62/2+34,-62) {};
\node[box=1-for-negatives] (p-62-35) at (-62/2+35,-62) {};
\node[box=0-for-negatives] (p-62-36) at (-62/2+36,-62) {};
\node[box=0-for-negatives] (p-62-37) at (-62/2+37,-62) {};
\node[box=0-for-negatives] (p-62-38) at (-62/2+38,-62) {};
\node[box=0-for-negatives] (p-62-39) at (-62/2+39,-62) {};
\node[box=0-for-negatives] (p-62-40) at (-62/2+40,-62) {};
\node[box=0-for-negatives] (p-62-41) at (-62/2+41,-62) {};
\node[box=0-for-negatives] (p-62-42) at (-62/2+42,-62) {};
\node[box=0-for-negatives] (p-62-43) at (-62/2+43,-62) {};
\node[box=0-for-negatives] (p-62-44) at (-62/2+44,-62) {};
\node[box=0-for-negatives] (p-62-45) at (-62/2+45,-62) {};
\node[box=0-for-negatives] (p-62-46) at (-62/2+46,-62) {};
\node[box=0-for-negatives] (p-62-47) at (-62/2+47,-62) {};
\node[box=0-for-negatives] (p-62-48) at (-62/2+48,-62) {};
\node[box=0-for-negatives] (p-62-49) at (-62/2+49,-62) {};
\node[box=0-for-negatives] (p-62-50) at (-62/2+50,-62) {};
\node[box=0-for-negatives] (p-62-51) at (-62/2+51,-62) {};
\node[box=0-for-negatives] (p-62-52) at (-62/2+52,-62) {};
\node[box=0-for-negatives] (p-62-53) at (-62/2+53,-62) {};
\node[box=1-for-negatives] (p-62-54) at (-62/2+54,-62) {};
\node[box=1-for-negatives] (p-62-55) at (-62/2+55,-62) {};
\node[box=1-for-negatives] (p-62-56) at (-62/2+56,-62) {};
\node[box=1-for-negatives] (p-62-57) at (-62/2+57,-62) {};
\node[box=1-for-negatives] (p-62-58) at (-62/2+58,-62) {};
\node[box=1-for-negatives] (p-62-59) at (-62/2+59,-62) {};
\node[box=1-for-negatives] (p-62-60) at (-62/2+60,-62) {};
\node[box=1-for-negatives] (p-62-61) at (-62/2+61,-62) {};
\node[box=1-for-negatives] (p-62-62) at (-62/2+62,-62) {};
\node[box=2] (p-63-0) at (-63/2+0,-63) {};
\node[box=0-for-negatives] (p-63-1) at (-63/2+1,-63) {};
\node[box=0-for-negatives] (p-63-2) at (-63/2+2,-63) {};
\node[box=0-for-negatives] (p-63-3) at (-63/2+3,-63) {};
\node[box=0-for-negatives] (p-63-4) at (-63/2+4,-63) {};
\node[box=0-for-negatives] (p-63-5) at (-63/2+5,-63) {};
\node[box=0-for-negatives] (p-63-6) at (-63/2+6,-63) {};
\node[box=0-for-negatives] (p-63-7) at (-63/2+7,-63) {};
\node[box=0-for-negatives] (p-63-8) at (-63/2+8,-63) {};
\node[box=1-for-negatives] (p-63-9) at (-63/2+9,-63) {};
\node[box=0-for-negatives] (p-63-10) at (-63/2+10,-63) {};
\node[box=0-for-negatives] (p-63-11) at (-63/2+11,-63) {};
\node[box=0-for-negatives] (p-63-12) at (-63/2+12,-63) {};
\node[box=0-for-negatives] (p-63-13) at (-63/2+13,-63) {};
\node[box=0-for-negatives] (p-63-14) at (-63/2+14,-63) {};
\node[box=0-for-negatives] (p-63-15) at (-63/2+15,-63) {};
\node[box=0-for-negatives] (p-63-16) at (-63/2+16,-63) {};
\node[box=0-for-negatives] (p-63-17) at (-63/2+17,-63) {};
\node[box=0-for-negatives] (p-63-18) at (-63/2+18,-63) {};
\node[box=0-for-negatives] (p-63-19) at (-63/2+19,-63) {};
\node[box=0-for-negatives] (p-63-20) at (-63/2+20,-63) {};
\node[box=0-for-negatives] (p-63-21) at (-63/2+21,-63) {};
\node[box=0-for-negatives] (p-63-22) at (-63/2+22,-63) {};
\node[box=0-for-negatives] (p-63-23) at (-63/2+23,-63) {};
\node[box=0-for-negatives] (p-63-24) at (-63/2+24,-63) {};
\node[box=0-for-negatives] (p-63-25) at (-63/2+25,-63) {};
\node[box=0-for-negatives] (p-63-26) at (-63/2+26,-63) {};
\node[box=2-for-negatives] (p-63-27) at (-63/2+27,-63) {};
\node[box=0-for-negatives] (p-63-28) at (-63/2+28,-63) {};
\node[box=0-for-negatives] (p-63-29) at (-63/2+29,-63) {};
\node[box=0-for-negatives] (p-63-30) at (-63/2+30,-63) {};
\node[box=0-for-negatives] (p-63-31) at (-63/2+31,-63) {};
\node[box=0-for-negatives] (p-63-32) at (-63/2+32,-63) {};
\node[box=0-for-negatives] (p-63-33) at (-63/2+33,-63) {};
\node[box=0-for-negatives] (p-63-34) at (-63/2+34,-63) {};
\node[box=0-for-negatives] (p-63-35) at (-63/2+35,-63) {};
\node[box=1-for-negatives] (p-63-36) at (-63/2+36,-63) {};
\node[box=0-for-negatives] (p-63-37) at (-63/2+37,-63) {};
\node[box=0-for-negatives] (p-63-38) at (-63/2+38,-63) {};
\node[box=0-for-negatives] (p-63-39) at (-63/2+39,-63) {};
\node[box=0-for-negatives] (p-63-40) at (-63/2+40,-63) {};
\node[box=0-for-negatives] (p-63-41) at (-63/2+41,-63) {};
\node[box=0-for-negatives] (p-63-42) at (-63/2+42,-63) {};
\node[box=0-for-negatives] (p-63-43) at (-63/2+43,-63) {};
\node[box=0-for-negatives] (p-63-44) at (-63/2+44,-63) {};
\node[box=0-for-negatives] (p-63-45) at (-63/2+45,-63) {};
\node[box=0-for-negatives] (p-63-46) at (-63/2+46,-63) {};
\node[box=0-for-negatives] (p-63-47) at (-63/2+47,-63) {};
\node[box=0-for-negatives] (p-63-48) at (-63/2+48,-63) {};
\node[box=0-for-negatives] (p-63-49) at (-63/2+49,-63) {};
\node[box=0-for-negatives] (p-63-50) at (-63/2+50,-63) {};
\node[box=0-for-negatives] (p-63-51) at (-63/2+51,-63) {};
\node[box=0-for-negatives] (p-63-52) at (-63/2+52,-63) {};
\node[box=0-for-negatives] (p-63-53) at (-63/2+53,-63) {};
\node[box=2-for-negatives] (p-63-54) at (-63/2+54,-63) {};
\node[box=0-for-negatives] (p-63-55) at (-63/2+55,-63) {};
\node[box=0-for-negatives] (p-63-56) at (-63/2+56,-63) {};
\node[box=0-for-negatives] (p-63-57) at (-63/2+57,-63) {};
\node[box=0-for-negatives] (p-63-58) at (-63/2+58,-63) {};
\node[box=0-for-negatives] (p-63-59) at (-63/2+59,-63) {};
\node[box=0-for-negatives] (p-63-60) at (-63/2+60,-63) {};
\node[box=0-for-negatives] (p-63-61) at (-63/2+61,-63) {};
\node[box=0-for-negatives] (p-63-62) at (-63/2+62,-63) {};
\node[box=1-for-negatives] (p-63-63) at (-63/2+63,-63) {};
\node[box=1] (p-64-0) at (-64/2+0,-64) {};
\node[box=2-for-negatives] (p-64-1) at (-64/2+1,-64) {};
\node[box=0-for-negatives] (p-64-2) at (-64/2+2,-64) {};
\node[box=0-for-negatives] (p-64-3) at (-64/2+3,-64) {};
\node[box=0-for-negatives] (p-64-4) at (-64/2+4,-64) {};
\node[box=0-for-negatives] (p-64-5) at (-64/2+5,-64) {};
\node[box=0-for-negatives] (p-64-6) at (-64/2+6,-64) {};
\node[box=0-for-negatives] (p-64-7) at (-64/2+7,-64) {};
\node[box=0-for-negatives] (p-64-8) at (-64/2+8,-64) {};
\node[box=2-for-negatives] (p-64-9) at (-64/2+9,-64) {};
\node[box=1-for-negatives] (p-64-10) at (-64/2+10,-64) {};
\node[box=0-for-negatives] (p-64-11) at (-64/2+11,-64) {};
\node[box=0-for-negatives] (p-64-12) at (-64/2+12,-64) {};
\node[box=0-for-negatives] (p-64-13) at (-64/2+13,-64) {};
\node[box=0-for-negatives] (p-64-14) at (-64/2+14,-64) {};
\node[box=0-for-negatives] (p-64-15) at (-64/2+15,-64) {};
\node[box=0-for-negatives] (p-64-16) at (-64/2+16,-64) {};
\node[box=0-for-negatives] (p-64-17) at (-64/2+17,-64) {};
\node[box=0-for-negatives] (p-64-18) at (-64/2+18,-64) {};
\node[box=0-for-negatives] (p-64-19) at (-64/2+19,-64) {};
\node[box=0-for-negatives] (p-64-20) at (-64/2+20,-64) {};
\node[box=0-for-negatives] (p-64-21) at (-64/2+21,-64) {};
\node[box=0-for-negatives] (p-64-22) at (-64/2+22,-64) {};
\node[box=0-for-negatives] (p-64-23) at (-64/2+23,-64) {};
\node[box=0-for-negatives] (p-64-24) at (-64/2+24,-64) {};
\node[box=0-for-negatives] (p-64-25) at (-64/2+25,-64) {};
\node[box=0-for-negatives] (p-64-26) at (-64/2+26,-64) {};
\node[box=1-for-negatives] (p-64-27) at (-64/2+27,-64) {};
\node[box=2-for-negatives] (p-64-28) at (-64/2+28,-64) {};
\node[box=0-for-negatives] (p-64-29) at (-64/2+29,-64) {};
\node[box=0-for-negatives] (p-64-30) at (-64/2+30,-64) {};
\node[box=0-for-negatives] (p-64-31) at (-64/2+31,-64) {};
\node[box=0-for-negatives] (p-64-32) at (-64/2+32,-64) {};
\node[box=0-for-negatives] (p-64-33) at (-64/2+33,-64) {};
\node[box=0-for-negatives] (p-64-34) at (-64/2+34,-64) {};
\node[box=0-for-negatives] (p-64-35) at (-64/2+35,-64) {};
\node[box=2-for-negatives] (p-64-36) at (-64/2+36,-64) {};
\node[box=1-for-negatives] (p-64-37) at (-64/2+37,-64) {};
\node[box=0-for-negatives] (p-64-38) at (-64/2+38,-64) {};
\node[box=0-for-negatives] (p-64-39) at (-64/2+39,-64) {};
\node[box=0-for-negatives] (p-64-40) at (-64/2+40,-64) {};
\node[box=0-for-negatives] (p-64-41) at (-64/2+41,-64) {};
\node[box=0-for-negatives] (p-64-42) at (-64/2+42,-64) {};
\node[box=0-for-negatives] (p-64-43) at (-64/2+43,-64) {};
\node[box=0-for-negatives] (p-64-44) at (-64/2+44,-64) {};
\node[box=0-for-negatives] (p-64-45) at (-64/2+45,-64) {};
\node[box=0-for-negatives] (p-64-46) at (-64/2+46,-64) {};
\node[box=0-for-negatives] (p-64-47) at (-64/2+47,-64) {};
\node[box=0-for-negatives] (p-64-48) at (-64/2+48,-64) {};
\node[box=0-for-negatives] (p-64-49) at (-64/2+49,-64) {};
\node[box=0-for-negatives] (p-64-50) at (-64/2+50,-64) {};
\node[box=0-for-negatives] (p-64-51) at (-64/2+51,-64) {};
\node[box=0-for-negatives] (p-64-52) at (-64/2+52,-64) {};
\node[box=0-for-negatives] (p-64-53) at (-64/2+53,-64) {};
\node[box=1-for-negatives] (p-64-54) at (-64/2+54,-64) {};
\node[box=2-for-negatives] (p-64-55) at (-64/2+55,-64) {};
\node[box=0-for-negatives] (p-64-56) at (-64/2+56,-64) {};
\node[box=0-for-negatives] (p-64-57) at (-64/2+57,-64) {};
\node[box=0-for-negatives] (p-64-58) at (-64/2+58,-64) {};
\node[box=0-for-negatives] (p-64-59) at (-64/2+59,-64) {};
\node[box=0-for-negatives] (p-64-60) at (-64/2+60,-64) {};
\node[box=0-for-negatives] (p-64-61) at (-64/2+61,-64) {};
\node[box=0-for-negatives] (p-64-62) at (-64/2+62,-64) {};
\node[box=2-for-negatives] (p-64-63) at (-64/2+63,-64) {};
\node[box=1-for-negatives] (p-64-64) at (-64/2+64,-64) {};
\node[box=2] (p-65-0) at (-65/2+0,-65) {};
\node[box=2-for-negatives] (p-65-1) at (-65/2+1,-65) {};
\node[box=2-for-negatives] (p-65-2) at (-65/2+2,-65) {};
\node[box=0-for-negatives] (p-65-3) at (-65/2+3,-65) {};
\node[box=0-for-negatives] (p-65-4) at (-65/2+4,-65) {};
\node[box=0-for-negatives] (p-65-5) at (-65/2+5,-65) {};
\node[box=0-for-negatives] (p-65-6) at (-65/2+6,-65) {};
\node[box=0-for-negatives] (p-65-7) at (-65/2+7,-65) {};
\node[box=0-for-negatives] (p-65-8) at (-65/2+8,-65) {};
\node[box=1-for-negatives] (p-65-9) at (-65/2+9,-65) {};
\node[box=1-for-negatives] (p-65-10) at (-65/2+10,-65) {};
\node[box=1-for-negatives] (p-65-11) at (-65/2+11,-65) {};
\node[box=0-for-negatives] (p-65-12) at (-65/2+12,-65) {};
\node[box=0-for-negatives] (p-65-13) at (-65/2+13,-65) {};
\node[box=0-for-negatives] (p-65-14) at (-65/2+14,-65) {};
\node[box=0-for-negatives] (p-65-15) at (-65/2+15,-65) {};
\node[box=0-for-negatives] (p-65-16) at (-65/2+16,-65) {};
\node[box=0-for-negatives] (p-65-17) at (-65/2+17,-65) {};
\node[box=0-for-negatives] (p-65-18) at (-65/2+18,-65) {};
\node[box=0-for-negatives] (p-65-19) at (-65/2+19,-65) {};
\node[box=0-for-negatives] (p-65-20) at (-65/2+20,-65) {};
\node[box=0-for-negatives] (p-65-21) at (-65/2+21,-65) {};
\node[box=0-for-negatives] (p-65-22) at (-65/2+22,-65) {};
\node[box=0-for-negatives] (p-65-23) at (-65/2+23,-65) {};
\node[box=0-for-negatives] (p-65-24) at (-65/2+24,-65) {};
\node[box=0-for-negatives] (p-65-25) at (-65/2+25,-65) {};
\node[box=0-for-negatives] (p-65-26) at (-65/2+26,-65) {};
\node[box=2-for-negatives] (p-65-27) at (-65/2+27,-65) {};
\node[box=2-for-negatives] (p-65-28) at (-65/2+28,-65) {};
\node[box=2-for-negatives] (p-65-29) at (-65/2+29,-65) {};
\node[box=0-for-negatives] (p-65-30) at (-65/2+30,-65) {};
\node[box=0-for-negatives] (p-65-31) at (-65/2+31,-65) {};
\node[box=0-for-negatives] (p-65-32) at (-65/2+32,-65) {};
\node[box=0-for-negatives] (p-65-33) at (-65/2+33,-65) {};
\node[box=0-for-negatives] (p-65-34) at (-65/2+34,-65) {};
\node[box=0-for-negatives] (p-65-35) at (-65/2+35,-65) {};
\node[box=1-for-negatives] (p-65-36) at (-65/2+36,-65) {};
\node[box=1-for-negatives] (p-65-37) at (-65/2+37,-65) {};
\node[box=1-for-negatives] (p-65-38) at (-65/2+38,-65) {};
\node[box=0-for-negatives] (p-65-39) at (-65/2+39,-65) {};
\node[box=0-for-negatives] (p-65-40) at (-65/2+40,-65) {};
\node[box=0-for-negatives] (p-65-41) at (-65/2+41,-65) {};
\node[box=0-for-negatives] (p-65-42) at (-65/2+42,-65) {};
\node[box=0-for-negatives] (p-65-43) at (-65/2+43,-65) {};
\node[box=0-for-negatives] (p-65-44) at (-65/2+44,-65) {};
\node[box=0-for-negatives] (p-65-45) at (-65/2+45,-65) {};
\node[box=0-for-negatives] (p-65-46) at (-65/2+46,-65) {};
\node[box=0-for-negatives] (p-65-47) at (-65/2+47,-65) {};
\node[box=0-for-negatives] (p-65-48) at (-65/2+48,-65) {};
\node[box=0-for-negatives] (p-65-49) at (-65/2+49,-65) {};
\node[box=0-for-negatives] (p-65-50) at (-65/2+50,-65) {};
\node[box=0-for-negatives] (p-65-51) at (-65/2+51,-65) {};
\node[box=0-for-negatives] (p-65-52) at (-65/2+52,-65) {};
\node[box=0-for-negatives] (p-65-53) at (-65/2+53,-65) {};
\node[box=2-for-negatives] (p-65-54) at (-65/2+54,-65) {};
\node[box=2-for-negatives] (p-65-55) at (-65/2+55,-65) {};
\node[box=2-for-negatives] (p-65-56) at (-65/2+56,-65) {};
\node[box=0-for-negatives] (p-65-57) at (-65/2+57,-65) {};
\node[box=0-for-negatives] (p-65-58) at (-65/2+58,-65) {};
\node[box=0-for-negatives] (p-65-59) at (-65/2+59,-65) {};
\node[box=0-for-negatives] (p-65-60) at (-65/2+60,-65) {};
\node[box=0-for-negatives] (p-65-61) at (-65/2+61,-65) {};
\node[box=0-for-negatives] (p-65-62) at (-65/2+62,-65) {};
\node[box=1-for-negatives] (p-65-63) at (-65/2+63,-65) {};
\node[box=1-for-negatives] (p-65-64) at (-65/2+64,-65) {};
\node[box=1-for-negatives] (p-65-65) at (-65/2+65,-65) {};
\node[box=1] (p-66-0) at (-66/2+0,-66) {};
\node[box=0-for-negatives] (p-66-1) at (-66/2+1,-66) {};
\node[box=0-for-negatives] (p-66-2) at (-66/2+2,-66) {};
\node[box=2-for-negatives] (p-66-3) at (-66/2+3,-66) {};
\node[box=0-for-negatives] (p-66-4) at (-66/2+4,-66) {};
\node[box=0-for-negatives] (p-66-5) at (-66/2+5,-66) {};
\node[box=0-for-negatives] (p-66-6) at (-66/2+6,-66) {};
\node[box=0-for-negatives] (p-66-7) at (-66/2+7,-66) {};
\node[box=0-for-negatives] (p-66-8) at (-66/2+8,-66) {};
\node[box=2-for-negatives] (p-66-9) at (-66/2+9,-66) {};
\node[box=0-for-negatives] (p-66-10) at (-66/2+10,-66) {};
\node[box=0-for-negatives] (p-66-11) at (-66/2+11,-66) {};
\node[box=1-for-negatives] (p-66-12) at (-66/2+12,-66) {};
\node[box=0-for-negatives] (p-66-13) at (-66/2+13,-66) {};
\node[box=0-for-negatives] (p-66-14) at (-66/2+14,-66) {};
\node[box=0-for-negatives] (p-66-15) at (-66/2+15,-66) {};
\node[box=0-for-negatives] (p-66-16) at (-66/2+16,-66) {};
\node[box=0-for-negatives] (p-66-17) at (-66/2+17,-66) {};
\node[box=0-for-negatives] (p-66-18) at (-66/2+18,-66) {};
\node[box=0-for-negatives] (p-66-19) at (-66/2+19,-66) {};
\node[box=0-for-negatives] (p-66-20) at (-66/2+20,-66) {};
\node[box=0-for-negatives] (p-66-21) at (-66/2+21,-66) {};
\node[box=0-for-negatives] (p-66-22) at (-66/2+22,-66) {};
\node[box=0-for-negatives] (p-66-23) at (-66/2+23,-66) {};
\node[box=0-for-negatives] (p-66-24) at (-66/2+24,-66) {};
\node[box=0-for-negatives] (p-66-25) at (-66/2+25,-66) {};
\node[box=0-for-negatives] (p-66-26) at (-66/2+26,-66) {};
\node[box=1-for-negatives] (p-66-27) at (-66/2+27,-66) {};
\node[box=0-for-negatives] (p-66-28) at (-66/2+28,-66) {};
\node[box=0-for-negatives] (p-66-29) at (-66/2+29,-66) {};
\node[box=2-for-negatives] (p-66-30) at (-66/2+30,-66) {};
\node[box=0-for-negatives] (p-66-31) at (-66/2+31,-66) {};
\node[box=0-for-negatives] (p-66-32) at (-66/2+32,-66) {};
\node[box=0-for-negatives] (p-66-33) at (-66/2+33,-66) {};
\node[box=0-for-negatives] (p-66-34) at (-66/2+34,-66) {};
\node[box=0-for-negatives] (p-66-35) at (-66/2+35,-66) {};
\node[box=2-for-negatives] (p-66-36) at (-66/2+36,-66) {};
\node[box=0-for-negatives] (p-66-37) at (-66/2+37,-66) {};
\node[box=0-for-negatives] (p-66-38) at (-66/2+38,-66) {};
\node[box=1-for-negatives] (p-66-39) at (-66/2+39,-66) {};
\node[box=0-for-negatives] (p-66-40) at (-66/2+40,-66) {};
\node[box=0-for-negatives] (p-66-41) at (-66/2+41,-66) {};
\node[box=0-for-negatives] (p-66-42) at (-66/2+42,-66) {};
\node[box=0-for-negatives] (p-66-43) at (-66/2+43,-66) {};
\node[box=0-for-negatives] (p-66-44) at (-66/2+44,-66) {};
\node[box=0-for-negatives] (p-66-45) at (-66/2+45,-66) {};
\node[box=0-for-negatives] (p-66-46) at (-66/2+46,-66) {};
\node[box=0-for-negatives] (p-66-47) at (-66/2+47,-66) {};
\node[box=0-for-negatives] (p-66-48) at (-66/2+48,-66) {};
\node[box=0-for-negatives] (p-66-49) at (-66/2+49,-66) {};
\node[box=0-for-negatives] (p-66-50) at (-66/2+50,-66) {};
\node[box=0-for-negatives] (p-66-51) at (-66/2+51,-66) {};
\node[box=0-for-negatives] (p-66-52) at (-66/2+52,-66) {};
\node[box=0-for-negatives] (p-66-53) at (-66/2+53,-66) {};
\node[box=1-for-negatives] (p-66-54) at (-66/2+54,-66) {};
\node[box=0-for-negatives] (p-66-55) at (-66/2+55,-66) {};
\node[box=0-for-negatives] (p-66-56) at (-66/2+56,-66) {};
\node[box=2-for-negatives] (p-66-57) at (-66/2+57,-66) {};
\node[box=0-for-negatives] (p-66-58) at (-66/2+58,-66) {};
\node[box=0-for-negatives] (p-66-59) at (-66/2+59,-66) {};
\node[box=0-for-negatives] (p-66-60) at (-66/2+60,-66) {};
\node[box=0-for-negatives] (p-66-61) at (-66/2+61,-66) {};
\node[box=0-for-negatives] (p-66-62) at (-66/2+62,-66) {};
\node[box=2-for-negatives] (p-66-63) at (-66/2+63,-66) {};
\node[box=0-for-negatives] (p-66-64) at (-66/2+64,-66) {};
\node[box=0-for-negatives] (p-66-65) at (-66/2+65,-66) {};
\node[box=1-for-negatives] (p-66-66) at (-66/2+66,-66) {};
\node[box=2] (p-67-0) at (-67/2+0,-67) {};
\node[box=1-for-negatives] (p-67-1) at (-67/2+1,-67) {};
\node[box=0-for-negatives] (p-67-2) at (-67/2+2,-67) {};
\node[box=1-for-negatives] (p-67-3) at (-67/2+3,-67) {};
\node[box=2-for-negatives] (p-67-4) at (-67/2+4,-67) {};
\node[box=0-for-negatives] (p-67-5) at (-67/2+5,-67) {};
\node[box=0-for-negatives] (p-67-6) at (-67/2+6,-67) {};
\node[box=0-for-negatives] (p-67-7) at (-67/2+7,-67) {};
\node[box=0-for-negatives] (p-67-8) at (-67/2+8,-67) {};
\node[box=1-for-negatives] (p-67-9) at (-67/2+9,-67) {};
\node[box=2-for-negatives] (p-67-10) at (-67/2+10,-67) {};
\node[box=0-for-negatives] (p-67-11) at (-67/2+11,-67) {};
\node[box=2-for-negatives] (p-67-12) at (-67/2+12,-67) {};
\node[box=1-for-negatives] (p-67-13) at (-67/2+13,-67) {};
\node[box=0-for-negatives] (p-67-14) at (-67/2+14,-67) {};
\node[box=0-for-negatives] (p-67-15) at (-67/2+15,-67) {};
\node[box=0-for-negatives] (p-67-16) at (-67/2+16,-67) {};
\node[box=0-for-negatives] (p-67-17) at (-67/2+17,-67) {};
\node[box=0-for-negatives] (p-67-18) at (-67/2+18,-67) {};
\node[box=0-for-negatives] (p-67-19) at (-67/2+19,-67) {};
\node[box=0-for-negatives] (p-67-20) at (-67/2+20,-67) {};
\node[box=0-for-negatives] (p-67-21) at (-67/2+21,-67) {};
\node[box=0-for-negatives] (p-67-22) at (-67/2+22,-67) {};
\node[box=0-for-negatives] (p-67-23) at (-67/2+23,-67) {};
\node[box=0-for-negatives] (p-67-24) at (-67/2+24,-67) {};
\node[box=0-for-negatives] (p-67-25) at (-67/2+25,-67) {};
\node[box=0-for-negatives] (p-67-26) at (-67/2+26,-67) {};
\node[box=2-for-negatives] (p-67-27) at (-67/2+27,-67) {};
\node[box=1-for-negatives] (p-67-28) at (-67/2+28,-67) {};
\node[box=0-for-negatives] (p-67-29) at (-67/2+29,-67) {};
\node[box=1-for-negatives] (p-67-30) at (-67/2+30,-67) {};
\node[box=2-for-negatives] (p-67-31) at (-67/2+31,-67) {};
\node[box=0-for-negatives] (p-67-32) at (-67/2+32,-67) {};
\node[box=0-for-negatives] (p-67-33) at (-67/2+33,-67) {};
\node[box=0-for-negatives] (p-67-34) at (-67/2+34,-67) {};
\node[box=0-for-negatives] (p-67-35) at (-67/2+35,-67) {};
\node[box=1-for-negatives] (p-67-36) at (-67/2+36,-67) {};
\node[box=2-for-negatives] (p-67-37) at (-67/2+37,-67) {};
\node[box=0-for-negatives] (p-67-38) at (-67/2+38,-67) {};
\node[box=2-for-negatives] (p-67-39) at (-67/2+39,-67) {};
\node[box=1-for-negatives] (p-67-40) at (-67/2+40,-67) {};
\node[box=0-for-negatives] (p-67-41) at (-67/2+41,-67) {};
\node[box=0-for-negatives] (p-67-42) at (-67/2+42,-67) {};
\node[box=0-for-negatives] (p-67-43) at (-67/2+43,-67) {};
\node[box=0-for-negatives] (p-67-44) at (-67/2+44,-67) {};
\node[box=0-for-negatives] (p-67-45) at (-67/2+45,-67) {};
\node[box=0-for-negatives] (p-67-46) at (-67/2+46,-67) {};
\node[box=0-for-negatives] (p-67-47) at (-67/2+47,-67) {};
\node[box=0-for-negatives] (p-67-48) at (-67/2+48,-67) {};
\node[box=0-for-negatives] (p-67-49) at (-67/2+49,-67) {};
\node[box=0-for-negatives] (p-67-50) at (-67/2+50,-67) {};
\node[box=0-for-negatives] (p-67-51) at (-67/2+51,-67) {};
\node[box=0-for-negatives] (p-67-52) at (-67/2+52,-67) {};
\node[box=0-for-negatives] (p-67-53) at (-67/2+53,-67) {};
\node[box=2-for-negatives] (p-67-54) at (-67/2+54,-67) {};
\node[box=1-for-negatives] (p-67-55) at (-67/2+55,-67) {};
\node[box=0-for-negatives] (p-67-56) at (-67/2+56,-67) {};
\node[box=1-for-negatives] (p-67-57) at (-67/2+57,-67) {};
\node[box=2-for-negatives] (p-67-58) at (-67/2+58,-67) {};
\node[box=0-for-negatives] (p-67-59) at (-67/2+59,-67) {};
\node[box=0-for-negatives] (p-67-60) at (-67/2+60,-67) {};
\node[box=0-for-negatives] (p-67-61) at (-67/2+61,-67) {};
\node[box=0-for-negatives] (p-67-62) at (-67/2+62,-67) {};
\node[box=1-for-negatives] (p-67-63) at (-67/2+63,-67) {};
\node[box=2-for-negatives] (p-67-64) at (-67/2+64,-67) {};
\node[box=0-for-negatives] (p-67-65) at (-67/2+65,-67) {};
\node[box=2-for-negatives] (p-67-66) at (-67/2+66,-67) {};
\node[box=1-for-negatives] (p-67-67) at (-67/2+67,-67) {};
\node[box=1] (p-68-0) at (-68/2+0,-68) {};
\node[box=1-for-negatives] (p-68-1) at (-68/2+1,-68) {};
\node[box=1-for-negatives] (p-68-2) at (-68/2+2,-68) {};
\node[box=2-for-negatives] (p-68-3) at (-68/2+3,-68) {};
\node[box=2-for-negatives] (p-68-4) at (-68/2+4,-68) {};
\node[box=2-for-negatives] (p-68-5) at (-68/2+5,-68) {};
\node[box=0-for-negatives] (p-68-6) at (-68/2+6,-68) {};
\node[box=0-for-negatives] (p-68-7) at (-68/2+7,-68) {};
\node[box=0-for-negatives] (p-68-8) at (-68/2+8,-68) {};
\node[box=2-for-negatives] (p-68-9) at (-68/2+9,-68) {};
\node[box=2-for-negatives] (p-68-10) at (-68/2+10,-68) {};
\node[box=2-for-negatives] (p-68-11) at (-68/2+11,-68) {};
\node[box=1-for-negatives] (p-68-12) at (-68/2+12,-68) {};
\node[box=1-for-negatives] (p-68-13) at (-68/2+13,-68) {};
\node[box=1-for-negatives] (p-68-14) at (-68/2+14,-68) {};
\node[box=0-for-negatives] (p-68-15) at (-68/2+15,-68) {};
\node[box=0-for-negatives] (p-68-16) at (-68/2+16,-68) {};
\node[box=0-for-negatives] (p-68-17) at (-68/2+17,-68) {};
\node[box=0-for-negatives] (p-68-18) at (-68/2+18,-68) {};
\node[box=0-for-negatives] (p-68-19) at (-68/2+19,-68) {};
\node[box=0-for-negatives] (p-68-20) at (-68/2+20,-68) {};
\node[box=0-for-negatives] (p-68-21) at (-68/2+21,-68) {};
\node[box=0-for-negatives] (p-68-22) at (-68/2+22,-68) {};
\node[box=0-for-negatives] (p-68-23) at (-68/2+23,-68) {};
\node[box=0-for-negatives] (p-68-24) at (-68/2+24,-68) {};
\node[box=0-for-negatives] (p-68-25) at (-68/2+25,-68) {};
\node[box=0-for-negatives] (p-68-26) at (-68/2+26,-68) {};
\node[box=1-for-negatives] (p-68-27) at (-68/2+27,-68) {};
\node[box=1-for-negatives] (p-68-28) at (-68/2+28,-68) {};
\node[box=1-for-negatives] (p-68-29) at (-68/2+29,-68) {};
\node[box=2-for-negatives] (p-68-30) at (-68/2+30,-68) {};
\node[box=2-for-negatives] (p-68-31) at (-68/2+31,-68) {};
\node[box=2-for-negatives] (p-68-32) at (-68/2+32,-68) {};
\node[box=0-for-negatives] (p-68-33) at (-68/2+33,-68) {};
\node[box=0-for-negatives] (p-68-34) at (-68/2+34,-68) {};
\node[box=0-for-negatives] (p-68-35) at (-68/2+35,-68) {};
\node[box=2-for-negatives] (p-68-36) at (-68/2+36,-68) {};
\node[box=2-for-negatives] (p-68-37) at (-68/2+37,-68) {};
\node[box=2-for-negatives] (p-68-38) at (-68/2+38,-68) {};
\node[box=1-for-negatives] (p-68-39) at (-68/2+39,-68) {};
\node[box=1-for-negatives] (p-68-40) at (-68/2+40,-68) {};
\node[box=1-for-negatives] (p-68-41) at (-68/2+41,-68) {};
\node[box=0-for-negatives] (p-68-42) at (-68/2+42,-68) {};
\node[box=0-for-negatives] (p-68-43) at (-68/2+43,-68) {};
\node[box=0-for-negatives] (p-68-44) at (-68/2+44,-68) {};
\node[box=0-for-negatives] (p-68-45) at (-68/2+45,-68) {};
\node[box=0-for-negatives] (p-68-46) at (-68/2+46,-68) {};
\node[box=0-for-negatives] (p-68-47) at (-68/2+47,-68) {};
\node[box=0-for-negatives] (p-68-48) at (-68/2+48,-68) {};
\node[box=0-for-negatives] (p-68-49) at (-68/2+49,-68) {};
\node[box=0-for-negatives] (p-68-50) at (-68/2+50,-68) {};
\node[box=0-for-negatives] (p-68-51) at (-68/2+51,-68) {};
\node[box=0-for-negatives] (p-68-52) at (-68/2+52,-68) {};
\node[box=0-for-negatives] (p-68-53) at (-68/2+53,-68) {};
\node[box=1-for-negatives] (p-68-54) at (-68/2+54,-68) {};
\node[box=1-for-negatives] (p-68-55) at (-68/2+55,-68) {};
\node[box=1-for-negatives] (p-68-56) at (-68/2+56,-68) {};
\node[box=2-for-negatives] (p-68-57) at (-68/2+57,-68) {};
\node[box=2-for-negatives] (p-68-58) at (-68/2+58,-68) {};
\node[box=2-for-negatives] (p-68-59) at (-68/2+59,-68) {};
\node[box=0-for-negatives] (p-68-60) at (-68/2+60,-68) {};
\node[box=0-for-negatives] (p-68-61) at (-68/2+61,-68) {};
\node[box=0-for-negatives] (p-68-62) at (-68/2+62,-68) {};
\node[box=2-for-negatives] (p-68-63) at (-68/2+63,-68) {};
\node[box=2-for-negatives] (p-68-64) at (-68/2+64,-68) {};
\node[box=2-for-negatives] (p-68-65) at (-68/2+65,-68) {};
\node[box=1-for-negatives] (p-68-66) at (-68/2+66,-68) {};
\node[box=1-for-negatives] (p-68-67) at (-68/2+67,-68) {};
\node[box=1-for-negatives] (p-68-68) at (-68/2+68,-68) {};
\node[box=2] (p-69-0) at (-69/2+0,-69) {};
\node[box=0-for-negatives] (p-69-1) at (-69/2+1,-69) {};
\node[box=0-for-negatives] (p-69-2) at (-69/2+2,-69) {};
\node[box=2-for-negatives] (p-69-3) at (-69/2+3,-69) {};
\node[box=0-for-negatives] (p-69-4) at (-69/2+4,-69) {};
\node[box=0-for-negatives] (p-69-5) at (-69/2+5,-69) {};
\node[box=2-for-negatives] (p-69-6) at (-69/2+6,-69) {};
\node[box=0-for-negatives] (p-69-7) at (-69/2+7,-69) {};
\node[box=0-for-negatives] (p-69-8) at (-69/2+8,-69) {};
\node[box=1-for-negatives] (p-69-9) at (-69/2+9,-69) {};
\node[box=0-for-negatives] (p-69-10) at (-69/2+10,-69) {};
\node[box=0-for-negatives] (p-69-11) at (-69/2+11,-69) {};
\node[box=1-for-negatives] (p-69-12) at (-69/2+12,-69) {};
\node[box=0-for-negatives] (p-69-13) at (-69/2+13,-69) {};
\node[box=0-for-negatives] (p-69-14) at (-69/2+14,-69) {};
\node[box=1-for-negatives] (p-69-15) at (-69/2+15,-69) {};
\node[box=0-for-negatives] (p-69-16) at (-69/2+16,-69) {};
\node[box=0-for-negatives] (p-69-17) at (-69/2+17,-69) {};
\node[box=0-for-negatives] (p-69-18) at (-69/2+18,-69) {};
\node[box=0-for-negatives] (p-69-19) at (-69/2+19,-69) {};
\node[box=0-for-negatives] (p-69-20) at (-69/2+20,-69) {};
\node[box=0-for-negatives] (p-69-21) at (-69/2+21,-69) {};
\node[box=0-for-negatives] (p-69-22) at (-69/2+22,-69) {};
\node[box=0-for-negatives] (p-69-23) at (-69/2+23,-69) {};
\node[box=0-for-negatives] (p-69-24) at (-69/2+24,-69) {};
\node[box=0-for-negatives] (p-69-25) at (-69/2+25,-69) {};
\node[box=0-for-negatives] (p-69-26) at (-69/2+26,-69) {};
\node[box=2-for-negatives] (p-69-27) at (-69/2+27,-69) {};
\node[box=0-for-negatives] (p-69-28) at (-69/2+28,-69) {};
\node[box=0-for-negatives] (p-69-29) at (-69/2+29,-69) {};
\node[box=2-for-negatives] (p-69-30) at (-69/2+30,-69) {};
\node[box=0-for-negatives] (p-69-31) at (-69/2+31,-69) {};
\node[box=0-for-negatives] (p-69-32) at (-69/2+32,-69) {};
\node[box=2-for-negatives] (p-69-33) at (-69/2+33,-69) {};
\node[box=0-for-negatives] (p-69-34) at (-69/2+34,-69) {};
\node[box=0-for-negatives] (p-69-35) at (-69/2+35,-69) {};
\node[box=1-for-negatives] (p-69-36) at (-69/2+36,-69) {};
\node[box=0-for-negatives] (p-69-37) at (-69/2+37,-69) {};
\node[box=0-for-negatives] (p-69-38) at (-69/2+38,-69) {};
\node[box=1-for-negatives] (p-69-39) at (-69/2+39,-69) {};
\node[box=0-for-negatives] (p-69-40) at (-69/2+40,-69) {};
\node[box=0-for-negatives] (p-69-41) at (-69/2+41,-69) {};
\node[box=1-for-negatives] (p-69-42) at (-69/2+42,-69) {};
\node[box=0-for-negatives] (p-69-43) at (-69/2+43,-69) {};
\node[box=0-for-negatives] (p-69-44) at (-69/2+44,-69) {};
\node[box=0-for-negatives] (p-69-45) at (-69/2+45,-69) {};
\node[box=0-for-negatives] (p-69-46) at (-69/2+46,-69) {};
\node[box=0-for-negatives] (p-69-47) at (-69/2+47,-69) {};
\node[box=0-for-negatives] (p-69-48) at (-69/2+48,-69) {};
\node[box=0-for-negatives] (p-69-49) at (-69/2+49,-69) {};
\node[box=0-for-negatives] (p-69-50) at (-69/2+50,-69) {};
\node[box=0-for-negatives] (p-69-51) at (-69/2+51,-69) {};
\node[box=0-for-negatives] (p-69-52) at (-69/2+52,-69) {};
\node[box=0-for-negatives] (p-69-53) at (-69/2+53,-69) {};
\node[box=2-for-negatives] (p-69-54) at (-69/2+54,-69) {};
\node[box=0-for-negatives] (p-69-55) at (-69/2+55,-69) {};
\node[box=0-for-negatives] (p-69-56) at (-69/2+56,-69) {};
\node[box=2-for-negatives] (p-69-57) at (-69/2+57,-69) {};
\node[box=0-for-negatives] (p-69-58) at (-69/2+58,-69) {};
\node[box=0-for-negatives] (p-69-59) at (-69/2+59,-69) {};
\node[box=2-for-negatives] (p-69-60) at (-69/2+60,-69) {};
\node[box=0-for-negatives] (p-69-61) at (-69/2+61,-69) {};
\node[box=0-for-negatives] (p-69-62) at (-69/2+62,-69) {};
\node[box=1-for-negatives] (p-69-63) at (-69/2+63,-69) {};
\node[box=0-for-negatives] (p-69-64) at (-69/2+64,-69) {};
\node[box=0-for-negatives] (p-69-65) at (-69/2+65,-69) {};
\node[box=1-for-negatives] (p-69-66) at (-69/2+66,-69) {};
\node[box=0-for-negatives] (p-69-67) at (-69/2+67,-69) {};
\node[box=0-for-negatives] (p-69-68) at (-69/2+68,-69) {};
\node[box=1-for-negatives] (p-69-69) at (-69/2+69,-69) {};
\node[box=1] (p-70-0) at (-70/2+0,-70) {};
\node[box=2-for-negatives] (p-70-1) at (-70/2+1,-70) {};
\node[box=0-for-negatives] (p-70-2) at (-70/2+2,-70) {};
\node[box=1-for-negatives] (p-70-3) at (-70/2+3,-70) {};
\node[box=2-for-negatives] (p-70-4) at (-70/2+4,-70) {};
\node[box=0-for-negatives] (p-70-5) at (-70/2+5,-70) {};
\node[box=1-for-negatives] (p-70-6) at (-70/2+6,-70) {};
\node[box=2-for-negatives] (p-70-7) at (-70/2+7,-70) {};
\node[box=0-for-negatives] (p-70-8) at (-70/2+8,-70) {};
\node[box=2-for-negatives] (p-70-9) at (-70/2+9,-70) {};
\node[box=1-for-negatives] (p-70-10) at (-70/2+10,-70) {};
\node[box=0-for-negatives] (p-70-11) at (-70/2+11,-70) {};
\node[box=2-for-negatives] (p-70-12) at (-70/2+12,-70) {};
\node[box=1-for-negatives] (p-70-13) at (-70/2+13,-70) {};
\node[box=0-for-negatives] (p-70-14) at (-70/2+14,-70) {};
\node[box=2-for-negatives] (p-70-15) at (-70/2+15,-70) {};
\node[box=1-for-negatives] (p-70-16) at (-70/2+16,-70) {};
\node[box=0-for-negatives] (p-70-17) at (-70/2+17,-70) {};
\node[box=0-for-negatives] (p-70-18) at (-70/2+18,-70) {};
\node[box=0-for-negatives] (p-70-19) at (-70/2+19,-70) {};
\node[box=0-for-negatives] (p-70-20) at (-70/2+20,-70) {};
\node[box=0-for-negatives] (p-70-21) at (-70/2+21,-70) {};
\node[box=0-for-negatives] (p-70-22) at (-70/2+22,-70) {};
\node[box=0-for-negatives] (p-70-23) at (-70/2+23,-70) {};
\node[box=0-for-negatives] (p-70-24) at (-70/2+24,-70) {};
\node[box=0-for-negatives] (p-70-25) at (-70/2+25,-70) {};
\node[box=0-for-negatives] (p-70-26) at (-70/2+26,-70) {};
\node[box=1-for-negatives] (p-70-27) at (-70/2+27,-70) {};
\node[box=2-for-negatives] (p-70-28) at (-70/2+28,-70) {};
\node[box=0-for-negatives] (p-70-29) at (-70/2+29,-70) {};
\node[box=1-for-negatives] (p-70-30) at (-70/2+30,-70) {};
\node[box=2-for-negatives] (p-70-31) at (-70/2+31,-70) {};
\node[box=0-for-negatives] (p-70-32) at (-70/2+32,-70) {};
\node[box=1-for-negatives] (p-70-33) at (-70/2+33,-70) {};
\node[box=2-for-negatives] (p-70-34) at (-70/2+34,-70) {};
\node[box=0-for-negatives] (p-70-35) at (-70/2+35,-70) {};
\node[box=2-for-negatives] (p-70-36) at (-70/2+36,-70) {};
\node[box=1-for-negatives] (p-70-37) at (-70/2+37,-70) {};
\node[box=0-for-negatives] (p-70-38) at (-70/2+38,-70) {};
\node[box=2-for-negatives] (p-70-39) at (-70/2+39,-70) {};
\node[box=1-for-negatives] (p-70-40) at (-70/2+40,-70) {};
\node[box=0-for-negatives] (p-70-41) at (-70/2+41,-70) {};
\node[box=2-for-negatives] (p-70-42) at (-70/2+42,-70) {};
\node[box=1-for-negatives] (p-70-43) at (-70/2+43,-70) {};
\node[box=0-for-negatives] (p-70-44) at (-70/2+44,-70) {};
\node[box=0-for-negatives] (p-70-45) at (-70/2+45,-70) {};
\node[box=0-for-negatives] (p-70-46) at (-70/2+46,-70) {};
\node[box=0-for-negatives] (p-70-47) at (-70/2+47,-70) {};
\node[box=0-for-negatives] (p-70-48) at (-70/2+48,-70) {};
\node[box=0-for-negatives] (p-70-49) at (-70/2+49,-70) {};
\node[box=0-for-negatives] (p-70-50) at (-70/2+50,-70) {};
\node[box=0-for-negatives] (p-70-51) at (-70/2+51,-70) {};
\node[box=0-for-negatives] (p-70-52) at (-70/2+52,-70) {};
\node[box=0-for-negatives] (p-70-53) at (-70/2+53,-70) {};
\node[box=1-for-negatives] (p-70-54) at (-70/2+54,-70) {};
\node[box=2-for-negatives] (p-70-55) at (-70/2+55,-70) {};
\node[box=0-for-negatives] (p-70-56) at (-70/2+56,-70) {};
\node[box=1-for-negatives] (p-70-57) at (-70/2+57,-70) {};
\node[box=2-for-negatives] (p-70-58) at (-70/2+58,-70) {};
\node[box=0-for-negatives] (p-70-59) at (-70/2+59,-70) {};
\node[box=1-for-negatives] (p-70-60) at (-70/2+60,-70) {};
\node[box=2-for-negatives] (p-70-61) at (-70/2+61,-70) {};
\node[box=0-for-negatives] (p-70-62) at (-70/2+62,-70) {};
\node[box=2-for-negatives] (p-70-63) at (-70/2+63,-70) {};
\node[box=1-for-negatives] (p-70-64) at (-70/2+64,-70) {};
\node[box=0-for-negatives] (p-70-65) at (-70/2+65,-70) {};
\node[box=2-for-negatives] (p-70-66) at (-70/2+66,-70) {};
\node[box=1-for-negatives] (p-70-67) at (-70/2+67,-70) {};
\node[box=0-for-negatives] (p-70-68) at (-70/2+68,-70) {};
\node[box=2-for-negatives] (p-70-69) at (-70/2+69,-70) {};
\node[box=1-for-negatives] (p-70-70) at (-70/2+70,-70) {};
\node[box=2] (p-71-0) at (-71/2+0,-71) {};
\node[box=2-for-negatives] (p-71-1) at (-71/2+1,-71) {};
\node[box=2-for-negatives] (p-71-2) at (-71/2+2,-71) {};
\node[box=2-for-negatives] (p-71-3) at (-71/2+3,-71) {};
\node[box=2-for-negatives] (p-71-4) at (-71/2+4,-71) {};
\node[box=2-for-negatives] (p-71-5) at (-71/2+5,-71) {};
\node[box=2-for-negatives] (p-71-6) at (-71/2+6,-71) {};
\node[box=2-for-negatives] (p-71-7) at (-71/2+7,-71) {};
\node[box=2-for-negatives] (p-71-8) at (-71/2+8,-71) {};
\node[box=1-for-negatives] (p-71-9) at (-71/2+9,-71) {};
\node[box=1-for-negatives] (p-71-10) at (-71/2+10,-71) {};
\node[box=1-for-negatives] (p-71-11) at (-71/2+11,-71) {};
\node[box=1-for-negatives] (p-71-12) at (-71/2+12,-71) {};
\node[box=1-for-negatives] (p-71-13) at (-71/2+13,-71) {};
\node[box=1-for-negatives] (p-71-14) at (-71/2+14,-71) {};
\node[box=1-for-negatives] (p-71-15) at (-71/2+15,-71) {};
\node[box=1-for-negatives] (p-71-16) at (-71/2+16,-71) {};
\node[box=1-for-negatives] (p-71-17) at (-71/2+17,-71) {};
\node[box=0-for-negatives] (p-71-18) at (-71/2+18,-71) {};
\node[box=0-for-negatives] (p-71-19) at (-71/2+19,-71) {};
\node[box=0-for-negatives] (p-71-20) at (-71/2+20,-71) {};
\node[box=0-for-negatives] (p-71-21) at (-71/2+21,-71) {};
\node[box=0-for-negatives] (p-71-22) at (-71/2+22,-71) {};
\node[box=0-for-negatives] (p-71-23) at (-71/2+23,-71) {};
\node[box=0-for-negatives] (p-71-24) at (-71/2+24,-71) {};
\node[box=0-for-negatives] (p-71-25) at (-71/2+25,-71) {};
\node[box=0-for-negatives] (p-71-26) at (-71/2+26,-71) {};
\node[box=2-for-negatives] (p-71-27) at (-71/2+27,-71) {};
\node[box=2-for-negatives] (p-71-28) at (-71/2+28,-71) {};
\node[box=2-for-negatives] (p-71-29) at (-71/2+29,-71) {};
\node[box=2-for-negatives] (p-71-30) at (-71/2+30,-71) {};
\node[box=2-for-negatives] (p-71-31) at (-71/2+31,-71) {};
\node[box=2-for-negatives] (p-71-32) at (-71/2+32,-71) {};
\node[box=2-for-negatives] (p-71-33) at (-71/2+33,-71) {};
\node[box=2-for-negatives] (p-71-34) at (-71/2+34,-71) {};
\node[box=2-for-negatives] (p-71-35) at (-71/2+35,-71) {};
\node[box=1-for-negatives] (p-71-36) at (-71/2+36,-71) {};
\node[box=1-for-negatives] (p-71-37) at (-71/2+37,-71) {};
\node[box=1-for-negatives] (p-71-38) at (-71/2+38,-71) {};
\node[box=1-for-negatives] (p-71-39) at (-71/2+39,-71) {};
\node[box=1-for-negatives] (p-71-40) at (-71/2+40,-71) {};
\node[box=1-for-negatives] (p-71-41) at (-71/2+41,-71) {};
\node[box=1-for-negatives] (p-71-42) at (-71/2+42,-71) {};
\node[box=1-for-negatives] (p-71-43) at (-71/2+43,-71) {};
\node[box=1-for-negatives] (p-71-44) at (-71/2+44,-71) {};
\node[box=0-for-negatives] (p-71-45) at (-71/2+45,-71) {};
\node[box=0-for-negatives] (p-71-46) at (-71/2+46,-71) {};
\node[box=0-for-negatives] (p-71-47) at (-71/2+47,-71) {};
\node[box=0-for-negatives] (p-71-48) at (-71/2+48,-71) {};
\node[box=0-for-negatives] (p-71-49) at (-71/2+49,-71) {};
\node[box=0-for-negatives] (p-71-50) at (-71/2+50,-71) {};
\node[box=0-for-negatives] (p-71-51) at (-71/2+51,-71) {};
\node[box=0-for-negatives] (p-71-52) at (-71/2+52,-71) {};
\node[box=0-for-negatives] (p-71-53) at (-71/2+53,-71) {};
\node[box=2-for-negatives] (p-71-54) at (-71/2+54,-71) {};
\node[box=2-for-negatives] (p-71-55) at (-71/2+55,-71) {};
\node[box=2-for-negatives] (p-71-56) at (-71/2+56,-71) {};
\node[box=2-for-negatives] (p-71-57) at (-71/2+57,-71) {};
\node[box=2-for-negatives] (p-71-58) at (-71/2+58,-71) {};
\node[box=2-for-negatives] (p-71-59) at (-71/2+59,-71) {};
\node[box=2-for-negatives] (p-71-60) at (-71/2+60,-71) {};
\node[box=2-for-negatives] (p-71-61) at (-71/2+61,-71) {};
\node[box=2-for-negatives] (p-71-62) at (-71/2+62,-71) {};
\node[box=1-for-negatives] (p-71-63) at (-71/2+63,-71) {};
\node[box=1-for-negatives] (p-71-64) at (-71/2+64,-71) {};
\node[box=1-for-negatives] (p-71-65) at (-71/2+65,-71) {};
\node[box=1-for-negatives] (p-71-66) at (-71/2+66,-71) {};
\node[box=1-for-negatives] (p-71-67) at (-71/2+67,-71) {};
\node[box=1-for-negatives] (p-71-68) at (-71/2+68,-71) {};
\node[box=1-for-negatives] (p-71-69) at (-71/2+69,-71) {};
\node[box=1-for-negatives] (p-71-70) at (-71/2+70,-71) {};
\node[box=1-for-negatives] (p-71-71) at (-71/2+71,-71) {};
\node[box=1] (p-72-0) at (-72/2+0,-72) {};
\node[box=0-for-negatives] (p-72-1) at (-72/2+1,-72) {};
\node[box=0-for-negatives] (p-72-2) at (-72/2+2,-72) {};
\node[box=0-for-negatives] (p-72-3) at (-72/2+3,-72) {};
\node[box=0-for-negatives] (p-72-4) at (-72/2+4,-72) {};
\node[box=0-for-negatives] (p-72-5) at (-72/2+5,-72) {};
\node[box=0-for-negatives] (p-72-6) at (-72/2+6,-72) {};
\node[box=0-for-negatives] (p-72-7) at (-72/2+7,-72) {};
\node[box=0-for-negatives] (p-72-8) at (-72/2+8,-72) {};
\node[box=1-for-negatives] (p-72-9) at (-72/2+9,-72) {};
\node[box=0-for-negatives] (p-72-10) at (-72/2+10,-72) {};
\node[box=0-for-negatives] (p-72-11) at (-72/2+11,-72) {};
\node[box=0-for-negatives] (p-72-12) at (-72/2+12,-72) {};
\node[box=0-for-negatives] (p-72-13) at (-72/2+13,-72) {};
\node[box=0-for-negatives] (p-72-14) at (-72/2+14,-72) {};
\node[box=0-for-negatives] (p-72-15) at (-72/2+15,-72) {};
\node[box=0-for-negatives] (p-72-16) at (-72/2+16,-72) {};
\node[box=0-for-negatives] (p-72-17) at (-72/2+17,-72) {};
\node[box=1-for-negatives] (p-72-18) at (-72/2+18,-72) {};
\node[box=0-for-negatives] (p-72-19) at (-72/2+19,-72) {};
\node[box=0-for-negatives] (p-72-20) at (-72/2+20,-72) {};
\node[box=0-for-negatives] (p-72-21) at (-72/2+21,-72) {};
\node[box=0-for-negatives] (p-72-22) at (-72/2+22,-72) {};
\node[box=0-for-negatives] (p-72-23) at (-72/2+23,-72) {};
\node[box=0-for-negatives] (p-72-24) at (-72/2+24,-72) {};
\node[box=0-for-negatives] (p-72-25) at (-72/2+25,-72) {};
\node[box=0-for-negatives] (p-72-26) at (-72/2+26,-72) {};
\node[box=1-for-negatives] (p-72-27) at (-72/2+27,-72) {};
\node[box=0-for-negatives] (p-72-28) at (-72/2+28,-72) {};
\node[box=0-for-negatives] (p-72-29) at (-72/2+29,-72) {};
\node[box=0-for-negatives] (p-72-30) at (-72/2+30,-72) {};
\node[box=0-for-negatives] (p-72-31) at (-72/2+31,-72) {};
\node[box=0-for-negatives] (p-72-32) at (-72/2+32,-72) {};
\node[box=0-for-negatives] (p-72-33) at (-72/2+33,-72) {};
\node[box=0-for-negatives] (p-72-34) at (-72/2+34,-72) {};
\node[box=0-for-negatives] (p-72-35) at (-72/2+35,-72) {};
\node[box=1-for-negatives] (p-72-36) at (-72/2+36,-72) {};
\node[box=0-for-negatives] (p-72-37) at (-72/2+37,-72) {};
\node[box=0-for-negatives] (p-72-38) at (-72/2+38,-72) {};
\node[box=0-for-negatives] (p-72-39) at (-72/2+39,-72) {};
\node[box=0-for-negatives] (p-72-40) at (-72/2+40,-72) {};
\node[box=0-for-negatives] (p-72-41) at (-72/2+41,-72) {};
\node[box=0-for-negatives] (p-72-42) at (-72/2+42,-72) {};
\node[box=0-for-negatives] (p-72-43) at (-72/2+43,-72) {};
\node[box=0-for-negatives] (p-72-44) at (-72/2+44,-72) {};
\node[box=1-for-negatives] (p-72-45) at (-72/2+45,-72) {};
\node[box=0-for-negatives] (p-72-46) at (-72/2+46,-72) {};
\node[box=0-for-negatives] (p-72-47) at (-72/2+47,-72) {};
\node[box=0-for-negatives] (p-72-48) at (-72/2+48,-72) {};
\node[box=0-for-negatives] (p-72-49) at (-72/2+49,-72) {};
\node[box=0-for-negatives] (p-72-50) at (-72/2+50,-72) {};
\node[box=0-for-negatives] (p-72-51) at (-72/2+51,-72) {};
\node[box=0-for-negatives] (p-72-52) at (-72/2+52,-72) {};
\node[box=0-for-negatives] (p-72-53) at (-72/2+53,-72) {};
\node[box=1-for-negatives] (p-72-54) at (-72/2+54,-72) {};
\node[box=0-for-negatives] (p-72-55) at (-72/2+55,-72) {};
\node[box=0-for-negatives] (p-72-56) at (-72/2+56,-72) {};
\node[box=0-for-negatives] (p-72-57) at (-72/2+57,-72) {};
\node[box=0-for-negatives] (p-72-58) at (-72/2+58,-72) {};
\node[box=0-for-negatives] (p-72-59) at (-72/2+59,-72) {};
\node[box=0-for-negatives] (p-72-60) at (-72/2+60,-72) {};
\node[box=0-for-negatives] (p-72-61) at (-72/2+61,-72) {};
\node[box=0-for-negatives] (p-72-62) at (-72/2+62,-72) {};
\node[box=1-for-negatives] (p-72-63) at (-72/2+63,-72) {};
\node[box=0-for-negatives] (p-72-64) at (-72/2+64,-72) {};
\node[box=0-for-negatives] (p-72-65) at (-72/2+65,-72) {};
\node[box=0-for-negatives] (p-72-66) at (-72/2+66,-72) {};
\node[box=0-for-negatives] (p-72-67) at (-72/2+67,-72) {};
\node[box=0-for-negatives] (p-72-68) at (-72/2+68,-72) {};
\node[box=0-for-negatives] (p-72-69) at (-72/2+69,-72) {};
\node[box=0-for-negatives] (p-72-70) at (-72/2+70,-72) {};
\node[box=0-for-negatives] (p-72-71) at (-72/2+71,-72) {};
\node[box=1-for-negatives] (p-72-72) at (-72/2+72,-72) {};
\node[box=2] (p-73-0) at (-73/2+0,-73) {};
\node[box=1-for-negatives] (p-73-1) at (-73/2+1,-73) {};
\node[box=0-for-negatives] (p-73-2) at (-73/2+2,-73) {};
\node[box=0-for-negatives] (p-73-3) at (-73/2+3,-73) {};
\node[box=0-for-negatives] (p-73-4) at (-73/2+4,-73) {};
\node[box=0-for-negatives] (p-73-5) at (-73/2+5,-73) {};
\node[box=0-for-negatives] (p-73-6) at (-73/2+6,-73) {};
\node[box=0-for-negatives] (p-73-7) at (-73/2+7,-73) {};
\node[box=0-for-negatives] (p-73-8) at (-73/2+8,-73) {};
\node[box=2-for-negatives] (p-73-9) at (-73/2+9,-73) {};
\node[box=1-for-negatives] (p-73-10) at (-73/2+10,-73) {};
\node[box=0-for-negatives] (p-73-11) at (-73/2+11,-73) {};
\node[box=0-for-negatives] (p-73-12) at (-73/2+12,-73) {};
\node[box=0-for-negatives] (p-73-13) at (-73/2+13,-73) {};
\node[box=0-for-negatives] (p-73-14) at (-73/2+14,-73) {};
\node[box=0-for-negatives] (p-73-15) at (-73/2+15,-73) {};
\node[box=0-for-negatives] (p-73-16) at (-73/2+16,-73) {};
\node[box=0-for-negatives] (p-73-17) at (-73/2+17,-73) {};
\node[box=2-for-negatives] (p-73-18) at (-73/2+18,-73) {};
\node[box=1-for-negatives] (p-73-19) at (-73/2+19,-73) {};
\node[box=0-for-negatives] (p-73-20) at (-73/2+20,-73) {};
\node[box=0-for-negatives] (p-73-21) at (-73/2+21,-73) {};
\node[box=0-for-negatives] (p-73-22) at (-73/2+22,-73) {};
\node[box=0-for-negatives] (p-73-23) at (-73/2+23,-73) {};
\node[box=0-for-negatives] (p-73-24) at (-73/2+24,-73) {};
\node[box=0-for-negatives] (p-73-25) at (-73/2+25,-73) {};
\node[box=0-for-negatives] (p-73-26) at (-73/2+26,-73) {};
\node[box=2-for-negatives] (p-73-27) at (-73/2+27,-73) {};
\node[box=1-for-negatives] (p-73-28) at (-73/2+28,-73) {};
\node[box=0-for-negatives] (p-73-29) at (-73/2+29,-73) {};
\node[box=0-for-negatives] (p-73-30) at (-73/2+30,-73) {};
\node[box=0-for-negatives] (p-73-31) at (-73/2+31,-73) {};
\node[box=0-for-negatives] (p-73-32) at (-73/2+32,-73) {};
\node[box=0-for-negatives] (p-73-33) at (-73/2+33,-73) {};
\node[box=0-for-negatives] (p-73-34) at (-73/2+34,-73) {};
\node[box=0-for-negatives] (p-73-35) at (-73/2+35,-73) {};
\node[box=2-for-negatives] (p-73-36) at (-73/2+36,-73) {};
\node[box=1-for-negatives] (p-73-37) at (-73/2+37,-73) {};
\node[box=0-for-negatives] (p-73-38) at (-73/2+38,-73) {};
\node[box=0-for-negatives] (p-73-39) at (-73/2+39,-73) {};
\node[box=0-for-negatives] (p-73-40) at (-73/2+40,-73) {};
\node[box=0-for-negatives] (p-73-41) at (-73/2+41,-73) {};
\node[box=0-for-negatives] (p-73-42) at (-73/2+42,-73) {};
\node[box=0-for-negatives] (p-73-43) at (-73/2+43,-73) {};
\node[box=0-for-negatives] (p-73-44) at (-73/2+44,-73) {};
\node[box=2-for-negatives] (p-73-45) at (-73/2+45,-73) {};
\node[box=1-for-negatives] (p-73-46) at (-73/2+46,-73) {};
\node[box=0-for-negatives] (p-73-47) at (-73/2+47,-73) {};
\node[box=0-for-negatives] (p-73-48) at (-73/2+48,-73) {};
\node[box=0-for-negatives] (p-73-49) at (-73/2+49,-73) {};
\node[box=0-for-negatives] (p-73-50) at (-73/2+50,-73) {};
\node[box=0-for-negatives] (p-73-51) at (-73/2+51,-73) {};
\node[box=0-for-negatives] (p-73-52) at (-73/2+52,-73) {};
\node[box=0-for-negatives] (p-73-53) at (-73/2+53,-73) {};
\node[box=2-for-negatives] (p-73-54) at (-73/2+54,-73) {};
\node[box=1-for-negatives] (p-73-55) at (-73/2+55,-73) {};
\node[box=0-for-negatives] (p-73-56) at (-73/2+56,-73) {};
\node[box=0-for-negatives] (p-73-57) at (-73/2+57,-73) {};
\node[box=0-for-negatives] (p-73-58) at (-73/2+58,-73) {};
\node[box=0-for-negatives] (p-73-59) at (-73/2+59,-73) {};
\node[box=0-for-negatives] (p-73-60) at (-73/2+60,-73) {};
\node[box=0-for-negatives] (p-73-61) at (-73/2+61,-73) {};
\node[box=0-for-negatives] (p-73-62) at (-73/2+62,-73) {};
\node[box=2-for-negatives] (p-73-63) at (-73/2+63,-73) {};
\node[box=1-for-negatives] (p-73-64) at (-73/2+64,-73) {};
\node[box=0-for-negatives] (p-73-65) at (-73/2+65,-73) {};
\node[box=0-for-negatives] (p-73-66) at (-73/2+66,-73) {};
\node[box=0-for-negatives] (p-73-67) at (-73/2+67,-73) {};
\node[box=0-for-negatives] (p-73-68) at (-73/2+68,-73) {};
\node[box=0-for-negatives] (p-73-69) at (-73/2+69,-73) {};
\node[box=0-for-negatives] (p-73-70) at (-73/2+70,-73) {};
\node[box=0-for-negatives] (p-73-71) at (-73/2+71,-73) {};
\node[box=2-for-negatives] (p-73-72) at (-73/2+72,-73) {};
\node[box=1-for-negatives] (p-73-73) at (-73/2+73,-73) {};
\node[box=1] (p-74-0) at (-74/2+0,-74) {};
\node[box=1-for-negatives] (p-74-1) at (-74/2+1,-74) {};
\node[box=1-for-negatives] (p-74-2) at (-74/2+2,-74) {};
\node[box=0-for-negatives] (p-74-3) at (-74/2+3,-74) {};
\node[box=0-for-negatives] (p-74-4) at (-74/2+4,-74) {};
\node[box=0-for-negatives] (p-74-5) at (-74/2+5,-74) {};
\node[box=0-for-negatives] (p-74-6) at (-74/2+6,-74) {};
\node[box=0-for-negatives] (p-74-7) at (-74/2+7,-74) {};
\node[box=0-for-negatives] (p-74-8) at (-74/2+8,-74) {};
\node[box=1-for-negatives] (p-74-9) at (-74/2+9,-74) {};
\node[box=1-for-negatives] (p-74-10) at (-74/2+10,-74) {};
\node[box=1-for-negatives] (p-74-11) at (-74/2+11,-74) {};
\node[box=0-for-negatives] (p-74-12) at (-74/2+12,-74) {};
\node[box=0-for-negatives] (p-74-13) at (-74/2+13,-74) {};
\node[box=0-for-negatives] (p-74-14) at (-74/2+14,-74) {};
\node[box=0-for-negatives] (p-74-15) at (-74/2+15,-74) {};
\node[box=0-for-negatives] (p-74-16) at (-74/2+16,-74) {};
\node[box=0-for-negatives] (p-74-17) at (-74/2+17,-74) {};
\node[box=1-for-negatives] (p-74-18) at (-74/2+18,-74) {};
\node[box=1-for-negatives] (p-74-19) at (-74/2+19,-74) {};
\node[box=1-for-negatives] (p-74-20) at (-74/2+20,-74) {};
\node[box=0-for-negatives] (p-74-21) at (-74/2+21,-74) {};
\node[box=0-for-negatives] (p-74-22) at (-74/2+22,-74) {};
\node[box=0-for-negatives] (p-74-23) at (-74/2+23,-74) {};
\node[box=0-for-negatives] (p-74-24) at (-74/2+24,-74) {};
\node[box=0-for-negatives] (p-74-25) at (-74/2+25,-74) {};
\node[box=0-for-negatives] (p-74-26) at (-74/2+26,-74) {};
\node[box=1-for-negatives] (p-74-27) at (-74/2+27,-74) {};
\node[box=1-for-negatives] (p-74-28) at (-74/2+28,-74) {};
\node[box=1-for-negatives] (p-74-29) at (-74/2+29,-74) {};
\node[box=0-for-negatives] (p-74-30) at (-74/2+30,-74) {};
\node[box=0-for-negatives] (p-74-31) at (-74/2+31,-74) {};
\node[box=0-for-negatives] (p-74-32) at (-74/2+32,-74) {};
\node[box=0-for-negatives] (p-74-33) at (-74/2+33,-74) {};
\node[box=0-for-negatives] (p-74-34) at (-74/2+34,-74) {};
\node[box=0-for-negatives] (p-74-35) at (-74/2+35,-74) {};
\node[box=1-for-negatives] (p-74-36) at (-74/2+36,-74) {};
\node[box=1-for-negatives] (p-74-37) at (-74/2+37,-74) {};
\node[box=1-for-negatives] (p-74-38) at (-74/2+38,-74) {};
\node[box=0-for-negatives] (p-74-39) at (-74/2+39,-74) {};
\node[box=0-for-negatives] (p-74-40) at (-74/2+40,-74) {};
\node[box=0-for-negatives] (p-74-41) at (-74/2+41,-74) {};
\node[box=0-for-negatives] (p-74-42) at (-74/2+42,-74) {};
\node[box=0-for-negatives] (p-74-43) at (-74/2+43,-74) {};
\node[box=0-for-negatives] (p-74-44) at (-74/2+44,-74) {};
\node[box=1-for-negatives] (p-74-45) at (-74/2+45,-74) {};
\node[box=1-for-negatives] (p-74-46) at (-74/2+46,-74) {};
\node[box=1-for-negatives] (p-74-47) at (-74/2+47,-74) {};
\node[box=0-for-negatives] (p-74-48) at (-74/2+48,-74) {};
\node[box=0-for-negatives] (p-74-49) at (-74/2+49,-74) {};
\node[box=0-for-negatives] (p-74-50) at (-74/2+50,-74) {};
\node[box=0-for-negatives] (p-74-51) at (-74/2+51,-74) {};
\node[box=0-for-negatives] (p-74-52) at (-74/2+52,-74) {};
\node[box=0-for-negatives] (p-74-53) at (-74/2+53,-74) {};
\node[box=1-for-negatives] (p-74-54) at (-74/2+54,-74) {};
\node[box=1-for-negatives] (p-74-55) at (-74/2+55,-74) {};
\node[box=1-for-negatives] (p-74-56) at (-74/2+56,-74) {};
\node[box=0-for-negatives] (p-74-57) at (-74/2+57,-74) {};
\node[box=0-for-negatives] (p-74-58) at (-74/2+58,-74) {};
\node[box=0-for-negatives] (p-74-59) at (-74/2+59,-74) {};
\node[box=0-for-negatives] (p-74-60) at (-74/2+60,-74) {};
\node[box=0-for-negatives] (p-74-61) at (-74/2+61,-74) {};
\node[box=0-for-negatives] (p-74-62) at (-74/2+62,-74) {};
\node[box=1-for-negatives] (p-74-63) at (-74/2+63,-74) {};
\node[box=1-for-negatives] (p-74-64) at (-74/2+64,-74) {};
\node[box=1-for-negatives] (p-74-65) at (-74/2+65,-74) {};
\node[box=0-for-negatives] (p-74-66) at (-74/2+66,-74) {};
\node[box=0-for-negatives] (p-74-67) at (-74/2+67,-74) {};
\node[box=0-for-negatives] (p-74-68) at (-74/2+68,-74) {};
\node[box=0-for-negatives] (p-74-69) at (-74/2+69,-74) {};
\node[box=0-for-negatives] (p-74-70) at (-74/2+70,-74) {};
\node[box=0-for-negatives] (p-74-71) at (-74/2+71,-74) {};
\node[box=1-for-negatives] (p-74-72) at (-74/2+72,-74) {};
\node[box=1-for-negatives] (p-74-73) at (-74/2+73,-74) {};
\node[box=1-for-negatives] (p-74-74) at (-74/2+74,-74) {};
\node[box=2] (p-75-0) at (-75/2+0,-75) {};
\node[box=0-for-negatives] (p-75-1) at (-75/2+1,-75) {};
\node[box=0-for-negatives] (p-75-2) at (-75/2+2,-75) {};
\node[box=1-for-negatives] (p-75-3) at (-75/2+3,-75) {};
\node[box=0-for-negatives] (p-75-4) at (-75/2+4,-75) {};
\node[box=0-for-negatives] (p-75-5) at (-75/2+5,-75) {};
\node[box=0-for-negatives] (p-75-6) at (-75/2+6,-75) {};
\node[box=0-for-negatives] (p-75-7) at (-75/2+7,-75) {};
\node[box=0-for-negatives] (p-75-8) at (-75/2+8,-75) {};
\node[box=2-for-negatives] (p-75-9) at (-75/2+9,-75) {};
\node[box=0-for-negatives] (p-75-10) at (-75/2+10,-75) {};
\node[box=0-for-negatives] (p-75-11) at (-75/2+11,-75) {};
\node[box=1-for-negatives] (p-75-12) at (-75/2+12,-75) {};
\node[box=0-for-negatives] (p-75-13) at (-75/2+13,-75) {};
\node[box=0-for-negatives] (p-75-14) at (-75/2+14,-75) {};
\node[box=0-for-negatives] (p-75-15) at (-75/2+15,-75) {};
\node[box=0-for-negatives] (p-75-16) at (-75/2+16,-75) {};
\node[box=0-for-negatives] (p-75-17) at (-75/2+17,-75) {};
\node[box=2-for-negatives] (p-75-18) at (-75/2+18,-75) {};
\node[box=0-for-negatives] (p-75-19) at (-75/2+19,-75) {};
\node[box=0-for-negatives] (p-75-20) at (-75/2+20,-75) {};
\node[box=1-for-negatives] (p-75-21) at (-75/2+21,-75) {};
\node[box=0-for-negatives] (p-75-22) at (-75/2+22,-75) {};
\node[box=0-for-negatives] (p-75-23) at (-75/2+23,-75) {};
\node[box=0-for-negatives] (p-75-24) at (-75/2+24,-75) {};
\node[box=0-for-negatives] (p-75-25) at (-75/2+25,-75) {};
\node[box=0-for-negatives] (p-75-26) at (-75/2+26,-75) {};
\node[box=2-for-negatives] (p-75-27) at (-75/2+27,-75) {};
\node[box=0-for-negatives] (p-75-28) at (-75/2+28,-75) {};
\node[box=0-for-negatives] (p-75-29) at (-75/2+29,-75) {};
\node[box=1-for-negatives] (p-75-30) at (-75/2+30,-75) {};
\node[box=0-for-negatives] (p-75-31) at (-75/2+31,-75) {};
\node[box=0-for-negatives] (p-75-32) at (-75/2+32,-75) {};
\node[box=0-for-negatives] (p-75-33) at (-75/2+33,-75) {};
\node[box=0-for-negatives] (p-75-34) at (-75/2+34,-75) {};
\node[box=0-for-negatives] (p-75-35) at (-75/2+35,-75) {};
\node[box=2-for-negatives] (p-75-36) at (-75/2+36,-75) {};
\node[box=0-for-negatives] (p-75-37) at (-75/2+37,-75) {};
\node[box=0-for-negatives] (p-75-38) at (-75/2+38,-75) {};
\node[box=1-for-negatives] (p-75-39) at (-75/2+39,-75) {};
\node[box=0-for-negatives] (p-75-40) at (-75/2+40,-75) {};
\node[box=0-for-negatives] (p-75-41) at (-75/2+41,-75) {};
\node[box=0-for-negatives] (p-75-42) at (-75/2+42,-75) {};
\node[box=0-for-negatives] (p-75-43) at (-75/2+43,-75) {};
\node[box=0-for-negatives] (p-75-44) at (-75/2+44,-75) {};
\node[box=2-for-negatives] (p-75-45) at (-75/2+45,-75) {};
\node[box=0-for-negatives] (p-75-46) at (-75/2+46,-75) {};
\node[box=0-for-negatives] (p-75-47) at (-75/2+47,-75) {};
\node[box=1-for-negatives] (p-75-48) at (-75/2+48,-75) {};
\node[box=0-for-negatives] (p-75-49) at (-75/2+49,-75) {};
\node[box=0-for-negatives] (p-75-50) at (-75/2+50,-75) {};
\node[box=0-for-negatives] (p-75-51) at (-75/2+51,-75) {};
\node[box=0-for-negatives] (p-75-52) at (-75/2+52,-75) {};
\node[box=0-for-negatives] (p-75-53) at (-75/2+53,-75) {};
\node[box=2-for-negatives] (p-75-54) at (-75/2+54,-75) {};
\node[box=0-for-negatives] (p-75-55) at (-75/2+55,-75) {};
\node[box=0-for-negatives] (p-75-56) at (-75/2+56,-75) {};
\node[box=1-for-negatives] (p-75-57) at (-75/2+57,-75) {};
\node[box=0-for-negatives] (p-75-58) at (-75/2+58,-75) {};
\node[box=0-for-negatives] (p-75-59) at (-75/2+59,-75) {};
\node[box=0-for-negatives] (p-75-60) at (-75/2+60,-75) {};
\node[box=0-for-negatives] (p-75-61) at (-75/2+61,-75) {};
\node[box=0-for-negatives] (p-75-62) at (-75/2+62,-75) {};
\node[box=2-for-negatives] (p-75-63) at (-75/2+63,-75) {};
\node[box=0-for-negatives] (p-75-64) at (-75/2+64,-75) {};
\node[box=0-for-negatives] (p-75-65) at (-75/2+65,-75) {};
\node[box=1-for-negatives] (p-75-66) at (-75/2+66,-75) {};
\node[box=0-for-negatives] (p-75-67) at (-75/2+67,-75) {};
\node[box=0-for-negatives] (p-75-68) at (-75/2+68,-75) {};
\node[box=0-for-negatives] (p-75-69) at (-75/2+69,-75) {};
\node[box=0-for-negatives] (p-75-70) at (-75/2+70,-75) {};
\node[box=0-for-negatives] (p-75-71) at (-75/2+71,-75) {};
\node[box=2-for-negatives] (p-75-72) at (-75/2+72,-75) {};
\node[box=0-for-negatives] (p-75-73) at (-75/2+73,-75) {};
\node[box=0-for-negatives] (p-75-74) at (-75/2+74,-75) {};
\node[box=1-for-negatives] (p-75-75) at (-75/2+75,-75) {};
\node[box=1] (p-76-0) at (-76/2+0,-76) {};
\node[box=2-for-negatives] (p-76-1) at (-76/2+1,-76) {};
\node[box=0-for-negatives] (p-76-2) at (-76/2+2,-76) {};
\node[box=2-for-negatives] (p-76-3) at (-76/2+3,-76) {};
\node[box=1-for-negatives] (p-76-4) at (-76/2+4,-76) {};
\node[box=0-for-negatives] (p-76-5) at (-76/2+5,-76) {};
\node[box=0-for-negatives] (p-76-6) at (-76/2+6,-76) {};
\node[box=0-for-negatives] (p-76-7) at (-76/2+7,-76) {};
\node[box=0-for-negatives] (p-76-8) at (-76/2+8,-76) {};
\node[box=1-for-negatives] (p-76-9) at (-76/2+9,-76) {};
\node[box=2-for-negatives] (p-76-10) at (-76/2+10,-76) {};
\node[box=0-for-negatives] (p-76-11) at (-76/2+11,-76) {};
\node[box=2-for-negatives] (p-76-12) at (-76/2+12,-76) {};
\node[box=1-for-negatives] (p-76-13) at (-76/2+13,-76) {};
\node[box=0-for-negatives] (p-76-14) at (-76/2+14,-76) {};
\node[box=0-for-negatives] (p-76-15) at (-76/2+15,-76) {};
\node[box=0-for-negatives] (p-76-16) at (-76/2+16,-76) {};
\node[box=0-for-negatives] (p-76-17) at (-76/2+17,-76) {};
\node[box=1-for-negatives] (p-76-18) at (-76/2+18,-76) {};
\node[box=2-for-negatives] (p-76-19) at (-76/2+19,-76) {};
\node[box=0-for-negatives] (p-76-20) at (-76/2+20,-76) {};
\node[box=2-for-negatives] (p-76-21) at (-76/2+21,-76) {};
\node[box=1-for-negatives] (p-76-22) at (-76/2+22,-76) {};
\node[box=0-for-negatives] (p-76-23) at (-76/2+23,-76) {};
\node[box=0-for-negatives] (p-76-24) at (-76/2+24,-76) {};
\node[box=0-for-negatives] (p-76-25) at (-76/2+25,-76) {};
\node[box=0-for-negatives] (p-76-26) at (-76/2+26,-76) {};
\node[box=1-for-negatives] (p-76-27) at (-76/2+27,-76) {};
\node[box=2-for-negatives] (p-76-28) at (-76/2+28,-76) {};
\node[box=0-for-negatives] (p-76-29) at (-76/2+29,-76) {};
\node[box=2-for-negatives] (p-76-30) at (-76/2+30,-76) {};
\node[box=1-for-negatives] (p-76-31) at (-76/2+31,-76) {};
\node[box=0-for-negatives] (p-76-32) at (-76/2+32,-76) {};
\node[box=0-for-negatives] (p-76-33) at (-76/2+33,-76) {};
\node[box=0-for-negatives] (p-76-34) at (-76/2+34,-76) {};
\node[box=0-for-negatives] (p-76-35) at (-76/2+35,-76) {};
\node[box=1-for-negatives] (p-76-36) at (-76/2+36,-76) {};
\node[box=2-for-negatives] (p-76-37) at (-76/2+37,-76) {};
\node[box=0-for-negatives] (p-76-38) at (-76/2+38,-76) {};
\node[box=2-for-negatives] (p-76-39) at (-76/2+39,-76) {};
\node[box=1-for-negatives] (p-76-40) at (-76/2+40,-76) {};
\node[box=0-for-negatives] (p-76-41) at (-76/2+41,-76) {};
\node[box=0-for-negatives] (p-76-42) at (-76/2+42,-76) {};
\node[box=0-for-negatives] (p-76-43) at (-76/2+43,-76) {};
\node[box=0-for-negatives] (p-76-44) at (-76/2+44,-76) {};
\node[box=1-for-negatives] (p-76-45) at (-76/2+45,-76) {};
\node[box=2-for-negatives] (p-76-46) at (-76/2+46,-76) {};
\node[box=0-for-negatives] (p-76-47) at (-76/2+47,-76) {};
\node[box=2-for-negatives] (p-76-48) at (-76/2+48,-76) {};
\node[box=1-for-negatives] (p-76-49) at (-76/2+49,-76) {};
\node[box=0-for-negatives] (p-76-50) at (-76/2+50,-76) {};
\node[box=0-for-negatives] (p-76-51) at (-76/2+51,-76) {};
\node[box=0-for-negatives] (p-76-52) at (-76/2+52,-76) {};
\node[box=0-for-negatives] (p-76-53) at (-76/2+53,-76) {};
\node[box=1-for-negatives] (p-76-54) at (-76/2+54,-76) {};
\node[box=2-for-negatives] (p-76-55) at (-76/2+55,-76) {};
\node[box=0-for-negatives] (p-76-56) at (-76/2+56,-76) {};
\node[box=2-for-negatives] (p-76-57) at (-76/2+57,-76) {};
\node[box=1-for-negatives] (p-76-58) at (-76/2+58,-76) {};
\node[box=0-for-negatives] (p-76-59) at (-76/2+59,-76) {};
\node[box=0-for-negatives] (p-76-60) at (-76/2+60,-76) {};
\node[box=0-for-negatives] (p-76-61) at (-76/2+61,-76) {};
\node[box=0-for-negatives] (p-76-62) at (-76/2+62,-76) {};
\node[box=1-for-negatives] (p-76-63) at (-76/2+63,-76) {};
\node[box=2-for-negatives] (p-76-64) at (-76/2+64,-76) {};
\node[box=0-for-negatives] (p-76-65) at (-76/2+65,-76) {};
\node[box=2-for-negatives] (p-76-66) at (-76/2+66,-76) {};
\node[box=1-for-negatives] (p-76-67) at (-76/2+67,-76) {};
\node[box=0-for-negatives] (p-76-68) at (-76/2+68,-76) {};
\node[box=0-for-negatives] (p-76-69) at (-76/2+69,-76) {};
\node[box=0-for-negatives] (p-76-70) at (-76/2+70,-76) {};
\node[box=0-for-negatives] (p-76-71) at (-76/2+71,-76) {};
\node[box=1-for-negatives] (p-76-72) at (-76/2+72,-76) {};
\node[box=2-for-negatives] (p-76-73) at (-76/2+73,-76) {};
\node[box=0-for-negatives] (p-76-74) at (-76/2+74,-76) {};
\node[box=2-for-negatives] (p-76-75) at (-76/2+75,-76) {};
\node[box=1-for-negatives] (p-76-76) at (-76/2+76,-76) {};
\node[box=2] (p-77-0) at (-77/2+0,-77) {};
\node[box=2-for-negatives] (p-77-1) at (-77/2+1,-77) {};
\node[box=2-for-negatives] (p-77-2) at (-77/2+2,-77) {};
\node[box=1-for-negatives] (p-77-3) at (-77/2+3,-77) {};
\node[box=1-for-negatives] (p-77-4) at (-77/2+4,-77) {};
\node[box=1-for-negatives] (p-77-5) at (-77/2+5,-77) {};
\node[box=0-for-negatives] (p-77-6) at (-77/2+6,-77) {};
\node[box=0-for-negatives] (p-77-7) at (-77/2+7,-77) {};
\node[box=0-for-negatives] (p-77-8) at (-77/2+8,-77) {};
\node[box=2-for-negatives] (p-77-9) at (-77/2+9,-77) {};
\node[box=2-for-negatives] (p-77-10) at (-77/2+10,-77) {};
\node[box=2-for-negatives] (p-77-11) at (-77/2+11,-77) {};
\node[box=1-for-negatives] (p-77-12) at (-77/2+12,-77) {};
\node[box=1-for-negatives] (p-77-13) at (-77/2+13,-77) {};
\node[box=1-for-negatives] (p-77-14) at (-77/2+14,-77) {};
\node[box=0-for-negatives] (p-77-15) at (-77/2+15,-77) {};
\node[box=0-for-negatives] (p-77-16) at (-77/2+16,-77) {};
\node[box=0-for-negatives] (p-77-17) at (-77/2+17,-77) {};
\node[box=2-for-negatives] (p-77-18) at (-77/2+18,-77) {};
\node[box=2-for-negatives] (p-77-19) at (-77/2+19,-77) {};
\node[box=2-for-negatives] (p-77-20) at (-77/2+20,-77) {};
\node[box=1-for-negatives] (p-77-21) at (-77/2+21,-77) {};
\node[box=1-for-negatives] (p-77-22) at (-77/2+22,-77) {};
\node[box=1-for-negatives] (p-77-23) at (-77/2+23,-77) {};
\node[box=0-for-negatives] (p-77-24) at (-77/2+24,-77) {};
\node[box=0-for-negatives] (p-77-25) at (-77/2+25,-77) {};
\node[box=0-for-negatives] (p-77-26) at (-77/2+26,-77) {};
\node[box=2-for-negatives] (p-77-27) at (-77/2+27,-77) {};
\node[box=2-for-negatives] (p-77-28) at (-77/2+28,-77) {};
\node[box=2-for-negatives] (p-77-29) at (-77/2+29,-77) {};
\node[box=1-for-negatives] (p-77-30) at (-77/2+30,-77) {};
\node[box=1-for-negatives] (p-77-31) at (-77/2+31,-77) {};
\node[box=1-for-negatives] (p-77-32) at (-77/2+32,-77) {};
\node[box=0-for-negatives] (p-77-33) at (-77/2+33,-77) {};
\node[box=0-for-negatives] (p-77-34) at (-77/2+34,-77) {};
\node[box=0-for-negatives] (p-77-35) at (-77/2+35,-77) {};
\node[box=2-for-negatives] (p-77-36) at (-77/2+36,-77) {};
\node[box=2-for-negatives] (p-77-37) at (-77/2+37,-77) {};
\node[box=2-for-negatives] (p-77-38) at (-77/2+38,-77) {};
\node[box=1-for-negatives] (p-77-39) at (-77/2+39,-77) {};
\node[box=1-for-negatives] (p-77-40) at (-77/2+40,-77) {};
\node[box=1-for-negatives] (p-77-41) at (-77/2+41,-77) {};
\node[box=0-for-negatives] (p-77-42) at (-77/2+42,-77) {};
\node[box=0-for-negatives] (p-77-43) at (-77/2+43,-77) {};
\node[box=0-for-negatives] (p-77-44) at (-77/2+44,-77) {};
\node[box=2-for-negatives] (p-77-45) at (-77/2+45,-77) {};
\node[box=2-for-negatives] (p-77-46) at (-77/2+46,-77) {};
\node[box=2-for-negatives] (p-77-47) at (-77/2+47,-77) {};
\node[box=1-for-negatives] (p-77-48) at (-77/2+48,-77) {};
\node[box=1-for-negatives] (p-77-49) at (-77/2+49,-77) {};
\node[box=1-for-negatives] (p-77-50) at (-77/2+50,-77) {};
\node[box=0-for-negatives] (p-77-51) at (-77/2+51,-77) {};
\node[box=0-for-negatives] (p-77-52) at (-77/2+52,-77) {};
\node[box=0-for-negatives] (p-77-53) at (-77/2+53,-77) {};
\node[box=2-for-negatives] (p-77-54) at (-77/2+54,-77) {};
\node[box=2-for-negatives] (p-77-55) at (-77/2+55,-77) {};
\node[box=2-for-negatives] (p-77-56) at (-77/2+56,-77) {};
\node[box=1-for-negatives] (p-77-57) at (-77/2+57,-77) {};
\node[box=1-for-negatives] (p-77-58) at (-77/2+58,-77) {};
\node[box=1-for-negatives] (p-77-59) at (-77/2+59,-77) {};
\node[box=0-for-negatives] (p-77-60) at (-77/2+60,-77) {};
\node[box=0-for-negatives] (p-77-61) at (-77/2+61,-77) {};
\node[box=0-for-negatives] (p-77-62) at (-77/2+62,-77) {};
\node[box=2-for-negatives] (p-77-63) at (-77/2+63,-77) {};
\node[box=2-for-negatives] (p-77-64) at (-77/2+64,-77) {};
\node[box=2-for-negatives] (p-77-65) at (-77/2+65,-77) {};
\node[box=1-for-negatives] (p-77-66) at (-77/2+66,-77) {};
\node[box=1-for-negatives] (p-77-67) at (-77/2+67,-77) {};
\node[box=1-for-negatives] (p-77-68) at (-77/2+68,-77) {};
\node[box=0-for-negatives] (p-77-69) at (-77/2+69,-77) {};
\node[box=0-for-negatives] (p-77-70) at (-77/2+70,-77) {};
\node[box=0-for-negatives] (p-77-71) at (-77/2+71,-77) {};
\node[box=2-for-negatives] (p-77-72) at (-77/2+72,-77) {};
\node[box=2-for-negatives] (p-77-73) at (-77/2+73,-77) {};
\node[box=2-for-negatives] (p-77-74) at (-77/2+74,-77) {};
\node[box=1-for-negatives] (p-77-75) at (-77/2+75,-77) {};
\node[box=1-for-negatives] (p-77-76) at (-77/2+76,-77) {};
\node[box=1-for-negatives] (p-77-77) at (-77/2+77,-77) {};
\node[box=1] (p-78-0) at (-78/2+0,-78) {};
\node[box=0-for-negatives] (p-78-1) at (-78/2+1,-78) {};
\node[box=0-for-negatives] (p-78-2) at (-78/2+2,-78) {};
\node[box=1-for-negatives] (p-78-3) at (-78/2+3,-78) {};
\node[box=0-for-negatives] (p-78-4) at (-78/2+4,-78) {};
\node[box=0-for-negatives] (p-78-5) at (-78/2+5,-78) {};
\node[box=1-for-negatives] (p-78-6) at (-78/2+6,-78) {};
\node[box=0-for-negatives] (p-78-7) at (-78/2+7,-78) {};
\node[box=0-for-negatives] (p-78-8) at (-78/2+8,-78) {};
\node[box=1-for-negatives] (p-78-9) at (-78/2+9,-78) {};
\node[box=0-for-negatives] (p-78-10) at (-78/2+10,-78) {};
\node[box=0-for-negatives] (p-78-11) at (-78/2+11,-78) {};
\node[box=1-for-negatives] (p-78-12) at (-78/2+12,-78) {};
\node[box=0-for-negatives] (p-78-13) at (-78/2+13,-78) {};
\node[box=0-for-negatives] (p-78-14) at (-78/2+14,-78) {};
\node[box=1-for-negatives] (p-78-15) at (-78/2+15,-78) {};
\node[box=0-for-negatives] (p-78-16) at (-78/2+16,-78) {};
\node[box=0-for-negatives] (p-78-17) at (-78/2+17,-78) {};
\node[box=1-for-negatives] (p-78-18) at (-78/2+18,-78) {};
\node[box=0-for-negatives] (p-78-19) at (-78/2+19,-78) {};
\node[box=0-for-negatives] (p-78-20) at (-78/2+20,-78) {};
\node[box=1-for-negatives] (p-78-21) at (-78/2+21,-78) {};
\node[box=0-for-negatives] (p-78-22) at (-78/2+22,-78) {};
\node[box=0-for-negatives] (p-78-23) at (-78/2+23,-78) {};
\node[box=1-for-negatives] (p-78-24) at (-78/2+24,-78) {};
\node[box=0-for-negatives] (p-78-25) at (-78/2+25,-78) {};
\node[box=0-for-negatives] (p-78-26) at (-78/2+26,-78) {};
\node[box=1-for-negatives] (p-78-27) at (-78/2+27,-78) {};
\node[box=0-for-negatives] (p-78-28) at (-78/2+28,-78) {};
\node[box=0-for-negatives] (p-78-29) at (-78/2+29,-78) {};
\node[box=1-for-negatives] (p-78-30) at (-78/2+30,-78) {};
\node[box=0-for-negatives] (p-78-31) at (-78/2+31,-78) {};
\node[box=0-for-negatives] (p-78-32) at (-78/2+32,-78) {};
\node[box=1-for-negatives] (p-78-33) at (-78/2+33,-78) {};
\node[box=0-for-negatives] (p-78-34) at (-78/2+34,-78) {};
\node[box=0-for-negatives] (p-78-35) at (-78/2+35,-78) {};
\node[box=1-for-negatives] (p-78-36) at (-78/2+36,-78) {};
\node[box=0-for-negatives] (p-78-37) at (-78/2+37,-78) {};
\node[box=0-for-negatives] (p-78-38) at (-78/2+38,-78) {};
\node[box=1-for-negatives] (p-78-39) at (-78/2+39,-78) {};
\node[box=0-for-negatives] (p-78-40) at (-78/2+40,-78) {};
\node[box=0-for-negatives] (p-78-41) at (-78/2+41,-78) {};
\node[box=1-for-negatives] (p-78-42) at (-78/2+42,-78) {};
\node[box=0-for-negatives] (p-78-43) at (-78/2+43,-78) {};
\node[box=0-for-negatives] (p-78-44) at (-78/2+44,-78) {};
\node[box=1-for-negatives] (p-78-45) at (-78/2+45,-78) {};
\node[box=0-for-negatives] (p-78-46) at (-78/2+46,-78) {};
\node[box=0-for-negatives] (p-78-47) at (-78/2+47,-78) {};
\node[box=1-for-negatives] (p-78-48) at (-78/2+48,-78) {};
\node[box=0-for-negatives] (p-78-49) at (-78/2+49,-78) {};
\node[box=0-for-negatives] (p-78-50) at (-78/2+50,-78) {};
\node[box=1-for-negatives] (p-78-51) at (-78/2+51,-78) {};
\node[box=0-for-negatives] (p-78-52) at (-78/2+52,-78) {};
\node[box=0-for-negatives] (p-78-53) at (-78/2+53,-78) {};
\node[box=1-for-negatives] (p-78-54) at (-78/2+54,-78) {};
\node[box=0-for-negatives] (p-78-55) at (-78/2+55,-78) {};
\node[box=0-for-negatives] (p-78-56) at (-78/2+56,-78) {};
\node[box=1-for-negatives] (p-78-57) at (-78/2+57,-78) {};
\node[box=0-for-negatives] (p-78-58) at (-78/2+58,-78) {};
\node[box=0-for-negatives] (p-78-59) at (-78/2+59,-78) {};
\node[box=1-for-negatives] (p-78-60) at (-78/2+60,-78) {};
\node[box=0-for-negatives] (p-78-61) at (-78/2+61,-78) {};
\node[box=0-for-negatives] (p-78-62) at (-78/2+62,-78) {};
\node[box=1-for-negatives] (p-78-63) at (-78/2+63,-78) {};
\node[box=0-for-negatives] (p-78-64) at (-78/2+64,-78) {};
\node[box=0-for-negatives] (p-78-65) at (-78/2+65,-78) {};
\node[box=1-for-negatives] (p-78-66) at (-78/2+66,-78) {};
\node[box=0-for-negatives] (p-78-67) at (-78/2+67,-78) {};
\node[box=0-for-negatives] (p-78-68) at (-78/2+68,-78) {};
\node[box=1-for-negatives] (p-78-69) at (-78/2+69,-78) {};
\node[box=0-for-negatives] (p-78-70) at (-78/2+70,-78) {};
\node[box=0-for-negatives] (p-78-71) at (-78/2+71,-78) {};
\node[box=1-for-negatives] (p-78-72) at (-78/2+72,-78) {};
\node[box=0-for-negatives] (p-78-73) at (-78/2+73,-78) {};
\node[box=0-for-negatives] (p-78-74) at (-78/2+74,-78) {};
\node[box=1-for-negatives] (p-78-75) at (-78/2+75,-78) {};
\node[box=0-for-negatives] (p-78-76) at (-78/2+76,-78) {};
\node[box=0-for-negatives] (p-78-77) at (-78/2+77,-78) {};
\node[box=1-for-negatives] (p-78-78) at (-78/2+78,-78) {};
\node[box=2] (p-79-0) at (-79/2+0,-79) {};
\node[box=1-for-negatives] (p-79-1) at (-79/2+1,-79) {};
\node[box=0-for-negatives] (p-79-2) at (-79/2+2,-79) {};
\node[box=2-for-negatives] (p-79-3) at (-79/2+3,-79) {};
\node[box=1-for-negatives] (p-79-4) at (-79/2+4,-79) {};
\node[box=0-for-negatives] (p-79-5) at (-79/2+5,-79) {};
\node[box=2-for-negatives] (p-79-6) at (-79/2+6,-79) {};
\node[box=1-for-negatives] (p-79-7) at (-79/2+7,-79) {};
\node[box=0-for-negatives] (p-79-8) at (-79/2+8,-79) {};
\node[box=2-for-negatives] (p-79-9) at (-79/2+9,-79) {};
\node[box=1-for-negatives] (p-79-10) at (-79/2+10,-79) {};
\node[box=0-for-negatives] (p-79-11) at (-79/2+11,-79) {};
\node[box=2-for-negatives] (p-79-12) at (-79/2+12,-79) {};
\node[box=1-for-negatives] (p-79-13) at (-79/2+13,-79) {};
\node[box=0-for-negatives] (p-79-14) at (-79/2+14,-79) {};
\node[box=2-for-negatives] (p-79-15) at (-79/2+15,-79) {};
\node[box=1-for-negatives] (p-79-16) at (-79/2+16,-79) {};
\node[box=0-for-negatives] (p-79-17) at (-79/2+17,-79) {};
\node[box=2-for-negatives] (p-79-18) at (-79/2+18,-79) {};
\node[box=1-for-negatives] (p-79-19) at (-79/2+19,-79) {};
\node[box=0-for-negatives] (p-79-20) at (-79/2+20,-79) {};
\node[box=2-for-negatives] (p-79-21) at (-79/2+21,-79) {};
\node[box=1-for-negatives] (p-79-22) at (-79/2+22,-79) {};
\node[box=0-for-negatives] (p-79-23) at (-79/2+23,-79) {};
\node[box=2-for-negatives] (p-79-24) at (-79/2+24,-79) {};
\node[box=1-for-negatives] (p-79-25) at (-79/2+25,-79) {};
\node[box=0-for-negatives] (p-79-26) at (-79/2+26,-79) {};
\node[box=2-for-negatives] (p-79-27) at (-79/2+27,-79) {};
\node[box=1-for-negatives] (p-79-28) at (-79/2+28,-79) {};
\node[box=0-for-negatives] (p-79-29) at (-79/2+29,-79) {};
\node[box=2-for-negatives] (p-79-30) at (-79/2+30,-79) {};
\node[box=1-for-negatives] (p-79-31) at (-79/2+31,-79) {};
\node[box=0-for-negatives] (p-79-32) at (-79/2+32,-79) {};
\node[box=2-for-negatives] (p-79-33) at (-79/2+33,-79) {};
\node[box=1-for-negatives] (p-79-34) at (-79/2+34,-79) {};
\node[box=0-for-negatives] (p-79-35) at (-79/2+35,-79) {};
\node[box=2-for-negatives] (p-79-36) at (-79/2+36,-79) {};
\node[box=1-for-negatives] (p-79-37) at (-79/2+37,-79) {};
\node[box=0-for-negatives] (p-79-38) at (-79/2+38,-79) {};
\node[box=2-for-negatives] (p-79-39) at (-79/2+39,-79) {};
\node[box=1-for-negatives] (p-79-40) at (-79/2+40,-79) {};
\node[box=0-for-negatives] (p-79-41) at (-79/2+41,-79) {};
\node[box=2-for-negatives] (p-79-42) at (-79/2+42,-79) {};
\node[box=1-for-negatives] (p-79-43) at (-79/2+43,-79) {};
\node[box=0-for-negatives] (p-79-44) at (-79/2+44,-79) {};
\node[box=2-for-negatives] (p-79-45) at (-79/2+45,-79) {};
\node[box=1-for-negatives] (p-79-46) at (-79/2+46,-79) {};
\node[box=0-for-negatives] (p-79-47) at (-79/2+47,-79) {};
\node[box=2-for-negatives] (p-79-48) at (-79/2+48,-79) {};
\node[box=1-for-negatives] (p-79-49) at (-79/2+49,-79) {};
\node[box=0-for-negatives] (p-79-50) at (-79/2+50,-79) {};
\node[box=2-for-negatives] (p-79-51) at (-79/2+51,-79) {};
\node[box=1-for-negatives] (p-79-52) at (-79/2+52,-79) {};
\node[box=0-for-negatives] (p-79-53) at (-79/2+53,-79) {};
\node[box=2-for-negatives] (p-79-54) at (-79/2+54,-79) {};
\node[box=1-for-negatives] (p-79-55) at (-79/2+55,-79) {};
\node[box=0-for-negatives] (p-79-56) at (-79/2+56,-79) {};
\node[box=2-for-negatives] (p-79-57) at (-79/2+57,-79) {};
\node[box=1-for-negatives] (p-79-58) at (-79/2+58,-79) {};
\node[box=0-for-negatives] (p-79-59) at (-79/2+59,-79) {};
\node[box=2-for-negatives] (p-79-60) at (-79/2+60,-79) {};
\node[box=1-for-negatives] (p-79-61) at (-79/2+61,-79) {};
\node[box=0-for-negatives] (p-79-62) at (-79/2+62,-79) {};
\node[box=2-for-negatives] (p-79-63) at (-79/2+63,-79) {};
\node[box=1-for-negatives] (p-79-64) at (-79/2+64,-79) {};
\node[box=0-for-negatives] (p-79-65) at (-79/2+65,-79) {};
\node[box=2-for-negatives] (p-79-66) at (-79/2+66,-79) {};
\node[box=1-for-negatives] (p-79-67) at (-79/2+67,-79) {};
\node[box=0-for-negatives] (p-79-68) at (-79/2+68,-79) {};
\node[box=2-for-negatives] (p-79-69) at (-79/2+69,-79) {};
\node[box=1-for-negatives] (p-79-70) at (-79/2+70,-79) {};
\node[box=0-for-negatives] (p-79-71) at (-79/2+71,-79) {};
\node[box=2-for-negatives] (p-79-72) at (-79/2+72,-79) {};
\node[box=1-for-negatives] (p-79-73) at (-79/2+73,-79) {};
\node[box=0-for-negatives] (p-79-74) at (-79/2+74,-79) {};
\node[box=2-for-negatives] (p-79-75) at (-79/2+75,-79) {};
\node[box=1-for-negatives] (p-79-76) at (-79/2+76,-79) {};
\node[box=0-for-negatives] (p-79-77) at (-79/2+77,-79) {};
\node[box=2-for-negatives] (p-79-78) at (-79/2+78,-79) {};
\node[box=1-for-negatives] (p-79-79) at (-79/2+79,-79) {};
\node[box=1] (p-80-0) at (-80/2+0,-80) {};
\node[box=1-for-negatives] (p-80-1) at (-80/2+1,-80) {};
\node[box=1-for-negatives] (p-80-2) at (-80/2+2,-80) {};
\node[box=1-for-negatives] (p-80-3) at (-80/2+3,-80) {};
\node[box=1-for-negatives] (p-80-4) at (-80/2+4,-80) {};
\node[box=1-for-negatives] (p-80-5) at (-80/2+5,-80) {};
\node[box=1-for-negatives] (p-80-6) at (-80/2+6,-80) {};
\node[box=1-for-negatives] (p-80-7) at (-80/2+7,-80) {};
\node[box=1-for-negatives] (p-80-8) at (-80/2+8,-80) {};
\node[box=1-for-negatives] (p-80-9) at (-80/2+9,-80) {};
\node[box=1-for-negatives] (p-80-10) at (-80/2+10,-80) {};
\node[box=1-for-negatives] (p-80-11) at (-80/2+11,-80) {};
\node[box=1-for-negatives] (p-80-12) at (-80/2+12,-80) {};
\node[box=1-for-negatives] (p-80-13) at (-80/2+13,-80) {};
\node[box=1-for-negatives] (p-80-14) at (-80/2+14,-80) {};
\node[box=1-for-negatives] (p-80-15) at (-80/2+15,-80) {};
\node[box=1-for-negatives] (p-80-16) at (-80/2+16,-80) {};
\node[box=1-for-negatives] (p-80-17) at (-80/2+17,-80) {};
\node[box=1-for-negatives] (p-80-18) at (-80/2+18,-80) {};
\node[box=1-for-negatives] (p-80-19) at (-80/2+19,-80) {};
\node[box=1-for-negatives] (p-80-20) at (-80/2+20,-80) {};
\node[box=1-for-negatives] (p-80-21) at (-80/2+21,-80) {};
\node[box=1-for-negatives] (p-80-22) at (-80/2+22,-80) {};
\node[box=1-for-negatives] (p-80-23) at (-80/2+23,-80) {};
\node[box=1-for-negatives] (p-80-24) at (-80/2+24,-80) {};
\node[box=1-for-negatives] (p-80-25) at (-80/2+25,-80) {};
\node[box=1-for-negatives] (p-80-26) at (-80/2+26,-80) {};
\node[box=1-for-negatives] (p-80-27) at (-80/2+27,-80) {};
\node[box=1-for-negatives] (p-80-28) at (-80/2+28,-80) {};
\node[box=1-for-negatives] (p-80-29) at (-80/2+29,-80) {};
\node[box=1-for-negatives] (p-80-30) at (-80/2+30,-80) {};
\node[box=1-for-negatives] (p-80-31) at (-80/2+31,-80) {};
\node[box=1-for-negatives] (p-80-32) at (-80/2+32,-80) {};
\node[box=1-for-negatives] (p-80-33) at (-80/2+33,-80) {};
\node[box=1-for-negatives] (p-80-34) at (-80/2+34,-80) {};
\node[box=1-for-negatives] (p-80-35) at (-80/2+35,-80) {};
\node[box=1-for-negatives] (p-80-36) at (-80/2+36,-80) {};
\node[box=1-for-negatives] (p-80-37) at (-80/2+37,-80) {};
\node[box=1-for-negatives] (p-80-38) at (-80/2+38,-80) {};
\node[box=1-for-negatives] (p-80-39) at (-80/2+39,-80) {};
\node[box=1-for-negatives] (p-80-40) at (-80/2+40,-80) {};
\node[box=1-for-negatives] (p-80-41) at (-80/2+41,-80) {};
\node[box=1-for-negatives] (p-80-42) at (-80/2+42,-80) {};
\node[box=1-for-negatives] (p-80-43) at (-80/2+43,-80) {};
\node[box=1-for-negatives] (p-80-44) at (-80/2+44,-80) {};
\node[box=1-for-negatives] (p-80-45) at (-80/2+45,-80) {};
\node[box=1-for-negatives] (p-80-46) at (-80/2+46,-80) {};
\node[box=1-for-negatives] (p-80-47) at (-80/2+47,-80) {};
\node[box=1-for-negatives] (p-80-48) at (-80/2+48,-80) {};
\node[box=1-for-negatives] (p-80-49) at (-80/2+49,-80) {};
\node[box=1-for-negatives] (p-80-50) at (-80/2+50,-80) {};
\node[box=1-for-negatives] (p-80-51) at (-80/2+51,-80) {};
\node[box=1-for-negatives] (p-80-52) at (-80/2+52,-80) {};
\node[box=1-for-negatives] (p-80-53) at (-80/2+53,-80) {};
\node[box=1-for-negatives] (p-80-54) at (-80/2+54,-80) {};
\node[box=1-for-negatives] (p-80-55) at (-80/2+55,-80) {};
\node[box=1-for-negatives] (p-80-56) at (-80/2+56,-80) {};
\node[box=1-for-negatives] (p-80-57) at (-80/2+57,-80) {};
\node[box=1-for-negatives] (p-80-58) at (-80/2+58,-80) {};
\node[box=1-for-negatives] (p-80-59) at (-80/2+59,-80) {};
\node[box=1-for-negatives] (p-80-60) at (-80/2+60,-80) {};
\node[box=1-for-negatives] (p-80-61) at (-80/2+61,-80) {};
\node[box=1-for-negatives] (p-80-62) at (-80/2+62,-80) {};
\node[box=1-for-negatives] (p-80-63) at (-80/2+63,-80) {};
\node[box=1-for-negatives] (p-80-64) at (-80/2+64,-80) {};
\node[box=1-for-negatives] (p-80-65) at (-80/2+65,-80) {};
\node[box=1-for-negatives] (p-80-66) at (-80/2+66,-80) {};
\node[box=1-for-negatives] (p-80-67) at (-80/2+67,-80) {};
\node[box=1-for-negatives] (p-80-68) at (-80/2+68,-80) {};
\node[box=1-for-negatives] (p-80-69) at (-80/2+69,-80) {};
\node[box=1-for-negatives] (p-80-70) at (-80/2+70,-80) {};
\node[box=1-for-negatives] (p-80-71) at (-80/2+71,-80) {};
\node[box=1-for-negatives] (p-80-72) at (-80/2+72,-80) {};
\node[box=1-for-negatives] (p-80-73) at (-80/2+73,-80) {};
\node[box=1-for-negatives] (p-80-74) at (-80/2+74,-80) {};
\node[box=1-for-negatives] (p-80-75) at (-80/2+75,-80) {};
\node[box=1-for-negatives] (p-80-76) at (-80/2+76,-80) {};
\node[box=1-for-negatives] (p-80-77) at (-80/2+77,-80) {};
\node[box=1-for-negatives] (p-80-78) at (-80/2+78,-80) {};
\node[box=1-for-negatives] (p-80-79) at (-80/2+79,-80) {};
\node[box=1-for-negatives] (p-80-80) at (-80/2+80,-80) {};
\node[box=2-for-negatives] (p-81-0) at (-81/2+0,-81) {};
\node[box=0-for-negatives] (p-81-1) at (-81/2+1,-81) {};
\node[box=0-for-negatives] (p-81-2) at (-81/2+2,-81) {};
\node[box=0-for-negatives] (p-81-3) at (-81/2+3,-81) {};
\node[box=0-for-negatives] (p-81-4) at (-81/2+4,-81) {};
\node[box=0-for-negatives] (p-81-5) at (-81/2+5,-81) {};
\node[box=0-for-negatives] (p-81-6) at (-81/2+6,-81) {};
\node[box=0-for-negatives] (p-81-7) at (-81/2+7,-81) {};
\node[box=0-for-negatives] (p-81-8) at (-81/2+8,-81) {};
\node[box=0-for-negatives] (p-81-9) at (-81/2+9,-81) {};
\node[box=0-for-negatives] (p-81-10) at (-81/2+10,-81) {};
\node[box=0-for-negatives] (p-81-11) at (-81/2+11,-81) {};
\node[box=0-for-negatives] (p-81-12) at (-81/2+12,-81) {};
\node[box=0-for-negatives] (p-81-13) at (-81/2+13,-81) {};
\node[box=0-for-negatives] (p-81-14) at (-81/2+14,-81) {};
\node[box=0-for-negatives] (p-81-15) at (-81/2+15,-81) {};
\node[box=0-for-negatives] (p-81-16) at (-81/2+16,-81) {};
\node[box=0-for-negatives] (p-81-17) at (-81/2+17,-81) {};
\node[box=0-for-negatives] (p-81-18) at (-81/2+18,-81) {};
\node[box=0-for-negatives] (p-81-19) at (-81/2+19,-81) {};
\node[box=0-for-negatives] (p-81-20) at (-81/2+20,-81) {};
\node[box=0-for-negatives] (p-81-21) at (-81/2+21,-81) {};
\node[box=0-for-negatives] (p-81-22) at (-81/2+22,-81) {};
\node[box=0-for-negatives] (p-81-23) at (-81/2+23,-81) {};
\node[box=0-for-negatives] (p-81-24) at (-81/2+24,-81) {};
\node[box=0-for-negatives] (p-81-25) at (-81/2+25,-81) {};
\node[box=0-for-negatives] (p-81-26) at (-81/2+26,-81) {};
\node[box=0-for-negatives] (p-81-27) at (-81/2+27,-81) {};
\node[box=0-for-negatives] (p-81-28) at (-81/2+28,-81) {};
\node[box=0-for-negatives] (p-81-29) at (-81/2+29,-81) {};
\node[box=0-for-negatives] (p-81-30) at (-81/2+30,-81) {};
\node[box=0-for-negatives] (p-81-31) at (-81/2+31,-81) {};
\node[box=0-for-negatives] (p-81-32) at (-81/2+32,-81) {};
\node[box=0-for-negatives] (p-81-33) at (-81/2+33,-81) {};
\node[box=0-for-negatives] (p-81-34) at (-81/2+34,-81) {};
\node[box=0-for-negatives] (p-81-35) at (-81/2+35,-81) {};
\node[box=0-for-negatives] (p-81-36) at (-81/2+36,-81) {};
\node[box=0-for-negatives] (p-81-37) at (-81/2+37,-81) {};
\node[box=0-for-negatives] (p-81-38) at (-81/2+38,-81) {};
\node[box=0-for-negatives] (p-81-39) at (-81/2+39,-81) {};
\node[box=0-for-negatives] (p-81-40) at (-81/2+40,-81) {};
\node[box=0-for-negatives] (p-81-41) at (-81/2+41,-81) {};
\node[box=0-for-negatives] (p-81-42) at (-81/2+42,-81) {};
\node[box=0-for-negatives] (p-81-43) at (-81/2+43,-81) {};
\node[box=0-for-negatives] (p-81-44) at (-81/2+44,-81) {};
\node[box=0-for-negatives] (p-81-45) at (-81/2+45,-81) {};
\node[box=0-for-negatives] (p-81-46) at (-81/2+46,-81) {};
\node[box=0-for-negatives] (p-81-47) at (-81/2+47,-81) {};
\node[box=0-for-negatives] (p-81-48) at (-81/2+48,-81) {};
\node[box=0-for-negatives] (p-81-49) at (-81/2+49,-81) {};
\node[box=0-for-negatives] (p-81-50) at (-81/2+50,-81) {};
\node[box=0-for-negatives] (p-81-51) at (-81/2+51,-81) {};
\node[box=0-for-negatives] (p-81-52) at (-81/2+52,-81) {};
\node[box=0-for-negatives] (p-81-53) at (-81/2+53,-81) {};
\node[box=0-for-negatives] (p-81-54) at (-81/2+54,-81) {};
\node[box=0-for-negatives] (p-81-55) at (-81/2+55,-81) {};
\node[box=0-for-negatives] (p-81-56) at (-81/2+56,-81) {};
\node[box=0-for-negatives] (p-81-57) at (-81/2+57,-81) {};
\node[box=0-for-negatives] (p-81-58) at (-81/2+58,-81) {};
\node[box=0-for-negatives] (p-81-59) at (-81/2+59,-81) {};
\node[box=0-for-negatives] (p-81-60) at (-81/2+60,-81) {};
\node[box=0-for-negatives] (p-81-61) at (-81/2+61,-81) {};
\node[box=0-for-negatives] (p-81-62) at (-81/2+62,-81) {};
\node[box=0-for-negatives] (p-81-63) at (-81/2+63,-81) {};
\node[box=0-for-negatives] (p-81-64) at (-81/2+64,-81) {};
\node[box=0-for-negatives] (p-81-65) at (-81/2+65,-81) {};
\node[box=0-for-negatives] (p-81-66) at (-81/2+66,-81) {};
\node[box=0-for-negatives] (p-81-67) at (-81/2+67,-81) {};
\node[box=0-for-negatives] (p-81-68) at (-81/2+68,-81) {};
\node[box=0-for-negatives] (p-81-69) at (-81/2+69,-81) {};
\node[box=0-for-negatives] (p-81-70) at (-81/2+70,-81) {};
\node[box=0-for-negatives] (p-81-71) at (-81/2+71,-81) {};
\node[box=0-for-negatives] (p-81-72) at (-81/2+72,-81) {};
\node[box=0-for-negatives] (p-81-73) at (-81/2+73,-81) {};
\node[box=0-for-negatives] (p-81-74) at (-81/2+74,-81) {};
\node[box=0-for-negatives] (p-81-75) at (-81/2+75,-81) {};
\node[box=0-for-negatives] (p-81-76) at (-81/2+76,-81) {};
\node[box=0-for-negatives] (p-81-77) at (-81/2+77,-81) {};
\node[box=0-for-negatives] (p-81-78) at (-81/2+78,-81) {};
\node[box=0-for-negatives] (p-81-79) at (-81/2+79,-81) {};
\node[box=0-for-negatives] (p-81-80) at (-81/2+80,-81) {};
\node[box=1-for-negatives] (p-81-81) at (-81/2+81,-81) {};

\end{tikzpicture}

\end{document}
%%%%%%%%%%%%%%%%%%%%%%%%%%%%%%%%%
