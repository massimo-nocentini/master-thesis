
\begin{figure}[p]

    \hspace{-1cm}
    \noindent\makebox[\textwidth]{
        \centering
        %\includegraphics[width=0.8\textwidth]{../../sympy/catalan/coloured.pdf}

        % using *angle* property to rotate it is difficult to properly align it
        % in order to have a "real" matrix representation.
        \includegraphics[width=14cm, height=14cm, keepaspectratio=true]{Chapters/mod-p-characterization/pascal/multiples-over-antidiagonal.pdf}
    }

    \vskip1cm

    \hspace{-1cm}
    \noindent\makebox[\textwidth]{
        \centering
        %\includegraphics[width=0.8\textwidth]{../../sympy/catalan/coloured.pdf}

        % using *angle* property to rotate it is difficult to properly align it
        % in order to have a "real" matrix representation.
        \includegraphics[width=14cm, height=14cm, keepaspectratio=true]{Chapters/mod-p-characterization/pascal/pascal-inverse-ignore-negatives-centered-colouring-127-rows-mod3-partitioning-triangle}
    }

    % this 'particular' line is necessary to use `displaymath' environment
    % into the caption environment, togheter with the inclusion of 
    % `caption' package. See here for more explanation:
    % http://stackoverflow.com/questions/2716227/adding-an-equation-or-formula-to-a-figure-caption-in-latex
    \captionsetup{singlelinecheck=off}
    \caption[Congruences over antidiagonals within $\mathcal{P}_{\equiv_{3}}^{(3)}$
        and $\left(\mathcal{P}^{-1}\right)_{\equiv_{3}}^{(3)}$]{
        On both clusters is shown the set $\Theta_{\diagdown}^{(1)}$.
        
        On \emph{top}, $\mathcal{P}_{\equiv_{3}}^{(1)}$ is highlighted, while on \emph{bottom} is highlighted 
            $\left(\mathcal{P}_{\equiv_{3}}^{-1}\right)^{(1)}$:
            for both clusters, some congruent coefficients lying on antidiagonals $6$ and $8$ are highlighted,
                distance $3^{2}$ spaces them all. 
        }

    \label{fig:pascal-multiples-over-antidiagonal}

\end{figure}
