
\section{Pascal characterization}

In this section we present some results about Pascal array and
its inverse, taken modulo some prime $p$. 

Associating different colours to elements belonging to different remainder
classes, we show formally that for $p$ even, ie. $p=2$, elements in the same
position get the same colour, hence both triangles are coloured in the same
way, ignoring signs for elements in the inverse array, which can happen to be
negative. On the other hand, for $p$ odd there is not such a simple
correspondence and we attempt to observe some repeating pattern in the mapping
among elements in those arrays.

\subsection{A general congruence $\equiv_{p}$ over $\mathcal{P}$ and $\mathcal{P}^{-1}$,
    for $p$ prime}

Let $\mathcal{P}$ be the Riordan array for the Pascal triangle,
defined as:
\begin{displaymath} 
    \mathcal{P} = \left(\frac{1}{1-t}, \frac{t}{1-t}  \right)
\end{displaymath} 
and let $\mathcal{P}^{-1}$ be the inverse of $\mathcal{P}$:
\begin{displaymath} 
    \mathcal{P}^{-1} = \left(\frac{1}{1+t}, \frac{t}{1+t}  \right)
\end{displaymath} 

Let $d_{nk}$ and $\hat{d}_{nk}$ be the generic elements of $\mathcal{P}$ and of
$\mathcal{P}^{-1}$, respectively. Since both arrays are in the Riordan group,
$d_{nk}$ can be written as:
\begin{displaymath}
    \begin{split}
        d_{nk} &= [t^n]\frac{1}{1-t}\left(\frac{t}{1-t}\right)^k = [t^{n-k}](1-t)^{-(k+1)} \\
            &= {{-(k+1)} \choose {n-k}}(-1)^{n-k} = {{k+1 +n-k -1} \choose {n-k}} = {{n} \choose {n-k}} \\
    \end{split}
\end{displaymath}
using the same approach $\hat{d}_{nk}$ can be written as:
\begin{displaymath}
  \begin{split}
    \hat{d}_{nk} &= [t^n]\frac{1}{1+t}\left(\frac{t}{1+t}\right)^k = [t^{n-k}](1+t)^{-(k+1)} = 
    {{-(k+1)} \choose {n-k}} \\
    &= {{k+1 +n-k -1} \choose {n-k}} (-1)^{n-k} = {{n} \choose {n-k}} (-1)^{n-k}\\
  \end{split}
\end{displaymath}
Hence, equating binomial coefficients yields:
\begin{displaymath}
  [t^n]\frac{1}{1-t}\left(\frac{t}{1-t}\right)^k = (-1)^{k-n}[t^n]\frac{1}{1+t}\left(\frac{t}{1+t}\right)^k 
\end{displaymath}
Choose a prime $p$ and apply the modulo operator on both members:
\begin{displaymath}
  [t^n]\frac{1}{1-t}\left(\frac{t}{1-t}\right)^k \equiv_{p} (-1)^{k-n}[t^n]\frac{1}{1+t}\left(\frac{t}{1+t}\right)^k 
\end{displaymath}


\marginpar{$b^{-1}\mod p$ exists if and only if $(b,p)=1$} From now on we have
to reason according modular arithmetic, in particular we have to keep in mind
that multiplying by a term $a$ both member of an equation for the sake of
simplification a term $b$, requests to show that $a$ is the
\emph{multiplicative inverse} of $b$ modulo $p$, denoted by: 
\begin{displaymath}
    a = b^{-1}\mod p
\end{displaymath}

First of all, observe that $-1 \equiv_{p} p-1$ and since $p$ is a
prime by hp, it follows that $(p, p-1)=1$ (this result holds in
general, not just for $p$ prime), which proofs the existence of
both $-1$ and $p-1$ inverses, denoted by $(-1)^{-1}\mod p$ and
$(p-1)^{-1}\mod p$ respectively.

In order to find $(p-1)^{-1}\mod p$ we have to satisfy the
congruence equation $(p-1) \cdot (p-1)^{-1} \equiv_{p} 1$. Choose
$(p-1)^{-1}\mod p = p-1$ and verify
$(p-1) \cdot (p-1) \equiv_{p} p^{2} -2\,p +1 \equiv_{p} 1$ as required.

Another useful observation concerns raising to negative powers:
\begin{displaymath}
    (-1)^{-k} \equiv_{p} \left((-1)^{-1}\right)^{k} \equiv_{p} (p-1)^k
\end{displaymath}
where $k \geq 0$ and, since $-1 \equiv_{p} p-1$, it follows that
$(-1)^{-k} \equiv_{p} (-1)^k$.

Now we can use previous observation on the main congruence:
\begin{displaymath}
    \begin{split}
        [t^n]\frac{1}{1-t}\left(\frac{t}{1-t}\right)^k 
            &\equiv_{p} (-1)^{k-n}[t^n]\frac{1}{1+t}\left(\frac{t}{1+t}\right)^k \\
            &\equiv_{p} (-1)^{k+n}[t^n]\frac{1}{1+t}\left(\frac{t}{1+t}\right)^k \\
            &\equiv_{p} (p-1)^{k+n}[t^n]\frac{1}{1+t}\left(\frac{t}{1+t}\right)^k \\
    \end{split}
\end{displaymath}
Hence, multiplying by $(p-1)^{-1}\mod p$ both members $k+n$ times:
\begin{displaymath}
    (p-1)^{k+n}[t^n]\frac{1}{1-t}\left(\frac{t}{1-t}\right)^k \equiv_{p} [t^n]\frac{1}{1+t}\left(\frac{t}{1+t}\right)^k 
\end{displaymath}
which is the same as:
\begin{displaymath}
    (-1)^{k+n}[t^n]\frac{1}{1-t}\left(\frac{t}{1-t}\right)^k \equiv_{p} [t^n]\frac{1}{1+t}\left(\frac{t}{1+t}\right)^k 
\end{displaymath}
and relating generic elements $d_{nk}$ and $\hat{d}_{nk}$:
\marginpar{a general congruence $\equiv_{p}$ over $\mathcal{P}$ and $\mathcal{P}^{-1}$, where $p$
    is an arbitrary prime}
\begin{equation}
    \label{eq:general:congruence:over:pascal:arrays}
    (p-1)^{k+n}d_{nk}\equiv_{p}(-1)^{k+n}d_{nk} \equiv_{p} \hat{d}_{nk}
\end{equation}

\subsection{$\mathcal{P}_{\equiv_{2}}$} 

The case where $p$ is the even prime produces the colouring reported in
\autoref{fig:pascal-standard-ignore-negatives-centered-colouring-127-rows-mod2-partitioning-triangle} for $\mathcal{P}$,
and the colouring reported in
\autoref{fig:pascal-inverse-ignore-negatives-centered-colouring-127-rows-mod2-partitioning-triangle} for $\mathcal{P}^{-1}$,
where negatives entries are ignored, while in 
\autoref{fig:pascal-inverse-handle-negatives-centered-colouring-127-rows-mod2-partitioning-triangle} such entries are handled.
Formally, $\mathcal{P}$ and $\mathcal{P}^{-1}$ look the same because if we fix $p=2$, 
\autoref{eq:general:congruence:over:pascal:arrays} yields:
\begin{displaymath} 
    d_{nk} \equiv_{2} \hat{d}_{nk} 
\end{displaymath} 
for any choice of $n$ and $k$ in $\mathbb{N}$.



\begin{table}
    \begin{displaymath} 
        \hspace{-5cm}
        \mathcal{P}_{10}\left(\begin{array}{rrrrrrrrrr}
        1 & 0 & 0 & 0 & 0 & 0 & 0 & 0 & 0 & 0 \\
        1 & 1 & 0 & 0 & 0 & 0 & 0 & 0 & 0 & 0 \\
        1 & 2 & 1 & 0 & 0 & 0 & 0 & 0 & 0 & 0 \\
        1 & 3 & 3 & 1 & 0 & 0 & 0 & 0 & 0 & 0 \\
        1 & 4 & 6 & 4 & 1 & 0 & 0 & 0 & 0 & 0 \\
        1 & 5 & 10 & 10 & 5 & 1 & 0 & 0 & 0 & 0 \\
        1 & 6 & 15 & 20 & 15 & 6 & 1 & 0 & 0 & 0 \\
        1 & 7 & 21 & 35 & 35 & 21 & 7 & 1 & 0 & 0 \\
        1 & 8 & 28 & 56 & 70 & 56 & 28 & 8 & 1 & 0 \\
        1 & 9 & 36 & 84 & 126 & 126 & 84 & 36 & 9 & 1
        \end{array}\right) 
        \quad
        \mathcal{P}_{10}^{-1}\left(\begin{array}{rrrrrrrrrr}
        1 & 0 & 0 & 0 & 0 & 0 & 0 & 0 & 0 & 0 \\
        -1 & 1 & 0 & 0 & 0 & 0 & 0 & 0 & 0 & 0 \\
        1 & -2 & 1 & 0 & 0 & 0 & 0 & 0 & 0 & 0 \\
        -1 & 3 & -3 & 1 & 0 & 0 & 0 & 0 & 0 & 0 \\
        1 & -4 & 6 & -4 & 1 & 0 & 0 & 0 & 0 & 0 \\
        -1 & 5 & -10 & 10 & -5 & 1 & 0 & 0 & 0 & 0 \\
        1 & -6 & 15 & -20 & 15 & -6 & 1 & 0 & 0 & 0 \\
        -1 & 7 & -21 & 35 & -35 & 21 & -7 & 1 & 0 & 0 \\
        1 & -8 & 28 & -56 & 70 & -56 & 28 & -8 & 1 & 0 \\
        -1 & 9 & -36 & 84 & -126 & 126 & -84 & 36 & -9 & 1
        \end{array}\right) 
    \end{displaymath}

  \caption[$\mathcal{P}$ and $\mathcal{P}^{-1}$]{Two $10$-minors of
  $\mathcal{P}$ and $\mathcal{P}^{-1}$ matrix expansions, respectively}

  \label{tab:pascal:array} 
  
  \end{table}





\begin{figure}[p]
    
    \hspace{-1.5cm}
    \noindent\makebox[\textwidth]{
        \centering
        %\includegraphics[width=0.8\textwidth]{../../sympy/catalan/coloured.pdf}

        % using *angle* property to rotate it is difficult to properly align it
        % in order to have a "real" matrix representation.
        \includegraphics[width=20cm, height=20cm, keepaspectratio=true]{../sympy/pascal/pascal-standard-ignore-negatives-centered-colouring-127-rows-mod2-partitioning-triangle.pdf}
    }

    % this 'particular' line is necessary to use `displaymath' environment
    % into the caption environment, togheter with the inclusion of 
    % `caption' package. See here for more explanation:
    % http://stackoverflow.com/questions/2716227/adding-an-equation-or-formula-to-a-figure-caption-in-latex
    \captionsetup{singlelinecheck=off}
    \caption[$\mathcal{P}_{\equiv_{2}}$]{
        $\mathcal{P}_{\equiv_{2}}$
        \iffalse
        Pascal triangle, formally: 
        \begin{displaymath}
            \mathcal{P}=\left(\frac{1}{1-t}, \frac{t}{1-t}\right)
        \end{displaymath} % \newline % new line no more necessary
        standard, ignore negatives, centered colouring, 127 rows, mod2 partitioning
        \fi
        }

    \label{fig:pascal-standard-ignore-negatives-centered-colouring-127-rows-mod2-partitioning-triangle}

\end{figure}


\begin{figure}[p]

    \noindent\makebox[\textwidth]{
        \centering
        %\includegraphics[width=0.8\textwidth]{../../sympy/catalan/coloured.pdf}

        % using *angle* property to rotate it is difficult to properly align it
        % in order to have a "real" matrix representation.
        \includegraphics[width=20cm, height=20cm, keepaspectratio=true]{../sympy/pascal/pascal-inverse-ignore-negatives-centered-colouring-127-rows-mod2-partitioning-triangle.pdf}
    }

    % this 'particular' line is necessary to use `displaymath' environment
    % into the caption environment, togheter with the inclusion of 
    % `caption' package. See here for more explanation:
    % http://stackoverflow.com/questions/2716227/adding-an-equation-or-formula-to-a-figure-caption-in-latex
    \captionsetup{singlelinecheck=off}
    \caption[$\mathcal{P}_{\equiv_{2}}^{-1}$, ignore negative entries]{
        Pascal triangle, formally: 
        \begin{displaymath}
            \mathcal{P}^{-1}=\left(\frac{1}{t + 1}, \frac{t}{t + 1}\right)
        \end{displaymath} % \newline % new line no more necessary
        inverse, ignore negatives, centered colouring, 127 rows, mod2 partitioning}

    \label{fig:pascal-inverse-ignore-negatives-centered-colouring-127-rows-mod2-partitioning-triangle}

\end{figure}


\begin{figure}[p]

    \noindent\makebox[\textwidth]{
        \centering
        %\includegraphics[width=0.8\textwidth]{../../sympy/catalan/coloured.pdf}

        % using *angle* property to rotate it is difficult to properly align it
        % in order to have a "real" matrix representation.
        \includegraphics[width=20cm, height=20cm, keepaspectratio=true]{../sympy/pascal/pascal-inverse-handle-negatives-centered-colouring-127-rows-mod2-partitioning-triangle.pdf}
    }

    % this 'particular' line is necessary to use `displaymath' environment
    % into the caption environment, togheter with the inclusion of 
    % `caption' package. See here for more explanation:
    % http://stackoverflow.com/questions/2716227/adding-an-equation-or-formula-to-a-figure-caption-in-latex
    \captionsetup{singlelinecheck=off}
    \caption[$\mathcal{P}_{\equiv_{2}}^{-1}$, handle negative entries]{
        Pascal triangle, formally: 
        \begin{displaymath}
            \mathcal{P}^{-1}=\left(\frac{1}{t + 1}, \frac{t}{t + 1}\right)
        \end{displaymath} % \newline % new line no more necessary
        inverse, handle negatives, centered colouring, 127 rows, mod2 partitioning}

    \label{fig:pascal-inverse-handle-negatives-centered-colouring-127-rows-mod2-partitioning-triangle}

\end{figure}


\subsection{$\mathcal{P}_{\equiv_{3}}$}

Here there's a more interesting pattern to study, and in general,
for any odd prime $p$, colourings of $\mathcal{P}$ and its inverse 
mismatch. In this section we tackle the case for $p=3$,
instantiating \autoref{eq:general:congruence:over:pascal:arrays} yields:
\begin{displaymath}
  2^{k+n}d_{nk}\equiv_{3}(-1)^{k+n}d_{nk} \equiv_{3} \hat{d}_{nk}
\end{displaymath}

For the sake of clarity, let us consider row $4$ of both triangles:
\begin{itemize}
\item $\mathcal{P}_{[3,\,:]} = (1 \quad 3 \quad 3 \quad 1) \equiv_{3}(1 \quad 0 \quad 0 \quad 1)$
\item $\mathcal{P}^{-1}_{[3,\,:]} = (-1 \quad 3 \quad -3 \quad 1) \equiv_{3}(2 \quad 0 \quad 0 \quad 1)$
\end{itemize}
\marginpar{$[\alpha,\beta]$
 in $\mathcal{R}_{[\alpha,\beta]}$ is the \emph{slice operator}, 
 where $\alpha,\beta\in\mathbb{Z}\cup\lbrace:\rbrace$, as defined in NumPy or in Octave}
Hence element $d_{30}$ gets a color $c$ while $\hat{d}_{30}$ gets
a color $c'$ different from $c$: no general mapping about this relationship appears, so we
carefully study the two coloured triangles. 

Our exercise here is as follow: first fix the order $\alpha$ of a principal
cluster $\mathcal{P}^{(\alpha)}$, say $\alpha=4$; then move upwards from the
last row of $\mathcal{P}^{(4)}$ to the top, one row by one, repeatedly. For
each considered row, move toward the right over columns, trying to understand a
possible congruence over coefficients in the two arrays:
\begin{displaymath}
    \begin{split}
        d_{3^4 -1,0} &\equiv_{3} \hat{d}_{3^4 -1,0} \\
        d_{3^4 -1,1} &\equiv_{3} \hat{d}_{3^4 -2,0} \\
        d_{3^4 -1,2} &\equiv_{3} \hat{d}_{3^4 -3,0} \\
        d_{3^4 -1,3} &\equiv_{3} \hat{d}_{3^4 -4,0} \\
        &\vdots
    \end{split}
\end{displaymath}
next, a congruence going up one row:
\marginpar{remember that the last row of $\mathcal{P}^{(\alpha)}$ is $p^{\alpha}-1$,
    for $p$ given }
\begin{displaymath}
    \begin{split}
        d_{3^4 -2,0} &\equiv_{3} \hat{d}_{3^4 -1,1} \\
        d_{3^4 -2,1} &\equiv_{3} \hat{d}_{3^4 -2,1} \\
        d_{3^4 -2,2} &\equiv_{3} \hat{d}_{3^4 -3,1} \\
        d_{3^4 -2,3} &\equiv_{3} \hat{d}_{3^4 -4,1} \\
        &\vdots
    \end{split}
\end{displaymath}
next, a congruence going up two rows:
\begin{displaymath}
    \begin{split}
        d_{3^4 -3,0} &\equiv_{3} \hat{d}_{3^4 -1,2} \\
        d_{3^4 -3,1} &\equiv_{3} \hat{d}_{3^4 -2,2} \\
        d_{3^4 -3,2} &\equiv_{3} \hat{d}_{3^4 -3,2} \\
        d_{3^4 -3,3} &\equiv_{3} \hat{d}_{3^4 -4,2} \\
        &\vdots
    \end{split}
\end{displaymath}

A structure over indices can be caught, so let us introduce variable
$b\in\lbrace1,\ldots,3^{4}\rbrace$ running over rows, and variable
$a\in\lbrace 0,\ldots,3^{4}-b\rbrace$, running over columns. So we can state the following 
congruence:
\begin{equation}
    \label{eq:pascal:arrays:congruence:rows:cols}
    d_{3^4 -b,a} \equiv_{3} \hat{d}_{3^4 -1-a,b-1} 
\end{equation}
\marginpar{column segments of $\mathcal{P}$ are congruent to row segment of
    $\mathcal{P}^{-1}$ and viceversa}

Let's say, assume the colouring for $\mathcal{P}^{-1}$ triangle is
given and the colouring for row $26$ of $\mathcal{P}$ is desired. It
is necessary to find $b$: by $3^4 -b=26$ get $b=55$, so the
coefficients lying on the required row satisfy the following congruence:
\begin{displaymath}
        d_{26,a} \equiv_{3} \hat{d}_{80-a,54} 
\end{displaymath}
On the other hand, we could have choosed a column, compute the value 
for $a$ and then state a congruence with $b$ as free variable.

\iffalse
\\\\
A question is still open: why do we choose row $3^4$ as reference
row?

It is useful to recall a theorem due to Fine:
\begin{theorem}
  A necessary and sufficient condition for a binomial coefficient
  ${{n} \choose {m}}$ to be divisible by a prime $p$ is that $n$
  be a power of $p$.
\end{theorem}
Consider the colouring for triangle $\mathcal{P}$, we can use the
given theorem to point out ``interesting'' rows, namely those rows
affected by the theorem, they correspond to powers
$3^1, 3^2, 3^3, 3^4, \ldots$, each one of them can be easily
recognized since dots lying on it have all the same colour. In the
triangle $127$ former rows are drawn and in order to have ``more
space'' to find a modular relationship between $d_{nk}$ and
$\hat{d}_{nk}$ we choose as \emph{reference} row the one with
index $3^4$.  From here we start moving backwards by rows toward
the root: observe that the entire row $3^4 -1$, containing
$3^4 -1$ remainders, of triangle $\mathcal{P}$ is the first
segment of the first column of $\mathcal{P}^{-1}$, in other words
$d_{3^4 -1,a} \equiv_{3} \hat{d}_{3^4 -1 -a, 0}$ for
$a \in \lbrace 0, \ldots, 3^4 -1\rbrace$.

\fi

It seems that coefficient $d_{3^4-1,0}$ acts as a pivot on which
the triangle ``flips'': the root moves toward the reader while the
bottom edge moves toward opposite the reader. This rigid motion is
captured by the following modular relationships among three
important points:
\begin{displaymath}
    \begin{split}
        d_{3^4 -1,0} &\equiv_{3} \hat{d}_{3^4 -1,0} \\
        d_{3^4 -1,3^4 -1} &\equiv_{3} \hat{d}_{0,0} \\
        d_{3^4 -3^3,3^3-1} &\equiv_{3} \hat{d}_{3^4 -3^3,3^3-1} \\
    \end{split}
\end{displaymath}

It is natural to ask if \autoref{eq:pascal:arrays:congruence:rows:cols}
holds for an arbitrary prime $p$.
\begin{conjecture}
    Let $p$ be a prime and $\alpha\geq0$. \\
    If $d_{nk}\in\mathcal{P}^{(\alpha)}$ and
    $\hat{d}_{nk}\in\left(\mathcal{P}^{-1}\right)^{(\alpha)}$, then:
    \begin{equation}
        d_{p^\alpha -b,a} \equiv_{p} \hat{d}_{p^\alpha -1-a,b-1} 
    \end{equation}
    where $b\in\lbrace1,\ldots,p^{\alpha}\rbrace$ and 
    $a\in\lbrace 0,\ldots,p^{\alpha}-b\rbrace$.

\end{conjecture}
