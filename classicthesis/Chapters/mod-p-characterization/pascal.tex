
\section{Pascal characterization}

In this section we present some results about Pascal array and
its inverse, taken modulo some prime $p$. 

Associating different colours to elements belonging to different remainder
classes, we show formally that for $p$ even, ie. $p=2$, elements in the same
position get the same colour, hence both triangles are coloured in the same
way, ignoring signs for elements in the inverse array, which can happen to be
negative. On the other hand, for $p$ odd there is not such a simple
correspondence and we attempt to observe some repeating pattern in the mapping
among elements in those arrays.

\subsection{$\equiv_{p}$ over $\mathcal{P}$ and $\mathcal{P}^{-1}$,
    for $p$ prime}

Let $\mathcal{P}$ be the Riordan array for the Pascal triangle,
defined as:
\begin{displaymath} 
    \mathcal{P} = \left(\frac{1}{1-t}, \frac{t}{1-t}  \right)
\end{displaymath} 
and let $\mathcal{P}^{-1}$ be the inverse of $\mathcal{P}$:
\begin{displaymath} 
    \mathcal{P}^{-1} = \left(\frac{1}{1+t}, \frac{t}{1+t}  \right)
\end{displaymath} 

Let $d_{nk}$ and $\hat{d}_{nk}$ be the generic elements of $\mathcal{P}$ and of
$\mathcal{P}^{-1}$, respectively. Since both arrays are in the Riordan group,
$d_{nk}$ can be written as:
\begin{displaymath}
    \begin{split}
        d_{nk} &= [t^n]\frac{1}{1-t}\left(\frac{t}{1-t}\right)^k = [t^{n-k}](1-t)^{-(k+1)} \\
            &= {{-(k+1)} \choose {n-k}}(-1)^{n-k} = {{k+1 +n-k -1} \choose {n-k}} = {{n} \choose {n-k}} \\
    \end{split}
\end{displaymath}
using the same approach $\hat{d}_{nk}$ can be written as:
\begin{displaymath}
  \begin{split}
    \hat{d}_{nk} &= [t^n]\frac{1}{1+t}\left(\frac{t}{1+t}\right)^k = [t^{n-k}](1+t)^{-(k+1)} = 
    {{-(k+1)} \choose {n-k}} \\
    &= {{k+1 +n-k -1} \choose {n-k}} (-1)^{n-k} = {{n} \choose {n-k}} (-1)^{n-k}\\
  \end{split}
\end{displaymath}
Hence, equating binomial coefficients yields:
\begin{displaymath}
  [t^n]\frac{1}{1-t}\left(\frac{t}{1-t}\right)^k = (-1)^{k-n}[t^n]\frac{1}{1+t}\left(\frac{t}{1+t}\right)^k 
\end{displaymath}
Choose a prime $p$ and apply the modulo operator on both members:
\begin{displaymath}
  [t^n]\frac{1}{1-t}\left(\frac{t}{1-t}\right)^k \equiv_{p} (-1)^{k-n}[t^n]\frac{1}{1+t}\left(\frac{t}{1+t}\right)^k 
\end{displaymath}


\marginpar{$b^{-1}\mod p$ exists if and only if $(b,p)=1$} From now on we have
to reason according modular arithmetic, in particular we have to keep in mind
that multiplying by a term $a$ both member of an equation for the sake of
simplification a term $b$, requests to show that $a$ is the
\emph{multiplicative inverse} of $b$ modulo $p$, denoted by: 
\begin{displaymath}
    a = b^{-1}\mod p
\end{displaymath}

First of all, observe that $-1 \equiv_{p} p-1$ and since $p$ is a
prime by hp, it follows that $(p, p-1)=1$ (this result holds in
general, not just for $p$ prime), which proofs the existence of
both $-1$ and $p-1$ inverses, denoted by $(-1)^{-1}\mod p$ and
$(p-1)^{-1}\mod p$ respectively.

In order to find $(p-1)^{-1}\mod p$ we have to satisfy the
congruence equation $(p-1) \cdot (p-1)^{-1} \equiv_{p} 1$. Choose
$(p-1)^{-1}\mod p = p-1$ and verify
$(p-1) \cdot (p-1) \equiv_{p} p^{2} -2\,p +1 \equiv_{p} 1$ as required.

Another useful observation concerns raising to negative powers:
\begin{displaymath}
    (-1)^{-k} \equiv_{p} \left((-1)^{-1}\right)^{k} \equiv_{p} (p-1)^k
\end{displaymath}
where $k \geq 0$ and, since $-1 \equiv_{p} p-1$, it follows that
$(-1)^{-k} \equiv_{p} (-1)^k$.

Now we can use previous observation on the main congruence:
\begin{displaymath}
    \begin{split}
        [t^n]\frac{1}{1-t}\left(\frac{t}{1-t}\right)^k 
            &\equiv_{p} (-1)^{k-n}[t^n]\frac{1}{1+t}\left(\frac{t}{1+t}\right)^k \\
            &\equiv_{p} (-1)^{k+n}[t^n]\frac{1}{1+t}\left(\frac{t}{1+t}\right)^k \\
            &\equiv_{p} (p-1)^{k+n}[t^n]\frac{1}{1+t}\left(\frac{t}{1+t}\right)^k \\
    \end{split}
\end{displaymath}
Hence, multiplying by $(p-1)^{-1}\mod p$ both members $k+n$ times:
\begin{displaymath}
    (p-1)^{k+n}[t^n]\frac{1}{1-t}\left(\frac{t}{1-t}\right)^k \equiv_{p} [t^n]\frac{1}{1+t}\left(\frac{t}{1+t}\right)^k 
\end{displaymath}
which is the same as:
\begin{displaymath}
    (-1)^{k+n}[t^n]\frac{1}{1-t}\left(\frac{t}{1-t}\right)^k \equiv_{p} [t^n]\frac{1}{1+t}\left(\frac{t}{1+t}\right)^k 
\end{displaymath}
and relating generic elements $d_{nk}$ and $\hat{d}_{nk}$:
\marginpar{a general congruence $\equiv_{p}$ over $\mathcal{P}$ and $\mathcal{P}^{-1}$, where $p$
    is an arbitrary prime}
\begin{equation}
    \label{eq:general:congruence:over:pascal:arrays}
    (p-1)^{k+n}d_{nk}\equiv_{p}(-1)^{k+n}d_{nk} \equiv_{p} \hat{d}_{nk}
\end{equation}

\subsection{$\mathcal{P}_{\equiv_{2}}$} 

The case where $p$ is the even prime produces the colouring reported in
\autoref{fig:pascal-standard-ignore-negatives-centered-colouring-127-rows-mod2-partitioning-triangle} for $\mathcal{P}$,
and the colouring reported in
\autoref{fig:pascal-inverse-ignore-negatives-centered-colouring-127-rows-mod2-partitioning-triangle} for $\mathcal{P}^{-1}$,
where negatives entries are ignored, while in 
\autoref{fig:pascal-inverse-handle-negatives-centered-colouring-127-rows-mod2-partitioning-triangle} such entries are handled.
Formally, $\mathcal{P}$ and $\mathcal{P}^{-1}$ look the same because if we fix $p=2$, 
\autoref{eq:general:congruence:over:pascal:arrays} yields:
\begin{displaymath} 
    d_{nk} \equiv_{2} \hat{d}_{nk} 
\end{displaymath} 
for any choice of $n$ and $k$ in $\mathbb{N}$.

\input{Chapters/mod-p-characterization/pascal/pascal-standard-ignore-negatives-centered-colouring-127-rows-mod2-partitioning-include-matrix.tex}
\input{Chapters/mod-p-characterization/pascal/pascal-standard-ignore-negatives-centered-colouring-127-rows-mod2-partitioning-include-figure.tex}

\begin{figure}[p]

    \noindent\makebox[\textwidth]{
        \centering
        %\includegraphics[width=0.8\textwidth]{../../sympy/catalan/coloured.pdf}

        % using *angle* property to rotate it is difficult to properly align it
        % in order to have a "real" matrix representation.
        \includegraphics[width=20cm, height=20cm, keepaspectratio=true]{../sympy/pascal/pascal-inverse-ignore-negatives-centered-colouring-127-rows-mod2-partitioning-triangle.pdf}
    }

    % this 'particular' line is necessary to use `displaymath' environment
    % into the caption environment, togheter with the inclusion of 
    % `caption' package. See here for more explanation:
    % http://stackoverflow.com/questions/2716227/adding-an-equation-or-formula-to-a-figure-caption-in-latex
    \captionsetup{singlelinecheck=off}
    \caption[$\mathcal{P}_{\equiv_{2}}^{-1}$, ignore negative entries]{
        Pascal triangle, formally: 
        \begin{displaymath}
            \mathcal{P}^{-1}=\left(\frac{1}{t + 1}, \frac{t}{t + 1}\right)
        \end{displaymath} % \newline % new line no more necessary
        inverse, ignore negatives, centered colouring, 127 rows, mod2 partitioning}

    \label{fig:pascal-inverse-ignore-negatives-centered-colouring-127-rows-mod2-partitioning-triangle}

\end{figure}

\input{Chapters/mod-p-characterization/pascal/pascal-inverse-handle-negatives-centered-colouring-127-rows-mod2-partitioning-include-figure.tex}


\subsection{$\mathcal{P}_{\equiv_{3}}$}

Here there's a more interesting pattern to study, and in general,
for any odd prime $p$, colourings of $\mathcal{P}$ and its inverse 
mismatch. In this section we tackle the case for $p=3$,
instantiating \autoref{eq:general:congruence:over:pascal:arrays} yields:
\begin{displaymath}
  2^{k+n}d_{nk}\equiv_{3}(-1)^{k+n}d_{nk} \equiv_{3} \hat{d}_{nk}
\end{displaymath}

For the sake of clarity, let us consider row $4$ of both triangles:
\begin{itemize}
\item $\mathcal{P}_{[3,\,:]} = (1 \quad 3 \quad 3 \quad 1) \equiv_{3}(1 \quad 0 \quad 0 \quad 1)$
\item $\mathcal{P}^{-1}_{[3,\,:]} = (-1 \quad 3 \quad -3 \quad 1) \equiv_{3}(2 \quad 0 \quad 0 \quad 1)$
\end{itemize}
\marginpar{$[\alpha,\beta]$
 in $\mathcal{R}_{[\alpha,\beta]}$ is the \emph{slice operator}, 
 where $\alpha,\beta\in\mathbb{Z}\cup\lbrace:\rbrace$, as defined in NumPy or in Octave}
Hence element $d_{30}$ gets a color $c$ while $\hat{d}_{30}$ gets
a color $c'$ different from $c$: no general mapping about this relationship appears, so we
study the two coloured triangles carefully. 

Our exercise here is as follow: first fix the order $\alpha$ of a principal
cluster $\mathcal{P}^{(\alpha)}$, say $\alpha=4$; then move upwards from the
last row of $\mathcal{P}^{(4)}$ to the top, one row by one, repeatedly. For
each considered row, move toward the right over columns, trying to understand a
possible congruence over coefficients in the two arrays, see
\autoref{fig:pascal:standard:mod3:congruence:zero:row:above}
for some highlights:

\begin{figure}[p]

    \noindent\makebox[\textwidth]{
        \centering
        %\includegraphics[width=0.8\textwidth]{../../sympy/catalan/coloured.pdf}

        % using *angle* property to rotate it is difficult to properly align it
        % in order to have a "real" matrix representation.
        \includegraphics[width=14cm, height=14cm, keepaspectratio=true]{Chapters/mod-p-characterization/pascal/pascal-standard-mod3-zero-row-above-end.pdf}
    }

    \vskip2cm

    \noindent\makebox[\textwidth]{
        \centering
        %\includegraphics[width=0.8\textwidth]{../../sympy/catalan/coloured.pdf}

        % using *angle* property to rotate it is difficult to properly align it
        % in order to have a "real" matrix representation.
        \includegraphics[width=14cm, height=14cm, keepaspectratio=true]{Chapters/mod-p-characterization/pascal/pascal-inverse-mod3-zero-row-above-end.pdf}
    }

    % this 'particular' line is necessary to use `displaymath' environment
    % into the caption environment, togheter with the inclusion of 
    % `caption' package. See here for more explanation:
    % http://stackoverflow.com/questions/2716227/adding-an-equation-or-formula-to-a-figure-caption-in-latex
    \captionsetup{singlelinecheck=off}
    \caption[$\mathcal{P}$'s row $3^{4}-1$ and $\mathcal{P}^{-1}$'s column $0$ ]
        { Highlighting row $3^{4}-1$ of $\mathcal{P}$, on top, and column $0$ of $\mathcal{P}^{-1}$, on bottom,
            to show congruence: 
            \begin{displaymath}
                d_{3^4 -1,a} \equiv_{3} \hat{d}_{3^4 -1-a,0} 
            \end{displaymath}
            for $a\in\lbrace0,\ldots,3^{4}-1\rbrace$ 
        }


    \label{fig:pascal:standard:mod3:congruence:zero:row:above}

\end{figure}

\begin{displaymath}
    \begin{split}
        d_{3^4 -1,0} &\equiv_{3} \hat{d}_{3^4 -1,0} \\
        d_{3^4 -1,1} &\equiv_{3} \hat{d}_{3^4 -2,0} \\
        d_{3^4 -1,2} &\equiv_{3} \hat{d}_{3^4 -3,0} \\
        d_{3^4 -1,3} &\equiv_{3} \hat{d}_{3^4 -4,0} \\
        &\vdots
    \end{split}
\end{displaymath}
next, a congruence going up one row:
\marginpar{remember that the last row of $\mathcal{P}^{(\alpha)}$ is $p^{\alpha}-1$,
    for $p$ given }
\begin{displaymath}
    \begin{split}
        d_{3^4 -2,0} &\equiv_{3} \hat{d}_{3^4 -1,1} \\
        d_{3^4 -2,1} &\equiv_{3} \hat{d}_{3^4 -2,1} \\
        d_{3^4 -2,2} &\equiv_{3} \hat{d}_{3^4 -3,1} \\
        d_{3^4 -2,3} &\equiv_{3} \hat{d}_{3^4 -4,1} \\
        &\vdots
    \end{split}
\end{displaymath}
next, a congruence going up two rows:
\begin{displaymath}
    \begin{split}
        d_{3^4 -3,0} &\equiv_{3} \hat{d}_{3^4 -1,2} \\
        d_{3^4 -3,1} &\equiv_{3} \hat{d}_{3^4 -2,2} \\
        d_{3^4 -3,2} &\equiv_{3} \hat{d}_{3^4 -3,2} \\
        d_{3^4 -3,3} &\equiv_{3} \hat{d}_{3^4 -4,2} \\
        &\vdots
    \end{split}
\end{displaymath}

A structure over indices can be caught, so let us introduce variable
$b\in\lbrace1,\ldots,3^{4}\rbrace$ running over rows, and variable
$a\in\lbrace 0,\ldots,3^{4}-b\rbrace$, running over columns. So we can state the following 
congruence:
\begin{equation}
    \label{eq:pascal:arrays:congruence:rows:cols}
    d_{3^4 -b,a} \equiv_{3} \hat{d}_{3^4 -1-a,b-1} 
\end{equation}
\marginpar{column segments of $\mathcal{P}$ are congruent to row segment of
    $\mathcal{P}^{-1}$ and viceversa}

Let's say, assume the colouring for $\mathcal{P}^{-1}$ triangle is
given and the colouring for row $26$ of $\mathcal{P}$ is desired. It
is necessary to find $b$: by $3^4 -b=26$ get $b=55$, so the
coefficients lying on the required row satisfy the following congruence:
\begin{displaymath}
        d_{26,a} \equiv_{3} \hat{d}_{80-a,54} 
\end{displaymath}

\begin{figure}[p]

    \noindent\makebox[\textwidth]{
        \centering
        %\includegraphics[width=0.8\textwidth]{../../sympy/catalan/coloured.pdf}

        % using *angle* property to rotate it is difficult to properly align it
        % in order to have a "real" matrix representation.
        \includegraphics[width=14cm, height=14cm, keepaspectratio=true]{Chapters/mod-p-characterization/pascal/pascal-standard-mod3-zero-row-26.pdf}
    }

    \vskip1cm

    \noindent\makebox[\textwidth]{
        \centering
        %\includegraphics[width=0.8\textwidth]{../../sympy/catalan/coloured.pdf}

        % using *angle* property to rotate it is difficult to properly align it
        % in order to have a "real" matrix representation.
        \includegraphics[width=14cm, height=14cm, keepaspectratio=true]{Chapters/mod-p-characterization/pascal/pascal-inverse-mod3-zero-row-26.pdf}
    }

    % this 'particular' line is necessary to use `displaymath' environment
    % into the caption environment, togheter with the inclusion of 
    % `caption' package. See here for more explanation:
    % http://stackoverflow.com/questions/2716227/adding-an-equation-or-formula-to-a-figure-caption-in-latex
    \captionsetup{singlelinecheck=off}
    \caption[$\mathcal{P}$'s row $26$ and $\mathcal{P}^{-1}$'s column $54$ ]
        { Highlighting row $26$ of $\mathcal{P}$, on top, and column $54$ of $\mathcal{P}^{-1}$, on bottom,
            to show congruence: 
            \begin{displaymath}
                d_{26,a} \equiv_{3} \hat{d}_{3^4 -1-a,54} 
            \end{displaymath}
            for $a\in\lbrace0,\ldots,26\rbrace$ 
        }


    \label{fig:pascal:standard:mod3:congruence:row:26}

\end{figure}

an highlighting is shown in \autoref{fig:pascal:standard:mod3:congruence:row:26}.
On the other hand, we could have choosed a column, compute the value 
for $a$ and then state a congruence with $b$ as free variable.

% the following chunk of statements is not clearly written and
% I believe it is not so important, so for now skip it.
\iffalse
\\\\
A question is still open: why do we choose row $3^4$ as reference
row?

It is useful to recall a theorem due to Fine:
\begin{theorem}
  A necessary and sufficient condition for a binomial coefficient
  ${{n} \choose {m}}$ to be divisible by a prime $p$ is that $n$
  be a power of $p$.
\end{theorem}
Consider the colouring for triangle $\mathcal{P}$, we can use the
given theorem to point out ``interesting'' rows, namely those rows
affected by the theorem, they correspond to powers
$3^1, 3^2, 3^3, 3^4, \ldots$, each one of them can be easily
recognized since dots lying on it have all the same colour. In the
triangle $127$ former rows are drawn and in order to have ``more
space'' to find a modular relationship between $d_{nk}$ and
$\hat{d}_{nk}$ we choose as \emph{reference} row the one with
index $3^4$.  From here we start moving backwards by rows toward
the root: observe that the entire row $3^4 -1$, containing
$3^4 -1$ remainders, of triangle $\mathcal{P}$ is the first
segment of the first column of $\mathcal{P}^{-1}$, in other words
$d_{3^4 -1,a} \equiv_{3} \hat{d}_{3^4 -1 -a, 0}$ for
$a \in \lbrace 0, \ldots, 3^4 -1\rbrace$.

\fi

% again the following paragraph is a technical detail that can 
% be understood only by talking.
\iffalse
It seems that coefficient $d_{3^4-1,0}$ acts as a pivot on which
the triangle ``flips'': the root moves toward the reader while the
bottom edge moves toward opposite the reader. This rigid motion is
captured by the following modular relationships among three
important points:
\begin{displaymath}
    \begin{split}
        d_{3^4 -1,0} &\equiv_{3} \hat{d}_{3^4 -1,0} \\
        d_{3^4 -1,3^4 -1} &\equiv_{3} \hat{d}_{0,0} \\
        d_{3^4 -3^3,3^3-1} &\equiv_{3} \hat{d}_{3^4 -3^3,3^3-1} \\
    \end{split}
\end{displaymath}
\fi

\subsection{$\mathcal{P}_{\equiv_{p}}$, where $p$ is a prime greater than $3$}

It is natural to ask if \autoref{eq:pascal:arrays:congruence:rows:cols}
holds for an arbitrary prime $p$.
\begin{conjecture}
    Let $p$ be a prime and let $\alpha\in\mathbb{N}$. 
    If $d_{nk}\in\mathcal{P}^{(\alpha)}$ and
    $\hat{d}_{nk}\in\left(\mathcal{P}^{-1}\right)^{(\alpha)}$, then:
    \marginpar{a general congruence over rows of $\mathcal{P}$
        and columns of $\mathcal{P}^{-1}$, for any prime $p$?}
    \begin{equation}
        d_{p^\alpha -b,a} \equiv_{p} \hat{d}_{p^\alpha -1-a,b-1} 
    \end{equation}
    where $b\in\lbrace1,\ldots,p^{\alpha}\rbrace$ and 
    $a\in\lbrace 0,\ldots,p^{\alpha}-b\rbrace$.
\end{conjecture}
We provide a proof sketch:
\begin{proof}
    Using closed formula for coefficients $d_{p^\alpha -b,a}$ and $\hat{d}_{p^\alpha -1-a,b-1}$
    we can rewrite:
    \begin{displaymath}
        {{p^\alpha -b} \choose {a}} \equiv_{p} {{p^\alpha -1-a}\choose{b-1} }\left(-1\right)^{p^{\alpha}-a-b}
    \end{displaymath}
    and simple manipulation of binomial coefficients yield:
    \begin{displaymath}
        {{p^\alpha -b} \choose {p^\alpha -a-b}} \equiv_{p} 
            {{p^\alpha -1-a}\choose{p^\alpha -a-b} }\left(-1\right)^{p^{\alpha}-a-b}
    \end{displaymath}
    rewrite the right hand side using the identity ${{-n}\choose{k}}={{n+k-1}\choose{k}}(-1)^{k}$:
    \begin{displaymath}
        {{p^\alpha -b} \choose {p^\alpha -a-b}} \equiv_{p} 
            {{-b}\choose{p^\alpha -a-b} }
    \end{displaymath}
    to close the proof we could represent $-b$ in base $p$ and, 
    since $b\in\lbrace1,\ldots,p^{\alpha}\rbrace$, the application of
    Lucas theorem to both members produces the same product of binomial
    coefficients, as required.

\end{proof}

\subsection{A proof of \emph{Sierpinski}'s gasket}

In this section we show that in a Pascal array $\mathcal{P}$ it is possible to
recognize a structure similar to the one observed by Sierpinski during his
studies on fractals. More precisely, we show that $\mathcal{P}^{(\alpha)}$, a
principal cluster of order $\alpha$, repeats itself $3$ times within a
\emph{chunk} of the principal cluster of one order bigger: such repetitions
happens in the first $2\,p^{\alpha}$ rows of $\mathcal{P}^{(\alpha+1)}$.
Moreover, those copies of $\mathcal{P}^{(\alpha)}$ surround a zero-hole of
order $\alpha$. Note that for even prime $p$, $\mathcal{P}^{(\alpha)}$ repeats
itself $3$ times in $\mathcal{P}^{(\alpha+1)}$, exactly as in the
\emph{Sierpinski's gasket}, having a pure recursive structure.

On the other hand, for $p$ odd prime, $\mathcal{P}^{(\alpha)}$ repeats itself
in $\mathcal{P}^{(\alpha+1)}$ somewhere, without a complete recursive
structure, due to the oddness of $p$, ie. other coloured subtriangles appears
in $\mathcal{P}^{(\alpha+1)}$. However if we study the repetition up to row
$2\,p^{\alpha}$, the proof still hold for arbitrary prime $p$.

% the following paragraph should be put in a dedicated section, after this one.
\iffalse
, while in general we can say that maximal
triangles of coefficients multiples of $p$, appear with regularity in
$\mathcal{P}_{n+1}$ a number of times equals to:
\begin{displaymath}
    \frac{(p-1)p}{2}
\end{displaymath}
in other words, considering triangles $\mathcal{P}_{n}$ and
$\mathcal{P}_{n+1}$, there are $\frac{(p-1)p}{2}$ upside-down
maximal triangles, with all coefficients multiple of $p$, from row $p^n$
to row $p^{n+1}-1$; this result will be prove at the end of this 
section.
\fi
% ----------------------------------------------------------------------

\begin{proof}

  Let $\mathcal{P}^{(\alpha)}$ be the principal cluster of order $\alpha$ of
  Pascal array $\mathcal{P}$ and let $r$ be a row index for
  $\mathcal{P}^{(\alpha)}$, so $r \in \lbrace 0, \ldots, p^{\alpha} -1
  \rbrace$.
    
  Let $\mathcal{P}_{2;\alpha}$ be a \emph{chunk} of $\mathcal{P}^{(\alpha +1)}$
  that starts at the root and extends downward $2\,p^{\alpha}$ rows, therefore
  including $\mathcal{P}^{(\alpha)}$ as its first halve. It is requested to
  prove that coefficients at \emph{equivalent positions} in the bottom left and
  bottom right triangles of $\mathcal{P}_{2;\alpha}$ are congruent, modulo $p$,
  to coefficients at \emph{equivalent positions} in $\mathcal{P}^{\alpha}$. 
  
  In order to formalize \marginpar{proving congruences over coefficients at
  equivalent positions} the concept of \emph{equivalent positions} we
  introduce the following objects: 
  \begin{itemize}
    \item the pair $(r,c)$ of indices is an \emph{equivalent position} 
        for coefficient $d_{rc}\in\mathcal{P}^{(\alpha)}$ and coefficient 
            $d_{rc}^{\swarrow}$ in the \emph{bottom left} triangle of $\mathcal{P}_{2;\alpha}$ if and only if
                $d_{rc}^{\swarrow} = d_{p^{\alpha}+r,c}$
    \item the pair $(r,c)$ of indices is an \emph{equivalent position} 
        for coefficient $d_{rc}\in\mathcal{P}^{(\alpha)}$ and coefficient 
            $d_{rc}^{\searrow}$ in the \emph{bottom right} triangle of $\mathcal{P}_{2;\alpha}$ if and only if
                $d_{rc}^{\searrow} = d_{p^{\alpha}+r,p^{\alpha}+c}$
  \end{itemize}
  
  After these definitions, we've to prove:
  \begin{displaymath}
    d_{rc} \equiv_p d_{rc}^{\swarrow} \equiv_p d_{rc}^{\searrow} 
  \end{displaymath}
  or, in other words:
  \begin{displaymath}
    {{r} \choose {c}} \equiv_p {{p^{\alpha}+r} \choose {c}} \equiv_p {{p^{\alpha}+r} \choose {p^{\alpha}+c}} 
  \end{displaymath}

  In order to prove such congruences we'll use Lucas theorem:
  first of all, observe that $c \leq r$ because $\mathcal{P}^{(\alpha)}$ 
  is a triangle, therefore $c \in \lbrace 0, \ldots, p^{\alpha} -1 \rbrace$,
  as $r$ satisfies.  By basis representation theorem, there exists
  sequences $\lbrace r_i\rbrace$ and $\lbrace c_i\rbrace$, 
  with both $r_i< p$ and $c_i < p$, for $i \in \lbrace 0, \ldots, p^{\alpha} -1 \rbrace$,
  such that:
  \begin{displaymath}
    \begin{split}
      r &= r_0 + r_1 p + r_2 p^2 + \ldots + r_{{\alpha}-1}p^{{\alpha}-1} \\
      c &= c_0 + c_1 p + c_2 p^2 + \ldots + c_{{\alpha}-1}p^{{\alpha}-1} \\
    \end{split}
  \end{displaymath}
  Settings for Lucas theorem are ready, hence apply it:
  \begin{displaymath}
    \begin{split}
      {{p^{\alpha}+r} \choose {c}} &\equiv_{p} {{r_0} \choose {c_0}} {{r_1} \choose {c_1}}{{r_2} \choose {c_2}} \ldots 
      {{r_{{\alpha}-1}} \choose {c_{{\alpha}-1}}}{{1} \choose {0}} \equiv_{p} {{r} \choose {c}}\\
      {{p^{\alpha}+r} \choose {p^{\alpha}+c}} &\equiv_{p} {{r_0} \choose {c_0}} {{r_1} \choose {c_1}}{{r_2} \choose {c_2}} \ldots 
      {{r_{{\alpha}-1}} \choose {c_{{\alpha}-1}}}{{1} \choose {1}} \equiv_{p} {{r} \choose {c}}\\
    \end{split}
  \end{displaymath}
  therefore coefficients located at \emph{equivalent position} 
  $(r,c)$ belong to the same remainder class of congruence relation
  modulo $p$, as required.
  \\\\

  For the second part \marginpar{proving that a zero-hole of order $\alpha$ 
  is surrounded within $\mathcal{P}_{2;\alpha}$} of the statement, we've to show that in
  $\mathcal{P}_{2;\alpha}$ a zero-hole of order $\alpha$ is
  surrounded by the three ``congruent'' subtriangles composed of
  coefficients $d_{rc}, d_{rc}^{\swarrow}$ and $d_{rc}^{\searrow} $, respectively.

  Observe that the very first coefficient $d_{p^{\alpha}, 0}$ and very last $d_{p^{\alpha}, p^{\alpha}}$ of row
  with index $p^{\alpha}$ are congruent to the unit, modulo $p$:
  \begin{displaymath}
    {{p^{\alpha}} \choose {0}} \equiv_{p}{{p^{\alpha}} \choose {p^{\alpha}}} \equiv_{p} 1
  \end{displaymath}
  while $p$ divides every coefficient between them, let
  $c\in\lbrace1,\ldots, p^{\alpha}-1 \rbrace$:
  \begin{displaymath}
    {{p^{\alpha}} \choose {c}} \equiv_{p} {{0} \choose {c_0}} {{0} \choose {c_1}}{{0} \choose {c_2}} \ldots 
    {{0} \choose {c_{{\alpha}-1}}}{{1} \choose {0}} \equiv_{p} 0
  \end{displaymath}
  By the recurrence rule ${{n+1}\choose {k+1}} = {{n} \choose {k}} + {{n}\choose{ k+1}}$
  characterizing $\mathcal{P}$, observe that:
  \begin{displaymath}
    \begin{split}
      d_{p^{\alpha}+1, 1} &\equiv_{p} d_{p^{\alpha}, 0} + d_{p^{\alpha}, 1}\equiv_{p} 1 \\
      d_{p^{\alpha}+1, p^{\alpha}} &\equiv_{p} d_{p^{\alpha}, p^{\alpha}-1} + d_{p^{\alpha}, p^{\alpha}}\equiv_{p} 1 \\
      d_{p^{\alpha}+1, i} &\equiv_{p} d_{p^{\alpha}, i-1} + d_{p^{\alpha}, i}\equiv_{p} 0 \quad \forall i \in \lbrace 2, \ldots, p^{\alpha} -1\rbrace \\
    \end{split}
  \end{displaymath}
  therefore row $p^{\alpha} + 1$ has one less coefficient multiple of $p$ 
  than row $p^{\alpha}$: $\left|\lbrace2,\ldots, p^{\alpha}-1 \rbrace\right|=
    \left|\lbrace1,\ldots, p^{\alpha}-1 \rbrace\right|-1$. 
    Since in row $p^{\alpha}$ there are $p^{\alpha}+1$ coefficients, where
  $p^{\alpha}-1$ of them are multiples of $p$, it follows that after $p^{\alpha}-1$ 
  rows there are no such coefficients at all, that happens at row $2\,p^{\alpha} -1$. 
  Formally, for any column index $c$:
  \begin{displaymath}
    \begin{split}
      {{2p^{\alpha} - 1} \choose {c}} &\equiv_{p} {{p^{\alpha} +(p^{\alpha}- 1)} \choose {c}} \\
      &\equiv_{p} {{p-1} \choose {c_0}} {{p-1} \choose {c_1}}{{p-1} \choose {c_2}} \ldots 
      {{p-1} \choose {c_{{\alpha}-1}}}{{1} \choose {0}} \\
      &\not\equiv_{p} 0
    \end{split}
  \end{displaymath}
  by representing $c=(c_{0}, \ldots,c_{\alpha-1})_{p}$ in base $p$, where
  $c_i \in \lbrace 0, \ldots, p-1 \rbrace$, for any $i\in\lbrace0,\ldots,\alpha-1\rbrace$.

  % the following derivation is simply a proof of coefficient extraction
  % using the definition of Riordan array for Pascal triangle, redundant.
  % Recall $\mathcal{P}$ is defined as the Riordan array :
  % \begin{displaymath}
  %   \mathcal{P} = \left(\frac{1}{1-t}, \frac{t}{1-t}  \right)
  % \end{displaymath}
  % hence the following derivation holds:
  % \begin{displaymath}
  %   \begin{split}
  %     {{p^n+r} \choose {c}} &\equiv_p [t^{p^n +r}]\frac{1}{1-t} \left(\frac{t}{1-t}\right)^c \\
  %     &\equiv_p [t^{p^n +r-c}](1-t)^{-(c+1)} \\
  %     &\equiv_p [t^{p^n +r-c}]\mathcal{G}\left\lbrace {{-(c+1)} \choose {k}}(-1)^k \right\rbrace_{k\in\mathbb{N}} \\
  %     &\equiv_p  {{-(c+1)} \choose {p^n +r-c}}(-1)^{p^n +r-c}  \\
  %     &\equiv_p  {{ p^n +r} \choose {p^n +r-c}} \left((-1)^{p^n +r-c}\right)^2  \\
  %     &\equiv_p  {{ p^n +r} \choose {c}}  \\
  %   \end{split}
  % \end{displaymath}

\end{proof}

  The \marginpar{for every $k\in\mathbb{N}$, row $p^{k}-1$ does not contain a coefficient multiple of $p$}
  last argument can be applied to every row with index of
  the form $p^k -1$, for each $k\in\mathbb{N}$:
  \begin{displaymath}
    \begin{split}
      {{p^k-1} \choose {c}} &\equiv_{p} {{p-1} \choose {c_0}} {{p-1} \choose {c_1}} \ldots 
      {{p-1} \choose {c_{{\alpha}-1}}}{{p-1} \choose {0}}\ldots{{p-1} \choose {c_{k-1}=0}} \not\equiv_{p} 0
    \end{split}
  \end{displaymath}

\subsection{On the number of \emph{zero-holes} within $\mathcal{P}^{(\alpha)}$}

Let us finish this section with the proof of an observation about
maximal upside down triangle of coefficients, each one of them 
multiple of $p$.

\begin{lemma}
    Let $\mathcal{P}_n$ be a chunk of $\mathcal{P}$ and let $j\in
    \lbrace 1, \ldots, p-1 \rbrace$. Between row index $j p^n$ and
    row index $(j+1)p^n -1$, there are $j$ maximal upside down triangles
    such that if a coefficient $d_{nk}$ belongs to one of them, then 
    $d_{nk} \equiv_{p} 0$.
\end{lemma}

\begin{proof}
    By absurd, choose any $j\in \lbrace 1, \ldots, p-1 \rbrace$ and show
    that between row index $j p^n$ and row index $(j+1)p^n -1$, 
    a maximal upside down triangle $\mathcal{T}$,
    such that $d_{nk}\in\mathcal{T} \rightarrow d_{nk} \equiv_{p} 0$,
    either is missing or is over. Proceed by cases:
    \begin{itemize}
        \item assume triangle $\mathcal{T}$ is missing, so there exists a coefficient
                $\tilde{d}_{nk}\in\mathcal{T}$ such that $\tilde{d}_{nk}\not\equiv_{p}0$, while all 
                other coefficient $d_{nk}\in\mathcal{T}$ satisfies $d_{nk} \equiv_{p}0$.
                Proceed by cases on parity of $j$:
            \begin{description}
                \item[$j=2k+1$ for some $k$] Without loss of generality, suppose $\mathcal{T}$ is the one in the 
                very middle, due to symmetry of $\mathcal{P}$ (the cases where it is on the left or 
                on the right are less interesting) and
                suppose that $\tilde{d}_{(j+1) p^n -2, (k+1)p^n -1}\not\equiv_{p}0$, the coefficient in the very bottom corner.
                But this is impossible because $d_{nk} = d_{n-1,k-1} + d_{n-1, k}$ if $d_{nk}\in \mathcal{P}$:
                \begin{displaymath}
                0\not\equiv_{p}\tilde{d}_{(j+1) p^n -2, (k+1)p^n -1} \equiv_{p} 
                    \tilde{d}_{(j+1) p^n -3, (k+1)p^n -1} + \tilde{d}_{(j+1) p^n -3, (k+1)p^n-2 }\equiv_{p}0
                \end{displaymath}

                \item[$j=2k$ for some $k$] Without loss of generality, suppose  $\mathcal{T}$ is the one on the very left. 
                Choose $r \in\lbrace 0,\ldots,p^n-2\rbrace$ and $c \in\lbrace r+1,\ldots,p^n-1\rbrace$,
                suppose that $\tilde{d}_{(j+1) p^n +r, c}\not\equiv_{p}0$. 
                By symmetry of $\mathcal{P}$, we get another contradiction: 
                \begin{displaymath}
                    0\not\equiv_{p}\tilde{d}_{(j+1) p^n +r, c} \equiv_{p} \tilde{d}_{(j+1) p^n +r, (j+1) p^n +r-c}\equiv_{p}0
                \end{displaymath}
            \end{description}
        \item assume triangle $\mathcal{T}$ is over, therefore the very first coefficient 
            $\tilde{d}$ in the top-left corner of $\mathcal{T}$ have to satisfy:
            \begin{displaymath}
                 \tilde{d} \equiv_{p} d_{j\,p^{n},j\,p^{n}+1}
            \end{displaymath}
            by structure of $\mathcal{P}$, coefficient $d_{j\,p^{n},j\,p^{n}+1}$ has no
            meaning so $\tilde{d}$ cannot satisfy the requested congruence.
    \end{itemize}
\end{proof}


\begin{theorem}
    In two adjacent chunks of $\mathcal{P}$, denote them by $\mathcal{P}_n$
    and $\mathcal{P}_{n+1}$ respectively, the
    number of upside down maximal triangles of coefficients $d_{nk}$, with
    $d_{nk} \equiv_p 0$, equals:
    \begin{displaymath}
        {{p}\choose{2}}
    \end{displaymath}
\end{theorem}

\begin{proof}
    Choose any $n\in\mathbb{N}$ and let $\mathcal{P}_n$ and $\mathcal{P}_{n+1}$
    be the two chunk of $\mathcal{P}$ of interest. Let $j\in\lbrace 1, \ldots, p-1 \rbrace$, so
    by previous lemma between row index $j p^n$ and
    row index $(j+1)p^n -1$, there are $j$ maximal upside down triangles, therefore consider the sum:
    \begin{displaymath}
        \sum_{i=1}^{p-1}{i} = \frac{(p-1)p}{2}
    \end{displaymath}
    as required.

\end{proof}
