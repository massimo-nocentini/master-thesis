
\section{Previous work}

In this section we recall existing literature about the application of modular
arithmetic to infinite lower triangular matrices. In particular, we report a
fundamental theorem about binomial coefficients, proved by
\citeauthor{lucas:theorie:des:nombres} in \cite{lucas:theorie:des:nombres}, and
how it has been used by various authors to establish new and curious results.
Finally, it allows us to apply a proof about the \emph{Sierpinski} gasket to 
\emph{Pascal} Riordan array.

\subsection{Lucas theorem exposed}

In \cite{fine:1947}, \citeauthor{fine:1947} exposes a theorem given by
\citeauthor{lucas:theorie:des:nombres}. We state it and report a proof since
it is a fundamental result for our work. 
\begin{theorem}[Lucas theorem]
    Let $p$ be a prime and $m,n\in\mathbb{N}$. Then:
    \begin{displaymath}
        {{m}\choose{n}} \equiv_{p} 
            {{m_{0}}\choose{n_{0}}} 
            {{m_{1}}\choose{n_{1}}} 
            \cdots 
            {{m_{k}}\choose{n_{k}}} 
    \end{displaymath}
    representing $m=\left(m_{0},m_{1},\ldots,m_{k}\right)_{p}$ and
    $n=\left(n_{0},n_{1},\ldots,n_{k}\right)_{p}$ in base $p$, for some $k\in\mathbb{N}$.
    So both $m_{j}$ and $n_{j}$ are in $\lbrace 0,\ldots, p-1 \rbrace$, for
    $j\in\lbrace 0,\ldots,k\rbrace$.
    \label{thm:lucas:theorem}
\end{theorem}

\begin{proof}
    Consider the sequence $\vect{\alpha}_{m}=\left\lbrace{{m}\choose{n}}\right\rbrace_{n\in\mathbb{N}}$ 
    of coefficients which counts the number of subset
    of a set composed by $m$ elements. Use sequence $\vect{\alpha}$ to define a \ac{fps}:
    \begin{displaymath}
        \begin{split}
            \sum_{n=0}^{m}{{{m}\choose{n}}\,t^{n}} &= \left(1+t\right)^{m}
                = \left(1+t\right)^{m_{0}+m_{1}p+\ldots+m_{k}p^{k}}\\
                &= \prod_{r=0}^{k}{{\left(1+t\right)^{m_{r}p^{r}}}}
                = \prod_{r=0}^{k}{\left(\left(1+t\right)^{p^{r}}\right)^{m_{r}}}
        \end{split}
    \end{displaymath}
    expansion of $\left(1+t\right)^{p^{r}}$ yield: 
    \begin{displaymath}
            \left(1+t\right)^{p^{r}} = \sum_{s=0}^{p^{r}}{{{p^{r}}\choose{s}}\,t^{s}}
                = 1+\sum_{s=1}^{p^{r}-1}{{{p^{r}}\choose{s}}\,t^{s}}+t^{p^{r}}
    \end{displaymath}
    application of congruence relation modulo $p$ makes the sum vanish since ${{p^{r}}\choose{s}}=p\,u_{s}$,
    for some $u_{s}\in\mathbb{N}$ and $s\in\lbrace 1,\ldots,p^{r}-1\rbrace$, therefore:
    \begin{displaymath}
        \left(1+t\right)^{p^{r}} = 1+t^{p^{r}}
    \end{displaymath}
    we can rewrite as:
    \begin{displaymath}
        \begin{split}
            \prod_{r=0}^{k}{\left(\left(1+t\right)^{p^{r}}\right)^{m_{r}}}
                &\equiv_{p} \prod_{r=0}^{k}{\left(1+t^{p^{r}}\right)^{m_{r}}}\\
                &\equiv_{p} \prod_{r=0}^{k}{\sum_{s_{r}=0}^{m_{r}}{{{m_{r}}\choose{s_{r}}}t^{s_{r}p^{r}}}}
        \end{split}
    \end{displaymath}
    \marginpar{abstracting a convolution introducing set $\Omega_{n}$}
    on the very \ac{rhs} there is a product of $k+1$ \ac{fps}, properly shifted according to coefficients
    $m_{0},m_{1},\ldots,m_{k}$. Such product yields a new \ac{fps} which satisfies:
    \begin{displaymath}
        \sum_{n=0}^{m}{{{m}\choose{n}}\,t^{n}} 
        \equiv_{p}
        \sum_{n=0}^{m}{\left(\sum_{\vect{\omega}\in\Omega_{n}}{\prod_{i=0}^{k}{{{m_{i}}\choose{\omega_{i}}}}}\right)\,t^{n}}
    \end{displaymath}
    where $\Omega_{n}=\left\lbrace\vect{\omega}=\left(\omega_{0},\ldots,\omega_{k}\right):
        \sum_{j=0}^{k}{\omega_{j}p^{j}}=n\right\rbrace$.
    By the representation theorem, there exists a \emph{unique} set of coefficients
    $\lbrace n_{0},n_{1},\ldots,n_{k}\rbrace$ such that $n=\sum_{j=0}^{k}{n_{j}p^{j}}$, therefore
    each set $\Omega_{n}$ contains only one element, namely $\vect{\omega}=(n_{0},\ldots,n_{k})$, so:
    \begin{displaymath}
        \equiv_{p} \sum_{n=0}^{m}{\left({\prod_{i=0}^{k}{{{m_{i}}\choose{n_{i}}}}}\right)\,t^{n}}
    \end{displaymath}
    equating coefficients the argument follows, as required.

\end{proof}

\citeauthor{fine:1947} uses such powerful theorem to prove additional facts
about binomial coefficients modulo a prime $p$. The following results concerns
part of our work and we report them without a complete proof.

\begin{theorem}
    Let $p$ be a prime, $m\in\mathbb{N}$ and $\Theta_{m}$ be a set of integers defined as: 
    \begin{displaymath}
        \Theta_{m}=\left\lbrace n\in\mathbb{N}: 
            {{m}\choose{n}}\not\equiv_{p}0\right\rbrace
    \end{displaymath}
    representing $m=\left(m_{0},m_{1},\ldots,m_{k}\right)_{p}$, then:
    \marginpar{this identity appears in \autoref{subsection:modular:alter:works} also,
    proved by \citeauthor{broomhead:1972}}
    \begin{displaymath}
        \left|\Theta_{m}\right|=\prod_{r=0}^{k}{(m_{r}+1)}
    \end{displaymath}
    \label{thm:fine:divisibility:binomial:coeffs:on:row}
\end{theorem}

Previous little theorem together with \autoref{thm:lucas:theorem} 
allow \citeauthor{fine:1947} to prove the following arguments.

\begin{theorem}
    Let $p$ be a prime, $m\in\mathbb{N}$ and $\Theta_{m}$ 
    be a set of integers defined as before. Then:   
    \begin{displaymath}
        \left|\Theta_{m}\right|=2\leftrightarrow m=p^{u}
    \end{displaymath}
    for some $u\in\mathbb{N}$.
\end{theorem}

Which can be read as: \flqq $m$ being a power of $p$ is a necessary and 
sufficient condition for coefficients ${{m}\choose{n}}$, for $n\in\lbrace1,\ldots,m-1\rbrace$, 
    to be multiples of $p$\frqq.
\marginpar{such rows can be found in \autoref{fig:pascal-standard-ignore-negatives-centered-colouring-127-rows-mod2-partitioning-triangle}}
\begin{theorem}
    Let $p$ be a prime, $m\in\mathbb{N}$ and $\Theta_{m}$ 
    be a set of integers defined as before.
    If $m=\left(m_{0},m_{1},\ldots,m_{k}\right)_{p}$, for some $k\in\mathbb{N}$, then:
    \begin{displaymath}
        \left|\Theta_{m}\right|=p^{k+1}\leftrightarrow m=p^{k+1}-1
    \end{displaymath}
    which is the same to say $m_{j}=p-1$, for each $j\in\lbrace 0,\ldots,k\rbrace$.
\end{theorem}

Again, last theorem can be read as: \flqq $m$ being one less a power of $p$ is
a necessary and sufficient condition for coefficients ${{m}\choose{n}}$ to be
\emph{not} multiples of $p$, where $n\in\lbrace0,\ldots,m\rbrace$\frqq.


\subsection{Divisibility--with visibility}

In \cite{sved:1988}, \citeauthor{sved:1988} gives solid bases to understand
divisibility properties of some \emph{counting numbers}, that is coefficients
occurring in combinatorics, such as binomials and Stirling numbers. Many of
such coefficients shows divisibility structures of remarkable design. 

She starts working on the Pascal triangle, namely the lower triangular infinite
matrix defined inductively by the recurrence: 
\begin{equation}
    {{n}\choose{k}}={{n-1}\choose{k-1}}+{{n-1}\choose{k}}
    \label{eq:binomial:recurrence}
\end{equation}
and applies a modification to build the Pascal array modulo a prime $p$, 
reporting some interesting tables for some values of $p$.

We believe that one of the core concepts she introduces is a little language
that allows us to characterize these arrays and we adopt it in the next chapter
as well. In the following paragraphs we describe this language. 
\\\\
Let $\mathcal{M}=\lbrace m_{nk}\rbrace_{n,k\in\mathbb{N}}$ 
\marginpar{a little language for modular infinite arrays}
be an infinite lower triangular array, where $m_{nk}=0$ if 
$n<k$. Choose a prime $p$, then build a modular array
$\lbrace m_{nk}\mod p\rbrace_{n,k\in\mathbb{N}}$ denoted by $\mathcal{M}_{\equiv_{p}}$.

A \emph{principal cell} is a chunk of $\mathcal{M}_{\equiv_{p}}$ denoted by:
\begin{displaymath}
    \mathcal{M}_{\equiv_{p}}^{\bigtriangleup} = 
        \left\lbrace m_{nk}\in \mathcal{M}_{\equiv_{p}}: n\in\lbrace 0,\ldots,p-1\rbrace\right\rbrace
\end{displaymath}

A \emph{cell} is a copy of the \emph{principal cell} with each entry multiplied 
by the same constant $\beta\in\mathbb{N}$, formally:
\begin{displaymath}
    \left\lbrace \beta\,m_{nk}\mod p:m_{nk}\in \mathcal{M}_{\equiv_{p}}^{\bigtriangleup}\right\rbrace
\end{displaymath}

A \emph{principal cluster} of order $\alpha$ is a chunk of $\mathcal{M}_{\equiv_{p}}$ denoted by:
\begin{displaymath}
    \mathcal{M}_{\equiv_{p}}^{(\alpha)} = 
        \left\lbrace m_{nk}\in \mathcal{M}_{\equiv_{p}}: n\in\lbrace 0,\ldots,p^{\alpha}-1\rbrace\right\rbrace
\end{displaymath}

A \emph{cluster} of order $\alpha$ is a copy of the \emph{principal cluster}
of order $\alpha$ with each entry multiplied 
by the same constant $\beta\in\mathbb{N}$, formally:
\begin{displaymath}
    \left\lbrace \beta\,m_{nk}\mod p:m_{nk}\in \mathcal{M}_{\equiv_{p}}^{(\alpha)}\right\rbrace
\end{displaymath}

Finally, a \emph{zero-hole} of order $\alpha$ is an inverted triangular array of
\marginpar{as we will see, modular Catalan array has upside-down zero-holes, placed
    somewhere on the right of the first column, have a look at
    \autoref{fig:catalan-traditional-standard-ignore-negatives-centered-colouring-127-rows-mod2-partitioning-triangle}}
coefficients multiples of $p$, from here the name, consisting of $p^{\alpha}-1$
coefficients lying on the first row, $p^{\alpha}-2$ coefficients on the second row
and a single coefficient on the $(p^{\alpha}-1)$-th row. 
Observe that a \emph{zero-hole} of order $\alpha$ has $p^{\alpha}-1$ rows, one less
than a \emph{principal cluster} of the same order.

Although a formal proof can be stated by induction using
\autoref{eq:binomial:recurrence}, the following fact about the modular Pascal
array has an easier formulation using the proposed language.

\begin{theorem}
    Let $p$ be a prime and $\mathcal{P}_{\equiv_{p}}$ be the 
    modular Pascal array. The principal cluster $\mathcal{P}_{\equiv_{p}}^{(\alpha+1)}$,
    of order $\alpha + 1$, consists of $p$ layers of clusters of order $\alpha$, alternating
    with zero-holes of order $\alpha$ too. The first layer containing only the principal
    cluster of order $\alpha$.
\end{theorem}
\quad
\\\\
A second core concepts she introduces is a \emph{graphical interpretation} of
\marginpar{a graphical interpretation of Lucas theorem}
\autoref{thm:lucas:theorem}. We follow her original explanation in the
following exposition: first of all, observe that within
$\mathcal{P}_{\equiv_{p}}^{(\alpha+1)}$ the \emph{head} of the $(c+1)$-th cluster
of order $\alpha$ in the $(r+1)$-th layer is ${{r}\choose{c}}\mod p$.

Consider $\mathcal{P}_{\equiv_{p}}^{(\alpha+1)}$ and introduce 
two indices, $n$ and $k$, running over rows and columns, respectively.
Expand both of them in base $p$, so $n=(n_{0},\ldots,n_{\alpha})_{p}$ and 
$k=(k_{0},\ldots,k_{\alpha})_{p}$. Proceed by cases:
\begin{itemize}

    \item $k_{i}\leq n_{i}$, for all $i\in\lbrace0,\ldots,\alpha\rbrace$: in
    this case $m_{nk}\in\mathcal{P}_{\equiv_{p}}^{(\alpha+1)}$ is in a
    \emph{nest of clusters}.  Coefficients of order $\alpha$ state that
    $m_{nk}$ is inside the $(k_{\alpha}+1)$-th cluster
    $\mathcal{C}^{k_{\alpha}}$, of order $\alpha$, of the $(n_{\alpha}+1)$-th
    layer.  Within cluster $\mathcal{C}^{k_{\alpha}}$ there are $p$ layers of
    clusters of order $\alpha-1$ and coefficients of order $\alpha-1$ state
    that $m_{nk}$ is inside the $(k_{\alpha-1}+1)$-th cluster
    $\mathcal{C}^{k_{\alpha-1}}$ of the $(n_{\alpha-1}+1)$-th layer of cluster
    $\mathcal{C}^{k_{\alpha}}$. Using coefficients of orders
    $\alpha-2,\alpha-3,\ldots,0$ repeatedly, we finally arrive to the
    \emph{cell} containing $m_{nk}$, which is the $(k+1)$-th element of the
    $(n+1)$-th row of $\mathcal{P}_{\equiv_{p}}^{(\alpha+1)}$.

    Now, the head of cluster $\mathcal{C}^{k_{\alpha}}$ is
    ${{n_{\alpha}}\choose{k_{\alpha}}}\mod p$ therefore, by definition, cluster
    $\mathcal{C}^{k_{\alpha}}$ is ${{n_{\alpha}}\choose{k_{\alpha}}}\mod p$
    times the copy of $\mathcal{P}_{\equiv_{p}}^{(\alpha)}$.  Using the same
    argument, the head of cluster $\mathcal{C}^{k_{\alpha-1}}$ is
    ${{n_{\alpha}}\choose{k_{\alpha}}}{{n_{\alpha-1}}\choose{k_{\alpha-1}}}\mod
    p$ therefore, by definition, cluster $\mathcal{C}^{k_{\alpha-1}}$ is
    ${{n_{\alpha}}\choose{k_{\alpha}}}{{n_{\alpha-1}}\choose{k_{\alpha-1}}}\mod
    p$ times the copy of $\mathcal{P}_{\equiv_{p}}^{(\alpha-1)}$.

    Continuing in this manner, elements of the \emph{cell} containing $m_{nk}$
    are ${{n_{\alpha}}\choose{k_{\alpha}}}{{n_{\alpha-1}}\choose
    {k_{\alpha-1}}}\ldots{{n_{1}}\choose{k_{1}}}\mod p$ times the elements of 
    the \emph{principal cell} $\mathcal{P}_{\equiv_{p}}^{\bigtriangleup}$, whose
    \emph{head} is ${{n_{0}}\choose{k_{0}}}$. 

    \item let $i$ be the greater index such that $n_{i}<k_{i}$: $m_{nk}$ cannot
    be in the $(k_{i}+1)$-th cluster, of order $i$, of the $(n_{i}+1)$-th layer
    of cluster $\mathcal{C}^{k_{i+1}}$, since such cluster does not exist,
    therefore $m_{nk}$ falls into the \emph{zero-hole} of order $i$ to the
    right of $\mathcal{C}^{k_{i+1}}$.

\end{itemize}
\quad
\\\\
\citeauthor{sved:1988} continues to apply modular transformation to triangles
of Stirling and Gaussian numbers and finishes with an advice about how modular
transformation using \emph{higher powers of primes} can be done.

\subsection{McLean, Broomhead and Wolfram works}
\label{subsection:modular:alter:works}
Although not stricly necessary for our work, we would like to recall some
additional results. Several other authors study interesting aspects when
modular transformation is applied to infinite lower triangular matrices.

Consider again the modular Pascal array $\mathcal{P}_{\equiv_{p}}$.
\citeauthor{broomhead:1972}, in \cite{broomhead:1972}, shows that:
\begin{displaymath}
    N(n)=\sum_{k=0}^{n}{\mathbb{1}_{\left\lbrace p\text{ does not divide }{{n}\choose{k}}\right\rbrace}}
\end{displaymath}
where $\mathbb{1}_{\lbrace \gamma \rbrace}=1$ if and only if predicate $\gamma$ is true,
has a closed form:
\begin{displaymath}
    N((n_{0},n_{1},\ldots,n_{m})_{2})=(n_{0}+1)(n_{1}+1)\cdots(n_{m}+1)
\end{displaymath}
which confirms \autoref{thm:fine:divisibility:binomial:coeffs:on:row}, stated in different words.
Moreover, he attempts a system that is quite similar to the language introduced by \citeauthor{sved:1988},
although with an operational nature instead of a declarative one as her language has.

\citeauthor{mclean:1974}, in \cite{mclean:1974}, works on modular transformations using
higher powers of primes:
\marginpar{The method of \citeauthor{sved:1988} is studied in depth by \citeauthor{mclean:1974}}
\begin{itemize}
    \item defines function $\nu_{p}$, called \emph{$p$-index}, which satisfies the following relation,
        for some $t\in\mathbb{N}$: 
        \begin{displaymath}
            \nu_{p}\lbrace a\rbrace =t \leftrightarrow p^{t}\mid a \wedge p^{t+1}\nmid a
        \end{displaymath}
    \item proves that {$\nu_{p}\left\lbrace {{n}\choose{k}}\right\rbrace $} 
        equals the \emph{number of carries} in $n-_{p}k$
\end{itemize}

Finally, \citeauthor{wolfram:1984} in \cite{wolfram:1984}:
\marginpar{\citeauthor{wolfram:1984} applies some ideas about the theory of fractals
    introduced by \citeauthor{mandlebrot:1978} in \cite{mandlebrot:1978}}
\begin{itemize}
    \item finds the \emph{fractal dimensionality} of self-similar patterns in $\mathcal{P}_{\equiv_{2}}$,
        solving recurrence $T\left(\frac{i}{2}\right)=3\,T(i)$ yield $T(i)\approx i^{-\log_{2}{3}}$
    \item observes $N(n)=\sum_{k=0}^{n}{\mathbb{1}_{\left\lbrace{{n}\choose{k}}\equiv_{2}1\right\rbrace}}$ 
        is a very irregular function. Also $N(n)=2^{\phi_{1}(n)}$, where $ \phi_{1}((n_{0},n_{1},\ldots,n_{m})_{2})=
            \sum_{j=0}^{m}{\mathbb{1}_{\lbrace n_{j}=1\rbrace} }$
\end{itemize}
