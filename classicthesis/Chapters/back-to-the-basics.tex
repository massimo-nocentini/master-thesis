
\chapter{Riordan arrays: a chronological development}
\label{ch:chronologial:development}

In this chapter we would like to introduce some important mathematical objects
necessary to tackle topics in later chapters. Such objects are the foundations
of the \emph{Riordan group theory} and we would like to describe them in our
words, after a quick review of fundamental concepts, such as sequences,
\ac{fps} and \ac{gf}.  We aim at descriptions that point out why a particular
concept is important and how it allows generalizations and further
developments. Because of this approach, our prose will be longer than
traditional ones.

%In the first part we enter into the playground, where we will recall concepts
%according chronological and dependency order; in the last part we describe our
%motivations and ideas underlying this work. 

So, start the journey!

\section{Basic concepts}


In this section we recall some fundamental concepts that are essential building
blocks for the treatment that follows. First of all we denote a sequence
$\vect{\omega}$ of \emph{countable} integers with:
\marginpar{sequences and matrices}
\begin{displaymath}
    \vect{\omega}=\lbrace \omega_{n} \rbrace_{n\in\mathbb{N}}
\end{displaymath}
where term $\omega_{n}$ is called the \emph{generic coefficient} of
sequence $\vect{\omega}$, which can be either simply a costant $\omega_{n}\in\mathbb{Z}$
or a function in the variable $n$, denoting a rule for building each
coefficient belonging to sequence $\vect{\omega}$.

When a sequence depends on \emph{two} indices, we denote it with:
\begin{displaymath}
    \lbrace \omega_{nk} \rbrace_{n,k\in\mathbb{N}}
\end{displaymath}
which is usually called an \emph{infinite matrix}.

We will be interested to handle coefficients in a sequence $\vect{\omega}$,
using them to build \ac{fps} over the ring $\mathbb{Z}[\![t]\!]$, as follows:
\begin{equation}
    \sum_{n\geq0}{\omega_{n}\,t^{n}}
    \label{eq:fps:formally}
\end{equation}
where $t$ is an undeterminate variable, it usually does \emph{not} take
any value in order to \emph{evaluate} the sum at a given point. 
Moreover, such \ac{fps} is the Taylor expansion of the \ac{gf}
$\Omega$ for sequence $\vect{\omega}$, formally:
\marginpar{\ac{fps} $\Omega$ over a sequence $\vect{\omega}$}
\begin{equation}
    \sum_{n\geq0}{\omega_{n}\,t^{n}} = \sum_{i\geq0}{\frac{1}{i!}\frac{
        \partial^{i}\Omega(0)}{\partial\,t^{i}}t^{i}}
    \label{eq:fps:gf:relation}
\end{equation}
where function $\Omega$ satisfies:
\begin{displaymath}
    \Omega(t) = \mathcal{G}_{t}\left(\lbrace\omega_{n}\rbrace_{n\in\mathbb{N}}\right)
\end{displaymath}
denoting with $\mathcal{G}_{t}$ the operator which consumes a sequence
$\vect{\omega}$ and produces the function $\Omega$, in the undeterminate
variable $t$, satisfying \autoref{eq:fps:gf:relation}. From now on, we
omit the subscript $_{n\in\mathbb{N}}$ when ambiguities do not arise 
in the context where it should appear.

We don't care about questions of convergence in infinite series, such as in
\marginpar{\ac{fps} convergence doesn't matter}
\autoref{eq:fps:formally}, since a systematic theory of \ac{fps} has been
proposed by \citeauthor{niven:1969} in \cite{niven:1969} from the theoretical
point of view, and by \citeauthor{riordan:1964}, in \cite{riordan:1964}, for
the combinatorial aspect.  Those theories do not involve any question of
convergence or divergence; on the other hand they provide some
characterizations of solutions of functional equations, how coefficients
attached to the same power of an underterminate variable can be equated, how
\ac{fps} can be multiplied or derived and so on. Additional treatment of \ac{gf}
using this kind of approach can be found in \cite{Wilf:2006:GEN:1204575}.
\\\\
We report a simple example involving two basic sequences, first the one
composed of all $1$s, namely $\vect{1}=\lbrace1\rbrace_{n\in\mathbb{N}}$; second, the
one composed of natural numbers, namely $\vect{n}=\lbrace n\rbrace_{n\in\mathbb{N}}$.
We start recalling an important result about sequences.
\begin{theorem}[Identity Principle]
    Let $\vect{\omega}=\lbrace\omega_{n}\rbrace_{n\in\mathbb{N}}$ and 
        $\vect{\theta}=\lbrace\theta_{n}\rbrace_{n\in\mathbb{N}}$ be two sequences.
        Then:
    \begin{displaymath}
        \omega_{n}=\theta_{n}\leftrightarrow\mathcal{G}_{t}\left(\vect{\omega}\right)
            =\mathcal{G}_{t}\left(\vect{\theta}\right)
    \end{displaymath}
    \label{thm:identity:principle}
    for each $n\in\mathbb{N}$.
\end{theorem}
In order to find the \ac{gf} $\Omega$ of sequence
\marginpar{an example: computing \ac{gf} of sequence $\vect{1}=\lbrace1\rbrace_{n\in\mathbb{N}}$}
$\vect{\omega}=\lbrace\omega_{n}\rbrace_{n\in\mathbb{N}}
=\lbrace1\rbrace_{n\in\mathbb{N}}=\vect{1}$ we reason as follows.  Define a sequence
$\vect{\theta}=\lbrace\theta_{n}\rbrace_{n\in\mathbb{N}}=\lbrace\omega_{n+1}\rbrace_{n\in\mathbb{N}}$.
Since $\omega_{n}=\theta_{n}$ because $\omega_{n}=\omega_{n+1}=1$, for each
$n\in\mathbb{N}$, then it is possible to apply
\autoref{thm:identity:principle}, so:
\begin{displaymath}
    \mathcal{G}_{t}\left(\lbrace\omega_{n}\rbrace\right) 
        =\mathcal{G}_{t}\left(\lbrace\omega_{n+1}\rbrace\right)
        =\frac{\mathcal{G}_{t}\left(\lbrace\omega_{n}\rbrace\right)-\omega_{0}}{t}
        =\frac{\mathcal{G}_{t}\left(\lbrace\omega_{n}\rbrace\right)-1}{t}
\end{displaymath}
where the middle step hold because $\lbrace\omega_{n+1}\rbrace_{n\in\mathbb{N}}$ is
just sequence $\vect{\omega}$ shifted by one position. A simple manipulation yield:
\begin{equation}
    \Omega(t)=\mathcal{G}_{t}\left(\vect{1}\right) =\frac{1}{1-t}
    \label{eq:1:gf}
\end{equation}
which we are going to expand as \ac{fps}. Taylor expansion of function $\Omega$
yields:
\begin{displaymath}
    \Omega(t)=1+t+t^{2}+t^{3}+t^{4}+t^{5}+\mathcal{O}(t^{6})
\end{displaymath}
as required.

In order to find the \ac{gf} $N$ of sequence $\vect{n}=\lbrace
n\rbrace_{n\in\mathbb{N}}$, namely the sequence of natural numbers, observe
\marginpar{an example: computing \ac{gf} of sequence $\vect{n}=\lbrace n\rbrace_{n\in\mathbb{N}}$}
that sequence $\vect{n}$ can be written, more verbosly, as $\vect{n}=\lbrace
n\cdot1\rbrace_{n\in\mathbb{N}}$, which has shape $\lbrace
n\cdot\omega_{n}\rbrace_{n\in\mathbb{N}}$, where $\omega_{n}=1$, for each
$n\in\mathbb{N}$, so $\vect{\omega}=\vect{1}$. Therefore:
\begin{equation}
    N(t)=\mathcal{G}_{t}\left(\vect{n}\right) =\frac{t}{(1-t)^{2}}
    \label{eq:n:gf}
\end{equation}
because $\mathcal{G}_{t}\left(\lbrace n\omega_{n}\rbrace\right) =t\,
\frac{\partial\,\mathcal{G}_{t}\left(\lbrace
\omega_{n}\rbrace\right)}{\partial\,t}$, in our case
$\mathcal{G}_{t}\left(\lbrace n\cdot1\rbrace\right) =
t\,\frac{\partial\,\mathcal{G}_{t}\left(\vect{1}\right)}{\partial\,t}$.
Again, do a little check, expanding function $N$ as a \ac{fps}:
\begin{displaymath}
    N(t)=t+2t^{2}+3t^{3}+4t^{4}+5t^{5}+\mathcal{O}(t^{6})
\end{displaymath}
as required.
\\\\
Another important concept that will be used in the definition of
\marginpar{operator $\left[t^{n}\right]$: the coefficient extractor}
\emph{Riordan arrays} is the $\left[t^{n}\right]$ operator, where
$t$ is an undeterminate function and $n\in\mathbb{N}$, also called
\emph{coefficient extractor}. This operator is defined by the following
relation:
\begin{equation}
    \left[t^{n}\right]\mathcal{G}_{t}\left(\lbrace\omega_{n}\rbrace\right) = \omega_{n}
    \label{eq:coefficient:extractor:def}
\end{equation}
for some sequence $\vect{\omega}=\lbrace\omega_{n}\rbrace_{n\in\mathbb{N}}$. More properties
of $\left[t^{n}\right]$ operator can be found in \cite{merlini:method:of:coefficient}.



% put here something about the method of coefficient
%\cite{merlini:method:of:coefficient}



\section{Understanding the playground}


\subsection{Catalan triangle, by Shapiro}

The very first work that allows researchers to develop the theory of
\emph{Riordan arrays} is the one by \citeauthor{shapiro:1976}. 
In \cite{shapiro:1976}, he builds the following triangle, called
\emph{Catalan} because of its relation with \emph{Catalan} numbers in the first
column:
\begin{equation}
    \begin{array}{c|ccccccc}
        n\diagdown k & 1 & 2 & 3 & 4 & 5 & 6&\ldots\\
        \hline
        1&1 & & & & & &\\
        2&2 &1 & & & & &\\
        3&5 &4 &1 & & & &\\
        4&14 &14 &6 &1 & & &\\
        5&42 &48 &27 &8 &1 & &\\
        6&132 &165 &110 &44 &10 &1& \\
        \vdots&\vdots&\vdots&\vdots&\vdots&\vdots&\vdots&\ddots \\
    \end{array}
    \label{eq:shapiro:catalan:triangle:expanded}
\end{equation}

Such a triangle counts the pairs of \emph{non-intersecting} paths of length $n$
with distance $k$, in an integral lattice domain. This interpretation would
allow us to give meaning to other triangles, 
%one of them is generated by the \emph{Riordan array} $\mathcal{C}$, 
therefore we formalize it a little more.

A path $\vect{\rho}$ of length $n$ is a sequence $(\rho_{0},\rho_{1},\ldots,\rho_{n-1})$,
where each $\rho_{i}$ is a pair $(a_{i},b_{i})$ of natural numbers, such that:
\marginpar{Path definition, \emph{upwards} and \emph{rightward} steps}
\begin{displaymath}
    (\rho_{i},\rho_{i+1})\subset\vect{\rho} \leftrightarrow
    (\rho_{i+1}=(a_{i}+1,b_{i})) \vee 
    (\rho_{i+1}=(a_{i},b_{i}+1)) 
\end{displaymath}
the first term in the \ac{rhs} denotes a \emph{vertical} step, while the second
term denotes an \emph{horizontal} step. Two paths $\vect{\rho}$ and
$\vect{\sigma}$ \emph{intersect} after $j$ steps if $\rho_{j}=\sigma_{j}$.
Moreover, they have distance $d$ if
$\left|\rho_{n-1}^{(1)}-\sigma_{n-1}^{(1)}\right|=d$, where $\gamma_{j}^{(1)}$
is the first component of a generic pair $\gamma_{j}$, for each
$j\in\lbrace0,\ldots,n-1\rbrace$.

\citeauthor{shapiro:1976} observes a recurrence relation over elements in the triangle.
Denote with $d_{nk}$ such an element, then the following holds:
\marginpar{$A(t)=1+2t+t^{2}$ is the $A$-sequence of this triangle, as we will soon see}
\begin{displaymath}
    d_{nk}=d_{n-1,k-1}+2\,d_{n-1,k}+d_{n-1,k+1}
\end{displaymath}
Using this recurrence, he proposes a closed formula for element $d_{nk}$, found by 
``trial and error'' as he says:
\begin{displaymath}
    d_{nk}=\frac{k}{n}{{2n}\choose{n-k}}
\end{displaymath}
\emph{verifying} that it satisfies the given characterizing recurrence, giving
no \emph{constructive} proof at all. Since the $j$-th Catalan number $c_{j}$ is $d_{j1}$,
another closed formula for $c_{j}$ pops out:
\begin{displaymath}
    c_{j}=\frac{1}{j}{{2j}\choose{j-1}}
\end{displaymath}
and, by combinatorial interpretation of $d_{nk}$, $c_j$ is the number of pairs
of \emph{non-intersecting} path of length $j$ with distance $1$. Using path definition,
and the \emph{intersect} property on a pair of paths, it is possible to give another
combinatorial interpretation to $c_{j}$:
\marginpar{a combinatorial interpretation for the Catalan numbers based on non-intersecting paths}
\begin{quote}
    \emph{$c_j$ counts pairs of paths that intersect, \\
    \indent for the first time, at step $j+1$ }
\end{quote}
\begin{proof}
    Let $\vect{\rho}$ and $\vect{\sigma}$ be two paths. Proceed by 
    induction on length $j$:
    \begin{itemize}
        \item base $j=1$. Since paths are \emph{non-intersecting}, without loss
            of generality, let $\vect{\rho}=((1,0))$ and $\vect{\sigma}=((0,1))$.
            There is only one combination to have $\vect{\rho}$ and $\vect{\sigma}$
            intersect at step $2$, namely $\vect{\rho}$ has to move upwards and
            $\vect{\sigma}$ has to move rightwards. Therefore there are $c_{1}=1$ pairs
            of paths that intersect, for the \emph{first} time, at step $2$; 
        \item assume that $c_j$ counts pairs of paths that intersect, 
            for the \emph{first} time, at step $j+1$; then show for $c_{j+1}$. 
            According to this hypothesis, paths $\vect{\rho}$ and $\vect{\sigma}$ 
            do not intersect until step $j+1$, therefore one of them is \emph{above}
            the other one for the first $j$ steps. 
            Without loss of generality, let $\vect{\rho}$ be the higher one. So, 
            $\vect{\rho}$ can be written, for some natural numbers $a$ and $b$, as:
            \begin{displaymath}
                \vect{\rho}=(\rho_{0},\rho_{1},\ldots,\rho_{j-2},(a, b))
            \end{displaymath}
            and $\vect{\sigma}$ can be written as:
            \begin{displaymath}
                \vect{\sigma}=(\sigma_{0},\sigma_{1},\ldots,\sigma_{j-2},(a+1, b-1))
            \end{displaymath}
            Do one more step which keeps distance $1$. Both paths have to move
            either upwards, and in this case they can be written as:
            \begin{displaymath}
                \begin{split}
                    \vect{\rho}&=(\rho_{0},\rho_{1},\ldots,\rho_{j-2},(a, b),(a, b+1))\\
                    \vect{\sigma}&=(\sigma_{0},\sigma_{1},\ldots,\sigma_{j-2},(a+1,b-1), (a+1, b))
                \end{split}
            \end{displaymath}
            or rightwards, and in this case they can be written as:
            \begin{displaymath}
                \begin{split}
                    \vect{\rho}&=(\rho_{0},\rho_{1},\ldots,\rho_{j-2},(a, b),(a+1, b))\\
                    \vect{\sigma}&=(\sigma_{0},\sigma_{1},\ldots,\sigma_{j-2},(a+1,b-1),(a+2, b-1))
                \end{split}
            \end{displaymath}
            in both cases, there is only one combination that make $\vect{\rho}$ and $\vect{\sigma}$
            intersect at step $j+2$, namely $\vect{\rho}$ has to move rightward and $\vect{\sigma}$
            has to move upwards. Therefore $c_{j+1}$ counts pairs of paths that intersect,
            for the \emph{first} time, at step $j+2$, as required.
            
                 
    \end{itemize}
\end{proof}

Another interesting observation of \citeauthor{shapiro:1976} is about the computation
of \emph{row sums}. Choose a row $n$, then: 
\marginpar{row sums: $s_{n}$ is the number of \emph{non-intersecting} pairs of 
    paths of length $n$, no matter the distance between them}
\begin{displaymath}
    \sum_{k=1}^{n}{d_{nk}}=s_{n}
\end{displaymath}
where $s_{n}\in\mathbb{N}$ which satisfies:
\begin{displaymath}
    s_{n}=\frac{1}{2}{{2n}\choose{n}}
\end{displaymath}

Again, \citeauthor{shapiro:1976} \emph{verifies} previous relation against the
following recurrence, without giving a \emph{constructive} proof:
\begin{displaymath}
    s_{n}=4\,s_{n-1}-c_{n-1}
\end{displaymath}
\marginpar{\emph{verifying} against a recurrence relation, again}
The recurrence holds because $s_{n}$ denotes the number of
\emph{non-intersecting} paths of length $n$ and, since two paths can be
extended in $4$ ways, then we've to remove all of them that intersect after the
extension, which are $c_{n-1}$ in number.

Yet another interesting observation is the following recurrence:
\begin{displaymath}
    d_{nk}=\sum_{j=1}^{n-k+1}{d_{n-j,k-1}c_{j}}
\end{displaymath}

\begin{proof}
    If two \emph{non-intersecting} paths have distance $k$ then there
    exists $j\in\mathbb{N}$ such that, at step $n-j$, they have distance 
    $k-1$ for the last time. For the remaining $j$ steps, starting from
    step $n-j+1$ to step $n$, the two paths have to be extended so as to keep
    distance at least $k$ and, at step $n$, distance has to be $k$ exactly.
    One way to do this extensions is to apply the same sequence of steps to both paths,
    which keep distance $k$ for all the remaining $j$ steps, which is the same to say
    that paths can be extended in $c_{j}$ ways.
\end{proof}

Previous recurrence is important when it is used to characterize columns. Take the second
column, so let $k=2$:
\marginpar{$2$-fold convolution of function $C$ with itself}
\begin{displaymath}
    d_{n,2}=\sum_{j=1}^{n-1}{d_{n-j,1}c_{j}}
           =\sum_{j=1}^{n-1}{c_{n-j}c_{j}}
           =[t^{n}]C(t)^{2}-c_{0}c_{n}
           =[t^{n}]C(t)^{2}
\end{displaymath}
where $C(t)=t+2t^{2}+5t^{3}+14t^{4}+42t^{5}+132t^{6}+\mathcal{O}(t^{7})$ is the
\ac{fps} for the Catalan numbers, shifted by one place, so $c_{0}=0$.

Take the third column, so let $k=3$:
\marginpar{$3$-fold convolution of function $C$ with itself}
\begin{displaymath}
    \begin{split}
        d_{n,3}&=\sum_{j=1}^{n-2}{d_{n-j,2}c_{j}}
               =\sum_{j=1}^{n-2}{\left(\sum_{i=1}^{n-j-1}{d_{n-j-i,1}c_{i}}\right)c_{j}}\\
               &=\sum_{j=1}^{n-2}{\sum_{i=1}^{n-j-1}{c_{n-j-i}c_{i}}c_{j}}
               =\sum_{i+j+l=n}{c_{i}c_{j}c_{l}}
               =[t^{n}]C(t)^{3}
    \end{split}
\end{displaymath}

So, take any column $k$, then coefficients lying on it are coefficients
of the $k$-fold convolution of function $C$ with itself.
\begin{equation}
    d_{nk}=\sum_{i_{1}+\ldots+i_{k}=n}{c_{i_{1}}\cdots c_{i_{k}}}
    \label{eq:shapiro:catalan:triangle:convolution:for:generic:coefficient}
\end{equation}
\\\\
Finally, \citeauthor{shapiro:1976} asks the following question:
\begin{quote}
    \emph{Is there a theory of arithmetic triangles where a simple function
    of the generating function (or \ac{fps}) of the first column yields
    the generating function (or \ac{fps}) of the $n$-th column?}
\end{quote}
Rogers found such a theory, as we will see in the next section.












\subsection{Renewal arrays, by Rogers}
\label{sec:back:to:the:basics:rogers}

\citeauthor{rogers:1977}, in \cite{rogers:1977}, proposes a theory
as a solution to the question left open by \citeauthor{shapiro:1976}, 
introducing the concept of \emph{renewal arrays}.

In order to understand his work, we formalize first the concept of $k$-fold
convolution of a sequence $\lbrace s_{i}\rbrace_{i\in\mathbb{N}}$ 
with itself and, second, the concept of \emph{renewal arrays}.
\\\\
Let $\lbrace \alpha_{i}\rbrace_{i\in\mathbb{N}}$ and 
$\lbrace \beta_{i}\rbrace_{i\in\mathbb{N}}$ be two sequences
\marginpar{to not clutter notations, we discard subscript in 
notation $\lbrace \alpha_{i} \rbrace$ for sequences}. Denote
with $\cdot$ the \emph{convolution} operator which consumes a pair
of sequences and produces a new sequence, written in \emph{infix} 
notation:
\begin{displaymath}
    \lbrace \alpha_{i}\rbrace_{i\in\mathbb{N}}\cdot
    \lbrace \beta_{i}\rbrace_{i\in\mathbb{N}} =
    \lbrace \theta_{i}\rbrace_{i\in\mathbb{N}}
\end{displaymath}
where sequence  $\lbrace \theta_{i}\rbrace_{i\in\mathbb{N}}$ satisfies:
\begin{displaymath}
    \theta_{n}=\sum_{j=0}^{n}{\alpha_{j}\beta_{n-j}}
\end{displaymath}

The $k$-fold convolution of a sequence $\lbrace
\theta_{i}\rbrace_{i\in\mathbb{N}}$ \marginpar{$k$-fold convolution, precisely}
with itself, denoted with  $\left\lbrace
\theta_{i}^{(k)}\right\rbrace_{i\in\mathbb{N}}$, is defined recursively as:
\begin{displaymath}
    \left\lbrace\theta_{n}^{(k+1)}\right\rbrace= \left\lbrace\theta_{n}^{(k)}\right\rbrace\cdot
        \lbrace\theta_{n}\rbrace
\end{displaymath}
with boundary condition $\left\lbrace\theta_{n}^{(0)}\right\rbrace=(1,\underbrace{0,0,\ldots}_{\text{infinitely many zeros}})$ so as:
\begin{displaymath}
    \left\lbrace\theta_{n}^{(1)}\right\rbrace= \left\lbrace\theta_{n}^{(0)}\right\rbrace\cdot
        \lbrace\theta_{n}\rbrace=
        \lbrace\theta_{n}+0\theta_{n-1}+\ldots+0\theta_{0}\rbrace=
        \lbrace\theta_{n}\rbrace
\end{displaymath}
\\\\
Let $\vect{b}=\lbrace b_{i}\rbrace_{i\in\mathbb{N}}$ be a sequence. 
The \emph{renewal array} \marginpar{renewal arrays, precisely}
$\mathcal{B}=\lbrace b_{nm}\rbrace_{n,m\in\mathbb{N}}$
generated by sequence $\vect{b}$ is defined as follows:
\begin{displaymath}
    b_{nm}=b_{n-m}^{(m+1)} \quad\wedge\quad n < m\rightarrow b_{nm}=0
\end{displaymath}
For the sake of clarity, let $n\in\mathbb{N}$, then
% the following set should be helpful if subscript index change `n-m'
% hasn't be done in the definition of Renewal array 
%let $S_{m}=\mathbb{N}\setminus\lbrace0,\ldots,m-1\rbrace$, then 
the following is the sequence on the very first column, where $m=0$:
\begin{displaymath}
    \lbrace b_{n0}\rbrace
        =\lbrace b_{n}^{(1)}\rbrace
        =\lbrace b_{n}\rbrace
\end{displaymath}
the following is the sequence on the second column, where $m=1$:
\begin{displaymath}
    \lbrace b_{n1}\rbrace
        =\lbrace b_{n-1}^{(2)}\rbrace
        =\left\lbrace \sum_{j=0}^{n-1}{b_{j}b_{n-1-j}}\right\rbrace
\end{displaymath}
the following is the sequence on the third column, where $m=2$:
\begin{displaymath}
    \lbrace b_{n2}\rbrace
        =\lbrace b_{n-2}^{(3)}\rbrace
        =\left\lbrace \sum_{i+j+l=n-2}{b_{i}b_{j}b_{l}}\right\rbrace
\end{displaymath}
and, in general, the following is the sequence on column $m\in\mathbb{N}$:
\begin{displaymath}
    \lbrace b_{nm}\rbrace
        =\lbrace b_{n-m}^{(m+1)}\rbrace
        =\left\lbrace \sum_{i_{1}+i_{2}+\ldots+i_{m+1} =n-m}
            {b_{i_{1}}b_{i_{2}}\cdots b_{i_{m+1}}}\right\rbrace
\end{displaymath}
Choose a column index $m\in\mathbb{N}$ and let $B$ a \ac{fps} over sequence $\vect{b}$.
Denote with $B^{(m)}$ a \ac{fps} over the \emph{renewal array} 
$\lbrace b_{nm} \rbrace_{n,m\in\mathbb{N}}$, defined by sequence $\vect{b}$, 
such that:
\begin{displaymath}
    B^{(m)}(t) = \sum_{n\geq m}{b_{nm}t^{n-m}}
\end{displaymath}
the following relation exists between functions $B$ and $B^{(m)}$:
\begin{displaymath}
    B^{(m)}(t) = B(t)^{m+1}
\end{displaymath}
Consider a sequence\marginpar{$A$-sequence, informally}
$\vect{a}=\lbrace a_{n} \rbrace_{n\in\mathbb{n}}$ and let $A$ be
a \ac{fps} over $\vect{a}$.
\begin{displaymath}
    A\left(t\,B(t)\right) 
        = \sum_{r\geq 0}{a_{r}\left(t\,B(t)\right)^{r}}
        = \sum_{r\geq 0}{a_{r}B(t)^{r}t^{r}}
        = \sum_{r\geq 0}{a_{r}B^{(r-1)}(t)t^{r}}
\end{displaymath}
by definition of $B^{(r-1)}$ rewrite:
\begin{displaymath}
    \sum_{r\geq 0}{a_{r}B^{(r-1)}(t)t^{r}}
        =\sum_{r\geq 0}{a_{r}\left(\sum_{n\geq r-1}{b_{n,r-1}t^{n-r+1}}\right)t^{r}}
        =\sum_{r\geq 0}{\sum_{n\geq r-1}{a_{r}b_{n,r-1}t^{n+1}}}
\end{displaymath}
now we tabulate sums respect to index $r$:
\begin{displaymath}
    \begin{split}
        r=0 &\rightarrow \sum_{n\geq -1}{a_{0}b_{n,-1}t^{n+1}}\\
        r=1 &\rightarrow \sum_{n\geq 0}{a_{1}b_{n,0}t^{n+1}}\\
        r=2 &\rightarrow \sum_{n\geq 1}{a_{2}b_{n,1}t^{n+1}}\\
        &\ldots\\
        r=k+1 &\rightarrow \sum_{n\geq k}{a_{k+1}b_{n,k}t^{n+1}}\\
        &\ldots\\
    \end{split}
\end{displaymath}
therefore extracting coefficient of $t^{n+1}$ yields:
\begin{displaymath}
    [t^{n+1}]A(t\,B(t))=\sum_{k\geq 0}{a_{k}\,b_{n,k-1}}
\end{displaymath}
if \marginpar{$A$-sequence, precisely}function $A$ satisfies 
$B(t)=A(t\,B(t))$, which is the same to say:
\begin{equation}
    b_{n+1}=\sum_{k\geq 0}{a_{k}\,b_{n,k-1}}
    \label{eq:A:sequence:for:Renewal:arrays}
\end{equation}
then, sequence $\vect{a}$ is the $A$-sequence for the \emph{renewal array}
$\lbrace b_{nk}\rbrace_{n,k\in\mathbb{N}}$, generated by sequence $\vect{b}$.
A subtle point underlying all this theory is shown in 
\autoref{eq:A:sequence:for:Renewal:arrays}: it says that coeffient
$b_{n+1}$, in sequence $\vect{b}$, is a combination
of all coefficients lying on row $n$ of the \emph{Renewal array}
$\lbrace b_{nm}\rbrace_{n,m\in\mathbb{N}}$,\marginpar{a recursion trick}
which is generated by sequence $\vect{b}$ itself!
\\\\
Allow us to generalize a little further. Choose 
$s\in\mathbb{N}\setminus\lbrace0\rbrace$ and consider the following:
\begin{displaymath}
    (t\,B(t))^{s-1}\,A\left(t\,B(t)\right) 
        = \sum_{r\geq 0}{a_{r}B(t)^{r+s-1}t^{r+s-1}}
        = \sum_{r\geq 0}{a_{r}B^{(r+s-2)}(t)t^{r+s-1}}
\end{displaymath}
by definition of $B^{(r+s-2)}$ rewrite:
\begin{displaymath}
    \sum_{r\geq 0}{a_{r}B^{(r+s-2)}(t)t^{r+s-1}}
        =\sum_{r\geq 0}{a_{r}\sum_{n\geq r+s-2}{b_{n,r+s-2}t^{n+1}}}
\end{displaymath}
finally:
\begin{displaymath}
    =\sum_{r\geq 0}{\sum_{n\geq r+s-2}{a_{r}b_{n,r+s-2}t^{n+1}}}
\end{displaymath}
now we tabulate sums respect to index $r$, as done in the previous derivation:
\begin{displaymath}
    \begin{split}
        r=0 &\rightarrow \sum_{n\geq s-2}{a_{0}b_{n,s-2}t^{n+1}}\\
        r=1 &\rightarrow \sum_{n\geq s-1}{a_{1}b_{n,s-1}t^{n+1}}\\
        r=2 &\rightarrow \sum_{n\geq s}{a_{2}b_{n,s}t^{n+1}}\\
        r=3 &\rightarrow \sum_{n\geq s+1}{a_{3}b_{n,s+1}t^{n+1}}\\
        &\ldots\\
        r=k+2 &\rightarrow \sum_{n\geq s+k}{a_{k+2}b_{n,s+k}t^{n+1}}\\
        &\ldots\\
    \end{split}
\end{displaymath}
therefore extracting coefficient of $t^{n+1}$ yields:
\begin{displaymath}
    [t^{n+1}](t\,B(t))^{s-1}A(t\,B(t))=\sum_{k\geq 0}{a_{k}\,b_{n,s+k-2}}
\end{displaymath}
if \marginpar{$A$-sequence, generally}function $A$ satisfies 
$B(t)^{s}=(t\,B(t))^{s-1}A(t\,B(t))$, which is the same to say:
\begin{equation}
    \sum_{i_{1}+\ldots+i_{s}=n+1}{b_{i_{1}}\cdots b_{i_{s}}}
        =b_{n+1}^{(s)}
        =b_{n+s,s-1}
        =\sum_{k\geq 0}{a_{k}\,b_{n,s+k-2}}
    \label{eq:generalized:A:sequence:for:Renewal:arrays}
\end{equation}
then, sequence $\vect{a}$ is the (\emph{generalized}) $A$-sequence for the
\emph{renewal array} $\lbrace b_{nk}\rbrace_{n,k\in\mathbb{N}}$, generated by
sequence $\vect{b}$ \marginpar{a dependency within array $\lbrace
b_{nm}\rbrace_{n,m\in\mathbb{N}}$}.

Fixing $s=1$, we go back to result shown in
\autoref{eq:A:sequence:for:Renewal:arrays}.



















\subsection{Eplett identity, by Eplett}

\citeauthor{riordan:intro:combinatorial:analysis},
in his book \cite{riordan:intro:combinatorial:analysis},
asks for a solution of the following exercise:

\begin{quote}
    let $\lbrace a_{n} \rbrace_{n\in\mathbb{N}}$ be a sequence, with $a_{0}\neq0$, 
    and $\lbrace a_{n}^{\prime} \rbrace_{n\in\mathbb{N}}$ be its \emph{inverse}:
    if $A(t)$ and $A^{\prime}(t)$ are two \ac{fps} over them, respectively, 
    then $A(t)A^{\prime}(t)=1$. Show that:
    \begin{displaymath}                
        a_{n}^{\prime} = \frac{(-1)^{n}}{a_{0}^{n+1}}
            \left|
            \begin{array}{ccccc}
                a_1 & a_2 & \ldots & a_{n-1} & a_{n}\\
                a_0 & a_1 & \ldots & a_{n-2} & a_{n-1}\\
                0   & a_0 & \ldots & a_{n-3} & a_{n-2}\\
                \vdots & \vdots & \vdots & \vdots & \vdots\\
                0 & 0 & \ldots & a_{1} & a_{2}\\
                0 & 0 & \ldots & a_{0} & a_{1}\\
            \end{array}
            \right|
    \end{displaymath}                
    for $n \geq 2$.
\end{quote}

\begin{proof}
Although not explicitly requested, %to show for $n\geq2$, 
it's useful to recall formula about $a_{0}^{\prime}$ and $a_{1}^{\prime}$, 
since they'll occur in the following. 

First of all we interpret the product
$A(t)A^{\prime}(t)=1$ as the Cauchy product over the ring of \ac{fps}, 
namely as a convolution, therefore: 
\begin{displaymath}
    \left[t^{0}\right]A(t)A^{\prime}(t)=a_{0}a_{0}^{\prime}=1
\end{displaymath}
is the relation which defines $a_{0}^{\prime}=\frac{1}{a_0}$.  On the other hand: 
\begin{displaymath}
    \left[t^{1}\right]A(t)A^{\prime}(t)=a_{0}a_{1}^{\prime}+a_{1}a_{0}^{\prime}=0
\end{displaymath}
is the relation which defines $a_{1}^{\prime}=-\frac{a_1}{a_{0}^{2}}$.

The proof proceeds by induction on $n$:
\begin{itemize}
    \item base $n=2$:
        \begin{displaymath}                
            a_{2}^{\prime} = \frac{1}{a_{0}^{3}}
                \left|
                \begin{array}{cc}
                    a_1 & a_2 \\
                    a_0 & a_1 \\
                \end{array}
                \right| 
                = \frac{a_{1}^{2}-a_{0}a_{2}}{a_{0}^{3}}
        \end{displaymath}                
        since $\big[t^{2}\big]A(t)A^{\prime}(t)=a_{0}a_{2}^{\prime}
            +a_{1}a_{1}^{\prime}+a_{2}a_{0}^{\prime}=0$, we have:
            \begin{displaymath}
                \begin{split}
                    a_{0}a_{2}^{\prime} &= -\left(a_{1}a_{1}^{\prime}+a_{2}a_{0}^{\prime}\right)
                        = -\left(-\left(\frac{a_{1}}{a_{0}}\right)^{2}+\frac{a_{2}}{a_{0}}\right)\\
                    a_{2}^{\prime} &= \frac{a_{1}^{2}}{a_{0}^{3}}-\frac{a_{2}}{a_{0}^{2}}
                \end{split}
            \end{displaymath}
        which proves the base case;

    \item assume the statement holds for $n$ and show it still holds for $n+1$, therefore
        we're left to prove the following:
    \begin{displaymath}                
        a_{n+1}^{\prime} = 
            -\frac{1}{a_{0}}\frac{(-1)^{n}}{a_{0}^{n+1}}
            \left|
            \begin{array}{ccccc}
                a_1 & a_2 & \ldots & a_{n} & a_{n+1}\\
                a_0 & a_1 & \ldots & a_{n-1} & a_{n}\\
                0   & a_0 & \ldots & a_{n-2} & a_{n-1}\\
                \vdots & \vdots & \vdots & \vdots & \vdots\\
                0 & 0 & \ldots & a_{1} & a_{2}\\
                0 & 0 & \ldots & a_{0} & a_{1}\\
            \end{array}
            \right|
    \end{displaymath}                
    expanding the determinant:
    \begin{displaymath}                
        a_{1}
            \left|
            \begin{array}{ccccc}
                a_1 & a_2 & \ldots & a_{n-1} & a_{n}\\
                a_0 & a_1 & \ldots & a_{n-2} & a_{n-1}\\
                0   & a_0 & \ldots & a_{n-3} & a_{n-2}\\
                \vdots & \vdots & \vdots & \vdots & \vdots\\
                0 & 0 & \ldots & a_{1} & a_{2}\\
                0 & 0 & \ldots & a_{0} & a_{1}\\
            \end{array}
            \right|
        - a_{0}
            \left|
            \begin{array}{ccccc}
                a_2 & a_3 & \ldots & a_{n} & a_{n+1}\\
                a_0 & a_1 & \ldots & a_{n-2} & a_{n-1}\\
                0   & a_0 & \ldots & a_{n-3} & a_{n-2}\\
                \vdots & \vdots & \vdots & \vdots & \vdots\\
                0 & 0 & \ldots & a_{1} & a_{2}\\
                0 & 0 & \ldots & a_{0} & a_{1}\\
            \end{array}
            \right|
    \end{displaymath}                
    therefore:
    \begin{displaymath}                
        a_{n+1}^{\prime} = 
            -\frac{a_{1}}{a_{0}}a_{n}^{\prime}
        - 
            \frac{1}{a_{0}}\frac{(-1)^{n-1}}{a_{0}^{n}}
            \left|
            \begin{array}{ccccc}
                a_2 & a_3 & \ldots & a_{n} & a_{n+1}\\
                a_0 & a_1 & \ldots & a_{n-2} & a_{n-1}\\
                0   & a_0 & \ldots & a_{n-3} & a_{n-2}\\
                \vdots & \vdots & \vdots & \vdots & \vdots\\
                0 & 0 & \ldots & a_{1} & a_{2}\\
                0 & 0 & \ldots & a_{0} & a_{1}\\
            \end{array}
            \right|
    \end{displaymath}                
    expanding the determinant a second time:
    \begin{displaymath}                
        a_{2}
            \left|
            \begin{array}{ccccc}
                a_1 & a_2 & \ldots & a_{n-2} & a_{n-1}\\
                a_0 & a_1 & \ldots & a_{n-3} & a_{n-2}\\
                \vdots & \vdots & \vdots & \vdots & \vdots\\
                0 & 0 & \ldots & a_{1} & a_{2}\\
                0 & 0 & \ldots & a_{0} & a_{1}\\
            \end{array}
            \right|
        - a_{0}
            \left|
            \begin{array}{ccccc}
                a_3 & a_4 & \ldots & a_{n} & a_{n+1}\\
                a_0 & a_1 & \ldots & a_{n-3} & a_{n-2}\\
                0   & a_0 & \ldots & a_{n-4} & a_{n-3}\\
                \vdots & \vdots & \vdots & \vdots & \vdots\\
                0 & 0 & \ldots & a_{1} & a_{2}\\
                0 & 0 & \ldots & a_{0} & a_{1}\\
            \end{array}
            \right|
    \end{displaymath}                
    therefore:
    \begin{displaymath}                
        a_{n+1}^{\prime} = 
            -\frac{a_{1}}{a_{0}}a_{n}^{\prime}
            -\frac{a_{2}}{a_{0}}a_{n-1}^{\prime}
            -\frac{1}{a_{0}}\frac{(-1)^{n-2}}{a_{0}^{n-1}}
            \left|
            \begin{array}{ccccc}
                a_3 & a_4 & \ldots & a_{n} & a_{n+1}\\
                a_0 & a_1 & \ldots & a_{n-3} & a_{n-2}\\
                0   & a_0 & \ldots & a_{n-4} & a_{n-3}\\
                \vdots & \vdots & \vdots & \vdots & \vdots\\
                0 & 0 & \ldots & a_{1} & a_{2}\\
                0 & 0 & \ldots & a_{0} & a_{1}\\
            \end{array}
            \right|
    \end{displaymath}                
    keep expanding determinants $n$ times yield:
    \begin{displaymath}                
        a_{n+1}^{\prime} = 
            -\frac{a_{1}}{a_{0}}a_{n}^{\prime}
            -\frac{a_{2}}{a_{0}}a_{n-1}^{\prime}
            \ldots
            -\frac{a_{n}}{a_{0}}a_{1}^{\prime}
            -\frac{a_{n+1}}{a_{0}}a_{0}^{\prime}
    \end{displaymath}                
    which is the same to say:
    \begin{displaymath}                
        a_{0}a_{n+1}^{\prime}  
            +a_{1}a_{n}^{\prime}
            +a_{2}a_{n-1}^{\prime}
            \ldots
            +a_{n}a_{1}^{\prime}
            +a_{n+1}a_{0}^{\prime}
            = 0
    \end{displaymath}                
    which is the relation we get from $\big[t^{n+1}\big]A(t)A^{\prime}(t)=0$
    so the induction step holds and the argument is proved as required.

\end{itemize}

\end{proof}

Previous exercise is actually a lemma for an identity which
\citeauthor{eplett:1979} finds and discusses in \cite{eplett:1979}.  Let
$\lbrace d_{nk}\rbrace_{n,k\in\mathbb{N}}$ be the Catalan triangle as defined
in \autoref{eq:shapiro:catalan:triangle:expanded}, so \citeauthor{eplett:1979}'s
result reads as follow.

\begin{theorem}
    \marginpar{in other words, $c_{j-1}$, the first coefficient of row $j-1$,
        combines coefficients lying on the next row $j$, summing with alternating signs}
    \begin{displaymath}                
        c_{j-1}=\sum_{k=1}^{j}{(-1)^{k-1}d_{jk}}
    \end{displaymath}                
\end{theorem}
\begin{proof}
    Let $\vect{a}=\lbrace a_{n}\rbrace_{n\in\mathbb{N}}$ be a sequence such
    that $a_{0}=1$ and consider its \emph{inverse} sequence, call it $\vect{b}$,
    where $b_{0}=1$ by inverse relation on the very first coefficient.
    By previous exercise observe that:
    \begin{equation}                
        \hspace{-2cm}
        a_{n} = b_{n}+a_{1}b_{n-1}+a_{2}b_{n-2}+\ldots+a_{n-1}b_{1}
        \leftrightarrow
        b_{n} = (-1)^{n-1}
            \left|
            \begin{array}{ccccc}
                a_1 & a_2 & \ldots & a_{n-1} & a_{n}\\
                1 & a_1 & \ldots & a_{n-2} & a_{n-1}\\
                0   & 1 & \ldots & a_{n-3} & a_{n-2}\\
                \vdots & \vdots & \vdots & \vdots & \vdots\\
                0 & 0 & \ldots & a_{1} & a_{2}\\
                0 & 0 & \ldots & 1 & a_{1}\\
            \end{array}
            \right|
        \label{eq:eplett:det:inverse:identity:lemma}
    \end{equation}                
    As we have seen in
    \autoref{eq:shapiro:catalan:triangle:convolution:for:generic:coefficient}, 
    we can rewrite theorem's \ac{rhs}, calling the new term $b_{j}$:
    \begin{displaymath}                
        c_{j-1}=\sum_{k=1}^{j}{(-1)^{k-1}d_{jk}}
            = \sum_{k=1}^{j}{(-1)^{k-1}\sum_{i_{1}+\ldots+i_{k}=j}{c_{i_{1}}\cdots c_{i_{k}}}}=b_{j}
    \end{displaymath}                
    % TODO in order to be truly precise, we should prove that the term abstracted out by `b_{j}'
    % actually equals the determinant of Catalan matrix.
    Applying \autoref{eq:eplett:det:inverse:identity:lemma} \emph{from right to left}, 
    it is necessary to find a sequence $\vect{a}$ such that:
    \begin{displaymath}                
        a_{n} = c_{n-1}+a_{1}c_{n-2}+a_{2}c_{n-3}+\ldots+a_{n-1}c_{0}
    \end{displaymath}                
    according \citeauthor{feller:intro:combinatorial:analysis}, page $94$ of
    \cite{feller:intro:combinatorial:analysis}, the following combination over
    Catalan numbers holds, for $n>0$:
    \begin{displaymath}                
        c_{n}=\sum_{k=1}^{n}{c_{k-1}c_{n-k}}\quad\wedge\quad c_{0}=1
    \end{displaymath}                
    therefore setting $a_{n}=c_{n}$ sequence $\vect{a}$ is found and theorem
    is proved.

\end{proof}

The above proof shows another interesting identity which relates Catalan numbers
with matrix determinants\marginpar{yet another characterization for $c_{j}$ using
    matrix theory}:
\begin{displaymath}                
    c_{n-1} = (-1)^{n-1}
        \left|
        \begin{array}{ccccc}
            c_1 & c_2 & \ldots & c_{n-1} & c_{n}\\
            1 & c_1 & \ldots & c_{n-2} & c_{n-1}\\
            0   & 1 & \ldots & c_{n-3} & c_{n-2}\\
            \vdots & \vdots & \vdots & \vdots & \vdots\\
            0 & 0 & \ldots & c_{1} & c_{2}\\
            0 & 0 & \ldots & 1 & c_{1}\\
        \end{array}
        \right|
\end{displaymath}                

Taking apart the sum in theorem's \ac{rhs}, we can restate it as:
\begin{displaymath}                
    c_{j-1} + \sum_{{k \text{ is even}}}{d_{jk}}
        = \sum_{{k \text{ is odd}}}{d_{jk}}\qquad\text{where } k\in\mathbb{N}\setminus \lbrace0\rbrace
\end{displaymath}
Since $d_{j0}$ has no meaning by definition of the triangle, we can
attach to it a combinatorial meaning: let $d_{j0}$ denotes the
number of pairs of paths that intersect for the first time at
step $j$ and possibly after again. Therefore $d_{j0}=c_{j-1}$, so 
we can rewrite one more time:
\begin{displaymath}                
    \sum_{{k \text{ is even}}}{d_{jk}}
        = \sum_{{k \text{ is odd}}}{d_{jk}} \qquad\text{where } k\in\mathbb{N}
\end{displaymath}
This last version supplies a combinatorial 
interpretation in terms of \emph{non-intersecting} pairs of paths:
\begin{quote}
    \marginpar{a combinatorial interpretation of \citeauthor{eplett:1979}'s result}
    consider the set $S$ of pairs of paths with distance $j$. Then,
    the subset of $S$ composed of \emph{non-intersecting} pairs of paths 
    with \emph{even} distance augmented by the subset of $S$ of 
    pairs of paths that \emph{intersect} for the first time at step $j$
    equals, in number, the subset of $S$ of pairs of paths with \emph{odd}
    distance
\end{quote}
\begin{proof}

    Consider a pair of paths $(\vect{\rho},\vect{\sigma})$ of length $j$ with
    \emph{even} distance $k$.  If a pair of steps, such that they \emph{do not
    point to the same direction}, is applied to $(\vect{\rho},\vect{\sigma})$
    then a new pair of paths $(\vect{\rho}^{\prime},\vect{\sigma}^{\prime})$ of
    length $j+1$ with \emph{odd} distance $k+1$ is built. This production is a
    bijection, therefore the number of pairs of paths with \emph{even} distance
    equal the one with \emph{odd} distance, under the constraint that paths
    have the same length them all, as required.

\end{proof}





















\subsection{The Riordan group}

In \cite{shapiro:1991}, \citeauthor{shapiro:1991} introduces a
generalization of the concepts developed so far: a group called 
\emph{Riordan group}, in honor of professor John Riordan
\marginpar{Thank you John Riordan, $1903-1988$}.
This group has been defined in order to unify many themes in 
enumeration and combinatoric problems; to solve binomial, 
inverse identities and, more recently, combinatorial
sums and binary words counting. 
\\\\
Let $\mathcal{M}=\lbrace m_{nk}\rbrace_{n,k\in\mathbb{N}}$ be an \emph{infinite
lower triangular matrix}, namely:
\begin{displaymath}
     m_{nk}\in\mathbb{Z} \quad\wedge\quad n < k \rightarrow m_{nk} = 0
\end{displaymath}
Call $\mathcal{M}_{(t)}$ the vector produced by the following product, 
where $t$ is an \emph{dummy variable}\marginpar{$\mathcal{M}$ is a matrix of integers, 
$\mathcal{M}_{(t)}$ is a vector of \ac{fps} in $\mathbb{Z}\llbracket t \rrbracket$}:
\begin{displaymath}
    \left[
        \begin{array}{cccccc}
            1 & t & t^{2} & t^{3} & t^{4} &\ldots
        \end{array}
    \right]
    \left[
        \begin{array}{cccccc}
            m_{00} & & & &  &\\
            m_{10} & m_{11} & & &  &\\
            m_{20} & m_{21}& m_{22}& &  &\\
            m_{30} & m_{31}& m_{32}& m_{33}&  &\\
            m_{40} & m_{41}& m_{42}& m_{43}& m_{44} &\\
            \vdots & \vdots& \vdots& \vdots& \vdots & \ddots\\
        \end{array}
    \right]
\end{displaymath}
therefore $\mathcal{M}_{(t)} =
    \left[
        \begin{array}{cccccc}
            m_{0}(t) & m_{1}(t) & m_{2}(t) & m_{3}(t) &m_{4}(t) & \ldots
        \end{array}
    \right]$ 
with function $m_{j}$, for $j\in\mathbb{N}$, is a \ac{fps} in the ring 
$\mathbb{Z}\llbracket t \rrbracket$. If there exists two analytic functions $d$
and $h$, such that $d(0)\neq0$ and $h(0)=0 \wedge h^{\prime}(0)\neq0$, which
satisfy \marginpar{matrix $\mathcal{M}$ will be denoted by the pair $(d(t),h(t))$}:
\begin{displaymath}
    m_{j}(t)=d(t)\,h(t)^{j} \quad \forall j\in\mathbb{N}
\end{displaymath}
then matrix $\mathcal{M}$ is called a \emph{Riordan array}. Such an array is
directly related to coefficients matrix $\lbrace m_{nk}\rbrace_{n,k\in\mathbb{N}}$ as follows:
\begin{displaymath}
    [t^{n}]m_{j}(t)=m_{nj}
\end{displaymath}


\begin{theorem}[Fundamental theorem]
    Let $\mathcal{M}=(d(t),h(t))$ be a Riordan matrix and $\vect{b}=
    \left[\begin{array}{cccc}b_{0}&b_{1}&b_{2}&\ldots\end{array}\right]^{T}$ 
    be an infinite column vector with $b_{j}\in\mathbb{Z}$, for $j\in\mathbb{N}$. Then:
    \begin{displaymath}
        \mathcal{M}_{(t)}\cdot\vect{b}=d(t)b(h(t))
    \end{displaymath}
    where function $b$ is a \ac{fps} over coefficients in $\vect{b}$.
    \label{thm:riordan:group:fundamental:theorem}
\end{theorem}
\begin{proof}
    By definition of vector $\mathcal{M}_{(t)}$:
    \begin{displaymath}
        \mathcal{M}_{(t)}\cdot\vect{b}= b_{0}\,m_{0}(t) + b_{1}\,m_{1}(t) + b_{2}\,m_{2}(t) 
            + b_{3}\,m_{3}(t) + b_{4}\,m_{4}(t) + \ldots
    \end{displaymath}
    by definition of function $m_{j}$, for any $j\in\mathbb{N}$:
    \begin{displaymath}
        = b_{0}\,d(t)h(t)^{0} + b_{1}\,d(t)h(t) + b_{2}\,d(t)h(t)^{2} 
            + b_{3}\,d(t)h(t)^{3} + b_{4}\,d(t)h(t)^{4} + \ldots
    \end{displaymath}
    factor out function $d$:
    \begin{displaymath}
        = d(t)\left(b_{0}\,h(t)^{0} + b_{1}\,h(t) + b_{2}\,h(t)^{2} 
            + b_{3}\,h(t)^{3} + b_{4}\,h(t)^{4} + \ldots\right)
    \end{displaymath}
    finally, rewrite composing function $h$ with function $b$:
    \begin{displaymath}
        = d(t)b(h(t))  
    \end{displaymath}
    as required.

\end{proof}

In the introductory paragraph we said that the current generalization
we are studying is a \emph{group}. More formally, we are going to define a 
\emph{group} over a \emph{set of Riordan matrices}, 
denoted by $(\lbrace \mathcal{M}_{i}\rbrace_{i\in\mathbb{N}},\cdot)$, 
where $\cdot$ is the \emph{group
operator}: additionally it is required to show the \emph{identity element} and,
finally, to prove that there exists an \emph{inverse} for every element in the group.
\\\\
Let $\mathcal{M}=\left(d_{\mathcal{M}}(t),h_{\mathcal{M}}(t)\right)$ and
$\mathcal{N}=\left(d_{\mathcal{N}}(t),h_{\mathcal{N}}(t)\right)$ be two Riordan matrices.
Define the \emph{group operator} as their product
\marginpar{matrix multiplication as group operator}, formally:
\begin{displaymath}
    \left(d_{\mathcal{M}}(t),h_{\mathcal{M}}(t)\right)\cdot
        \left(d_{\mathcal{N}}(t),h_{\mathcal{N}}(t)\right) = 
        \left(d_{\mathcal{M}}(t)d_{\mathcal{N}}(h_{\mathcal{M}}(t)),
            h_{\mathcal{N}}(h_{\mathcal{M}}(t))\right)
\end{displaymath}
\begin{proof}
    Consider a column $j$ of matrix $\mathcal{N}$,
    for some $j\in\mathbb{N}$. Denoting by $\vect{e}_{j}$ the $j$-th versor and
    by $n_{j}(t)=d_{\mathcal{N}}(t)h_{\mathcal{N}}(t)^{j}$ the \ac{fps} over such column, 
    by \autoref{thm:riordan:group:fundamental:theorem} the following holds:
    \begin{displaymath}
        \mathcal{M}_{(t)}\cdot\left(\mathcal{N}\cdot\vect{e}_{j}\right)
            = d_{\mathcal{M}}(t) d_{\mathcal{N}}(h_{\mathcal{M}}(t))
                h_{\mathcal{N}}(h_{\mathcal{M}}(t))^{j}
    \end{displaymath}
    Let $d_{\mathcal{M}\mathcal{N}}$ be a function such that 
    $d_{\mathcal{M}\mathcal{N}}(t)=d_{\mathcal{M}}(t) d_{\mathcal{N}}( h_{\mathcal{M}}(t))$,
    and let $h_{\mathcal{M}\mathcal{N}}$ be a function such that 
    $h_{\mathcal{M}\mathcal{N}}(t)=h_{\mathcal{N}}(h_{\mathcal{M}}(t))$, then the \ac{rhs}
    has the shape
    $d_{\mathcal{M}\mathcal{N}}(t)h_{\mathcal{M}\mathcal{N}}(t)^{j}$, which denotes the
    Riordan matrix $\mathcal{M}\mathcal{N}$, as required.

\end{proof}

The \marginpar{$(1,t)$ is the identity element} identity element of the group
is the Riordan matrix $\mathcal{I}=(1,t)$.
\begin{proof}
    Let $\mathcal{M}=\left(d_{\mathcal{M}}(t),h_{\mathcal{M}}(t)\right)$ be a Riordan matrix.
    By definition of group operator:
    \begin{displaymath}
        \left(d_{\mathcal{M}}(t),h_{\mathcal{M}}(t)\right)\cdot \left(1,t\right) = 
            \left(d_{\mathcal{M}}(t), h_{\mathcal{M}}(t)\right)
    \end{displaymath}
    since group operator $\cdot$ is no commutative in general, check the following also:
    \begin{displaymath}
        \left(1,t\right) \cdot \left(d_{\mathcal{M}}(t),h_{\mathcal{M}}(t)\right)= 
            \left(d_{\mathcal{M}}(t), h_{\mathcal{M}}(t)\right)
    \end{displaymath}
    as required.
\end{proof}

Finally, in order to find the \emph{inverse} of a Riordan matrix, we have to
introduce the concept of \marginpar{compositional inverse of a function}
\emph{compositional inverse} of an analytic function $h$.
Let $h$ be a function such that $h(0)=0 \wedge h^{\prime}(0)\neq0$, then
$\hat{h}$ is the \emph{compositional inverse} of function $h$ if and only if 
$\hat{h}(h(t))=h(\hat{h}(t))=t$. 
\\\\
Let $\mathcal{M}=\left(d_{\mathcal{M}}(t),h_{\mathcal{M}}(t)\right)$ be a Riordan matrix.
The inverse matrix $\mathcal{M}^{-1}$ of $\mathcal{M}$ satisfies the following relation:
\begin{displaymath}
    \mathcal{M}^{-1}=\left(
        \frac{1}{d_{\mathcal{M}}(\hat{h}_{\mathcal{M}}(t))}, \hat{h}_{\mathcal{M}}(t)
    \right)
\end{displaymath}

\begin{proof}
    By definition of inverse element\marginpar{inverse of an element in 
    $\left(\lbrace \mathcal{M}_{i}\rbrace,\cdot\right)$}, 
    we have to find a matrix $\mathcal{M}^{-1}$ which satisfies the following relation:
    \begin{displaymath}
        \mathcal{M}\cdot\mathcal{M}^{-1}
            =\left(d_{\mathcal{M}}(t)d_{\mathcal{M}^{-1}}(h_{\mathcal{M}}(t)),
                    h_{\mathcal{M}^{-1}}(h_{\mathcal{M}}(t))\right)
            =\left(1,t\right)
    \end{displaymath}

    Looking at the first component we get:
    \begin{displaymath}
        d_{\mathcal{M}}(t)d_{\mathcal{M}^{-1}}(h_{\mathcal{M}}(t))=1    
    \end{displaymath}
    which can be rewritten as\marginpar{abstracting function $g$ in the
        composition $f(g(t))$ yield $\left.\left[f(y)\right|y=g(t)\right]$,
        a trick also called ``changing variable''}:
    \begin{displaymath}
        d_{\mathcal{M}^{-1}}(h_{\mathcal{M}}(t))=\frac{1}{d_{\mathcal{M}}(t)}    
            = \left.\left[
                d_{\mathcal{M}^{-1}}(y)=\frac{1}{d_{\mathcal{M}}(\hat{h}_{\mathcal{M}}(y))}
                    \right|y=h_{\mathcal{M}}(t)
                \right]
    \end{displaymath}

    On the other hand, looking at the second component we get:
    \begin{displaymath}
        h_{\mathcal{M}^{-1}}(h_{\mathcal{M}}(t))=t
    \end{displaymath}
    therefore $h_{\mathcal{M}^{-1}}=\hat{h}_{\mathcal{M}}$, as required.
\end{proof}

It \marginpar{subgroups of the Riordan group}
is interesting to dig a little on the \emph{Riordan group}, looking at
common patterns that arises in Riordan matrices definitions. Denoting with
$\cdot$ the group operator defined as before, we report
these patterns by introducing the following \emph{subgroups}:
\begin{itemize}
    \item $\left(\lbrace \mathcal{M}_{i}=(d_{\mathcal{M}_{i}}(t),t)
        \rbrace_{i\in\mathbb{N}},\cdot\right)$, the \emph{Appell subgroup};
    \item $\left(\lbrace \mathcal{M}_{i}=(1,h_{\mathcal{M}_{i}}(t))
        \rbrace_{i\in\mathbb{N}},\cdot\right)$, the \emph{Associated subgroup};
    \item $\left(\lbrace \mathcal{M}_{i}=(d_{\mathcal{M}_{i}}(t),td_{\mathcal{M}_{i}}(t))
        \rbrace_{i\in\mathbb{N}},\cdot\right)$, the \emph{Renewal subgroup};
    \item $\left(\lbrace \mathcal{M}_{i}=(d_{\mathcal{M}_{i}}(t),h_{\mathcal{M}_{i}}(t))
        \rbrace_{i\in\mathbb{N}},\cdot\right)$, where function $d$ is \emph{even} and 
        function $h$ is \emph{odd}, the \emph{Checkerboard subgroup};
    \item $\left(\left\lbrace \mathcal{M}_{i}=\left(\frac{t\,h_{\mathcal{M}_{i}}^{\prime}(t)}
            {h_{\mathcal{M}_{i}}(t)},h_{\mathcal{M}_{i}}(t)\right)
        \right\rbrace_{i\in\mathbb{N}},\cdot\right)$, the \emph{Hitting-time subgroup};
\end{itemize}

Finally, \marginpar{a characterization using a bivariate function}
we would like to discuss another interesting definition which characterize
a Riordan array $\mathcal{M}=(d_{\mathcal{M}}(t),h_{\mathcal{M}}(t))$. 
The main point is to associate a bivariate function $m$ 
to the infinite lower triangular matrix $\lbrace m_{nk}\rbrace_{n,k\in\mathbb{N}}$
of coefficients, used to build $\mathcal{M}$. Function $m$ is defined as follows:
\begin{displaymath}
    m(t,u) = \sum_{k\in\mathbb{N}}{\sum_{n\geq k}{m_{nk} t^{n} u^{k}}}
        = \sum_{k\in\mathbb{N}}{m_{k}(t)u^{k}}
\end{displaymath}
by definition of function $m_{k}$:
\begin{displaymath}
    = \sum_{k\in\mathbb{N}}{d_{\mathcal{M}}(t)\,h_{\mathcal{M}}(t)^{k}\,u^{k}}
    = \frac{d_{\mathcal{M}}(t)}{1-h_{\mathcal{M}}(t)\,u}
\end{displaymath}
this characterization is more general and allows us to have \ac{fps} $m_{j}^{(t)}(u)$
for a column $j$, if it is expanded respect variable $t$; and, by symmetry,
to have \ac{fps} $m_{l}^{(u)}(t)$ for a row $l$, if it is expanded respect variable $u$.












\subsection{Applications to combinatorial sums}

\citeauthor{sprugnoli:1991}, in \cite{sprugnoli:1991}, shows how the
\emph{Riordan group} and its elements \emph{Riordan arrays} provide an elegant
methodology to solve combinatorial sums. This method is quite different from
ones by others\footnote{put here some bib references} since it offer a
\emph{constructive} proof for a sum identity $\sum_{k}f_{k}=v$: in addition to
be a \emph{certification} for the identity, it shows how to find such value
$v$.

\begin{theorem}[Sprugnoli]
    Let $\mathcal{M}=(d(t),h(t))$ be a Riordan matrix and $\vect{\omega}$ 
    be a \emph{given} sequence $\lbrace\omega_{i}\rbrace_{i\in\mathbb{N}}$. Choose
    a row index $n\in\mathbb{N}$, then:
    \begin{displaymath}
        \sum_{k=0}^{n}{m_{nk}\,\omega_{k}}=[t^{n}]d(t)\Omega(h(t))
    \end{displaymath}
    where function $\Omega$ is a \ac{fps} over sequence $\vect{\omega}$.
    \label{thm:sprugnoli:riordan:combinatorial:sums}
\end{theorem}

\begin{proof}
    By relation among Riordan array $\mathcal{M}$ and matrix 
    $\lbrace m_{nk}\rbrace_{n,k\in\mathbb{N}}$:
    \begin{displaymath}
        \sum_{k=0}^{n}{m_{nk}\omega_{k}}
            =\sum_{k=0}^{n}{[t^{n}]m_{k}(t)\,\omega_{k}}
            =d(t)\sum_{k=0}^{n}{[t^{n}]h(t)^{k}\,\omega_{k}}
    \end{displaymath}
    since summing $n+1$ coefficients extracted from $n+1$
    convolutions of function $h$ with itself, is the same as
    do $n+1$ convolutions of function $h$ with itself, sum them
    and, lately, do \emph{one} coefficient extraction, the following holds:
    \begin{displaymath}
        =[t^{n}]d(t)\sum_{k=0}^{n}{\omega_{k}\,h(t)^{k}}
        =[t^{n}]d(t)\Omega(h(t))
    \end{displaymath}
    as required.
\end{proof}

It is very interesting the following result that allows us to solve
yet another combinatorial sum using both a Riordan array $\mathcal{M}$
and its inverse.

\begin{theorem}
    Let $\mathcal{M}=(d(t),h(t))$ be a Riordan matrix and $\vect{\omega}$ 
    be a sequence $\lbrace\omega_{i}\rbrace_{i\in\mathbb{N}}$ with an \emph{unknown} 
    \ac{gf} $\Omega$. Choose a row index $n\in\mathbb{N}$, if $\vect{\theta}$ is a  
    sequence $\lbrace\theta_{i}\rbrace_{i\in\mathbb{N}}$ with a \emph{given} \ac{gf} $\Theta$, 
    then:
    \begin{displaymath}
        \sum_{k=0}^{n}{m_{nk}\,\omega_{k}}=\theta_{n}
            \quad\leftarrow\quad 
            \sum_{k=0}^{n}{\tilde{m}_{nk}\,\theta_{k}}=\omega_{n}
    \end{displaymath}
    where $\tilde{m}_{nk}\in\mathcal{M}^{-1}$.
\end{theorem}

\begin{proof}
    By \autoref{thm:sprugnoli:riordan:combinatorial:sums} we can rewrite:
    \begin{displaymath}
        \sum_{k=0}^{n}{\tilde{m}_{nk}\,\theta_{k}}
            = [t^{n}]\frac{1}{d(\hat{h}(t))}\,\Theta(\hat{h}(t)) 
            = [t^{n}]\Omega(t) = \omega_{n}
    \end{displaymath}
    which sets unknown function $\Omega$, which is the \ac{gf} for 
    sequence $\vect{\omega}$, such that $\Omega(t)=\frac{1}{d(\hat{h}(t))}\,\Theta(\hat{h}(t))$. 
    Then:
    \begin{displaymath}
        \sum_{k=0}^{n}{m_{nk}\,\omega_{k}}
            = [u^{n}]d(u)\Omega(h(u))
            = [u^{n}]\Theta(u)
            = \theta_{n}
    \end{displaymath}
    as required.
\end{proof}


\subsection{$A$-sequence and $A$-matrix}
\label{sec:back:to:the:basics:sequences}

In this section we rework the concept of $A$-sequence introduced by
\citeauthor{rogers:1977}, subject of our discussion in
\autoref{sec:back:to:the:basics:rogers}, using the \emph{Riordan group}
to characterize it for \emph{Riordan arrays}. Furthermore, we will see
a generalization of this concept, introducing $A$-matrices and some
applications are reported for the sake of clarity.

\begin{theorem}
    $\mathcal{M}$ is a Riordan array over a matrix of coefficients
    $\lbrace m_{n,k}\rbrace_{n,k\in\mathbb{N}}$ if and only if
    there exists a sequence $\vect{a}=\lbrace a_{n}\rbrace_{n\in\mathbb{N}}$,
    with $a_{0}\neq0$, such that:
    \begin{displaymath}
        m_{n+1,k+1}=a_{0}m_{n,k}+a_{1}m_{n,k+1}+a_{2}m_{n,k+2}+\ldots+a_{j}m_{n,k+j}
    \end{displaymath}
    where $n,k,j\in\mathbb{N}$ and $k+j=n$. Sequence $\vect{a}$ is called the $A$-sequence
    of Riordan array $\mathcal{M}$. 
    \label{thm:merlini:A:sequence:characterization} 
\end{theorem}
\marginpar{The idea behind this proof comes from \cite{he:sprugnoli:2009}}
\begin{proof}[Proof of ($\rightarrow$) direction]
    Let $\mathcal{M}=(d_{\mathcal{M}}(t),h_{\mathcal{M}}(t))$ be a Riordan array. Since it is requested
    to prove that such a sequence $\vect{a}$ \emph{exists}, we proceed to
    build a Riordan array $\mathcal{A}=(d_{\mathcal{A}}(t),h_{\mathcal{A}}(t))$ such
    that function $d_{\mathcal{A}}$ is the \ac{gf} for sequence $\vect{a}$, proving this direction.
    Riordan array $\mathcal{A}$ is the defined as the solution of the following relation:
    \begin{displaymath}
        (d_{\mathcal{M}}(t),h_{\mathcal{M}}(t))\cdot(d_{\mathcal{A}}(t),h_{\mathcal{A}}(t))
            = \left(d_{\mathcal{M}}(t)\frac{h_{\mathcal{M}}(t)}{t},h_{\mathcal{M}}(t)\right)
    \end{displaymath}
    by definition of group operator $\cdot$, the \ac{lhs} can be rewritten as:
    \begin{displaymath}
        (d_{\mathcal{M}}(t)d_{\mathcal{A}}(h_{\mathcal{M}}(t)),h_{\mathcal{A}}(h_{\mathcal{M}}(t)))
            = \left(d_{\mathcal{M}}(t)\frac{h_{\mathcal{M}}(t)}{t},h_{\mathcal{M}}(t)\right)
    \end{displaymath}
    
    Looking at second components, the following relation defines function $h_{\mathcal{A}}$:
    \begin{displaymath}
         \left[h_{\mathcal{A}}(y)=y \mid y = h_{\mathcal{M}}(t)\right]
    \end{displaymath}
    while looking at the first component, the following relation defines function $d_{\mathcal{A}}$:
    \marginpar{a \ac{gf} over $A$-sequence $\vect{a}$ for Riordan arrays}
    \begin{displaymath}
         \left[d_{\mathcal{A}}(y)=\frac{y}{\hat{h}_{\mathcal{M}}(y)} \mid y = h_{\mathcal{M}}(t)\right]
    \end{displaymath}
    therefore it is possible to write the complete definition for array $\mathcal{A}$:
    \begin{displaymath}
         \mathcal{A}=(d_{\mathcal{A}}(y),h_{\mathcal{A}}(y))
            =\left(\frac{y}{\hat{h}_{\mathcal{M}}(y)}, y \right)
    \end{displaymath}
    
    \marginpar{although we are in the Riordan group, here we go back to 
        matrix theory}
    Assume function $d_{\mathcal{A}}$ can be expanded as \ac{fps} over a sequence
    $\vect{a}=\lbrace a_{n}\rbrace_{n\in\mathbb{N}}$, therefore expanding array $\mathcal{A}$
    as a matrix $\lbrace a_{nk}\rbrace_{n,k\in\mathbb{N}}$ yield:
    \begin{displaymath}
        \mathcal{A} =
        \left[
            \begin{array}{cccccc}
                a_{0} & & & &  &\\
                a_{1} & a_{0} & & &  &\\
                a_{2} & a_{1}& a_{0}& &  &\\
                a_{3} & a_{2}& a_{1}& a_{0}&  &\\
                a_{4} & a_{3}& a_{2}& a_{1}& a_{0} &\\
                \vdots & \vdots& \vdots& \vdots& \vdots & \ddots\\
            \end{array}
        \right]
    \end{displaymath}
    so doing the \emph{matrix} product of array $\mathcal{M}$ with array $\mathcal{A}$ yield:
    \begin{displaymath}
        \mathcal{M}\mathcal{A}
        = 
        \left[
            \begin{array}{cccccc}
                m_{00} & & & &  &\\
                m_{10} & m_{11} & & &  &\\
                m_{20} & m_{21}& m_{22}& &  &\\
                m_{30} & m_{31}& m_{32}& m_{33}&  &\\
                m_{40} & m_{41}& m_{42}& m_{43}& m_{44} &\\
                \vdots & \vdots& \vdots& \vdots& \vdots & \ddots\\
            \end{array}
        \right]
        \left[
            \begin{array}{cccccc}
                a_{0} & & & &  &\\
                a_{1} & a_{0} & & &  &\\
                a_{2} & a_{1}& a_{0}& &  &\\
                a_{3} & a_{2}& a_{1}& a_{0}&  &\\
                a_{4} & a_{3}& a_{2}& a_{1}& a_{0} &\\
                \vdots & \vdots& \vdots& \vdots& \vdots & \ddots\\
            \end{array}
        \right]
    \end{displaymath}
    Consider coefficient at row $n$ and column $k$ of $\mathcal{M}\mathcal{A}$, by definition
    of \emph{matrix} product:
    \begin{displaymath}
        \left(\mathcal{M}\mathcal{A}\right)_{nk}=
        \sum_{j=0}^{n}{m_{nj}{a_{jk}}}
            =\sum_{j=0}^{n-k}{m_{n,k+j}{a_{k+j,k}}}
            =\sum_{j=0}^{n-k}{m_{n,k+j}{a_{j}}}
    \end{displaymath}
    Do the same for coefficient at row $n$ and column $k$ of
    $\left(d_{\mathcal{M}}(t)\frac{h_{\mathcal{M}}(t)}{t},h_{\mathcal{M}}(t)\right)$:
    \begin{displaymath}
        [t^{n+1}]d_{\mathcal{M}}(t)\,h_{\mathcal{M}}(t)^{k+1} = m_{n+1,k+1}
    \end{displaymath}
    so equating this and the previous result yield:
    \begin{displaymath}
        m_{n+1,k+1}=\sum_{j=0}^{n}{m_{n,k+j}{a_{j}}}
    \end{displaymath}
    as required.
\end{proof}

The above proof leave space for another result. Start again from the following equation:
\begin{displaymath}
    (d_{\mathcal{M}}(t)d_{\mathcal{A}}(h_{\mathcal{M}}(t)),h_{\mathcal{M}}(t))
        = \left(d_{\mathcal{M}}(t)\frac{h_{\mathcal{M}}(t)}{t},h_{\mathcal{M}}(t)\right)
\end{displaymath}
and consider coefficient at row $n$ and column $k$ in both arrays, as done before:
\begin{displaymath}
    [t^{n}]d_{\mathcal{A}}(h_{\mathcal{M}}(t))\left(d_{\mathcal{M}}(t)\,h_{\mathcal{M}}(t)^{k}\right)
        = [t^{n+1}]d_{\mathcal{M}}(t)\,h_{\mathcal{M}}(t)^{k+1}
\end{displaymath}
\marginpar{sequence $\vect{\hat{a}}$, with \ac{gf} $\hat{A}(t)$, 
is yet another characterization}
on the \ac{lhs} appears a convolution of function $\hat{A}(t)=d_{\mathcal{A}}(h_{\mathcal{M}}(t))$,
which can be written as \ac{fps} over sequence $\vect{\hat{a}}=\lbrace\hat{a}_{i}\rbrace_{i\in\mathbb{N}}$,
with function $m_{k}(t)=d_{\mathcal{M}}(t)\,h_{\mathcal{M}}(t)^{k}$. Therefore the following identity
holds as well:
\begin{displaymath}
    \sum_{j=0}^{n}{\hat{a}_{j}\,m_{n-j,k}}
        = m_{n+1,k+1}
\end{displaymath}
combining this result with the one stated in the above theorem, the following holds:
\begin{displaymath}
    \sum_{j=0}^{n}{\hat{a}_{j}\,m_{n-j,k}}
        =\sum_{j=0}^{n}{m_{n,k+j}{a_{j}}}
\end{displaymath}

Yet another characterization for a coefficient $m_{nk}$ in a Riordan array $\mathcal{M}$
can be stated and is closely related to $\mathcal{M}$'s $A$-sequence.
\begin{theorem}
    Let $\mathcal{M}$ be a Riordan array over a matrix of coefficients
    $\lbrace m_{n,k}\rbrace_{n,k\in\mathbb{N}}$, then there exists a sequence
    $\vect{\omega}=\lbrace \omega_{n}\rbrace_{n\in\mathbb{N}}$ such that:
    \begin{displaymath}
        m_{nk}=\sum_{j=0}^{n-k}{\omega_{j}\,m_{n+1,k+1+j}}
    \end{displaymath}
    moreover, denoting with $\Omega$ a \ac{gf} over sequence $\vect{\omega}$ and
    with $A$ the \ac{gf} for $\mathcal{M}$'s $A$-sequence, then:
    \begin{displaymath}
        A(t)\,\Omega(t)=1
    \end{displaymath}
    \label{thm:characterization:next:row}
\end{theorem}
\begin{proof}
    Assume not, therefore for \emph{each} sequence $\vect{\omega}=\lbrace \omega_{n}\rbrace_{n\in\mathbb{N}}$,
    from:
    \begin{displaymath}
        m_{nk}\neq\sum_{j=0}^{n-k}{\omega_{j}\,m_{n+1,k+1+j}}
    \end{displaymath}
    we have to show a contraddiction. By $A$-sequence characterization, 
    there exists a sequence $\vect{a}=\lbrace a_{n}\rbrace_{n\in\mathbb{N}}$
    such that allows us to rewrite terms $m_{n+1,k+1+j}$
    within the sum in the \ac{rhs}:
    \begin{displaymath}
        \begin{split}
            m_{nk} &\neq \omega_{0}\left(a_{0}m_{n,k}+a_{1}m_{n,k+1}+a_{2}m_{n,k+2}+a_{3}m_{n,k+3}+\ldots+a_{j_{0}}m_{n,k+j_{0}}\right)\\
                   &+ \omega_{1}\left(a_{0}m_{n,k+1}+a_{1}m_{n,k+2}+a_{2}m_{n,k+3}+a_{3}m_{n,k+4}+\ldots+a_{j_{1}}m_{n,k+j_{1}}\right)\\
                   &+ \omega_{2}\left(a_{0}m_{n,k+2}+a_{1}m_{n,k+3}+a_{2}m_{n,k+4}+a_{3}m_{n,k+5}+\ldots+a_{j_{2}}m_{n,k+j_{2}}\right)\\
                   &+ \omega_{3}\left(a_{0}m_{n,k+3}+a_{1}m_{n,k+4}+a_{2}m_{n,k+5}+a_{3}m_{n,k+6}+\ldots+a_{j_{3}}m_{n,k+j_{3}}\right)\\
                   &\ldots\\
                   &+ \omega_{n-k-1}\left(a_{0}m_{n,n-1}+a_{1}m_{n,n}\right)\\
                   &+ \omega_{n-k}\left(a_{0}m_{n,n}\right)\\
        \end{split}
    \end{displaymath}
    factoring coefficients $m_{nk}$, the \ac{rhs} can be rewritten as:
    \begin{displaymath}
        \begin{split}
            m_{nk}&\neq \omega_{0}a_{0}m_{n,k} + \left(\sum_{i_{1}+i_{2}=1}{\omega_{i_{1}}a_{i_{2}}}\right)m_{n,k+1}
                + \left(\sum_{i_{1}+i_{2}=2}{\omega_{i_{1}}a_{i_{2}}}\right)m_{n,k+2}\\
                &+ \left(\sum_{i_{1}+i_{2}=3}{\omega_{i_{1}}a_{i_{2}}}\right)m_{n,k+3}
                + \ldots 
                + \left(\sum_{i_{1}+i_{2}=n-k}{\omega_{i_{1}}a_{i_{2}}}\right)m_{n,n}
        \end{split}
    \end{displaymath}
    which can be written more compactly as:
    \begin{displaymath}
        m_{nk} \neq \omega_{0}a_{0}m_{n,k} +
            \sum_{j=1}^{n-k}{\left(\sum_{i_{1}+i_{2}=j}{\omega_{i_{1}}a_{i_{2}}}\right)m_{n,k+j}}
    \end{displaymath}
    but if we take sequence $\vect{\omega}$ to be the \emph{inverse} sequence of 
    $\mathcal{M}$'s $A$-sequence $\vect{a}$, then the previous equation is false,
    since $\omega_{0}a_{0}=1$ and $\sum_{i_{1}+i_{2}=j}{\omega_{i_{1}}a_{i_{2}}}=0$,
    for each $j\in\lbrace1,\ldots,n-k\rbrace$, equality \emph{holds}. 
    A contraddiction occurs, as required.        
\end{proof}

\subsubsection{$A$-matrix}

We are now ready to tackle the most general characterization for a coefficient
$m_{nk}$ in a Riordan array $\mathcal{M}$, the concept of $A$-matrix. Actually,
we will see only one formulation but there exist another two of them, yet more
general indeed. The following theorem and the proof idea comes from 
\cite{merlini:some:alternative:characterizations:1997}, where additional
formulation of the $A$-matrix concept can be found.

\begin{theorem}
    $\mathcal{M}$ is a Riordan array, over a matrix of coefficients
    $\lbrace m_{n,k}\rbrace_{n,k\in\mathbb{N}}$, if there exists a matrix
    $\lbrace \sigma_{nk}\rbrace_{n,k\in\mathbb{N}}$ such that:
    \begin{displaymath}
        m_{n+1,k+1}=\sum_{i\in\mathbb{N}}{\sum_{j\in\mathbb{N}}{\sigma_{ij}m_{n-i,k+j}}}
    \end{displaymath}
\end{theorem}

\begin{proof}
    The idea underlying this proof is to show that a sequence 
    $\vect{a}=\lbrace a_{n}\rbrace_{n\in\mathbb{N}}$ can be built from
    matrix $\lbrace \sigma_{nk}\rbrace_{n,k\in\mathbb{N}}$,
    such that $\vect{a}$ can be used to combine elements in matrix
    $\lbrace m_{n,k}\rbrace_{n,k\in\mathbb{N}}$ according: 
    \begin{displaymath}
        m_{n+1,k+1}=a_{0}m_{nk}+a_{1}m_{n,k+1}+\ldots+a_{j}m_{n,k+j}
    \end{displaymath}
    where $k+j=n$, as usual.  If we success, then sequence $\vect{a}$ is the $A$-sequence for 
    $\lbrace m_{n,k}\rbrace_{n,k\in\mathbb{N}}$, therefore $\mathcal{M}$ is a Riordan array.
    \\\\
    Choose $n\in\mathbb{N}$, then proceed by natural induction on $n-k$:
    \begin{itemize}
        \item base case, $n-k=0$. By assumption the following holds:
            \begin{displaymath}
                m_{n+1,n+1}=\sigma_{00}\,m_{nn}
            \end{displaymath}
            which suggests to set $a_{0}=\sigma_{00}$;
        \item although not required, consider case $n-k=1$. By assumption the following holds:
            \begin{displaymath}
                m_{n+1,n}=\sigma_{00}m_{n,n-1}+\sigma_{01}m_{n,n}+\sigma_{10}m_{n-1,n-1}
            \end{displaymath}
            by \autoref{thm:characterization:next:row}, there exist a sequence $\vect{\omega}=
            \lbrace \omega_{n}\rbrace_{n\in\mathbb{N}}$ which allow us to rewrite $m_{n-1,n-1}$
            as a combination of $m_{nn}$:
            \begin{displaymath}
                m_{n+1,n}=\sigma_{00}m_{n,n-1}+\left(\sigma_{01}+\sigma_{10}\omega_{0}\right)m_{nn}
            \end{displaymath}
            which reinforces $a_{0}=\sigma_{00}$ and suggests to set 
            $a_{1}=\sigma_{01}+\sigma_{10}\omega_{0}$; moreover, using sequence $\vect{\omega}$
            to rewrite terms in the \ac{rhs}:
            \begin{displaymath}
                m_{n+1,n}=\sigma_{00}\left(\omega_{0}m_{n+1,n}+\omega_{1}m_{n+1,n+1}\right)+
                    \left(\frac{\sigma_{01}+\sigma_{10}\omega_{0}}{\sigma_{00}}\right)m_{n+1,n+1}
            \end{displaymath}
            manipulating:
            \begin{displaymath}
                m_{n+1,n}=
                    \left(\frac{\sigma_{00}^{2}\omega_{1}+\sigma_{01}+\sigma_{10}\omega_{0}}
                    {(1-\sigma_{00}\omega_{0})\sigma_{00}}\right)m_{n+1,n+1}
            \end{displaymath}
        \item \emph{induction hp} assume that, for $n-k=n-1$, ie. $k=1$, if:
            \begin{displaymath}
                m_{n+1,2}=\sum_{i\in\mathbb{N}}{\sum_{j\in\mathbb{N}}{\sigma_{ij}m_{n-i,1+j}}}
            \end{displaymath}
            then there exists a sequence $\vect{\hat{a}}$ such that:
            \begin{displaymath}
                m_{n+1,2}=\hat{a}_{0}m_{n1}+\hat{a}_{1}m_{n,2}+\ldots+\hat{a}_{j}m_{n,1+j}
            \end{displaymath}
            where $1+j=n$;
        \item \emph{induction step} show that, for $n-k=n$, ie. $k=0$, if:
            \begin{displaymath}
                m_{n+1,1}=\sum_{i\in\mathbb{N}}{\sum_{j\in\mathbb{N}}{\sigma_{ij}m_{n-i,j}}}
            \end{displaymath}
            then there exists a sequence $\vect{a}$ such that:
            \begin{displaymath}
                m_{n+1,1}=a_{0}m_{n0}+a_{1}m_{n,1}+\ldots+a_{n}m_{n,n}
            \end{displaymath}

            Begin by expanding the assumption:
            \begin{displaymath}
                \hspace{-2cm}
                \begin{split}
                    m_{n+1,1}&=\sigma_{00}m_{n0}+\sigma_{01}m_{n1}+\sigma_{02}m_{n2}+\ldots
                                +\sigma_{0,n-2}m_{n,n-2}+\sigma_{0,n-1}m_{n,n-1}+\sigma_{0n}m_{nn}\\
                             &+\sigma_{10}m_{n-1,0}+\sigma_{11}m_{n-1,1}+\sigma_{12}m_{n-1,2}+\ldots
                                +\sigma_{1,n-2}m_{n-1,n-2}+\sigma_{1,n-1}m_{n-1,n-1}\\
                             &+\sigma_{20}m_{n-2,0}+\sigma_{21}m_{n-2,1}+\sigma_{22}m_{n-2,2}+\ldots+\sigma_{2,n-2}m_{n-2,n-2}\\
                             &\ldots\\
                             &+\sigma_{n-1,0}m_{10}+\sigma_{n-1,1}m_{11}\\
                             &+\sigma_{n0}m_{00}\\
                \end{split}
            \end{displaymath}
            from the bottom line of the previous sum expansion, keep applying
            \autoref{thm:characterization:next:row} to every coefficient $m_{rc}$, for
            $r\in\lbrace0,\ldots,n-1\rbrace$ and, consequently, $c\in\lbrace0,\ldots,r\rbrace$.
            When every coefficient $m_{n-1,c}$, for $c\in\lbrace0,\ldots,n-1\rbrace$, has been
            expanded, coefficient $m_{n+1,1}$ is a combination of coefficients 
            $\lbrace m_{n0}\rbrace\cup\lbrace m_{n,1},m_{n,2}\ldots,m_{n,n}\rbrace$ using
            a sequence $\vect{\beta}$, namely:
            \begin{displaymath}
                m_{n+1,1}=\beta_{0}m_{n0}+\beta_{1}m_{n1}+\ldots+\beta_{n}m_{nn}
            \end{displaymath}
            By induction hypothesis, there exists a sequence $\vect{\hat{a}}$ such that:
            \begin{displaymath}
                m_{n+1,2}=\hat{a}_{0}m_{n1}+\hat{a}_{1}m_{n2}+\ldots+\hat{a}_{n-1}m_{nn}
            \end{displaymath}
            so, we can build a sequence $\vect{a}$ as follows:
            \begin{displaymath}
                \vect{a}=\left(\hat{a}_{0},\hat{a}_{1},\hat{a}_{2},\ldots,\hat{a}_{n-1},\beta_{n}\right)
            \end{displaymath}
            where $\beta_{n}$ can be possibly different from $\hat{a}_{n-1}$. 
            Sequence $\vect{a}$ is the $A$-sequence
            for matrix $\lbrace m_{nk}\rbrace_{n,k\in\mathbb{N}}$, therefore $\mathcal{M}$ is a Riordan
            array, as required.
    \end{itemize}
\end{proof}




















