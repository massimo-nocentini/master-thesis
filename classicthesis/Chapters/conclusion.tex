
\chapter{Conclusions}
\label{ch:conclusions}

We are approaching the end of our work, in this space we would like
to do a quick review of what we have done e point out some directions
for further studies.
\\\\

At the beginning we explain motivations underlying our interest for the
\emph{Riordan array} concepts and make a list of target we aims at.  Then, we
recall some basic definitions and facts about essential building blocks, such
as sequences, matrices, \ac{gf} and \ac{fps}. From here, a chronological
description of concepts, starting from a combinatorial triangle definition
toward alternative characterizations, has been provided to show how
\emph{Riordan arrays} have been defined. Two parallel studies have been shown:
a first one about the application of modular transformations to coefficients of
an array; a second one about an alternative way to denote an array.  Finally,
we have described our implementation of a subset of the \emph{Riordan group
theory} using the Python language, providing a detailed comment to an usual
\emph{study script}.
\\\\
\indent Modular transformations yield interesting patterns when modified arrays
are represented as triangles composed of coloured dots, where each colour is
associated to a class of the congruence relation. This should suggest relations
about combinatorial objects that are counted by coefficients lying on the
pattern of interest. Therefore, it could be possible to explain relations among
objects from a combinatorial point of view, in the spirit of
\cite{benjamin:quinn:proofs:really:count}. Moreover, we believe that further
interest studies concerns modular transformations which use congruence
relations where moduloes are higher powers of primes. Starting points are
\cite{sved:1988} and \cite{mclean:1974}.  Finally, we are interested to look
for a way of manipulation of modular arrays which avoid the computation of a
closed formula for their generic elements: at RART$2015$ conference, where
part of this work has been presented in a contributed talk, some experts of
the field suggests to investigate \emph{Mandelbrot} and \emph{Julia} sets.
\\\\
On a parallel track, a new way to denote a Riordan array has been provided, we
have called it $h$-characterization. Starting from an array $\mathcal{R}$
defined as usual, $\mathcal{R}=(d(t),h(t))$, it is possible to combine function $d$
and the compositional inverse $\hat{h}$ with a simple algebraic trick, to denote
the same array with $\mathcal{R}_{h(t)}(\gamma(h(t)),h(t))$, where $h(t)$ has
to been considered a \emph{variable}, for some function $\gamma$. This characterization
allows us to revise group operations from a new point of view and if the given array
belongs to the \emph{renewal subgroup}, then the first component of $\mathcal{R}_{h(t)}$
shows the $A$-sequence in function of $h(t)$, pretty nice. An open question is to look for 
meanings of function $\gamma$ when the given array is not in the renewal subgroup, take 
as a starting point \emph{Delannoy array} $\mathcal{D}$.

Using the proposed characterization, we have restated $A$-sequence and
$A$-matrix concepts, providing a new point of view to look at combinations of
coefficients when additional constraint in the combinations are desired. We
would like to enhance this power combinator device in order to consider
combinatorial sums where terms could be product of coefficients lying on
different rows in the array: this allows us to count those combined combinatorial
objects.
\\\\
The last point we left open to further developments is our Python implementation.
It is very minimal but strongly oriented to be extended, thanks to the core principle
underlying the design of our classes: \emph{messaging}. We are interested to implement
all subgroups and work together to the Korean team met at RART$2015$ in order to 
apply modular transformation to \emph{block-wise} \emph{Riordan arrays} belonging to 
the Riordan group in \emph{several variables}. 

