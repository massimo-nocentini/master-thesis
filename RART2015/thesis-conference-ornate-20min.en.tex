% $Header: /Users/joseph/Documents/LaTeX/beamer/solutions/conference-talks/conference-ornate-20min.en.tex,v 90e850259b8b 2007/01/28 20:48:30 tantau $

\documentclass[10pt,serif, professionalfont]{beamer}

% This file is a solution template for:

% - Talk at a conference/colloquium.
% - Talk length is about 20min.
% - Style is ornate.



% Copyright 2004 by Till Tantau <tantau@users.sourceforge.net>.
%
% In principle, this file can be redistributed and/or modified under
% the terms of the GNU Public License, version 2.
%
% However, this file is supposed to be a template to be modified
% for your own needs. For this reason, if you use this file as a
% template and not specifically distribute it as part of a another
% package/program, I grant the extra permission to freely copy and
% modify this file as you see fit and even to delete this copyright
% notice. 


\mode<presentation>
{
  %\usetheme{Singapore}
  %\usetheme{Goettingen}
  \usetheme{Pittsburgh}
  % or ...

  %\setbeamercovered{transparent}
  % or whatever (possibly just delete it)

  \usecolortheme{dove}
  %\usecolortheme{seagull}
  %\setbeamertemplate{blocks}[rounded][shadow=false]
  %\setbeamertemplate{background canvas}[vertical shading][bottom=white,top=structure.fg!25]
  %\setbeamercolor{block body}{bg=normal text.bg!90!black}
  %\setbeamertemplate{sidebar canvas left}[horizontal shading][left=white!40!black,right=black]
}


\usepackage[english]{babel}
% or whatever

\usepackage[utf8]{inputenc}
% or whatever

%\usepackage{times}
\usepackage[T1]{fontenc}
% Or whatever. Note that the encoding and the font should match. If T1
% does not look nice, try deleting the line with the fontenc.

\usepackage{concrete}

\usepackage{graphicx}
\usepackage{float}
\usepackage{amsmath}
\usepackage{amsthm}
\usepackage{amssymb}
\usepackage{caption}
\usepackage{bbold}

\newcommand{\myTitle}{Regolarit\`a nei Riordan array\xspace}
\newcommand{\myItalianTitle}{Regolarit\`a nei Riordan array\xspace}
\newcommand{\myEnglishTitle}{Patterns in Riordan arrays\xspace}
%\newcommand{\mySubtitle}{$\equiv_{p}$ and $h$ characterizations\xspace}
\newcommand{\myDegree}{\xspace}
\newcommand{\myName}{Massimo Nocentini\xspace}
\newcommand{\myProf}{Donatella Merlini\xspace}
\newcommand{\myOtherProf}{Put name here\xspace}
\newcommand{\mySupervisor}{Put name here\xspace}
\newcommand{\myFaculty}{{S}cuola di {S}cienze {M}atematiche, {F}isiche e {N}aturali\xspace}
\newcommand{\myDepartment}{Put data here\xspace}
\newcommand{\myUni}{Universit\`a degli Studi di Firenze\xspace}
\newcommand{\myLocation}{Firenze\xspace}
\newcommand{\myTime}{Anno Accademico 2014-2015\xspace}
\newcommand{\myVersion}{version 1.0\xspace}

\title%[Short Paper Title] % (optional, use only with long paper titles)
{Patterns in Riordan arrays}

%\subtitle {Include Only If Paper Has a Subtitle}

\author[Nocentini, Massimo] % (optional, use only with lots of authors)
{Massimo Nocentini}
% - Give the names in the same order as the appear in the paper.
% - Use the \inst{?} command only if the authors have different
%   affiliation.

\institute[Universit\`a degli Studi di Firenze] % (optional, but mostly needed)
{
  %\inst{1}%
  \inst{}%
    {S}cuola di {S}cienze {M}atematiche, {F}isiche e {N}aturali\\
    Corso di Laurea Magistrale in Informatica
  %\and
  %\inst{2}%
  %Department of Theoretical Philosophy\\
  %University of Elsewhere
    \begin{center}
        \includegraphics[width=3cm]{logo/unifi} \\ \medskip
    \end{center}
    {\small Professor: \emph{Donatella Merlini}}
  }
% - Use the \inst command only if there are several affiliations.
% - Keep it simple, no one is interested in your street address.


\date[\today] % (optional, should be abbreviation of conference name)
{Anno Accademico $2014-2015$}
% - Either use conference name or its abbreviation.
% - Not really informative to the audience, more for people (including
%   yourself) who are reading the slides online

\subject{Theoretical Computer Science}
% This is only inserted into the PDF information catalog. Can be left
% out. 



% If you have a file called "university-logo-filename.xxx", where xxx
% is a graphic format that can be processed by latex or pdflatex,
% resp., then you can add a logo as follows:

\pgfdeclareimage[height=0.5cm]{university-logo}{logo/unifi}
\logo{\pgfuseimage{university-logo}}



% Delete this, if you do not want the table of contents to pop up at
% the beginning of each subsection:
\AtBeginSubsection[]
{
  \begin{frame}<beamer>{Outline}
    \tableofcontents[currentsection,currentsubsection]
  \end{frame}
}


% If you wish to uncover everything in a step-wise fashion, uncomment
% the following command: 

%\beamerdefaultoverlayspecification{<+->}


\begin{document}


\begin{frame}
  \titlepage
\end{frame}

\begin{frame}{Outline}
  \tableofcontents
  % You might wish to add the option [pausesections]
\end{frame}


% Structuring a talk is a difficult task and the following structure
% may not be suitable. Here are some rules that apply for this
% solution: 

% - Exactly two or three sections (other than the summary).
% - At *most* three subsections per section.
% - Talk about 30s to 2min per frame. So there should be between about
%   15 and 30 frames, all told.

% - A conference audience is likely to know very little of what you
%   are going to talk about. So *simplify*!
% - In a 20min talk, getting the main ideas across is hard
%   enough. Leave out details, even if it means being less precise than
%   you think necessary.
% - If you omit details that are vital to the proof/implementation,
%   just say so once. Everybody will be happy with that.

\section{Motivation}

\subsection{Some definitions}

\begin{frame}{Riordan arrays, formally}
    Let $\mathcal{M}=\lbrace m_{nk}\rbrace_{n,k\in\mathbb{N}}$ be an \emph{infinite
    lower triangular matrix}:
    \begin{displaymath}
         m_{nk}\in\mathbb{Z} \quad\wedge\quad n < k \rightarrow m_{nk} = 0
    \end{displaymath}
    Let $t$ be an \emph{undeterminate variable} and denote with 
    $\mathcal{M}_{(t)}$ the following product: 
    %$\mathcal{M}_{(t)}$ is a vector of {fps} in $\mathbb{Z}[\![t]\!]$}:
    \begin{displaymath}
        \small
        \left[
            \begin{array}{cccccc}
                1 & t & t^{2} & t^{3} & t^{4} &\ldots
            \end{array}
        \right]
        \left[
            \begin{array}{cccccc}
                m_{00} & & & &  &\\
                m_{10} & m_{11} & & &  &\\
                m_{20} & m_{21}& m_{22}& &  &\\
                m_{30} & m_{31}& m_{32}& m_{33}&  &\\
                m_{40} & m_{41}& m_{42}& m_{43}& m_{44} &\\
                \vdots & \vdots& \vdots& \vdots& \vdots & \ddots\\
            \end{array}
        \right]
    \end{displaymath}
\end{frame}

\begin{frame}{Riordan arrays, formally}
    Therefore $\mathcal{M}_{(t)} =
        \left[
            \begin{array}{cccccc}
                m_{0}(t) & m_{1}(t) & m_{2}(t) & m_{3}(t) &m_{4}(t) & \ldots
            \end{array}
        \right]$ 
    where function $m_{j}$, for $j\in\mathbb{N}$, is a {fps} in the ring 
    $\mathbb{Z}[\![t]\!]$.
    
    \begin{definition}
        If there exists two analytic functions $d$
        and $h$, such that $d(0)\neq0$ and $h(0)=0 \wedge h^{\prime}(0)\neq0$, which
        satisfy:
        \begin{displaymath}
            m_{j}(t)=d(t)\,h(t)^{j} \quad \forall j\in\mathbb{N}
        \end{displaymath}
        then matrix $\mathcal{M}$ is called a \emph{Riordan array}, denoted by 
        $\mathcal{M}(d(t),h(t))$ 
    \end{definition}
\end{frame}

\begin{frame}{Riordan arrays, formally}

    Let $\mathcal{R}$ be a Riordan array and $p$ be a prime.
    
    \begin{block}{$\mathcal{R}_{\equiv_{p}}$}
        Define $\mathcal{R}_{\equiv_{p}}$
        as the \emph{congruent of} $\mathcal{R}$, modulo $p$, such that:
        \begin{displaymath}
            d_{nk}\in\mathcal{R} \leftrightarrow a \in\mathcal{R}_{\equiv_{p}}
            \quad \text{where} \quad d_{nk}\equiv_{p} a
        \end{displaymath}
    \end{block}

    \begin{block}{$\mathcal{R}^{(\alpha)}$}
        Define $\mathcal{R}^{(\alpha)}$
        as the \emph{principal $\alpha$-cluster of} $\mathcal{R}$, such that:
        \begin{displaymath}
            d_{nk}\in\mathcal{R}^{(\alpha)}\rightarrow d_{nk}\in\mathcal{R}
            \quad \text{where}\quad n\in\lbrace0,\ldots,p^{\alpha}-1\rbrace
        \end{displaymath}
    \end{block}
    

\end{frame}

\subsection{An example: the binomial triangle}

\begin{frame}{Riordan arrays, an example}
    What's the Riordan array $\mathcal{P}$ of the following matrix?
    \begin{displaymath} 
        %\hspace{-2.5cm}
        \left[
        \begin{array}{rrrrrrrrrr}
        1 &  &  &  &  &  &  &  &  &  \\
        1 & 1 &  &  &  &  &  &  &  &  \\
        1 & 2 & 1 &  &  &  &  &  &  &  \\
        1 & 3 & 3 & 1 &  &  &  &  &  &  \\
        1 & 4 & 6 & 4 & 1 &  &  &  &  &  \\
        1 & 5 & 10 & 10 & 5 & 1 &  &  &  &  \\
        1 & 6 & 15 & 20 & 15 & 6 & 1 &  &  &  \\
        1 & 7 & 21 & 35 & 35 & 21 & 7 & 1 &  &  \\
        1 & 8 & 28 & 56 & 70 & 56 & 28 & 8 & 1 &  \\
        1 & 9 & 36 & 84 & 126 & 126 & 84 & 36 & 9 & 1
        \end{array}
        \right] 
    \end{displaymath}
    $m_{nk}\in\mathcal{P}$ counts the number of subset of length $k$ of a set of length $n$
\end{frame}

\begin{frame}{Riordan arrays, an example}
    Coefficients lying on the very first column belongs to 
    sequence $\lbrace1\rbrace_{n\in\mathbb{N}}$, therefore function $m_{0}$
    satisfies:
    \begin{displaymath}
        m_{0}(t)=\frac{1}{1-t}
    \end{displaymath}
    therefore $m_{0}(t)=d(t)$, so function $d$ has been defined. \\
    \pause
    In order to find function $h$, coeffiecients lying on the second column 
    belongs to sequence $\lbrace n\rbrace_{n\in\mathbb{N}}$ of natural numbers,
    therefore function $m_{1}$ satisfies:
    \begin{displaymath}
        m_{1}(t)=d(t)\,h(t)=\frac{t}{(1-t)^{2}} \rightarrow
        h(t)=\frac{t}{1-t}
    \end{displaymath}
    defining function $h$. Therefore the array $\mathcal{P}$ we are looking for is:
    \begin{equation}
        \mathcal{P}=\left(\frac{1}{1-t},\frac{t}{1-t}\right)
        \label{eq:pascal:array:derived:for:example}
    \end{equation}
\end{frame}

\section{Modular transformations}

\subsection{Main idea}

\begin{frame}{Sierpinski gasket and main idea}
    \begin{columns}[T] % contents are top vertically aligned
        \begin{column}[T]{5cm} % alternative top-align that's better for graphics
              \onslide<1->{
                  \vskip-20pt plus.5fill
                  \input{pascal-tikz/main-idea-four-splitted/plain-numbers-include-figure}
              }
              \onslide<3->{
                  \vskip-20pt plus.5fill
                  
\begin{figure}[p]

    \noindent\makebox[\textwidth]{
        \centering
        %\includegraphics[width=0.8\textwidth]{../../sympy/catalan/coloured.pdf}

        % using *angle* property to rotate it is difficult to properly align it
        % in order to have a "real" matrix representation.
        \includegraphics[width=5cm, height=5cm, keepaspectratio=true]{pascal-tikz/main-idea-four-splitted/remainders.pdf}
    }

    % this 'particular' line is necessary to use `displaymath' environment
    % into the caption environment, togheter with the inclusion of 
    % `caption' package. See here for more explanation:
    % http://stackoverflow.com/questions/2716227/adding-an-equation-or-formula-to-a-figure-caption-in-latex
    \captionsetup{singlelinecheck=off}
    %\caption[.]{ }

    \label{fig:main:idea:remainders}

\end{figure}

              }
        \end{column}
        \begin{column}[T]{5cm} % each column can also be its own environment
              \onslide<2->{
                  \vskip-20pt plus.5fill
                  \input{pascal-tikz/main-idea-four-splitted/centered-numbers-include-figure}
              }
              \onslide<4->{
                  \vskip-20pt plus.5fill
                  \input{pascal-tikz/main-idea-four-splitted/dots-include-figure}
              }
        \end{column}
    \end{columns}
\end{frame}

\subsection{Previous Work}

\begin{frame}{Fine, Sved et al.}

    \begin{block}{Fine in \cite{Fine1947} exposes \emph{Lucas theorem}}
        Let $p$ be a prime, $a=(a_{0},\ldots,a_{k})_{p}$ and $b=(b_{0},\ldots,b_{k})_{p}$, for some $k\in\mathbb{N}$:
        \begin{displaymath}
            {{a}\choose{b}} \equiv_{p} {{a_{0}}\choose{b_{0}}}{{a_{1}}\choose{b_{1}}}
                \ldots{{a_{k}}\choose{b_{k}}}
        \end{displaymath}
    \end{block}
    \pause
    \begin{block}{Sved in \cite{Sved1998} introduces}
        \begin{itemize}
            \item {\normalsize a language to talk about triangles modulo some prime $p$}\\
                \footnotesize{\emph{principal cell}, \emph{principal $\alpha$-cluster}, \emph{zero-holes of order $\alpha$}}
            \item {\normalsize a graphic interpretation of \emph{Lucas theorem} }\\
                \footnotesize{each ${{a_i}\choose{b_i}}$ locates ${{a}\choose{b}}$ within a \emph{nest of triangles}}
            \item {\normalsize an application of \emph{Lucas theorem} to Stirling triangle}\\
                \footnotesize{via a \emph{mapping} over the numbers of the second kind}
            \item {\normalsize higher powers of primes}\\
                \footnotesize{emphasizing coefficient $d_{nk}$ such that 
                    $p^{\alpha}\mid d_{nk}$ while $p^{\alpha+1}\nmid d_{nk}$}
        \end{itemize}
    \end{block}
\end{frame}

\begin{frame}{Fine, Sved et al.}
    \begin{block}{Wolfram in \cite{Wolfram1984} shows}
        \begin{itemize}
            \item {\normalsize \emph{fractal dimensionality} of self-similar patterns in $\mathcal{P}_{\equiv_{2}}$}\\
                \footnotesize{solving recurrence $T(\frac{i}{2})=3\,T(i)$ yield $T(i)\approx i^{-\log_{2}{3}}$}
            \item {\normalsize $N(n)=\sum_{k=0}^{n}{\mathbb{1}_{{{n}\choose{k}}\equiv_{2}1}}$ is a very irregular function }\\
                \footnotesize{also $N(n)=2^{\phi_{1}(n)}$, where $ \phi_{1}((n_{0},n_{1},\ldots,n_{m})_{2})=
                    \sum_{j=0}^{m}{\mathbb{1}_{n_{j}=1} }$}
        \end{itemize}
    \end{block}
    \pause
    \begin{block}{Broomhead in \cite{Broomhead1972} shows}
        \begin{itemize}
            \item {\normalsize $N(n)=\sum_{k=0}^{n}{\mathbb{1}_{p\nmid{{n}\choose{k}}}}$ has a closed form in $\mathcal{P}$}\\
                \footnotesize{namely $N((n_{0},n_{1},\ldots,n_{m})_{2})=(n_{0}+1)(n_{1}+1)\cdots(n_{m}+1)$}
        \end{itemize}
    \end{block}
    \pause
    \begin{block}{McLean in \cite{Lean1974} generalizes}
        Broomhead and Sved work over higher powers of a prime $p$:
        \begin{itemize}
            \item defines \emph{$p$-index of $a$} as {$\nu_{p}\lbrace a\rbrace =t \leftrightarrow p^{t}\mid a \wedge p^{t+1}\nmid a$}
            \item proves {$\nu_{p}\lbrace {{n}\choose{k}}\rbrace $} equals the \emph{number of carries} in $n-_{p}k$
        \end{itemize}
    \end{block}
\end{frame}

\subsection{Associating a colour to each remainder class of $\equiv_{p}$}

%\subsection{Pascal}

\begin{frame}{$\mathcal{P}_{\equiv_{2}}$ where 
    $\textcolor{blue}{[0]_{2}},\textcolor{orange}{[1]_{2}}$}

    \input{../sympy/pascal/pascal-standard-ignore-negatives-centered-colouring-127-rows-mod2-partitioning-include-figure}
\end{frame}

\begin{frame}{$\mathcal{P}_{\equiv_{3}}$ where 
    $\textcolor{blue}{[0]_{3}},\textcolor{orange}{[1]_{3}},\textcolor{red}{[2]_{3}}$}

    \input{../sympy/pascal/pascal-standard-ignore-negatives-centered-colouring-127-rows-mod3-partitioning-include-figure}
\end{frame}
\begin{frame}{$\mathcal{P}_{\stackrel{\circ}{\equiv_{3}}}$ where 
    $\textcolor{blue}{[0]_{3}},\textcolor{orange}{\lbrace [1],[2] \rbrace_{3}}$}

    \input{../sympy/pascal/pascal-standard-ignore-negatives-centered-colouring-127-rows-multiples-of-3-partitioning-include-figure}
\end{frame}
\begin{frame}{$\mathcal{P}_{\stackrel{\circ}{\equiv_{5}}}$ where 
    $\textcolor{blue}{[0]_{5}},\textcolor{orange}{\lbrace [1],[2],[3],[4] \rbrace_{5}}$}

    \input{../sympy/pascal/pascal-standard-ignore-negatives-centered-colouring-127-rows-multiples-of-5-partitioning-include-figure}
\end{frame}
\begin{frame}{$\mathcal{P}_{\stackrel{\circ}{\equiv_{7}}}$ where 
    $\textcolor{blue}{[0]_{7}},\textcolor{orange}{\lbrace [1],[2],[3],[4],[5],[6] \rbrace_{7}}$}

    
\begin{figure}[H]

    \hspace{1cm}
    \noindent\makebox[\textwidth]{
        \centering
        %\includegraphics[width=0.8\textwidth]{../../sympy/catalan/coloured.pdf}

        % using *angle* property to rotate it is difficult to properly align it
        % in order to have a "real" matrix representation.
        \includegraphics[width=15cm, height=15cm, keepaspectratio=true]{../sympy/pascal/pascal-standard-ignore-negatives-centered-colouring-127-rows-multiples-of-7-partitioning-triangle.pdf}
    }

    % this 'particular' line is necessary to use `displaymath' environment
    % into the caption environment, togheter with the inclusion of 
    % `caption' package. See here for more explanation:
    % http://stackoverflow.com/questions/2716227/adding-an-equation-or-formula-to-a-figure-caption-in-latex
    \captionsetup{singlelinecheck=off}
    \caption[$\mathcal{P}_{\stackrel{\circ}{\equiv_{7}}}$]{ $\mathcal{P}_{\stackrel{\circ}{\equiv_{7}}}$ }

    \label{fig:pascal-standard-ignore-negatives-centered-colouring-127-rows-multiples-of-7-partitioning-triangle}

\end{figure}

\end{frame}



%\subsection{Catalan}
\begin{frame}{$\mathcal{C}_{\equiv_{2}}$ where 
    $\textcolor{blue}{[0]_{2}},\textcolor{orange}{[1]_{2}}$}

    \input{../sympy/catalan/Catalan-traditional-standard-handle-negatives-centered-colouring-127-rows-mod2-partitioning-include-figure.tex}
\end{frame}
%\begin{frame}{$\mathcal{C}_{\equiv_{3}}$ where 
    %$\textcolor{blue}{[0]_{3}},\textcolor{orange}{[1]_{3}},\textcolor{red}{[2]_{3}}$}

    %\input{../sympy/catalan/catalan-traditional-standard-handle-negatives-centered-colouring-127-rows-mod3-partitioning-include-figure}
%\end{frame}
\begin{frame}{$\mathcal{C}_{\stackrel{\circ}{\equiv_{3}}}$ where 
    $\textcolor{blue}{[0]_{3}},\textcolor{orange}{\lbrace [1],[2] \rbrace_{3}}$}

    \input{../sympy/catalan/catalan-traditional-standard-handle-negatives-centered-colouring-127-rows-multiples-of-3-partitioning-include-figure}
\end{frame}
\begin{frame}{$\mathcal{C}_{\stackrel{\circ}{\equiv_{5}}}$ where 
    $\textcolor{blue}{[0]_{5}},\textcolor{orange}{\lbrace [1],[2],[3],[4] \rbrace_{5}}$}

    \input{../sympy/catalan/catalan-traditional-standard-handle-negatives-centered-colouring-127-rows-multiples-of-5-partitioning-include-figure}
\end{frame}
\begin{frame}{$\mathcal{C}_{\stackrel{\circ}{\equiv_{7}}}$ where 
    $\textcolor{blue}{[0]_{7}},\textcolor{orange}{\lbrace [1],[2],[3],[4],[5],[6] \rbrace_{7}}$}

    \input{../sympy/catalan/catalan-traditional-standard-handle-negatives-centered-colouring-127-rows-multiples-of-7-partitioning-include-figure}
\end{frame}

%\subsection{Motzkin}

\begin{frame}{$\mathcal{M}_{\equiv_{2}}$ where 
    $\textcolor{blue}{[0]_{2}},\textcolor{orange}{[1]_{2}}$}

    \input{../sympy/motzkin/motzkin-standard-ignore-negatives-centered-colouring-127-rows-mod2-partitioning-include-figure.tex}
\end{frame}
%\begin{frame}{$\mathcal{M}_{\equiv_{3}}$ where 
    %$\textcolor{blue}{[0]_{3}},\textcolor{orange}{[1]_{3}},\textcolor{red}{[2]_{3}}$}

    %\input{../sympy/motzkin/motzkin-standard-ignore-negatives-centered-colouring-127-rows-mod3-partitioning-include-figure}
%\end{frame}
\begin{frame}{$\mathcal{M}_{\stackrel{\circ}{\equiv_{3}}}$ where 
    $\textcolor{blue}{[0]_{3}},\textcolor{orange}{\lbrace [1],[2] \rbrace_{3}}$}

    
\begin{figure}[H]

    \noindent\makebox[\textwidth]{
        \centering
        %\includegraphics[width=0.8\textwidth]{../../sympy/catalan/coloured.pdf}

        % using *angle* property to rotate it is difficult to properly align it
        % in order to have a "real" matrix representation.
        \includegraphics[width=20cm, height=20cm, keepaspectratio=true]{../sympy/motzkin/motzkin-standard-ignore-negatives-centered-colouring-127-rows-multiples-of-3-partitioning-triangle.pdf}
    }

    % this 'particular' line is necessary to use `displaymath' environment
    % into the caption environment, togheter with the inclusion of 
    % `caption' package. See here for more explanation:
    % http://stackoverflow.com/questions/2716227/adding-an-equation-or-formula-to-a-figure-caption-in-latex
    \captionsetup{singlelinecheck=off}
    \caption[$\mathcal{M}_{\stackrel{\circ}{\equiv_{3}}}$]{ $\mathcal{M}_{\stackrel{\circ}{\equiv_{3}}}$ }

    \label{fig:motzkin-standard-ignore-negatives-centered-colouring-127-rows-multiples-of-3-partitioning-triangle}

\end{figure}

\end{frame}
\begin{frame}{$\mathcal{M}_{\stackrel{\circ}{\equiv_{5}}}$ where 
    $\textcolor{blue}{[0]_{5}},\textcolor{orange}{\lbrace [1],[2],[3],[4] \rbrace_{5}}$}

    \input{../sympy/motzkin/motzkin-standard-ignore-negatives-centered-colouring-127-rows-multiples-of-5-partitioning-include-figure}
\end{frame}
\begin{frame}{$\mathcal{M}_{\stackrel{\circ}{\equiv_{7}}}$ where 
    $\textcolor{blue}{[0]_{7}},\textcolor{orange}{\lbrace [1],[2],[3],[4],[5],[6] \rbrace_{7}}$}

    \input{../sympy/motzkin/motzkin-standard-ignore-negatives-centered-colouring-127-rows-multiples-of-7-partitioning-include-figure}
\end{frame}


%\subsection{Delannoy}

\begin{frame}{$\mathcal{D}_{\stackrel{\circ}{\equiv_{3}}}$ where 
    $\textcolor{blue}{[0]_{3}},\textcolor{orange}{\lbrace [1],[2] \rbrace_{3}}$}

    \input{../sympy/delannoy/delannoy-standard-handle-negatives-centered-colouring-127-rows-multiples-of-3-partitioning-include-figure}
\end{frame}
\begin{frame}{$\mathcal{D}_{\stackrel{\circ}{\equiv_{5}}}$ where 
    $\textcolor{blue}{[0]_{5}},\textcolor{orange}{\lbrace [1],[2],[3],[4] \rbrace_{5}}$}

    
\begin{figure}[H]

    \hspace{1cm}
    \noindent\makebox[\textwidth]{
        \centering
        %\includegraphics[width=0.8\textwidth]{../../sympy/catalan/coloured.pdf}

        % using *angle* property to rotate it is difficult to properly align it
        % in order to have a "real" matrix representation.
        \includegraphics[width=15cm, height=15cm, keepaspectratio=true]{../sympy/delannoy/delannoy-standard-handle-negatives-centered-colouring-127-rows-multiples-of-5-partitioning-triangle.pdf}
    }

    % this 'particular' line is necessary to use `displaymath' environment
    % into the caption environment, togheter with the inclusion of 
    % `caption' package. See here for more explanation:
    % http://stackoverflow.com/questions/2716227/adding-an-equation-or-formula-to-a-figure-caption-in-latex
    \captionsetup{singlelinecheck=off}
    \caption[$\mathcal{D}_{\stackrel{\circ}{\equiv_{5}}}$]{ $\mathcal{D}_{\stackrel{\circ}{\equiv_{5}}}$ }

    \label{fig:delannoy-standard-handle-negatives-centered-colouring-127-rows-multiples-of-5-partitioning-triangle}

\end{figure}

\end{frame}
\begin{frame}{$\mathcal{D}_{\stackrel{\circ}{\equiv_{7}}}$ where 
    $\textcolor{blue}{[0]_{7}},\textcolor{orange}{\lbrace [1],[2],[3],[4],[5],[6] \rbrace_{7}}$}

    \input{../sympy/delannoy/delannoy-standard-handle-negatives-centered-colouring-127-rows-multiples-of-7-partitioning-include-figure}
\end{frame}

%\subsection{Fibonacci}

\begin{frame}{$\mathcal{F}_{\equiv_{2}}$ where 
    $\textcolor{blue}{[0]_{2}},\textcolor{orange}{[1]_{2}}$}

    
\begin{figure}[H]

    \hspace{-1.5cm}
    \noindent\makebox[\textwidth]{
        \centering
        %\includegraphics[width=0.8\textwidth]{../../sympy/catalan/coloured.pdf}

        % using *angle* property to rotate it is difficult to properly align it
        % in order to have a "real" matrix representation.
        \includegraphics[width=15cm, height=15cm, keepaspectratio=true]{../sympy/fibonacci/fibonacci-standard-handle-negatives-centered-colouring-127-rows-mod2-partitioning-triangle.pdf}
    }

    % this 'particular' line is necessary to use `displaymath' environment
    % into the caption environment, togheter with the inclusion of 
    % `caption' package. See here for more explanation:
    % http://stackoverflow.com/questions/2716227/adding-an-equation-or-formula-to-a-figure-caption-in-latex
    \captionsetup{singlelinecheck=off}
    \caption[$\mathcal{F}_{\stackrel{\circ}{\equiv_{2}}}$]{ $\mathcal{F}_{\stackrel{\circ}{\equiv_{2}}}$ }

    \label{fig:fibonacci-standard-handle-negatives-centered-colouring-127-rows-mod2-partitioning-triangle}

\end{figure}

\end{frame}
%\begin{frame}{$\mathcal{F}_{\equiv_{3}}$ where 
    %$\textcolor{blue}{[0]_{3}},\textcolor{orange}{[1]_{3}},\textcolor{red}{[2]_{3}}$}

    %\input{../sympy/fibonacci/fibonacci-standard-ignore-negatives-centered-colouring-127-rows-mod3-partitioning-include-figure}
%\end{frame}
\begin{frame}{$\mathcal{F}_{\stackrel{\circ}{\equiv_{3}}}$ where 
    $\textcolor{blue}{[0]_{3}},\textcolor{orange}{\lbrace [1],[2] \rbrace_{3}}$}

    \input{../sympy/fibonacci/fibonacci-standard-handle-negatives-centered-colouring-127-rows-multiples-of-3-partitioning-include-figure}
\end{frame}
%\begin{frame}{$\mathcal{F}_{\stackrel{\circ}{\equiv_{5}}}$ where 
    %$\textcolor{blue}{[0]_{5}},\textcolor{orange}{\lbrace [1],[2],[3],[4] \rbrace_{5}}$}

    %\input{../sympy/fibonacci/fibonacci-standard-handle-negatives-centered-colouring-127-rows-multiples-of-5-partitioning-include-figure}
%\end{frame}
%\begin{frame}{$\mathcal{F}_{\stackrel{\circ}{\equiv_{7}}}$ where 
    %$\textcolor{blue}{[0]_{7}},\textcolor{orange}{\lbrace [1],[2],[3],[4],[5],[6] \rbrace_{7}}$}

    %\input{../sympy/fibonacci/fibonacci-standard-handle-negatives-centered-colouring-127-rows-multiples-of-7-partitioning-include-figure}
%\end{frame}

\subsection{Observations}

\begin{frame}{Observations}
    \begin{itemize}
        \item for increasing values of prime $p$:
        \begin{itemize} 
            \item $\mathcal{P}_{\stackrel{\circ}{\equiv_{p}}}$ and
                $\mathcal{C}_{\stackrel{\circ}{\equiv_{p}}}$ have a \emph{recursive structure}, 
                like a fractal;
            \item $\mathcal{D}_{\stackrel{\circ}{\equiv_{p}}}$ has a common pattern but
                not completely recursive;
            \item $\mathcal{F}_{\stackrel{\circ}{\equiv_{p}}}$ and
                $\mathcal{M}_{\stackrel{\circ}{\equiv_{p}}}$ have a pattern but it seems
                to be augmented for each increment of $p$. 
        \end{itemize} 
        By \emph{recursive structure} we mean that $\mathcal{R}^{(\alpha+1)}$ can
        be built using combinations of  $\mathcal{R}^{(\alpha)}$ and \emph{zero-holes}
        of order $\alpha$ only

        \pause
        \item if $p$ isn't a prime, colouring gets ``noise'' and it's difficult
        to recognize the underlying pattern

        %\pause
        %\item $\mathcal{D}_{\equiv_{2}}$ isn't very interesting since
        %$d_{nk}\in\mathcal{D}_{\equiv_{2}}\rightarrow d_{nk}\equiv_{2}1$

        \pause
        \item colouring $\mathcal{R}^{-1}$, inverse of $\mathcal{R}$, uses darker
        and lighter colour variants for positive and negative entries, respectively
    \end{itemize} 
\end{frame}

\begin{frame}{$\mathcal{C}_{\equiv_{2}}^{-1}$ where 
    $\textcolor{blue}{[0]_{2}},\textcolor{orange}{[1]_{2}}$ \emph{ignoring} negatives}

    \input{../sympy/catalan/catalan-traditional-inverse-ignore-negatives-centered-colouring-127-rows-mod2-partitioning-include-figure}
\end{frame}

\begin{frame}{$\mathcal{C}_{\equiv_{3}}^{-1}$ where 
    $\textcolor{blue}{[0]_{3}},\textcolor{orange}{[1]_{3}},\textcolor{red}{[2]_{3}}$ \emph{with} negatives}

    \input{../sympy/catalan/catalan-traditional-inverse-handle-negatives-centered-colouring-127-rows-mod3-partitioning-include-figure}
\end{frame}

\section{Our Contribution}

\subsection{On $\mathcal{P}_{\equiv_{p}}$ and $\mathcal{P}_{\equiv_{p}}^{-1}$}

\begin{frame}{On $\mathcal{P}_{\equiv_{p}}$ and $\mathcal{P}_{\equiv_{p}}^{-1}$}
    \begin{theorem}
        let $d_{nk}\in\mathcal{P}$ and  $\hat{d}_{nk}\in\mathcal{P}^{-1}$, then:
            \begin{displaymath}
                (p-1)^{k+n}d_{nk}\equiv_{p} \hat{d}_{nk}
            \end{displaymath}
    \end{theorem}
      \vskip-20pt plus.5fill
    \begin{itemize}
        \item \emph{proof}: apply $\big[t^{n}\big]$ both to $\mathcal{P}$ and $\mathcal{P}^{-1}$,
            then set congruence $\equiv_{p}$ 
        \item $p=2 \rightarrow d_{nk}\equiv_{2}\hat{d}_{nk}$, therefore $\mathcal{P}_{\equiv_{2}}$ and 
            $\mathcal{P}_{\equiv_{2}}^{-1}$ looks the same
    \end{itemize}
    \pause
    \begin{theorem}
        $d_{nk}\in\mathcal{P}$ and $\hat{d}_{nk}\in\mathcal{P}^{-1}$, then:
            \begin{displaymath}
                d_{n,n-p^{m}} \equiv_{p} c \leftrightarrow \hat{d}_{n,n-p^{m}} \equiv_{p} p-c
            \end{displaymath}
    \end{theorem}
      \vskip-20pt plus.5fill
    \begin{itemize}
        \item \emph{proof}: simple application of the above theorem
        \item $c\in\lbrace0,\ldots,p-1\rbrace$ is usually seen as a colour
    \end{itemize}
\end{frame}

\begin{frame}{On $\mathcal{P}_{\equiv_{p}}$ and $\mathcal{P}_{\equiv_{p}}^{-1}$}
    \begin{theorem}
         $\mathcal{P}^{(\alpha+1)}$ contains 
        ${{p}\choose{2}}$ \emph{zero-holes} of order $\alpha$
    \end{theorem}
    \begin{itemize}
        \item \emph{proof}: $\mathcal{P}^{(\alpha+1)}$ contains $\lbrace l_{0},\ldots,l_{p-1}\rbrace$ 
            layers of order $\alpha$, each $l_{j}$ contains $j$ \emph{zero-holes}, therefore
            $\sum_{j=0}^{p-1}{j}=\frac{(p-1)p}{2}$
    \end{itemize}
    
    
\begin{figure}[p]

    \noindent\makebox[\textwidth]{
        \centering
        %\includegraphics[width=0.8\textwidth]{../../sympy/catalan/coloured.pdf}

        % using *angle* property to rotate it is difficult to properly align it
        % in order to have a "real" matrix representation.
        \includegraphics[width=6cm, height=6cm, keepaspectratio=true]{pascal-tikz/zero-holes/zero-holes.pdf}
    }

    % this 'particular' line is necessary to use `displaymath' environment
    % into the caption environment, togheter with the inclusion of 
    % `caption' package. See here for more explanation:
    % http://stackoverflow.com/questions/2716227/adding-an-equation-or-formula-to-a-figure-caption-in-latex
    \captionsetup{singlelinecheck=off}
    \caption[.]{$21$ zero-holes of order $1$ in $\mathcal{P}_{\equiv_{7}}^{(2)}$}

    \label{fig:pascal-zero-holes}

\end{figure}

\end{frame}

\begin{frame}{On $\mathcal{P}_{\equiv_{p}}$ and $\mathcal{P}_{\equiv_{p}}^{-1}$}
    \begin{theorem}
        consider $\mathcal{P}^{(m)}$, let $p^{m} \leq n < p^{m+1}$ and 
        $k\in\lbrace0,\ldots,n-p^{m}\rbrace$, then:
        \begin{displaymath}
            d_{n,k} \equiv_{p} d_{n+\gamma p^{m+1}, k+\gamma p^{m+1}}
                \quad\forall\gamma\in\mathbb{N}
        \end{displaymath}
    \end{theorem}
    \begin{itemize}
        \item \emph{proof}: represent subscripts in base $p$ and apply \emph{Lucas theorem}
        \item holds both if $d_{nk}\in\mathcal{P}$, both if $d_{nk}\in\mathcal{P}^{-1}$
    \end{itemize}

    
\begin{figure}[p]

    \noindent\makebox[\textwidth]{
        \centering
        %\includegraphics[width=0.8\textwidth]{../../sympy/catalan/coloured.pdf}

        % using *angle* property to rotate it is difficult to properly align it
        % in order to have a "real" matrix representation.
        \includegraphics[width=5cm, height=5cm, keepaspectratio=true]{pascal-tikz/multiples-over-antidiagonal/multiples-over-antidiagonal.pdf}
    }

    % this 'particular' line is necessary to use `displaymath' environment
    % into the caption environment, togheter with the inclusion of 
    % `caption' package. See here for more explanation:
    % http://stackoverflow.com/questions/2716227/adding-an-equation-or-formula-to-a-figure-caption-in-latex
    \captionsetup{singlelinecheck=off}
    \caption[.]{$\mathcal{P}^{(1)}$:
        \footnotesize{
        $\textcolor{blue}{d_{6,1}\equiv_{3}d_{15,10}\equiv_{3}d_{24,19}}$
        $\textcolor{red}{d_{8,3}\equiv_{3}d_{17,12}\equiv_{3}d_{26,21}}$
        $\textcolor{orange}{d_{8,0}\equiv_{3}d_{17,9}\equiv_{3}d_{26,18}}$ 
        }}

    \label{fig:pascal-multiples-over-antidiagonal}

\end{figure}

\end{frame}

\subsection{On $\mathcal{C}_{\equiv_{2}}$}

\begin{frame}{On $\mathcal{C}_{\equiv_{2}}$}
    \begin{columns}[T] % contents are top vertically aligned
        \begin{column}[T]{5cm} % alternative top-align that's better for graphics
            
\begin{figure}[p]

    \noindent\makebox[\textwidth]{
        \centering
        %\includegraphics[width=0.8\textwidth]{../../sympy/catalan/coloured.pdf}

        % using *angle* property to rotate it is difficult to properly align it
        % in order to have a "real" matrix representation.
        \includegraphics[
            width=7cm, 
            height=6cm, 
            keepaspectratio=true]{catalan-tikz/first-column/first-column.pdf}
    }

    % this 'particular' line is necessary to use `displaymath' environment
    % into the caption environment, togheter with the inclusion of 
    % `caption' package. See here for more explanation:
    % http://stackoverflow.com/questions/2716227/adding-an-equation-or-formula-to-a-figure-caption-in-latex
    \captionsetup{singlelinecheck=off}
    \caption[.]{ \textcolor{blue}{$d_{nk}$} if even,
        \textcolor{orange}{$d_{nk}$} otherwise  }

    \label{fig:catalan-first-column}

\end{figure}

        \end{column}
        \begin{column}[T]{5cm} % each column can also be its own environment
            let $d_{nk}\in\mathcal{C}$ and $c_{j}=d_{j0}$ be a Catalan number
            \begin{theorem}
                \begin{displaymath}
                    c_{j} \equiv_{2}1 \leftrightarrow j=2^{\alpha}-1
                \end{displaymath}
                where $\alpha\in\mathbb{N}$
            \end{theorem}
            \emph{proof}: $c_{j} = {{2j}\choose{j}} - {{2j}\choose{j+1}}$ and by cases on $j$'s parity
        \end{column}
    \end{columns}
\end{frame}

\begin{frame}{On $\mathcal{C}_{\equiv_{2}}$}
     %$\mathcal{C}_{-(0,p^{h}-1)}^{h}$ be a copy of 
        %$\mathcal{C}_{h}$:
        %\begin{itemize}
            %\item without $d_{nn}$ for each $n$ (antidiagonal $0$)
            %\item without $d_{p^{h}-1,k}$ for each $k$ (row $p^{h}-1$)
        %\end{itemize}
    %\pause

      % in order to use a `displaymath' environment for the proof
      \vskip-20pt plus.5fill
    \begin{theorem}
        let $h\in\mathbb{N}$ and $s\in\lbrace0,\ldots,2^{h}-1 \rbrace$, then:
        \begin{displaymath}
            d_{2^{h}-1,s} \equiv_{2} 1
        \end{displaymath}
    \end{theorem}
    \emph{proof}: 
        \begin{displaymath}
            d_{2^{\alpha}-1,s} = \sum_{i_{1}+i_{2}+\ldots+i_{s+1}=2^{\alpha}}
                {c_{i_{1}-1}\,c_{i_{2}-1}\ldots\,c_{i_{s+1}-1}}
        \end{displaymath}

    
\begin{figure}[p]

    \noindent\makebox[\textwidth]{
        \centering
        %\includegraphics[width=0.8\textwidth]{../../sympy/catalan/coloured.pdf}

        % using *angle* property to rotate it is difficult to properly align it
        % in order to have a "real" matrix representation.
        \includegraphics[width=10cm, height=3cm, keepaspectratio=true]{catalan-tikz/odd-row/odd-row.pdf}
    }

    % this 'particular' line is necessary to use `displaymath' environment
    % into the caption environment, togheter with the inclusion of 
    % `caption' package. See here for more explanation:
    % http://stackoverflow.com/questions/2716227/adding-an-equation-or-formula-to-a-figure-caption-in-latex
    \captionsetup{singlelinecheck=off}
    \caption[.]{Row of coefficients $\textcolor{orange}{d_{2^{4}-1,s}} \equiv_{2} 1$ 
        for $s\in\lbrace1,\ldots,2^{4}-1 \rbrace$ }

    \label{fig:catalan-odd-row}

\end{figure}

\end{frame}

\begin{frame}{On $\mathcal{C}_{\equiv_{2}}$}
      % in order to use a `displaymath' environment for the proof
      \vskip-15pt plus.5fill
    \begin{theorem}
        let  $S=\lbrace2^{\alpha},\ldots,2^{\alpha+1}-2\rbrace$ be a 
        \flqq mirror\frqq segment of column $2^{\alpha}-1$, then:
        \begin{displaymath} 
            d_{s,2^{\alpha}-1}\equiv_{2}0\quad\text{for}\quad s\in S
        \end{displaymath} 
    \end{theorem}
      \vskip-10pt plus.5fill
    \emph{proof}: 
        \begin{displaymath} 
            d_{s, 2^{\alpha}-1} = \sum_{i_{1}+i_{2}+\ldots+i_{2^{\alpha}}=s+1}
                {c_{i_{1}-1}\,c_{i_{2}-1}\ldots\,c_{i_{2^{\alpha}}-1}}
        \end{displaymath} 
    
    
\begin{figure}[p]

    \noindent\makebox[\textwidth]{
        \centering
        %\includegraphics[width=0.8\textwidth]{../../sympy/catalan/coloured.pdf}

        % using *angle* property to rotate it is difficult to properly align it
        % in order to have a "real" matrix representation.
        \includegraphics[width=10cm, height=10cm, keepaspectratio=true]{catalan-tikz/mirror-segment/mirror-segment.pdf}
    }

    % this 'particular' line is necessary to use `displaymath' environment
    % into the caption environment, togheter with the inclusion of 
    % `caption' package. See here for more explanation:
    % http://stackoverflow.com/questions/2716227/adding-an-equation-or-formula-to-a-figure-caption-in-latex
    \captionsetup{singlelinecheck=off}
    \caption[.]{for $s\in S=\lbrace2^{\alpha},\ldots,2^{\alpha+1}-2\rbrace$, segment of 
        $\textcolor{blue}{d_{s,2^{\alpha}-1}}\equiv_{2}0$ lying on column $2^{\alpha}-1$,
        where $\alpha=4$}



    \label{fig:mirror-segment}

\end{figure}

        
\end{frame}

\begin{frame}{On $\mathcal{C}_{\equiv_{2}}$}
    \begin{theorem}
    let $\hat{d}_{s,2^{h}-1}$ for $s\in S_{2^{h}-1}=\lbrace 2^{h}+1,\ldots,2^{h+1}-2 \rbrace$, then:
    \begin{displaymath}
        d_{s-e,2^{h}-1-e} \equiv_{2} d_{s,2^{h}-1+e}\quad\text{where}\quad
            e\in\lbrace1,\ldots,s-2^{h}\rbrace
    \end{displaymath}
    \end{theorem}
    \emph{proof}: $ d_{nk}\in\mathcal{C}\leftrightarrow 
        d_{nk}={{2n-k}\choose{n-k}} - {{2n-k}\choose{n-k-1}}$ and \emph{Lucas theorem}

    
\begin{figure}[p]

    \noindent\makebox[\textwidth]{
        \centering
        %\includegraphics[width=0.8\textwidth]{../../sympy/catalan/coloured.pdf}

        % using *angle* property to rotate it is difficult to properly align it
        % in order to have a "real" matrix representation.
        \includegraphics[width=10cm, height=10cm, keepaspectratio=true]
            {../RART2015/catalan-tikz/mirrored-clusters/mirrored-clusters.pdf}
    }

    % this 'particular' line is necessary to use `displaymath' environment
    % into the caption environment, togheter with the inclusion of 
    % `caption' package. See here for more explanation:
    % http://stackoverflow.com/questions/2716227/adding-an-equation-or-formula-to-a-figure-caption-in-latex
    \captionsetup{singlelinecheck=off}
    \caption[\emph{Mirrored} principal clusters $\mathcal{C}_{\equiv_{2}}^{(4)}$]
        {Some coefficients in \emph{mirrored} cluster $\mathcal{C}_{\equiv_{2}}^{(4)}$ 
        over $\hat{d}_{20,2^{4}-1}$: congruences $d_{20-e,2^{4}-1-e} \equiv_{2} d_{20,2^{4}-1+e}$ 
        for $e\in\lbrace1,\ldots,4\rbrace$ }

    \label{fig:catalan-mirrored-clusters}

\end{figure}

        
\end{frame}

\begin{frame}{On $\mathcal{C}_{\equiv_{2}}$}
    \begin{theorem}
    let $\hat{d}_{s,2^{h}-1}$ for $s\in S_{2^{h}-1}=\lbrace 2^{h}+1,\ldots,2^{h+1}-2 \rbrace$, then:
    \begin{displaymath}
        d_{s,2^{h}-1+e} \equiv_{2} d_{s-2^{h},e-1}\quad\text{where}\quad e\in\lbrace1,\ldots,s-2^{h}\rbrace
    \end{displaymath}
    \end{theorem}
    \emph{proof}: $ d_{nk}\in\mathcal{C}\leftrightarrow 
        d_{nk}={{2n-k}\choose{n-k}} - {{2n-k}\choose{n-k-1}}$ and \emph{Lucas theorem}

    
\begin{figure}[p]

    \noindent\makebox[\textwidth]{
        \centering
        %\includegraphics[width=0.8\textwidth]{../../sympy/catalan/coloured.pdf}

        % using *angle* property to rotate it is difficult to properly align it
        % in order to have a "real" matrix representation.
        \includegraphics[width=6cm, height=6cm, keepaspectratio=true]{../RART2015/catalan-tikz/principal-cluster/principal-cluster.pdf}
    }

    % this 'particular' line is necessary to use `displaymath' environment
    % into the caption environment, togheter with the inclusion of 
    % `caption' package. See here for more explanation:
    % http://stackoverflow.com/questions/2716227/adding-an-equation-or-formula-to-a-figure-caption-in-latex
    \captionsetup{singlelinecheck=off}
    \caption[$\mathcal{C}_{\equiv_{2}}^{(4)}$ 
    contains a copy of $\mathcal{C}_{\equiv_{2}}^{(3)}$]{$\hat{d}_{s,2^{3}-1}$ for $s\in S_{2^{3}-1}$,
    $d_{s,2^{3}-1+e} \equiv_{2} d_{s-2^{3},e-1}$ with $e\in\lbrace1,\ldots,s-2^{3}\rbrace$ }

    \label{fig:catalan-principal-cluster}

\end{figure}

        
\end{frame}

\begin{frame}{On $\mathcal{C}_{\equiv_{2}}$}
      % in order to use a `displaymath' environment for the proof
      \vskip-20pt plus.5fill
    \begin{theorem}
    let $\mathcal{T}_{\equiv_{2}0} \subset \mathcal{C}^{(\alpha+1)}$ be a %upside-down 
    \emph{zero-hole} with side $2^{\alpha}-1$, then:
    \begin{displaymath}
        d_{nk}\in\mathcal{T}_{\equiv_{2}0}\rightarrow d_{nk}\equiv_{2}0
    \end{displaymath}
    where $n\in\lbrace2^{\alpha},\ldots,2^{\alpha+1}-2\rbrace$ and $k\in\lbrace0,\ldots,n-2^{\alpha}\rbrace$. \emph{Proof}: 
    \end{theorem}
      \vskip-10pt plus.5fill
    \begin{displaymath}
        d_{s, u} = \sum_{i_{1}+\ldots+i_{u+1}=s+1}
            {c_{i_{1}-1}\cdots\,c_{i_{u+1}-1}}, \quad u\in\lbrace0,\ldots,2^{\alpha}-2\rbrace
    \end{displaymath}

    
\begin{figure}[htb]

    \noindent\makebox[\textwidth]{
        \centering
        %\includegraphics[width=0.8\textwidth]{../../sympy/catalan/coloured.pdf}

        % using *angle* property to rotate it is difficult to properly align it
        % in order to have a "real" matrix representation.
        \includegraphics[width=10cm, height=10cm, keepaspectratio=true]
            {../RART2015/catalan-tikz/zero-hole/zero-hole.pdf}
    }

    % this 'particular' line is necessary to use `displaymath' environment
    % into the caption environment, togheter with the inclusion of 
    % `caption' package. See here for more explanation:
    % http://stackoverflow.com/questions/2716227/adding-an-equation-or-formula-to-a-figure-caption-in-latex
    \captionsetup{singlelinecheck=off}
    \caption[Upside-down zero-hole $H_{\bigtriangleup}^{(4)}$ within
        $\mathcal{C}_{\equiv_{2}}^{(\alpha+1)}$]{Zero hole $H_{\bigtriangleup}^{(4)} \subset \mathcal{C}_{\equiv_{2}}^{(5)}$}

    \label{fig:catalan-zero-hole}

\end{figure}

        
\end{frame}

\subsection{$h$-characterization}

\begin{frame}{Another way to denote a Riordan array}
    Consider a \emph{Riordan array} $\mathcal{R}\left(d(t),h(t)\right)$ 
    By definition, coefficients lying on 
    column $k$ are the coefficients of $d(t)h(t)^k$.\\
    \pause
    Characterize $\mathcal{R}$ \emph{changing} variable $t$ by a simple
    algebraic trick:
    \begin{displaymath}
        d(t)h(t)^k = d(t)(1 + (h(t)-1))^k = \left[ \left. d(\hat{h}(1+y))(1+y)^k \right|y = h(t)-1  \right]
    \end{displaymath}
    where function $\hat{h}$ is the compositional inverse of $h$, the one that
    satisfies $\hat{h}(h(t)) = t$.
    \pause
    Inside the square brackets there's a shape of a Riordan array, therefore:
    \begin{displaymath}
        \begin{split}
            \mathcal{R}\left(d(t),h(t)\right) &= \left[ \mathcal{R}\left(d(\hat{h}(1+y)), 1+y\right) \left| y = h(t)-1 \right. \right]\\
            &= \mathcal{R}_{y=h(t)-1}\left( f(y), 1+y \right) =  \mathcal{R}_{h(t)}\left( g(h(t)), h(t) \right) 
        \end{split}
    \end{displaymath}
    We call $\mathcal{R}_{h(t)}$ the \emph{$h$-characterization} of array $\mathcal{R}$ 
\end{frame}

\begin{frame}{Another way to denote a Riordan array}
    \begin{displaymath} 
        \mathcal{P}_{h(t)}\left( 1+h(t), h(t) \right)
    \end{displaymath} 
    \begin{displaymath} 
        \mathcal{C}_{h(t)}\left(\frac{1}{1-h\left(t\right)}, h(t) \right)
    \end{displaymath} 
    \begin{displaymath} 
         \mathcal{D}_{h(t)}\left( \frac{2}{3+h(t)-\sqrt{h(t)^2+6h(t)+1}}, h(t) \right)
    \end{displaymath} 
    \pause
    \begin{theorem}
        Let $\mathcal{M}(d(t), h(t))$ be a Riordan array and $\mathcal{M}_{h(t)}\left( g(h(t)), h(t) \right) $
        its $h$-characterization. Then: $h(t)=t\,d(t) \rightarrow g(t)=A(t)$\\
        where function $A$ is the $A$-sequence of array $\mathcal{M}$, that is:
        \begin{displaymath}
            A(t)=\sum_{i\in\mathbb{N}}{a_{i}t^i} \leftrightarrow m_{n+1,k+1}=a_{0}m_{n,k}+a_{1}m_{n,k+1}+\ldots+a_{n-k}m_{n,n}
        \end{displaymath}
    \end{theorem}
\end{frame}

\section*{Summary}

\begin{frame}{Summary}

  % Keep the summary *very short*.
  \begin{itemize}
      \item representation of modular Riordan arrays in \LaTeX-ready code% (\TikZ)
      \item some properties about $\mathcal{P}$ and $\mathcal{P}^{-1}$ modulo a prime $p$
      \item a formal characterization of $\mathcal{C}_{\equiv_{2}}$
  \end{itemize}
  
  % The following outlook is optional.
  \vskip0pt plus.5fill
  \begin{itemize}
  \item
    WIP:
    \begin{itemize}
        \item on modular characterization of $\mathcal{C}^{-1}$, 
            nice pictures but hard to handle formally
        \item on understanding correspondence between modular results and 
            \emph{combinatorial interpretation} of coefficients in some arrays
        \item \emph{very in progress}: $h$-characterization!
    \end{itemize}
  \end{itemize}

  % The following outlook is optional.
  \vskip0pt plus.5fill
  \begin{itemize}
  \item
    Open problems:
    \begin{itemize}
        \item how to build a formal characterization of $\mathcal{F},\mathcal{M},\mathcal{D}$ and
            other arrays from the modular point of view?
        \item is there a way to build proofs manipulating functions $d$ and $h$ only? 
    \end{itemize}
  \end{itemize}
\end{frame}



% All of the following is optional and typically not needed. 
\appendix
\section<presentation>*{\appendixname}
\subsection<presentation>*{For Further Reading}

\begin{frame}[allowframebreaks]
  \frametitle<presentation>{For Further Reading}
    
  \begin{thebibliography}{10}
    
  \beamertemplatebookbibitems
  % Start with overview books.

    
  \beamertemplatearticlebibitems
  % Followed by interesting articles. Keep the list short. 

  \bibitem{Sved1998}
    Marta Sved.
    \newblock Divisibility - with visibility.
    \newblock {\em The Mathematical Intelligencer}, 10(2):56--64, 1998.

  \bibitem{Wolfram1984}
    Stephen Wolfram.
    \newblock Geometry of Binomial Coefficients.
    \newblock {\em The American Mathematical Monthly}, 91(9):566--571, 1984.

  \bibitem{Broomhead1972}
    Antony Broomhead.
    \newblock Pascal (mod $p$).
    \newblock {\em The Mathematical Gazette}, 56(398):267--271, 1972.

  \bibitem{Lean1974}
    K. R. McLean.
    \newblock Divisibility Properties of Binomial Coefficients.
    \newblock {\em The Mathematical Gazette}, 58(403):17--24, 1974.

  \bibitem{Fine1947}
    N. J. Fine.
    \newblock Binomial Coefficients Modulo a Prime.
    \newblock {\em The American Mathematical Monthly}, 54(10):589--592, 1947.
  \end{thebibliography}
\end{frame}

\begin{frame}
    \begin{center}
        \Large Thanks everybody!
    \end{center}
\end{frame}

\end{document}


