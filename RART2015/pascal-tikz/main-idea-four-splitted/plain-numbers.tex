%%%%%%%%%%%%%%%%%%%%%%%%%%%
% Author : Paul Gaborit (2009)
% under Creative Commons attribution license.
% Title : Pascal's triangle and Sierpinski triangle
% Note : 17 lines maximum
\documentclass[landscape]{article}
\usepackage[landscape,margin=1cm]{geometry}
\pagestyle{empty}
\usepackage[T1]{fontenc}
\usepackage{lmodern}
\usepackage{concrete}

\usepackage{tikz}
\usetikzlibrary{positioning,shadows,backgrounds}
\usetikzlibrary{external}

%%%<
\usepackage{verbatim}
\usepackage[active,tightpage]{preview}
\PreviewEnvironment{tikzpicture}
\setlength\PreviewBorder{5pt}%
%%%>

%\tikzexternalize % activate!
\begin{comment}
:Title:  Pascal's triangle and Sierpinski triangle

\end{comment}


\begin{document}
\centering

\begin{tikzpicture}[x=13mm,y=9mm]
  % some colors
  
%  \colorlet{even}{cyan!60!black}
%  \colorlet{odd}{orange!100!black}
%  \colorlet{links}{red!70!black}
%  \colorlet{back}{yellow!20!white}

%  \colorlet{zero}{cyan!80!black}
%  \colorlet{one}{orange!80!black}
%  \colorlet{two}{blue!80!black}
%  \colorlet{three}{red!80!black}

  \colorlet{0}{blue!100!black}
  \colorlet{1}{orange!100!black}
  \colorlet{2}{red!100!black}
  \colorlet{3}{cyan!100!black}
  \colorlet{4}{magenta!100!black}
  \colorlet{5}{violet!100!black}
  \colorlet{6}{black!100!black}
  \colorlet{7}{gray!100!black}
  \colorlet{8}{darkgray!100!black}
  \colorlet{9}{lightgray!100!black}
  \colorlet{10}{brown!100!black}
  \colorlet{11}{lime!100!black}
  \colorlet{12}{olive!100!black}
  \colorlet{13}{green!100!black}
  \colorlet{14}{pink!100!black}
  \colorlet{15}{purple!100!black}
  \colorlet{16}{teal!100!black}
  \colorlet{17}{yellow!100!black}
  \colorlet{18}{white!100!black}

  \colorlet{0-for-negatives}{blue!40!white}
  \colorlet{1-for-negatives}{orange!40!white}
  \colorlet{2-for-negatives}{red!40!white}
  \colorlet{3-for-negatives}{cyan!40!white}
  \colorlet{4-for-negatives}{magenta!40!white}
  \colorlet{5-for-negatives}{violet!40!white}
  \colorlet{6-for-negatives}{white!40!white}
  \colorlet{7-for-negatives}{gray!40!white}
  \colorlet{8-for-negatives}{darkgray!40!white}
  \colorlet{9-for-negatives}{lightgray!40!white}
  \colorlet{10-for-negatives}{brown!40!white}
  \colorlet{11-for-negatives}{lime!40!white}
  \colorlet{12-for-negatives}{olive!40!white}
  \colorlet{13-for-negatives}{green!40!white}
  \colorlet{14-for-negatives}{pink!40!white}
  \colorlet{15-for-negatives}{purple!40!white}
  \colorlet{16-for-negatives}{teal!40!white}
  \colorlet{17-for-negatives}{yellow!40!white}
  \colorlet{18-for-negatives}{white!40!white}

  \colorlet{blue}{blue!100!black}
  \colorlet{orange}{orange!100!black}
  \colorlet{red}{red!100!black}
  \colorlet{cyan}{cyan!100!black}
  \colorlet{magenta}{magenta!100!black}
  \colorlet{violet}{violet!100!black}
  \colorlet{black}{black!100!black}
  \colorlet{gray}{gray!100!black}
  \colorlet{darkgray}{darkgray!100!black}
  \colorlet{lightgray}{lightgray!100!black}
  \colorlet{brown}{brown!100!black}
  \colorlet{lime}{lime!100!black}
  \colorlet{olive}{olive!100!black}
  \colorlet{green}{green!100!black}
  \colorlet{pink}{pink!100!black}
  \colorlet{purple}{purple!100!black}
  \colorlet{teal}{teal!100!black}
  \colorlet{yellow}{yellow!100!black}
  \colorlet{white}{white!100!black}

  \colorlet{blue-for-negatives}{blue!40!white}
  \colorlet{orange-for-negatives}{orange!40!white}
  \colorlet{red-for-negatives}{red!40!white}
  \colorlet{cyan-for-negatives}{cyan!40!white}
  \colorlet{magenta-for-negatives}{magenta!40!white}
  \colorlet{violet-for-negatives}{violet!40!white}
  \colorlet{white-for-negatives}{white!40!white}
  \colorlet{gray-for-negatives}{gray!40!white}
  \colorlet{darkgray-for-negatives}{darkgray!40!white}
  \colorlet{lightgray-for-negatives}{lightgray!40!white}
  \colorlet{brown-for-negatives}{brown!40!white}
  \colorlet{lime-for-negatives}{lime!40!white}
  \colorlet{olive-for-negatives}{olive!40!white}
  \colorlet{green-for-negatives}{green!40!white}
  \colorlet{pink-for-negatives}{pink!40!white}
  \colorlet{purple-for-negatives}{purple!40!white}
  \colorlet{teal-for-negatives}{teal!40!white}
  \colorlet{yellow-for-negatives}{yellow!40!white}
  \colorlet{white-for-negatives}{white!40!white}


  % some styles
  \tikzset{
    box/.style={
        circle,
      minimum height=5mm,
      inner sep=.7mm,
      outer sep=0mm,
      text width=10mm,
      text centered,
      font=\small\bfseries\sffamily,
      text=#1!50!black,
      draw=#1,
      line width=.25mm,
      top color=#1!5,
      bottom color=#1!40,
      shading angle=0,
      rounded corners=2.3mm,
      drop shadow={fill=#1!40!gray,fill opacity=.8},
      rotate=0,
    },
%    link/.style={-latex,links,line width=.3mm},
%    plus/.style={text=links,font=\footnotesize\bfseries\sffamily},
  }
  % the following key `pascal-standard-ignore-negatives-plain-colouring-127-rows-mod2-partitioning-tikz-nodes.tex' is used as a template key
  % to allow Python to do string substitution
  \LARGE
\node (p-0-0) at (0,-0) {$1$};
\node (p-1-0) at (0,-1) {$1$};
\node (p-1-1) at (1,-1) {$1$};
\node (p-2-0) at (0,-2) {$1$};
\node (p-2-1) at (1,-2) {$2$};
\node (p-2-2) at (2,-2) {$1$};
\node (p-3-0) at (0,-3) {$1$};
\node (p-3-1) at (1,-3) {$3$};
\node (p-3-2) at (2,-3) {$3$};
\node (p-3-3) at (3,-3) {$1$};
\node (p-4-0) at (0,-4) {$1$};
\node (p-4-1) at (1,-4) {$4$};
\node (p-4-2) at (2,-4) {$6$};
\node (p-4-3) at (3,-4) {$4$};
\node (p-4-4) at (4,-4) {$1$};
\node (p-5-0) at (0,-5) {$1$};
\node (p-5-1) at (1,-5) {$5$};
\node (p-5-2) at (2,-5) {$10$};
\node (p-5-3) at (3,-5) {$10$};
\node (p-5-4) at (4,-5) {$5$};
\node (p-5-5) at (5,-5) {$1$};
\node (p-6-0) at (0,-6) {$1$};
\node (p-6-1) at (1,-6) {$6$};
\node (p-6-2) at (2,-6) {$15$};
\node (p-6-3) at (3,-6) {$20$};
\node (p-6-4) at (4,-6) {$15$};
\node (p-6-5) at (5,-6) {$6$};
\node (p-6-6) at (6,-6) {$1$};
\node (p-7-0) at (0,-7) {$1$};
\node (p-7-1) at (1,-7) {$7$};
\node (p-7-2) at (2,-7) {$21$};
\node (p-7-3) at (3,-7) {$35$};
\node (p-7-4) at (4,-7) {$35$};
\node (p-7-5) at (5,-7) {$21$};
\node (p-7-6) at (6,-7) {$7$};
\node (p-7-7) at (7,-7) {$1$};
%\node[box=1] (p-8-0) at (0,-8) {};
%\node[box=0] (p-8-1) at (1,-8) {};
%\node[box=0] (p-8-2) at (2,-8) {};
%\node[box=0] (p-8-3) at (3,-8) {};
%\node[box=0] (p-8-4) at (4,-8) {};
%\node[box=0] (p-8-5) at (5,-8) {};
%\node[box=0] (p-8-6) at (6,-8) {};
%\node[box=0] (p-8-7) at (7,-8) {};
%\node[box=1] (p-8-8) at (8,-8) {};
%\node[box=1] (p-9-0) at (0,-9) {};
%\node[box=1] (p-9-1) at (1,-9) {};
%\node[box=0] (p-9-2) at (2,-9) {};
%\node[box=0] (p-9-3) at (3,-9) {};
%\node[box=0] (p-9-4) at (4,-9) {};
%\node[box=0] (p-9-5) at (5,-9) {};
%\node[box=0] (p-9-6) at (6,-9) {};
%\node[box=0] (p-9-7) at (7,-9) {};
%\node[box=1] (p-9-8) at (8,-9) {};
%\node[box=1] (p-9-9) at (9,-9) {};
%\node[box=1] (p-10-0) at (0,-10) {};
%\node[box=0] (p-10-1) at (1,-10) {};
%\node[box=1] (p-10-2) at (2,-10) {};
%\node[box=0] (p-10-3) at (3,-10) {};
%\node[box=0] (p-10-4) at (4,-10) {};
%\node[box=0] (p-10-5) at (5,-10) {};
%\node[box=0] (p-10-6) at (6,-10) {};
%\node[box=0] (p-10-7) at (7,-10) {};
%\node[box=1] (p-10-8) at (8,-10) {};
%\node[box=0] (p-10-9) at (9,-10) {};
%\node[box=1] (p-10-10) at (10,-10) {};
%\node[box=1] (p-11-0) at (0,-11) {};
%\node[box=1] (p-11-1) at (1,-11) {};
%\node[box=1] (p-11-2) at (2,-11) {};
%\node[box=1] (p-11-3) at (3,-11) {};
%\node[box=0] (p-11-4) at (4,-11) {};
%\node[box=0] (p-11-5) at (5,-11) {};
%\node[box=0] (p-11-6) at (6,-11) {};
%\node[box=0] (p-11-7) at (7,-11) {};
%\node[box=1] (p-11-8) at (8,-11) {};
%\node[box=1] (p-11-9) at (9,-11) {};
%\node[box=1] (p-11-10) at (10,-11) {};
%\node[box=1] (p-11-11) at (11,-11) {};
%\node[box=1] (p-12-0) at (0,-12) {};
%\node[box=0] (p-12-1) at (1,-12) {};
%\node[box=0] (p-12-2) at (2,-12) {};
%\node[box=0] (p-12-3) at (3,-12) {};
%\node[box=1] (p-12-4) at (4,-12) {};
%\node[box=0] (p-12-5) at (5,-12) {};
%\node[box=0] (p-12-6) at (6,-12) {};
%\node[box=0] (p-12-7) at (7,-12) {};
%\node[box=1] (p-12-8) at (8,-12) {};
%\node[box=0] (p-12-9) at (9,-12) {};
%\node[box=0] (p-12-10) at (10,-12) {};
%\node[box=0] (p-12-11) at (11,-12) {};
%\node[box=1] (p-12-12) at (12,-12) {};
%\node[box=1] (p-13-0) at (0,-13) {};
%\node[box=1] (p-13-1) at (1,-13) {};
%\node[box=0] (p-13-2) at (2,-13) {};
%\node[box=0] (p-13-3) at (3,-13) {};
%\node[box=1] (p-13-4) at (4,-13) {};
%\node[box=1] (p-13-5) at (5,-13) {};
%\node[box=0] (p-13-6) at (6,-13) {};
%\node[box=0] (p-13-7) at (7,-13) {};
%\node[box=1] (p-13-8) at (8,-13) {};
%\node[box=1] (p-13-9) at (9,-13) {};
%\node[box=0] (p-13-10) at (10,-13) {};
%\node[box=0] (p-13-11) at (11,-13) {};
%\node[box=1] (p-13-12) at (12,-13) {};
%\node[box=1] (p-13-13) at (13,-13) {};
%\node[box=1] (p-14-0) at (0,-14) {};
%\node[box=0] (p-14-1) at (1,-14) {};
%\node[box=1] (p-14-2) at (2,-14) {};
%\node[box=0] (p-14-3) at (3,-14) {};
%\node[box=1] (p-14-4) at (4,-14) {};
%\node[box=0] (p-14-5) at (5,-14) {};
%\node[box=1] (p-14-6) at (6,-14) {};
%\node[box=0] (p-14-7) at (7,-14) {};
%\node[box=1] (p-14-8) at (8,-14) {};
%\node[box=0] (p-14-9) at (9,-14) {};
%\node[box=1] (p-14-10) at (10,-14) {};
%\node[box=0] (p-14-11) at (11,-14) {};
%\node[box=1] (p-14-12) at (12,-14) {};
%\node[box=0] (p-14-13) at (13,-14) {};
%\node[box=1] (p-14-14) at (14,-14) {};
%\node[box=1] (p-15-0) at (0,-15) {};
%\node[box=1] (p-15-1) at (1,-15) {};
%\node[box=1] (p-15-2) at (2,-15) {};
%\node[box=1] (p-15-3) at (3,-15) {};
%\node[box=1] (p-15-4) at (4,-15) {};
%\node[box=1] (p-15-5) at (5,-15) {};
%\node[box=1] (p-15-6) at (6,-15) {};
%\node[box=1] (p-15-7) at (7,-15) {};
%\node[box=1] (p-15-8) at (8,-15) {};
%\node[box=1] (p-15-9) at (9,-15) {};
%\node[box=1] (p-15-10) at (10,-15) {};
%\node[box=1] (p-15-11) at (11,-15) {};
%\node[box=1] (p-15-12) at (12,-15) {};
%\node[box=1] (p-15-13) at (13,-15) {};
%\node[box=1] (p-15-14) at (14,-15) {};
%\node[box=1] (p-15-15) at (15,-15) {};
%\node[box=1] (p-16-0) at (0,-16) {};
%\node[box=0] (p-16-1) at (1,-16) {};
%\node[box=0] (p-16-2) at (2,-16) {};
%\node[box=0] (p-16-3) at (3,-16) {};
%\node[box=0] (p-16-4) at (4,-16) {};
%\node[box=0] (p-16-5) at (5,-16) {};
%\node[box=0] (p-16-6) at (6,-16) {};
%\node[box=0] (p-16-7) at (7,-16) {};
%\node[box=0] (p-16-8) at (8,-16) {};
%\node[box=0] (p-16-9) at (9,-16) {};
%\node[box=0] (p-16-10) at (10,-16) {};
%\node[box=0] (p-16-11) at (11,-16) {};
%\node[box=0] (p-16-12) at (12,-16) {};
%\node[box=0] (p-16-13) at (13,-16) {};
%\node[box=0] (p-16-14) at (14,-16) {};
%\node[box=0] (p-16-15) at (15,-16) {};
%\node[box=1] (p-16-16) at (16,-16) {};
%\node[box=1] (p-17-0) at (0,-17) {};
%\node[box=1] (p-17-1) at (1,-17) {};
%\node[box=0] (p-17-2) at (2,-17) {};
%\node[box=0] (p-17-3) at (3,-17) {};
%\node[box=0] (p-17-4) at (4,-17) {};
%\node[box=0] (p-17-5) at (5,-17) {};
%\node[box=0] (p-17-6) at (6,-17) {};
%\node[box=0] (p-17-7) at (7,-17) {};
%\node[box=0] (p-17-8) at (8,-17) {};
%\node[box=0] (p-17-9) at (9,-17) {};
%\node[box=0] (p-17-10) at (10,-17) {};
%\node[box=0] (p-17-11) at (11,-17) {};
%\node[box=0] (p-17-12) at (12,-17) {};
%\node[box=0] (p-17-13) at (13,-17) {};
%\node[box=0] (p-17-14) at (14,-17) {};
%\node[box=0] (p-17-15) at (15,-17) {};
%\node[box=1] (p-17-16) at (16,-17) {};
%\node[box=1] (p-17-17) at (17,-17) {};
%\node[box=1] (p-18-0) at (0,-18) {};
%\node[box=0] (p-18-1) at (1,-18) {};
%\node[box=1] (p-18-2) at (2,-18) {};
%\node[box=0] (p-18-3) at (3,-18) {};
%\node[box=0] (p-18-4) at (4,-18) {};
%\node[box=0] (p-18-5) at (5,-18) {};
%\node[box=0] (p-18-6) at (6,-18) {};
%\node[box=0] (p-18-7) at (7,-18) {};
%\node[box=0] (p-18-8) at (8,-18) {};
%\node[box=0] (p-18-9) at (9,-18) {};
%\node[box=0] (p-18-10) at (10,-18) {};
%\node[box=0] (p-18-11) at (11,-18) {};
%\node[box=0] (p-18-12) at (12,-18) {};
%\node[box=0] (p-18-13) at (13,-18) {};
%\node[box=0] (p-18-14) at (14,-18) {};
%\node[box=0] (p-18-15) at (15,-18) {};
%\node[box=1] (p-18-16) at (16,-18) {};
%\node[box=0] (p-18-17) at (17,-18) {};
%\node[box=1] (p-18-18) at (18,-18) {};
%\node[box=1] (p-19-0) at (0,-19) {};
%\node[box=1] (p-19-1) at (1,-19) {};
%\node[box=1] (p-19-2) at (2,-19) {};
%\node[box=1] (p-19-3) at (3,-19) {};
%\node[box=0] (p-19-4) at (4,-19) {};
%\node[box=0] (p-19-5) at (5,-19) {};
%\node[box=0] (p-19-6) at (6,-19) {};
%\node[box=0] (p-19-7) at (7,-19) {};
%\node[box=0] (p-19-8) at (8,-19) {};
%\node[box=0] (p-19-9) at (9,-19) {};
%\node[box=0] (p-19-10) at (10,-19) {};
%\node[box=0] (p-19-11) at (11,-19) {};
%\node[box=0] (p-19-12) at (12,-19) {};
%\node[box=0] (p-19-13) at (13,-19) {};
%\node[box=0] (p-19-14) at (14,-19) {};
%\node[box=0] (p-19-15) at (15,-19) {};
%\node[box=1] (p-19-16) at (16,-19) {};
%\node[box=1] (p-19-17) at (17,-19) {};
%\node[box=1] (p-19-18) at (18,-19) {};
%\node[box=1] (p-19-19) at (19,-19) {};
%\node[box=1] (p-20-0) at (0,-20) {};
%\node[box=0] (p-20-1) at (1,-20) {};
%\node[box=0] (p-20-2) at (2,-20) {};
%\node[box=0] (p-20-3) at (3,-20) {};
%\node[box=1] (p-20-4) at (4,-20) {};
%\node[box=0] (p-20-5) at (5,-20) {};
%\node[box=0] (p-20-6) at (6,-20) {};
%\node[box=0] (p-20-7) at (7,-20) {};
%\node[box=0] (p-20-8) at (8,-20) {};
%\node[box=0] (p-20-9) at (9,-20) {};
%\node[box=0] (p-20-10) at (10,-20) {};
%\node[box=0] (p-20-11) at (11,-20) {};
%\node[box=0] (p-20-12) at (12,-20) {};
%\node[box=0] (p-20-13) at (13,-20) {};
%\node[box=0] (p-20-14) at (14,-20) {};
%\node[box=0] (p-20-15) at (15,-20) {};
%\node[box=1] (p-20-16) at (16,-20) {};
%\node[box=0] (p-20-17) at (17,-20) {};
%\node[box=0] (p-20-18) at (18,-20) {};
%\node[box=0] (p-20-19) at (19,-20) {};
%\node[box=1] (p-20-20) at (20,-20) {};
%\node[box=1] (p-21-0) at (0,-21) {};
%\node[box=1] (p-21-1) at (1,-21) {};
%\node[box=0] (p-21-2) at (2,-21) {};
%\node[box=0] (p-21-3) at (3,-21) {};
%\node[box=1] (p-21-4) at (4,-21) {};
%\node[box=1] (p-21-5) at (5,-21) {};
%\node[box=0] (p-21-6) at (6,-21) {};
%\node[box=0] (p-21-7) at (7,-21) {};
%\node[box=0] (p-21-8) at (8,-21) {};
%\node[box=0] (p-21-9) at (9,-21) {};
%\node[box=0] (p-21-10) at (10,-21) {};
%\node[box=0] (p-21-11) at (11,-21) {};
%\node[box=0] (p-21-12) at (12,-21) {};
%\node[box=0] (p-21-13) at (13,-21) {};
%\node[box=0] (p-21-14) at (14,-21) {};
%\node[box=0] (p-21-15) at (15,-21) {};
%\node[box=1] (p-21-16) at (16,-21) {};
%\node[box=1] (p-21-17) at (17,-21) {};
%\node[box=0] (p-21-18) at (18,-21) {};
%\node[box=0] (p-21-19) at (19,-21) {};
%\node[box=1] (p-21-20) at (20,-21) {};
%\node[box=1] (p-21-21) at (21,-21) {};
%\node[box=1] (p-22-0) at (0,-22) {};
%\node[box=0] (p-22-1) at (1,-22) {};
%\node[box=1] (p-22-2) at (2,-22) {};
%\node[box=0] (p-22-3) at (3,-22) {};
%\node[box=1] (p-22-4) at (4,-22) {};
%\node[box=0] (p-22-5) at (5,-22) {};
%\node[box=1] (p-22-6) at (6,-22) {};
%\node[box=0] (p-22-7) at (7,-22) {};
%\node[box=0] (p-22-8) at (8,-22) {};
%\node[box=0] (p-22-9) at (9,-22) {};
%\node[box=0] (p-22-10) at (10,-22) {};
%\node[box=0] (p-22-11) at (11,-22) {};
%\node[box=0] (p-22-12) at (12,-22) {};
%\node[box=0] (p-22-13) at (13,-22) {};
%\node[box=0] (p-22-14) at (14,-22) {};
%\node[box=0] (p-22-15) at (15,-22) {};
%\node[box=1] (p-22-16) at (16,-22) {};
%\node[box=0] (p-22-17) at (17,-22) {};
%\node[box=1] (p-22-18) at (18,-22) {};
%\node[box=0] (p-22-19) at (19,-22) {};
%\node[box=1] (p-22-20) at (20,-22) {};
%\node[box=0] (p-22-21) at (21,-22) {};
%\node[box=1] (p-22-22) at (22,-22) {};
%\node[box=1] (p-23-0) at (0,-23) {};
%\node[box=1] (p-23-1) at (1,-23) {};
%\node[box=1] (p-23-2) at (2,-23) {};
%\node[box=1] (p-23-3) at (3,-23) {};
%\node[box=1] (p-23-4) at (4,-23) {};
%\node[box=1] (p-23-5) at (5,-23) {};
%\node[box=1] (p-23-6) at (6,-23) {};
%\node[box=1] (p-23-7) at (7,-23) {};
%\node[box=0] (p-23-8) at (8,-23) {};
%\node[box=0] (p-23-9) at (9,-23) {};
%\node[box=0] (p-23-10) at (10,-23) {};
%\node[box=0] (p-23-11) at (11,-23) {};
%\node[box=0] (p-23-12) at (12,-23) {};
%\node[box=0] (p-23-13) at (13,-23) {};
%\node[box=0] (p-23-14) at (14,-23) {};
%\node[box=0] (p-23-15) at (15,-23) {};
%\node[box=1] (p-23-16) at (16,-23) {};
%\node[box=1] (p-23-17) at (17,-23) {};
%\node[box=1] (p-23-18) at (18,-23) {};
%\node[box=1] (p-23-19) at (19,-23) {};
%\node[box=1] (p-23-20) at (20,-23) {};
%\node[box=1] (p-23-21) at (21,-23) {};
%\node[box=1] (p-23-22) at (22,-23) {};
%\node[box=1] (p-23-23) at (23,-23) {};
%\node[box=1] (p-24-0) at (0,-24) {};
%\node[box=0] (p-24-1) at (1,-24) {};
%\node[box=0] (p-24-2) at (2,-24) {};
%\node[box=0] (p-24-3) at (3,-24) {};
%\node[box=0] (p-24-4) at (4,-24) {};
%\node[box=0] (p-24-5) at (5,-24) {};
%\node[box=0] (p-24-6) at (6,-24) {};
%\node[box=0] (p-24-7) at (7,-24) {};
%\node[box=1] (p-24-8) at (8,-24) {};
%\node[box=0] (p-24-9) at (9,-24) {};
%\node[box=0] (p-24-10) at (10,-24) {};
%\node[box=0] (p-24-11) at (11,-24) {};
%\node[box=0] (p-24-12) at (12,-24) {};
%\node[box=0] (p-24-13) at (13,-24) {};
%\node[box=0] (p-24-14) at (14,-24) {};
%\node[box=0] (p-24-15) at (15,-24) {};
%\node[box=1] (p-24-16) at (16,-24) {};
%\node[box=0] (p-24-17) at (17,-24) {};
%\node[box=0] (p-24-18) at (18,-24) {};
%\node[box=0] (p-24-19) at (19,-24) {};
%\node[box=0] (p-24-20) at (20,-24) {};
%\node[box=0] (p-24-21) at (21,-24) {};
%\node[box=0] (p-24-22) at (22,-24) {};
%\node[box=0] (p-24-23) at (23,-24) {};
%\node[box=1] (p-24-24) at (24,-24) {};
%\node[box=1] (p-25-0) at (0,-25) {};
%\node[box=1] (p-25-1) at (1,-25) {};
%\node[box=0] (p-25-2) at (2,-25) {};
%\node[box=0] (p-25-3) at (3,-25) {};
%\node[box=0] (p-25-4) at (4,-25) {};
%\node[box=0] (p-25-5) at (5,-25) {};
%\node[box=0] (p-25-6) at (6,-25) {};
%\node[box=0] (p-25-7) at (7,-25) {};
%\node[box=1] (p-25-8) at (8,-25) {};
%\node[box=1] (p-25-9) at (9,-25) {};
%\node[box=0] (p-25-10) at (10,-25) {};
%\node[box=0] (p-25-11) at (11,-25) {};
%\node[box=0] (p-25-12) at (12,-25) {};
%\node[box=0] (p-25-13) at (13,-25) {};
%\node[box=0] (p-25-14) at (14,-25) {};
%\node[box=0] (p-25-15) at (15,-25) {};
%\node[box=1] (p-25-16) at (16,-25) {};
%\node[box=1] (p-25-17) at (17,-25) {};
%\node[box=0] (p-25-18) at (18,-25) {};
%\node[box=0] (p-25-19) at (19,-25) {};
%\node[box=0] (p-25-20) at (20,-25) {};
%\node[box=0] (p-25-21) at (21,-25) {};
%\node[box=0] (p-25-22) at (22,-25) {};
%\node[box=0] (p-25-23) at (23,-25) {};
%\node[box=1] (p-25-24) at (24,-25) {};
%\node[box=1] (p-25-25) at (25,-25) {};
%\node[box=1] (p-26-0) at (0,-26) {};
%\node[box=0] (p-26-1) at (1,-26) {};
%\node[box=1] (p-26-2) at (2,-26) {};
%\node[box=0] (p-26-3) at (3,-26) {};
%\node[box=0] (p-26-4) at (4,-26) {};
%\node[box=0] (p-26-5) at (5,-26) {};
%\node[box=0] (p-26-6) at (6,-26) {};
%\node[box=0] (p-26-7) at (7,-26) {};
%\node[box=1] (p-26-8) at (8,-26) {};
%\node[box=0] (p-26-9) at (9,-26) {};
%\node[box=1] (p-26-10) at (10,-26) {};
%\node[box=0] (p-26-11) at (11,-26) {};
%\node[box=0] (p-26-12) at (12,-26) {};
%\node[box=0] (p-26-13) at (13,-26) {};
%\node[box=0] (p-26-14) at (14,-26) {};
%\node[box=0] (p-26-15) at (15,-26) {};
%\node[box=1] (p-26-16) at (16,-26) {};
%\node[box=0] (p-26-17) at (17,-26) {};
%\node[box=1] (p-26-18) at (18,-26) {};
%\node[box=0] (p-26-19) at (19,-26) {};
%\node[box=0] (p-26-20) at (20,-26) {};
%\node[box=0] (p-26-21) at (21,-26) {};
%\node[box=0] (p-26-22) at (22,-26) {};
%\node[box=0] (p-26-23) at (23,-26) {};
%\node[box=1] (p-26-24) at (24,-26) {};
%\node[box=0] (p-26-25) at (25,-26) {};
%\node[box=1] (p-26-26) at (26,-26) {};
%\node[box=1] (p-27-0) at (0,-27) {};
%\node[box=1] (p-27-1) at (1,-27) {};
%\node[box=1] (p-27-2) at (2,-27) {};
%\node[box=1] (p-27-3) at (3,-27) {};
%\node[box=0] (p-27-4) at (4,-27) {};
%\node[box=0] (p-27-5) at (5,-27) {};
%\node[box=0] (p-27-6) at (6,-27) {};
%\node[box=0] (p-27-7) at (7,-27) {};
%\node[box=1] (p-27-8) at (8,-27) {};
%\node[box=1] (p-27-9) at (9,-27) {};
%\node[box=1] (p-27-10) at (10,-27) {};
%\node[box=1] (p-27-11) at (11,-27) {};
%\node[box=0] (p-27-12) at (12,-27) {};
%\node[box=0] (p-27-13) at (13,-27) {};
%\node[box=0] (p-27-14) at (14,-27) {};
%\node[box=0] (p-27-15) at (15,-27) {};
%\node[box=1] (p-27-16) at (16,-27) {};
%\node[box=1] (p-27-17) at (17,-27) {};
%\node[box=1] (p-27-18) at (18,-27) {};
%\node[box=1] (p-27-19) at (19,-27) {};
%\node[box=0] (p-27-20) at (20,-27) {};
%\node[box=0] (p-27-21) at (21,-27) {};
%\node[box=0] (p-27-22) at (22,-27) {};
%\node[box=0] (p-27-23) at (23,-27) {};
%\node[box=1] (p-27-24) at (24,-27) {};
%\node[box=1] (p-27-25) at (25,-27) {};
%\node[box=1] (p-27-26) at (26,-27) {};
%\node[box=1] (p-27-27) at (27,-27) {};
%\node[box=1] (p-28-0) at (0,-28) {};
%\node[box=0] (p-28-1) at (1,-28) {};
%\node[box=0] (p-28-2) at (2,-28) {};
%\node[box=0] (p-28-3) at (3,-28) {};
%\node[box=1] (p-28-4) at (4,-28) {};
%\node[box=0] (p-28-5) at (5,-28) {};
%\node[box=0] (p-28-6) at (6,-28) {};
%\node[box=0] (p-28-7) at (7,-28) {};
%\node[box=1] (p-28-8) at (8,-28) {};
%\node[box=0] (p-28-9) at (9,-28) {};
%\node[box=0] (p-28-10) at (10,-28) {};
%\node[box=0] (p-28-11) at (11,-28) {};
%\node[box=1] (p-28-12) at (12,-28) {};
%\node[box=0] (p-28-13) at (13,-28) {};
%\node[box=0] (p-28-14) at (14,-28) {};
%\node[box=0] (p-28-15) at (15,-28) {};
%\node[box=1] (p-28-16) at (16,-28) {};
%\node[box=0] (p-28-17) at (17,-28) {};
%\node[box=0] (p-28-18) at (18,-28) {};
%\node[box=0] (p-28-19) at (19,-28) {};
%\node[box=1] (p-28-20) at (20,-28) {};
%\node[box=0] (p-28-21) at (21,-28) {};
%\node[box=0] (p-28-22) at (22,-28) {};
%\node[box=0] (p-28-23) at (23,-28) {};
%\node[box=1] (p-28-24) at (24,-28) {};
%\node[box=0] (p-28-25) at (25,-28) {};
%\node[box=0] (p-28-26) at (26,-28) {};
%\node[box=0] (p-28-27) at (27,-28) {};
%\node[box=1] (p-28-28) at (28,-28) {};
%\node[box=1] (p-29-0) at (0,-29) {};
%\node[box=1] (p-29-1) at (1,-29) {};
%\node[box=0] (p-29-2) at (2,-29) {};
%\node[box=0] (p-29-3) at (3,-29) {};
%\node[box=1] (p-29-4) at (4,-29) {};
%\node[box=1] (p-29-5) at (5,-29) {};
%\node[box=0] (p-29-6) at (6,-29) {};
%\node[box=0] (p-29-7) at (7,-29) {};
%\node[box=1] (p-29-8) at (8,-29) {};
%\node[box=1] (p-29-9) at (9,-29) {};
%\node[box=0] (p-29-10) at (10,-29) {};
%\node[box=0] (p-29-11) at (11,-29) {};
%\node[box=1] (p-29-12) at (12,-29) {};
%\node[box=1] (p-29-13) at (13,-29) {};
%\node[box=0] (p-29-14) at (14,-29) {};
%\node[box=0] (p-29-15) at (15,-29) {};
%\node[box=1] (p-29-16) at (16,-29) {};
%\node[box=1] (p-29-17) at (17,-29) {};
%\node[box=0] (p-29-18) at (18,-29) {};
%\node[box=0] (p-29-19) at (19,-29) {};
%\node[box=1] (p-29-20) at (20,-29) {};
%\node[box=1] (p-29-21) at (21,-29) {};
%\node[box=0] (p-29-22) at (22,-29) {};
%\node[box=0] (p-29-23) at (23,-29) {};
%\node[box=1] (p-29-24) at (24,-29) {};
%\node[box=1] (p-29-25) at (25,-29) {};
%\node[box=0] (p-29-26) at (26,-29) {};
%\node[box=0] (p-29-27) at (27,-29) {};
%\node[box=1] (p-29-28) at (28,-29) {};
%\node[box=1] (p-29-29) at (29,-29) {};
%\node[box=1] (p-30-0) at (0,-30) {};
%\node[box=0] (p-30-1) at (1,-30) {};
%\node[box=1] (p-30-2) at (2,-30) {};
%\node[box=0] (p-30-3) at (3,-30) {};
%\node[box=1] (p-30-4) at (4,-30) {};
%\node[box=0] (p-30-5) at (5,-30) {};
%\node[box=1] (p-30-6) at (6,-30) {};
%\node[box=0] (p-30-7) at (7,-30) {};
%\node[box=1] (p-30-8) at (8,-30) {};
%\node[box=0] (p-30-9) at (9,-30) {};
%\node[box=1] (p-30-10) at (10,-30) {};
%\node[box=0] (p-30-11) at (11,-30) {};
%\node[box=1] (p-30-12) at (12,-30) {};
%\node[box=0] (p-30-13) at (13,-30) {};
%\node[box=1] (p-30-14) at (14,-30) {};
%\node[box=0] (p-30-15) at (15,-30) {};
%\node[box=1] (p-30-16) at (16,-30) {};
%\node[box=0] (p-30-17) at (17,-30) {};
%\node[box=1] (p-30-18) at (18,-30) {};
%\node[box=0] (p-30-19) at (19,-30) {};
%\node[box=1] (p-30-20) at (20,-30) {};
%\node[box=0] (p-30-21) at (21,-30) {};
%\node[box=1] (p-30-22) at (22,-30) {};
%\node[box=0] (p-30-23) at (23,-30) {};
%\node[box=1] (p-30-24) at (24,-30) {};
%\node[box=0] (p-30-25) at (25,-30) {};
%\node[box=1] (p-30-26) at (26,-30) {};
%\node[box=0] (p-30-27) at (27,-30) {};
%\node[box=1] (p-30-28) at (28,-30) {};
%\node[box=0] (p-30-29) at (29,-30) {};
%\node[box=1] (p-30-30) at (30,-30) {};
%\node[box=1] (p-31-0) at (0,-31) {};
%\node[box=1] (p-31-1) at (1,-31) {};
%\node[box=1] (p-31-2) at (2,-31) {};
%\node[box=1] (p-31-3) at (3,-31) {};
%\node[box=1] (p-31-4) at (4,-31) {};
%\node[box=1] (p-31-5) at (5,-31) {};
%\node[box=1] (p-31-6) at (6,-31) {};
%\node[box=1] (p-31-7) at (7,-31) {};
%\node[box=1] (p-31-8) at (8,-31) {};
%\node[box=1] (p-31-9) at (9,-31) {};
%\node[box=1] (p-31-10) at (10,-31) {};
%\node[box=1] (p-31-11) at (11,-31) {};
%\node[box=1] (p-31-12) at (12,-31) {};
%\node[box=1] (p-31-13) at (13,-31) {};
%\node[box=1] (p-31-14) at (14,-31) {};
%\node[box=1] (p-31-15) at (15,-31) {};
%\node[box=1] (p-31-16) at (16,-31) {};
%\node[box=1] (p-31-17) at (17,-31) {};
%\node[box=1] (p-31-18) at (18,-31) {};
%\node[box=1] (p-31-19) at (19,-31) {};
%\node[box=1] (p-31-20) at (20,-31) {};
%\node[box=1] (p-31-21) at (21,-31) {};
%\node[box=1] (p-31-22) at (22,-31) {};
%\node[box=1] (p-31-23) at (23,-31) {};
%\node[box=1] (p-31-24) at (24,-31) {};
%\node[box=1] (p-31-25) at (25,-31) {};
%\node[box=1] (p-31-26) at (26,-31) {};
%\node[box=1] (p-31-27) at (27,-31) {};
%\node[box=1] (p-31-28) at (28,-31) {};
%\node[box=1] (p-31-29) at (29,-31) {};
%\node[box=1] (p-31-30) at (30,-31) {};
%\node[box=1] (p-31-31) at (31,-31) {};
%\node[box=1] (p-32-0) at (0,-32) {};
%\node[box=0] (p-32-1) at (1,-32) {};
%\node[box=0] (p-32-2) at (2,-32) {};
%\node[box=0] (p-32-3) at (3,-32) {};
%\node[box=0] (p-32-4) at (4,-32) {};
%\node[box=0] (p-32-5) at (5,-32) {};
%\node[box=0] (p-32-6) at (6,-32) {};
%\node[box=0] (p-32-7) at (7,-32) {};
%\node[box=0] (p-32-8) at (8,-32) {};
%\node[box=0] (p-32-9) at (9,-32) {};
%\node[box=0] (p-32-10) at (10,-32) {};
%\node[box=0] (p-32-11) at (11,-32) {};
%\node[box=0] (p-32-12) at (12,-32) {};
%\node[box=0] (p-32-13) at (13,-32) {};
%\node[box=0] (p-32-14) at (14,-32) {};
%\node[box=0] (p-32-15) at (15,-32) {};
%\node[box=0] (p-32-16) at (16,-32) {};
%\node[box=0] (p-32-17) at (17,-32) {};
%\node[box=0] (p-32-18) at (18,-32) {};
%\node[box=0] (p-32-19) at (19,-32) {};
%\node[box=0] (p-32-20) at (20,-32) {};
%\node[box=0] (p-32-21) at (21,-32) {};
%\node[box=0] (p-32-22) at (22,-32) {};
%\node[box=0] (p-32-23) at (23,-32) {};
%\node[box=0] (p-32-24) at (24,-32) {};
%\node[box=0] (p-32-25) at (25,-32) {};
%\node[box=0] (p-32-26) at (26,-32) {};
%\node[box=0] (p-32-27) at (27,-32) {};
%\node[box=0] (p-32-28) at (28,-32) {};
%\node[box=0] (p-32-29) at (29,-32) {};
%\node[box=0] (p-32-30) at (30,-32) {};
%\node[box=0] (p-32-31) at (31,-32) {};
%\node[box=1] (p-32-32) at (32,-32) {};
%\node[box=1] (p-33-0) at (0,-33) {};
%\node[box=1] (p-33-1) at (1,-33) {};
%\node[box=0] (p-33-2) at (2,-33) {};
%\node[box=0] (p-33-3) at (3,-33) {};
%\node[box=0] (p-33-4) at (4,-33) {};
%\node[box=0] (p-33-5) at (5,-33) {};
%\node[box=0] (p-33-6) at (6,-33) {};
%\node[box=0] (p-33-7) at (7,-33) {};
%\node[box=0] (p-33-8) at (8,-33) {};
%\node[box=0] (p-33-9) at (9,-33) {};
%\node[box=0] (p-33-10) at (10,-33) {};
%\node[box=0] (p-33-11) at (11,-33) {};
%\node[box=0] (p-33-12) at (12,-33) {};
%\node[box=0] (p-33-13) at (13,-33) {};
%\node[box=0] (p-33-14) at (14,-33) {};
%\node[box=0] (p-33-15) at (15,-33) {};
%\node[box=0] (p-33-16) at (16,-33) {};
%\node[box=0] (p-33-17) at (17,-33) {};
%\node[box=0] (p-33-18) at (18,-33) {};
%\node[box=0] (p-33-19) at (19,-33) {};
%\node[box=0] (p-33-20) at (20,-33) {};
%\node[box=0] (p-33-21) at (21,-33) {};
%\node[box=0] (p-33-22) at (22,-33) {};
%\node[box=0] (p-33-23) at (23,-33) {};
%\node[box=0] (p-33-24) at (24,-33) {};
%\node[box=0] (p-33-25) at (25,-33) {};
%\node[box=0] (p-33-26) at (26,-33) {};
%\node[box=0] (p-33-27) at (27,-33) {};
%\node[box=0] (p-33-28) at (28,-33) {};
%\node[box=0] (p-33-29) at (29,-33) {};
%\node[box=0] (p-33-30) at (30,-33) {};
%\node[box=0] (p-33-31) at (31,-33) {};
%\node[box=1] (p-33-32) at (32,-33) {};
%\node[box=1] (p-33-33) at (33,-33) {};
%\node[box=1] (p-34-0) at (0,-34) {};
%\node[box=0] (p-34-1) at (1,-34) {};
%\node[box=1] (p-34-2) at (2,-34) {};
%\node[box=0] (p-34-3) at (3,-34) {};
%\node[box=0] (p-34-4) at (4,-34) {};
%\node[box=0] (p-34-5) at (5,-34) {};
%\node[box=0] (p-34-6) at (6,-34) {};
%\node[box=0] (p-34-7) at (7,-34) {};
%\node[box=0] (p-34-8) at (8,-34) {};
%\node[box=0] (p-34-9) at (9,-34) {};
%\node[box=0] (p-34-10) at (10,-34) {};
%\node[box=0] (p-34-11) at (11,-34) {};
%\node[box=0] (p-34-12) at (12,-34) {};
%\node[box=0] (p-34-13) at (13,-34) {};
%\node[box=0] (p-34-14) at (14,-34) {};
%\node[box=0] (p-34-15) at (15,-34) {};
%\node[box=0] (p-34-16) at (16,-34) {};
%\node[box=0] (p-34-17) at (17,-34) {};
%\node[box=0] (p-34-18) at (18,-34) {};
%\node[box=0] (p-34-19) at (19,-34) {};
%\node[box=0] (p-34-20) at (20,-34) {};
%\node[box=0] (p-34-21) at (21,-34) {};
%\node[box=0] (p-34-22) at (22,-34) {};
%\node[box=0] (p-34-23) at (23,-34) {};
%\node[box=0] (p-34-24) at (24,-34) {};
%\node[box=0] (p-34-25) at (25,-34) {};
%\node[box=0] (p-34-26) at (26,-34) {};
%\node[box=0] (p-34-27) at (27,-34) {};
%\node[box=0] (p-34-28) at (28,-34) {};
%\node[box=0] (p-34-29) at (29,-34) {};
%\node[box=0] (p-34-30) at (30,-34) {};
%\node[box=0] (p-34-31) at (31,-34) {};
%\node[box=1] (p-34-32) at (32,-34) {};
%\node[box=0] (p-34-33) at (33,-34) {};
%\node[box=1] (p-34-34) at (34,-34) {};
%\node[box=1] (p-35-0) at (0,-35) {};
%\node[box=1] (p-35-1) at (1,-35) {};
%\node[box=1] (p-35-2) at (2,-35) {};
%\node[box=1] (p-35-3) at (3,-35) {};
%\node[box=0] (p-35-4) at (4,-35) {};
%\node[box=0] (p-35-5) at (5,-35) {};
%\node[box=0] (p-35-6) at (6,-35) {};
%\node[box=0] (p-35-7) at (7,-35) {};
%\node[box=0] (p-35-8) at (8,-35) {};
%\node[box=0] (p-35-9) at (9,-35) {};
%\node[box=0] (p-35-10) at (10,-35) {};
%\node[box=0] (p-35-11) at (11,-35) {};
%\node[box=0] (p-35-12) at (12,-35) {};
%\node[box=0] (p-35-13) at (13,-35) {};
%\node[box=0] (p-35-14) at (14,-35) {};
%\node[box=0] (p-35-15) at (15,-35) {};
%\node[box=0] (p-35-16) at (16,-35) {};
%\node[box=0] (p-35-17) at (17,-35) {};
%\node[box=0] (p-35-18) at (18,-35) {};
%\node[box=0] (p-35-19) at (19,-35) {};
%\node[box=0] (p-35-20) at (20,-35) {};
%\node[box=0] (p-35-21) at (21,-35) {};
%\node[box=0] (p-35-22) at (22,-35) {};
%\node[box=0] (p-35-23) at (23,-35) {};
%\node[box=0] (p-35-24) at (24,-35) {};
%\node[box=0] (p-35-25) at (25,-35) {};
%\node[box=0] (p-35-26) at (26,-35) {};
%\node[box=0] (p-35-27) at (27,-35) {};
%\node[box=0] (p-35-28) at (28,-35) {};
%\node[box=0] (p-35-29) at (29,-35) {};
%\node[box=0] (p-35-30) at (30,-35) {};
%\node[box=0] (p-35-31) at (31,-35) {};
%\node[box=1] (p-35-32) at (32,-35) {};
%\node[box=1] (p-35-33) at (33,-35) {};
%\node[box=1] (p-35-34) at (34,-35) {};
%\node[box=1] (p-35-35) at (35,-35) {};
%\node[box=1] (p-36-0) at (0,-36) {};
%\node[box=0] (p-36-1) at (1,-36) {};
%\node[box=0] (p-36-2) at (2,-36) {};
%\node[box=0] (p-36-3) at (3,-36) {};
%\node[box=1] (p-36-4) at (4,-36) {};
%\node[box=0] (p-36-5) at (5,-36) {};
%\node[box=0] (p-36-6) at (6,-36) {};
%\node[box=0] (p-36-7) at (7,-36) {};
%\node[box=0] (p-36-8) at (8,-36) {};
%\node[box=0] (p-36-9) at (9,-36) {};
%\node[box=0] (p-36-10) at (10,-36) {};
%\node[box=0] (p-36-11) at (11,-36) {};
%\node[box=0] (p-36-12) at (12,-36) {};
%\node[box=0] (p-36-13) at (13,-36) {};
%\node[box=0] (p-36-14) at (14,-36) {};
%\node[box=0] (p-36-15) at (15,-36) {};
%\node[box=0] (p-36-16) at (16,-36) {};
%\node[box=0] (p-36-17) at (17,-36) {};
%\node[box=0] (p-36-18) at (18,-36) {};
%\node[box=0] (p-36-19) at (19,-36) {};
%\node[box=0] (p-36-20) at (20,-36) {};
%\node[box=0] (p-36-21) at (21,-36) {};
%\node[box=0] (p-36-22) at (22,-36) {};
%\node[box=0] (p-36-23) at (23,-36) {};
%\node[box=0] (p-36-24) at (24,-36) {};
%\node[box=0] (p-36-25) at (25,-36) {};
%\node[box=0] (p-36-26) at (26,-36) {};
%\node[box=0] (p-36-27) at (27,-36) {};
%\node[box=0] (p-36-28) at (28,-36) {};
%\node[box=0] (p-36-29) at (29,-36) {};
%\node[box=0] (p-36-30) at (30,-36) {};
%\node[box=0] (p-36-31) at (31,-36) {};
%\node[box=1] (p-36-32) at (32,-36) {};
%\node[box=0] (p-36-33) at (33,-36) {};
%\node[box=0] (p-36-34) at (34,-36) {};
%\node[box=0] (p-36-35) at (35,-36) {};
%\node[box=1] (p-36-36) at (36,-36) {};
%\node[box=1] (p-37-0) at (0,-37) {};
%\node[box=1] (p-37-1) at (1,-37) {};
%\node[box=0] (p-37-2) at (2,-37) {};
%\node[box=0] (p-37-3) at (3,-37) {};
%\node[box=1] (p-37-4) at (4,-37) {};
%\node[box=1] (p-37-5) at (5,-37) {};
%\node[box=0] (p-37-6) at (6,-37) {};
%\node[box=0] (p-37-7) at (7,-37) {};
%\node[box=0] (p-37-8) at (8,-37) {};
%\node[box=0] (p-37-9) at (9,-37) {};
%\node[box=0] (p-37-10) at (10,-37) {};
%\node[box=0] (p-37-11) at (11,-37) {};
%\node[box=0] (p-37-12) at (12,-37) {};
%\node[box=0] (p-37-13) at (13,-37) {};
%\node[box=0] (p-37-14) at (14,-37) {};
%\node[box=0] (p-37-15) at (15,-37) {};
%\node[box=0] (p-37-16) at (16,-37) {};
%\node[box=0] (p-37-17) at (17,-37) {};
%\node[box=0] (p-37-18) at (18,-37) {};
%\node[box=0] (p-37-19) at (19,-37) {};
%\node[box=0] (p-37-20) at (20,-37) {};
%\node[box=0] (p-37-21) at (21,-37) {};
%\node[box=0] (p-37-22) at (22,-37) {};
%\node[box=0] (p-37-23) at (23,-37) {};
%\node[box=0] (p-37-24) at (24,-37) {};
%\node[box=0] (p-37-25) at (25,-37) {};
%\node[box=0] (p-37-26) at (26,-37) {};
%\node[box=0] (p-37-27) at (27,-37) {};
%\node[box=0] (p-37-28) at (28,-37) {};
%\node[box=0] (p-37-29) at (29,-37) {};
%\node[box=0] (p-37-30) at (30,-37) {};
%\node[box=0] (p-37-31) at (31,-37) {};
%\node[box=1] (p-37-32) at (32,-37) {};
%\node[box=1] (p-37-33) at (33,-37) {};
%\node[box=0] (p-37-34) at (34,-37) {};
%\node[box=0] (p-37-35) at (35,-37) {};
%\node[box=1] (p-37-36) at (36,-37) {};
%\node[box=1] (p-37-37) at (37,-37) {};
%\node[box=1] (p-38-0) at (0,-38) {};
%\node[box=0] (p-38-1) at (1,-38) {};
%\node[box=1] (p-38-2) at (2,-38) {};
%\node[box=0] (p-38-3) at (3,-38) {};
%\node[box=1] (p-38-4) at (4,-38) {};
%\node[box=0] (p-38-5) at (5,-38) {};
%\node[box=1] (p-38-6) at (6,-38) {};
%\node[box=0] (p-38-7) at (7,-38) {};
%\node[box=0] (p-38-8) at (8,-38) {};
%\node[box=0] (p-38-9) at (9,-38) {};
%\node[box=0] (p-38-10) at (10,-38) {};
%\node[box=0] (p-38-11) at (11,-38) {};
%\node[box=0] (p-38-12) at (12,-38) {};
%\node[box=0] (p-38-13) at (13,-38) {};
%\node[box=0] (p-38-14) at (14,-38) {};
%\node[box=0] (p-38-15) at (15,-38) {};
%\node[box=0] (p-38-16) at (16,-38) {};
%\node[box=0] (p-38-17) at (17,-38) {};
%\node[box=0] (p-38-18) at (18,-38) {};
%\node[box=0] (p-38-19) at (19,-38) {};
%\node[box=0] (p-38-20) at (20,-38) {};
%\node[box=0] (p-38-21) at (21,-38) {};
%\node[box=0] (p-38-22) at (22,-38) {};
%\node[box=0] (p-38-23) at (23,-38) {};
%\node[box=0] (p-38-24) at (24,-38) {};
%\node[box=0] (p-38-25) at (25,-38) {};
%\node[box=0] (p-38-26) at (26,-38) {};
%\node[box=0] (p-38-27) at (27,-38) {};
%\node[box=0] (p-38-28) at (28,-38) {};
%\node[box=0] (p-38-29) at (29,-38) {};
%\node[box=0] (p-38-30) at (30,-38) {};
%\node[box=0] (p-38-31) at (31,-38) {};
%\node[box=1] (p-38-32) at (32,-38) {};
%\node[box=0] (p-38-33) at (33,-38) {};
%\node[box=1] (p-38-34) at (34,-38) {};
%\node[box=0] (p-38-35) at (35,-38) {};
%\node[box=1] (p-38-36) at (36,-38) {};
%\node[box=0] (p-38-37) at (37,-38) {};
%\node[box=1] (p-38-38) at (38,-38) {};
%\node[box=1] (p-39-0) at (0,-39) {};
%\node[box=1] (p-39-1) at (1,-39) {};
%\node[box=1] (p-39-2) at (2,-39) {};
%\node[box=1] (p-39-3) at (3,-39) {};
%\node[box=1] (p-39-4) at (4,-39) {};
%\node[box=1] (p-39-5) at (5,-39) {};
%\node[box=1] (p-39-6) at (6,-39) {};
%\node[box=1] (p-39-7) at (7,-39) {};
%\node[box=0] (p-39-8) at (8,-39) {};
%\node[box=0] (p-39-9) at (9,-39) {};
%\node[box=0] (p-39-10) at (10,-39) {};
%\node[box=0] (p-39-11) at (11,-39) {};
%\node[box=0] (p-39-12) at (12,-39) {};
%\node[box=0] (p-39-13) at (13,-39) {};
%\node[box=0] (p-39-14) at (14,-39) {};
%\node[box=0] (p-39-15) at (15,-39) {};
%\node[box=0] (p-39-16) at (16,-39) {};
%\node[box=0] (p-39-17) at (17,-39) {};
%\node[box=0] (p-39-18) at (18,-39) {};
%\node[box=0] (p-39-19) at (19,-39) {};
%\node[box=0] (p-39-20) at (20,-39) {};
%\node[box=0] (p-39-21) at (21,-39) {};
%\node[box=0] (p-39-22) at (22,-39) {};
%\node[box=0] (p-39-23) at (23,-39) {};
%\node[box=0] (p-39-24) at (24,-39) {};
%\node[box=0] (p-39-25) at (25,-39) {};
%\node[box=0] (p-39-26) at (26,-39) {};
%\node[box=0] (p-39-27) at (27,-39) {};
%\node[box=0] (p-39-28) at (28,-39) {};
%\node[box=0] (p-39-29) at (29,-39) {};
%\node[box=0] (p-39-30) at (30,-39) {};
%\node[box=0] (p-39-31) at (31,-39) {};
%\node[box=1] (p-39-32) at (32,-39) {};
%\node[box=1] (p-39-33) at (33,-39) {};
%\node[box=1] (p-39-34) at (34,-39) {};
%\node[box=1] (p-39-35) at (35,-39) {};
%\node[box=1] (p-39-36) at (36,-39) {};
%\node[box=1] (p-39-37) at (37,-39) {};
%\node[box=1] (p-39-38) at (38,-39) {};
%\node[box=1] (p-39-39) at (39,-39) {};
%\node[box=1] (p-40-0) at (0,-40) {};
%\node[box=0] (p-40-1) at (1,-40) {};
%\node[box=0] (p-40-2) at (2,-40) {};
%\node[box=0] (p-40-3) at (3,-40) {};
%\node[box=0] (p-40-4) at (4,-40) {};
%\node[box=0] (p-40-5) at (5,-40) {};
%\node[box=0] (p-40-6) at (6,-40) {};
%\node[box=0] (p-40-7) at (7,-40) {};
%\node[box=1] (p-40-8) at (8,-40) {};
%\node[box=0] (p-40-9) at (9,-40) {};
%\node[box=0] (p-40-10) at (10,-40) {};
%\node[box=0] (p-40-11) at (11,-40) {};
%\node[box=0] (p-40-12) at (12,-40) {};
%\node[box=0] (p-40-13) at (13,-40) {};
%\node[box=0] (p-40-14) at (14,-40) {};
%\node[box=0] (p-40-15) at (15,-40) {};
%\node[box=0] (p-40-16) at (16,-40) {};
%\node[box=0] (p-40-17) at (17,-40) {};
%\node[box=0] (p-40-18) at (18,-40) {};
%\node[box=0] (p-40-19) at (19,-40) {};
%\node[box=0] (p-40-20) at (20,-40) {};
%\node[box=0] (p-40-21) at (21,-40) {};
%\node[box=0] (p-40-22) at (22,-40) {};
%\node[box=0] (p-40-23) at (23,-40) {};
%\node[box=0] (p-40-24) at (24,-40) {};
%\node[box=0] (p-40-25) at (25,-40) {};
%\node[box=0] (p-40-26) at (26,-40) {};
%\node[box=0] (p-40-27) at (27,-40) {};
%\node[box=0] (p-40-28) at (28,-40) {};
%\node[box=0] (p-40-29) at (29,-40) {};
%\node[box=0] (p-40-30) at (30,-40) {};
%\node[box=0] (p-40-31) at (31,-40) {};
%\node[box=1] (p-40-32) at (32,-40) {};
%\node[box=0] (p-40-33) at (33,-40) {};
%\node[box=0] (p-40-34) at (34,-40) {};
%\node[box=0] (p-40-35) at (35,-40) {};
%\node[box=0] (p-40-36) at (36,-40) {};
%\node[box=0] (p-40-37) at (37,-40) {};
%\node[box=0] (p-40-38) at (38,-40) {};
%\node[box=0] (p-40-39) at (39,-40) {};
%\node[box=1] (p-40-40) at (40,-40) {};
%\node[box=1] (p-41-0) at (0,-41) {};
%\node[box=1] (p-41-1) at (1,-41) {};
%\node[box=0] (p-41-2) at (2,-41) {};
%\node[box=0] (p-41-3) at (3,-41) {};
%\node[box=0] (p-41-4) at (4,-41) {};
%\node[box=0] (p-41-5) at (5,-41) {};
%\node[box=0] (p-41-6) at (6,-41) {};
%\node[box=0] (p-41-7) at (7,-41) {};
%\node[box=1] (p-41-8) at (8,-41) {};
%\node[box=1] (p-41-9) at (9,-41) {};
%\node[box=0] (p-41-10) at (10,-41) {};
%\node[box=0] (p-41-11) at (11,-41) {};
%\node[box=0] (p-41-12) at (12,-41) {};
%\node[box=0] (p-41-13) at (13,-41) {};
%\node[box=0] (p-41-14) at (14,-41) {};
%\node[box=0] (p-41-15) at (15,-41) {};
%\node[box=0] (p-41-16) at (16,-41) {};
%\node[box=0] (p-41-17) at (17,-41) {};
%\node[box=0] (p-41-18) at (18,-41) {};
%\node[box=0] (p-41-19) at (19,-41) {};
%\node[box=0] (p-41-20) at (20,-41) {};
%\node[box=0] (p-41-21) at (21,-41) {};
%\node[box=0] (p-41-22) at (22,-41) {};
%\node[box=0] (p-41-23) at (23,-41) {};
%\node[box=0] (p-41-24) at (24,-41) {};
%\node[box=0] (p-41-25) at (25,-41) {};
%\node[box=0] (p-41-26) at (26,-41) {};
%\node[box=0] (p-41-27) at (27,-41) {};
%\node[box=0] (p-41-28) at (28,-41) {};
%\node[box=0] (p-41-29) at (29,-41) {};
%\node[box=0] (p-41-30) at (30,-41) {};
%\node[box=0] (p-41-31) at (31,-41) {};
%\node[box=1] (p-41-32) at (32,-41) {};
%\node[box=1] (p-41-33) at (33,-41) {};
%\node[box=0] (p-41-34) at (34,-41) {};
%\node[box=0] (p-41-35) at (35,-41) {};
%\node[box=0] (p-41-36) at (36,-41) {};
%\node[box=0] (p-41-37) at (37,-41) {};
%\node[box=0] (p-41-38) at (38,-41) {};
%\node[box=0] (p-41-39) at (39,-41) {};
%\node[box=1] (p-41-40) at (40,-41) {};
%\node[box=1] (p-41-41) at (41,-41) {};
%\node[box=1] (p-42-0) at (0,-42) {};
%\node[box=0] (p-42-1) at (1,-42) {};
%\node[box=1] (p-42-2) at (2,-42) {};
%\node[box=0] (p-42-3) at (3,-42) {};
%\node[box=0] (p-42-4) at (4,-42) {};
%\node[box=0] (p-42-5) at (5,-42) {};
%\node[box=0] (p-42-6) at (6,-42) {};
%\node[box=0] (p-42-7) at (7,-42) {};
%\node[box=1] (p-42-8) at (8,-42) {};
%\node[box=0] (p-42-9) at (9,-42) {};
%\node[box=1] (p-42-10) at (10,-42) {};
%\node[box=0] (p-42-11) at (11,-42) {};
%\node[box=0] (p-42-12) at (12,-42) {};
%\node[box=0] (p-42-13) at (13,-42) {};
%\node[box=0] (p-42-14) at (14,-42) {};
%\node[box=0] (p-42-15) at (15,-42) {};
%\node[box=0] (p-42-16) at (16,-42) {};
%\node[box=0] (p-42-17) at (17,-42) {};
%\node[box=0] (p-42-18) at (18,-42) {};
%\node[box=0] (p-42-19) at (19,-42) {};
%\node[box=0] (p-42-20) at (20,-42) {};
%\node[box=0] (p-42-21) at (21,-42) {};
%\node[box=0] (p-42-22) at (22,-42) {};
%\node[box=0] (p-42-23) at (23,-42) {};
%\node[box=0] (p-42-24) at (24,-42) {};
%\node[box=0] (p-42-25) at (25,-42) {};
%\node[box=0] (p-42-26) at (26,-42) {};
%\node[box=0] (p-42-27) at (27,-42) {};
%\node[box=0] (p-42-28) at (28,-42) {};
%\node[box=0] (p-42-29) at (29,-42) {};
%\node[box=0] (p-42-30) at (30,-42) {};
%\node[box=0] (p-42-31) at (31,-42) {};
%\node[box=1] (p-42-32) at (32,-42) {};
%\node[box=0] (p-42-33) at (33,-42) {};
%\node[box=1] (p-42-34) at (34,-42) {};
%\node[box=0] (p-42-35) at (35,-42) {};
%\node[box=0] (p-42-36) at (36,-42) {};
%\node[box=0] (p-42-37) at (37,-42) {};
%\node[box=0] (p-42-38) at (38,-42) {};
%\node[box=0] (p-42-39) at (39,-42) {};
%\node[box=1] (p-42-40) at (40,-42) {};
%\node[box=0] (p-42-41) at (41,-42) {};
%\node[box=1] (p-42-42) at (42,-42) {};
%\node[box=1] (p-43-0) at (0,-43) {};
%\node[box=1] (p-43-1) at (1,-43) {};
%\node[box=1] (p-43-2) at (2,-43) {};
%\node[box=1] (p-43-3) at (3,-43) {};
%\node[box=0] (p-43-4) at (4,-43) {};
%\node[box=0] (p-43-5) at (5,-43) {};
%\node[box=0] (p-43-6) at (6,-43) {};
%\node[box=0] (p-43-7) at (7,-43) {};
%\node[box=1] (p-43-8) at (8,-43) {};
%\node[box=1] (p-43-9) at (9,-43) {};
%\node[box=1] (p-43-10) at (10,-43) {};
%\node[box=1] (p-43-11) at (11,-43) {};
%\node[box=0] (p-43-12) at (12,-43) {};
%\node[box=0] (p-43-13) at (13,-43) {};
%\node[box=0] (p-43-14) at (14,-43) {};
%\node[box=0] (p-43-15) at (15,-43) {};
%\node[box=0] (p-43-16) at (16,-43) {};
%\node[box=0] (p-43-17) at (17,-43) {};
%\node[box=0] (p-43-18) at (18,-43) {};
%\node[box=0] (p-43-19) at (19,-43) {};
%\node[box=0] (p-43-20) at (20,-43) {};
%\node[box=0] (p-43-21) at (21,-43) {};
%\node[box=0] (p-43-22) at (22,-43) {};
%\node[box=0] (p-43-23) at (23,-43) {};
%\node[box=0] (p-43-24) at (24,-43) {};
%\node[box=0] (p-43-25) at (25,-43) {};
%\node[box=0] (p-43-26) at (26,-43) {};
%\node[box=0] (p-43-27) at (27,-43) {};
%\node[box=0] (p-43-28) at (28,-43) {};
%\node[box=0] (p-43-29) at (29,-43) {};
%\node[box=0] (p-43-30) at (30,-43) {};
%\node[box=0] (p-43-31) at (31,-43) {};
%\node[box=1] (p-43-32) at (32,-43) {};
%\node[box=1] (p-43-33) at (33,-43) {};
%\node[box=1] (p-43-34) at (34,-43) {};
%\node[box=1] (p-43-35) at (35,-43) {};
%\node[box=0] (p-43-36) at (36,-43) {};
%\node[box=0] (p-43-37) at (37,-43) {};
%\node[box=0] (p-43-38) at (38,-43) {};
%\node[box=0] (p-43-39) at (39,-43) {};
%\node[box=1] (p-43-40) at (40,-43) {};
%\node[box=1] (p-43-41) at (41,-43) {};
%\node[box=1] (p-43-42) at (42,-43) {};
%\node[box=1] (p-43-43) at (43,-43) {};
%\node[box=1] (p-44-0) at (0,-44) {};
%\node[box=0] (p-44-1) at (1,-44) {};
%\node[box=0] (p-44-2) at (2,-44) {};
%\node[box=0] (p-44-3) at (3,-44) {};
%\node[box=1] (p-44-4) at (4,-44) {};
%\node[box=0] (p-44-5) at (5,-44) {};
%\node[box=0] (p-44-6) at (6,-44) {};
%\node[box=0] (p-44-7) at (7,-44) {};
%\node[box=1] (p-44-8) at (8,-44) {};
%\node[box=0] (p-44-9) at (9,-44) {};
%\node[box=0] (p-44-10) at (10,-44) {};
%\node[box=0] (p-44-11) at (11,-44) {};
%\node[box=1] (p-44-12) at (12,-44) {};
%\node[box=0] (p-44-13) at (13,-44) {};
%\node[box=0] (p-44-14) at (14,-44) {};
%\node[box=0] (p-44-15) at (15,-44) {};
%\node[box=0] (p-44-16) at (16,-44) {};
%\node[box=0] (p-44-17) at (17,-44) {};
%\node[box=0] (p-44-18) at (18,-44) {};
%\node[box=0] (p-44-19) at (19,-44) {};
%\node[box=0] (p-44-20) at (20,-44) {};
%\node[box=0] (p-44-21) at (21,-44) {};
%\node[box=0] (p-44-22) at (22,-44) {};
%\node[box=0] (p-44-23) at (23,-44) {};
%\node[box=0] (p-44-24) at (24,-44) {};
%\node[box=0] (p-44-25) at (25,-44) {};
%\node[box=0] (p-44-26) at (26,-44) {};
%\node[box=0] (p-44-27) at (27,-44) {};
%\node[box=0] (p-44-28) at (28,-44) {};
%\node[box=0] (p-44-29) at (29,-44) {};
%\node[box=0] (p-44-30) at (30,-44) {};
%\node[box=0] (p-44-31) at (31,-44) {};
%\node[box=1] (p-44-32) at (32,-44) {};
%\node[box=0] (p-44-33) at (33,-44) {};
%\node[box=0] (p-44-34) at (34,-44) {};
%\node[box=0] (p-44-35) at (35,-44) {};
%\node[box=1] (p-44-36) at (36,-44) {};
%\node[box=0] (p-44-37) at (37,-44) {};
%\node[box=0] (p-44-38) at (38,-44) {};
%\node[box=0] (p-44-39) at (39,-44) {};
%\node[box=1] (p-44-40) at (40,-44) {};
%\node[box=0] (p-44-41) at (41,-44) {};
%\node[box=0] (p-44-42) at (42,-44) {};
%\node[box=0] (p-44-43) at (43,-44) {};
%\node[box=1] (p-44-44) at (44,-44) {};
%\node[box=1] (p-45-0) at (0,-45) {};
%\node[box=1] (p-45-1) at (1,-45) {};
%\node[box=0] (p-45-2) at (2,-45) {};
%\node[box=0] (p-45-3) at (3,-45) {};
%\node[box=1] (p-45-4) at (4,-45) {};
%\node[box=1] (p-45-5) at (5,-45) {};
%\node[box=0] (p-45-6) at (6,-45) {};
%\node[box=0] (p-45-7) at (7,-45) {};
%\node[box=1] (p-45-8) at (8,-45) {};
%\node[box=1] (p-45-9) at (9,-45) {};
%\node[box=0] (p-45-10) at (10,-45) {};
%\node[box=0] (p-45-11) at (11,-45) {};
%\node[box=1] (p-45-12) at (12,-45) {};
%\node[box=1] (p-45-13) at (13,-45) {};
%\node[box=0] (p-45-14) at (14,-45) {};
%\node[box=0] (p-45-15) at (15,-45) {};
%\node[box=0] (p-45-16) at (16,-45) {};
%\node[box=0] (p-45-17) at (17,-45) {};
%\node[box=0] (p-45-18) at (18,-45) {};
%\node[box=0] (p-45-19) at (19,-45) {};
%\node[box=0] (p-45-20) at (20,-45) {};
%\node[box=0] (p-45-21) at (21,-45) {};
%\node[box=0] (p-45-22) at (22,-45) {};
%\node[box=0] (p-45-23) at (23,-45) {};
%\node[box=0] (p-45-24) at (24,-45) {};
%\node[box=0] (p-45-25) at (25,-45) {};
%\node[box=0] (p-45-26) at (26,-45) {};
%\node[box=0] (p-45-27) at (27,-45) {};
%\node[box=0] (p-45-28) at (28,-45) {};
%\node[box=0] (p-45-29) at (29,-45) {};
%\node[box=0] (p-45-30) at (30,-45) {};
%\node[box=0] (p-45-31) at (31,-45) {};
%\node[box=1] (p-45-32) at (32,-45) {};
%\node[box=1] (p-45-33) at (33,-45) {};
%\node[box=0] (p-45-34) at (34,-45) {};
%\node[box=0] (p-45-35) at (35,-45) {};
%\node[box=1] (p-45-36) at (36,-45) {};
%\node[box=1] (p-45-37) at (37,-45) {};
%\node[box=0] (p-45-38) at (38,-45) {};
%\node[box=0] (p-45-39) at (39,-45) {};
%\node[box=1] (p-45-40) at (40,-45) {};
%\node[box=1] (p-45-41) at (41,-45) {};
%\node[box=0] (p-45-42) at (42,-45) {};
%\node[box=0] (p-45-43) at (43,-45) {};
%\node[box=1] (p-45-44) at (44,-45) {};
%\node[box=1] (p-45-45) at (45,-45) {};
%\node[box=1] (p-46-0) at (0,-46) {};
%\node[box=0] (p-46-1) at (1,-46) {};
%\node[box=1] (p-46-2) at (2,-46) {};
%\node[box=0] (p-46-3) at (3,-46) {};
%\node[box=1] (p-46-4) at (4,-46) {};
%\node[box=0] (p-46-5) at (5,-46) {};
%\node[box=1] (p-46-6) at (6,-46) {};
%\node[box=0] (p-46-7) at (7,-46) {};
%\node[box=1] (p-46-8) at (8,-46) {};
%\node[box=0] (p-46-9) at (9,-46) {};
%\node[box=1] (p-46-10) at (10,-46) {};
%\node[box=0] (p-46-11) at (11,-46) {};
%\node[box=1] (p-46-12) at (12,-46) {};
%\node[box=0] (p-46-13) at (13,-46) {};
%\node[box=1] (p-46-14) at (14,-46) {};
%\node[box=0] (p-46-15) at (15,-46) {};
%\node[box=0] (p-46-16) at (16,-46) {};
%\node[box=0] (p-46-17) at (17,-46) {};
%\node[box=0] (p-46-18) at (18,-46) {};
%\node[box=0] (p-46-19) at (19,-46) {};
%\node[box=0] (p-46-20) at (20,-46) {};
%\node[box=0] (p-46-21) at (21,-46) {};
%\node[box=0] (p-46-22) at (22,-46) {};
%\node[box=0] (p-46-23) at (23,-46) {};
%\node[box=0] (p-46-24) at (24,-46) {};
%\node[box=0] (p-46-25) at (25,-46) {};
%\node[box=0] (p-46-26) at (26,-46) {};
%\node[box=0] (p-46-27) at (27,-46) {};
%\node[box=0] (p-46-28) at (28,-46) {};
%\node[box=0] (p-46-29) at (29,-46) {};
%\node[box=0] (p-46-30) at (30,-46) {};
%\node[box=0] (p-46-31) at (31,-46) {};
%\node[box=1] (p-46-32) at (32,-46) {};
%\node[box=0] (p-46-33) at (33,-46) {};
%\node[box=1] (p-46-34) at (34,-46) {};
%\node[box=0] (p-46-35) at (35,-46) {};
%\node[box=1] (p-46-36) at (36,-46) {};
%\node[box=0] (p-46-37) at (37,-46) {};
%\node[box=1] (p-46-38) at (38,-46) {};
%\node[box=0] (p-46-39) at (39,-46) {};
%\node[box=1] (p-46-40) at (40,-46) {};
%\node[box=0] (p-46-41) at (41,-46) {};
%\node[box=1] (p-46-42) at (42,-46) {};
%\node[box=0] (p-46-43) at (43,-46) {};
%\node[box=1] (p-46-44) at (44,-46) {};
%\node[box=0] (p-46-45) at (45,-46) {};
%\node[box=1] (p-46-46) at (46,-46) {};
%\node[box=1] (p-47-0) at (0,-47) {};
%\node[box=1] (p-47-1) at (1,-47) {};
%\node[box=1] (p-47-2) at (2,-47) {};
%\node[box=1] (p-47-3) at (3,-47) {};
%\node[box=1] (p-47-4) at (4,-47) {};
%\node[box=1] (p-47-5) at (5,-47) {};
%\node[box=1] (p-47-6) at (6,-47) {};
%\node[box=1] (p-47-7) at (7,-47) {};
%\node[box=1] (p-47-8) at (8,-47) {};
%\node[box=1] (p-47-9) at (9,-47) {};
%\node[box=1] (p-47-10) at (10,-47) {};
%\node[box=1] (p-47-11) at (11,-47) {};
%\node[box=1] (p-47-12) at (12,-47) {};
%\node[box=1] (p-47-13) at (13,-47) {};
%\node[box=1] (p-47-14) at (14,-47) {};
%\node[box=1] (p-47-15) at (15,-47) {};
%\node[box=0] (p-47-16) at (16,-47) {};
%\node[box=0] (p-47-17) at (17,-47) {};
%\node[box=0] (p-47-18) at (18,-47) {};
%\node[box=0] (p-47-19) at (19,-47) {};
%\node[box=0] (p-47-20) at (20,-47) {};
%\node[box=0] (p-47-21) at (21,-47) {};
%\node[box=0] (p-47-22) at (22,-47) {};
%\node[box=0] (p-47-23) at (23,-47) {};
%\node[box=0] (p-47-24) at (24,-47) {};
%\node[box=0] (p-47-25) at (25,-47) {};
%\node[box=0] (p-47-26) at (26,-47) {};
%\node[box=0] (p-47-27) at (27,-47) {};
%\node[box=0] (p-47-28) at (28,-47) {};
%\node[box=0] (p-47-29) at (29,-47) {};
%\node[box=0] (p-47-30) at (30,-47) {};
%\node[box=0] (p-47-31) at (31,-47) {};
%\node[box=1] (p-47-32) at (32,-47) {};
%\node[box=1] (p-47-33) at (33,-47) {};
%\node[box=1] (p-47-34) at (34,-47) {};
%\node[box=1] (p-47-35) at (35,-47) {};
%\node[box=1] (p-47-36) at (36,-47) {};
%\node[box=1] (p-47-37) at (37,-47) {};
%\node[box=1] (p-47-38) at (38,-47) {};
%\node[box=1] (p-47-39) at (39,-47) {};
%\node[box=1] (p-47-40) at (40,-47) {};
%\node[box=1] (p-47-41) at (41,-47) {};
%\node[box=1] (p-47-42) at (42,-47) {};
%\node[box=1] (p-47-43) at (43,-47) {};
%\node[box=1] (p-47-44) at (44,-47) {};
%\node[box=1] (p-47-45) at (45,-47) {};
%\node[box=1] (p-47-46) at (46,-47) {};
%\node[box=1] (p-47-47) at (47,-47) {};
%\node[box=1] (p-48-0) at (0,-48) {};
%\node[box=0] (p-48-1) at (1,-48) {};
%\node[box=0] (p-48-2) at (2,-48) {};
%\node[box=0] (p-48-3) at (3,-48) {};
%\node[box=0] (p-48-4) at (4,-48) {};
%\node[box=0] (p-48-5) at (5,-48) {};
%\node[box=0] (p-48-6) at (6,-48) {};
%\node[box=0] (p-48-7) at (7,-48) {};
%\node[box=0] (p-48-8) at (8,-48) {};
%\node[box=0] (p-48-9) at (9,-48) {};
%\node[box=0] (p-48-10) at (10,-48) {};
%\node[box=0] (p-48-11) at (11,-48) {};
%\node[box=0] (p-48-12) at (12,-48) {};
%\node[box=0] (p-48-13) at (13,-48) {};
%\node[box=0] (p-48-14) at (14,-48) {};
%\node[box=0] (p-48-15) at (15,-48) {};
%\node[box=1] (p-48-16) at (16,-48) {};
%\node[box=0] (p-48-17) at (17,-48) {};
%\node[box=0] (p-48-18) at (18,-48) {};
%\node[box=0] (p-48-19) at (19,-48) {};
%\node[box=0] (p-48-20) at (20,-48) {};
%\node[box=0] (p-48-21) at (21,-48) {};
%\node[box=0] (p-48-22) at (22,-48) {};
%\node[box=0] (p-48-23) at (23,-48) {};
%\node[box=0] (p-48-24) at (24,-48) {};
%\node[box=0] (p-48-25) at (25,-48) {};
%\node[box=0] (p-48-26) at (26,-48) {};
%\node[box=0] (p-48-27) at (27,-48) {};
%\node[box=0] (p-48-28) at (28,-48) {};
%\node[box=0] (p-48-29) at (29,-48) {};
%\node[box=0] (p-48-30) at (30,-48) {};
%\node[box=0] (p-48-31) at (31,-48) {};
%\node[box=1] (p-48-32) at (32,-48) {};
%\node[box=0] (p-48-33) at (33,-48) {};
%\node[box=0] (p-48-34) at (34,-48) {};
%\node[box=0] (p-48-35) at (35,-48) {};
%\node[box=0] (p-48-36) at (36,-48) {};
%\node[box=0] (p-48-37) at (37,-48) {};
%\node[box=0] (p-48-38) at (38,-48) {};
%\node[box=0] (p-48-39) at (39,-48) {};
%\node[box=0] (p-48-40) at (40,-48) {};
%\node[box=0] (p-48-41) at (41,-48) {};
%\node[box=0] (p-48-42) at (42,-48) {};
%\node[box=0] (p-48-43) at (43,-48) {};
%\node[box=0] (p-48-44) at (44,-48) {};
%\node[box=0] (p-48-45) at (45,-48) {};
%\node[box=0] (p-48-46) at (46,-48) {};
%\node[box=0] (p-48-47) at (47,-48) {};
%\node[box=1] (p-48-48) at (48,-48) {};
%\node[box=1] (p-49-0) at (0,-49) {};
%\node[box=1] (p-49-1) at (1,-49) {};
%\node[box=0] (p-49-2) at (2,-49) {};
%\node[box=0] (p-49-3) at (3,-49) {};
%\node[box=0] (p-49-4) at (4,-49) {};
%\node[box=0] (p-49-5) at (5,-49) {};
%\node[box=0] (p-49-6) at (6,-49) {};
%\node[box=0] (p-49-7) at (7,-49) {};
%\node[box=0] (p-49-8) at (8,-49) {};
%\node[box=0] (p-49-9) at (9,-49) {};
%\node[box=0] (p-49-10) at (10,-49) {};
%\node[box=0] (p-49-11) at (11,-49) {};
%\node[box=0] (p-49-12) at (12,-49) {};
%\node[box=0] (p-49-13) at (13,-49) {};
%\node[box=0] (p-49-14) at (14,-49) {};
%\node[box=0] (p-49-15) at (15,-49) {};
%\node[box=1] (p-49-16) at (16,-49) {};
%\node[box=1] (p-49-17) at (17,-49) {};
%\node[box=0] (p-49-18) at (18,-49) {};
%\node[box=0] (p-49-19) at (19,-49) {};
%\node[box=0] (p-49-20) at (20,-49) {};
%\node[box=0] (p-49-21) at (21,-49) {};
%\node[box=0] (p-49-22) at (22,-49) {};
%\node[box=0] (p-49-23) at (23,-49) {};
%\node[box=0] (p-49-24) at (24,-49) {};
%\node[box=0] (p-49-25) at (25,-49) {};
%\node[box=0] (p-49-26) at (26,-49) {};
%\node[box=0] (p-49-27) at (27,-49) {};
%\node[box=0] (p-49-28) at (28,-49) {};
%\node[box=0] (p-49-29) at (29,-49) {};
%\node[box=0] (p-49-30) at (30,-49) {};
%\node[box=0] (p-49-31) at (31,-49) {};
%\node[box=1] (p-49-32) at (32,-49) {};
%\node[box=1] (p-49-33) at (33,-49) {};
%\node[box=0] (p-49-34) at (34,-49) {};
%\node[box=0] (p-49-35) at (35,-49) {};
%\node[box=0] (p-49-36) at (36,-49) {};
%\node[box=0] (p-49-37) at (37,-49) {};
%\node[box=0] (p-49-38) at (38,-49) {};
%\node[box=0] (p-49-39) at (39,-49) {};
%\node[box=0] (p-49-40) at (40,-49) {};
%\node[box=0] (p-49-41) at (41,-49) {};
%\node[box=0] (p-49-42) at (42,-49) {};
%\node[box=0] (p-49-43) at (43,-49) {};
%\node[box=0] (p-49-44) at (44,-49) {};
%\node[box=0] (p-49-45) at (45,-49) {};
%\node[box=0] (p-49-46) at (46,-49) {};
%\node[box=0] (p-49-47) at (47,-49) {};
%\node[box=1] (p-49-48) at (48,-49) {};
%\node[box=1] (p-49-49) at (49,-49) {};
%\node[box=1] (p-50-0) at (0,-50) {};
%\node[box=0] (p-50-1) at (1,-50) {};
%\node[box=1] (p-50-2) at (2,-50) {};
%\node[box=0] (p-50-3) at (3,-50) {};
%\node[box=0] (p-50-4) at (4,-50) {};
%\node[box=0] (p-50-5) at (5,-50) {};
%\node[box=0] (p-50-6) at (6,-50) {};
%\node[box=0] (p-50-7) at (7,-50) {};
%\node[box=0] (p-50-8) at (8,-50) {};
%\node[box=0] (p-50-9) at (9,-50) {};
%\node[box=0] (p-50-10) at (10,-50) {};
%\node[box=0] (p-50-11) at (11,-50) {};
%\node[box=0] (p-50-12) at (12,-50) {};
%\node[box=0] (p-50-13) at (13,-50) {};
%\node[box=0] (p-50-14) at (14,-50) {};
%\node[box=0] (p-50-15) at (15,-50) {};
%\node[box=1] (p-50-16) at (16,-50) {};
%\node[box=0] (p-50-17) at (17,-50) {};
%\node[box=1] (p-50-18) at (18,-50) {};
%\node[box=0] (p-50-19) at (19,-50) {};
%\node[box=0] (p-50-20) at (20,-50) {};
%\node[box=0] (p-50-21) at (21,-50) {};
%\node[box=0] (p-50-22) at (22,-50) {};
%\node[box=0] (p-50-23) at (23,-50) {};
%\node[box=0] (p-50-24) at (24,-50) {};
%\node[box=0] (p-50-25) at (25,-50) {};
%\node[box=0] (p-50-26) at (26,-50) {};
%\node[box=0] (p-50-27) at (27,-50) {};
%\node[box=0] (p-50-28) at (28,-50) {};
%\node[box=0] (p-50-29) at (29,-50) {};
%\node[box=0] (p-50-30) at (30,-50) {};
%\node[box=0] (p-50-31) at (31,-50) {};
%\node[box=1] (p-50-32) at (32,-50) {};
%\node[box=0] (p-50-33) at (33,-50) {};
%\node[box=1] (p-50-34) at (34,-50) {};
%\node[box=0] (p-50-35) at (35,-50) {};
%\node[box=0] (p-50-36) at (36,-50) {};
%\node[box=0] (p-50-37) at (37,-50) {};
%\node[box=0] (p-50-38) at (38,-50) {};
%\node[box=0] (p-50-39) at (39,-50) {};
%\node[box=0] (p-50-40) at (40,-50) {};
%\node[box=0] (p-50-41) at (41,-50) {};
%\node[box=0] (p-50-42) at (42,-50) {};
%\node[box=0] (p-50-43) at (43,-50) {};
%\node[box=0] (p-50-44) at (44,-50) {};
%\node[box=0] (p-50-45) at (45,-50) {};
%\node[box=0] (p-50-46) at (46,-50) {};
%\node[box=0] (p-50-47) at (47,-50) {};
%\node[box=1] (p-50-48) at (48,-50) {};
%\node[box=0] (p-50-49) at (49,-50) {};
%\node[box=1] (p-50-50) at (50,-50) {};
%\node[box=1] (p-51-0) at (0,-51) {};
%\node[box=1] (p-51-1) at (1,-51) {};
%\node[box=1] (p-51-2) at (2,-51) {};
%\node[box=1] (p-51-3) at (3,-51) {};
%\node[box=0] (p-51-4) at (4,-51) {};
%\node[box=0] (p-51-5) at (5,-51) {};
%\node[box=0] (p-51-6) at (6,-51) {};
%\node[box=0] (p-51-7) at (7,-51) {};
%\node[box=0] (p-51-8) at (8,-51) {};
%\node[box=0] (p-51-9) at (9,-51) {};
%\node[box=0] (p-51-10) at (10,-51) {};
%\node[box=0] (p-51-11) at (11,-51) {};
%\node[box=0] (p-51-12) at (12,-51) {};
%\node[box=0] (p-51-13) at (13,-51) {};
%\node[box=0] (p-51-14) at (14,-51) {};
%\node[box=0] (p-51-15) at (15,-51) {};
%\node[box=1] (p-51-16) at (16,-51) {};
%\node[box=1] (p-51-17) at (17,-51) {};
%\node[box=1] (p-51-18) at (18,-51) {};
%\node[box=1] (p-51-19) at (19,-51) {};
%\node[box=0] (p-51-20) at (20,-51) {};
%\node[box=0] (p-51-21) at (21,-51) {};
%\node[box=0] (p-51-22) at (22,-51) {};
%\node[box=0] (p-51-23) at (23,-51) {};
%\node[box=0] (p-51-24) at (24,-51) {};
%\node[box=0] (p-51-25) at (25,-51) {};
%\node[box=0] (p-51-26) at (26,-51) {};
%\node[box=0] (p-51-27) at (27,-51) {};
%\node[box=0] (p-51-28) at (28,-51) {};
%\node[box=0] (p-51-29) at (29,-51) {};
%\node[box=0] (p-51-30) at (30,-51) {};
%\node[box=0] (p-51-31) at (31,-51) {};
%\node[box=1] (p-51-32) at (32,-51) {};
%\node[box=1] (p-51-33) at (33,-51) {};
%\node[box=1] (p-51-34) at (34,-51) {};
%\node[box=1] (p-51-35) at (35,-51) {};
%\node[box=0] (p-51-36) at (36,-51) {};
%\node[box=0] (p-51-37) at (37,-51) {};
%\node[box=0] (p-51-38) at (38,-51) {};
%\node[box=0] (p-51-39) at (39,-51) {};
%\node[box=0] (p-51-40) at (40,-51) {};
%\node[box=0] (p-51-41) at (41,-51) {};
%\node[box=0] (p-51-42) at (42,-51) {};
%\node[box=0] (p-51-43) at (43,-51) {};
%\node[box=0] (p-51-44) at (44,-51) {};
%\node[box=0] (p-51-45) at (45,-51) {};
%\node[box=0] (p-51-46) at (46,-51) {};
%\node[box=0] (p-51-47) at (47,-51) {};
%\node[box=1] (p-51-48) at (48,-51) {};
%\node[box=1] (p-51-49) at (49,-51) {};
%\node[box=1] (p-51-50) at (50,-51) {};
%\node[box=1] (p-51-51) at (51,-51) {};
%\node[box=1] (p-52-0) at (0,-52) {};
%\node[box=0] (p-52-1) at (1,-52) {};
%\node[box=0] (p-52-2) at (2,-52) {};
%\node[box=0] (p-52-3) at (3,-52) {};
%\node[box=1] (p-52-4) at (4,-52) {};
%\node[box=0] (p-52-5) at (5,-52) {};
%\node[box=0] (p-52-6) at (6,-52) {};
%\node[box=0] (p-52-7) at (7,-52) {};
%\node[box=0] (p-52-8) at (8,-52) {};
%\node[box=0] (p-52-9) at (9,-52) {};
%\node[box=0] (p-52-10) at (10,-52) {};
%\node[box=0] (p-52-11) at (11,-52) {};
%\node[box=0] (p-52-12) at (12,-52) {};
%\node[box=0] (p-52-13) at (13,-52) {};
%\node[box=0] (p-52-14) at (14,-52) {};
%\node[box=0] (p-52-15) at (15,-52) {};
%\node[box=1] (p-52-16) at (16,-52) {};
%\node[box=0] (p-52-17) at (17,-52) {};
%\node[box=0] (p-52-18) at (18,-52) {};
%\node[box=0] (p-52-19) at (19,-52) {};
%\node[box=1] (p-52-20) at (20,-52) {};
%\node[box=0] (p-52-21) at (21,-52) {};
%\node[box=0] (p-52-22) at (22,-52) {};
%\node[box=0] (p-52-23) at (23,-52) {};
%\node[box=0] (p-52-24) at (24,-52) {};
%\node[box=0] (p-52-25) at (25,-52) {};
%\node[box=0] (p-52-26) at (26,-52) {};
%\node[box=0] (p-52-27) at (27,-52) {};
%\node[box=0] (p-52-28) at (28,-52) {};
%\node[box=0] (p-52-29) at (29,-52) {};
%\node[box=0] (p-52-30) at (30,-52) {};
%\node[box=0] (p-52-31) at (31,-52) {};
%\node[box=1] (p-52-32) at (32,-52) {};
%\node[box=0] (p-52-33) at (33,-52) {};
%\node[box=0] (p-52-34) at (34,-52) {};
%\node[box=0] (p-52-35) at (35,-52) {};
%\node[box=1] (p-52-36) at (36,-52) {};
%\node[box=0] (p-52-37) at (37,-52) {};
%\node[box=0] (p-52-38) at (38,-52) {};
%\node[box=0] (p-52-39) at (39,-52) {};
%\node[box=0] (p-52-40) at (40,-52) {};
%\node[box=0] (p-52-41) at (41,-52) {};
%\node[box=0] (p-52-42) at (42,-52) {};
%\node[box=0] (p-52-43) at (43,-52) {};
%\node[box=0] (p-52-44) at (44,-52) {};
%\node[box=0] (p-52-45) at (45,-52) {};
%\node[box=0] (p-52-46) at (46,-52) {};
%\node[box=0] (p-52-47) at (47,-52) {};
%\node[box=1] (p-52-48) at (48,-52) {};
%\node[box=0] (p-52-49) at (49,-52) {};
%\node[box=0] (p-52-50) at (50,-52) {};
%\node[box=0] (p-52-51) at (51,-52) {};
%\node[box=1] (p-52-52) at (52,-52) {};
%\node[box=1] (p-53-0) at (0,-53) {};
%\node[box=1] (p-53-1) at (1,-53) {};
%\node[box=0] (p-53-2) at (2,-53) {};
%\node[box=0] (p-53-3) at (3,-53) {};
%\node[box=1] (p-53-4) at (4,-53) {};
%\node[box=1] (p-53-5) at (5,-53) {};
%\node[box=0] (p-53-6) at (6,-53) {};
%\node[box=0] (p-53-7) at (7,-53) {};
%\node[box=0] (p-53-8) at (8,-53) {};
%\node[box=0] (p-53-9) at (9,-53) {};
%\node[box=0] (p-53-10) at (10,-53) {};
%\node[box=0] (p-53-11) at (11,-53) {};
%\node[box=0] (p-53-12) at (12,-53) {};
%\node[box=0] (p-53-13) at (13,-53) {};
%\node[box=0] (p-53-14) at (14,-53) {};
%\node[box=0] (p-53-15) at (15,-53) {};
%\node[box=1] (p-53-16) at (16,-53) {};
%\node[box=1] (p-53-17) at (17,-53) {};
%\node[box=0] (p-53-18) at (18,-53) {};
%\node[box=0] (p-53-19) at (19,-53) {};
%\node[box=1] (p-53-20) at (20,-53) {};
%\node[box=1] (p-53-21) at (21,-53) {};
%\node[box=0] (p-53-22) at (22,-53) {};
%\node[box=0] (p-53-23) at (23,-53) {};
%\node[box=0] (p-53-24) at (24,-53) {};
%\node[box=0] (p-53-25) at (25,-53) {};
%\node[box=0] (p-53-26) at (26,-53) {};
%\node[box=0] (p-53-27) at (27,-53) {};
%\node[box=0] (p-53-28) at (28,-53) {};
%\node[box=0] (p-53-29) at (29,-53) {};
%\node[box=0] (p-53-30) at (30,-53) {};
%\node[box=0] (p-53-31) at (31,-53) {};
%\node[box=1] (p-53-32) at (32,-53) {};
%\node[box=1] (p-53-33) at (33,-53) {};
%\node[box=0] (p-53-34) at (34,-53) {};
%\node[box=0] (p-53-35) at (35,-53) {};
%\node[box=1] (p-53-36) at (36,-53) {};
%\node[box=1] (p-53-37) at (37,-53) {};
%\node[box=0] (p-53-38) at (38,-53) {};
%\node[box=0] (p-53-39) at (39,-53) {};
%\node[box=0] (p-53-40) at (40,-53) {};
%\node[box=0] (p-53-41) at (41,-53) {};
%\node[box=0] (p-53-42) at (42,-53) {};
%\node[box=0] (p-53-43) at (43,-53) {};
%\node[box=0] (p-53-44) at (44,-53) {};
%\node[box=0] (p-53-45) at (45,-53) {};
%\node[box=0] (p-53-46) at (46,-53) {};
%\node[box=0] (p-53-47) at (47,-53) {};
%\node[box=1] (p-53-48) at (48,-53) {};
%\node[box=1] (p-53-49) at (49,-53) {};
%\node[box=0] (p-53-50) at (50,-53) {};
%\node[box=0] (p-53-51) at (51,-53) {};
%\node[box=1] (p-53-52) at (52,-53) {};
%\node[box=1] (p-53-53) at (53,-53) {};
%\node[box=1] (p-54-0) at (0,-54) {};
%\node[box=0] (p-54-1) at (1,-54) {};
%\node[box=1] (p-54-2) at (2,-54) {};
%\node[box=0] (p-54-3) at (3,-54) {};
%\node[box=1] (p-54-4) at (4,-54) {};
%\node[box=0] (p-54-5) at (5,-54) {};
%\node[box=1] (p-54-6) at (6,-54) {};
%\node[box=0] (p-54-7) at (7,-54) {};
%\node[box=0] (p-54-8) at (8,-54) {};
%\node[box=0] (p-54-9) at (9,-54) {};
%\node[box=0] (p-54-10) at (10,-54) {};
%\node[box=0] (p-54-11) at (11,-54) {};
%\node[box=0] (p-54-12) at (12,-54) {};
%\node[box=0] (p-54-13) at (13,-54) {};
%\node[box=0] (p-54-14) at (14,-54) {};
%\node[box=0] (p-54-15) at (15,-54) {};
%\node[box=1] (p-54-16) at (16,-54) {};
%\node[box=0] (p-54-17) at (17,-54) {};
%\node[box=1] (p-54-18) at (18,-54) {};
%\node[box=0] (p-54-19) at (19,-54) {};
%\node[box=1] (p-54-20) at (20,-54) {};
%\node[box=0] (p-54-21) at (21,-54) {};
%\node[box=1] (p-54-22) at (22,-54) {};
%\node[box=0] (p-54-23) at (23,-54) {};
%\node[box=0] (p-54-24) at (24,-54) {};
%\node[box=0] (p-54-25) at (25,-54) {};
%\node[box=0] (p-54-26) at (26,-54) {};
%\node[box=0] (p-54-27) at (27,-54) {};
%\node[box=0] (p-54-28) at (28,-54) {};
%\node[box=0] (p-54-29) at (29,-54) {};
%\node[box=0] (p-54-30) at (30,-54) {};
%\node[box=0] (p-54-31) at (31,-54) {};
%\node[box=1] (p-54-32) at (32,-54) {};
%\node[box=0] (p-54-33) at (33,-54) {};
%\node[box=1] (p-54-34) at (34,-54) {};
%\node[box=0] (p-54-35) at (35,-54) {};
%\node[box=1] (p-54-36) at (36,-54) {};
%\node[box=0] (p-54-37) at (37,-54) {};
%\node[box=1] (p-54-38) at (38,-54) {};
%\node[box=0] (p-54-39) at (39,-54) {};
%\node[box=0] (p-54-40) at (40,-54) {};
%\node[box=0] (p-54-41) at (41,-54) {};
%\node[box=0] (p-54-42) at (42,-54) {};
%\node[box=0] (p-54-43) at (43,-54) {};
%\node[box=0] (p-54-44) at (44,-54) {};
%\node[box=0] (p-54-45) at (45,-54) {};
%\node[box=0] (p-54-46) at (46,-54) {};
%\node[box=0] (p-54-47) at (47,-54) {};
%\node[box=1] (p-54-48) at (48,-54) {};
%\node[box=0] (p-54-49) at (49,-54) {};
%\node[box=1] (p-54-50) at (50,-54) {};
%\node[box=0] (p-54-51) at (51,-54) {};
%\node[box=1] (p-54-52) at (52,-54) {};
%\node[box=0] (p-54-53) at (53,-54) {};
%\node[box=1] (p-54-54) at (54,-54) {};
%\node[box=1] (p-55-0) at (0,-55) {};
%\node[box=1] (p-55-1) at (1,-55) {};
%\node[box=1] (p-55-2) at (2,-55) {};
%\node[box=1] (p-55-3) at (3,-55) {};
%\node[box=1] (p-55-4) at (4,-55) {};
%\node[box=1] (p-55-5) at (5,-55) {};
%\node[box=1] (p-55-6) at (6,-55) {};
%\node[box=1] (p-55-7) at (7,-55) {};
%\node[box=0] (p-55-8) at (8,-55) {};
%\node[box=0] (p-55-9) at (9,-55) {};
%\node[box=0] (p-55-10) at (10,-55) {};
%\node[box=0] (p-55-11) at (11,-55) {};
%\node[box=0] (p-55-12) at (12,-55) {};
%\node[box=0] (p-55-13) at (13,-55) {};
%\node[box=0] (p-55-14) at (14,-55) {};
%\node[box=0] (p-55-15) at (15,-55) {};
%\node[box=1] (p-55-16) at (16,-55) {};
%\node[box=1] (p-55-17) at (17,-55) {};
%\node[box=1] (p-55-18) at (18,-55) {};
%\node[box=1] (p-55-19) at (19,-55) {};
%\node[box=1] (p-55-20) at (20,-55) {};
%\node[box=1] (p-55-21) at (21,-55) {};
%\node[box=1] (p-55-22) at (22,-55) {};
%\node[box=1] (p-55-23) at (23,-55) {};
%\node[box=0] (p-55-24) at (24,-55) {};
%\node[box=0] (p-55-25) at (25,-55) {};
%\node[box=0] (p-55-26) at (26,-55) {};
%\node[box=0] (p-55-27) at (27,-55) {};
%\node[box=0] (p-55-28) at (28,-55) {};
%\node[box=0] (p-55-29) at (29,-55) {};
%\node[box=0] (p-55-30) at (30,-55) {};
%\node[box=0] (p-55-31) at (31,-55) {};
%\node[box=1] (p-55-32) at (32,-55) {};
%\node[box=1] (p-55-33) at (33,-55) {};
%\node[box=1] (p-55-34) at (34,-55) {};
%\node[box=1] (p-55-35) at (35,-55) {};
%\node[box=1] (p-55-36) at (36,-55) {};
%\node[box=1] (p-55-37) at (37,-55) {};
%\node[box=1] (p-55-38) at (38,-55) {};
%\node[box=1] (p-55-39) at (39,-55) {};
%\node[box=0] (p-55-40) at (40,-55) {};
%\node[box=0] (p-55-41) at (41,-55) {};
%\node[box=0] (p-55-42) at (42,-55) {};
%\node[box=0] (p-55-43) at (43,-55) {};
%\node[box=0] (p-55-44) at (44,-55) {};
%\node[box=0] (p-55-45) at (45,-55) {};
%\node[box=0] (p-55-46) at (46,-55) {};
%\node[box=0] (p-55-47) at (47,-55) {};
%\node[box=1] (p-55-48) at (48,-55) {};
%\node[box=1] (p-55-49) at (49,-55) {};
%\node[box=1] (p-55-50) at (50,-55) {};
%\node[box=1] (p-55-51) at (51,-55) {};
%\node[box=1] (p-55-52) at (52,-55) {};
%\node[box=1] (p-55-53) at (53,-55) {};
%\node[box=1] (p-55-54) at (54,-55) {};
%\node[box=1] (p-55-55) at (55,-55) {};
%\node[box=1] (p-56-0) at (0,-56) {};
%\node[box=0] (p-56-1) at (1,-56) {};
%\node[box=0] (p-56-2) at (2,-56) {};
%\node[box=0] (p-56-3) at (3,-56) {};
%\node[box=0] (p-56-4) at (4,-56) {};
%\node[box=0] (p-56-5) at (5,-56) {};
%\node[box=0] (p-56-6) at (6,-56) {};
%\node[box=0] (p-56-7) at (7,-56) {};
%\node[box=1] (p-56-8) at (8,-56) {};
%\node[box=0] (p-56-9) at (9,-56) {};
%\node[box=0] (p-56-10) at (10,-56) {};
%\node[box=0] (p-56-11) at (11,-56) {};
%\node[box=0] (p-56-12) at (12,-56) {};
%\node[box=0] (p-56-13) at (13,-56) {};
%\node[box=0] (p-56-14) at (14,-56) {};
%\node[box=0] (p-56-15) at (15,-56) {};
%\node[box=1] (p-56-16) at (16,-56) {};
%\node[box=0] (p-56-17) at (17,-56) {};
%\node[box=0] (p-56-18) at (18,-56) {};
%\node[box=0] (p-56-19) at (19,-56) {};
%\node[box=0] (p-56-20) at (20,-56) {};
%\node[box=0] (p-56-21) at (21,-56) {};
%\node[box=0] (p-56-22) at (22,-56) {};
%\node[box=0] (p-56-23) at (23,-56) {};
%\node[box=1] (p-56-24) at (24,-56) {};
%\node[box=0] (p-56-25) at (25,-56) {};
%\node[box=0] (p-56-26) at (26,-56) {};
%\node[box=0] (p-56-27) at (27,-56) {};
%\node[box=0] (p-56-28) at (28,-56) {};
%\node[box=0] (p-56-29) at (29,-56) {};
%\node[box=0] (p-56-30) at (30,-56) {};
%\node[box=0] (p-56-31) at (31,-56) {};
%\node[box=1] (p-56-32) at (32,-56) {};
%\node[box=0] (p-56-33) at (33,-56) {};
%\node[box=0] (p-56-34) at (34,-56) {};
%\node[box=0] (p-56-35) at (35,-56) {};
%\node[box=0] (p-56-36) at (36,-56) {};
%\node[box=0] (p-56-37) at (37,-56) {};
%\node[box=0] (p-56-38) at (38,-56) {};
%\node[box=0] (p-56-39) at (39,-56) {};
%\node[box=1] (p-56-40) at (40,-56) {};
%\node[box=0] (p-56-41) at (41,-56) {};
%\node[box=0] (p-56-42) at (42,-56) {};
%\node[box=0] (p-56-43) at (43,-56) {};
%\node[box=0] (p-56-44) at (44,-56) {};
%\node[box=0] (p-56-45) at (45,-56) {};
%\node[box=0] (p-56-46) at (46,-56) {};
%\node[box=0] (p-56-47) at (47,-56) {};
%\node[box=1] (p-56-48) at (48,-56) {};
%\node[box=0] (p-56-49) at (49,-56) {};
%\node[box=0] (p-56-50) at (50,-56) {};
%\node[box=0] (p-56-51) at (51,-56) {};
%\node[box=0] (p-56-52) at (52,-56) {};
%\node[box=0] (p-56-53) at (53,-56) {};
%\node[box=0] (p-56-54) at (54,-56) {};
%\node[box=0] (p-56-55) at (55,-56) {};
%\node[box=1] (p-56-56) at (56,-56) {};
%\node[box=1] (p-57-0) at (0,-57) {};
%\node[box=1] (p-57-1) at (1,-57) {};
%\node[box=0] (p-57-2) at (2,-57) {};
%\node[box=0] (p-57-3) at (3,-57) {};
%\node[box=0] (p-57-4) at (4,-57) {};
%\node[box=0] (p-57-5) at (5,-57) {};
%\node[box=0] (p-57-6) at (6,-57) {};
%\node[box=0] (p-57-7) at (7,-57) {};
%\node[box=1] (p-57-8) at (8,-57) {};
%\node[box=1] (p-57-9) at (9,-57) {};
%\node[box=0] (p-57-10) at (10,-57) {};
%\node[box=0] (p-57-11) at (11,-57) {};
%\node[box=0] (p-57-12) at (12,-57) {};
%\node[box=0] (p-57-13) at (13,-57) {};
%\node[box=0] (p-57-14) at (14,-57) {};
%\node[box=0] (p-57-15) at (15,-57) {};
%\node[box=1] (p-57-16) at (16,-57) {};
%\node[box=1] (p-57-17) at (17,-57) {};
%\node[box=0] (p-57-18) at (18,-57) {};
%\node[box=0] (p-57-19) at (19,-57) {};
%\node[box=0] (p-57-20) at (20,-57) {};
%\node[box=0] (p-57-21) at (21,-57) {};
%\node[box=0] (p-57-22) at (22,-57) {};
%\node[box=0] (p-57-23) at (23,-57) {};
%\node[box=1] (p-57-24) at (24,-57) {};
%\node[box=1] (p-57-25) at (25,-57) {};
%\node[box=0] (p-57-26) at (26,-57) {};
%\node[box=0] (p-57-27) at (27,-57) {};
%\node[box=0] (p-57-28) at (28,-57) {};
%\node[box=0] (p-57-29) at (29,-57) {};
%\node[box=0] (p-57-30) at (30,-57) {};
%\node[box=0] (p-57-31) at (31,-57) {};
%\node[box=1] (p-57-32) at (32,-57) {};
%\node[box=1] (p-57-33) at (33,-57) {};
%\node[box=0] (p-57-34) at (34,-57) {};
%\node[box=0] (p-57-35) at (35,-57) {};
%\node[box=0] (p-57-36) at (36,-57) {};
%\node[box=0] (p-57-37) at (37,-57) {};
%\node[box=0] (p-57-38) at (38,-57) {};
%\node[box=0] (p-57-39) at (39,-57) {};
%\node[box=1] (p-57-40) at (40,-57) {};
%\node[box=1] (p-57-41) at (41,-57) {};
%\node[box=0] (p-57-42) at (42,-57) {};
%\node[box=0] (p-57-43) at (43,-57) {};
%\node[box=0] (p-57-44) at (44,-57) {};
%\node[box=0] (p-57-45) at (45,-57) {};
%\node[box=0] (p-57-46) at (46,-57) {};
%\node[box=0] (p-57-47) at (47,-57) {};
%\node[box=1] (p-57-48) at (48,-57) {};
%\node[box=1] (p-57-49) at (49,-57) {};
%\node[box=0] (p-57-50) at (50,-57) {};
%\node[box=0] (p-57-51) at (51,-57) {};
%\node[box=0] (p-57-52) at (52,-57) {};
%\node[box=0] (p-57-53) at (53,-57) {};
%\node[box=0] (p-57-54) at (54,-57) {};
%\node[box=0] (p-57-55) at (55,-57) {};
%\node[box=1] (p-57-56) at (56,-57) {};
%\node[box=1] (p-57-57) at (57,-57) {};
%\node[box=1] (p-58-0) at (0,-58) {};
%\node[box=0] (p-58-1) at (1,-58) {};
%\node[box=1] (p-58-2) at (2,-58) {};
%\node[box=0] (p-58-3) at (3,-58) {};
%\node[box=0] (p-58-4) at (4,-58) {};
%\node[box=0] (p-58-5) at (5,-58) {};
%\node[box=0] (p-58-6) at (6,-58) {};
%\node[box=0] (p-58-7) at (7,-58) {};
%\node[box=1] (p-58-8) at (8,-58) {};
%\node[box=0] (p-58-9) at (9,-58) {};
%\node[box=1] (p-58-10) at (10,-58) {};
%\node[box=0] (p-58-11) at (11,-58) {};
%\node[box=0] (p-58-12) at (12,-58) {};
%\node[box=0] (p-58-13) at (13,-58) {};
%\node[box=0] (p-58-14) at (14,-58) {};
%\node[box=0] (p-58-15) at (15,-58) {};
%\node[box=1] (p-58-16) at (16,-58) {};
%\node[box=0] (p-58-17) at (17,-58) {};
%\node[box=1] (p-58-18) at (18,-58) {};
%\node[box=0] (p-58-19) at (19,-58) {};
%\node[box=0] (p-58-20) at (20,-58) {};
%\node[box=0] (p-58-21) at (21,-58) {};
%\node[box=0] (p-58-22) at (22,-58) {};
%\node[box=0] (p-58-23) at (23,-58) {};
%\node[box=1] (p-58-24) at (24,-58) {};
%\node[box=0] (p-58-25) at (25,-58) {};
%\node[box=1] (p-58-26) at (26,-58) {};
%\node[box=0] (p-58-27) at (27,-58) {};
%\node[box=0] (p-58-28) at (28,-58) {};
%\node[box=0] (p-58-29) at (29,-58) {};
%\node[box=0] (p-58-30) at (30,-58) {};
%\node[box=0] (p-58-31) at (31,-58) {};
%\node[box=1] (p-58-32) at (32,-58) {};
%\node[box=0] (p-58-33) at (33,-58) {};
%\node[box=1] (p-58-34) at (34,-58) {};
%\node[box=0] (p-58-35) at (35,-58) {};
%\node[box=0] (p-58-36) at (36,-58) {};
%\node[box=0] (p-58-37) at (37,-58) {};
%\node[box=0] (p-58-38) at (38,-58) {};
%\node[box=0] (p-58-39) at (39,-58) {};
%\node[box=1] (p-58-40) at (40,-58) {};
%\node[box=0] (p-58-41) at (41,-58) {};
%\node[box=1] (p-58-42) at (42,-58) {};
%\node[box=0] (p-58-43) at (43,-58) {};
%\node[box=0] (p-58-44) at (44,-58) {};
%\node[box=0] (p-58-45) at (45,-58) {};
%\node[box=0] (p-58-46) at (46,-58) {};
%\node[box=0] (p-58-47) at (47,-58) {};
%\node[box=1] (p-58-48) at (48,-58) {};
%\node[box=0] (p-58-49) at (49,-58) {};
%\node[box=1] (p-58-50) at (50,-58) {};
%\node[box=0] (p-58-51) at (51,-58) {};
%\node[box=0] (p-58-52) at (52,-58) {};
%\node[box=0] (p-58-53) at (53,-58) {};
%\node[box=0] (p-58-54) at (54,-58) {};
%\node[box=0] (p-58-55) at (55,-58) {};
%\node[box=1] (p-58-56) at (56,-58) {};
%\node[box=0] (p-58-57) at (57,-58) {};
%\node[box=1] (p-58-58) at (58,-58) {};
%\node[box=1] (p-59-0) at (0,-59) {};
%\node[box=1] (p-59-1) at (1,-59) {};
%\node[box=1] (p-59-2) at (2,-59) {};
%\node[box=1] (p-59-3) at (3,-59) {};
%\node[box=0] (p-59-4) at (4,-59) {};
%\node[box=0] (p-59-5) at (5,-59) {};
%\node[box=0] (p-59-6) at (6,-59) {};
%\node[box=0] (p-59-7) at (7,-59) {};
%\node[box=1] (p-59-8) at (8,-59) {};
%\node[box=1] (p-59-9) at (9,-59) {};
%\node[box=1] (p-59-10) at (10,-59) {};
%\node[box=1] (p-59-11) at (11,-59) {};
%\node[box=0] (p-59-12) at (12,-59) {};
%\node[box=0] (p-59-13) at (13,-59) {};
%\node[box=0] (p-59-14) at (14,-59) {};
%\node[box=0] (p-59-15) at (15,-59) {};
%\node[box=1] (p-59-16) at (16,-59) {};
%\node[box=1] (p-59-17) at (17,-59) {};
%\node[box=1] (p-59-18) at (18,-59) {};
%\node[box=1] (p-59-19) at (19,-59) {};
%\node[box=0] (p-59-20) at (20,-59) {};
%\node[box=0] (p-59-21) at (21,-59) {};
%\node[box=0] (p-59-22) at (22,-59) {};
%\node[box=0] (p-59-23) at (23,-59) {};
%\node[box=1] (p-59-24) at (24,-59) {};
%\node[box=1] (p-59-25) at (25,-59) {};
%\node[box=1] (p-59-26) at (26,-59) {};
%\node[box=1] (p-59-27) at (27,-59) {};
%\node[box=0] (p-59-28) at (28,-59) {};
%\node[box=0] (p-59-29) at (29,-59) {};
%\node[box=0] (p-59-30) at (30,-59) {};
%\node[box=0] (p-59-31) at (31,-59) {};
%\node[box=1] (p-59-32) at (32,-59) {};
%\node[box=1] (p-59-33) at (33,-59) {};
%\node[box=1] (p-59-34) at (34,-59) {};
%\node[box=1] (p-59-35) at (35,-59) {};
%\node[box=0] (p-59-36) at (36,-59) {};
%\node[box=0] (p-59-37) at (37,-59) {};
%\node[box=0] (p-59-38) at (38,-59) {};
%\node[box=0] (p-59-39) at (39,-59) {};
%\node[box=1] (p-59-40) at (40,-59) {};
%\node[box=1] (p-59-41) at (41,-59) {};
%\node[box=1] (p-59-42) at (42,-59) {};
%\node[box=1] (p-59-43) at (43,-59) {};
%\node[box=0] (p-59-44) at (44,-59) {};
%\node[box=0] (p-59-45) at (45,-59) {};
%\node[box=0] (p-59-46) at (46,-59) {};
%\node[box=0] (p-59-47) at (47,-59) {};
%\node[box=1] (p-59-48) at (48,-59) {};
%\node[box=1] (p-59-49) at (49,-59) {};
%\node[box=1] (p-59-50) at (50,-59) {};
%\node[box=1] (p-59-51) at (51,-59) {};
%\node[box=0] (p-59-52) at (52,-59) {};
%\node[box=0] (p-59-53) at (53,-59) {};
%\node[box=0] (p-59-54) at (54,-59) {};
%\node[box=0] (p-59-55) at (55,-59) {};
%\node[box=1] (p-59-56) at (56,-59) {};
%\node[box=1] (p-59-57) at (57,-59) {};
%\node[box=1] (p-59-58) at (58,-59) {};
%\node[box=1] (p-59-59) at (59,-59) {};
%\node[box=1] (p-60-0) at (0,-60) {};
%\node[box=0] (p-60-1) at (1,-60) {};
%\node[box=0] (p-60-2) at (2,-60) {};
%\node[box=0] (p-60-3) at (3,-60) {};
%\node[box=1] (p-60-4) at (4,-60) {};
%\node[box=0] (p-60-5) at (5,-60) {};
%\node[box=0] (p-60-6) at (6,-60) {};
%\node[box=0] (p-60-7) at (7,-60) {};
%\node[box=1] (p-60-8) at (8,-60) {};
%\node[box=0] (p-60-9) at (9,-60) {};
%\node[box=0] (p-60-10) at (10,-60) {};
%\node[box=0] (p-60-11) at (11,-60) {};
%\node[box=1] (p-60-12) at (12,-60) {};
%\node[box=0] (p-60-13) at (13,-60) {};
%\node[box=0] (p-60-14) at (14,-60) {};
%\node[box=0] (p-60-15) at (15,-60) {};
%\node[box=1] (p-60-16) at (16,-60) {};
%\node[box=0] (p-60-17) at (17,-60) {};
%\node[box=0] (p-60-18) at (18,-60) {};
%\node[box=0] (p-60-19) at (19,-60) {};
%\node[box=1] (p-60-20) at (20,-60) {};
%\node[box=0] (p-60-21) at (21,-60) {};
%\node[box=0] (p-60-22) at (22,-60) {};
%\node[box=0] (p-60-23) at (23,-60) {};
%\node[box=1] (p-60-24) at (24,-60) {};
%\node[box=0] (p-60-25) at (25,-60) {};
%\node[box=0] (p-60-26) at (26,-60) {};
%\node[box=0] (p-60-27) at (27,-60) {};
%\node[box=1] (p-60-28) at (28,-60) {};
%\node[box=0] (p-60-29) at (29,-60) {};
%\node[box=0] (p-60-30) at (30,-60) {};
%\node[box=0] (p-60-31) at (31,-60) {};
%\node[box=1] (p-60-32) at (32,-60) {};
%\node[box=0] (p-60-33) at (33,-60) {};
%\node[box=0] (p-60-34) at (34,-60) {};
%\node[box=0] (p-60-35) at (35,-60) {};
%\node[box=1] (p-60-36) at (36,-60) {};
%\node[box=0] (p-60-37) at (37,-60) {};
%\node[box=0] (p-60-38) at (38,-60) {};
%\node[box=0] (p-60-39) at (39,-60) {};
%\node[box=1] (p-60-40) at (40,-60) {};
%\node[box=0] (p-60-41) at (41,-60) {};
%\node[box=0] (p-60-42) at (42,-60) {};
%\node[box=0] (p-60-43) at (43,-60) {};
%\node[box=1] (p-60-44) at (44,-60) {};
%\node[box=0] (p-60-45) at (45,-60) {};
%\node[box=0] (p-60-46) at (46,-60) {};
%\node[box=0] (p-60-47) at (47,-60) {};
%\node[box=1] (p-60-48) at (48,-60) {};
%\node[box=0] (p-60-49) at (49,-60) {};
%\node[box=0] (p-60-50) at (50,-60) {};
%\node[box=0] (p-60-51) at (51,-60) {};
%\node[box=1] (p-60-52) at (52,-60) {};
%\node[box=0] (p-60-53) at (53,-60) {};
%\node[box=0] (p-60-54) at (54,-60) {};
%\node[box=0] (p-60-55) at (55,-60) {};
%\node[box=1] (p-60-56) at (56,-60) {};
%\node[box=0] (p-60-57) at (57,-60) {};
%\node[box=0] (p-60-58) at (58,-60) {};
%\node[box=0] (p-60-59) at (59,-60) {};
%\node[box=1] (p-60-60) at (60,-60) {};
%\node[box=1] (p-61-0) at (0,-61) {};
%\node[box=1] (p-61-1) at (1,-61) {};
%\node[box=0] (p-61-2) at (2,-61) {};
%\node[box=0] (p-61-3) at (3,-61) {};
%\node[box=1] (p-61-4) at (4,-61) {};
%\node[box=1] (p-61-5) at (5,-61) {};
%\node[box=0] (p-61-6) at (6,-61) {};
%\node[box=0] (p-61-7) at (7,-61) {};
%\node[box=1] (p-61-8) at (8,-61) {};
%\node[box=1] (p-61-9) at (9,-61) {};
%\node[box=0] (p-61-10) at (10,-61) {};
%\node[box=0] (p-61-11) at (11,-61) {};
%\node[box=1] (p-61-12) at (12,-61) {};
%\node[box=1] (p-61-13) at (13,-61) {};
%\node[box=0] (p-61-14) at (14,-61) {};
%\node[box=0] (p-61-15) at (15,-61) {};
%\node[box=1] (p-61-16) at (16,-61) {};
%\node[box=1] (p-61-17) at (17,-61) {};
%\node[box=0] (p-61-18) at (18,-61) {};
%\node[box=0] (p-61-19) at (19,-61) {};
%\node[box=1] (p-61-20) at (20,-61) {};
%\node[box=1] (p-61-21) at (21,-61) {};
%\node[box=0] (p-61-22) at (22,-61) {};
%\node[box=0] (p-61-23) at (23,-61) {};
%\node[box=1] (p-61-24) at (24,-61) {};
%\node[box=1] (p-61-25) at (25,-61) {};
%\node[box=0] (p-61-26) at (26,-61) {};
%\node[box=0] (p-61-27) at (27,-61) {};
%\node[box=1] (p-61-28) at (28,-61) {};
%\node[box=1] (p-61-29) at (29,-61) {};
%\node[box=0] (p-61-30) at (30,-61) {};
%\node[box=0] (p-61-31) at (31,-61) {};
%\node[box=1] (p-61-32) at (32,-61) {};
%\node[box=1] (p-61-33) at (33,-61) {};
%\node[box=0] (p-61-34) at (34,-61) {};
%\node[box=0] (p-61-35) at (35,-61) {};
%\node[box=1] (p-61-36) at (36,-61) {};
%\node[box=1] (p-61-37) at (37,-61) {};
%\node[box=0] (p-61-38) at (38,-61) {};
%\node[box=0] (p-61-39) at (39,-61) {};
%\node[box=1] (p-61-40) at (40,-61) {};
%\node[box=1] (p-61-41) at (41,-61) {};
%\node[box=0] (p-61-42) at (42,-61) {};
%\node[box=0] (p-61-43) at (43,-61) {};
%\node[box=1] (p-61-44) at (44,-61) {};
%\node[box=1] (p-61-45) at (45,-61) {};
%\node[box=0] (p-61-46) at (46,-61) {};
%\node[box=0] (p-61-47) at (47,-61) {};
%\node[box=1] (p-61-48) at (48,-61) {};
%\node[box=1] (p-61-49) at (49,-61) {};
%\node[box=0] (p-61-50) at (50,-61) {};
%\node[box=0] (p-61-51) at (51,-61) {};
%\node[box=1] (p-61-52) at (52,-61) {};
%\node[box=1] (p-61-53) at (53,-61) {};
%\node[box=0] (p-61-54) at (54,-61) {};
%\node[box=0] (p-61-55) at (55,-61) {};
%\node[box=1] (p-61-56) at (56,-61) {};
%\node[box=1] (p-61-57) at (57,-61) {};
%\node[box=0] (p-61-58) at (58,-61) {};
%\node[box=0] (p-61-59) at (59,-61) {};
%\node[box=1] (p-61-60) at (60,-61) {};
%\node[box=1] (p-61-61) at (61,-61) {};
%\node[box=1] (p-62-0) at (0,-62) {};
%\node[box=0] (p-62-1) at (1,-62) {};
%\node[box=1] (p-62-2) at (2,-62) {};
%\node[box=0] (p-62-3) at (3,-62) {};
%\node[box=1] (p-62-4) at (4,-62) {};
%\node[box=0] (p-62-5) at (5,-62) {};
%\node[box=1] (p-62-6) at (6,-62) {};
%\node[box=0] (p-62-7) at (7,-62) {};
%\node[box=1] (p-62-8) at (8,-62) {};
%\node[box=0] (p-62-9) at (9,-62) {};
%\node[box=1] (p-62-10) at (10,-62) {};
%\node[box=0] (p-62-11) at (11,-62) {};
%\node[box=1] (p-62-12) at (12,-62) {};
%\node[box=0] (p-62-13) at (13,-62) {};
%\node[box=1] (p-62-14) at (14,-62) {};
%\node[box=0] (p-62-15) at (15,-62) {};
%\node[box=1] (p-62-16) at (16,-62) {};
%\node[box=0] (p-62-17) at (17,-62) {};
%\node[box=1] (p-62-18) at (18,-62) {};
%\node[box=0] (p-62-19) at (19,-62) {};
%\node[box=1] (p-62-20) at (20,-62) {};
%\node[box=0] (p-62-21) at (21,-62) {};
%\node[box=1] (p-62-22) at (22,-62) {};
%\node[box=0] (p-62-23) at (23,-62) {};
%\node[box=1] (p-62-24) at (24,-62) {};
%\node[box=0] (p-62-25) at (25,-62) {};
%\node[box=1] (p-62-26) at (26,-62) {};
%\node[box=0] (p-62-27) at (27,-62) {};
%\node[box=1] (p-62-28) at (28,-62) {};
%\node[box=0] (p-62-29) at (29,-62) {};
%\node[box=1] (p-62-30) at (30,-62) {};
%\node[box=0] (p-62-31) at (31,-62) {};
%\node[box=1] (p-62-32) at (32,-62) {};
%\node[box=0] (p-62-33) at (33,-62) {};
%\node[box=1] (p-62-34) at (34,-62) {};
%\node[box=0] (p-62-35) at (35,-62) {};
%\node[box=1] (p-62-36) at (36,-62) {};
%\node[box=0] (p-62-37) at (37,-62) {};
%\node[box=1] (p-62-38) at (38,-62) {};
%\node[box=0] (p-62-39) at (39,-62) {};
%\node[box=1] (p-62-40) at (40,-62) {};
%\node[box=0] (p-62-41) at (41,-62) {};
%\node[box=1] (p-62-42) at (42,-62) {};
%\node[box=0] (p-62-43) at (43,-62) {};
%\node[box=1] (p-62-44) at (44,-62) {};
%\node[box=0] (p-62-45) at (45,-62) {};
%\node[box=1] (p-62-46) at (46,-62) {};
%\node[box=0] (p-62-47) at (47,-62) {};
%\node[box=1] (p-62-48) at (48,-62) {};
%\node[box=0] (p-62-49) at (49,-62) {};
%\node[box=1] (p-62-50) at (50,-62) {};
%\node[box=0] (p-62-51) at (51,-62) {};
%\node[box=1] (p-62-52) at (52,-62) {};
%\node[box=0] (p-62-53) at (53,-62) {};
%\node[box=1] (p-62-54) at (54,-62) {};
%\node[box=0] (p-62-55) at (55,-62) {};
%\node[box=1] (p-62-56) at (56,-62) {};
%\node[box=0] (p-62-57) at (57,-62) {};
%\node[box=1] (p-62-58) at (58,-62) {};
%\node[box=0] (p-62-59) at (59,-62) {};
%\node[box=1] (p-62-60) at (60,-62) {};
%\node[box=0] (p-62-61) at (61,-62) {};
%\node[box=1] (p-62-62) at (62,-62) {};
%\node[box=1] (p-63-0) at (0,-63) {};
%\node[box=1] (p-63-1) at (1,-63) {};
%\node[box=1] (p-63-2) at (2,-63) {};
%\node[box=1] (p-63-3) at (3,-63) {};
%\node[box=1] (p-63-4) at (4,-63) {};
%\node[box=1] (p-63-5) at (5,-63) {};
%\node[box=1] (p-63-6) at (6,-63) {};
%\node[box=1] (p-63-7) at (7,-63) {};
%\node[box=1] (p-63-8) at (8,-63) {};
%\node[box=1] (p-63-9) at (9,-63) {};
%\node[box=1] (p-63-10) at (10,-63) {};
%\node[box=1] (p-63-11) at (11,-63) {};
%\node[box=1] (p-63-12) at (12,-63) {};
%\node[box=1] (p-63-13) at (13,-63) {};
%\node[box=1] (p-63-14) at (14,-63) {};
%\node[box=1] (p-63-15) at (15,-63) {};
%\node[box=1] (p-63-16) at (16,-63) {};
%\node[box=1] (p-63-17) at (17,-63) {};
%\node[box=1] (p-63-18) at (18,-63) {};
%\node[box=1] (p-63-19) at (19,-63) {};
%\node[box=1] (p-63-20) at (20,-63) {};
%\node[box=1] (p-63-21) at (21,-63) {};
%\node[box=1] (p-63-22) at (22,-63) {};
%\node[box=1] (p-63-23) at (23,-63) {};
%\node[box=1] (p-63-24) at (24,-63) {};
%\node[box=1] (p-63-25) at (25,-63) {};
%\node[box=1] (p-63-26) at (26,-63) {};
%\node[box=1] (p-63-27) at (27,-63) {};
%\node[box=1] (p-63-28) at (28,-63) {};
%\node[box=1] (p-63-29) at (29,-63) {};
%\node[box=1] (p-63-30) at (30,-63) {};
%\node[box=1] (p-63-31) at (31,-63) {};
%\node[box=1] (p-63-32) at (32,-63) {};
%\node[box=1] (p-63-33) at (33,-63) {};
%\node[box=1] (p-63-34) at (34,-63) {};
%\node[box=1] (p-63-35) at (35,-63) {};
%\node[box=1] (p-63-36) at (36,-63) {};
%\node[box=1] (p-63-37) at (37,-63) {};
%\node[box=1] (p-63-38) at (38,-63) {};
%\node[box=1] (p-63-39) at (39,-63) {};
%\node[box=1] (p-63-40) at (40,-63) {};
%\node[box=1] (p-63-41) at (41,-63) {};
%\node[box=1] (p-63-42) at (42,-63) {};
%\node[box=1] (p-63-43) at (43,-63) {};
%\node[box=1] (p-63-44) at (44,-63) {};
%\node[box=1] (p-63-45) at (45,-63) {};
%\node[box=1] (p-63-46) at (46,-63) {};
%\node[box=1] (p-63-47) at (47,-63) {};
%\node[box=1] (p-63-48) at (48,-63) {};
%\node[box=1] (p-63-49) at (49,-63) {};
%\node[box=1] (p-63-50) at (50,-63) {};
%\node[box=1] (p-63-51) at (51,-63) {};
%\node[box=1] (p-63-52) at (52,-63) {};
%\node[box=1] (p-63-53) at (53,-63) {};
%\node[box=1] (p-63-54) at (54,-63) {};
%\node[box=1] (p-63-55) at (55,-63) {};
%\node[box=1] (p-63-56) at (56,-63) {};
%\node[box=1] (p-63-57) at (57,-63) {};
%\node[box=1] (p-63-58) at (58,-63) {};
%\node[box=1] (p-63-59) at (59,-63) {};
%\node[box=1] (p-63-60) at (60,-63) {};
%\node[box=1] (p-63-61) at (61,-63) {};
%\node[box=1] (p-63-62) at (62,-63) {};
%\node[box=1] (p-63-63) at (63,-63) {};
%\node[box=1] (p-64-0) at (0,-64) {};
%\node[box=0] (p-64-1) at (1,-64) {};
%\node[box=0] (p-64-2) at (2,-64) {};
%\node[box=0] (p-64-3) at (3,-64) {};
%\node[box=0] (p-64-4) at (4,-64) {};
%\node[box=0] (p-64-5) at (5,-64) {};
%\node[box=0] (p-64-6) at (6,-64) {};
%\node[box=0] (p-64-7) at (7,-64) {};
%\node[box=0] (p-64-8) at (8,-64) {};
%\node[box=0] (p-64-9) at (9,-64) {};
%\node[box=0] (p-64-10) at (10,-64) {};
%\node[box=0] (p-64-11) at (11,-64) {};
%\node[box=0] (p-64-12) at (12,-64) {};
%\node[box=0] (p-64-13) at (13,-64) {};
%\node[box=0] (p-64-14) at (14,-64) {};
%\node[box=0] (p-64-15) at (15,-64) {};
%\node[box=0] (p-64-16) at (16,-64) {};
%\node[box=0] (p-64-17) at (17,-64) {};
%\node[box=0] (p-64-18) at (18,-64) {};
%\node[box=0] (p-64-19) at (19,-64) {};
%\node[box=0] (p-64-20) at (20,-64) {};
%\node[box=0] (p-64-21) at (21,-64) {};
%\node[box=0] (p-64-22) at (22,-64) {};
%\node[box=0] (p-64-23) at (23,-64) {};
%\node[box=0] (p-64-24) at (24,-64) {};
%\node[box=0] (p-64-25) at (25,-64) {};
%\node[box=0] (p-64-26) at (26,-64) {};
%\node[box=0] (p-64-27) at (27,-64) {};
%\node[box=0] (p-64-28) at (28,-64) {};
%\node[box=0] (p-64-29) at (29,-64) {};
%\node[box=0] (p-64-30) at (30,-64) {};
%\node[box=0] (p-64-31) at (31,-64) {};
%\node[box=0] (p-64-32) at (32,-64) {};
%\node[box=0] (p-64-33) at (33,-64) {};
%\node[box=0] (p-64-34) at (34,-64) {};
%\node[box=0] (p-64-35) at (35,-64) {};
%\node[box=0] (p-64-36) at (36,-64) {};
%\node[box=0] (p-64-37) at (37,-64) {};
%\node[box=0] (p-64-38) at (38,-64) {};
%\node[box=0] (p-64-39) at (39,-64) {};
%\node[box=0] (p-64-40) at (40,-64) {};
%\node[box=0] (p-64-41) at (41,-64) {};
%\node[box=0] (p-64-42) at (42,-64) {};
%\node[box=0] (p-64-43) at (43,-64) {};
%\node[box=0] (p-64-44) at (44,-64) {};
%\node[box=0] (p-64-45) at (45,-64) {};
%\node[box=0] (p-64-46) at (46,-64) {};
%\node[box=0] (p-64-47) at (47,-64) {};
%\node[box=0] (p-64-48) at (48,-64) {};
%\node[box=0] (p-64-49) at (49,-64) {};
%\node[box=0] (p-64-50) at (50,-64) {};
%\node[box=0] (p-64-51) at (51,-64) {};
%\node[box=0] (p-64-52) at (52,-64) {};
%\node[box=0] (p-64-53) at (53,-64) {};
%\node[box=0] (p-64-54) at (54,-64) {};
%\node[box=0] (p-64-55) at (55,-64) {};
%\node[box=0] (p-64-56) at (56,-64) {};
%\node[box=0] (p-64-57) at (57,-64) {};
%\node[box=0] (p-64-58) at (58,-64) {};
%\node[box=0] (p-64-59) at (59,-64) {};
%\node[box=0] (p-64-60) at (60,-64) {};
%\node[box=0] (p-64-61) at (61,-64) {};
%\node[box=0] (p-64-62) at (62,-64) {};
%\node[box=0] (p-64-63) at (63,-64) {};
%\node[box=1] (p-64-64) at (64,-64) {};
%\node[box=1] (p-65-0) at (0,-65) {};
%\node[box=1] (p-65-1) at (1,-65) {};
%\node[box=0] (p-65-2) at (2,-65) {};
%\node[box=0] (p-65-3) at (3,-65) {};
%\node[box=0] (p-65-4) at (4,-65) {};
%\node[box=0] (p-65-5) at (5,-65) {};
%\node[box=0] (p-65-6) at (6,-65) {};
%\node[box=0] (p-65-7) at (7,-65) {};
%\node[box=0] (p-65-8) at (8,-65) {};
%\node[box=0] (p-65-9) at (9,-65) {};
%\node[box=0] (p-65-10) at (10,-65) {};
%\node[box=0] (p-65-11) at (11,-65) {};
%\node[box=0] (p-65-12) at (12,-65) {};
%\node[box=0] (p-65-13) at (13,-65) {};
%\node[box=0] (p-65-14) at (14,-65) {};
%\node[box=0] (p-65-15) at (15,-65) {};
%\node[box=0] (p-65-16) at (16,-65) {};
%\node[box=0] (p-65-17) at (17,-65) {};
%\node[box=0] (p-65-18) at (18,-65) {};
%\node[box=0] (p-65-19) at (19,-65) {};
%\node[box=0] (p-65-20) at (20,-65) {};
%\node[box=0] (p-65-21) at (21,-65) {};
%\node[box=0] (p-65-22) at (22,-65) {};
%\node[box=0] (p-65-23) at (23,-65) {};
%\node[box=0] (p-65-24) at (24,-65) {};
%\node[box=0] (p-65-25) at (25,-65) {};
%\node[box=0] (p-65-26) at (26,-65) {};
%\node[box=0] (p-65-27) at (27,-65) {};
%\node[box=0] (p-65-28) at (28,-65) {};
%\node[box=0] (p-65-29) at (29,-65) {};
%\node[box=0] (p-65-30) at (30,-65) {};
%\node[box=0] (p-65-31) at (31,-65) {};
%\node[box=0] (p-65-32) at (32,-65) {};
%\node[box=0] (p-65-33) at (33,-65) {};
%\node[box=0] (p-65-34) at (34,-65) {};
%\node[box=0] (p-65-35) at (35,-65) {};
%\node[box=0] (p-65-36) at (36,-65) {};
%\node[box=0] (p-65-37) at (37,-65) {};
%\node[box=0] (p-65-38) at (38,-65) {};
%\node[box=0] (p-65-39) at (39,-65) {};
%\node[box=0] (p-65-40) at (40,-65) {};
%\node[box=0] (p-65-41) at (41,-65) {};
%\node[box=0] (p-65-42) at (42,-65) {};
%\node[box=0] (p-65-43) at (43,-65) {};
%\node[box=0] (p-65-44) at (44,-65) {};
%\node[box=0] (p-65-45) at (45,-65) {};
%\node[box=0] (p-65-46) at (46,-65) {};
%\node[box=0] (p-65-47) at (47,-65) {};
%\node[box=0] (p-65-48) at (48,-65) {};
%\node[box=0] (p-65-49) at (49,-65) {};
%\node[box=0] (p-65-50) at (50,-65) {};
%\node[box=0] (p-65-51) at (51,-65) {};
%\node[box=0] (p-65-52) at (52,-65) {};
%\node[box=0] (p-65-53) at (53,-65) {};
%\node[box=0] (p-65-54) at (54,-65) {};
%\node[box=0] (p-65-55) at (55,-65) {};
%\node[box=0] (p-65-56) at (56,-65) {};
%\node[box=0] (p-65-57) at (57,-65) {};
%\node[box=0] (p-65-58) at (58,-65) {};
%\node[box=0] (p-65-59) at (59,-65) {};
%\node[box=0] (p-65-60) at (60,-65) {};
%\node[box=0] (p-65-61) at (61,-65) {};
%\node[box=0] (p-65-62) at (62,-65) {};
%\node[box=0] (p-65-63) at (63,-65) {};
%\node[box=1] (p-65-64) at (64,-65) {};
%\node[box=1] (p-65-65) at (65,-65) {};
%\node[box=1] (p-66-0) at (0,-66) {};
%\node[box=0] (p-66-1) at (1,-66) {};
%\node[box=1] (p-66-2) at (2,-66) {};
%\node[box=0] (p-66-3) at (3,-66) {};
%\node[box=0] (p-66-4) at (4,-66) {};
%\node[box=0] (p-66-5) at (5,-66) {};
%\node[box=0] (p-66-6) at (6,-66) {};
%\node[box=0] (p-66-7) at (7,-66) {};
%\node[box=0] (p-66-8) at (8,-66) {};
%\node[box=0] (p-66-9) at (9,-66) {};
%\node[box=0] (p-66-10) at (10,-66) {};
%\node[box=0] (p-66-11) at (11,-66) {};
%\node[box=0] (p-66-12) at (12,-66) {};
%\node[box=0] (p-66-13) at (13,-66) {};
%\node[box=0] (p-66-14) at (14,-66) {};
%\node[box=0] (p-66-15) at (15,-66) {};
%\node[box=0] (p-66-16) at (16,-66) {};
%\node[box=0] (p-66-17) at (17,-66) {};
%\node[box=0] (p-66-18) at (18,-66) {};
%\node[box=0] (p-66-19) at (19,-66) {};
%\node[box=0] (p-66-20) at (20,-66) {};
%\node[box=0] (p-66-21) at (21,-66) {};
%\node[box=0] (p-66-22) at (22,-66) {};
%\node[box=0] (p-66-23) at (23,-66) {};
%\node[box=0] (p-66-24) at (24,-66) {};
%\node[box=0] (p-66-25) at (25,-66) {};
%\node[box=0] (p-66-26) at (26,-66) {};
%\node[box=0] (p-66-27) at (27,-66) {};
%\node[box=0] (p-66-28) at (28,-66) {};
%\node[box=0] (p-66-29) at (29,-66) {};
%\node[box=0] (p-66-30) at (30,-66) {};
%\node[box=0] (p-66-31) at (31,-66) {};
%\node[box=0] (p-66-32) at (32,-66) {};
%\node[box=0] (p-66-33) at (33,-66) {};
%\node[box=0] (p-66-34) at (34,-66) {};
%\node[box=0] (p-66-35) at (35,-66) {};
%\node[box=0] (p-66-36) at (36,-66) {};
%\node[box=0] (p-66-37) at (37,-66) {};
%\node[box=0] (p-66-38) at (38,-66) {};
%\node[box=0] (p-66-39) at (39,-66) {};
%\node[box=0] (p-66-40) at (40,-66) {};
%\node[box=0] (p-66-41) at (41,-66) {};
%\node[box=0] (p-66-42) at (42,-66) {};
%\node[box=0] (p-66-43) at (43,-66) {};
%\node[box=0] (p-66-44) at (44,-66) {};
%\node[box=0] (p-66-45) at (45,-66) {};
%\node[box=0] (p-66-46) at (46,-66) {};
%\node[box=0] (p-66-47) at (47,-66) {};
%\node[box=0] (p-66-48) at (48,-66) {};
%\node[box=0] (p-66-49) at (49,-66) {};
%\node[box=0] (p-66-50) at (50,-66) {};
%\node[box=0] (p-66-51) at (51,-66) {};
%\node[box=0] (p-66-52) at (52,-66) {};
%\node[box=0] (p-66-53) at (53,-66) {};
%\node[box=0] (p-66-54) at (54,-66) {};
%\node[box=0] (p-66-55) at (55,-66) {};
%\node[box=0] (p-66-56) at (56,-66) {};
%\node[box=0] (p-66-57) at (57,-66) {};
%\node[box=0] (p-66-58) at (58,-66) {};
%\node[box=0] (p-66-59) at (59,-66) {};
%\node[box=0] (p-66-60) at (60,-66) {};
%\node[box=0] (p-66-61) at (61,-66) {};
%\node[box=0] (p-66-62) at (62,-66) {};
%\node[box=0] (p-66-63) at (63,-66) {};
%\node[box=1] (p-66-64) at (64,-66) {};
%\node[box=0] (p-66-65) at (65,-66) {};
%\node[box=1] (p-66-66) at (66,-66) {};
%\node[box=1] (p-67-0) at (0,-67) {};
%\node[box=1] (p-67-1) at (1,-67) {};
%\node[box=1] (p-67-2) at (2,-67) {};
%\node[box=1] (p-67-3) at (3,-67) {};
%\node[box=0] (p-67-4) at (4,-67) {};
%\node[box=0] (p-67-5) at (5,-67) {};
%\node[box=0] (p-67-6) at (6,-67) {};
%\node[box=0] (p-67-7) at (7,-67) {};
%\node[box=0] (p-67-8) at (8,-67) {};
%\node[box=0] (p-67-9) at (9,-67) {};
%\node[box=0] (p-67-10) at (10,-67) {};
%\node[box=0] (p-67-11) at (11,-67) {};
%\node[box=0] (p-67-12) at (12,-67) {};
%\node[box=0] (p-67-13) at (13,-67) {};
%\node[box=0] (p-67-14) at (14,-67) {};
%\node[box=0] (p-67-15) at (15,-67) {};
%\node[box=0] (p-67-16) at (16,-67) {};
%\node[box=0] (p-67-17) at (17,-67) {};
%\node[box=0] (p-67-18) at (18,-67) {};
%\node[box=0] (p-67-19) at (19,-67) {};
%\node[box=0] (p-67-20) at (20,-67) {};
%\node[box=0] (p-67-21) at (21,-67) {};
%\node[box=0] (p-67-22) at (22,-67) {};
%\node[box=0] (p-67-23) at (23,-67) {};
%\node[box=0] (p-67-24) at (24,-67) {};
%\node[box=0] (p-67-25) at (25,-67) {};
%\node[box=0] (p-67-26) at (26,-67) {};
%\node[box=0] (p-67-27) at (27,-67) {};
%\node[box=0] (p-67-28) at (28,-67) {};
%\node[box=0] (p-67-29) at (29,-67) {};
%\node[box=0] (p-67-30) at (30,-67) {};
%\node[box=0] (p-67-31) at (31,-67) {};
%\node[box=0] (p-67-32) at (32,-67) {};
%\node[box=0] (p-67-33) at (33,-67) {};
%\node[box=0] (p-67-34) at (34,-67) {};
%\node[box=0] (p-67-35) at (35,-67) {};
%\node[box=0] (p-67-36) at (36,-67) {};
%\node[box=0] (p-67-37) at (37,-67) {};
%\node[box=0] (p-67-38) at (38,-67) {};
%\node[box=0] (p-67-39) at (39,-67) {};
%\node[box=0] (p-67-40) at (40,-67) {};
%\node[box=0] (p-67-41) at (41,-67) {};
%\node[box=0] (p-67-42) at (42,-67) {};
%\node[box=0] (p-67-43) at (43,-67) {};
%\node[box=0] (p-67-44) at (44,-67) {};
%\node[box=0] (p-67-45) at (45,-67) {};
%\node[box=0] (p-67-46) at (46,-67) {};
%\node[box=0] (p-67-47) at (47,-67) {};
%\node[box=0] (p-67-48) at (48,-67) {};
%\node[box=0] (p-67-49) at (49,-67) {};
%\node[box=0] (p-67-50) at (50,-67) {};
%\node[box=0] (p-67-51) at (51,-67) {};
%\node[box=0] (p-67-52) at (52,-67) {};
%\node[box=0] (p-67-53) at (53,-67) {};
%\node[box=0] (p-67-54) at (54,-67) {};
%\node[box=0] (p-67-55) at (55,-67) {};
%\node[box=0] (p-67-56) at (56,-67) {};
%\node[box=0] (p-67-57) at (57,-67) {};
%\node[box=0] (p-67-58) at (58,-67) {};
%\node[box=0] (p-67-59) at (59,-67) {};
%\node[box=0] (p-67-60) at (60,-67) {};
%\node[box=0] (p-67-61) at (61,-67) {};
%\node[box=0] (p-67-62) at (62,-67) {};
%\node[box=0] (p-67-63) at (63,-67) {};
%\node[box=1] (p-67-64) at (64,-67) {};
%\node[box=1] (p-67-65) at (65,-67) {};
%\node[box=1] (p-67-66) at (66,-67) {};
%\node[box=1] (p-67-67) at (67,-67) {};
%\node[box=1] (p-68-0) at (0,-68) {};
%\node[box=0] (p-68-1) at (1,-68) {};
%\node[box=0] (p-68-2) at (2,-68) {};
%\node[box=0] (p-68-3) at (3,-68) {};
%\node[box=1] (p-68-4) at (4,-68) {};
%\node[box=0] (p-68-5) at (5,-68) {};
%\node[box=0] (p-68-6) at (6,-68) {};
%\node[box=0] (p-68-7) at (7,-68) {};
%\node[box=0] (p-68-8) at (8,-68) {};
%\node[box=0] (p-68-9) at (9,-68) {};
%\node[box=0] (p-68-10) at (10,-68) {};
%\node[box=0] (p-68-11) at (11,-68) {};
%\node[box=0] (p-68-12) at (12,-68) {};
%\node[box=0] (p-68-13) at (13,-68) {};
%\node[box=0] (p-68-14) at (14,-68) {};
%\node[box=0] (p-68-15) at (15,-68) {};
%\node[box=0] (p-68-16) at (16,-68) {};
%\node[box=0] (p-68-17) at (17,-68) {};
%\node[box=0] (p-68-18) at (18,-68) {};
%\node[box=0] (p-68-19) at (19,-68) {};
%\node[box=0] (p-68-20) at (20,-68) {};
%\node[box=0] (p-68-21) at (21,-68) {};
%\node[box=0] (p-68-22) at (22,-68) {};
%\node[box=0] (p-68-23) at (23,-68) {};
%\node[box=0] (p-68-24) at (24,-68) {};
%\node[box=0] (p-68-25) at (25,-68) {};
%\node[box=0] (p-68-26) at (26,-68) {};
%\node[box=0] (p-68-27) at (27,-68) {};
%\node[box=0] (p-68-28) at (28,-68) {};
%\node[box=0] (p-68-29) at (29,-68) {};
%\node[box=0] (p-68-30) at (30,-68) {};
%\node[box=0] (p-68-31) at (31,-68) {};
%\node[box=0] (p-68-32) at (32,-68) {};
%\node[box=0] (p-68-33) at (33,-68) {};
%\node[box=0] (p-68-34) at (34,-68) {};
%\node[box=0] (p-68-35) at (35,-68) {};
%\node[box=0] (p-68-36) at (36,-68) {};
%\node[box=0] (p-68-37) at (37,-68) {};
%\node[box=0] (p-68-38) at (38,-68) {};
%\node[box=0] (p-68-39) at (39,-68) {};
%\node[box=0] (p-68-40) at (40,-68) {};
%\node[box=0] (p-68-41) at (41,-68) {};
%\node[box=0] (p-68-42) at (42,-68) {};
%\node[box=0] (p-68-43) at (43,-68) {};
%\node[box=0] (p-68-44) at (44,-68) {};
%\node[box=0] (p-68-45) at (45,-68) {};
%\node[box=0] (p-68-46) at (46,-68) {};
%\node[box=0] (p-68-47) at (47,-68) {};
%\node[box=0] (p-68-48) at (48,-68) {};
%\node[box=0] (p-68-49) at (49,-68) {};
%\node[box=0] (p-68-50) at (50,-68) {};
%\node[box=0] (p-68-51) at (51,-68) {};
%\node[box=0] (p-68-52) at (52,-68) {};
%\node[box=0] (p-68-53) at (53,-68) {};
%\node[box=0] (p-68-54) at (54,-68) {};
%\node[box=0] (p-68-55) at (55,-68) {};
%\node[box=0] (p-68-56) at (56,-68) {};
%\node[box=0] (p-68-57) at (57,-68) {};
%\node[box=0] (p-68-58) at (58,-68) {};
%\node[box=0] (p-68-59) at (59,-68) {};
%\node[box=0] (p-68-60) at (60,-68) {};
%\node[box=0] (p-68-61) at (61,-68) {};
%\node[box=0] (p-68-62) at (62,-68) {};
%\node[box=0] (p-68-63) at (63,-68) {};
%\node[box=1] (p-68-64) at (64,-68) {};
%\node[box=0] (p-68-65) at (65,-68) {};
%\node[box=0] (p-68-66) at (66,-68) {};
%\node[box=0] (p-68-67) at (67,-68) {};
%\node[box=1] (p-68-68) at (68,-68) {};
%\node[box=1] (p-69-0) at (0,-69) {};
%\node[box=1] (p-69-1) at (1,-69) {};
%\node[box=0] (p-69-2) at (2,-69) {};
%\node[box=0] (p-69-3) at (3,-69) {};
%\node[box=1] (p-69-4) at (4,-69) {};
%\node[box=1] (p-69-5) at (5,-69) {};
%\node[box=0] (p-69-6) at (6,-69) {};
%\node[box=0] (p-69-7) at (7,-69) {};
%\node[box=0] (p-69-8) at (8,-69) {};
%\node[box=0] (p-69-9) at (9,-69) {};
%\node[box=0] (p-69-10) at (10,-69) {};
%\node[box=0] (p-69-11) at (11,-69) {};
%\node[box=0] (p-69-12) at (12,-69) {};
%\node[box=0] (p-69-13) at (13,-69) {};
%\node[box=0] (p-69-14) at (14,-69) {};
%\node[box=0] (p-69-15) at (15,-69) {};
%\node[box=0] (p-69-16) at (16,-69) {};
%\node[box=0] (p-69-17) at (17,-69) {};
%\node[box=0] (p-69-18) at (18,-69) {};
%\node[box=0] (p-69-19) at (19,-69) {};
%\node[box=0] (p-69-20) at (20,-69) {};
%\node[box=0] (p-69-21) at (21,-69) {};
%\node[box=0] (p-69-22) at (22,-69) {};
%\node[box=0] (p-69-23) at (23,-69) {};
%\node[box=0] (p-69-24) at (24,-69) {};
%\node[box=0] (p-69-25) at (25,-69) {};
%\node[box=0] (p-69-26) at (26,-69) {};
%\node[box=0] (p-69-27) at (27,-69) {};
%\node[box=0] (p-69-28) at (28,-69) {};
%\node[box=0] (p-69-29) at (29,-69) {};
%\node[box=0] (p-69-30) at (30,-69) {};
%\node[box=0] (p-69-31) at (31,-69) {};
%\node[box=0] (p-69-32) at (32,-69) {};
%\node[box=0] (p-69-33) at (33,-69) {};
%\node[box=0] (p-69-34) at (34,-69) {};
%\node[box=0] (p-69-35) at (35,-69) {};
%\node[box=0] (p-69-36) at (36,-69) {};
%\node[box=0] (p-69-37) at (37,-69) {};
%\node[box=0] (p-69-38) at (38,-69) {};
%\node[box=0] (p-69-39) at (39,-69) {};
%\node[box=0] (p-69-40) at (40,-69) {};
%\node[box=0] (p-69-41) at (41,-69) {};
%\node[box=0] (p-69-42) at (42,-69) {};
%\node[box=0] (p-69-43) at (43,-69) {};
%\node[box=0] (p-69-44) at (44,-69) {};
%\node[box=0] (p-69-45) at (45,-69) {};
%\node[box=0] (p-69-46) at (46,-69) {};
%\node[box=0] (p-69-47) at (47,-69) {};
%\node[box=0] (p-69-48) at (48,-69) {};
%\node[box=0] (p-69-49) at (49,-69) {};
%\node[box=0] (p-69-50) at (50,-69) {};
%\node[box=0] (p-69-51) at (51,-69) {};
%\node[box=0] (p-69-52) at (52,-69) {};
%\node[box=0] (p-69-53) at (53,-69) {};
%\node[box=0] (p-69-54) at (54,-69) {};
%\node[box=0] (p-69-55) at (55,-69) {};
%\node[box=0] (p-69-56) at (56,-69) {};
%\node[box=0] (p-69-57) at (57,-69) {};
%\node[box=0] (p-69-58) at (58,-69) {};
%\node[box=0] (p-69-59) at (59,-69) {};
%\node[box=0] (p-69-60) at (60,-69) {};
%\node[box=0] (p-69-61) at (61,-69) {};
%\node[box=0] (p-69-62) at (62,-69) {};
%\node[box=0] (p-69-63) at (63,-69) {};
%\node[box=1] (p-69-64) at (64,-69) {};
%\node[box=1] (p-69-65) at (65,-69) {};
%\node[box=0] (p-69-66) at (66,-69) {};
%\node[box=0] (p-69-67) at (67,-69) {};
%\node[box=1] (p-69-68) at (68,-69) {};
%\node[box=1] (p-69-69) at (69,-69) {};
%\node[box=1] (p-70-0) at (0,-70) {};
%\node[box=0] (p-70-1) at (1,-70) {};
%\node[box=1] (p-70-2) at (2,-70) {};
%\node[box=0] (p-70-3) at (3,-70) {};
%\node[box=1] (p-70-4) at (4,-70) {};
%\node[box=0] (p-70-5) at (5,-70) {};
%\node[box=1] (p-70-6) at (6,-70) {};
%\node[box=0] (p-70-7) at (7,-70) {};
%\node[box=0] (p-70-8) at (8,-70) {};
%\node[box=0] (p-70-9) at (9,-70) {};
%\node[box=0] (p-70-10) at (10,-70) {};
%\node[box=0] (p-70-11) at (11,-70) {};
%\node[box=0] (p-70-12) at (12,-70) {};
%\node[box=0] (p-70-13) at (13,-70) {};
%\node[box=0] (p-70-14) at (14,-70) {};
%\node[box=0] (p-70-15) at (15,-70) {};
%\node[box=0] (p-70-16) at (16,-70) {};
%\node[box=0] (p-70-17) at (17,-70) {};
%\node[box=0] (p-70-18) at (18,-70) {};
%\node[box=0] (p-70-19) at (19,-70) {};
%\node[box=0] (p-70-20) at (20,-70) {};
%\node[box=0] (p-70-21) at (21,-70) {};
%\node[box=0] (p-70-22) at (22,-70) {};
%\node[box=0] (p-70-23) at (23,-70) {};
%\node[box=0] (p-70-24) at (24,-70) {};
%\node[box=0] (p-70-25) at (25,-70) {};
%\node[box=0] (p-70-26) at (26,-70) {};
%\node[box=0] (p-70-27) at (27,-70) {};
%\node[box=0] (p-70-28) at (28,-70) {};
%\node[box=0] (p-70-29) at (29,-70) {};
%\node[box=0] (p-70-30) at (30,-70) {};
%\node[box=0] (p-70-31) at (31,-70) {};
%\node[box=0] (p-70-32) at (32,-70) {};
%\node[box=0] (p-70-33) at (33,-70) {};
%\node[box=0] (p-70-34) at (34,-70) {};
%\node[box=0] (p-70-35) at (35,-70) {};
%\node[box=0] (p-70-36) at (36,-70) {};
%\node[box=0] (p-70-37) at (37,-70) {};
%\node[box=0] (p-70-38) at (38,-70) {};
%\node[box=0] (p-70-39) at (39,-70) {};
%\node[box=0] (p-70-40) at (40,-70) {};
%\node[box=0] (p-70-41) at (41,-70) {};
%\node[box=0] (p-70-42) at (42,-70) {};
%\node[box=0] (p-70-43) at (43,-70) {};
%\node[box=0] (p-70-44) at (44,-70) {};
%\node[box=0] (p-70-45) at (45,-70) {};
%\node[box=0] (p-70-46) at (46,-70) {};
%\node[box=0] (p-70-47) at (47,-70) {};
%\node[box=0] (p-70-48) at (48,-70) {};
%\node[box=0] (p-70-49) at (49,-70) {};
%\node[box=0] (p-70-50) at (50,-70) {};
%\node[box=0] (p-70-51) at (51,-70) {};
%\node[box=0] (p-70-52) at (52,-70) {};
%\node[box=0] (p-70-53) at (53,-70) {};
%\node[box=0] (p-70-54) at (54,-70) {};
%\node[box=0] (p-70-55) at (55,-70) {};
%\node[box=0] (p-70-56) at (56,-70) {};
%\node[box=0] (p-70-57) at (57,-70) {};
%\node[box=0] (p-70-58) at (58,-70) {};
%\node[box=0] (p-70-59) at (59,-70) {};
%\node[box=0] (p-70-60) at (60,-70) {};
%\node[box=0] (p-70-61) at (61,-70) {};
%\node[box=0] (p-70-62) at (62,-70) {};
%\node[box=0] (p-70-63) at (63,-70) {};
%\node[box=1] (p-70-64) at (64,-70) {};
%\node[box=0] (p-70-65) at (65,-70) {};
%\node[box=1] (p-70-66) at (66,-70) {};
%\node[box=0] (p-70-67) at (67,-70) {};
%\node[box=1] (p-70-68) at (68,-70) {};
%\node[box=0] (p-70-69) at (69,-70) {};
%\node[box=1] (p-70-70) at (70,-70) {};
%\node[box=1] (p-71-0) at (0,-71) {};
%\node[box=1] (p-71-1) at (1,-71) {};
%\node[box=1] (p-71-2) at (2,-71) {};
%\node[box=1] (p-71-3) at (3,-71) {};
%\node[box=1] (p-71-4) at (4,-71) {};
%\node[box=1] (p-71-5) at (5,-71) {};
%\node[box=1] (p-71-6) at (6,-71) {};
%\node[box=1] (p-71-7) at (7,-71) {};
%\node[box=0] (p-71-8) at (8,-71) {};
%\node[box=0] (p-71-9) at (9,-71) {};
%\node[box=0] (p-71-10) at (10,-71) {};
%\node[box=0] (p-71-11) at (11,-71) {};
%\node[box=0] (p-71-12) at (12,-71) {};
%\node[box=0] (p-71-13) at (13,-71) {};
%\node[box=0] (p-71-14) at (14,-71) {};
%\node[box=0] (p-71-15) at (15,-71) {};
%\node[box=0] (p-71-16) at (16,-71) {};
%\node[box=0] (p-71-17) at (17,-71) {};
%\node[box=0] (p-71-18) at (18,-71) {};
%\node[box=0] (p-71-19) at (19,-71) {};
%\node[box=0] (p-71-20) at (20,-71) {};
%\node[box=0] (p-71-21) at (21,-71) {};
%\node[box=0] (p-71-22) at (22,-71) {};
%\node[box=0] (p-71-23) at (23,-71) {};
%\node[box=0] (p-71-24) at (24,-71) {};
%\node[box=0] (p-71-25) at (25,-71) {};
%\node[box=0] (p-71-26) at (26,-71) {};
%\node[box=0] (p-71-27) at (27,-71) {};
%\node[box=0] (p-71-28) at (28,-71) {};
%\node[box=0] (p-71-29) at (29,-71) {};
%\node[box=0] (p-71-30) at (30,-71) {};
%\node[box=0] (p-71-31) at (31,-71) {};
%\node[box=0] (p-71-32) at (32,-71) {};
%\node[box=0] (p-71-33) at (33,-71) {};
%\node[box=0] (p-71-34) at (34,-71) {};
%\node[box=0] (p-71-35) at (35,-71) {};
%\node[box=0] (p-71-36) at (36,-71) {};
%\node[box=0] (p-71-37) at (37,-71) {};
%\node[box=0] (p-71-38) at (38,-71) {};
%\node[box=0] (p-71-39) at (39,-71) {};
%\node[box=0] (p-71-40) at (40,-71) {};
%\node[box=0] (p-71-41) at (41,-71) {};
%\node[box=0] (p-71-42) at (42,-71) {};
%\node[box=0] (p-71-43) at (43,-71) {};
%\node[box=0] (p-71-44) at (44,-71) {};
%\node[box=0] (p-71-45) at (45,-71) {};
%\node[box=0] (p-71-46) at (46,-71) {};
%\node[box=0] (p-71-47) at (47,-71) {};
%\node[box=0] (p-71-48) at (48,-71) {};
%\node[box=0] (p-71-49) at (49,-71) {};
%\node[box=0] (p-71-50) at (50,-71) {};
%\node[box=0] (p-71-51) at (51,-71) {};
%\node[box=0] (p-71-52) at (52,-71) {};
%\node[box=0] (p-71-53) at (53,-71) {};
%\node[box=0] (p-71-54) at (54,-71) {};
%\node[box=0] (p-71-55) at (55,-71) {};
%\node[box=0] (p-71-56) at (56,-71) {};
%\node[box=0] (p-71-57) at (57,-71) {};
%\node[box=0] (p-71-58) at (58,-71) {};
%\node[box=0] (p-71-59) at (59,-71) {};
%\node[box=0] (p-71-60) at (60,-71) {};
%\node[box=0] (p-71-61) at (61,-71) {};
%\node[box=0] (p-71-62) at (62,-71) {};
%\node[box=0] (p-71-63) at (63,-71) {};
%\node[box=1] (p-71-64) at (64,-71) {};
%\node[box=1] (p-71-65) at (65,-71) {};
%\node[box=1] (p-71-66) at (66,-71) {};
%\node[box=1] (p-71-67) at (67,-71) {};
%\node[box=1] (p-71-68) at (68,-71) {};
%\node[box=1] (p-71-69) at (69,-71) {};
%\node[box=1] (p-71-70) at (70,-71) {};
%\node[box=1] (p-71-71) at (71,-71) {};
%\node[box=1] (p-72-0) at (0,-72) {};
%\node[box=0] (p-72-1) at (1,-72) {};
%\node[box=0] (p-72-2) at (2,-72) {};
%\node[box=0] (p-72-3) at (3,-72) {};
%\node[box=0] (p-72-4) at (4,-72) {};
%\node[box=0] (p-72-5) at (5,-72) {};
%\node[box=0] (p-72-6) at (6,-72) {};
%\node[box=0] (p-72-7) at (7,-72) {};
%\node[box=1] (p-72-8) at (8,-72) {};
%\node[box=0] (p-72-9) at (9,-72) {};
%\node[box=0] (p-72-10) at (10,-72) {};
%\node[box=0] (p-72-11) at (11,-72) {};
%\node[box=0] (p-72-12) at (12,-72) {};
%\node[box=0] (p-72-13) at (13,-72) {};
%\node[box=0] (p-72-14) at (14,-72) {};
%\node[box=0] (p-72-15) at (15,-72) {};
%\node[box=0] (p-72-16) at (16,-72) {};
%\node[box=0] (p-72-17) at (17,-72) {};
%\node[box=0] (p-72-18) at (18,-72) {};
%\node[box=0] (p-72-19) at (19,-72) {};
%\node[box=0] (p-72-20) at (20,-72) {};
%\node[box=0] (p-72-21) at (21,-72) {};
%\node[box=0] (p-72-22) at (22,-72) {};
%\node[box=0] (p-72-23) at (23,-72) {};
%\node[box=0] (p-72-24) at (24,-72) {};
%\node[box=0] (p-72-25) at (25,-72) {};
%\node[box=0] (p-72-26) at (26,-72) {};
%\node[box=0] (p-72-27) at (27,-72) {};
%\node[box=0] (p-72-28) at (28,-72) {};
%\node[box=0] (p-72-29) at (29,-72) {};
%\node[box=0] (p-72-30) at (30,-72) {};
%\node[box=0] (p-72-31) at (31,-72) {};
%\node[box=0] (p-72-32) at (32,-72) {};
%\node[box=0] (p-72-33) at (33,-72) {};
%\node[box=0] (p-72-34) at (34,-72) {};
%\node[box=0] (p-72-35) at (35,-72) {};
%\node[box=0] (p-72-36) at (36,-72) {};
%\node[box=0] (p-72-37) at (37,-72) {};
%\node[box=0] (p-72-38) at (38,-72) {};
%\node[box=0] (p-72-39) at (39,-72) {};
%\node[box=0] (p-72-40) at (40,-72) {};
%\node[box=0] (p-72-41) at (41,-72) {};
%\node[box=0] (p-72-42) at (42,-72) {};
%\node[box=0] (p-72-43) at (43,-72) {};
%\node[box=0] (p-72-44) at (44,-72) {};
%\node[box=0] (p-72-45) at (45,-72) {};
%\node[box=0] (p-72-46) at (46,-72) {};
%\node[box=0] (p-72-47) at (47,-72) {};
%\node[box=0] (p-72-48) at (48,-72) {};
%\node[box=0] (p-72-49) at (49,-72) {};
%\node[box=0] (p-72-50) at (50,-72) {};
%\node[box=0] (p-72-51) at (51,-72) {};
%\node[box=0] (p-72-52) at (52,-72) {};
%\node[box=0] (p-72-53) at (53,-72) {};
%\node[box=0] (p-72-54) at (54,-72) {};
%\node[box=0] (p-72-55) at (55,-72) {};
%\node[box=0] (p-72-56) at (56,-72) {};
%\node[box=0] (p-72-57) at (57,-72) {};
%\node[box=0] (p-72-58) at (58,-72) {};
%\node[box=0] (p-72-59) at (59,-72) {};
%\node[box=0] (p-72-60) at (60,-72) {};
%\node[box=0] (p-72-61) at (61,-72) {};
%\node[box=0] (p-72-62) at (62,-72) {};
%\node[box=0] (p-72-63) at (63,-72) {};
%\node[box=1] (p-72-64) at (64,-72) {};
%\node[box=0] (p-72-65) at (65,-72) {};
%\node[box=0] (p-72-66) at (66,-72) {};
%\node[box=0] (p-72-67) at (67,-72) {};
%\node[box=0] (p-72-68) at (68,-72) {};
%\node[box=0] (p-72-69) at (69,-72) {};
%\node[box=0] (p-72-70) at (70,-72) {};
%\node[box=0] (p-72-71) at (71,-72) {};
%\node[box=1] (p-72-72) at (72,-72) {};
%\node[box=1] (p-73-0) at (0,-73) {};
%\node[box=1] (p-73-1) at (1,-73) {};
%\node[box=0] (p-73-2) at (2,-73) {};
%\node[box=0] (p-73-3) at (3,-73) {};
%\node[box=0] (p-73-4) at (4,-73) {};
%\node[box=0] (p-73-5) at (5,-73) {};
%\node[box=0] (p-73-6) at (6,-73) {};
%\node[box=0] (p-73-7) at (7,-73) {};
%\node[box=1] (p-73-8) at (8,-73) {};
%\node[box=1] (p-73-9) at (9,-73) {};
%\node[box=0] (p-73-10) at (10,-73) {};
%\node[box=0] (p-73-11) at (11,-73) {};
%\node[box=0] (p-73-12) at (12,-73) {};
%\node[box=0] (p-73-13) at (13,-73) {};
%\node[box=0] (p-73-14) at (14,-73) {};
%\node[box=0] (p-73-15) at (15,-73) {};
%\node[box=0] (p-73-16) at (16,-73) {};
%\node[box=0] (p-73-17) at (17,-73) {};
%\node[box=0] (p-73-18) at (18,-73) {};
%\node[box=0] (p-73-19) at (19,-73) {};
%\node[box=0] (p-73-20) at (20,-73) {};
%\node[box=0] (p-73-21) at (21,-73) {};
%\node[box=0] (p-73-22) at (22,-73) {};
%\node[box=0] (p-73-23) at (23,-73) {};
%\node[box=0] (p-73-24) at (24,-73) {};
%\node[box=0] (p-73-25) at (25,-73) {};
%\node[box=0] (p-73-26) at (26,-73) {};
%\node[box=0] (p-73-27) at (27,-73) {};
%\node[box=0] (p-73-28) at (28,-73) {};
%\node[box=0] (p-73-29) at (29,-73) {};
%\node[box=0] (p-73-30) at (30,-73) {};
%\node[box=0] (p-73-31) at (31,-73) {};
%\node[box=0] (p-73-32) at (32,-73) {};
%\node[box=0] (p-73-33) at (33,-73) {};
%\node[box=0] (p-73-34) at (34,-73) {};
%\node[box=0] (p-73-35) at (35,-73) {};
%\node[box=0] (p-73-36) at (36,-73) {};
%\node[box=0] (p-73-37) at (37,-73) {};
%\node[box=0] (p-73-38) at (38,-73) {};
%\node[box=0] (p-73-39) at (39,-73) {};
%\node[box=0] (p-73-40) at (40,-73) {};
%\node[box=0] (p-73-41) at (41,-73) {};
%\node[box=0] (p-73-42) at (42,-73) {};
%\node[box=0] (p-73-43) at (43,-73) {};
%\node[box=0] (p-73-44) at (44,-73) {};
%\node[box=0] (p-73-45) at (45,-73) {};
%\node[box=0] (p-73-46) at (46,-73) {};
%\node[box=0] (p-73-47) at (47,-73) {};
%\node[box=0] (p-73-48) at (48,-73) {};
%\node[box=0] (p-73-49) at (49,-73) {};
%\node[box=0] (p-73-50) at (50,-73) {};
%\node[box=0] (p-73-51) at (51,-73) {};
%\node[box=0] (p-73-52) at (52,-73) {};
%\node[box=0] (p-73-53) at (53,-73) {};
%\node[box=0] (p-73-54) at (54,-73) {};
%\node[box=0] (p-73-55) at (55,-73) {};
%\node[box=0] (p-73-56) at (56,-73) {};
%\node[box=0] (p-73-57) at (57,-73) {};
%\node[box=0] (p-73-58) at (58,-73) {};
%\node[box=0] (p-73-59) at (59,-73) {};
%\node[box=0] (p-73-60) at (60,-73) {};
%\node[box=0] (p-73-61) at (61,-73) {};
%\node[box=0] (p-73-62) at (62,-73) {};
%\node[box=0] (p-73-63) at (63,-73) {};
%\node[box=1] (p-73-64) at (64,-73) {};
%\node[box=1] (p-73-65) at (65,-73) {};
%\node[box=0] (p-73-66) at (66,-73) {};
%\node[box=0] (p-73-67) at (67,-73) {};
%\node[box=0] (p-73-68) at (68,-73) {};
%\node[box=0] (p-73-69) at (69,-73) {};
%\node[box=0] (p-73-70) at (70,-73) {};
%\node[box=0] (p-73-71) at (71,-73) {};
%\node[box=1] (p-73-72) at (72,-73) {};
%\node[box=1] (p-73-73) at (73,-73) {};
%\node[box=1] (p-74-0) at (0,-74) {};
%\node[box=0] (p-74-1) at (1,-74) {};
%\node[box=1] (p-74-2) at (2,-74) {};
%\node[box=0] (p-74-3) at (3,-74) {};
%\node[box=0] (p-74-4) at (4,-74) {};
%\node[box=0] (p-74-5) at (5,-74) {};
%\node[box=0] (p-74-6) at (6,-74) {};
%\node[box=0] (p-74-7) at (7,-74) {};
%\node[box=1] (p-74-8) at (8,-74) {};
%\node[box=0] (p-74-9) at (9,-74) {};
%\node[box=1] (p-74-10) at (10,-74) {};
%\node[box=0] (p-74-11) at (11,-74) {};
%\node[box=0] (p-74-12) at (12,-74) {};
%\node[box=0] (p-74-13) at (13,-74) {};
%\node[box=0] (p-74-14) at (14,-74) {};
%\node[box=0] (p-74-15) at (15,-74) {};
%\node[box=0] (p-74-16) at (16,-74) {};
%\node[box=0] (p-74-17) at (17,-74) {};
%\node[box=0] (p-74-18) at (18,-74) {};
%\node[box=0] (p-74-19) at (19,-74) {};
%\node[box=0] (p-74-20) at (20,-74) {};
%\node[box=0] (p-74-21) at (21,-74) {};
%\node[box=0] (p-74-22) at (22,-74) {};
%\node[box=0] (p-74-23) at (23,-74) {};
%\node[box=0] (p-74-24) at (24,-74) {};
%\node[box=0] (p-74-25) at (25,-74) {};
%\node[box=0] (p-74-26) at (26,-74) {};
%\node[box=0] (p-74-27) at (27,-74) {};
%\node[box=0] (p-74-28) at (28,-74) {};
%\node[box=0] (p-74-29) at (29,-74) {};
%\node[box=0] (p-74-30) at (30,-74) {};
%\node[box=0] (p-74-31) at (31,-74) {};
%\node[box=0] (p-74-32) at (32,-74) {};
%\node[box=0] (p-74-33) at (33,-74) {};
%\node[box=0] (p-74-34) at (34,-74) {};
%\node[box=0] (p-74-35) at (35,-74) {};
%\node[box=0] (p-74-36) at (36,-74) {};
%\node[box=0] (p-74-37) at (37,-74) {};
%\node[box=0] (p-74-38) at (38,-74) {};
%\node[box=0] (p-74-39) at (39,-74) {};
%\node[box=0] (p-74-40) at (40,-74) {};
%\node[box=0] (p-74-41) at (41,-74) {};
%\node[box=0] (p-74-42) at (42,-74) {};
%\node[box=0] (p-74-43) at (43,-74) {};
%\node[box=0] (p-74-44) at (44,-74) {};
%\node[box=0] (p-74-45) at (45,-74) {};
%\node[box=0] (p-74-46) at (46,-74) {};
%\node[box=0] (p-74-47) at (47,-74) {};
%\node[box=0] (p-74-48) at (48,-74) {};
%\node[box=0] (p-74-49) at (49,-74) {};
%\node[box=0] (p-74-50) at (50,-74) {};
%\node[box=0] (p-74-51) at (51,-74) {};
%\node[box=0] (p-74-52) at (52,-74) {};
%\node[box=0] (p-74-53) at (53,-74) {};
%\node[box=0] (p-74-54) at (54,-74) {};
%\node[box=0] (p-74-55) at (55,-74) {};
%\node[box=0] (p-74-56) at (56,-74) {};
%\node[box=0] (p-74-57) at (57,-74) {};
%\node[box=0] (p-74-58) at (58,-74) {};
%\node[box=0] (p-74-59) at (59,-74) {};
%\node[box=0] (p-74-60) at (60,-74) {};
%\node[box=0] (p-74-61) at (61,-74) {};
%\node[box=0] (p-74-62) at (62,-74) {};
%\node[box=0] (p-74-63) at (63,-74) {};
%\node[box=1] (p-74-64) at (64,-74) {};
%\node[box=0] (p-74-65) at (65,-74) {};
%\node[box=1] (p-74-66) at (66,-74) {};
%\node[box=0] (p-74-67) at (67,-74) {};
%\node[box=0] (p-74-68) at (68,-74) {};
%\node[box=0] (p-74-69) at (69,-74) {};
%\node[box=0] (p-74-70) at (70,-74) {};
%\node[box=0] (p-74-71) at (71,-74) {};
%\node[box=1] (p-74-72) at (72,-74) {};
%\node[box=0] (p-74-73) at (73,-74) {};
%\node[box=1] (p-74-74) at (74,-74) {};
%\node[box=1] (p-75-0) at (0,-75) {};
%\node[box=1] (p-75-1) at (1,-75) {};
%\node[box=1] (p-75-2) at (2,-75) {};
%\node[box=1] (p-75-3) at (3,-75) {};
%\node[box=0] (p-75-4) at (4,-75) {};
%\node[box=0] (p-75-5) at (5,-75) {};
%\node[box=0] (p-75-6) at (6,-75) {};
%\node[box=0] (p-75-7) at (7,-75) {};
%\node[box=1] (p-75-8) at (8,-75) {};
%\node[box=1] (p-75-9) at (9,-75) {};
%\node[box=1] (p-75-10) at (10,-75) {};
%\node[box=1] (p-75-11) at (11,-75) {};
%\node[box=0] (p-75-12) at (12,-75) {};
%\node[box=0] (p-75-13) at (13,-75) {};
%\node[box=0] (p-75-14) at (14,-75) {};
%\node[box=0] (p-75-15) at (15,-75) {};
%\node[box=0] (p-75-16) at (16,-75) {};
%\node[box=0] (p-75-17) at (17,-75) {};
%\node[box=0] (p-75-18) at (18,-75) {};
%\node[box=0] (p-75-19) at (19,-75) {};
%\node[box=0] (p-75-20) at (20,-75) {};
%\node[box=0] (p-75-21) at (21,-75) {};
%\node[box=0] (p-75-22) at (22,-75) {};
%\node[box=0] (p-75-23) at (23,-75) {};
%\node[box=0] (p-75-24) at (24,-75) {};
%\node[box=0] (p-75-25) at (25,-75) {};
%\node[box=0] (p-75-26) at (26,-75) {};
%\node[box=0] (p-75-27) at (27,-75) {};
%\node[box=0] (p-75-28) at (28,-75) {};
%\node[box=0] (p-75-29) at (29,-75) {};
%\node[box=0] (p-75-30) at (30,-75) {};
%\node[box=0] (p-75-31) at (31,-75) {};
%\node[box=0] (p-75-32) at (32,-75) {};
%\node[box=0] (p-75-33) at (33,-75) {};
%\node[box=0] (p-75-34) at (34,-75) {};
%\node[box=0] (p-75-35) at (35,-75) {};
%\node[box=0] (p-75-36) at (36,-75) {};
%\node[box=0] (p-75-37) at (37,-75) {};
%\node[box=0] (p-75-38) at (38,-75) {};
%\node[box=0] (p-75-39) at (39,-75) {};
%\node[box=0] (p-75-40) at (40,-75) {};
%\node[box=0] (p-75-41) at (41,-75) {};
%\node[box=0] (p-75-42) at (42,-75) {};
%\node[box=0] (p-75-43) at (43,-75) {};
%\node[box=0] (p-75-44) at (44,-75) {};
%\node[box=0] (p-75-45) at (45,-75) {};
%\node[box=0] (p-75-46) at (46,-75) {};
%\node[box=0] (p-75-47) at (47,-75) {};
%\node[box=0] (p-75-48) at (48,-75) {};
%\node[box=0] (p-75-49) at (49,-75) {};
%\node[box=0] (p-75-50) at (50,-75) {};
%\node[box=0] (p-75-51) at (51,-75) {};
%\node[box=0] (p-75-52) at (52,-75) {};
%\node[box=0] (p-75-53) at (53,-75) {};
%\node[box=0] (p-75-54) at (54,-75) {};
%\node[box=0] (p-75-55) at (55,-75) {};
%\node[box=0] (p-75-56) at (56,-75) {};
%\node[box=0] (p-75-57) at (57,-75) {};
%\node[box=0] (p-75-58) at (58,-75) {};
%\node[box=0] (p-75-59) at (59,-75) {};
%\node[box=0] (p-75-60) at (60,-75) {};
%\node[box=0] (p-75-61) at (61,-75) {};
%\node[box=0] (p-75-62) at (62,-75) {};
%\node[box=0] (p-75-63) at (63,-75) {};
%\node[box=1] (p-75-64) at (64,-75) {};
%\node[box=1] (p-75-65) at (65,-75) {};
%\node[box=1] (p-75-66) at (66,-75) {};
%\node[box=1] (p-75-67) at (67,-75) {};
%\node[box=0] (p-75-68) at (68,-75) {};
%\node[box=0] (p-75-69) at (69,-75) {};
%\node[box=0] (p-75-70) at (70,-75) {};
%\node[box=0] (p-75-71) at (71,-75) {};
%\node[box=1] (p-75-72) at (72,-75) {};
%\node[box=1] (p-75-73) at (73,-75) {};
%\node[box=1] (p-75-74) at (74,-75) {};
%\node[box=1] (p-75-75) at (75,-75) {};
%\node[box=1] (p-76-0) at (0,-76) {};
%\node[box=0] (p-76-1) at (1,-76) {};
%\node[box=0] (p-76-2) at (2,-76) {};
%\node[box=0] (p-76-3) at (3,-76) {};
%\node[box=1] (p-76-4) at (4,-76) {};
%\node[box=0] (p-76-5) at (5,-76) {};
%\node[box=0] (p-76-6) at (6,-76) {};
%\node[box=0] (p-76-7) at (7,-76) {};
%\node[box=1] (p-76-8) at (8,-76) {};
%\node[box=0] (p-76-9) at (9,-76) {};
%\node[box=0] (p-76-10) at (10,-76) {};
%\node[box=0] (p-76-11) at (11,-76) {};
%\node[box=1] (p-76-12) at (12,-76) {};
%\node[box=0] (p-76-13) at (13,-76) {};
%\node[box=0] (p-76-14) at (14,-76) {};
%\node[box=0] (p-76-15) at (15,-76) {};
%\node[box=0] (p-76-16) at (16,-76) {};
%\node[box=0] (p-76-17) at (17,-76) {};
%\node[box=0] (p-76-18) at (18,-76) {};
%\node[box=0] (p-76-19) at (19,-76) {};
%\node[box=0] (p-76-20) at (20,-76) {};
%\node[box=0] (p-76-21) at (21,-76) {};
%\node[box=0] (p-76-22) at (22,-76) {};
%\node[box=0] (p-76-23) at (23,-76) {};
%\node[box=0] (p-76-24) at (24,-76) {};
%\node[box=0] (p-76-25) at (25,-76) {};
%\node[box=0] (p-76-26) at (26,-76) {};
%\node[box=0] (p-76-27) at (27,-76) {};
%\node[box=0] (p-76-28) at (28,-76) {};
%\node[box=0] (p-76-29) at (29,-76) {};
%\node[box=0] (p-76-30) at (30,-76) {};
%\node[box=0] (p-76-31) at (31,-76) {};
%\node[box=0] (p-76-32) at (32,-76) {};
%\node[box=0] (p-76-33) at (33,-76) {};
%\node[box=0] (p-76-34) at (34,-76) {};
%\node[box=0] (p-76-35) at (35,-76) {};
%\node[box=0] (p-76-36) at (36,-76) {};
%\node[box=0] (p-76-37) at (37,-76) {};
%\node[box=0] (p-76-38) at (38,-76) {};
%\node[box=0] (p-76-39) at (39,-76) {};
%\node[box=0] (p-76-40) at (40,-76) {};
%\node[box=0] (p-76-41) at (41,-76) {};
%\node[box=0] (p-76-42) at (42,-76) {};
%\node[box=0] (p-76-43) at (43,-76) {};
%\node[box=0] (p-76-44) at (44,-76) {};
%\node[box=0] (p-76-45) at (45,-76) {};
%\node[box=0] (p-76-46) at (46,-76) {};
%\node[box=0] (p-76-47) at (47,-76) {};
%\node[box=0] (p-76-48) at (48,-76) {};
%\node[box=0] (p-76-49) at (49,-76) {};
%\node[box=0] (p-76-50) at (50,-76) {};
%\node[box=0] (p-76-51) at (51,-76) {};
%\node[box=0] (p-76-52) at (52,-76) {};
%\node[box=0] (p-76-53) at (53,-76) {};
%\node[box=0] (p-76-54) at (54,-76) {};
%\node[box=0] (p-76-55) at (55,-76) {};
%\node[box=0] (p-76-56) at (56,-76) {};
%\node[box=0] (p-76-57) at (57,-76) {};
%\node[box=0] (p-76-58) at (58,-76) {};
%\node[box=0] (p-76-59) at (59,-76) {};
%\node[box=0] (p-76-60) at (60,-76) {};
%\node[box=0] (p-76-61) at (61,-76) {};
%\node[box=0] (p-76-62) at (62,-76) {};
%\node[box=0] (p-76-63) at (63,-76) {};
%\node[box=1] (p-76-64) at (64,-76) {};
%\node[box=0] (p-76-65) at (65,-76) {};
%\node[box=0] (p-76-66) at (66,-76) {};
%\node[box=0] (p-76-67) at (67,-76) {};
%\node[box=1] (p-76-68) at (68,-76) {};
%\node[box=0] (p-76-69) at (69,-76) {};
%\node[box=0] (p-76-70) at (70,-76) {};
%\node[box=0] (p-76-71) at (71,-76) {};
%\node[box=1] (p-76-72) at (72,-76) {};
%\node[box=0] (p-76-73) at (73,-76) {};
%\node[box=0] (p-76-74) at (74,-76) {};
%\node[box=0] (p-76-75) at (75,-76) {};
%\node[box=1] (p-76-76) at (76,-76) {};
%\node[box=1] (p-77-0) at (0,-77) {};
%\node[box=1] (p-77-1) at (1,-77) {};
%\node[box=0] (p-77-2) at (2,-77) {};
%\node[box=0] (p-77-3) at (3,-77) {};
%\node[box=1] (p-77-4) at (4,-77) {};
%\node[box=1] (p-77-5) at (5,-77) {};
%\node[box=0] (p-77-6) at (6,-77) {};
%\node[box=0] (p-77-7) at (7,-77) {};
%\node[box=1] (p-77-8) at (8,-77) {};
%\node[box=1] (p-77-9) at (9,-77) {};
%\node[box=0] (p-77-10) at (10,-77) {};
%\node[box=0] (p-77-11) at (11,-77) {};
%\node[box=1] (p-77-12) at (12,-77) {};
%\node[box=1] (p-77-13) at (13,-77) {};
%\node[box=0] (p-77-14) at (14,-77) {};
%\node[box=0] (p-77-15) at (15,-77) {};
%\node[box=0] (p-77-16) at (16,-77) {};
%\node[box=0] (p-77-17) at (17,-77) {};
%\node[box=0] (p-77-18) at (18,-77) {};
%\node[box=0] (p-77-19) at (19,-77) {};
%\node[box=0] (p-77-20) at (20,-77) {};
%\node[box=0] (p-77-21) at (21,-77) {};
%\node[box=0] (p-77-22) at (22,-77) {};
%\node[box=0] (p-77-23) at (23,-77) {};
%\node[box=0] (p-77-24) at (24,-77) {};
%\node[box=0] (p-77-25) at (25,-77) {};
%\node[box=0] (p-77-26) at (26,-77) {};
%\node[box=0] (p-77-27) at (27,-77) {};
%\node[box=0] (p-77-28) at (28,-77) {};
%\node[box=0] (p-77-29) at (29,-77) {};
%\node[box=0] (p-77-30) at (30,-77) {};
%\node[box=0] (p-77-31) at (31,-77) {};
%\node[box=0] (p-77-32) at (32,-77) {};
%\node[box=0] (p-77-33) at (33,-77) {};
%\node[box=0] (p-77-34) at (34,-77) {};
%\node[box=0] (p-77-35) at (35,-77) {};
%\node[box=0] (p-77-36) at (36,-77) {};
%\node[box=0] (p-77-37) at (37,-77) {};
%\node[box=0] (p-77-38) at (38,-77) {};
%\node[box=0] (p-77-39) at (39,-77) {};
%\node[box=0] (p-77-40) at (40,-77) {};
%\node[box=0] (p-77-41) at (41,-77) {};
%\node[box=0] (p-77-42) at (42,-77) {};
%\node[box=0] (p-77-43) at (43,-77) {};
%\node[box=0] (p-77-44) at (44,-77) {};
%\node[box=0] (p-77-45) at (45,-77) {};
%\node[box=0] (p-77-46) at (46,-77) {};
%\node[box=0] (p-77-47) at (47,-77) {};
%\node[box=0] (p-77-48) at (48,-77) {};
%\node[box=0] (p-77-49) at (49,-77) {};
%\node[box=0] (p-77-50) at (50,-77) {};
%\node[box=0] (p-77-51) at (51,-77) {};
%\node[box=0] (p-77-52) at (52,-77) {};
%\node[box=0] (p-77-53) at (53,-77) {};
%\node[box=0] (p-77-54) at (54,-77) {};
%\node[box=0] (p-77-55) at (55,-77) {};
%\node[box=0] (p-77-56) at (56,-77) {};
%\node[box=0] (p-77-57) at (57,-77) {};
%\node[box=0] (p-77-58) at (58,-77) {};
%\node[box=0] (p-77-59) at (59,-77) {};
%\node[box=0] (p-77-60) at (60,-77) {};
%\node[box=0] (p-77-61) at (61,-77) {};
%\node[box=0] (p-77-62) at (62,-77) {};
%\node[box=0] (p-77-63) at (63,-77) {};
%\node[box=1] (p-77-64) at (64,-77) {};
%\node[box=1] (p-77-65) at (65,-77) {};
%\node[box=0] (p-77-66) at (66,-77) {};
%\node[box=0] (p-77-67) at (67,-77) {};
%\node[box=1] (p-77-68) at (68,-77) {};
%\node[box=1] (p-77-69) at (69,-77) {};
%\node[box=0] (p-77-70) at (70,-77) {};
%\node[box=0] (p-77-71) at (71,-77) {};
%\node[box=1] (p-77-72) at (72,-77) {};
%\node[box=1] (p-77-73) at (73,-77) {};
%\node[box=0] (p-77-74) at (74,-77) {};
%\node[box=0] (p-77-75) at (75,-77) {};
%\node[box=1] (p-77-76) at (76,-77) {};
%\node[box=1] (p-77-77) at (77,-77) {};
%\node[box=1] (p-78-0) at (0,-78) {};
%\node[box=0] (p-78-1) at (1,-78) {};
%\node[box=1] (p-78-2) at (2,-78) {};
%\node[box=0] (p-78-3) at (3,-78) {};
%\node[box=1] (p-78-4) at (4,-78) {};
%\node[box=0] (p-78-5) at (5,-78) {};
%\node[box=1] (p-78-6) at (6,-78) {};
%\node[box=0] (p-78-7) at (7,-78) {};
%\node[box=1] (p-78-8) at (8,-78) {};
%\node[box=0] (p-78-9) at (9,-78) {};
%\node[box=1] (p-78-10) at (10,-78) {};
%\node[box=0] (p-78-11) at (11,-78) {};
%\node[box=1] (p-78-12) at (12,-78) {};
%\node[box=0] (p-78-13) at (13,-78) {};
%\node[box=1] (p-78-14) at (14,-78) {};
%\node[box=0] (p-78-15) at (15,-78) {};
%\node[box=0] (p-78-16) at (16,-78) {};
%\node[box=0] (p-78-17) at (17,-78) {};
%\node[box=0] (p-78-18) at (18,-78) {};
%\node[box=0] (p-78-19) at (19,-78) {};
%\node[box=0] (p-78-20) at (20,-78) {};
%\node[box=0] (p-78-21) at (21,-78) {};
%\node[box=0] (p-78-22) at (22,-78) {};
%\node[box=0] (p-78-23) at (23,-78) {};
%\node[box=0] (p-78-24) at (24,-78) {};
%\node[box=0] (p-78-25) at (25,-78) {};
%\node[box=0] (p-78-26) at (26,-78) {};
%\node[box=0] (p-78-27) at (27,-78) {};
%\node[box=0] (p-78-28) at (28,-78) {};
%\node[box=0] (p-78-29) at (29,-78) {};
%\node[box=0] (p-78-30) at (30,-78) {};
%\node[box=0] (p-78-31) at (31,-78) {};
%\node[box=0] (p-78-32) at (32,-78) {};
%\node[box=0] (p-78-33) at (33,-78) {};
%\node[box=0] (p-78-34) at (34,-78) {};
%\node[box=0] (p-78-35) at (35,-78) {};
%\node[box=0] (p-78-36) at (36,-78) {};
%\node[box=0] (p-78-37) at (37,-78) {};
%\node[box=0] (p-78-38) at (38,-78) {};
%\node[box=0] (p-78-39) at (39,-78) {};
%\node[box=0] (p-78-40) at (40,-78) {};
%\node[box=0] (p-78-41) at (41,-78) {};
%\node[box=0] (p-78-42) at (42,-78) {};
%\node[box=0] (p-78-43) at (43,-78) {};
%\node[box=0] (p-78-44) at (44,-78) {};
%\node[box=0] (p-78-45) at (45,-78) {};
%\node[box=0] (p-78-46) at (46,-78) {};
%\node[box=0] (p-78-47) at (47,-78) {};
%\node[box=0] (p-78-48) at (48,-78) {};
%\node[box=0] (p-78-49) at (49,-78) {};
%\node[box=0] (p-78-50) at (50,-78) {};
%\node[box=0] (p-78-51) at (51,-78) {};
%\node[box=0] (p-78-52) at (52,-78) {};
%\node[box=0] (p-78-53) at (53,-78) {};
%\node[box=0] (p-78-54) at (54,-78) {};
%\node[box=0] (p-78-55) at (55,-78) {};
%\node[box=0] (p-78-56) at (56,-78) {};
%\node[box=0] (p-78-57) at (57,-78) {};
%\node[box=0] (p-78-58) at (58,-78) {};
%\node[box=0] (p-78-59) at (59,-78) {};
%\node[box=0] (p-78-60) at (60,-78) {};
%\node[box=0] (p-78-61) at (61,-78) {};
%\node[box=0] (p-78-62) at (62,-78) {};
%\node[box=0] (p-78-63) at (63,-78) {};
%\node[box=1] (p-78-64) at (64,-78) {};
%\node[box=0] (p-78-65) at (65,-78) {};
%\node[box=1] (p-78-66) at (66,-78) {};
%\node[box=0] (p-78-67) at (67,-78) {};
%\node[box=1] (p-78-68) at (68,-78) {};
%\node[box=0] (p-78-69) at (69,-78) {};
%\node[box=1] (p-78-70) at (70,-78) {};
%\node[box=0] (p-78-71) at (71,-78) {};
%\node[box=1] (p-78-72) at (72,-78) {};
%\node[box=0] (p-78-73) at (73,-78) {};
%\node[box=1] (p-78-74) at (74,-78) {};
%\node[box=0] (p-78-75) at (75,-78) {};
%\node[box=1] (p-78-76) at (76,-78) {};
%\node[box=0] (p-78-77) at (77,-78) {};
%\node[box=1] (p-78-78) at (78,-78) {};
%\node[box=1] (p-79-0) at (0,-79) {};
%\node[box=1] (p-79-1) at (1,-79) {};
%\node[box=1] (p-79-2) at (2,-79) {};
%\node[box=1] (p-79-3) at (3,-79) {};
%\node[box=1] (p-79-4) at (4,-79) {};
%\node[box=1] (p-79-5) at (5,-79) {};
%\node[box=1] (p-79-6) at (6,-79) {};
%\node[box=1] (p-79-7) at (7,-79) {};
%\node[box=1] (p-79-8) at (8,-79) {};
%\node[box=1] (p-79-9) at (9,-79) {};
%\node[box=1] (p-79-10) at (10,-79) {};
%\node[box=1] (p-79-11) at (11,-79) {};
%\node[box=1] (p-79-12) at (12,-79) {};
%\node[box=1] (p-79-13) at (13,-79) {};
%\node[box=1] (p-79-14) at (14,-79) {};
%\node[box=1] (p-79-15) at (15,-79) {};
%\node[box=0] (p-79-16) at (16,-79) {};
%\node[box=0] (p-79-17) at (17,-79) {};
%\node[box=0] (p-79-18) at (18,-79) {};
%\node[box=0] (p-79-19) at (19,-79) {};
%\node[box=0] (p-79-20) at (20,-79) {};
%\node[box=0] (p-79-21) at (21,-79) {};
%\node[box=0] (p-79-22) at (22,-79) {};
%\node[box=0] (p-79-23) at (23,-79) {};
%\node[box=0] (p-79-24) at (24,-79) {};
%\node[box=0] (p-79-25) at (25,-79) {};
%\node[box=0] (p-79-26) at (26,-79) {};
%\node[box=0] (p-79-27) at (27,-79) {};
%\node[box=0] (p-79-28) at (28,-79) {};
%\node[box=0] (p-79-29) at (29,-79) {};
%\node[box=0] (p-79-30) at (30,-79) {};
%\node[box=0] (p-79-31) at (31,-79) {};
%\node[box=0] (p-79-32) at (32,-79) {};
%\node[box=0] (p-79-33) at (33,-79) {};
%\node[box=0] (p-79-34) at (34,-79) {};
%\node[box=0] (p-79-35) at (35,-79) {};
%\node[box=0] (p-79-36) at (36,-79) {};
%\node[box=0] (p-79-37) at (37,-79) {};
%\node[box=0] (p-79-38) at (38,-79) {};
%\node[box=0] (p-79-39) at (39,-79) {};
%\node[box=0] (p-79-40) at (40,-79) {};
%\node[box=0] (p-79-41) at (41,-79) {};
%\node[box=0] (p-79-42) at (42,-79) {};
%\node[box=0] (p-79-43) at (43,-79) {};
%\node[box=0] (p-79-44) at (44,-79) {};
%\node[box=0] (p-79-45) at (45,-79) {};
%\node[box=0] (p-79-46) at (46,-79) {};
%\node[box=0] (p-79-47) at (47,-79) {};
%\node[box=0] (p-79-48) at (48,-79) {};
%\node[box=0] (p-79-49) at (49,-79) {};
%\node[box=0] (p-79-50) at (50,-79) {};
%\node[box=0] (p-79-51) at (51,-79) {};
%\node[box=0] (p-79-52) at (52,-79) {};
%\node[box=0] (p-79-53) at (53,-79) {};
%\node[box=0] (p-79-54) at (54,-79) {};
%\node[box=0] (p-79-55) at (55,-79) {};
%\node[box=0] (p-79-56) at (56,-79) {};
%\node[box=0] (p-79-57) at (57,-79) {};
%\node[box=0] (p-79-58) at (58,-79) {};
%\node[box=0] (p-79-59) at (59,-79) {};
%\node[box=0] (p-79-60) at (60,-79) {};
%\node[box=0] (p-79-61) at (61,-79) {};
%\node[box=0] (p-79-62) at (62,-79) {};
%\node[box=0] (p-79-63) at (63,-79) {};
%\node[box=1] (p-79-64) at (64,-79) {};
%\node[box=1] (p-79-65) at (65,-79) {};
%\node[box=1] (p-79-66) at (66,-79) {};
%\node[box=1] (p-79-67) at (67,-79) {};
%\node[box=1] (p-79-68) at (68,-79) {};
%\node[box=1] (p-79-69) at (69,-79) {};
%\node[box=1] (p-79-70) at (70,-79) {};
%\node[box=1] (p-79-71) at (71,-79) {};
%\node[box=1] (p-79-72) at (72,-79) {};
%\node[box=1] (p-79-73) at (73,-79) {};
%\node[box=1] (p-79-74) at (74,-79) {};
%\node[box=1] (p-79-75) at (75,-79) {};
%\node[box=1] (p-79-76) at (76,-79) {};
%\node[box=1] (p-79-77) at (77,-79) {};
%\node[box=1] (p-79-78) at (78,-79) {};
%\node[box=1] (p-79-79) at (79,-79) {};
%\node[box=1] (p-80-0) at (0,-80) {};
%\node[box=0] (p-80-1) at (1,-80) {};
%\node[box=0] (p-80-2) at (2,-80) {};
%\node[box=0] (p-80-3) at (3,-80) {};
%\node[box=0] (p-80-4) at (4,-80) {};
%\node[box=0] (p-80-5) at (5,-80) {};
%\node[box=0] (p-80-6) at (6,-80) {};
%\node[box=0] (p-80-7) at (7,-80) {};
%\node[box=0] (p-80-8) at (8,-80) {};
%\node[box=0] (p-80-9) at (9,-80) {};
%\node[box=0] (p-80-10) at (10,-80) {};
%\node[box=0] (p-80-11) at (11,-80) {};
%\node[box=0] (p-80-12) at (12,-80) {};
%\node[box=0] (p-80-13) at (13,-80) {};
%\node[box=0] (p-80-14) at (14,-80) {};
%\node[box=0] (p-80-15) at (15,-80) {};
%\node[box=1] (p-80-16) at (16,-80) {};
%\node[box=0] (p-80-17) at (17,-80) {};
%\node[box=0] (p-80-18) at (18,-80) {};
%\node[box=0] (p-80-19) at (19,-80) {};
%\node[box=0] (p-80-20) at (20,-80) {};
%\node[box=0] (p-80-21) at (21,-80) {};
%\node[box=0] (p-80-22) at (22,-80) {};
%\node[box=0] (p-80-23) at (23,-80) {};
%\node[box=0] (p-80-24) at (24,-80) {};
%\node[box=0] (p-80-25) at (25,-80) {};
%\node[box=0] (p-80-26) at (26,-80) {};
%\node[box=0] (p-80-27) at (27,-80) {};
%\node[box=0] (p-80-28) at (28,-80) {};
%\node[box=0] (p-80-29) at (29,-80) {};
%\node[box=0] (p-80-30) at (30,-80) {};
%\node[box=0] (p-80-31) at (31,-80) {};
%\node[box=0] (p-80-32) at (32,-80) {};
%\node[box=0] (p-80-33) at (33,-80) {};
%\node[box=0] (p-80-34) at (34,-80) {};
%\node[box=0] (p-80-35) at (35,-80) {};
%\node[box=0] (p-80-36) at (36,-80) {};
%\node[box=0] (p-80-37) at (37,-80) {};
%\node[box=0] (p-80-38) at (38,-80) {};
%\node[box=0] (p-80-39) at (39,-80) {};
%\node[box=0] (p-80-40) at (40,-80) {};
%\node[box=0] (p-80-41) at (41,-80) {};
%\node[box=0] (p-80-42) at (42,-80) {};
%\node[box=0] (p-80-43) at (43,-80) {};
%\node[box=0] (p-80-44) at (44,-80) {};
%\node[box=0] (p-80-45) at (45,-80) {};
%\node[box=0] (p-80-46) at (46,-80) {};
%\node[box=0] (p-80-47) at (47,-80) {};
%\node[box=0] (p-80-48) at (48,-80) {};
%\node[box=0] (p-80-49) at (49,-80) {};
%\node[box=0] (p-80-50) at (50,-80) {};
%\node[box=0] (p-80-51) at (51,-80) {};
%\node[box=0] (p-80-52) at (52,-80) {};
%\node[box=0] (p-80-53) at (53,-80) {};
%\node[box=0] (p-80-54) at (54,-80) {};
%\node[box=0] (p-80-55) at (55,-80) {};
%\node[box=0] (p-80-56) at (56,-80) {};
%\node[box=0] (p-80-57) at (57,-80) {};
%\node[box=0] (p-80-58) at (58,-80) {};
%\node[box=0] (p-80-59) at (59,-80) {};
%\node[box=0] (p-80-60) at (60,-80) {};
%\node[box=0] (p-80-61) at (61,-80) {};
%\node[box=0] (p-80-62) at (62,-80) {};
%\node[box=0] (p-80-63) at (63,-80) {};
%\node[box=1] (p-80-64) at (64,-80) {};
%\node[box=0] (p-80-65) at (65,-80) {};
%\node[box=0] (p-80-66) at (66,-80) {};
%\node[box=0] (p-80-67) at (67,-80) {};
%\node[box=0] (p-80-68) at (68,-80) {};
%\node[box=0] (p-80-69) at (69,-80) {};
%\node[box=0] (p-80-70) at (70,-80) {};
%\node[box=0] (p-80-71) at (71,-80) {};
%\node[box=0] (p-80-72) at (72,-80) {};
%\node[box=0] (p-80-73) at (73,-80) {};
%\node[box=0] (p-80-74) at (74,-80) {};
%\node[box=0] (p-80-75) at (75,-80) {};
%\node[box=0] (p-80-76) at (76,-80) {};
%\node[box=0] (p-80-77) at (77,-80) {};
%\node[box=0] (p-80-78) at (78,-80) {};
%\node[box=0] (p-80-79) at (79,-80) {};
%\node[box=1] (p-80-80) at (80,-80) {};
%\node[box=1] (p-81-0) at (0,-81) {};
%\node[box=1] (p-81-1) at (1,-81) {};
%\node[box=0] (p-81-2) at (2,-81) {};
%\node[box=0] (p-81-3) at (3,-81) {};
%\node[box=0] (p-81-4) at (4,-81) {};
%\node[box=0] (p-81-5) at (5,-81) {};
%\node[box=0] (p-81-6) at (6,-81) {};
%\node[box=0] (p-81-7) at (7,-81) {};
%\node[box=0] (p-81-8) at (8,-81) {};
%\node[box=0] (p-81-9) at (9,-81) {};
%\node[box=0] (p-81-10) at (10,-81) {};
%\node[box=0] (p-81-11) at (11,-81) {};
%\node[box=0] (p-81-12) at (12,-81) {};
%\node[box=0] (p-81-13) at (13,-81) {};
%\node[box=0] (p-81-14) at (14,-81) {};
%\node[box=0] (p-81-15) at (15,-81) {};
%\node[box=1] (p-81-16) at (16,-81) {};
%\node[box=1] (p-81-17) at (17,-81) {};
%\node[box=0] (p-81-18) at (18,-81) {};
%\node[box=0] (p-81-19) at (19,-81) {};
%\node[box=0] (p-81-20) at (20,-81) {};
%\node[box=0] (p-81-21) at (21,-81) {};
%\node[box=0] (p-81-22) at (22,-81) {};
%\node[box=0] (p-81-23) at (23,-81) {};
%\node[box=0] (p-81-24) at (24,-81) {};
%\node[box=0] (p-81-25) at (25,-81) {};
%\node[box=0] (p-81-26) at (26,-81) {};
%\node[box=0] (p-81-27) at (27,-81) {};
%\node[box=0] (p-81-28) at (28,-81) {};
%\node[box=0] (p-81-29) at (29,-81) {};
%\node[box=0] (p-81-30) at (30,-81) {};
%\node[box=0] (p-81-31) at (31,-81) {};
%\node[box=0] (p-81-32) at (32,-81) {};
%\node[box=0] (p-81-33) at (33,-81) {};
%\node[box=0] (p-81-34) at (34,-81) {};
%\node[box=0] (p-81-35) at (35,-81) {};
%\node[box=0] (p-81-36) at (36,-81) {};
%\node[box=0] (p-81-37) at (37,-81) {};
%\node[box=0] (p-81-38) at (38,-81) {};
%\node[box=0] (p-81-39) at (39,-81) {};
%\node[box=0] (p-81-40) at (40,-81) {};
%\node[box=0] (p-81-41) at (41,-81) {};
%\node[box=0] (p-81-42) at (42,-81) {};
%\node[box=0] (p-81-43) at (43,-81) {};
%\node[box=0] (p-81-44) at (44,-81) {};
%\node[box=0] (p-81-45) at (45,-81) {};
%\node[box=0] (p-81-46) at (46,-81) {};
%\node[box=0] (p-81-47) at (47,-81) {};
%\node[box=0] (p-81-48) at (48,-81) {};
%\node[box=0] (p-81-49) at (49,-81) {};
%\node[box=0] (p-81-50) at (50,-81) {};
%\node[box=0] (p-81-51) at (51,-81) {};
%\node[box=0] (p-81-52) at (52,-81) {};
%\node[box=0] (p-81-53) at (53,-81) {};
%\node[box=0] (p-81-54) at (54,-81) {};
%\node[box=0] (p-81-55) at (55,-81) {};
%\node[box=0] (p-81-56) at (56,-81) {};
%\node[box=0] (p-81-57) at (57,-81) {};
%\node[box=0] (p-81-58) at (58,-81) {};
%\node[box=0] (p-81-59) at (59,-81) {};
%\node[box=0] (p-81-60) at (60,-81) {};
%\node[box=0] (p-81-61) at (61,-81) {};
%\node[box=0] (p-81-62) at (62,-81) {};
%\node[box=0] (p-81-63) at (63,-81) {};
%\node[box=1] (p-81-64) at (64,-81) {};
%\node[box=1] (p-81-65) at (65,-81) {};
%\node[box=0] (p-81-66) at (66,-81) {};
%\node[box=0] (p-81-67) at (67,-81) {};
%\node[box=0] (p-81-68) at (68,-81) {};
%\node[box=0] (p-81-69) at (69,-81) {};
%\node[box=0] (p-81-70) at (70,-81) {};
%\node[box=0] (p-81-71) at (71,-81) {};
%\node[box=0] (p-81-72) at (72,-81) {};
%\node[box=0] (p-81-73) at (73,-81) {};
%\node[box=0] (p-81-74) at (74,-81) {};
%\node[box=0] (p-81-75) at (75,-81) {};
%\node[box=0] (p-81-76) at (76,-81) {};
%\node[box=0] (p-81-77) at (77,-81) {};
%\node[box=0] (p-81-78) at (78,-81) {};
%\node[box=0] (p-81-79) at (79,-81) {};
%\node[box=1] (p-81-80) at (80,-81) {};
%\node[box=1] (p-81-81) at (81,-81) {};
%\node[box=1] (p-82-0) at (0,-82) {};
%\node[box=0] (p-82-1) at (1,-82) {};
%\node[box=1] (p-82-2) at (2,-82) {};
%\node[box=0] (p-82-3) at (3,-82) {};
%\node[box=0] (p-82-4) at (4,-82) {};
%\node[box=0] (p-82-5) at (5,-82) {};
%\node[box=0] (p-82-6) at (6,-82) {};
%\node[box=0] (p-82-7) at (7,-82) {};
%\node[box=0] (p-82-8) at (8,-82) {};
%\node[box=0] (p-82-9) at (9,-82) {};
%\node[box=0] (p-82-10) at (10,-82) {};
%\node[box=0] (p-82-11) at (11,-82) {};
%\node[box=0] (p-82-12) at (12,-82) {};
%\node[box=0] (p-82-13) at (13,-82) {};
%\node[box=0] (p-82-14) at (14,-82) {};
%\node[box=0] (p-82-15) at (15,-82) {};
%\node[box=1] (p-82-16) at (16,-82) {};
%\node[box=0] (p-82-17) at (17,-82) {};
%\node[box=1] (p-82-18) at (18,-82) {};
%\node[box=0] (p-82-19) at (19,-82) {};
%\node[box=0] (p-82-20) at (20,-82) {};
%\node[box=0] (p-82-21) at (21,-82) {};
%\node[box=0] (p-82-22) at (22,-82) {};
%\node[box=0] (p-82-23) at (23,-82) {};
%\node[box=0] (p-82-24) at (24,-82) {};
%\node[box=0] (p-82-25) at (25,-82) {};
%\node[box=0] (p-82-26) at (26,-82) {};
%\node[box=0] (p-82-27) at (27,-82) {};
%\node[box=0] (p-82-28) at (28,-82) {};
%\node[box=0] (p-82-29) at (29,-82) {};
%\node[box=0] (p-82-30) at (30,-82) {};
%\node[box=0] (p-82-31) at (31,-82) {};
%\node[box=0] (p-82-32) at (32,-82) {};
%\node[box=0] (p-82-33) at (33,-82) {};
%\node[box=0] (p-82-34) at (34,-82) {};
%\node[box=0] (p-82-35) at (35,-82) {};
%\node[box=0] (p-82-36) at (36,-82) {};
%\node[box=0] (p-82-37) at (37,-82) {};
%\node[box=0] (p-82-38) at (38,-82) {};
%\node[box=0] (p-82-39) at (39,-82) {};
%\node[box=0] (p-82-40) at (40,-82) {};
%\node[box=0] (p-82-41) at (41,-82) {};
%\node[box=0] (p-82-42) at (42,-82) {};
%\node[box=0] (p-82-43) at (43,-82) {};
%\node[box=0] (p-82-44) at (44,-82) {};
%\node[box=0] (p-82-45) at (45,-82) {};
%\node[box=0] (p-82-46) at (46,-82) {};
%\node[box=0] (p-82-47) at (47,-82) {};
%\node[box=0] (p-82-48) at (48,-82) {};
%\node[box=0] (p-82-49) at (49,-82) {};
%\node[box=0] (p-82-50) at (50,-82) {};
%\node[box=0] (p-82-51) at (51,-82) {};
%\node[box=0] (p-82-52) at (52,-82) {};
%\node[box=0] (p-82-53) at (53,-82) {};
%\node[box=0] (p-82-54) at (54,-82) {};
%\node[box=0] (p-82-55) at (55,-82) {};
%\node[box=0] (p-82-56) at (56,-82) {};
%\node[box=0] (p-82-57) at (57,-82) {};
%\node[box=0] (p-82-58) at (58,-82) {};
%\node[box=0] (p-82-59) at (59,-82) {};
%\node[box=0] (p-82-60) at (60,-82) {};
%\node[box=0] (p-82-61) at (61,-82) {};
%\node[box=0] (p-82-62) at (62,-82) {};
%\node[box=0] (p-82-63) at (63,-82) {};
%\node[box=1] (p-82-64) at (64,-82) {};
%\node[box=0] (p-82-65) at (65,-82) {};
%\node[box=1] (p-82-66) at (66,-82) {};
%\node[box=0] (p-82-67) at (67,-82) {};
%\node[box=0] (p-82-68) at (68,-82) {};
%\node[box=0] (p-82-69) at (69,-82) {};
%\node[box=0] (p-82-70) at (70,-82) {};
%\node[box=0] (p-82-71) at (71,-82) {};
%\node[box=0] (p-82-72) at (72,-82) {};
%\node[box=0] (p-82-73) at (73,-82) {};
%\node[box=0] (p-82-74) at (74,-82) {};
%\node[box=0] (p-82-75) at (75,-82) {};
%\node[box=0] (p-82-76) at (76,-82) {};
%\node[box=0] (p-82-77) at (77,-82) {};
%\node[box=0] (p-82-78) at (78,-82) {};
%\node[box=0] (p-82-79) at (79,-82) {};
%\node[box=1] (p-82-80) at (80,-82) {};
%\node[box=0] (p-82-81) at (81,-82) {};
%\node[box=1] (p-82-82) at (82,-82) {};
%\node[box=1] (p-83-0) at (0,-83) {};
%\node[box=1] (p-83-1) at (1,-83) {};
%\node[box=1] (p-83-2) at (2,-83) {};
%\node[box=1] (p-83-3) at (3,-83) {};
%\node[box=0] (p-83-4) at (4,-83) {};
%\node[box=0] (p-83-5) at (5,-83) {};
%\node[box=0] (p-83-6) at (6,-83) {};
%\node[box=0] (p-83-7) at (7,-83) {};
%\node[box=0] (p-83-8) at (8,-83) {};
%\node[box=0] (p-83-9) at (9,-83) {};
%\node[box=0] (p-83-10) at (10,-83) {};
%\node[box=0] (p-83-11) at (11,-83) {};
%\node[box=0] (p-83-12) at (12,-83) {};
%\node[box=0] (p-83-13) at (13,-83) {};
%\node[box=0] (p-83-14) at (14,-83) {};
%\node[box=0] (p-83-15) at (15,-83) {};
%\node[box=1] (p-83-16) at (16,-83) {};
%\node[box=1] (p-83-17) at (17,-83) {};
%\node[box=1] (p-83-18) at (18,-83) {};
%\node[box=1] (p-83-19) at (19,-83) {};
%\node[box=0] (p-83-20) at (20,-83) {};
%\node[box=0] (p-83-21) at (21,-83) {};
%\node[box=0] (p-83-22) at (22,-83) {};
%\node[box=0] (p-83-23) at (23,-83) {};
%\node[box=0] (p-83-24) at (24,-83) {};
%\node[box=0] (p-83-25) at (25,-83) {};
%\node[box=0] (p-83-26) at (26,-83) {};
%\node[box=0] (p-83-27) at (27,-83) {};
%\node[box=0] (p-83-28) at (28,-83) {};
%\node[box=0] (p-83-29) at (29,-83) {};
%\node[box=0] (p-83-30) at (30,-83) {};
%\node[box=0] (p-83-31) at (31,-83) {};
%\node[box=0] (p-83-32) at (32,-83) {};
%\node[box=0] (p-83-33) at (33,-83) {};
%\node[box=0] (p-83-34) at (34,-83) {};
%\node[box=0] (p-83-35) at (35,-83) {};
%\node[box=0] (p-83-36) at (36,-83) {};
%\node[box=0] (p-83-37) at (37,-83) {};
%\node[box=0] (p-83-38) at (38,-83) {};
%\node[box=0] (p-83-39) at (39,-83) {};
%\node[box=0] (p-83-40) at (40,-83) {};
%\node[box=0] (p-83-41) at (41,-83) {};
%\node[box=0] (p-83-42) at (42,-83) {};
%\node[box=0] (p-83-43) at (43,-83) {};
%\node[box=0] (p-83-44) at (44,-83) {};
%\node[box=0] (p-83-45) at (45,-83) {};
%\node[box=0] (p-83-46) at (46,-83) {};
%\node[box=0] (p-83-47) at (47,-83) {};
%\node[box=0] (p-83-48) at (48,-83) {};
%\node[box=0] (p-83-49) at (49,-83) {};
%\node[box=0] (p-83-50) at (50,-83) {};
%\node[box=0] (p-83-51) at (51,-83) {};
%\node[box=0] (p-83-52) at (52,-83) {};
%\node[box=0] (p-83-53) at (53,-83) {};
%\node[box=0] (p-83-54) at (54,-83) {};
%\node[box=0] (p-83-55) at (55,-83) {};
%\node[box=0] (p-83-56) at (56,-83) {};
%\node[box=0] (p-83-57) at (57,-83) {};
%\node[box=0] (p-83-58) at (58,-83) {};
%\node[box=0] (p-83-59) at (59,-83) {};
%\node[box=0] (p-83-60) at (60,-83) {};
%\node[box=0] (p-83-61) at (61,-83) {};
%\node[box=0] (p-83-62) at (62,-83) {};
%\node[box=0] (p-83-63) at (63,-83) {};
%\node[box=1] (p-83-64) at (64,-83) {};
%\node[box=1] (p-83-65) at (65,-83) {};
%\node[box=1] (p-83-66) at (66,-83) {};
%\node[box=1] (p-83-67) at (67,-83) {};
%\node[box=0] (p-83-68) at (68,-83) {};
%\node[box=0] (p-83-69) at (69,-83) {};
%\node[box=0] (p-83-70) at (70,-83) {};
%\node[box=0] (p-83-71) at (71,-83) {};
%\node[box=0] (p-83-72) at (72,-83) {};
%\node[box=0] (p-83-73) at (73,-83) {};
%\node[box=0] (p-83-74) at (74,-83) {};
%\node[box=0] (p-83-75) at (75,-83) {};
%\node[box=0] (p-83-76) at (76,-83) {};
%\node[box=0] (p-83-77) at (77,-83) {};
%\node[box=0] (p-83-78) at (78,-83) {};
%\node[box=0] (p-83-79) at (79,-83) {};
%\node[box=1] (p-83-80) at (80,-83) {};
%\node[box=1] (p-83-81) at (81,-83) {};
%\node[box=1] (p-83-82) at (82,-83) {};
%\node[box=1] (p-83-83) at (83,-83) {};
%\node[box=1] (p-84-0) at (0,-84) {};
%\node[box=0] (p-84-1) at (1,-84) {};
%\node[box=0] (p-84-2) at (2,-84) {};
%\node[box=0] (p-84-3) at (3,-84) {};
%\node[box=1] (p-84-4) at (4,-84) {};
%\node[box=0] (p-84-5) at (5,-84) {};
%\node[box=0] (p-84-6) at (6,-84) {};
%\node[box=0] (p-84-7) at (7,-84) {};
%\node[box=0] (p-84-8) at (8,-84) {};
%\node[box=0] (p-84-9) at (9,-84) {};
%\node[box=0] (p-84-10) at (10,-84) {};
%\node[box=0] (p-84-11) at (11,-84) {};
%\node[box=0] (p-84-12) at (12,-84) {};
%\node[box=0] (p-84-13) at (13,-84) {};
%\node[box=0] (p-84-14) at (14,-84) {};
%\node[box=0] (p-84-15) at (15,-84) {};
%\node[box=1] (p-84-16) at (16,-84) {};
%\node[box=0] (p-84-17) at (17,-84) {};
%\node[box=0] (p-84-18) at (18,-84) {};
%\node[box=0] (p-84-19) at (19,-84) {};
%\node[box=1] (p-84-20) at (20,-84) {};
%\node[box=0] (p-84-21) at (21,-84) {};
%\node[box=0] (p-84-22) at (22,-84) {};
%\node[box=0] (p-84-23) at (23,-84) {};
%\node[box=0] (p-84-24) at (24,-84) {};
%\node[box=0] (p-84-25) at (25,-84) {};
%\node[box=0] (p-84-26) at (26,-84) {};
%\node[box=0] (p-84-27) at (27,-84) {};
%\node[box=0] (p-84-28) at (28,-84) {};
%\node[box=0] (p-84-29) at (29,-84) {};
%\node[box=0] (p-84-30) at (30,-84) {};
%\node[box=0] (p-84-31) at (31,-84) {};
%\node[box=0] (p-84-32) at (32,-84) {};
%\node[box=0] (p-84-33) at (33,-84) {};
%\node[box=0] (p-84-34) at (34,-84) {};
%\node[box=0] (p-84-35) at (35,-84) {};
%\node[box=0] (p-84-36) at (36,-84) {};
%\node[box=0] (p-84-37) at (37,-84) {};
%\node[box=0] (p-84-38) at (38,-84) {};
%\node[box=0] (p-84-39) at (39,-84) {};
%\node[box=0] (p-84-40) at (40,-84) {};
%\node[box=0] (p-84-41) at (41,-84) {};
%\node[box=0] (p-84-42) at (42,-84) {};
%\node[box=0] (p-84-43) at (43,-84) {};
%\node[box=0] (p-84-44) at (44,-84) {};
%\node[box=0] (p-84-45) at (45,-84) {};
%\node[box=0] (p-84-46) at (46,-84) {};
%\node[box=0] (p-84-47) at (47,-84) {};
%\node[box=0] (p-84-48) at (48,-84) {};
%\node[box=0] (p-84-49) at (49,-84) {};
%\node[box=0] (p-84-50) at (50,-84) {};
%\node[box=0] (p-84-51) at (51,-84) {};
%\node[box=0] (p-84-52) at (52,-84) {};
%\node[box=0] (p-84-53) at (53,-84) {};
%\node[box=0] (p-84-54) at (54,-84) {};
%\node[box=0] (p-84-55) at (55,-84) {};
%\node[box=0] (p-84-56) at (56,-84) {};
%\node[box=0] (p-84-57) at (57,-84) {};
%\node[box=0] (p-84-58) at (58,-84) {};
%\node[box=0] (p-84-59) at (59,-84) {};
%\node[box=0] (p-84-60) at (60,-84) {};
%\node[box=0] (p-84-61) at (61,-84) {};
%\node[box=0] (p-84-62) at (62,-84) {};
%\node[box=0] (p-84-63) at (63,-84) {};
%\node[box=1] (p-84-64) at (64,-84) {};
%\node[box=0] (p-84-65) at (65,-84) {};
%\node[box=0] (p-84-66) at (66,-84) {};
%\node[box=0] (p-84-67) at (67,-84) {};
%\node[box=1] (p-84-68) at (68,-84) {};
%\node[box=0] (p-84-69) at (69,-84) {};
%\node[box=0] (p-84-70) at (70,-84) {};
%\node[box=0] (p-84-71) at (71,-84) {};
%\node[box=0] (p-84-72) at (72,-84) {};
%\node[box=0] (p-84-73) at (73,-84) {};
%\node[box=0] (p-84-74) at (74,-84) {};
%\node[box=0] (p-84-75) at (75,-84) {};
%\node[box=0] (p-84-76) at (76,-84) {};
%\node[box=0] (p-84-77) at (77,-84) {};
%\node[box=0] (p-84-78) at (78,-84) {};
%\node[box=0] (p-84-79) at (79,-84) {};
%\node[box=1] (p-84-80) at (80,-84) {};
%\node[box=0] (p-84-81) at (81,-84) {};
%\node[box=0] (p-84-82) at (82,-84) {};
%\node[box=0] (p-84-83) at (83,-84) {};
%\node[box=1] (p-84-84) at (84,-84) {};
%\node[box=1] (p-85-0) at (0,-85) {};
%\node[box=1] (p-85-1) at (1,-85) {};
%\node[box=0] (p-85-2) at (2,-85) {};
%\node[box=0] (p-85-3) at (3,-85) {};
%\node[box=1] (p-85-4) at (4,-85) {};
%\node[box=1] (p-85-5) at (5,-85) {};
%\node[box=0] (p-85-6) at (6,-85) {};
%\node[box=0] (p-85-7) at (7,-85) {};
%\node[box=0] (p-85-8) at (8,-85) {};
%\node[box=0] (p-85-9) at (9,-85) {};
%\node[box=0] (p-85-10) at (10,-85) {};
%\node[box=0] (p-85-11) at (11,-85) {};
%\node[box=0] (p-85-12) at (12,-85) {};
%\node[box=0] (p-85-13) at (13,-85) {};
%\node[box=0] (p-85-14) at (14,-85) {};
%\node[box=0] (p-85-15) at (15,-85) {};
%\node[box=1] (p-85-16) at (16,-85) {};
%\node[box=1] (p-85-17) at (17,-85) {};
%\node[box=0] (p-85-18) at (18,-85) {};
%\node[box=0] (p-85-19) at (19,-85) {};
%\node[box=1] (p-85-20) at (20,-85) {};
%\node[box=1] (p-85-21) at (21,-85) {};
%\node[box=0] (p-85-22) at (22,-85) {};
%\node[box=0] (p-85-23) at (23,-85) {};
%\node[box=0] (p-85-24) at (24,-85) {};
%\node[box=0] (p-85-25) at (25,-85) {};
%\node[box=0] (p-85-26) at (26,-85) {};
%\node[box=0] (p-85-27) at (27,-85) {};
%\node[box=0] (p-85-28) at (28,-85) {};
%\node[box=0] (p-85-29) at (29,-85) {};
%\node[box=0] (p-85-30) at (30,-85) {};
%\node[box=0] (p-85-31) at (31,-85) {};
%\node[box=0] (p-85-32) at (32,-85) {};
%\node[box=0] (p-85-33) at (33,-85) {};
%\node[box=0] (p-85-34) at (34,-85) {};
%\node[box=0] (p-85-35) at (35,-85) {};
%\node[box=0] (p-85-36) at (36,-85) {};
%\node[box=0] (p-85-37) at (37,-85) {};
%\node[box=0] (p-85-38) at (38,-85) {};
%\node[box=0] (p-85-39) at (39,-85) {};
%\node[box=0] (p-85-40) at (40,-85) {};
%\node[box=0] (p-85-41) at (41,-85) {};
%\node[box=0] (p-85-42) at (42,-85) {};
%\node[box=0] (p-85-43) at (43,-85) {};
%\node[box=0] (p-85-44) at (44,-85) {};
%\node[box=0] (p-85-45) at (45,-85) {};
%\node[box=0] (p-85-46) at (46,-85) {};
%\node[box=0] (p-85-47) at (47,-85) {};
%\node[box=0] (p-85-48) at (48,-85) {};
%\node[box=0] (p-85-49) at (49,-85) {};
%\node[box=0] (p-85-50) at (50,-85) {};
%\node[box=0] (p-85-51) at (51,-85) {};
%\node[box=0] (p-85-52) at (52,-85) {};
%\node[box=0] (p-85-53) at (53,-85) {};
%\node[box=0] (p-85-54) at (54,-85) {};
%\node[box=0] (p-85-55) at (55,-85) {};
%\node[box=0] (p-85-56) at (56,-85) {};
%\node[box=0] (p-85-57) at (57,-85) {};
%\node[box=0] (p-85-58) at (58,-85) {};
%\node[box=0] (p-85-59) at (59,-85) {};
%\node[box=0] (p-85-60) at (60,-85) {};
%\node[box=0] (p-85-61) at (61,-85) {};
%\node[box=0] (p-85-62) at (62,-85) {};
%\node[box=0] (p-85-63) at (63,-85) {};
%\node[box=1] (p-85-64) at (64,-85) {};
%\node[box=1] (p-85-65) at (65,-85) {};
%\node[box=0] (p-85-66) at (66,-85) {};
%\node[box=0] (p-85-67) at (67,-85) {};
%\node[box=1] (p-85-68) at (68,-85) {};
%\node[box=1] (p-85-69) at (69,-85) {};
%\node[box=0] (p-85-70) at (70,-85) {};
%\node[box=0] (p-85-71) at (71,-85) {};
%\node[box=0] (p-85-72) at (72,-85) {};
%\node[box=0] (p-85-73) at (73,-85) {};
%\node[box=0] (p-85-74) at (74,-85) {};
%\node[box=0] (p-85-75) at (75,-85) {};
%\node[box=0] (p-85-76) at (76,-85) {};
%\node[box=0] (p-85-77) at (77,-85) {};
%\node[box=0] (p-85-78) at (78,-85) {};
%\node[box=0] (p-85-79) at (79,-85) {};
%\node[box=1] (p-85-80) at (80,-85) {};
%\node[box=1] (p-85-81) at (81,-85) {};
%\node[box=0] (p-85-82) at (82,-85) {};
%\node[box=0] (p-85-83) at (83,-85) {};
%\node[box=1] (p-85-84) at (84,-85) {};
%\node[box=1] (p-85-85) at (85,-85) {};
%\node[box=1] (p-86-0) at (0,-86) {};
%\node[box=0] (p-86-1) at (1,-86) {};
%\node[box=1] (p-86-2) at (2,-86) {};
%\node[box=0] (p-86-3) at (3,-86) {};
%\node[box=1] (p-86-4) at (4,-86) {};
%\node[box=0] (p-86-5) at (5,-86) {};
%\node[box=1] (p-86-6) at (6,-86) {};
%\node[box=0] (p-86-7) at (7,-86) {};
%\node[box=0] (p-86-8) at (8,-86) {};
%\node[box=0] (p-86-9) at (9,-86) {};
%\node[box=0] (p-86-10) at (10,-86) {};
%\node[box=0] (p-86-11) at (11,-86) {};
%\node[box=0] (p-86-12) at (12,-86) {};
%\node[box=0] (p-86-13) at (13,-86) {};
%\node[box=0] (p-86-14) at (14,-86) {};
%\node[box=0] (p-86-15) at (15,-86) {};
%\node[box=1] (p-86-16) at (16,-86) {};
%\node[box=0] (p-86-17) at (17,-86) {};
%\node[box=1] (p-86-18) at (18,-86) {};
%\node[box=0] (p-86-19) at (19,-86) {};
%\node[box=1] (p-86-20) at (20,-86) {};
%\node[box=0] (p-86-21) at (21,-86) {};
%\node[box=1] (p-86-22) at (22,-86) {};
%\node[box=0] (p-86-23) at (23,-86) {};
%\node[box=0] (p-86-24) at (24,-86) {};
%\node[box=0] (p-86-25) at (25,-86) {};
%\node[box=0] (p-86-26) at (26,-86) {};
%\node[box=0] (p-86-27) at (27,-86) {};
%\node[box=0] (p-86-28) at (28,-86) {};
%\node[box=0] (p-86-29) at (29,-86) {};
%\node[box=0] (p-86-30) at (30,-86) {};
%\node[box=0] (p-86-31) at (31,-86) {};
%\node[box=0] (p-86-32) at (32,-86) {};
%\node[box=0] (p-86-33) at (33,-86) {};
%\node[box=0] (p-86-34) at (34,-86) {};
%\node[box=0] (p-86-35) at (35,-86) {};
%\node[box=0] (p-86-36) at (36,-86) {};
%\node[box=0] (p-86-37) at (37,-86) {};
%\node[box=0] (p-86-38) at (38,-86) {};
%\node[box=0] (p-86-39) at (39,-86) {};
%\node[box=0] (p-86-40) at (40,-86) {};
%\node[box=0] (p-86-41) at (41,-86) {};
%\node[box=0] (p-86-42) at (42,-86) {};
%\node[box=0] (p-86-43) at (43,-86) {};
%\node[box=0] (p-86-44) at (44,-86) {};
%\node[box=0] (p-86-45) at (45,-86) {};
%\node[box=0] (p-86-46) at (46,-86) {};
%\node[box=0] (p-86-47) at (47,-86) {};
%\node[box=0] (p-86-48) at (48,-86) {};
%\node[box=0] (p-86-49) at (49,-86) {};
%\node[box=0] (p-86-50) at (50,-86) {};
%\node[box=0] (p-86-51) at (51,-86) {};
%\node[box=0] (p-86-52) at (52,-86) {};
%\node[box=0] (p-86-53) at (53,-86) {};
%\node[box=0] (p-86-54) at (54,-86) {};
%\node[box=0] (p-86-55) at (55,-86) {};
%\node[box=0] (p-86-56) at (56,-86) {};
%\node[box=0] (p-86-57) at (57,-86) {};
%\node[box=0] (p-86-58) at (58,-86) {};
%\node[box=0] (p-86-59) at (59,-86) {};
%\node[box=0] (p-86-60) at (60,-86) {};
%\node[box=0] (p-86-61) at (61,-86) {};
%\node[box=0] (p-86-62) at (62,-86) {};
%\node[box=0] (p-86-63) at (63,-86) {};
%\node[box=1] (p-86-64) at (64,-86) {};
%\node[box=0] (p-86-65) at (65,-86) {};
%\node[box=1] (p-86-66) at (66,-86) {};
%\node[box=0] (p-86-67) at (67,-86) {};
%\node[box=1] (p-86-68) at (68,-86) {};
%\node[box=0] (p-86-69) at (69,-86) {};
%\node[box=1] (p-86-70) at (70,-86) {};
%\node[box=0] (p-86-71) at (71,-86) {};
%\node[box=0] (p-86-72) at (72,-86) {};
%\node[box=0] (p-86-73) at (73,-86) {};
%\node[box=0] (p-86-74) at (74,-86) {};
%\node[box=0] (p-86-75) at (75,-86) {};
%\node[box=0] (p-86-76) at (76,-86) {};
%\node[box=0] (p-86-77) at (77,-86) {};
%\node[box=0] (p-86-78) at (78,-86) {};
%\node[box=0] (p-86-79) at (79,-86) {};
%\node[box=1] (p-86-80) at (80,-86) {};
%\node[box=0] (p-86-81) at (81,-86) {};
%\node[box=1] (p-86-82) at (82,-86) {};
%\node[box=0] (p-86-83) at (83,-86) {};
%\node[box=1] (p-86-84) at (84,-86) {};
%\node[box=0] (p-86-85) at (85,-86) {};
%\node[box=1] (p-86-86) at (86,-86) {};
%\node[box=1] (p-87-0) at (0,-87) {};
%\node[box=1] (p-87-1) at (1,-87) {};
%\node[box=1] (p-87-2) at (2,-87) {};
%\node[box=1] (p-87-3) at (3,-87) {};
%\node[box=1] (p-87-4) at (4,-87) {};
%\node[box=1] (p-87-5) at (5,-87) {};
%\node[box=1] (p-87-6) at (6,-87) {};
%\node[box=1] (p-87-7) at (7,-87) {};
%\node[box=0] (p-87-8) at (8,-87) {};
%\node[box=0] (p-87-9) at (9,-87) {};
%\node[box=0] (p-87-10) at (10,-87) {};
%\node[box=0] (p-87-11) at (11,-87) {};
%\node[box=0] (p-87-12) at (12,-87) {};
%\node[box=0] (p-87-13) at (13,-87) {};
%\node[box=0] (p-87-14) at (14,-87) {};
%\node[box=0] (p-87-15) at (15,-87) {};
%\node[box=1] (p-87-16) at (16,-87) {};
%\node[box=1] (p-87-17) at (17,-87) {};
%\node[box=1] (p-87-18) at (18,-87) {};
%\node[box=1] (p-87-19) at (19,-87) {};
%\node[box=1] (p-87-20) at (20,-87) {};
%\node[box=1] (p-87-21) at (21,-87) {};
%\node[box=1] (p-87-22) at (22,-87) {};
%\node[box=1] (p-87-23) at (23,-87) {};
%\node[box=0] (p-87-24) at (24,-87) {};
%\node[box=0] (p-87-25) at (25,-87) {};
%\node[box=0] (p-87-26) at (26,-87) {};
%\node[box=0] (p-87-27) at (27,-87) {};
%\node[box=0] (p-87-28) at (28,-87) {};
%\node[box=0] (p-87-29) at (29,-87) {};
%\node[box=0] (p-87-30) at (30,-87) {};
%\node[box=0] (p-87-31) at (31,-87) {};
%\node[box=0] (p-87-32) at (32,-87) {};
%\node[box=0] (p-87-33) at (33,-87) {};
%\node[box=0] (p-87-34) at (34,-87) {};
%\node[box=0] (p-87-35) at (35,-87) {};
%\node[box=0] (p-87-36) at (36,-87) {};
%\node[box=0] (p-87-37) at (37,-87) {};
%\node[box=0] (p-87-38) at (38,-87) {};
%\node[box=0] (p-87-39) at (39,-87) {};
%\node[box=0] (p-87-40) at (40,-87) {};
%\node[box=0] (p-87-41) at (41,-87) {};
%\node[box=0] (p-87-42) at (42,-87) {};
%\node[box=0] (p-87-43) at (43,-87) {};
%\node[box=0] (p-87-44) at (44,-87) {};
%\node[box=0] (p-87-45) at (45,-87) {};
%\node[box=0] (p-87-46) at (46,-87) {};
%\node[box=0] (p-87-47) at (47,-87) {};
%\node[box=0] (p-87-48) at (48,-87) {};
%\node[box=0] (p-87-49) at (49,-87) {};
%\node[box=0] (p-87-50) at (50,-87) {};
%\node[box=0] (p-87-51) at (51,-87) {};
%\node[box=0] (p-87-52) at (52,-87) {};
%\node[box=0] (p-87-53) at (53,-87) {};
%\node[box=0] (p-87-54) at (54,-87) {};
%\node[box=0] (p-87-55) at (55,-87) {};
%\node[box=0] (p-87-56) at (56,-87) {};
%\node[box=0] (p-87-57) at (57,-87) {};
%\node[box=0] (p-87-58) at (58,-87) {};
%\node[box=0] (p-87-59) at (59,-87) {};
%\node[box=0] (p-87-60) at (60,-87) {};
%\node[box=0] (p-87-61) at (61,-87) {};
%\node[box=0] (p-87-62) at (62,-87) {};
%\node[box=0] (p-87-63) at (63,-87) {};
%\node[box=1] (p-87-64) at (64,-87) {};
%\node[box=1] (p-87-65) at (65,-87) {};
%\node[box=1] (p-87-66) at (66,-87) {};
%\node[box=1] (p-87-67) at (67,-87) {};
%\node[box=1] (p-87-68) at (68,-87) {};
%\node[box=1] (p-87-69) at (69,-87) {};
%\node[box=1] (p-87-70) at (70,-87) {};
%\node[box=1] (p-87-71) at (71,-87) {};
%\node[box=0] (p-87-72) at (72,-87) {};
%\node[box=0] (p-87-73) at (73,-87) {};
%\node[box=0] (p-87-74) at (74,-87) {};
%\node[box=0] (p-87-75) at (75,-87) {};
%\node[box=0] (p-87-76) at (76,-87) {};
%\node[box=0] (p-87-77) at (77,-87) {};
%\node[box=0] (p-87-78) at (78,-87) {};
%\node[box=0] (p-87-79) at (79,-87) {};
%\node[box=1] (p-87-80) at (80,-87) {};
%\node[box=1] (p-87-81) at (81,-87) {};
%\node[box=1] (p-87-82) at (82,-87) {};
%\node[box=1] (p-87-83) at (83,-87) {};
%\node[box=1] (p-87-84) at (84,-87) {};
%\node[box=1] (p-87-85) at (85,-87) {};
%\node[box=1] (p-87-86) at (86,-87) {};
%\node[box=1] (p-87-87) at (87,-87) {};
%\node[box=1] (p-88-0) at (0,-88) {};
%\node[box=0] (p-88-1) at (1,-88) {};
%\node[box=0] (p-88-2) at (2,-88) {};
%\node[box=0] (p-88-3) at (3,-88) {};
%\node[box=0] (p-88-4) at (4,-88) {};
%\node[box=0] (p-88-5) at (5,-88) {};
%\node[box=0] (p-88-6) at (6,-88) {};
%\node[box=0] (p-88-7) at (7,-88) {};
%\node[box=1] (p-88-8) at (8,-88) {};
%\node[box=0] (p-88-9) at (9,-88) {};
%\node[box=0] (p-88-10) at (10,-88) {};
%\node[box=0] (p-88-11) at (11,-88) {};
%\node[box=0] (p-88-12) at (12,-88) {};
%\node[box=0] (p-88-13) at (13,-88) {};
%\node[box=0] (p-88-14) at (14,-88) {};
%\node[box=0] (p-88-15) at (15,-88) {};
%\node[box=1] (p-88-16) at (16,-88) {};
%\node[box=0] (p-88-17) at (17,-88) {};
%\node[box=0] (p-88-18) at (18,-88) {};
%\node[box=0] (p-88-19) at (19,-88) {};
%\node[box=0] (p-88-20) at (20,-88) {};
%\node[box=0] (p-88-21) at (21,-88) {};
%\node[box=0] (p-88-22) at (22,-88) {};
%\node[box=0] (p-88-23) at (23,-88) {};
%\node[box=1] (p-88-24) at (24,-88) {};
%\node[box=0] (p-88-25) at (25,-88) {};
%\node[box=0] (p-88-26) at (26,-88) {};
%\node[box=0] (p-88-27) at (27,-88) {};
%\node[box=0] (p-88-28) at (28,-88) {};
%\node[box=0] (p-88-29) at (29,-88) {};
%\node[box=0] (p-88-30) at (30,-88) {};
%\node[box=0] (p-88-31) at (31,-88) {};
%\node[box=0] (p-88-32) at (32,-88) {};
%\node[box=0] (p-88-33) at (33,-88) {};
%\node[box=0] (p-88-34) at (34,-88) {};
%\node[box=0] (p-88-35) at (35,-88) {};
%\node[box=0] (p-88-36) at (36,-88) {};
%\node[box=0] (p-88-37) at (37,-88) {};
%\node[box=0] (p-88-38) at (38,-88) {};
%\node[box=0] (p-88-39) at (39,-88) {};
%\node[box=0] (p-88-40) at (40,-88) {};
%\node[box=0] (p-88-41) at (41,-88) {};
%\node[box=0] (p-88-42) at (42,-88) {};
%\node[box=0] (p-88-43) at (43,-88) {};
%\node[box=0] (p-88-44) at (44,-88) {};
%\node[box=0] (p-88-45) at (45,-88) {};
%\node[box=0] (p-88-46) at (46,-88) {};
%\node[box=0] (p-88-47) at (47,-88) {};
%\node[box=0] (p-88-48) at (48,-88) {};
%\node[box=0] (p-88-49) at (49,-88) {};
%\node[box=0] (p-88-50) at (50,-88) {};
%\node[box=0] (p-88-51) at (51,-88) {};
%\node[box=0] (p-88-52) at (52,-88) {};
%\node[box=0] (p-88-53) at (53,-88) {};
%\node[box=0] (p-88-54) at (54,-88) {};
%\node[box=0] (p-88-55) at (55,-88) {};
%\node[box=0] (p-88-56) at (56,-88) {};
%\node[box=0] (p-88-57) at (57,-88) {};
%\node[box=0] (p-88-58) at (58,-88) {};
%\node[box=0] (p-88-59) at (59,-88) {};
%\node[box=0] (p-88-60) at (60,-88) {};
%\node[box=0] (p-88-61) at (61,-88) {};
%\node[box=0] (p-88-62) at (62,-88) {};
%\node[box=0] (p-88-63) at (63,-88) {};
%\node[box=1] (p-88-64) at (64,-88) {};
%\node[box=0] (p-88-65) at (65,-88) {};
%\node[box=0] (p-88-66) at (66,-88) {};
%\node[box=0] (p-88-67) at (67,-88) {};
%\node[box=0] (p-88-68) at (68,-88) {};
%\node[box=0] (p-88-69) at (69,-88) {};
%\node[box=0] (p-88-70) at (70,-88) {};
%\node[box=0] (p-88-71) at (71,-88) {};
%\node[box=1] (p-88-72) at (72,-88) {};
%\node[box=0] (p-88-73) at (73,-88) {};
%\node[box=0] (p-88-74) at (74,-88) {};
%\node[box=0] (p-88-75) at (75,-88) {};
%\node[box=0] (p-88-76) at (76,-88) {};
%\node[box=0] (p-88-77) at (77,-88) {};
%\node[box=0] (p-88-78) at (78,-88) {};
%\node[box=0] (p-88-79) at (79,-88) {};
%\node[box=1] (p-88-80) at (80,-88) {};
%\node[box=0] (p-88-81) at (81,-88) {};
%\node[box=0] (p-88-82) at (82,-88) {};
%\node[box=0] (p-88-83) at (83,-88) {};
%\node[box=0] (p-88-84) at (84,-88) {};
%\node[box=0] (p-88-85) at (85,-88) {};
%\node[box=0] (p-88-86) at (86,-88) {};
%\node[box=0] (p-88-87) at (87,-88) {};
%\node[box=1] (p-88-88) at (88,-88) {};
%\node[box=1] (p-89-0) at (0,-89) {};
%\node[box=1] (p-89-1) at (1,-89) {};
%\node[box=0] (p-89-2) at (2,-89) {};
%\node[box=0] (p-89-3) at (3,-89) {};
%\node[box=0] (p-89-4) at (4,-89) {};
%\node[box=0] (p-89-5) at (5,-89) {};
%\node[box=0] (p-89-6) at (6,-89) {};
%\node[box=0] (p-89-7) at (7,-89) {};
%\node[box=1] (p-89-8) at (8,-89) {};
%\node[box=1] (p-89-9) at (9,-89) {};
%\node[box=0] (p-89-10) at (10,-89) {};
%\node[box=0] (p-89-11) at (11,-89) {};
%\node[box=0] (p-89-12) at (12,-89) {};
%\node[box=0] (p-89-13) at (13,-89) {};
%\node[box=0] (p-89-14) at (14,-89) {};
%\node[box=0] (p-89-15) at (15,-89) {};
%\node[box=1] (p-89-16) at (16,-89) {};
%\node[box=1] (p-89-17) at (17,-89) {};
%\node[box=0] (p-89-18) at (18,-89) {};
%\node[box=0] (p-89-19) at (19,-89) {};
%\node[box=0] (p-89-20) at (20,-89) {};
%\node[box=0] (p-89-21) at (21,-89) {};
%\node[box=0] (p-89-22) at (22,-89) {};
%\node[box=0] (p-89-23) at (23,-89) {};
%\node[box=1] (p-89-24) at (24,-89) {};
%\node[box=1] (p-89-25) at (25,-89) {};
%\node[box=0] (p-89-26) at (26,-89) {};
%\node[box=0] (p-89-27) at (27,-89) {};
%\node[box=0] (p-89-28) at (28,-89) {};
%\node[box=0] (p-89-29) at (29,-89) {};
%\node[box=0] (p-89-30) at (30,-89) {};
%\node[box=0] (p-89-31) at (31,-89) {};
%\node[box=0] (p-89-32) at (32,-89) {};
%\node[box=0] (p-89-33) at (33,-89) {};
%\node[box=0] (p-89-34) at (34,-89) {};
%\node[box=0] (p-89-35) at (35,-89) {};
%\node[box=0] (p-89-36) at (36,-89) {};
%\node[box=0] (p-89-37) at (37,-89) {};
%\node[box=0] (p-89-38) at (38,-89) {};
%\node[box=0] (p-89-39) at (39,-89) {};
%\node[box=0] (p-89-40) at (40,-89) {};
%\node[box=0] (p-89-41) at (41,-89) {};
%\node[box=0] (p-89-42) at (42,-89) {};
%\node[box=0] (p-89-43) at (43,-89) {};
%\node[box=0] (p-89-44) at (44,-89) {};
%\node[box=0] (p-89-45) at (45,-89) {};
%\node[box=0] (p-89-46) at (46,-89) {};
%\node[box=0] (p-89-47) at (47,-89) {};
%\node[box=0] (p-89-48) at (48,-89) {};
%\node[box=0] (p-89-49) at (49,-89) {};
%\node[box=0] (p-89-50) at (50,-89) {};
%\node[box=0] (p-89-51) at (51,-89) {};
%\node[box=0] (p-89-52) at (52,-89) {};
%\node[box=0] (p-89-53) at (53,-89) {};
%\node[box=0] (p-89-54) at (54,-89) {};
%\node[box=0] (p-89-55) at (55,-89) {};
%\node[box=0] (p-89-56) at (56,-89) {};
%\node[box=0] (p-89-57) at (57,-89) {};
%\node[box=0] (p-89-58) at (58,-89) {};
%\node[box=0] (p-89-59) at (59,-89) {};
%\node[box=0] (p-89-60) at (60,-89) {};
%\node[box=0] (p-89-61) at (61,-89) {};
%\node[box=0] (p-89-62) at (62,-89) {};
%\node[box=0] (p-89-63) at (63,-89) {};
%\node[box=1] (p-89-64) at (64,-89) {};
%\node[box=1] (p-89-65) at (65,-89) {};
%\node[box=0] (p-89-66) at (66,-89) {};
%\node[box=0] (p-89-67) at (67,-89) {};
%\node[box=0] (p-89-68) at (68,-89) {};
%\node[box=0] (p-89-69) at (69,-89) {};
%\node[box=0] (p-89-70) at (70,-89) {};
%\node[box=0] (p-89-71) at (71,-89) {};
%\node[box=1] (p-89-72) at (72,-89) {};
%\node[box=1] (p-89-73) at (73,-89) {};
%\node[box=0] (p-89-74) at (74,-89) {};
%\node[box=0] (p-89-75) at (75,-89) {};
%\node[box=0] (p-89-76) at (76,-89) {};
%\node[box=0] (p-89-77) at (77,-89) {};
%\node[box=0] (p-89-78) at (78,-89) {};
%\node[box=0] (p-89-79) at (79,-89) {};
%\node[box=1] (p-89-80) at (80,-89) {};
%\node[box=1] (p-89-81) at (81,-89) {};
%\node[box=0] (p-89-82) at (82,-89) {};
%\node[box=0] (p-89-83) at (83,-89) {};
%\node[box=0] (p-89-84) at (84,-89) {};
%\node[box=0] (p-89-85) at (85,-89) {};
%\node[box=0] (p-89-86) at (86,-89) {};
%\node[box=0] (p-89-87) at (87,-89) {};
%\node[box=1] (p-89-88) at (88,-89) {};
%\node[box=1] (p-89-89) at (89,-89) {};
%\node[box=1] (p-90-0) at (0,-90) {};
%\node[box=0] (p-90-1) at (1,-90) {};
%\node[box=1] (p-90-2) at (2,-90) {};
%\node[box=0] (p-90-3) at (3,-90) {};
%\node[box=0] (p-90-4) at (4,-90) {};
%\node[box=0] (p-90-5) at (5,-90) {};
%\node[box=0] (p-90-6) at (6,-90) {};
%\node[box=0] (p-90-7) at (7,-90) {};
%\node[box=1] (p-90-8) at (8,-90) {};
%\node[box=0] (p-90-9) at (9,-90) {};
%\node[box=1] (p-90-10) at (10,-90) {};
%\node[box=0] (p-90-11) at (11,-90) {};
%\node[box=0] (p-90-12) at (12,-90) {};
%\node[box=0] (p-90-13) at (13,-90) {};
%\node[box=0] (p-90-14) at (14,-90) {};
%\node[box=0] (p-90-15) at (15,-90) {};
%\node[box=1] (p-90-16) at (16,-90) {};
%\node[box=0] (p-90-17) at (17,-90) {};
%\node[box=1] (p-90-18) at (18,-90) {};
%\node[box=0] (p-90-19) at (19,-90) {};
%\node[box=0] (p-90-20) at (20,-90) {};
%\node[box=0] (p-90-21) at (21,-90) {};
%\node[box=0] (p-90-22) at (22,-90) {};
%\node[box=0] (p-90-23) at (23,-90) {};
%\node[box=1] (p-90-24) at (24,-90) {};
%\node[box=0] (p-90-25) at (25,-90) {};
%\node[box=1] (p-90-26) at (26,-90) {};
%\node[box=0] (p-90-27) at (27,-90) {};
%\node[box=0] (p-90-28) at (28,-90) {};
%\node[box=0] (p-90-29) at (29,-90) {};
%\node[box=0] (p-90-30) at (30,-90) {};
%\node[box=0] (p-90-31) at (31,-90) {};
%\node[box=0] (p-90-32) at (32,-90) {};
%\node[box=0] (p-90-33) at (33,-90) {};
%\node[box=0] (p-90-34) at (34,-90) {};
%\node[box=0] (p-90-35) at (35,-90) {};
%\node[box=0] (p-90-36) at (36,-90) {};
%\node[box=0] (p-90-37) at (37,-90) {};
%\node[box=0] (p-90-38) at (38,-90) {};
%\node[box=0] (p-90-39) at (39,-90) {};
%\node[box=0] (p-90-40) at (40,-90) {};
%\node[box=0] (p-90-41) at (41,-90) {};
%\node[box=0] (p-90-42) at (42,-90) {};
%\node[box=0] (p-90-43) at (43,-90) {};
%\node[box=0] (p-90-44) at (44,-90) {};
%\node[box=0] (p-90-45) at (45,-90) {};
%\node[box=0] (p-90-46) at (46,-90) {};
%\node[box=0] (p-90-47) at (47,-90) {};
%\node[box=0] (p-90-48) at (48,-90) {};
%\node[box=0] (p-90-49) at (49,-90) {};
%\node[box=0] (p-90-50) at (50,-90) {};
%\node[box=0] (p-90-51) at (51,-90) {};
%\node[box=0] (p-90-52) at (52,-90) {};
%\node[box=0] (p-90-53) at (53,-90) {};
%\node[box=0] (p-90-54) at (54,-90) {};
%\node[box=0] (p-90-55) at (55,-90) {};
%\node[box=0] (p-90-56) at (56,-90) {};
%\node[box=0] (p-90-57) at (57,-90) {};
%\node[box=0] (p-90-58) at (58,-90) {};
%\node[box=0] (p-90-59) at (59,-90) {};
%\node[box=0] (p-90-60) at (60,-90) {};
%\node[box=0] (p-90-61) at (61,-90) {};
%\node[box=0] (p-90-62) at (62,-90) {};
%\node[box=0] (p-90-63) at (63,-90) {};
%\node[box=1] (p-90-64) at (64,-90) {};
%\node[box=0] (p-90-65) at (65,-90) {};
%\node[box=1] (p-90-66) at (66,-90) {};
%\node[box=0] (p-90-67) at (67,-90) {};
%\node[box=0] (p-90-68) at (68,-90) {};
%\node[box=0] (p-90-69) at (69,-90) {};
%\node[box=0] (p-90-70) at (70,-90) {};
%\node[box=0] (p-90-71) at (71,-90) {};
%\node[box=1] (p-90-72) at (72,-90) {};
%\node[box=0] (p-90-73) at (73,-90) {};
%\node[box=1] (p-90-74) at (74,-90) {};
%\node[box=0] (p-90-75) at (75,-90) {};
%\node[box=0] (p-90-76) at (76,-90) {};
%\node[box=0] (p-90-77) at (77,-90) {};
%\node[box=0] (p-90-78) at (78,-90) {};
%\node[box=0] (p-90-79) at (79,-90) {};
%\node[box=1] (p-90-80) at (80,-90) {};
%\node[box=0] (p-90-81) at (81,-90) {};
%\node[box=1] (p-90-82) at (82,-90) {};
%\node[box=0] (p-90-83) at (83,-90) {};
%\node[box=0] (p-90-84) at (84,-90) {};
%\node[box=0] (p-90-85) at (85,-90) {};
%\node[box=0] (p-90-86) at (86,-90) {};
%\node[box=0] (p-90-87) at (87,-90) {};
%\node[box=1] (p-90-88) at (88,-90) {};
%\node[box=0] (p-90-89) at (89,-90) {};
%\node[box=1] (p-90-90) at (90,-90) {};
%\node[box=1] (p-91-0) at (0,-91) {};
%\node[box=1] (p-91-1) at (1,-91) {};
%\node[box=1] (p-91-2) at (2,-91) {};
%\node[box=1] (p-91-3) at (3,-91) {};
%\node[box=0] (p-91-4) at (4,-91) {};
%\node[box=0] (p-91-5) at (5,-91) {};
%\node[box=0] (p-91-6) at (6,-91) {};
%\node[box=0] (p-91-7) at (7,-91) {};
%\node[box=1] (p-91-8) at (8,-91) {};
%\node[box=1] (p-91-9) at (9,-91) {};
%\node[box=1] (p-91-10) at (10,-91) {};
%\node[box=1] (p-91-11) at (11,-91) {};
%\node[box=0] (p-91-12) at (12,-91) {};
%\node[box=0] (p-91-13) at (13,-91) {};
%\node[box=0] (p-91-14) at (14,-91) {};
%\node[box=0] (p-91-15) at (15,-91) {};
%\node[box=1] (p-91-16) at (16,-91) {};
%\node[box=1] (p-91-17) at (17,-91) {};
%\node[box=1] (p-91-18) at (18,-91) {};
%\node[box=1] (p-91-19) at (19,-91) {};
%\node[box=0] (p-91-20) at (20,-91) {};
%\node[box=0] (p-91-21) at (21,-91) {};
%\node[box=0] (p-91-22) at (22,-91) {};
%\node[box=0] (p-91-23) at (23,-91) {};
%\node[box=1] (p-91-24) at (24,-91) {};
%\node[box=1] (p-91-25) at (25,-91) {};
%\node[box=1] (p-91-26) at (26,-91) {};
%\node[box=1] (p-91-27) at (27,-91) {};
%\node[box=0] (p-91-28) at (28,-91) {};
%\node[box=0] (p-91-29) at (29,-91) {};
%\node[box=0] (p-91-30) at (30,-91) {};
%\node[box=0] (p-91-31) at (31,-91) {};
%\node[box=0] (p-91-32) at (32,-91) {};
%\node[box=0] (p-91-33) at (33,-91) {};
%\node[box=0] (p-91-34) at (34,-91) {};
%\node[box=0] (p-91-35) at (35,-91) {};
%\node[box=0] (p-91-36) at (36,-91) {};
%\node[box=0] (p-91-37) at (37,-91) {};
%\node[box=0] (p-91-38) at (38,-91) {};
%\node[box=0] (p-91-39) at (39,-91) {};
%\node[box=0] (p-91-40) at (40,-91) {};
%\node[box=0] (p-91-41) at (41,-91) {};
%\node[box=0] (p-91-42) at (42,-91) {};
%\node[box=0] (p-91-43) at (43,-91) {};
%\node[box=0] (p-91-44) at (44,-91) {};
%\node[box=0] (p-91-45) at (45,-91) {};
%\node[box=0] (p-91-46) at (46,-91) {};
%\node[box=0] (p-91-47) at (47,-91) {};
%\node[box=0] (p-91-48) at (48,-91) {};
%\node[box=0] (p-91-49) at (49,-91) {};
%\node[box=0] (p-91-50) at (50,-91) {};
%\node[box=0] (p-91-51) at (51,-91) {};
%\node[box=0] (p-91-52) at (52,-91) {};
%\node[box=0] (p-91-53) at (53,-91) {};
%\node[box=0] (p-91-54) at (54,-91) {};
%\node[box=0] (p-91-55) at (55,-91) {};
%\node[box=0] (p-91-56) at (56,-91) {};
%\node[box=0] (p-91-57) at (57,-91) {};
%\node[box=0] (p-91-58) at (58,-91) {};
%\node[box=0] (p-91-59) at (59,-91) {};
%\node[box=0] (p-91-60) at (60,-91) {};
%\node[box=0] (p-91-61) at (61,-91) {};
%\node[box=0] (p-91-62) at (62,-91) {};
%\node[box=0] (p-91-63) at (63,-91) {};
%\node[box=1] (p-91-64) at (64,-91) {};
%\node[box=1] (p-91-65) at (65,-91) {};
%\node[box=1] (p-91-66) at (66,-91) {};
%\node[box=1] (p-91-67) at (67,-91) {};
%\node[box=0] (p-91-68) at (68,-91) {};
%\node[box=0] (p-91-69) at (69,-91) {};
%\node[box=0] (p-91-70) at (70,-91) {};
%\node[box=0] (p-91-71) at (71,-91) {};
%\node[box=1] (p-91-72) at (72,-91) {};
%\node[box=1] (p-91-73) at (73,-91) {};
%\node[box=1] (p-91-74) at (74,-91) {};
%\node[box=1] (p-91-75) at (75,-91) {};
%\node[box=0] (p-91-76) at (76,-91) {};
%\node[box=0] (p-91-77) at (77,-91) {};
%\node[box=0] (p-91-78) at (78,-91) {};
%\node[box=0] (p-91-79) at (79,-91) {};
%\node[box=1] (p-91-80) at (80,-91) {};
%\node[box=1] (p-91-81) at (81,-91) {};
%\node[box=1] (p-91-82) at (82,-91) {};
%\node[box=1] (p-91-83) at (83,-91) {};
%\node[box=0] (p-91-84) at (84,-91) {};
%\node[box=0] (p-91-85) at (85,-91) {};
%\node[box=0] (p-91-86) at (86,-91) {};
%\node[box=0] (p-91-87) at (87,-91) {};
%\node[box=1] (p-91-88) at (88,-91) {};
%\node[box=1] (p-91-89) at (89,-91) {};
%\node[box=1] (p-91-90) at (90,-91) {};
%\node[box=1] (p-91-91) at (91,-91) {};
%\node[box=1] (p-92-0) at (0,-92) {};
%\node[box=0] (p-92-1) at (1,-92) {};
%\node[box=0] (p-92-2) at (2,-92) {};
%\node[box=0] (p-92-3) at (3,-92) {};
%\node[box=1] (p-92-4) at (4,-92) {};
%\node[box=0] (p-92-5) at (5,-92) {};
%\node[box=0] (p-92-6) at (6,-92) {};
%\node[box=0] (p-92-7) at (7,-92) {};
%\node[box=1] (p-92-8) at (8,-92) {};
%\node[box=0] (p-92-9) at (9,-92) {};
%\node[box=0] (p-92-10) at (10,-92) {};
%\node[box=0] (p-92-11) at (11,-92) {};
%\node[box=1] (p-92-12) at (12,-92) {};
%\node[box=0] (p-92-13) at (13,-92) {};
%\node[box=0] (p-92-14) at (14,-92) {};
%\node[box=0] (p-92-15) at (15,-92) {};
%\node[box=1] (p-92-16) at (16,-92) {};
%\node[box=0] (p-92-17) at (17,-92) {};
%\node[box=0] (p-92-18) at (18,-92) {};
%\node[box=0] (p-92-19) at (19,-92) {};
%\node[box=1] (p-92-20) at (20,-92) {};
%\node[box=0] (p-92-21) at (21,-92) {};
%\node[box=0] (p-92-22) at (22,-92) {};
%\node[box=0] (p-92-23) at (23,-92) {};
%\node[box=1] (p-92-24) at (24,-92) {};
%\node[box=0] (p-92-25) at (25,-92) {};
%\node[box=0] (p-92-26) at (26,-92) {};
%\node[box=0] (p-92-27) at (27,-92) {};
%\node[box=1] (p-92-28) at (28,-92) {};
%\node[box=0] (p-92-29) at (29,-92) {};
%\node[box=0] (p-92-30) at (30,-92) {};
%\node[box=0] (p-92-31) at (31,-92) {};
%\node[box=0] (p-92-32) at (32,-92) {};
%\node[box=0] (p-92-33) at (33,-92) {};
%\node[box=0] (p-92-34) at (34,-92) {};
%\node[box=0] (p-92-35) at (35,-92) {};
%\node[box=0] (p-92-36) at (36,-92) {};
%\node[box=0] (p-92-37) at (37,-92) {};
%\node[box=0] (p-92-38) at (38,-92) {};
%\node[box=0] (p-92-39) at (39,-92) {};
%\node[box=0] (p-92-40) at (40,-92) {};
%\node[box=0] (p-92-41) at (41,-92) {};
%\node[box=0] (p-92-42) at (42,-92) {};
%\node[box=0] (p-92-43) at (43,-92) {};
%\node[box=0] (p-92-44) at (44,-92) {};
%\node[box=0] (p-92-45) at (45,-92) {};
%\node[box=0] (p-92-46) at (46,-92) {};
%\node[box=0] (p-92-47) at (47,-92) {};
%\node[box=0] (p-92-48) at (48,-92) {};
%\node[box=0] (p-92-49) at (49,-92) {};
%\node[box=0] (p-92-50) at (50,-92) {};
%\node[box=0] (p-92-51) at (51,-92) {};
%\node[box=0] (p-92-52) at (52,-92) {};
%\node[box=0] (p-92-53) at (53,-92) {};
%\node[box=0] (p-92-54) at (54,-92) {};
%\node[box=0] (p-92-55) at (55,-92) {};
%\node[box=0] (p-92-56) at (56,-92) {};
%\node[box=0] (p-92-57) at (57,-92) {};
%\node[box=0] (p-92-58) at (58,-92) {};
%\node[box=0] (p-92-59) at (59,-92) {};
%\node[box=0] (p-92-60) at (60,-92) {};
%\node[box=0] (p-92-61) at (61,-92) {};
%\node[box=0] (p-92-62) at (62,-92) {};
%\node[box=0] (p-92-63) at (63,-92) {};
%\node[box=1] (p-92-64) at (64,-92) {};
%\node[box=0] (p-92-65) at (65,-92) {};
%\node[box=0] (p-92-66) at (66,-92) {};
%\node[box=0] (p-92-67) at (67,-92) {};
%\node[box=1] (p-92-68) at (68,-92) {};
%\node[box=0] (p-92-69) at (69,-92) {};
%\node[box=0] (p-92-70) at (70,-92) {};
%\node[box=0] (p-92-71) at (71,-92) {};
%\node[box=1] (p-92-72) at (72,-92) {};
%\node[box=0] (p-92-73) at (73,-92) {};
%\node[box=0] (p-92-74) at (74,-92) {};
%\node[box=0] (p-92-75) at (75,-92) {};
%\node[box=1] (p-92-76) at (76,-92) {};
%\node[box=0] (p-92-77) at (77,-92) {};
%\node[box=0] (p-92-78) at (78,-92) {};
%\node[box=0] (p-92-79) at (79,-92) {};
%\node[box=1] (p-92-80) at (80,-92) {};
%\node[box=0] (p-92-81) at (81,-92) {};
%\node[box=0] (p-92-82) at (82,-92) {};
%\node[box=0] (p-92-83) at (83,-92) {};
%\node[box=1] (p-92-84) at (84,-92) {};
%\node[box=0] (p-92-85) at (85,-92) {};
%\node[box=0] (p-92-86) at (86,-92) {};
%\node[box=0] (p-92-87) at (87,-92) {};
%\node[box=1] (p-92-88) at (88,-92) {};
%\node[box=0] (p-92-89) at (89,-92) {};
%\node[box=0] (p-92-90) at (90,-92) {};
%\node[box=0] (p-92-91) at (91,-92) {};
%\node[box=1] (p-92-92) at (92,-92) {};
%\node[box=1] (p-93-0) at (0,-93) {};
%\node[box=1] (p-93-1) at (1,-93) {};
%\node[box=0] (p-93-2) at (2,-93) {};
%\node[box=0] (p-93-3) at (3,-93) {};
%\node[box=1] (p-93-4) at (4,-93) {};
%\node[box=1] (p-93-5) at (5,-93) {};
%\node[box=0] (p-93-6) at (6,-93) {};
%\node[box=0] (p-93-7) at (7,-93) {};
%\node[box=1] (p-93-8) at (8,-93) {};
%\node[box=1] (p-93-9) at (9,-93) {};
%\node[box=0] (p-93-10) at (10,-93) {};
%\node[box=0] (p-93-11) at (11,-93) {};
%\node[box=1] (p-93-12) at (12,-93) {};
%\node[box=1] (p-93-13) at (13,-93) {};
%\node[box=0] (p-93-14) at (14,-93) {};
%\node[box=0] (p-93-15) at (15,-93) {};
%\node[box=1] (p-93-16) at (16,-93) {};
%\node[box=1] (p-93-17) at (17,-93) {};
%\node[box=0] (p-93-18) at (18,-93) {};
%\node[box=0] (p-93-19) at (19,-93) {};
%\node[box=1] (p-93-20) at (20,-93) {};
%\node[box=1] (p-93-21) at (21,-93) {};
%\node[box=0] (p-93-22) at (22,-93) {};
%\node[box=0] (p-93-23) at (23,-93) {};
%\node[box=1] (p-93-24) at (24,-93) {};
%\node[box=1] (p-93-25) at (25,-93) {};
%\node[box=0] (p-93-26) at (26,-93) {};
%\node[box=0] (p-93-27) at (27,-93) {};
%\node[box=1] (p-93-28) at (28,-93) {};
%\node[box=1] (p-93-29) at (29,-93) {};
%\node[box=0] (p-93-30) at (30,-93) {};
%\node[box=0] (p-93-31) at (31,-93) {};
%\node[box=0] (p-93-32) at (32,-93) {};
%\node[box=0] (p-93-33) at (33,-93) {};
%\node[box=0] (p-93-34) at (34,-93) {};
%\node[box=0] (p-93-35) at (35,-93) {};
%\node[box=0] (p-93-36) at (36,-93) {};
%\node[box=0] (p-93-37) at (37,-93) {};
%\node[box=0] (p-93-38) at (38,-93) {};
%\node[box=0] (p-93-39) at (39,-93) {};
%\node[box=0] (p-93-40) at (40,-93) {};
%\node[box=0] (p-93-41) at (41,-93) {};
%\node[box=0] (p-93-42) at (42,-93) {};
%\node[box=0] (p-93-43) at (43,-93) {};
%\node[box=0] (p-93-44) at (44,-93) {};
%\node[box=0] (p-93-45) at (45,-93) {};
%\node[box=0] (p-93-46) at (46,-93) {};
%\node[box=0] (p-93-47) at (47,-93) {};
%\node[box=0] (p-93-48) at (48,-93) {};
%\node[box=0] (p-93-49) at (49,-93) {};
%\node[box=0] (p-93-50) at (50,-93) {};
%\node[box=0] (p-93-51) at (51,-93) {};
%\node[box=0] (p-93-52) at (52,-93) {};
%\node[box=0] (p-93-53) at (53,-93) {};
%\node[box=0] (p-93-54) at (54,-93) {};
%\node[box=0] (p-93-55) at (55,-93) {};
%\node[box=0] (p-93-56) at (56,-93) {};
%\node[box=0] (p-93-57) at (57,-93) {};
%\node[box=0] (p-93-58) at (58,-93) {};
%\node[box=0] (p-93-59) at (59,-93) {};
%\node[box=0] (p-93-60) at (60,-93) {};
%\node[box=0] (p-93-61) at (61,-93) {};
%\node[box=0] (p-93-62) at (62,-93) {};
%\node[box=0] (p-93-63) at (63,-93) {};
%\node[box=1] (p-93-64) at (64,-93) {};
%\node[box=1] (p-93-65) at (65,-93) {};
%\node[box=0] (p-93-66) at (66,-93) {};
%\node[box=0] (p-93-67) at (67,-93) {};
%\node[box=1] (p-93-68) at (68,-93) {};
%\node[box=1] (p-93-69) at (69,-93) {};
%\node[box=0] (p-93-70) at (70,-93) {};
%\node[box=0] (p-93-71) at (71,-93) {};
%\node[box=1] (p-93-72) at (72,-93) {};
%\node[box=1] (p-93-73) at (73,-93) {};
%\node[box=0] (p-93-74) at (74,-93) {};
%\node[box=0] (p-93-75) at (75,-93) {};
%\node[box=1] (p-93-76) at (76,-93) {};
%\node[box=1] (p-93-77) at (77,-93) {};
%\node[box=0] (p-93-78) at (78,-93) {};
%\node[box=0] (p-93-79) at (79,-93) {};
%\node[box=1] (p-93-80) at (80,-93) {};
%\node[box=1] (p-93-81) at (81,-93) {};
%\node[box=0] (p-93-82) at (82,-93) {};
%\node[box=0] (p-93-83) at (83,-93) {};
%\node[box=1] (p-93-84) at (84,-93) {};
%\node[box=1] (p-93-85) at (85,-93) {};
%\node[box=0] (p-93-86) at (86,-93) {};
%\node[box=0] (p-93-87) at (87,-93) {};
%\node[box=1] (p-93-88) at (88,-93) {};
%\node[box=1] (p-93-89) at (89,-93) {};
%\node[box=0] (p-93-90) at (90,-93) {};
%\node[box=0] (p-93-91) at (91,-93) {};
%\node[box=1] (p-93-92) at (92,-93) {};
%\node[box=1] (p-93-93) at (93,-93) {};
%\node[box=1] (p-94-0) at (0,-94) {};
%\node[box=0] (p-94-1) at (1,-94) {};
%\node[box=1] (p-94-2) at (2,-94) {};
%\node[box=0] (p-94-3) at (3,-94) {};
%\node[box=1] (p-94-4) at (4,-94) {};
%\node[box=0] (p-94-5) at (5,-94) {};
%\node[box=1] (p-94-6) at (6,-94) {};
%\node[box=0] (p-94-7) at (7,-94) {};
%\node[box=1] (p-94-8) at (8,-94) {};
%\node[box=0] (p-94-9) at (9,-94) {};
%\node[box=1] (p-94-10) at (10,-94) {};
%\node[box=0] (p-94-11) at (11,-94) {};
%\node[box=1] (p-94-12) at (12,-94) {};
%\node[box=0] (p-94-13) at (13,-94) {};
%\node[box=1] (p-94-14) at (14,-94) {};
%\node[box=0] (p-94-15) at (15,-94) {};
%\node[box=1] (p-94-16) at (16,-94) {};
%\node[box=0] (p-94-17) at (17,-94) {};
%\node[box=1] (p-94-18) at (18,-94) {};
%\node[box=0] (p-94-19) at (19,-94) {};
%\node[box=1] (p-94-20) at (20,-94) {};
%\node[box=0] (p-94-21) at (21,-94) {};
%\node[box=1] (p-94-22) at (22,-94) {};
%\node[box=0] (p-94-23) at (23,-94) {};
%\node[box=1] (p-94-24) at (24,-94) {};
%\node[box=0] (p-94-25) at (25,-94) {};
%\node[box=1] (p-94-26) at (26,-94) {};
%\node[box=0] (p-94-27) at (27,-94) {};
%\node[box=1] (p-94-28) at (28,-94) {};
%\node[box=0] (p-94-29) at (29,-94) {};
%\node[box=1] (p-94-30) at (30,-94) {};
%\node[box=0] (p-94-31) at (31,-94) {};
%\node[box=0] (p-94-32) at (32,-94) {};
%\node[box=0] (p-94-33) at (33,-94) {};
%\node[box=0] (p-94-34) at (34,-94) {};
%\node[box=0] (p-94-35) at (35,-94) {};
%\node[box=0] (p-94-36) at (36,-94) {};
%\node[box=0] (p-94-37) at (37,-94) {};
%\node[box=0] (p-94-38) at (38,-94) {};
%\node[box=0] (p-94-39) at (39,-94) {};
%\node[box=0] (p-94-40) at (40,-94) {};
%\node[box=0] (p-94-41) at (41,-94) {};
%\node[box=0] (p-94-42) at (42,-94) {};
%\node[box=0] (p-94-43) at (43,-94) {};
%\node[box=0] (p-94-44) at (44,-94) {};
%\node[box=0] (p-94-45) at (45,-94) {};
%\node[box=0] (p-94-46) at (46,-94) {};
%\node[box=0] (p-94-47) at (47,-94) {};
%\node[box=0] (p-94-48) at (48,-94) {};
%\node[box=0] (p-94-49) at (49,-94) {};
%\node[box=0] (p-94-50) at (50,-94) {};
%\node[box=0] (p-94-51) at (51,-94) {};
%\node[box=0] (p-94-52) at (52,-94) {};
%\node[box=0] (p-94-53) at (53,-94) {};
%\node[box=0] (p-94-54) at (54,-94) {};
%\node[box=0] (p-94-55) at (55,-94) {};
%\node[box=0] (p-94-56) at (56,-94) {};
%\node[box=0] (p-94-57) at (57,-94) {};
%\node[box=0] (p-94-58) at (58,-94) {};
%\node[box=0] (p-94-59) at (59,-94) {};
%\node[box=0] (p-94-60) at (60,-94) {};
%\node[box=0] (p-94-61) at (61,-94) {};
%\node[box=0] (p-94-62) at (62,-94) {};
%\node[box=0] (p-94-63) at (63,-94) {};
%\node[box=1] (p-94-64) at (64,-94) {};
%\node[box=0] (p-94-65) at (65,-94) {};
%\node[box=1] (p-94-66) at (66,-94) {};
%\node[box=0] (p-94-67) at (67,-94) {};
%\node[box=1] (p-94-68) at (68,-94) {};
%\node[box=0] (p-94-69) at (69,-94) {};
%\node[box=1] (p-94-70) at (70,-94) {};
%\node[box=0] (p-94-71) at (71,-94) {};
%\node[box=1] (p-94-72) at (72,-94) {};
%\node[box=0] (p-94-73) at (73,-94) {};
%\node[box=1] (p-94-74) at (74,-94) {};
%\node[box=0] (p-94-75) at (75,-94) {};
%\node[box=1] (p-94-76) at (76,-94) {};
%\node[box=0] (p-94-77) at (77,-94) {};
%\node[box=1] (p-94-78) at (78,-94) {};
%\node[box=0] (p-94-79) at (79,-94) {};
%\node[box=1] (p-94-80) at (80,-94) {};
%\node[box=0] (p-94-81) at (81,-94) {};
%\node[box=1] (p-94-82) at (82,-94) {};
%\node[box=0] (p-94-83) at (83,-94) {};
%\node[box=1] (p-94-84) at (84,-94) {};
%\node[box=0] (p-94-85) at (85,-94) {};
%\node[box=1] (p-94-86) at (86,-94) {};
%\node[box=0] (p-94-87) at (87,-94) {};
%\node[box=1] (p-94-88) at (88,-94) {};
%\node[box=0] (p-94-89) at (89,-94) {};
%\node[box=1] (p-94-90) at (90,-94) {};
%\node[box=0] (p-94-91) at (91,-94) {};
%\node[box=1] (p-94-92) at (92,-94) {};
%\node[box=0] (p-94-93) at (93,-94) {};
%\node[box=1] (p-94-94) at (94,-94) {};
%\node[box=1] (p-95-0) at (0,-95) {};
%\node[box=1] (p-95-1) at (1,-95) {};
%\node[box=1] (p-95-2) at (2,-95) {};
%\node[box=1] (p-95-3) at (3,-95) {};
%\node[box=1] (p-95-4) at (4,-95) {};
%\node[box=1] (p-95-5) at (5,-95) {};
%\node[box=1] (p-95-6) at (6,-95) {};
%\node[box=1] (p-95-7) at (7,-95) {};
%\node[box=1] (p-95-8) at (8,-95) {};
%\node[box=1] (p-95-9) at (9,-95) {};
%\node[box=1] (p-95-10) at (10,-95) {};
%\node[box=1] (p-95-11) at (11,-95) {};
%\node[box=1] (p-95-12) at (12,-95) {};
%\node[box=1] (p-95-13) at (13,-95) {};
%\node[box=1] (p-95-14) at (14,-95) {};
%\node[box=1] (p-95-15) at (15,-95) {};
%\node[box=1] (p-95-16) at (16,-95) {};
%\node[box=1] (p-95-17) at (17,-95) {};
%\node[box=1] (p-95-18) at (18,-95) {};
%\node[box=1] (p-95-19) at (19,-95) {};
%\node[box=1] (p-95-20) at (20,-95) {};
%\node[box=1] (p-95-21) at (21,-95) {};
%\node[box=1] (p-95-22) at (22,-95) {};
%\node[box=1] (p-95-23) at (23,-95) {};
%\node[box=1] (p-95-24) at (24,-95) {};
%\node[box=1] (p-95-25) at (25,-95) {};
%\node[box=1] (p-95-26) at (26,-95) {};
%\node[box=1] (p-95-27) at (27,-95) {};
%\node[box=1] (p-95-28) at (28,-95) {};
%\node[box=1] (p-95-29) at (29,-95) {};
%\node[box=1] (p-95-30) at (30,-95) {};
%\node[box=1] (p-95-31) at (31,-95) {};
%\node[box=0] (p-95-32) at (32,-95) {};
%\node[box=0] (p-95-33) at (33,-95) {};
%\node[box=0] (p-95-34) at (34,-95) {};
%\node[box=0] (p-95-35) at (35,-95) {};
%\node[box=0] (p-95-36) at (36,-95) {};
%\node[box=0] (p-95-37) at (37,-95) {};
%\node[box=0] (p-95-38) at (38,-95) {};
%\node[box=0] (p-95-39) at (39,-95) {};
%\node[box=0] (p-95-40) at (40,-95) {};
%\node[box=0] (p-95-41) at (41,-95) {};
%\node[box=0] (p-95-42) at (42,-95) {};
%\node[box=0] (p-95-43) at (43,-95) {};
%\node[box=0] (p-95-44) at (44,-95) {};
%\node[box=0] (p-95-45) at (45,-95) {};
%\node[box=0] (p-95-46) at (46,-95) {};
%\node[box=0] (p-95-47) at (47,-95) {};
%\node[box=0] (p-95-48) at (48,-95) {};
%\node[box=0] (p-95-49) at (49,-95) {};
%\node[box=0] (p-95-50) at (50,-95) {};
%\node[box=0] (p-95-51) at (51,-95) {};
%\node[box=0] (p-95-52) at (52,-95) {};
%\node[box=0] (p-95-53) at (53,-95) {};
%\node[box=0] (p-95-54) at (54,-95) {};
%\node[box=0] (p-95-55) at (55,-95) {};
%\node[box=0] (p-95-56) at (56,-95) {};
%\node[box=0] (p-95-57) at (57,-95) {};
%\node[box=0] (p-95-58) at (58,-95) {};
%\node[box=0] (p-95-59) at (59,-95) {};
%\node[box=0] (p-95-60) at (60,-95) {};
%\node[box=0] (p-95-61) at (61,-95) {};
%\node[box=0] (p-95-62) at (62,-95) {};
%\node[box=0] (p-95-63) at (63,-95) {};
%\node[box=1] (p-95-64) at (64,-95) {};
%\node[box=1] (p-95-65) at (65,-95) {};
%\node[box=1] (p-95-66) at (66,-95) {};
%\node[box=1] (p-95-67) at (67,-95) {};
%\node[box=1] (p-95-68) at (68,-95) {};
%\node[box=1] (p-95-69) at (69,-95) {};
%\node[box=1] (p-95-70) at (70,-95) {};
%\node[box=1] (p-95-71) at (71,-95) {};
%\node[box=1] (p-95-72) at (72,-95) {};
%\node[box=1] (p-95-73) at (73,-95) {};
%\node[box=1] (p-95-74) at (74,-95) {};
%\node[box=1] (p-95-75) at (75,-95) {};
%\node[box=1] (p-95-76) at (76,-95) {};
%\node[box=1] (p-95-77) at (77,-95) {};
%\node[box=1] (p-95-78) at (78,-95) {};
%\node[box=1] (p-95-79) at (79,-95) {};
%\node[box=1] (p-95-80) at (80,-95) {};
%\node[box=1] (p-95-81) at (81,-95) {};
%\node[box=1] (p-95-82) at (82,-95) {};
%\node[box=1] (p-95-83) at (83,-95) {};
%\node[box=1] (p-95-84) at (84,-95) {};
%\node[box=1] (p-95-85) at (85,-95) {};
%\node[box=1] (p-95-86) at (86,-95) {};
%\node[box=1] (p-95-87) at (87,-95) {};
%\node[box=1] (p-95-88) at (88,-95) {};
%\node[box=1] (p-95-89) at (89,-95) {};
%\node[box=1] (p-95-90) at (90,-95) {};
%\node[box=1] (p-95-91) at (91,-95) {};
%\node[box=1] (p-95-92) at (92,-95) {};
%\node[box=1] (p-95-93) at (93,-95) {};
%\node[box=1] (p-95-94) at (94,-95) {};
%\node[box=1] (p-95-95) at (95,-95) {};
%\node[box=1] (p-96-0) at (0,-96) {};
%\node[box=0] (p-96-1) at (1,-96) {};
%\node[box=0] (p-96-2) at (2,-96) {};
%\node[box=0] (p-96-3) at (3,-96) {};
%\node[box=0] (p-96-4) at (4,-96) {};
%\node[box=0] (p-96-5) at (5,-96) {};
%\node[box=0] (p-96-6) at (6,-96) {};
%\node[box=0] (p-96-7) at (7,-96) {};
%\node[box=0] (p-96-8) at (8,-96) {};
%\node[box=0] (p-96-9) at (9,-96) {};
%\node[box=0] (p-96-10) at (10,-96) {};
%\node[box=0] (p-96-11) at (11,-96) {};
%\node[box=0] (p-96-12) at (12,-96) {};
%\node[box=0] (p-96-13) at (13,-96) {};
%\node[box=0] (p-96-14) at (14,-96) {};
%\node[box=0] (p-96-15) at (15,-96) {};
%\node[box=0] (p-96-16) at (16,-96) {};
%\node[box=0] (p-96-17) at (17,-96) {};
%\node[box=0] (p-96-18) at (18,-96) {};
%\node[box=0] (p-96-19) at (19,-96) {};
%\node[box=0] (p-96-20) at (20,-96) {};
%\node[box=0] (p-96-21) at (21,-96) {};
%\node[box=0] (p-96-22) at (22,-96) {};
%\node[box=0] (p-96-23) at (23,-96) {};
%\node[box=0] (p-96-24) at (24,-96) {};
%\node[box=0] (p-96-25) at (25,-96) {};
%\node[box=0] (p-96-26) at (26,-96) {};
%\node[box=0] (p-96-27) at (27,-96) {};
%\node[box=0] (p-96-28) at (28,-96) {};
%\node[box=0] (p-96-29) at (29,-96) {};
%\node[box=0] (p-96-30) at (30,-96) {};
%\node[box=0] (p-96-31) at (31,-96) {};
%\node[box=1] (p-96-32) at (32,-96) {};
%\node[box=0] (p-96-33) at (33,-96) {};
%\node[box=0] (p-96-34) at (34,-96) {};
%\node[box=0] (p-96-35) at (35,-96) {};
%\node[box=0] (p-96-36) at (36,-96) {};
%\node[box=0] (p-96-37) at (37,-96) {};
%\node[box=0] (p-96-38) at (38,-96) {};
%\node[box=0] (p-96-39) at (39,-96) {};
%\node[box=0] (p-96-40) at (40,-96) {};
%\node[box=0] (p-96-41) at (41,-96) {};
%\node[box=0] (p-96-42) at (42,-96) {};
%\node[box=0] (p-96-43) at (43,-96) {};
%\node[box=0] (p-96-44) at (44,-96) {};
%\node[box=0] (p-96-45) at (45,-96) {};
%\node[box=0] (p-96-46) at (46,-96) {};
%\node[box=0] (p-96-47) at (47,-96) {};
%\node[box=0] (p-96-48) at (48,-96) {};
%\node[box=0] (p-96-49) at (49,-96) {};
%\node[box=0] (p-96-50) at (50,-96) {};
%\node[box=0] (p-96-51) at (51,-96) {};
%\node[box=0] (p-96-52) at (52,-96) {};
%\node[box=0] (p-96-53) at (53,-96) {};
%\node[box=0] (p-96-54) at (54,-96) {};
%\node[box=0] (p-96-55) at (55,-96) {};
%\node[box=0] (p-96-56) at (56,-96) {};
%\node[box=0] (p-96-57) at (57,-96) {};
%\node[box=0] (p-96-58) at (58,-96) {};
%\node[box=0] (p-96-59) at (59,-96) {};
%\node[box=0] (p-96-60) at (60,-96) {};
%\node[box=0] (p-96-61) at (61,-96) {};
%\node[box=0] (p-96-62) at (62,-96) {};
%\node[box=0] (p-96-63) at (63,-96) {};
%\node[box=1] (p-96-64) at (64,-96) {};
%\node[box=0] (p-96-65) at (65,-96) {};
%\node[box=0] (p-96-66) at (66,-96) {};
%\node[box=0] (p-96-67) at (67,-96) {};
%\node[box=0] (p-96-68) at (68,-96) {};
%\node[box=0] (p-96-69) at (69,-96) {};
%\node[box=0] (p-96-70) at (70,-96) {};
%\node[box=0] (p-96-71) at (71,-96) {};
%\node[box=0] (p-96-72) at (72,-96) {};
%\node[box=0] (p-96-73) at (73,-96) {};
%\node[box=0] (p-96-74) at (74,-96) {};
%\node[box=0] (p-96-75) at (75,-96) {};
%\node[box=0] (p-96-76) at (76,-96) {};
%\node[box=0] (p-96-77) at (77,-96) {};
%\node[box=0] (p-96-78) at (78,-96) {};
%\node[box=0] (p-96-79) at (79,-96) {};
%\node[box=0] (p-96-80) at (80,-96) {};
%\node[box=0] (p-96-81) at (81,-96) {};
%\node[box=0] (p-96-82) at (82,-96) {};
%\node[box=0] (p-96-83) at (83,-96) {};
%\node[box=0] (p-96-84) at (84,-96) {};
%\node[box=0] (p-96-85) at (85,-96) {};
%\node[box=0] (p-96-86) at (86,-96) {};
%\node[box=0] (p-96-87) at (87,-96) {};
%\node[box=0] (p-96-88) at (88,-96) {};
%\node[box=0] (p-96-89) at (89,-96) {};
%\node[box=0] (p-96-90) at (90,-96) {};
%\node[box=0] (p-96-91) at (91,-96) {};
%\node[box=0] (p-96-92) at (92,-96) {};
%\node[box=0] (p-96-93) at (93,-96) {};
%\node[box=0] (p-96-94) at (94,-96) {};
%\node[box=0] (p-96-95) at (95,-96) {};
%\node[box=1] (p-96-96) at (96,-96) {};
%\node[box=1] (p-97-0) at (0,-97) {};
%\node[box=1] (p-97-1) at (1,-97) {};
%\node[box=0] (p-97-2) at (2,-97) {};
%\node[box=0] (p-97-3) at (3,-97) {};
%\node[box=0] (p-97-4) at (4,-97) {};
%\node[box=0] (p-97-5) at (5,-97) {};
%\node[box=0] (p-97-6) at (6,-97) {};
%\node[box=0] (p-97-7) at (7,-97) {};
%\node[box=0] (p-97-8) at (8,-97) {};
%\node[box=0] (p-97-9) at (9,-97) {};
%\node[box=0] (p-97-10) at (10,-97) {};
%\node[box=0] (p-97-11) at (11,-97) {};
%\node[box=0] (p-97-12) at (12,-97) {};
%\node[box=0] (p-97-13) at (13,-97) {};
%\node[box=0] (p-97-14) at (14,-97) {};
%\node[box=0] (p-97-15) at (15,-97) {};
%\node[box=0] (p-97-16) at (16,-97) {};
%\node[box=0] (p-97-17) at (17,-97) {};
%\node[box=0] (p-97-18) at (18,-97) {};
%\node[box=0] (p-97-19) at (19,-97) {};
%\node[box=0] (p-97-20) at (20,-97) {};
%\node[box=0] (p-97-21) at (21,-97) {};
%\node[box=0] (p-97-22) at (22,-97) {};
%\node[box=0] (p-97-23) at (23,-97) {};
%\node[box=0] (p-97-24) at (24,-97) {};
%\node[box=0] (p-97-25) at (25,-97) {};
%\node[box=0] (p-97-26) at (26,-97) {};
%\node[box=0] (p-97-27) at (27,-97) {};
%\node[box=0] (p-97-28) at (28,-97) {};
%\node[box=0] (p-97-29) at (29,-97) {};
%\node[box=0] (p-97-30) at (30,-97) {};
%\node[box=0] (p-97-31) at (31,-97) {};
%\node[box=1] (p-97-32) at (32,-97) {};
%\node[box=1] (p-97-33) at (33,-97) {};
%\node[box=0] (p-97-34) at (34,-97) {};
%\node[box=0] (p-97-35) at (35,-97) {};
%\node[box=0] (p-97-36) at (36,-97) {};
%\node[box=0] (p-97-37) at (37,-97) {};
%\node[box=0] (p-97-38) at (38,-97) {};
%\node[box=0] (p-97-39) at (39,-97) {};
%\node[box=0] (p-97-40) at (40,-97) {};
%\node[box=0] (p-97-41) at (41,-97) {};
%\node[box=0] (p-97-42) at (42,-97) {};
%\node[box=0] (p-97-43) at (43,-97) {};
%\node[box=0] (p-97-44) at (44,-97) {};
%\node[box=0] (p-97-45) at (45,-97) {};
%\node[box=0] (p-97-46) at (46,-97) {};
%\node[box=0] (p-97-47) at (47,-97) {};
%\node[box=0] (p-97-48) at (48,-97) {};
%\node[box=0] (p-97-49) at (49,-97) {};
%\node[box=0] (p-97-50) at (50,-97) {};
%\node[box=0] (p-97-51) at (51,-97) {};
%\node[box=0] (p-97-52) at (52,-97) {};
%\node[box=0] (p-97-53) at (53,-97) {};
%\node[box=0] (p-97-54) at (54,-97) {};
%\node[box=0] (p-97-55) at (55,-97) {};
%\node[box=0] (p-97-56) at (56,-97) {};
%\node[box=0] (p-97-57) at (57,-97) {};
%\node[box=0] (p-97-58) at (58,-97) {};
%\node[box=0] (p-97-59) at (59,-97) {};
%\node[box=0] (p-97-60) at (60,-97) {};
%\node[box=0] (p-97-61) at (61,-97) {};
%\node[box=0] (p-97-62) at (62,-97) {};
%\node[box=0] (p-97-63) at (63,-97) {};
%\node[box=1] (p-97-64) at (64,-97) {};
%\node[box=1] (p-97-65) at (65,-97) {};
%\node[box=0] (p-97-66) at (66,-97) {};
%\node[box=0] (p-97-67) at (67,-97) {};
%\node[box=0] (p-97-68) at (68,-97) {};
%\node[box=0] (p-97-69) at (69,-97) {};
%\node[box=0] (p-97-70) at (70,-97) {};
%\node[box=0] (p-97-71) at (71,-97) {};
%\node[box=0] (p-97-72) at (72,-97) {};
%\node[box=0] (p-97-73) at (73,-97) {};
%\node[box=0] (p-97-74) at (74,-97) {};
%\node[box=0] (p-97-75) at (75,-97) {};
%\node[box=0] (p-97-76) at (76,-97) {};
%\node[box=0] (p-97-77) at (77,-97) {};
%\node[box=0] (p-97-78) at (78,-97) {};
%\node[box=0] (p-97-79) at (79,-97) {};
%\node[box=0] (p-97-80) at (80,-97) {};
%\node[box=0] (p-97-81) at (81,-97) {};
%\node[box=0] (p-97-82) at (82,-97) {};
%\node[box=0] (p-97-83) at (83,-97) {};
%\node[box=0] (p-97-84) at (84,-97) {};
%\node[box=0] (p-97-85) at (85,-97) {};
%\node[box=0] (p-97-86) at (86,-97) {};
%\node[box=0] (p-97-87) at (87,-97) {};
%\node[box=0] (p-97-88) at (88,-97) {};
%\node[box=0] (p-97-89) at (89,-97) {};
%\node[box=0] (p-97-90) at (90,-97) {};
%\node[box=0] (p-97-91) at (91,-97) {};
%\node[box=0] (p-97-92) at (92,-97) {};
%\node[box=0] (p-97-93) at (93,-97) {};
%\node[box=0] (p-97-94) at (94,-97) {};
%\node[box=0] (p-97-95) at (95,-97) {};
%\node[box=1] (p-97-96) at (96,-97) {};
%\node[box=1] (p-97-97) at (97,-97) {};
%\node[box=1] (p-98-0) at (0,-98) {};
%\node[box=0] (p-98-1) at (1,-98) {};
%\node[box=1] (p-98-2) at (2,-98) {};
%\node[box=0] (p-98-3) at (3,-98) {};
%\node[box=0] (p-98-4) at (4,-98) {};
%\node[box=0] (p-98-5) at (5,-98) {};
%\node[box=0] (p-98-6) at (6,-98) {};
%\node[box=0] (p-98-7) at (7,-98) {};
%\node[box=0] (p-98-8) at (8,-98) {};
%\node[box=0] (p-98-9) at (9,-98) {};
%\node[box=0] (p-98-10) at (10,-98) {};
%\node[box=0] (p-98-11) at (11,-98) {};
%\node[box=0] (p-98-12) at (12,-98) {};
%\node[box=0] (p-98-13) at (13,-98) {};
%\node[box=0] (p-98-14) at (14,-98) {};
%\node[box=0] (p-98-15) at (15,-98) {};
%\node[box=0] (p-98-16) at (16,-98) {};
%\node[box=0] (p-98-17) at (17,-98) {};
%\node[box=0] (p-98-18) at (18,-98) {};
%\node[box=0] (p-98-19) at (19,-98) {};
%\node[box=0] (p-98-20) at (20,-98) {};
%\node[box=0] (p-98-21) at (21,-98) {};
%\node[box=0] (p-98-22) at (22,-98) {};
%\node[box=0] (p-98-23) at (23,-98) {};
%\node[box=0] (p-98-24) at (24,-98) {};
%\node[box=0] (p-98-25) at (25,-98) {};
%\node[box=0] (p-98-26) at (26,-98) {};
%\node[box=0] (p-98-27) at (27,-98) {};
%\node[box=0] (p-98-28) at (28,-98) {};
%\node[box=0] (p-98-29) at (29,-98) {};
%\node[box=0] (p-98-30) at (30,-98) {};
%\node[box=0] (p-98-31) at (31,-98) {};
%\node[box=1] (p-98-32) at (32,-98) {};
%\node[box=0] (p-98-33) at (33,-98) {};
%\node[box=1] (p-98-34) at (34,-98) {};
%\node[box=0] (p-98-35) at (35,-98) {};
%\node[box=0] (p-98-36) at (36,-98) {};
%\node[box=0] (p-98-37) at (37,-98) {};
%\node[box=0] (p-98-38) at (38,-98) {};
%\node[box=0] (p-98-39) at (39,-98) {};
%\node[box=0] (p-98-40) at (40,-98) {};
%\node[box=0] (p-98-41) at (41,-98) {};
%\node[box=0] (p-98-42) at (42,-98) {};
%\node[box=0] (p-98-43) at (43,-98) {};
%\node[box=0] (p-98-44) at (44,-98) {};
%\node[box=0] (p-98-45) at (45,-98) {};
%\node[box=0] (p-98-46) at (46,-98) {};
%\node[box=0] (p-98-47) at (47,-98) {};
%\node[box=0] (p-98-48) at (48,-98) {};
%\node[box=0] (p-98-49) at (49,-98) {};
%\node[box=0] (p-98-50) at (50,-98) {};
%\node[box=0] (p-98-51) at (51,-98) {};
%\node[box=0] (p-98-52) at (52,-98) {};
%\node[box=0] (p-98-53) at (53,-98) {};
%\node[box=0] (p-98-54) at (54,-98) {};
%\node[box=0] (p-98-55) at (55,-98) {};
%\node[box=0] (p-98-56) at (56,-98) {};
%\node[box=0] (p-98-57) at (57,-98) {};
%\node[box=0] (p-98-58) at (58,-98) {};
%\node[box=0] (p-98-59) at (59,-98) {};
%\node[box=0] (p-98-60) at (60,-98) {};
%\node[box=0] (p-98-61) at (61,-98) {};
%\node[box=0] (p-98-62) at (62,-98) {};
%\node[box=0] (p-98-63) at (63,-98) {};
%\node[box=1] (p-98-64) at (64,-98) {};
%\node[box=0] (p-98-65) at (65,-98) {};
%\node[box=1] (p-98-66) at (66,-98) {};
%\node[box=0] (p-98-67) at (67,-98) {};
%\node[box=0] (p-98-68) at (68,-98) {};
%\node[box=0] (p-98-69) at (69,-98) {};
%\node[box=0] (p-98-70) at (70,-98) {};
%\node[box=0] (p-98-71) at (71,-98) {};
%\node[box=0] (p-98-72) at (72,-98) {};
%\node[box=0] (p-98-73) at (73,-98) {};
%\node[box=0] (p-98-74) at (74,-98) {};
%\node[box=0] (p-98-75) at (75,-98) {};
%\node[box=0] (p-98-76) at (76,-98) {};
%\node[box=0] (p-98-77) at (77,-98) {};
%\node[box=0] (p-98-78) at (78,-98) {};
%\node[box=0] (p-98-79) at (79,-98) {};
%\node[box=0] (p-98-80) at (80,-98) {};
%\node[box=0] (p-98-81) at (81,-98) {};
%\node[box=0] (p-98-82) at (82,-98) {};
%\node[box=0] (p-98-83) at (83,-98) {};
%\node[box=0] (p-98-84) at (84,-98) {};
%\node[box=0] (p-98-85) at (85,-98) {};
%\node[box=0] (p-98-86) at (86,-98) {};
%\node[box=0] (p-98-87) at (87,-98) {};
%\node[box=0] (p-98-88) at (88,-98) {};
%\node[box=0] (p-98-89) at (89,-98) {};
%\node[box=0] (p-98-90) at (90,-98) {};
%\node[box=0] (p-98-91) at (91,-98) {};
%\node[box=0] (p-98-92) at (92,-98) {};
%\node[box=0] (p-98-93) at (93,-98) {};
%\node[box=0] (p-98-94) at (94,-98) {};
%\node[box=0] (p-98-95) at (95,-98) {};
%\node[box=1] (p-98-96) at (96,-98) {};
%\node[box=0] (p-98-97) at (97,-98) {};
%\node[box=1] (p-98-98) at (98,-98) {};
%\node[box=1] (p-99-0) at (0,-99) {};
%\node[box=1] (p-99-1) at (1,-99) {};
%\node[box=1] (p-99-2) at (2,-99) {};
%\node[box=1] (p-99-3) at (3,-99) {};
%\node[box=0] (p-99-4) at (4,-99) {};
%\node[box=0] (p-99-5) at (5,-99) {};
%\node[box=0] (p-99-6) at (6,-99) {};
%\node[box=0] (p-99-7) at (7,-99) {};
%\node[box=0] (p-99-8) at (8,-99) {};
%\node[box=0] (p-99-9) at (9,-99) {};
%\node[box=0] (p-99-10) at (10,-99) {};
%\node[box=0] (p-99-11) at (11,-99) {};
%\node[box=0] (p-99-12) at (12,-99) {};
%\node[box=0] (p-99-13) at (13,-99) {};
%\node[box=0] (p-99-14) at (14,-99) {};
%\node[box=0] (p-99-15) at (15,-99) {};
%\node[box=0] (p-99-16) at (16,-99) {};
%\node[box=0] (p-99-17) at (17,-99) {};
%\node[box=0] (p-99-18) at (18,-99) {};
%\node[box=0] (p-99-19) at (19,-99) {};
%\node[box=0] (p-99-20) at (20,-99) {};
%\node[box=0] (p-99-21) at (21,-99) {};
%\node[box=0] (p-99-22) at (22,-99) {};
%\node[box=0] (p-99-23) at (23,-99) {};
%\node[box=0] (p-99-24) at (24,-99) {};
%\node[box=0] (p-99-25) at (25,-99) {};
%\node[box=0] (p-99-26) at (26,-99) {};
%\node[box=0] (p-99-27) at (27,-99) {};
%\node[box=0] (p-99-28) at (28,-99) {};
%\node[box=0] (p-99-29) at (29,-99) {};
%\node[box=0] (p-99-30) at (30,-99) {};
%\node[box=0] (p-99-31) at (31,-99) {};
%\node[box=1] (p-99-32) at (32,-99) {};
%\node[box=1] (p-99-33) at (33,-99) {};
%\node[box=1] (p-99-34) at (34,-99) {};
%\node[box=1] (p-99-35) at (35,-99) {};
%\node[box=0] (p-99-36) at (36,-99) {};
%\node[box=0] (p-99-37) at (37,-99) {};
%\node[box=0] (p-99-38) at (38,-99) {};
%\node[box=0] (p-99-39) at (39,-99) {};
%\node[box=0] (p-99-40) at (40,-99) {};
%\node[box=0] (p-99-41) at (41,-99) {};
%\node[box=0] (p-99-42) at (42,-99) {};
%\node[box=0] (p-99-43) at (43,-99) {};
%\node[box=0] (p-99-44) at (44,-99) {};
%\node[box=0] (p-99-45) at (45,-99) {};
%\node[box=0] (p-99-46) at (46,-99) {};
%\node[box=0] (p-99-47) at (47,-99) {};
%\node[box=0] (p-99-48) at (48,-99) {};
%\node[box=0] (p-99-49) at (49,-99) {};
%\node[box=0] (p-99-50) at (50,-99) {};
%\node[box=0] (p-99-51) at (51,-99) {};
%\node[box=0] (p-99-52) at (52,-99) {};
%\node[box=0] (p-99-53) at (53,-99) {};
%\node[box=0] (p-99-54) at (54,-99) {};
%\node[box=0] (p-99-55) at (55,-99) {};
%\node[box=0] (p-99-56) at (56,-99) {};
%\node[box=0] (p-99-57) at (57,-99) {};
%\node[box=0] (p-99-58) at (58,-99) {};
%\node[box=0] (p-99-59) at (59,-99) {};
%\node[box=0] (p-99-60) at (60,-99) {};
%\node[box=0] (p-99-61) at (61,-99) {};
%\node[box=0] (p-99-62) at (62,-99) {};
%\node[box=0] (p-99-63) at (63,-99) {};
%\node[box=1] (p-99-64) at (64,-99) {};
%\node[box=1] (p-99-65) at (65,-99) {};
%\node[box=1] (p-99-66) at (66,-99) {};
%\node[box=1] (p-99-67) at (67,-99) {};
%\node[box=0] (p-99-68) at (68,-99) {};
%\node[box=0] (p-99-69) at (69,-99) {};
%\node[box=0] (p-99-70) at (70,-99) {};
%\node[box=0] (p-99-71) at (71,-99) {};
%\node[box=0] (p-99-72) at (72,-99) {};
%\node[box=0] (p-99-73) at (73,-99) {};
%\node[box=0] (p-99-74) at (74,-99) {};
%\node[box=0] (p-99-75) at (75,-99) {};
%\node[box=0] (p-99-76) at (76,-99) {};
%\node[box=0] (p-99-77) at (77,-99) {};
%\node[box=0] (p-99-78) at (78,-99) {};
%\node[box=0] (p-99-79) at (79,-99) {};
%\node[box=0] (p-99-80) at (80,-99) {};
%\node[box=0] (p-99-81) at (81,-99) {};
%\node[box=0] (p-99-82) at (82,-99) {};
%\node[box=0] (p-99-83) at (83,-99) {};
%\node[box=0] (p-99-84) at (84,-99) {};
%\node[box=0] (p-99-85) at (85,-99) {};
%\node[box=0] (p-99-86) at (86,-99) {};
%\node[box=0] (p-99-87) at (87,-99) {};
%\node[box=0] (p-99-88) at (88,-99) {};
%\node[box=0] (p-99-89) at (89,-99) {};
%\node[box=0] (p-99-90) at (90,-99) {};
%\node[box=0] (p-99-91) at (91,-99) {};
%\node[box=0] (p-99-92) at (92,-99) {};
%\node[box=0] (p-99-93) at (93,-99) {};
%\node[box=0] (p-99-94) at (94,-99) {};
%\node[box=0] (p-99-95) at (95,-99) {};
%\node[box=1] (p-99-96) at (96,-99) {};
%\node[box=1] (p-99-97) at (97,-99) {};
%\node[box=1] (p-99-98) at (98,-99) {};
%\node[box=1] (p-99-99) at (99,-99) {};
%\node[box=1] (p-100-0) at (0,-100) {};
%\node[box=0] (p-100-1) at (1,-100) {};
%\node[box=0] (p-100-2) at (2,-100) {};
%\node[box=0] (p-100-3) at (3,-100) {};
%\node[box=1] (p-100-4) at (4,-100) {};
%\node[box=0] (p-100-5) at (5,-100) {};
%\node[box=0] (p-100-6) at (6,-100) {};
%\node[box=0] (p-100-7) at (7,-100) {};
%\node[box=0] (p-100-8) at (8,-100) {};
%\node[box=0] (p-100-9) at (9,-100) {};
%\node[box=0] (p-100-10) at (10,-100) {};
%\node[box=0] (p-100-11) at (11,-100) {};
%\node[box=0] (p-100-12) at (12,-100) {};
%\node[box=0] (p-100-13) at (13,-100) {};
%\node[box=0] (p-100-14) at (14,-100) {};
%\node[box=0] (p-100-15) at (15,-100) {};
%\node[box=0] (p-100-16) at (16,-100) {};
%\node[box=0] (p-100-17) at (17,-100) {};
%\node[box=0] (p-100-18) at (18,-100) {};
%\node[box=0] (p-100-19) at (19,-100) {};
%\node[box=0] (p-100-20) at (20,-100) {};
%\node[box=0] (p-100-21) at (21,-100) {};
%\node[box=0] (p-100-22) at (22,-100) {};
%\node[box=0] (p-100-23) at (23,-100) {};
%\node[box=0] (p-100-24) at (24,-100) {};
%\node[box=0] (p-100-25) at (25,-100) {};
%\node[box=0] (p-100-26) at (26,-100) {};
%\node[box=0] (p-100-27) at (27,-100) {};
%\node[box=0] (p-100-28) at (28,-100) {};
%\node[box=0] (p-100-29) at (29,-100) {};
%\node[box=0] (p-100-30) at (30,-100) {};
%\node[box=0] (p-100-31) at (31,-100) {};
%\node[box=1] (p-100-32) at (32,-100) {};
%\node[box=0] (p-100-33) at (33,-100) {};
%\node[box=0] (p-100-34) at (34,-100) {};
%\node[box=0] (p-100-35) at (35,-100) {};
%\node[box=1] (p-100-36) at (36,-100) {};
%\node[box=0] (p-100-37) at (37,-100) {};
%\node[box=0] (p-100-38) at (38,-100) {};
%\node[box=0] (p-100-39) at (39,-100) {};
%\node[box=0] (p-100-40) at (40,-100) {};
%\node[box=0] (p-100-41) at (41,-100) {};
%\node[box=0] (p-100-42) at (42,-100) {};
%\node[box=0] (p-100-43) at (43,-100) {};
%\node[box=0] (p-100-44) at (44,-100) {};
%\node[box=0] (p-100-45) at (45,-100) {};
%\node[box=0] (p-100-46) at (46,-100) {};
%\node[box=0] (p-100-47) at (47,-100) {};
%\node[box=0] (p-100-48) at (48,-100) {};
%\node[box=0] (p-100-49) at (49,-100) {};
%\node[box=0] (p-100-50) at (50,-100) {};
%\node[box=0] (p-100-51) at (51,-100) {};
%\node[box=0] (p-100-52) at (52,-100) {};
%\node[box=0] (p-100-53) at (53,-100) {};
%\node[box=0] (p-100-54) at (54,-100) {};
%\node[box=0] (p-100-55) at (55,-100) {};
%\node[box=0] (p-100-56) at (56,-100) {};
%\node[box=0] (p-100-57) at (57,-100) {};
%\node[box=0] (p-100-58) at (58,-100) {};
%\node[box=0] (p-100-59) at (59,-100) {};
%\node[box=0] (p-100-60) at (60,-100) {};
%\node[box=0] (p-100-61) at (61,-100) {};
%\node[box=0] (p-100-62) at (62,-100) {};
%\node[box=0] (p-100-63) at (63,-100) {};
%\node[box=1] (p-100-64) at (64,-100) {};
%\node[box=0] (p-100-65) at (65,-100) {};
%\node[box=0] (p-100-66) at (66,-100) {};
%\node[box=0] (p-100-67) at (67,-100) {};
%\node[box=1] (p-100-68) at (68,-100) {};
%\node[box=0] (p-100-69) at (69,-100) {};
%\node[box=0] (p-100-70) at (70,-100) {};
%\node[box=0] (p-100-71) at (71,-100) {};
%\node[box=0] (p-100-72) at (72,-100) {};
%\node[box=0] (p-100-73) at (73,-100) {};
%\node[box=0] (p-100-74) at (74,-100) {};
%\node[box=0] (p-100-75) at (75,-100) {};
%\node[box=0] (p-100-76) at (76,-100) {};
%\node[box=0] (p-100-77) at (77,-100) {};
%\node[box=0] (p-100-78) at (78,-100) {};
%\node[box=0] (p-100-79) at (79,-100) {};
%\node[box=0] (p-100-80) at (80,-100) {};
%\node[box=0] (p-100-81) at (81,-100) {};
%\node[box=0] (p-100-82) at (82,-100) {};
%\node[box=0] (p-100-83) at (83,-100) {};
%\node[box=0] (p-100-84) at (84,-100) {};
%\node[box=0] (p-100-85) at (85,-100) {};
%\node[box=0] (p-100-86) at (86,-100) {};
%\node[box=0] (p-100-87) at (87,-100) {};
%\node[box=0] (p-100-88) at (88,-100) {};
%\node[box=0] (p-100-89) at (89,-100) {};
%\node[box=0] (p-100-90) at (90,-100) {};
%\node[box=0] (p-100-91) at (91,-100) {};
%\node[box=0] (p-100-92) at (92,-100) {};
%\node[box=0] (p-100-93) at (93,-100) {};
%\node[box=0] (p-100-94) at (94,-100) {};
%\node[box=0] (p-100-95) at (95,-100) {};
%\node[box=1] (p-100-96) at (96,-100) {};
%\node[box=0] (p-100-97) at (97,-100) {};
%\node[box=0] (p-100-98) at (98,-100) {};
%\node[box=0] (p-100-99) at (99,-100) {};
%\node[box=1] (p-100-100) at (100,-100) {};
%\node[box=1] (p-101-0) at (0,-101) {};
%\node[box=1] (p-101-1) at (1,-101) {};
%\node[box=0] (p-101-2) at (2,-101) {};
%\node[box=0] (p-101-3) at (3,-101) {};
%\node[box=1] (p-101-4) at (4,-101) {};
%\node[box=1] (p-101-5) at (5,-101) {};
%\node[box=0] (p-101-6) at (6,-101) {};
%\node[box=0] (p-101-7) at (7,-101) {};
%\node[box=0] (p-101-8) at (8,-101) {};
%\node[box=0] (p-101-9) at (9,-101) {};
%\node[box=0] (p-101-10) at (10,-101) {};
%\node[box=0] (p-101-11) at (11,-101) {};
%\node[box=0] (p-101-12) at (12,-101) {};
%\node[box=0] (p-101-13) at (13,-101) {};
%\node[box=0] (p-101-14) at (14,-101) {};
%\node[box=0] (p-101-15) at (15,-101) {};
%\node[box=0] (p-101-16) at (16,-101) {};
%\node[box=0] (p-101-17) at (17,-101) {};
%\node[box=0] (p-101-18) at (18,-101) {};
%\node[box=0] (p-101-19) at (19,-101) {};
%\node[box=0] (p-101-20) at (20,-101) {};
%\node[box=0] (p-101-21) at (21,-101) {};
%\node[box=0] (p-101-22) at (22,-101) {};
%\node[box=0] (p-101-23) at (23,-101) {};
%\node[box=0] (p-101-24) at (24,-101) {};
%\node[box=0] (p-101-25) at (25,-101) {};
%\node[box=0] (p-101-26) at (26,-101) {};
%\node[box=0] (p-101-27) at (27,-101) {};
%\node[box=0] (p-101-28) at (28,-101) {};
%\node[box=0] (p-101-29) at (29,-101) {};
%\node[box=0] (p-101-30) at (30,-101) {};
%\node[box=0] (p-101-31) at (31,-101) {};
%\node[box=1] (p-101-32) at (32,-101) {};
%\node[box=1] (p-101-33) at (33,-101) {};
%\node[box=0] (p-101-34) at (34,-101) {};
%\node[box=0] (p-101-35) at (35,-101) {};
%\node[box=1] (p-101-36) at (36,-101) {};
%\node[box=1] (p-101-37) at (37,-101) {};
%\node[box=0] (p-101-38) at (38,-101) {};
%\node[box=0] (p-101-39) at (39,-101) {};
%\node[box=0] (p-101-40) at (40,-101) {};
%\node[box=0] (p-101-41) at (41,-101) {};
%\node[box=0] (p-101-42) at (42,-101) {};
%\node[box=0] (p-101-43) at (43,-101) {};
%\node[box=0] (p-101-44) at (44,-101) {};
%\node[box=0] (p-101-45) at (45,-101) {};
%\node[box=0] (p-101-46) at (46,-101) {};
%\node[box=0] (p-101-47) at (47,-101) {};
%\node[box=0] (p-101-48) at (48,-101) {};
%\node[box=0] (p-101-49) at (49,-101) {};
%\node[box=0] (p-101-50) at (50,-101) {};
%\node[box=0] (p-101-51) at (51,-101) {};
%\node[box=0] (p-101-52) at (52,-101) {};
%\node[box=0] (p-101-53) at (53,-101) {};
%\node[box=0] (p-101-54) at (54,-101) {};
%\node[box=0] (p-101-55) at (55,-101) {};
%\node[box=0] (p-101-56) at (56,-101) {};
%\node[box=0] (p-101-57) at (57,-101) {};
%\node[box=0] (p-101-58) at (58,-101) {};
%\node[box=0] (p-101-59) at (59,-101) {};
%\node[box=0] (p-101-60) at (60,-101) {};
%\node[box=0] (p-101-61) at (61,-101) {};
%\node[box=0] (p-101-62) at (62,-101) {};
%\node[box=0] (p-101-63) at (63,-101) {};
%\node[box=1] (p-101-64) at (64,-101) {};
%\node[box=1] (p-101-65) at (65,-101) {};
%\node[box=0] (p-101-66) at (66,-101) {};
%\node[box=0] (p-101-67) at (67,-101) {};
%\node[box=1] (p-101-68) at (68,-101) {};
%\node[box=1] (p-101-69) at (69,-101) {};
%\node[box=0] (p-101-70) at (70,-101) {};
%\node[box=0] (p-101-71) at (71,-101) {};
%\node[box=0] (p-101-72) at (72,-101) {};
%\node[box=0] (p-101-73) at (73,-101) {};
%\node[box=0] (p-101-74) at (74,-101) {};
%\node[box=0] (p-101-75) at (75,-101) {};
%\node[box=0] (p-101-76) at (76,-101) {};
%\node[box=0] (p-101-77) at (77,-101) {};
%\node[box=0] (p-101-78) at (78,-101) {};
%\node[box=0] (p-101-79) at (79,-101) {};
%\node[box=0] (p-101-80) at (80,-101) {};
%\node[box=0] (p-101-81) at (81,-101) {};
%\node[box=0] (p-101-82) at (82,-101) {};
%\node[box=0] (p-101-83) at (83,-101) {};
%\node[box=0] (p-101-84) at (84,-101) {};
%\node[box=0] (p-101-85) at (85,-101) {};
%\node[box=0] (p-101-86) at (86,-101) {};
%\node[box=0] (p-101-87) at (87,-101) {};
%\node[box=0] (p-101-88) at (88,-101) {};
%\node[box=0] (p-101-89) at (89,-101) {};
%\node[box=0] (p-101-90) at (90,-101) {};
%\node[box=0] (p-101-91) at (91,-101) {};
%\node[box=0] (p-101-92) at (92,-101) {};
%\node[box=0] (p-101-93) at (93,-101) {};
%\node[box=0] (p-101-94) at (94,-101) {};
%\node[box=0] (p-101-95) at (95,-101) {};
%\node[box=1] (p-101-96) at (96,-101) {};
%\node[box=1] (p-101-97) at (97,-101) {};
%\node[box=0] (p-101-98) at (98,-101) {};
%\node[box=0] (p-101-99) at (99,-101) {};
%\node[box=1] (p-101-100) at (100,-101) {};
%\node[box=1] (p-101-101) at (101,-101) {};
%\node[box=1] (p-102-0) at (0,-102) {};
%\node[box=0] (p-102-1) at (1,-102) {};
%\node[box=1] (p-102-2) at (2,-102) {};
%\node[box=0] (p-102-3) at (3,-102) {};
%\node[box=1] (p-102-4) at (4,-102) {};
%\node[box=0] (p-102-5) at (5,-102) {};
%\node[box=1] (p-102-6) at (6,-102) {};
%\node[box=0] (p-102-7) at (7,-102) {};
%\node[box=0] (p-102-8) at (8,-102) {};
%\node[box=0] (p-102-9) at (9,-102) {};
%\node[box=0] (p-102-10) at (10,-102) {};
%\node[box=0] (p-102-11) at (11,-102) {};
%\node[box=0] (p-102-12) at (12,-102) {};
%\node[box=0] (p-102-13) at (13,-102) {};
%\node[box=0] (p-102-14) at (14,-102) {};
%\node[box=0] (p-102-15) at (15,-102) {};
%\node[box=0] (p-102-16) at (16,-102) {};
%\node[box=0] (p-102-17) at (17,-102) {};
%\node[box=0] (p-102-18) at (18,-102) {};
%\node[box=0] (p-102-19) at (19,-102) {};
%\node[box=0] (p-102-20) at (20,-102) {};
%\node[box=0] (p-102-21) at (21,-102) {};
%\node[box=0] (p-102-22) at (22,-102) {};
%\node[box=0] (p-102-23) at (23,-102) {};
%\node[box=0] (p-102-24) at (24,-102) {};
%\node[box=0] (p-102-25) at (25,-102) {};
%\node[box=0] (p-102-26) at (26,-102) {};
%\node[box=0] (p-102-27) at (27,-102) {};
%\node[box=0] (p-102-28) at (28,-102) {};
%\node[box=0] (p-102-29) at (29,-102) {};
%\node[box=0] (p-102-30) at (30,-102) {};
%\node[box=0] (p-102-31) at (31,-102) {};
%\node[box=1] (p-102-32) at (32,-102) {};
%\node[box=0] (p-102-33) at (33,-102) {};
%\node[box=1] (p-102-34) at (34,-102) {};
%\node[box=0] (p-102-35) at (35,-102) {};
%\node[box=1] (p-102-36) at (36,-102) {};
%\node[box=0] (p-102-37) at (37,-102) {};
%\node[box=1] (p-102-38) at (38,-102) {};
%\node[box=0] (p-102-39) at (39,-102) {};
%\node[box=0] (p-102-40) at (40,-102) {};
%\node[box=0] (p-102-41) at (41,-102) {};
%\node[box=0] (p-102-42) at (42,-102) {};
%\node[box=0] (p-102-43) at (43,-102) {};
%\node[box=0] (p-102-44) at (44,-102) {};
%\node[box=0] (p-102-45) at (45,-102) {};
%\node[box=0] (p-102-46) at (46,-102) {};
%\node[box=0] (p-102-47) at (47,-102) {};
%\node[box=0] (p-102-48) at (48,-102) {};
%\node[box=0] (p-102-49) at (49,-102) {};
%\node[box=0] (p-102-50) at (50,-102) {};
%\node[box=0] (p-102-51) at (51,-102) {};
%\node[box=0] (p-102-52) at (52,-102) {};
%\node[box=0] (p-102-53) at (53,-102) {};
%\node[box=0] (p-102-54) at (54,-102) {};
%\node[box=0] (p-102-55) at (55,-102) {};
%\node[box=0] (p-102-56) at (56,-102) {};
%\node[box=0] (p-102-57) at (57,-102) {};
%\node[box=0] (p-102-58) at (58,-102) {};
%\node[box=0] (p-102-59) at (59,-102) {};
%\node[box=0] (p-102-60) at (60,-102) {};
%\node[box=0] (p-102-61) at (61,-102) {};
%\node[box=0] (p-102-62) at (62,-102) {};
%\node[box=0] (p-102-63) at (63,-102) {};
%\node[box=1] (p-102-64) at (64,-102) {};
%\node[box=0] (p-102-65) at (65,-102) {};
%\node[box=1] (p-102-66) at (66,-102) {};
%\node[box=0] (p-102-67) at (67,-102) {};
%\node[box=1] (p-102-68) at (68,-102) {};
%\node[box=0] (p-102-69) at (69,-102) {};
%\node[box=1] (p-102-70) at (70,-102) {};
%\node[box=0] (p-102-71) at (71,-102) {};
%\node[box=0] (p-102-72) at (72,-102) {};
%\node[box=0] (p-102-73) at (73,-102) {};
%\node[box=0] (p-102-74) at (74,-102) {};
%\node[box=0] (p-102-75) at (75,-102) {};
%\node[box=0] (p-102-76) at (76,-102) {};
%\node[box=0] (p-102-77) at (77,-102) {};
%\node[box=0] (p-102-78) at (78,-102) {};
%\node[box=0] (p-102-79) at (79,-102) {};
%\node[box=0] (p-102-80) at (80,-102) {};
%\node[box=0] (p-102-81) at (81,-102) {};
%\node[box=0] (p-102-82) at (82,-102) {};
%\node[box=0] (p-102-83) at (83,-102) {};
%\node[box=0] (p-102-84) at (84,-102) {};
%\node[box=0] (p-102-85) at (85,-102) {};
%\node[box=0] (p-102-86) at (86,-102) {};
%\node[box=0] (p-102-87) at (87,-102) {};
%\node[box=0] (p-102-88) at (88,-102) {};
%\node[box=0] (p-102-89) at (89,-102) {};
%\node[box=0] (p-102-90) at (90,-102) {};
%\node[box=0] (p-102-91) at (91,-102) {};
%\node[box=0] (p-102-92) at (92,-102) {};
%\node[box=0] (p-102-93) at (93,-102) {};
%\node[box=0] (p-102-94) at (94,-102) {};
%\node[box=0] (p-102-95) at (95,-102) {};
%\node[box=1] (p-102-96) at (96,-102) {};
%\node[box=0] (p-102-97) at (97,-102) {};
%\node[box=1] (p-102-98) at (98,-102) {};
%\node[box=0] (p-102-99) at (99,-102) {};
%\node[box=1] (p-102-100) at (100,-102) {};
%\node[box=0] (p-102-101) at (101,-102) {};
%\node[box=1] (p-102-102) at (102,-102) {};
%\node[box=1] (p-103-0) at (0,-103) {};
%\node[box=1] (p-103-1) at (1,-103) {};
%\node[box=1] (p-103-2) at (2,-103) {};
%\node[box=1] (p-103-3) at (3,-103) {};
%\node[box=1] (p-103-4) at (4,-103) {};
%\node[box=1] (p-103-5) at (5,-103) {};
%\node[box=1] (p-103-6) at (6,-103) {};
%\node[box=1] (p-103-7) at (7,-103) {};
%\node[box=0] (p-103-8) at (8,-103) {};
%\node[box=0] (p-103-9) at (9,-103) {};
%\node[box=0] (p-103-10) at (10,-103) {};
%\node[box=0] (p-103-11) at (11,-103) {};
%\node[box=0] (p-103-12) at (12,-103) {};
%\node[box=0] (p-103-13) at (13,-103) {};
%\node[box=0] (p-103-14) at (14,-103) {};
%\node[box=0] (p-103-15) at (15,-103) {};
%\node[box=0] (p-103-16) at (16,-103) {};
%\node[box=0] (p-103-17) at (17,-103) {};
%\node[box=0] (p-103-18) at (18,-103) {};
%\node[box=0] (p-103-19) at (19,-103) {};
%\node[box=0] (p-103-20) at (20,-103) {};
%\node[box=0] (p-103-21) at (21,-103) {};
%\node[box=0] (p-103-22) at (22,-103) {};
%\node[box=0] (p-103-23) at (23,-103) {};
%\node[box=0] (p-103-24) at (24,-103) {};
%\node[box=0] (p-103-25) at (25,-103) {};
%\node[box=0] (p-103-26) at (26,-103) {};
%\node[box=0] (p-103-27) at (27,-103) {};
%\node[box=0] (p-103-28) at (28,-103) {};
%\node[box=0] (p-103-29) at (29,-103) {};
%\node[box=0] (p-103-30) at (30,-103) {};
%\node[box=0] (p-103-31) at (31,-103) {};
%\node[box=1] (p-103-32) at (32,-103) {};
%\node[box=1] (p-103-33) at (33,-103) {};
%\node[box=1] (p-103-34) at (34,-103) {};
%\node[box=1] (p-103-35) at (35,-103) {};
%\node[box=1] (p-103-36) at (36,-103) {};
%\node[box=1] (p-103-37) at (37,-103) {};
%\node[box=1] (p-103-38) at (38,-103) {};
%\node[box=1] (p-103-39) at (39,-103) {};
%\node[box=0] (p-103-40) at (40,-103) {};
%\node[box=0] (p-103-41) at (41,-103) {};
%\node[box=0] (p-103-42) at (42,-103) {};
%\node[box=0] (p-103-43) at (43,-103) {};
%\node[box=0] (p-103-44) at (44,-103) {};
%\node[box=0] (p-103-45) at (45,-103) {};
%\node[box=0] (p-103-46) at (46,-103) {};
%\node[box=0] (p-103-47) at (47,-103) {};
%\node[box=0] (p-103-48) at (48,-103) {};
%\node[box=0] (p-103-49) at (49,-103) {};
%\node[box=0] (p-103-50) at (50,-103) {};
%\node[box=0] (p-103-51) at (51,-103) {};
%\node[box=0] (p-103-52) at (52,-103) {};
%\node[box=0] (p-103-53) at (53,-103) {};
%\node[box=0] (p-103-54) at (54,-103) {};
%\node[box=0] (p-103-55) at (55,-103) {};
%\node[box=0] (p-103-56) at (56,-103) {};
%\node[box=0] (p-103-57) at (57,-103) {};
%\node[box=0] (p-103-58) at (58,-103) {};
%\node[box=0] (p-103-59) at (59,-103) {};
%\node[box=0] (p-103-60) at (60,-103) {};
%\node[box=0] (p-103-61) at (61,-103) {};
%\node[box=0] (p-103-62) at (62,-103) {};
%\node[box=0] (p-103-63) at (63,-103) {};
%\node[box=1] (p-103-64) at (64,-103) {};
%\node[box=1] (p-103-65) at (65,-103) {};
%\node[box=1] (p-103-66) at (66,-103) {};
%\node[box=1] (p-103-67) at (67,-103) {};
%\node[box=1] (p-103-68) at (68,-103) {};
%\node[box=1] (p-103-69) at (69,-103) {};
%\node[box=1] (p-103-70) at (70,-103) {};
%\node[box=1] (p-103-71) at (71,-103) {};
%\node[box=0] (p-103-72) at (72,-103) {};
%\node[box=0] (p-103-73) at (73,-103) {};
%\node[box=0] (p-103-74) at (74,-103) {};
%\node[box=0] (p-103-75) at (75,-103) {};
%\node[box=0] (p-103-76) at (76,-103) {};
%\node[box=0] (p-103-77) at (77,-103) {};
%\node[box=0] (p-103-78) at (78,-103) {};
%\node[box=0] (p-103-79) at (79,-103) {};
%\node[box=0] (p-103-80) at (80,-103) {};
%\node[box=0] (p-103-81) at (81,-103) {};
%\node[box=0] (p-103-82) at (82,-103) {};
%\node[box=0] (p-103-83) at (83,-103) {};
%\node[box=0] (p-103-84) at (84,-103) {};
%\node[box=0] (p-103-85) at (85,-103) {};
%\node[box=0] (p-103-86) at (86,-103) {};
%\node[box=0] (p-103-87) at (87,-103) {};
%\node[box=0] (p-103-88) at (88,-103) {};
%\node[box=0] (p-103-89) at (89,-103) {};
%\node[box=0] (p-103-90) at (90,-103) {};
%\node[box=0] (p-103-91) at (91,-103) {};
%\node[box=0] (p-103-92) at (92,-103) {};
%\node[box=0] (p-103-93) at (93,-103) {};
%\node[box=0] (p-103-94) at (94,-103) {};
%\node[box=0] (p-103-95) at (95,-103) {};
%\node[box=1] (p-103-96) at (96,-103) {};
%\node[box=1] (p-103-97) at (97,-103) {};
%\node[box=1] (p-103-98) at (98,-103) {};
%\node[box=1] (p-103-99) at (99,-103) {};
%\node[box=1] (p-103-100) at (100,-103) {};
%\node[box=1] (p-103-101) at (101,-103) {};
%\node[box=1] (p-103-102) at (102,-103) {};
%\node[box=1] (p-103-103) at (103,-103) {};
%\node[box=1] (p-104-0) at (0,-104) {};
%\node[box=0] (p-104-1) at (1,-104) {};
%\node[box=0] (p-104-2) at (2,-104) {};
%\node[box=0] (p-104-3) at (3,-104) {};
%\node[box=0] (p-104-4) at (4,-104) {};
%\node[box=0] (p-104-5) at (5,-104) {};
%\node[box=0] (p-104-6) at (6,-104) {};
%\node[box=0] (p-104-7) at (7,-104) {};
%\node[box=1] (p-104-8) at (8,-104) {};
%\node[box=0] (p-104-9) at (9,-104) {};
%\node[box=0] (p-104-10) at (10,-104) {};
%\node[box=0] (p-104-11) at (11,-104) {};
%\node[box=0] (p-104-12) at (12,-104) {};
%\node[box=0] (p-104-13) at (13,-104) {};
%\node[box=0] (p-104-14) at (14,-104) {};
%\node[box=0] (p-104-15) at (15,-104) {};
%\node[box=0] (p-104-16) at (16,-104) {};
%\node[box=0] (p-104-17) at (17,-104) {};
%\node[box=0] (p-104-18) at (18,-104) {};
%\node[box=0] (p-104-19) at (19,-104) {};
%\node[box=0] (p-104-20) at (20,-104) {};
%\node[box=0] (p-104-21) at (21,-104) {};
%\node[box=0] (p-104-22) at (22,-104) {};
%\node[box=0] (p-104-23) at (23,-104) {};
%\node[box=0] (p-104-24) at (24,-104) {};
%\node[box=0] (p-104-25) at (25,-104) {};
%\node[box=0] (p-104-26) at (26,-104) {};
%\node[box=0] (p-104-27) at (27,-104) {};
%\node[box=0] (p-104-28) at (28,-104) {};
%\node[box=0] (p-104-29) at (29,-104) {};
%\node[box=0] (p-104-30) at (30,-104) {};
%\node[box=0] (p-104-31) at (31,-104) {};
%\node[box=1] (p-104-32) at (32,-104) {};
%\node[box=0] (p-104-33) at (33,-104) {};
%\node[box=0] (p-104-34) at (34,-104) {};
%\node[box=0] (p-104-35) at (35,-104) {};
%\node[box=0] (p-104-36) at (36,-104) {};
%\node[box=0] (p-104-37) at (37,-104) {};
%\node[box=0] (p-104-38) at (38,-104) {};
%\node[box=0] (p-104-39) at (39,-104) {};
%\node[box=1] (p-104-40) at (40,-104) {};
%\node[box=0] (p-104-41) at (41,-104) {};
%\node[box=0] (p-104-42) at (42,-104) {};
%\node[box=0] (p-104-43) at (43,-104) {};
%\node[box=0] (p-104-44) at (44,-104) {};
%\node[box=0] (p-104-45) at (45,-104) {};
%\node[box=0] (p-104-46) at (46,-104) {};
%\node[box=0] (p-104-47) at (47,-104) {};
%\node[box=0] (p-104-48) at (48,-104) {};
%\node[box=0] (p-104-49) at (49,-104) {};
%\node[box=0] (p-104-50) at (50,-104) {};
%\node[box=0] (p-104-51) at (51,-104) {};
%\node[box=0] (p-104-52) at (52,-104) {};
%\node[box=0] (p-104-53) at (53,-104) {};
%\node[box=0] (p-104-54) at (54,-104) {};
%\node[box=0] (p-104-55) at (55,-104) {};
%\node[box=0] (p-104-56) at (56,-104) {};
%\node[box=0] (p-104-57) at (57,-104) {};
%\node[box=0] (p-104-58) at (58,-104) {};
%\node[box=0] (p-104-59) at (59,-104) {};
%\node[box=0] (p-104-60) at (60,-104) {};
%\node[box=0] (p-104-61) at (61,-104) {};
%\node[box=0] (p-104-62) at (62,-104) {};
%\node[box=0] (p-104-63) at (63,-104) {};
%\node[box=1] (p-104-64) at (64,-104) {};
%\node[box=0] (p-104-65) at (65,-104) {};
%\node[box=0] (p-104-66) at (66,-104) {};
%\node[box=0] (p-104-67) at (67,-104) {};
%\node[box=0] (p-104-68) at (68,-104) {};
%\node[box=0] (p-104-69) at (69,-104) {};
%\node[box=0] (p-104-70) at (70,-104) {};
%\node[box=0] (p-104-71) at (71,-104) {};
%\node[box=1] (p-104-72) at (72,-104) {};
%\node[box=0] (p-104-73) at (73,-104) {};
%\node[box=0] (p-104-74) at (74,-104) {};
%\node[box=0] (p-104-75) at (75,-104) {};
%\node[box=0] (p-104-76) at (76,-104) {};
%\node[box=0] (p-104-77) at (77,-104) {};
%\node[box=0] (p-104-78) at (78,-104) {};
%\node[box=0] (p-104-79) at (79,-104) {};
%\node[box=0] (p-104-80) at (80,-104) {};
%\node[box=0] (p-104-81) at (81,-104) {};
%\node[box=0] (p-104-82) at (82,-104) {};
%\node[box=0] (p-104-83) at (83,-104) {};
%\node[box=0] (p-104-84) at (84,-104) {};
%\node[box=0] (p-104-85) at (85,-104) {};
%\node[box=0] (p-104-86) at (86,-104) {};
%\node[box=0] (p-104-87) at (87,-104) {};
%\node[box=0] (p-104-88) at (88,-104) {};
%\node[box=0] (p-104-89) at (89,-104) {};
%\node[box=0] (p-104-90) at (90,-104) {};
%\node[box=0] (p-104-91) at (91,-104) {};
%\node[box=0] (p-104-92) at (92,-104) {};
%\node[box=0] (p-104-93) at (93,-104) {};
%\node[box=0] (p-104-94) at (94,-104) {};
%\node[box=0] (p-104-95) at (95,-104) {};
%\node[box=1] (p-104-96) at (96,-104) {};
%\node[box=0] (p-104-97) at (97,-104) {};
%\node[box=0] (p-104-98) at (98,-104) {};
%\node[box=0] (p-104-99) at (99,-104) {};
%\node[box=0] (p-104-100) at (100,-104) {};
%\node[box=0] (p-104-101) at (101,-104) {};
%\node[box=0] (p-104-102) at (102,-104) {};
%\node[box=0] (p-104-103) at (103,-104) {};
%\node[box=1] (p-104-104) at (104,-104) {};
%\node[box=1] (p-105-0) at (0,-105) {};
%\node[box=1] (p-105-1) at (1,-105) {};
%\node[box=0] (p-105-2) at (2,-105) {};
%\node[box=0] (p-105-3) at (3,-105) {};
%\node[box=0] (p-105-4) at (4,-105) {};
%\node[box=0] (p-105-5) at (5,-105) {};
%\node[box=0] (p-105-6) at (6,-105) {};
%\node[box=0] (p-105-7) at (7,-105) {};
%\node[box=1] (p-105-8) at (8,-105) {};
%\node[box=1] (p-105-9) at (9,-105) {};
%\node[box=0] (p-105-10) at (10,-105) {};
%\node[box=0] (p-105-11) at (11,-105) {};
%\node[box=0] (p-105-12) at (12,-105) {};
%\node[box=0] (p-105-13) at (13,-105) {};
%\node[box=0] (p-105-14) at (14,-105) {};
%\node[box=0] (p-105-15) at (15,-105) {};
%\node[box=0] (p-105-16) at (16,-105) {};
%\node[box=0] (p-105-17) at (17,-105) {};
%\node[box=0] (p-105-18) at (18,-105) {};
%\node[box=0] (p-105-19) at (19,-105) {};
%\node[box=0] (p-105-20) at (20,-105) {};
%\node[box=0] (p-105-21) at (21,-105) {};
%\node[box=0] (p-105-22) at (22,-105) {};
%\node[box=0] (p-105-23) at (23,-105) {};
%\node[box=0] (p-105-24) at (24,-105) {};
%\node[box=0] (p-105-25) at (25,-105) {};
%\node[box=0] (p-105-26) at (26,-105) {};
%\node[box=0] (p-105-27) at (27,-105) {};
%\node[box=0] (p-105-28) at (28,-105) {};
%\node[box=0] (p-105-29) at (29,-105) {};
%\node[box=0] (p-105-30) at (30,-105) {};
%\node[box=0] (p-105-31) at (31,-105) {};
%\node[box=1] (p-105-32) at (32,-105) {};
%\node[box=1] (p-105-33) at (33,-105) {};
%\node[box=0] (p-105-34) at (34,-105) {};
%\node[box=0] (p-105-35) at (35,-105) {};
%\node[box=0] (p-105-36) at (36,-105) {};
%\node[box=0] (p-105-37) at (37,-105) {};
%\node[box=0] (p-105-38) at (38,-105) {};
%\node[box=0] (p-105-39) at (39,-105) {};
%\node[box=1] (p-105-40) at (40,-105) {};
%\node[box=1] (p-105-41) at (41,-105) {};
%\node[box=0] (p-105-42) at (42,-105) {};
%\node[box=0] (p-105-43) at (43,-105) {};
%\node[box=0] (p-105-44) at (44,-105) {};
%\node[box=0] (p-105-45) at (45,-105) {};
%\node[box=0] (p-105-46) at (46,-105) {};
%\node[box=0] (p-105-47) at (47,-105) {};
%\node[box=0] (p-105-48) at (48,-105) {};
%\node[box=0] (p-105-49) at (49,-105) {};
%\node[box=0] (p-105-50) at (50,-105) {};
%\node[box=0] (p-105-51) at (51,-105) {};
%\node[box=0] (p-105-52) at (52,-105) {};
%\node[box=0] (p-105-53) at (53,-105) {};
%\node[box=0] (p-105-54) at (54,-105) {};
%\node[box=0] (p-105-55) at (55,-105) {};
%\node[box=0] (p-105-56) at (56,-105) {};
%\node[box=0] (p-105-57) at (57,-105) {};
%\node[box=0] (p-105-58) at (58,-105) {};
%\node[box=0] (p-105-59) at (59,-105) {};
%\node[box=0] (p-105-60) at (60,-105) {};
%\node[box=0] (p-105-61) at (61,-105) {};
%\node[box=0] (p-105-62) at (62,-105) {};
%\node[box=0] (p-105-63) at (63,-105) {};
%\node[box=1] (p-105-64) at (64,-105) {};
%\node[box=1] (p-105-65) at (65,-105) {};
%\node[box=0] (p-105-66) at (66,-105) {};
%\node[box=0] (p-105-67) at (67,-105) {};
%\node[box=0] (p-105-68) at (68,-105) {};
%\node[box=0] (p-105-69) at (69,-105) {};
%\node[box=0] (p-105-70) at (70,-105) {};
%\node[box=0] (p-105-71) at (71,-105) {};
%\node[box=1] (p-105-72) at (72,-105) {};
%\node[box=1] (p-105-73) at (73,-105) {};
%\node[box=0] (p-105-74) at (74,-105) {};
%\node[box=0] (p-105-75) at (75,-105) {};
%\node[box=0] (p-105-76) at (76,-105) {};
%\node[box=0] (p-105-77) at (77,-105) {};
%\node[box=0] (p-105-78) at (78,-105) {};
%\node[box=0] (p-105-79) at (79,-105) {};
%\node[box=0] (p-105-80) at (80,-105) {};
%\node[box=0] (p-105-81) at (81,-105) {};
%\node[box=0] (p-105-82) at (82,-105) {};
%\node[box=0] (p-105-83) at (83,-105) {};
%\node[box=0] (p-105-84) at (84,-105) {};
%\node[box=0] (p-105-85) at (85,-105) {};
%\node[box=0] (p-105-86) at (86,-105) {};
%\node[box=0] (p-105-87) at (87,-105) {};
%\node[box=0] (p-105-88) at (88,-105) {};
%\node[box=0] (p-105-89) at (89,-105) {};
%\node[box=0] (p-105-90) at (90,-105) {};
%\node[box=0] (p-105-91) at (91,-105) {};
%\node[box=0] (p-105-92) at (92,-105) {};
%\node[box=0] (p-105-93) at (93,-105) {};
%\node[box=0] (p-105-94) at (94,-105) {};
%\node[box=0] (p-105-95) at (95,-105) {};
%\node[box=1] (p-105-96) at (96,-105) {};
%\node[box=1] (p-105-97) at (97,-105) {};
%\node[box=0] (p-105-98) at (98,-105) {};
%\node[box=0] (p-105-99) at (99,-105) {};
%\node[box=0] (p-105-100) at (100,-105) {};
%\node[box=0] (p-105-101) at (101,-105) {};
%\node[box=0] (p-105-102) at (102,-105) {};
%\node[box=0] (p-105-103) at (103,-105) {};
%\node[box=1] (p-105-104) at (104,-105) {};
%\node[box=1] (p-105-105) at (105,-105) {};
%\node[box=1] (p-106-0) at (0,-106) {};
%\node[box=0] (p-106-1) at (1,-106) {};
%\node[box=1] (p-106-2) at (2,-106) {};
%\node[box=0] (p-106-3) at (3,-106) {};
%\node[box=0] (p-106-4) at (4,-106) {};
%\node[box=0] (p-106-5) at (5,-106) {};
%\node[box=0] (p-106-6) at (6,-106) {};
%\node[box=0] (p-106-7) at (7,-106) {};
%\node[box=1] (p-106-8) at (8,-106) {};
%\node[box=0] (p-106-9) at (9,-106) {};
%\node[box=1] (p-106-10) at (10,-106) {};
%\node[box=0] (p-106-11) at (11,-106) {};
%\node[box=0] (p-106-12) at (12,-106) {};
%\node[box=0] (p-106-13) at (13,-106) {};
%\node[box=0] (p-106-14) at (14,-106) {};
%\node[box=0] (p-106-15) at (15,-106) {};
%\node[box=0] (p-106-16) at (16,-106) {};
%\node[box=0] (p-106-17) at (17,-106) {};
%\node[box=0] (p-106-18) at (18,-106) {};
%\node[box=0] (p-106-19) at (19,-106) {};
%\node[box=0] (p-106-20) at (20,-106) {};
%\node[box=0] (p-106-21) at (21,-106) {};
%\node[box=0] (p-106-22) at (22,-106) {};
%\node[box=0] (p-106-23) at (23,-106) {};
%\node[box=0] (p-106-24) at (24,-106) {};
%\node[box=0] (p-106-25) at (25,-106) {};
%\node[box=0] (p-106-26) at (26,-106) {};
%\node[box=0] (p-106-27) at (27,-106) {};
%\node[box=0] (p-106-28) at (28,-106) {};
%\node[box=0] (p-106-29) at (29,-106) {};
%\node[box=0] (p-106-30) at (30,-106) {};
%\node[box=0] (p-106-31) at (31,-106) {};
%\node[box=1] (p-106-32) at (32,-106) {};
%\node[box=0] (p-106-33) at (33,-106) {};
%\node[box=1] (p-106-34) at (34,-106) {};
%\node[box=0] (p-106-35) at (35,-106) {};
%\node[box=0] (p-106-36) at (36,-106) {};
%\node[box=0] (p-106-37) at (37,-106) {};
%\node[box=0] (p-106-38) at (38,-106) {};
%\node[box=0] (p-106-39) at (39,-106) {};
%\node[box=1] (p-106-40) at (40,-106) {};
%\node[box=0] (p-106-41) at (41,-106) {};
%\node[box=1] (p-106-42) at (42,-106) {};
%\node[box=0] (p-106-43) at (43,-106) {};
%\node[box=0] (p-106-44) at (44,-106) {};
%\node[box=0] (p-106-45) at (45,-106) {};
%\node[box=0] (p-106-46) at (46,-106) {};
%\node[box=0] (p-106-47) at (47,-106) {};
%\node[box=0] (p-106-48) at (48,-106) {};
%\node[box=0] (p-106-49) at (49,-106) {};
%\node[box=0] (p-106-50) at (50,-106) {};
%\node[box=0] (p-106-51) at (51,-106) {};
%\node[box=0] (p-106-52) at (52,-106) {};
%\node[box=0] (p-106-53) at (53,-106) {};
%\node[box=0] (p-106-54) at (54,-106) {};
%\node[box=0] (p-106-55) at (55,-106) {};
%\node[box=0] (p-106-56) at (56,-106) {};
%\node[box=0] (p-106-57) at (57,-106) {};
%\node[box=0] (p-106-58) at (58,-106) {};
%\node[box=0] (p-106-59) at (59,-106) {};
%\node[box=0] (p-106-60) at (60,-106) {};
%\node[box=0] (p-106-61) at (61,-106) {};
%\node[box=0] (p-106-62) at (62,-106) {};
%\node[box=0] (p-106-63) at (63,-106) {};
%\node[box=1] (p-106-64) at (64,-106) {};
%\node[box=0] (p-106-65) at (65,-106) {};
%\node[box=1] (p-106-66) at (66,-106) {};
%\node[box=0] (p-106-67) at (67,-106) {};
%\node[box=0] (p-106-68) at (68,-106) {};
%\node[box=0] (p-106-69) at (69,-106) {};
%\node[box=0] (p-106-70) at (70,-106) {};
%\node[box=0] (p-106-71) at (71,-106) {};
%\node[box=1] (p-106-72) at (72,-106) {};
%\node[box=0] (p-106-73) at (73,-106) {};
%\node[box=1] (p-106-74) at (74,-106) {};
%\node[box=0] (p-106-75) at (75,-106) {};
%\node[box=0] (p-106-76) at (76,-106) {};
%\node[box=0] (p-106-77) at (77,-106) {};
%\node[box=0] (p-106-78) at (78,-106) {};
%\node[box=0] (p-106-79) at (79,-106) {};
%\node[box=0] (p-106-80) at (80,-106) {};
%\node[box=0] (p-106-81) at (81,-106) {};
%\node[box=0] (p-106-82) at (82,-106) {};
%\node[box=0] (p-106-83) at (83,-106) {};
%\node[box=0] (p-106-84) at (84,-106) {};
%\node[box=0] (p-106-85) at (85,-106) {};
%\node[box=0] (p-106-86) at (86,-106) {};
%\node[box=0] (p-106-87) at (87,-106) {};
%\node[box=0] (p-106-88) at (88,-106) {};
%\node[box=0] (p-106-89) at (89,-106) {};
%\node[box=0] (p-106-90) at (90,-106) {};
%\node[box=0] (p-106-91) at (91,-106) {};
%\node[box=0] (p-106-92) at (92,-106) {};
%\node[box=0] (p-106-93) at (93,-106) {};
%\node[box=0] (p-106-94) at (94,-106) {};
%\node[box=0] (p-106-95) at (95,-106) {};
%\node[box=1] (p-106-96) at (96,-106) {};
%\node[box=0] (p-106-97) at (97,-106) {};
%\node[box=1] (p-106-98) at (98,-106) {};
%\node[box=0] (p-106-99) at (99,-106) {};
%\node[box=0] (p-106-100) at (100,-106) {};
%\node[box=0] (p-106-101) at (101,-106) {};
%\node[box=0] (p-106-102) at (102,-106) {};
%\node[box=0] (p-106-103) at (103,-106) {};
%\node[box=1] (p-106-104) at (104,-106) {};
%\node[box=0] (p-106-105) at (105,-106) {};
%\node[box=1] (p-106-106) at (106,-106) {};
%\node[box=1] (p-107-0) at (0,-107) {};
%\node[box=1] (p-107-1) at (1,-107) {};
%\node[box=1] (p-107-2) at (2,-107) {};
%\node[box=1] (p-107-3) at (3,-107) {};
%\node[box=0] (p-107-4) at (4,-107) {};
%\node[box=0] (p-107-5) at (5,-107) {};
%\node[box=0] (p-107-6) at (6,-107) {};
%\node[box=0] (p-107-7) at (7,-107) {};
%\node[box=1] (p-107-8) at (8,-107) {};
%\node[box=1] (p-107-9) at (9,-107) {};
%\node[box=1] (p-107-10) at (10,-107) {};
%\node[box=1] (p-107-11) at (11,-107) {};
%\node[box=0] (p-107-12) at (12,-107) {};
%\node[box=0] (p-107-13) at (13,-107) {};
%\node[box=0] (p-107-14) at (14,-107) {};
%\node[box=0] (p-107-15) at (15,-107) {};
%\node[box=0] (p-107-16) at (16,-107) {};
%\node[box=0] (p-107-17) at (17,-107) {};
%\node[box=0] (p-107-18) at (18,-107) {};
%\node[box=0] (p-107-19) at (19,-107) {};
%\node[box=0] (p-107-20) at (20,-107) {};
%\node[box=0] (p-107-21) at (21,-107) {};
%\node[box=0] (p-107-22) at (22,-107) {};
%\node[box=0] (p-107-23) at (23,-107) {};
%\node[box=0] (p-107-24) at (24,-107) {};
%\node[box=0] (p-107-25) at (25,-107) {};
%\node[box=0] (p-107-26) at (26,-107) {};
%\node[box=0] (p-107-27) at (27,-107) {};
%\node[box=0] (p-107-28) at (28,-107) {};
%\node[box=0] (p-107-29) at (29,-107) {};
%\node[box=0] (p-107-30) at (30,-107) {};
%\node[box=0] (p-107-31) at (31,-107) {};
%\node[box=1] (p-107-32) at (32,-107) {};
%\node[box=1] (p-107-33) at (33,-107) {};
%\node[box=1] (p-107-34) at (34,-107) {};
%\node[box=1] (p-107-35) at (35,-107) {};
%\node[box=0] (p-107-36) at (36,-107) {};
%\node[box=0] (p-107-37) at (37,-107) {};
%\node[box=0] (p-107-38) at (38,-107) {};
%\node[box=0] (p-107-39) at (39,-107) {};
%\node[box=1] (p-107-40) at (40,-107) {};
%\node[box=1] (p-107-41) at (41,-107) {};
%\node[box=1] (p-107-42) at (42,-107) {};
%\node[box=1] (p-107-43) at (43,-107) {};
%\node[box=0] (p-107-44) at (44,-107) {};
%\node[box=0] (p-107-45) at (45,-107) {};
%\node[box=0] (p-107-46) at (46,-107) {};
%\node[box=0] (p-107-47) at (47,-107) {};
%\node[box=0] (p-107-48) at (48,-107) {};
%\node[box=0] (p-107-49) at (49,-107) {};
%\node[box=0] (p-107-50) at (50,-107) {};
%\node[box=0] (p-107-51) at (51,-107) {};
%\node[box=0] (p-107-52) at (52,-107) {};
%\node[box=0] (p-107-53) at (53,-107) {};
%\node[box=0] (p-107-54) at (54,-107) {};
%\node[box=0] (p-107-55) at (55,-107) {};
%\node[box=0] (p-107-56) at (56,-107) {};
%\node[box=0] (p-107-57) at (57,-107) {};
%\node[box=0] (p-107-58) at (58,-107) {};
%\node[box=0] (p-107-59) at (59,-107) {};
%\node[box=0] (p-107-60) at (60,-107) {};
%\node[box=0] (p-107-61) at (61,-107) {};
%\node[box=0] (p-107-62) at (62,-107) {};
%\node[box=0] (p-107-63) at (63,-107) {};
%\node[box=1] (p-107-64) at (64,-107) {};
%\node[box=1] (p-107-65) at (65,-107) {};
%\node[box=1] (p-107-66) at (66,-107) {};
%\node[box=1] (p-107-67) at (67,-107) {};
%\node[box=0] (p-107-68) at (68,-107) {};
%\node[box=0] (p-107-69) at (69,-107) {};
%\node[box=0] (p-107-70) at (70,-107) {};
%\node[box=0] (p-107-71) at (71,-107) {};
%\node[box=1] (p-107-72) at (72,-107) {};
%\node[box=1] (p-107-73) at (73,-107) {};
%\node[box=1] (p-107-74) at (74,-107) {};
%\node[box=1] (p-107-75) at (75,-107) {};
%\node[box=0] (p-107-76) at (76,-107) {};
%\node[box=0] (p-107-77) at (77,-107) {};
%\node[box=0] (p-107-78) at (78,-107) {};
%\node[box=0] (p-107-79) at (79,-107) {};
%\node[box=0] (p-107-80) at (80,-107) {};
%\node[box=0] (p-107-81) at (81,-107) {};
%\node[box=0] (p-107-82) at (82,-107) {};
%\node[box=0] (p-107-83) at (83,-107) {};
%\node[box=0] (p-107-84) at (84,-107) {};
%\node[box=0] (p-107-85) at (85,-107) {};
%\node[box=0] (p-107-86) at (86,-107) {};
%\node[box=0] (p-107-87) at (87,-107) {};
%\node[box=0] (p-107-88) at (88,-107) {};
%\node[box=0] (p-107-89) at (89,-107) {};
%\node[box=0] (p-107-90) at (90,-107) {};
%\node[box=0] (p-107-91) at (91,-107) {};
%\node[box=0] (p-107-92) at (92,-107) {};
%\node[box=0] (p-107-93) at (93,-107) {};
%\node[box=0] (p-107-94) at (94,-107) {};
%\node[box=0] (p-107-95) at (95,-107) {};
%\node[box=1] (p-107-96) at (96,-107) {};
%\node[box=1] (p-107-97) at (97,-107) {};
%\node[box=1] (p-107-98) at (98,-107) {};
%\node[box=1] (p-107-99) at (99,-107) {};
%\node[box=0] (p-107-100) at (100,-107) {};
%\node[box=0] (p-107-101) at (101,-107) {};
%\node[box=0] (p-107-102) at (102,-107) {};
%\node[box=0] (p-107-103) at (103,-107) {};
%\node[box=1] (p-107-104) at (104,-107) {};
%\node[box=1] (p-107-105) at (105,-107) {};
%\node[box=1] (p-107-106) at (106,-107) {};
%\node[box=1] (p-107-107) at (107,-107) {};
%\node[box=1] (p-108-0) at (0,-108) {};
%\node[box=0] (p-108-1) at (1,-108) {};
%\node[box=0] (p-108-2) at (2,-108) {};
%\node[box=0] (p-108-3) at (3,-108) {};
%\node[box=1] (p-108-4) at (4,-108) {};
%\node[box=0] (p-108-5) at (5,-108) {};
%\node[box=0] (p-108-6) at (6,-108) {};
%\node[box=0] (p-108-7) at (7,-108) {};
%\node[box=1] (p-108-8) at (8,-108) {};
%\node[box=0] (p-108-9) at (9,-108) {};
%\node[box=0] (p-108-10) at (10,-108) {};
%\node[box=0] (p-108-11) at (11,-108) {};
%\node[box=1] (p-108-12) at (12,-108) {};
%\node[box=0] (p-108-13) at (13,-108) {};
%\node[box=0] (p-108-14) at (14,-108) {};
%\node[box=0] (p-108-15) at (15,-108) {};
%\node[box=0] (p-108-16) at (16,-108) {};
%\node[box=0] (p-108-17) at (17,-108) {};
%\node[box=0] (p-108-18) at (18,-108) {};
%\node[box=0] (p-108-19) at (19,-108) {};
%\node[box=0] (p-108-20) at (20,-108) {};
%\node[box=0] (p-108-21) at (21,-108) {};
%\node[box=0] (p-108-22) at (22,-108) {};
%\node[box=0] (p-108-23) at (23,-108) {};
%\node[box=0] (p-108-24) at (24,-108) {};
%\node[box=0] (p-108-25) at (25,-108) {};
%\node[box=0] (p-108-26) at (26,-108) {};
%\node[box=0] (p-108-27) at (27,-108) {};
%\node[box=0] (p-108-28) at (28,-108) {};
%\node[box=0] (p-108-29) at (29,-108) {};
%\node[box=0] (p-108-30) at (30,-108) {};
%\node[box=0] (p-108-31) at (31,-108) {};
%\node[box=1] (p-108-32) at (32,-108) {};
%\node[box=0] (p-108-33) at (33,-108) {};
%\node[box=0] (p-108-34) at (34,-108) {};
%\node[box=0] (p-108-35) at (35,-108) {};
%\node[box=1] (p-108-36) at (36,-108) {};
%\node[box=0] (p-108-37) at (37,-108) {};
%\node[box=0] (p-108-38) at (38,-108) {};
%\node[box=0] (p-108-39) at (39,-108) {};
%\node[box=1] (p-108-40) at (40,-108) {};
%\node[box=0] (p-108-41) at (41,-108) {};
%\node[box=0] (p-108-42) at (42,-108) {};
%\node[box=0] (p-108-43) at (43,-108) {};
%\node[box=1] (p-108-44) at (44,-108) {};
%\node[box=0] (p-108-45) at (45,-108) {};
%\node[box=0] (p-108-46) at (46,-108) {};
%\node[box=0] (p-108-47) at (47,-108) {};
%\node[box=0] (p-108-48) at (48,-108) {};
%\node[box=0] (p-108-49) at (49,-108) {};
%\node[box=0] (p-108-50) at (50,-108) {};
%\node[box=0] (p-108-51) at (51,-108) {};
%\node[box=0] (p-108-52) at (52,-108) {};
%\node[box=0] (p-108-53) at (53,-108) {};
%\node[box=0] (p-108-54) at (54,-108) {};
%\node[box=0] (p-108-55) at (55,-108) {};
%\node[box=0] (p-108-56) at (56,-108) {};
%\node[box=0] (p-108-57) at (57,-108) {};
%\node[box=0] (p-108-58) at (58,-108) {};
%\node[box=0] (p-108-59) at (59,-108) {};
%\node[box=0] (p-108-60) at (60,-108) {};
%\node[box=0] (p-108-61) at (61,-108) {};
%\node[box=0] (p-108-62) at (62,-108) {};
%\node[box=0] (p-108-63) at (63,-108) {};
%\node[box=1] (p-108-64) at (64,-108) {};
%\node[box=0] (p-108-65) at (65,-108) {};
%\node[box=0] (p-108-66) at (66,-108) {};
%\node[box=0] (p-108-67) at (67,-108) {};
%\node[box=1] (p-108-68) at (68,-108) {};
%\node[box=0] (p-108-69) at (69,-108) {};
%\node[box=0] (p-108-70) at (70,-108) {};
%\node[box=0] (p-108-71) at (71,-108) {};
%\node[box=1] (p-108-72) at (72,-108) {};
%\node[box=0] (p-108-73) at (73,-108) {};
%\node[box=0] (p-108-74) at (74,-108) {};
%\node[box=0] (p-108-75) at (75,-108) {};
%\node[box=1] (p-108-76) at (76,-108) {};
%\node[box=0] (p-108-77) at (77,-108) {};
%\node[box=0] (p-108-78) at (78,-108) {};
%\node[box=0] (p-108-79) at (79,-108) {};
%\node[box=0] (p-108-80) at (80,-108) {};
%\node[box=0] (p-108-81) at (81,-108) {};
%\node[box=0] (p-108-82) at (82,-108) {};
%\node[box=0] (p-108-83) at (83,-108) {};
%\node[box=0] (p-108-84) at (84,-108) {};
%\node[box=0] (p-108-85) at (85,-108) {};
%\node[box=0] (p-108-86) at (86,-108) {};
%\node[box=0] (p-108-87) at (87,-108) {};
%\node[box=0] (p-108-88) at (88,-108) {};
%\node[box=0] (p-108-89) at (89,-108) {};
%\node[box=0] (p-108-90) at (90,-108) {};
%\node[box=0] (p-108-91) at (91,-108) {};
%\node[box=0] (p-108-92) at (92,-108) {};
%\node[box=0] (p-108-93) at (93,-108) {};
%\node[box=0] (p-108-94) at (94,-108) {};
%\node[box=0] (p-108-95) at (95,-108) {};
%\node[box=1] (p-108-96) at (96,-108) {};
%\node[box=0] (p-108-97) at (97,-108) {};
%\node[box=0] (p-108-98) at (98,-108) {};
%\node[box=0] (p-108-99) at (99,-108) {};
%\node[box=1] (p-108-100) at (100,-108) {};
%\node[box=0] (p-108-101) at (101,-108) {};
%\node[box=0] (p-108-102) at (102,-108) {};
%\node[box=0] (p-108-103) at (103,-108) {};
%\node[box=1] (p-108-104) at (104,-108) {};
%\node[box=0] (p-108-105) at (105,-108) {};
%\node[box=0] (p-108-106) at (106,-108) {};
%\node[box=0] (p-108-107) at (107,-108) {};
%\node[box=1] (p-108-108) at (108,-108) {};
%\node[box=1] (p-109-0) at (0,-109) {};
%\node[box=1] (p-109-1) at (1,-109) {};
%\node[box=0] (p-109-2) at (2,-109) {};
%\node[box=0] (p-109-3) at (3,-109) {};
%\node[box=1] (p-109-4) at (4,-109) {};
%\node[box=1] (p-109-5) at (5,-109) {};
%\node[box=0] (p-109-6) at (6,-109) {};
%\node[box=0] (p-109-7) at (7,-109) {};
%\node[box=1] (p-109-8) at (8,-109) {};
%\node[box=1] (p-109-9) at (9,-109) {};
%\node[box=0] (p-109-10) at (10,-109) {};
%\node[box=0] (p-109-11) at (11,-109) {};
%\node[box=1] (p-109-12) at (12,-109) {};
%\node[box=1] (p-109-13) at (13,-109) {};
%\node[box=0] (p-109-14) at (14,-109) {};
%\node[box=0] (p-109-15) at (15,-109) {};
%\node[box=0] (p-109-16) at (16,-109) {};
%\node[box=0] (p-109-17) at (17,-109) {};
%\node[box=0] (p-109-18) at (18,-109) {};
%\node[box=0] (p-109-19) at (19,-109) {};
%\node[box=0] (p-109-20) at (20,-109) {};
%\node[box=0] (p-109-21) at (21,-109) {};
%\node[box=0] (p-109-22) at (22,-109) {};
%\node[box=0] (p-109-23) at (23,-109) {};
%\node[box=0] (p-109-24) at (24,-109) {};
%\node[box=0] (p-109-25) at (25,-109) {};
%\node[box=0] (p-109-26) at (26,-109) {};
%\node[box=0] (p-109-27) at (27,-109) {};
%\node[box=0] (p-109-28) at (28,-109) {};
%\node[box=0] (p-109-29) at (29,-109) {};
%\node[box=0] (p-109-30) at (30,-109) {};
%\node[box=0] (p-109-31) at (31,-109) {};
%\node[box=1] (p-109-32) at (32,-109) {};
%\node[box=1] (p-109-33) at (33,-109) {};
%\node[box=0] (p-109-34) at (34,-109) {};
%\node[box=0] (p-109-35) at (35,-109) {};
%\node[box=1] (p-109-36) at (36,-109) {};
%\node[box=1] (p-109-37) at (37,-109) {};
%\node[box=0] (p-109-38) at (38,-109) {};
%\node[box=0] (p-109-39) at (39,-109) {};
%\node[box=1] (p-109-40) at (40,-109) {};
%\node[box=1] (p-109-41) at (41,-109) {};
%\node[box=0] (p-109-42) at (42,-109) {};
%\node[box=0] (p-109-43) at (43,-109) {};
%\node[box=1] (p-109-44) at (44,-109) {};
%\node[box=1] (p-109-45) at (45,-109) {};
%\node[box=0] (p-109-46) at (46,-109) {};
%\node[box=0] (p-109-47) at (47,-109) {};
%\node[box=0] (p-109-48) at (48,-109) {};
%\node[box=0] (p-109-49) at (49,-109) {};
%\node[box=0] (p-109-50) at (50,-109) {};
%\node[box=0] (p-109-51) at (51,-109) {};
%\node[box=0] (p-109-52) at (52,-109) {};
%\node[box=0] (p-109-53) at (53,-109) {};
%\node[box=0] (p-109-54) at (54,-109) {};
%\node[box=0] (p-109-55) at (55,-109) {};
%\node[box=0] (p-109-56) at (56,-109) {};
%\node[box=0] (p-109-57) at (57,-109) {};
%\node[box=0] (p-109-58) at (58,-109) {};
%\node[box=0] (p-109-59) at (59,-109) {};
%\node[box=0] (p-109-60) at (60,-109) {};
%\node[box=0] (p-109-61) at (61,-109) {};
%\node[box=0] (p-109-62) at (62,-109) {};
%\node[box=0] (p-109-63) at (63,-109) {};
%\node[box=1] (p-109-64) at (64,-109) {};
%\node[box=1] (p-109-65) at (65,-109) {};
%\node[box=0] (p-109-66) at (66,-109) {};
%\node[box=0] (p-109-67) at (67,-109) {};
%\node[box=1] (p-109-68) at (68,-109) {};
%\node[box=1] (p-109-69) at (69,-109) {};
%\node[box=0] (p-109-70) at (70,-109) {};
%\node[box=0] (p-109-71) at (71,-109) {};
%\node[box=1] (p-109-72) at (72,-109) {};
%\node[box=1] (p-109-73) at (73,-109) {};
%\node[box=0] (p-109-74) at (74,-109) {};
%\node[box=0] (p-109-75) at (75,-109) {};
%\node[box=1] (p-109-76) at (76,-109) {};
%\node[box=1] (p-109-77) at (77,-109) {};
%\node[box=0] (p-109-78) at (78,-109) {};
%\node[box=0] (p-109-79) at (79,-109) {};
%\node[box=0] (p-109-80) at (80,-109) {};
%\node[box=0] (p-109-81) at (81,-109) {};
%\node[box=0] (p-109-82) at (82,-109) {};
%\node[box=0] (p-109-83) at (83,-109) {};
%\node[box=0] (p-109-84) at (84,-109) {};
%\node[box=0] (p-109-85) at (85,-109) {};
%\node[box=0] (p-109-86) at (86,-109) {};
%\node[box=0] (p-109-87) at (87,-109) {};
%\node[box=0] (p-109-88) at (88,-109) {};
%\node[box=0] (p-109-89) at (89,-109) {};
%\node[box=0] (p-109-90) at (90,-109) {};
%\node[box=0] (p-109-91) at (91,-109) {};
%\node[box=0] (p-109-92) at (92,-109) {};
%\node[box=0] (p-109-93) at (93,-109) {};
%\node[box=0] (p-109-94) at (94,-109) {};
%\node[box=0] (p-109-95) at (95,-109) {};
%\node[box=1] (p-109-96) at (96,-109) {};
%\node[box=1] (p-109-97) at (97,-109) {};
%\node[box=0] (p-109-98) at (98,-109) {};
%\node[box=0] (p-109-99) at (99,-109) {};
%\node[box=1] (p-109-100) at (100,-109) {};
%\node[box=1] (p-109-101) at (101,-109) {};
%\node[box=0] (p-109-102) at (102,-109) {};
%\node[box=0] (p-109-103) at (103,-109) {};
%\node[box=1] (p-109-104) at (104,-109) {};
%\node[box=1] (p-109-105) at (105,-109) {};
%\node[box=0] (p-109-106) at (106,-109) {};
%\node[box=0] (p-109-107) at (107,-109) {};
%\node[box=1] (p-109-108) at (108,-109) {};
%\node[box=1] (p-109-109) at (109,-109) {};
%\node[box=1] (p-110-0) at (0,-110) {};
%\node[box=0] (p-110-1) at (1,-110) {};
%\node[box=1] (p-110-2) at (2,-110) {};
%\node[box=0] (p-110-3) at (3,-110) {};
%\node[box=1] (p-110-4) at (4,-110) {};
%\node[box=0] (p-110-5) at (5,-110) {};
%\node[box=1] (p-110-6) at (6,-110) {};
%\node[box=0] (p-110-7) at (7,-110) {};
%\node[box=1] (p-110-8) at (8,-110) {};
%\node[box=0] (p-110-9) at (9,-110) {};
%\node[box=1] (p-110-10) at (10,-110) {};
%\node[box=0] (p-110-11) at (11,-110) {};
%\node[box=1] (p-110-12) at (12,-110) {};
%\node[box=0] (p-110-13) at (13,-110) {};
%\node[box=1] (p-110-14) at (14,-110) {};
%\node[box=0] (p-110-15) at (15,-110) {};
%\node[box=0] (p-110-16) at (16,-110) {};
%\node[box=0] (p-110-17) at (17,-110) {};
%\node[box=0] (p-110-18) at (18,-110) {};
%\node[box=0] (p-110-19) at (19,-110) {};
%\node[box=0] (p-110-20) at (20,-110) {};
%\node[box=0] (p-110-21) at (21,-110) {};
%\node[box=0] (p-110-22) at (22,-110) {};
%\node[box=0] (p-110-23) at (23,-110) {};
%\node[box=0] (p-110-24) at (24,-110) {};
%\node[box=0] (p-110-25) at (25,-110) {};
%\node[box=0] (p-110-26) at (26,-110) {};
%\node[box=0] (p-110-27) at (27,-110) {};
%\node[box=0] (p-110-28) at (28,-110) {};
%\node[box=0] (p-110-29) at (29,-110) {};
%\node[box=0] (p-110-30) at (30,-110) {};
%\node[box=0] (p-110-31) at (31,-110) {};
%\node[box=1] (p-110-32) at (32,-110) {};
%\node[box=0] (p-110-33) at (33,-110) {};
%\node[box=1] (p-110-34) at (34,-110) {};
%\node[box=0] (p-110-35) at (35,-110) {};
%\node[box=1] (p-110-36) at (36,-110) {};
%\node[box=0] (p-110-37) at (37,-110) {};
%\node[box=1] (p-110-38) at (38,-110) {};
%\node[box=0] (p-110-39) at (39,-110) {};
%\node[box=1] (p-110-40) at (40,-110) {};
%\node[box=0] (p-110-41) at (41,-110) {};
%\node[box=1] (p-110-42) at (42,-110) {};
%\node[box=0] (p-110-43) at (43,-110) {};
%\node[box=1] (p-110-44) at (44,-110) {};
%\node[box=0] (p-110-45) at (45,-110) {};
%\node[box=1] (p-110-46) at (46,-110) {};
%\node[box=0] (p-110-47) at (47,-110) {};
%\node[box=0] (p-110-48) at (48,-110) {};
%\node[box=0] (p-110-49) at (49,-110) {};
%\node[box=0] (p-110-50) at (50,-110) {};
%\node[box=0] (p-110-51) at (51,-110) {};
%\node[box=0] (p-110-52) at (52,-110) {};
%\node[box=0] (p-110-53) at (53,-110) {};
%\node[box=0] (p-110-54) at (54,-110) {};
%\node[box=0] (p-110-55) at (55,-110) {};
%\node[box=0] (p-110-56) at (56,-110) {};
%\node[box=0] (p-110-57) at (57,-110) {};
%\node[box=0] (p-110-58) at (58,-110) {};
%\node[box=0] (p-110-59) at (59,-110) {};
%\node[box=0] (p-110-60) at (60,-110) {};
%\node[box=0] (p-110-61) at (61,-110) {};
%\node[box=0] (p-110-62) at (62,-110) {};
%\node[box=0] (p-110-63) at (63,-110) {};
%\node[box=1] (p-110-64) at (64,-110) {};
%\node[box=0] (p-110-65) at (65,-110) {};
%\node[box=1] (p-110-66) at (66,-110) {};
%\node[box=0] (p-110-67) at (67,-110) {};
%\node[box=1] (p-110-68) at (68,-110) {};
%\node[box=0] (p-110-69) at (69,-110) {};
%\node[box=1] (p-110-70) at (70,-110) {};
%\node[box=0] (p-110-71) at (71,-110) {};
%\node[box=1] (p-110-72) at (72,-110) {};
%\node[box=0] (p-110-73) at (73,-110) {};
%\node[box=1] (p-110-74) at (74,-110) {};
%\node[box=0] (p-110-75) at (75,-110) {};
%\node[box=1] (p-110-76) at (76,-110) {};
%\node[box=0] (p-110-77) at (77,-110) {};
%\node[box=1] (p-110-78) at (78,-110) {};
%\node[box=0] (p-110-79) at (79,-110) {};
%\node[box=0] (p-110-80) at (80,-110) {};
%\node[box=0] (p-110-81) at (81,-110) {};
%\node[box=0] (p-110-82) at (82,-110) {};
%\node[box=0] (p-110-83) at (83,-110) {};
%\node[box=0] (p-110-84) at (84,-110) {};
%\node[box=0] (p-110-85) at (85,-110) {};
%\node[box=0] (p-110-86) at (86,-110) {};
%\node[box=0] (p-110-87) at (87,-110) {};
%\node[box=0] (p-110-88) at (88,-110) {};
%\node[box=0] (p-110-89) at (89,-110) {};
%\node[box=0] (p-110-90) at (90,-110) {};
%\node[box=0] (p-110-91) at (91,-110) {};
%\node[box=0] (p-110-92) at (92,-110) {};
%\node[box=0] (p-110-93) at (93,-110) {};
%\node[box=0] (p-110-94) at (94,-110) {};
%\node[box=0] (p-110-95) at (95,-110) {};
%\node[box=1] (p-110-96) at (96,-110) {};
%\node[box=0] (p-110-97) at (97,-110) {};
%\node[box=1] (p-110-98) at (98,-110) {};
%\node[box=0] (p-110-99) at (99,-110) {};
%\node[box=1] (p-110-100) at (100,-110) {};
%\node[box=0] (p-110-101) at (101,-110) {};
%\node[box=1] (p-110-102) at (102,-110) {};
%\node[box=0] (p-110-103) at (103,-110) {};
%\node[box=1] (p-110-104) at (104,-110) {};
%\node[box=0] (p-110-105) at (105,-110) {};
%\node[box=1] (p-110-106) at (106,-110) {};
%\node[box=0] (p-110-107) at (107,-110) {};
%\node[box=1] (p-110-108) at (108,-110) {};
%\node[box=0] (p-110-109) at (109,-110) {};
%\node[box=1] (p-110-110) at (110,-110) {};
%\node[box=1] (p-111-0) at (0,-111) {};
%\node[box=1] (p-111-1) at (1,-111) {};
%\node[box=1] (p-111-2) at (2,-111) {};
%\node[box=1] (p-111-3) at (3,-111) {};
%\node[box=1] (p-111-4) at (4,-111) {};
%\node[box=1] (p-111-5) at (5,-111) {};
%\node[box=1] (p-111-6) at (6,-111) {};
%\node[box=1] (p-111-7) at (7,-111) {};
%\node[box=1] (p-111-8) at (8,-111) {};
%\node[box=1] (p-111-9) at (9,-111) {};
%\node[box=1] (p-111-10) at (10,-111) {};
%\node[box=1] (p-111-11) at (11,-111) {};
%\node[box=1] (p-111-12) at (12,-111) {};
%\node[box=1] (p-111-13) at (13,-111) {};
%\node[box=1] (p-111-14) at (14,-111) {};
%\node[box=1] (p-111-15) at (15,-111) {};
%\node[box=0] (p-111-16) at (16,-111) {};
%\node[box=0] (p-111-17) at (17,-111) {};
%\node[box=0] (p-111-18) at (18,-111) {};
%\node[box=0] (p-111-19) at (19,-111) {};
%\node[box=0] (p-111-20) at (20,-111) {};
%\node[box=0] (p-111-21) at (21,-111) {};
%\node[box=0] (p-111-22) at (22,-111) {};
%\node[box=0] (p-111-23) at (23,-111) {};
%\node[box=0] (p-111-24) at (24,-111) {};
%\node[box=0] (p-111-25) at (25,-111) {};
%\node[box=0] (p-111-26) at (26,-111) {};
%\node[box=0] (p-111-27) at (27,-111) {};
%\node[box=0] (p-111-28) at (28,-111) {};
%\node[box=0] (p-111-29) at (29,-111) {};
%\node[box=0] (p-111-30) at (30,-111) {};
%\node[box=0] (p-111-31) at (31,-111) {};
%\node[box=1] (p-111-32) at (32,-111) {};
%\node[box=1] (p-111-33) at (33,-111) {};
%\node[box=1] (p-111-34) at (34,-111) {};
%\node[box=1] (p-111-35) at (35,-111) {};
%\node[box=1] (p-111-36) at (36,-111) {};
%\node[box=1] (p-111-37) at (37,-111) {};
%\node[box=1] (p-111-38) at (38,-111) {};
%\node[box=1] (p-111-39) at (39,-111) {};
%\node[box=1] (p-111-40) at (40,-111) {};
%\node[box=1] (p-111-41) at (41,-111) {};
%\node[box=1] (p-111-42) at (42,-111) {};
%\node[box=1] (p-111-43) at (43,-111) {};
%\node[box=1] (p-111-44) at (44,-111) {};
%\node[box=1] (p-111-45) at (45,-111) {};
%\node[box=1] (p-111-46) at (46,-111) {};
%\node[box=1] (p-111-47) at (47,-111) {};
%\node[box=0] (p-111-48) at (48,-111) {};
%\node[box=0] (p-111-49) at (49,-111) {};
%\node[box=0] (p-111-50) at (50,-111) {};
%\node[box=0] (p-111-51) at (51,-111) {};
%\node[box=0] (p-111-52) at (52,-111) {};
%\node[box=0] (p-111-53) at (53,-111) {};
%\node[box=0] (p-111-54) at (54,-111) {};
%\node[box=0] (p-111-55) at (55,-111) {};
%\node[box=0] (p-111-56) at (56,-111) {};
%\node[box=0] (p-111-57) at (57,-111) {};
%\node[box=0] (p-111-58) at (58,-111) {};
%\node[box=0] (p-111-59) at (59,-111) {};
%\node[box=0] (p-111-60) at (60,-111) {};
%\node[box=0] (p-111-61) at (61,-111) {};
%\node[box=0] (p-111-62) at (62,-111) {};
%\node[box=0] (p-111-63) at (63,-111) {};
%\node[box=1] (p-111-64) at (64,-111) {};
%\node[box=1] (p-111-65) at (65,-111) {};
%\node[box=1] (p-111-66) at (66,-111) {};
%\node[box=1] (p-111-67) at (67,-111) {};
%\node[box=1] (p-111-68) at (68,-111) {};
%\node[box=1] (p-111-69) at (69,-111) {};
%\node[box=1] (p-111-70) at (70,-111) {};
%\node[box=1] (p-111-71) at (71,-111) {};
%\node[box=1] (p-111-72) at (72,-111) {};
%\node[box=1] (p-111-73) at (73,-111) {};
%\node[box=1] (p-111-74) at (74,-111) {};
%\node[box=1] (p-111-75) at (75,-111) {};
%\node[box=1] (p-111-76) at (76,-111) {};
%\node[box=1] (p-111-77) at (77,-111) {};
%\node[box=1] (p-111-78) at (78,-111) {};
%\node[box=1] (p-111-79) at (79,-111) {};
%\node[box=0] (p-111-80) at (80,-111) {};
%\node[box=0] (p-111-81) at (81,-111) {};
%\node[box=0] (p-111-82) at (82,-111) {};
%\node[box=0] (p-111-83) at (83,-111) {};
%\node[box=0] (p-111-84) at (84,-111) {};
%\node[box=0] (p-111-85) at (85,-111) {};
%\node[box=0] (p-111-86) at (86,-111) {};
%\node[box=0] (p-111-87) at (87,-111) {};
%\node[box=0] (p-111-88) at (88,-111) {};
%\node[box=0] (p-111-89) at (89,-111) {};
%\node[box=0] (p-111-90) at (90,-111) {};
%\node[box=0] (p-111-91) at (91,-111) {};
%\node[box=0] (p-111-92) at (92,-111) {};
%\node[box=0] (p-111-93) at (93,-111) {};
%\node[box=0] (p-111-94) at (94,-111) {};
%\node[box=0] (p-111-95) at (95,-111) {};
%\node[box=1] (p-111-96) at (96,-111) {};
%\node[box=1] (p-111-97) at (97,-111) {};
%\node[box=1] (p-111-98) at (98,-111) {};
%\node[box=1] (p-111-99) at (99,-111) {};
%\node[box=1] (p-111-100) at (100,-111) {};
%\node[box=1] (p-111-101) at (101,-111) {};
%\node[box=1] (p-111-102) at (102,-111) {};
%\node[box=1] (p-111-103) at (103,-111) {};
%\node[box=1] (p-111-104) at (104,-111) {};
%\node[box=1] (p-111-105) at (105,-111) {};
%\node[box=1] (p-111-106) at (106,-111) {};
%\node[box=1] (p-111-107) at (107,-111) {};
%\node[box=1] (p-111-108) at (108,-111) {};
%\node[box=1] (p-111-109) at (109,-111) {};
%\node[box=1] (p-111-110) at (110,-111) {};
%\node[box=1] (p-111-111) at (111,-111) {};
%\node[box=1] (p-112-0) at (0,-112) {};
%\node[box=0] (p-112-1) at (1,-112) {};
%\node[box=0] (p-112-2) at (2,-112) {};
%\node[box=0] (p-112-3) at (3,-112) {};
%\node[box=0] (p-112-4) at (4,-112) {};
%\node[box=0] (p-112-5) at (5,-112) {};
%\node[box=0] (p-112-6) at (6,-112) {};
%\node[box=0] (p-112-7) at (7,-112) {};
%\node[box=0] (p-112-8) at (8,-112) {};
%\node[box=0] (p-112-9) at (9,-112) {};
%\node[box=0] (p-112-10) at (10,-112) {};
%\node[box=0] (p-112-11) at (11,-112) {};
%\node[box=0] (p-112-12) at (12,-112) {};
%\node[box=0] (p-112-13) at (13,-112) {};
%\node[box=0] (p-112-14) at (14,-112) {};
%\node[box=0] (p-112-15) at (15,-112) {};
%\node[box=1] (p-112-16) at (16,-112) {};
%\node[box=0] (p-112-17) at (17,-112) {};
%\node[box=0] (p-112-18) at (18,-112) {};
%\node[box=0] (p-112-19) at (19,-112) {};
%\node[box=0] (p-112-20) at (20,-112) {};
%\node[box=0] (p-112-21) at (21,-112) {};
%\node[box=0] (p-112-22) at (22,-112) {};
%\node[box=0] (p-112-23) at (23,-112) {};
%\node[box=0] (p-112-24) at (24,-112) {};
%\node[box=0] (p-112-25) at (25,-112) {};
%\node[box=0] (p-112-26) at (26,-112) {};
%\node[box=0] (p-112-27) at (27,-112) {};
%\node[box=0] (p-112-28) at (28,-112) {};
%\node[box=0] (p-112-29) at (29,-112) {};
%\node[box=0] (p-112-30) at (30,-112) {};
%\node[box=0] (p-112-31) at (31,-112) {};
%\node[box=1] (p-112-32) at (32,-112) {};
%\node[box=0] (p-112-33) at (33,-112) {};
%\node[box=0] (p-112-34) at (34,-112) {};
%\node[box=0] (p-112-35) at (35,-112) {};
%\node[box=0] (p-112-36) at (36,-112) {};
%\node[box=0] (p-112-37) at (37,-112) {};
%\node[box=0] (p-112-38) at (38,-112) {};
%\node[box=0] (p-112-39) at (39,-112) {};
%\node[box=0] (p-112-40) at (40,-112) {};
%\node[box=0] (p-112-41) at (41,-112) {};
%\node[box=0] (p-112-42) at (42,-112) {};
%\node[box=0] (p-112-43) at (43,-112) {};
%\node[box=0] (p-112-44) at (44,-112) {};
%\node[box=0] (p-112-45) at (45,-112) {};
%\node[box=0] (p-112-46) at (46,-112) {};
%\node[box=0] (p-112-47) at (47,-112) {};
%\node[box=1] (p-112-48) at (48,-112) {};
%\node[box=0] (p-112-49) at (49,-112) {};
%\node[box=0] (p-112-50) at (50,-112) {};
%\node[box=0] (p-112-51) at (51,-112) {};
%\node[box=0] (p-112-52) at (52,-112) {};
%\node[box=0] (p-112-53) at (53,-112) {};
%\node[box=0] (p-112-54) at (54,-112) {};
%\node[box=0] (p-112-55) at (55,-112) {};
%\node[box=0] (p-112-56) at (56,-112) {};
%\node[box=0] (p-112-57) at (57,-112) {};
%\node[box=0] (p-112-58) at (58,-112) {};
%\node[box=0] (p-112-59) at (59,-112) {};
%\node[box=0] (p-112-60) at (60,-112) {};
%\node[box=0] (p-112-61) at (61,-112) {};
%\node[box=0] (p-112-62) at (62,-112) {};
%\node[box=0] (p-112-63) at (63,-112) {};
%\node[box=1] (p-112-64) at (64,-112) {};
%\node[box=0] (p-112-65) at (65,-112) {};
%\node[box=0] (p-112-66) at (66,-112) {};
%\node[box=0] (p-112-67) at (67,-112) {};
%\node[box=0] (p-112-68) at (68,-112) {};
%\node[box=0] (p-112-69) at (69,-112) {};
%\node[box=0] (p-112-70) at (70,-112) {};
%\node[box=0] (p-112-71) at (71,-112) {};
%\node[box=0] (p-112-72) at (72,-112) {};
%\node[box=0] (p-112-73) at (73,-112) {};
%\node[box=0] (p-112-74) at (74,-112) {};
%\node[box=0] (p-112-75) at (75,-112) {};
%\node[box=0] (p-112-76) at (76,-112) {};
%\node[box=0] (p-112-77) at (77,-112) {};
%\node[box=0] (p-112-78) at (78,-112) {};
%\node[box=0] (p-112-79) at (79,-112) {};
%\node[box=1] (p-112-80) at (80,-112) {};
%\node[box=0] (p-112-81) at (81,-112) {};
%\node[box=0] (p-112-82) at (82,-112) {};
%\node[box=0] (p-112-83) at (83,-112) {};
%\node[box=0] (p-112-84) at (84,-112) {};
%\node[box=0] (p-112-85) at (85,-112) {};
%\node[box=0] (p-112-86) at (86,-112) {};
%\node[box=0] (p-112-87) at (87,-112) {};
%\node[box=0] (p-112-88) at (88,-112) {};
%\node[box=0] (p-112-89) at (89,-112) {};
%\node[box=0] (p-112-90) at (90,-112) {};
%\node[box=0] (p-112-91) at (91,-112) {};
%\node[box=0] (p-112-92) at (92,-112) {};
%\node[box=0] (p-112-93) at (93,-112) {};
%\node[box=0] (p-112-94) at (94,-112) {};
%\node[box=0] (p-112-95) at (95,-112) {};
%\node[box=1] (p-112-96) at (96,-112) {};
%\node[box=0] (p-112-97) at (97,-112) {};
%\node[box=0] (p-112-98) at (98,-112) {};
%\node[box=0] (p-112-99) at (99,-112) {};
%\node[box=0] (p-112-100) at (100,-112) {};
%\node[box=0] (p-112-101) at (101,-112) {};
%\node[box=0] (p-112-102) at (102,-112) {};
%\node[box=0] (p-112-103) at (103,-112) {};
%\node[box=0] (p-112-104) at (104,-112) {};
%\node[box=0] (p-112-105) at (105,-112) {};
%\node[box=0] (p-112-106) at (106,-112) {};
%\node[box=0] (p-112-107) at (107,-112) {};
%\node[box=0] (p-112-108) at (108,-112) {};
%\node[box=0] (p-112-109) at (109,-112) {};
%\node[box=0] (p-112-110) at (110,-112) {};
%\node[box=0] (p-112-111) at (111,-112) {};
%\node[box=1] (p-112-112) at (112,-112) {};
%\node[box=1] (p-113-0) at (0,-113) {};
%\node[box=1] (p-113-1) at (1,-113) {};
%\node[box=0] (p-113-2) at (2,-113) {};
%\node[box=0] (p-113-3) at (3,-113) {};
%\node[box=0] (p-113-4) at (4,-113) {};
%\node[box=0] (p-113-5) at (5,-113) {};
%\node[box=0] (p-113-6) at (6,-113) {};
%\node[box=0] (p-113-7) at (7,-113) {};
%\node[box=0] (p-113-8) at (8,-113) {};
%\node[box=0] (p-113-9) at (9,-113) {};
%\node[box=0] (p-113-10) at (10,-113) {};
%\node[box=0] (p-113-11) at (11,-113) {};
%\node[box=0] (p-113-12) at (12,-113) {};
%\node[box=0] (p-113-13) at (13,-113) {};
%\node[box=0] (p-113-14) at (14,-113) {};
%\node[box=0] (p-113-15) at (15,-113) {};
%\node[box=1] (p-113-16) at (16,-113) {};
%\node[box=1] (p-113-17) at (17,-113) {};
%\node[box=0] (p-113-18) at (18,-113) {};
%\node[box=0] (p-113-19) at (19,-113) {};
%\node[box=0] (p-113-20) at (20,-113) {};
%\node[box=0] (p-113-21) at (21,-113) {};
%\node[box=0] (p-113-22) at (22,-113) {};
%\node[box=0] (p-113-23) at (23,-113) {};
%\node[box=0] (p-113-24) at (24,-113) {};
%\node[box=0] (p-113-25) at (25,-113) {};
%\node[box=0] (p-113-26) at (26,-113) {};
%\node[box=0] (p-113-27) at (27,-113) {};
%\node[box=0] (p-113-28) at (28,-113) {};
%\node[box=0] (p-113-29) at (29,-113) {};
%\node[box=0] (p-113-30) at (30,-113) {};
%\node[box=0] (p-113-31) at (31,-113) {};
%\node[box=1] (p-113-32) at (32,-113) {};
%\node[box=1] (p-113-33) at (33,-113) {};
%\node[box=0] (p-113-34) at (34,-113) {};
%\node[box=0] (p-113-35) at (35,-113) {};
%\node[box=0] (p-113-36) at (36,-113) {};
%\node[box=0] (p-113-37) at (37,-113) {};
%\node[box=0] (p-113-38) at (38,-113) {};
%\node[box=0] (p-113-39) at (39,-113) {};
%\node[box=0] (p-113-40) at (40,-113) {};
%\node[box=0] (p-113-41) at (41,-113) {};
%\node[box=0] (p-113-42) at (42,-113) {};
%\node[box=0] (p-113-43) at (43,-113) {};
%\node[box=0] (p-113-44) at (44,-113) {};
%\node[box=0] (p-113-45) at (45,-113) {};
%\node[box=0] (p-113-46) at (46,-113) {};
%\node[box=0] (p-113-47) at (47,-113) {};
%\node[box=1] (p-113-48) at (48,-113) {};
%\node[box=1] (p-113-49) at (49,-113) {};
%\node[box=0] (p-113-50) at (50,-113) {};
%\node[box=0] (p-113-51) at (51,-113) {};
%\node[box=0] (p-113-52) at (52,-113) {};
%\node[box=0] (p-113-53) at (53,-113) {};
%\node[box=0] (p-113-54) at (54,-113) {};
%\node[box=0] (p-113-55) at (55,-113) {};
%\node[box=0] (p-113-56) at (56,-113) {};
%\node[box=0] (p-113-57) at (57,-113) {};
%\node[box=0] (p-113-58) at (58,-113) {};
%\node[box=0] (p-113-59) at (59,-113) {};
%\node[box=0] (p-113-60) at (60,-113) {};
%\node[box=0] (p-113-61) at (61,-113) {};
%\node[box=0] (p-113-62) at (62,-113) {};
%\node[box=0] (p-113-63) at (63,-113) {};
%\node[box=1] (p-113-64) at (64,-113) {};
%\node[box=1] (p-113-65) at (65,-113) {};
%\node[box=0] (p-113-66) at (66,-113) {};
%\node[box=0] (p-113-67) at (67,-113) {};
%\node[box=0] (p-113-68) at (68,-113) {};
%\node[box=0] (p-113-69) at (69,-113) {};
%\node[box=0] (p-113-70) at (70,-113) {};
%\node[box=0] (p-113-71) at (71,-113) {};
%\node[box=0] (p-113-72) at (72,-113) {};
%\node[box=0] (p-113-73) at (73,-113) {};
%\node[box=0] (p-113-74) at (74,-113) {};
%\node[box=0] (p-113-75) at (75,-113) {};
%\node[box=0] (p-113-76) at (76,-113) {};
%\node[box=0] (p-113-77) at (77,-113) {};
%\node[box=0] (p-113-78) at (78,-113) {};
%\node[box=0] (p-113-79) at (79,-113) {};
%\node[box=1] (p-113-80) at (80,-113) {};
%\node[box=1] (p-113-81) at (81,-113) {};
%\node[box=0] (p-113-82) at (82,-113) {};
%\node[box=0] (p-113-83) at (83,-113) {};
%\node[box=0] (p-113-84) at (84,-113) {};
%\node[box=0] (p-113-85) at (85,-113) {};
%\node[box=0] (p-113-86) at (86,-113) {};
%\node[box=0] (p-113-87) at (87,-113) {};
%\node[box=0] (p-113-88) at (88,-113) {};
%\node[box=0] (p-113-89) at (89,-113) {};
%\node[box=0] (p-113-90) at (90,-113) {};
%\node[box=0] (p-113-91) at (91,-113) {};
%\node[box=0] (p-113-92) at (92,-113) {};
%\node[box=0] (p-113-93) at (93,-113) {};
%\node[box=0] (p-113-94) at (94,-113) {};
%\node[box=0] (p-113-95) at (95,-113) {};
%\node[box=1] (p-113-96) at (96,-113) {};
%\node[box=1] (p-113-97) at (97,-113) {};
%\node[box=0] (p-113-98) at (98,-113) {};
%\node[box=0] (p-113-99) at (99,-113) {};
%\node[box=0] (p-113-100) at (100,-113) {};
%\node[box=0] (p-113-101) at (101,-113) {};
%\node[box=0] (p-113-102) at (102,-113) {};
%\node[box=0] (p-113-103) at (103,-113) {};
%\node[box=0] (p-113-104) at (104,-113) {};
%\node[box=0] (p-113-105) at (105,-113) {};
%\node[box=0] (p-113-106) at (106,-113) {};
%\node[box=0] (p-113-107) at (107,-113) {};
%\node[box=0] (p-113-108) at (108,-113) {};
%\node[box=0] (p-113-109) at (109,-113) {};
%\node[box=0] (p-113-110) at (110,-113) {};
%\node[box=0] (p-113-111) at (111,-113) {};
%\node[box=1] (p-113-112) at (112,-113) {};
%\node[box=1] (p-113-113) at (113,-113) {};
%\node[box=1] (p-114-0) at (0,-114) {};
%\node[box=0] (p-114-1) at (1,-114) {};
%\node[box=1] (p-114-2) at (2,-114) {};
%\node[box=0] (p-114-3) at (3,-114) {};
%\node[box=0] (p-114-4) at (4,-114) {};
%\node[box=0] (p-114-5) at (5,-114) {};
%\node[box=0] (p-114-6) at (6,-114) {};
%\node[box=0] (p-114-7) at (7,-114) {};
%\node[box=0] (p-114-8) at (8,-114) {};
%\node[box=0] (p-114-9) at (9,-114) {};
%\node[box=0] (p-114-10) at (10,-114) {};
%\node[box=0] (p-114-11) at (11,-114) {};
%\node[box=0] (p-114-12) at (12,-114) {};
%\node[box=0] (p-114-13) at (13,-114) {};
%\node[box=0] (p-114-14) at (14,-114) {};
%\node[box=0] (p-114-15) at (15,-114) {};
%\node[box=1] (p-114-16) at (16,-114) {};
%\node[box=0] (p-114-17) at (17,-114) {};
%\node[box=1] (p-114-18) at (18,-114) {};
%\node[box=0] (p-114-19) at (19,-114) {};
%\node[box=0] (p-114-20) at (20,-114) {};
%\node[box=0] (p-114-21) at (21,-114) {};
%\node[box=0] (p-114-22) at (22,-114) {};
%\node[box=0] (p-114-23) at (23,-114) {};
%\node[box=0] (p-114-24) at (24,-114) {};
%\node[box=0] (p-114-25) at (25,-114) {};
%\node[box=0] (p-114-26) at (26,-114) {};
%\node[box=0] (p-114-27) at (27,-114) {};
%\node[box=0] (p-114-28) at (28,-114) {};
%\node[box=0] (p-114-29) at (29,-114) {};
%\node[box=0] (p-114-30) at (30,-114) {};
%\node[box=0] (p-114-31) at (31,-114) {};
%\node[box=1] (p-114-32) at (32,-114) {};
%\node[box=0] (p-114-33) at (33,-114) {};
%\node[box=1] (p-114-34) at (34,-114) {};
%\node[box=0] (p-114-35) at (35,-114) {};
%\node[box=0] (p-114-36) at (36,-114) {};
%\node[box=0] (p-114-37) at (37,-114) {};
%\node[box=0] (p-114-38) at (38,-114) {};
%\node[box=0] (p-114-39) at (39,-114) {};
%\node[box=0] (p-114-40) at (40,-114) {};
%\node[box=0] (p-114-41) at (41,-114) {};
%\node[box=0] (p-114-42) at (42,-114) {};
%\node[box=0] (p-114-43) at (43,-114) {};
%\node[box=0] (p-114-44) at (44,-114) {};
%\node[box=0] (p-114-45) at (45,-114) {};
%\node[box=0] (p-114-46) at (46,-114) {};
%\node[box=0] (p-114-47) at (47,-114) {};
%\node[box=1] (p-114-48) at (48,-114) {};
%\node[box=0] (p-114-49) at (49,-114) {};
%\node[box=1] (p-114-50) at (50,-114) {};
%\node[box=0] (p-114-51) at (51,-114) {};
%\node[box=0] (p-114-52) at (52,-114) {};
%\node[box=0] (p-114-53) at (53,-114) {};
%\node[box=0] (p-114-54) at (54,-114) {};
%\node[box=0] (p-114-55) at (55,-114) {};
%\node[box=0] (p-114-56) at (56,-114) {};
%\node[box=0] (p-114-57) at (57,-114) {};
%\node[box=0] (p-114-58) at (58,-114) {};
%\node[box=0] (p-114-59) at (59,-114) {};
%\node[box=0] (p-114-60) at (60,-114) {};
%\node[box=0] (p-114-61) at (61,-114) {};
%\node[box=0] (p-114-62) at (62,-114) {};
%\node[box=0] (p-114-63) at (63,-114) {};
%\node[box=1] (p-114-64) at (64,-114) {};
%\node[box=0] (p-114-65) at (65,-114) {};
%\node[box=1] (p-114-66) at (66,-114) {};
%\node[box=0] (p-114-67) at (67,-114) {};
%\node[box=0] (p-114-68) at (68,-114) {};
%\node[box=0] (p-114-69) at (69,-114) {};
%\node[box=0] (p-114-70) at (70,-114) {};
%\node[box=0] (p-114-71) at (71,-114) {};
%\node[box=0] (p-114-72) at (72,-114) {};
%\node[box=0] (p-114-73) at (73,-114) {};
%\node[box=0] (p-114-74) at (74,-114) {};
%\node[box=0] (p-114-75) at (75,-114) {};
%\node[box=0] (p-114-76) at (76,-114) {};
%\node[box=0] (p-114-77) at (77,-114) {};
%\node[box=0] (p-114-78) at (78,-114) {};
%\node[box=0] (p-114-79) at (79,-114) {};
%\node[box=1] (p-114-80) at (80,-114) {};
%\node[box=0] (p-114-81) at (81,-114) {};
%\node[box=1] (p-114-82) at (82,-114) {};
%\node[box=0] (p-114-83) at (83,-114) {};
%\node[box=0] (p-114-84) at (84,-114) {};
%\node[box=0] (p-114-85) at (85,-114) {};
%\node[box=0] (p-114-86) at (86,-114) {};
%\node[box=0] (p-114-87) at (87,-114) {};
%\node[box=0] (p-114-88) at (88,-114) {};
%\node[box=0] (p-114-89) at (89,-114) {};
%\node[box=0] (p-114-90) at (90,-114) {};
%\node[box=0] (p-114-91) at (91,-114) {};
%\node[box=0] (p-114-92) at (92,-114) {};
%\node[box=0] (p-114-93) at (93,-114) {};
%\node[box=0] (p-114-94) at (94,-114) {};
%\node[box=0] (p-114-95) at (95,-114) {};
%\node[box=1] (p-114-96) at (96,-114) {};
%\node[box=0] (p-114-97) at (97,-114) {};
%\node[box=1] (p-114-98) at (98,-114) {};
%\node[box=0] (p-114-99) at (99,-114) {};
%\node[box=0] (p-114-100) at (100,-114) {};
%\node[box=0] (p-114-101) at (101,-114) {};
%\node[box=0] (p-114-102) at (102,-114) {};
%\node[box=0] (p-114-103) at (103,-114) {};
%\node[box=0] (p-114-104) at (104,-114) {};
%\node[box=0] (p-114-105) at (105,-114) {};
%\node[box=0] (p-114-106) at (106,-114) {};
%\node[box=0] (p-114-107) at (107,-114) {};
%\node[box=0] (p-114-108) at (108,-114) {};
%\node[box=0] (p-114-109) at (109,-114) {};
%\node[box=0] (p-114-110) at (110,-114) {};
%\node[box=0] (p-114-111) at (111,-114) {};
%\node[box=1] (p-114-112) at (112,-114) {};
%\node[box=0] (p-114-113) at (113,-114) {};
%\node[box=1] (p-114-114) at (114,-114) {};
%\node[box=1] (p-115-0) at (0,-115) {};
%\node[box=1] (p-115-1) at (1,-115) {};
%\node[box=1] (p-115-2) at (2,-115) {};
%\node[box=1] (p-115-3) at (3,-115) {};
%\node[box=0] (p-115-4) at (4,-115) {};
%\node[box=0] (p-115-5) at (5,-115) {};
%\node[box=0] (p-115-6) at (6,-115) {};
%\node[box=0] (p-115-7) at (7,-115) {};
%\node[box=0] (p-115-8) at (8,-115) {};
%\node[box=0] (p-115-9) at (9,-115) {};
%\node[box=0] (p-115-10) at (10,-115) {};
%\node[box=0] (p-115-11) at (11,-115) {};
%\node[box=0] (p-115-12) at (12,-115) {};
%\node[box=0] (p-115-13) at (13,-115) {};
%\node[box=0] (p-115-14) at (14,-115) {};
%\node[box=0] (p-115-15) at (15,-115) {};
%\node[box=1] (p-115-16) at (16,-115) {};
%\node[box=1] (p-115-17) at (17,-115) {};
%\node[box=1] (p-115-18) at (18,-115) {};
%\node[box=1] (p-115-19) at (19,-115) {};
%\node[box=0] (p-115-20) at (20,-115) {};
%\node[box=0] (p-115-21) at (21,-115) {};
%\node[box=0] (p-115-22) at (22,-115) {};
%\node[box=0] (p-115-23) at (23,-115) {};
%\node[box=0] (p-115-24) at (24,-115) {};
%\node[box=0] (p-115-25) at (25,-115) {};
%\node[box=0] (p-115-26) at (26,-115) {};
%\node[box=0] (p-115-27) at (27,-115) {};
%\node[box=0] (p-115-28) at (28,-115) {};
%\node[box=0] (p-115-29) at (29,-115) {};
%\node[box=0] (p-115-30) at (30,-115) {};
%\node[box=0] (p-115-31) at (31,-115) {};
%\node[box=1] (p-115-32) at (32,-115) {};
%\node[box=1] (p-115-33) at (33,-115) {};
%\node[box=1] (p-115-34) at (34,-115) {};
%\node[box=1] (p-115-35) at (35,-115) {};
%\node[box=0] (p-115-36) at (36,-115) {};
%\node[box=0] (p-115-37) at (37,-115) {};
%\node[box=0] (p-115-38) at (38,-115) {};
%\node[box=0] (p-115-39) at (39,-115) {};
%\node[box=0] (p-115-40) at (40,-115) {};
%\node[box=0] (p-115-41) at (41,-115) {};
%\node[box=0] (p-115-42) at (42,-115) {};
%\node[box=0] (p-115-43) at (43,-115) {};
%\node[box=0] (p-115-44) at (44,-115) {};
%\node[box=0] (p-115-45) at (45,-115) {};
%\node[box=0] (p-115-46) at (46,-115) {};
%\node[box=0] (p-115-47) at (47,-115) {};
%\node[box=1] (p-115-48) at (48,-115) {};
%\node[box=1] (p-115-49) at (49,-115) {};
%\node[box=1] (p-115-50) at (50,-115) {};
%\node[box=1] (p-115-51) at (51,-115) {};
%\node[box=0] (p-115-52) at (52,-115) {};
%\node[box=0] (p-115-53) at (53,-115) {};
%\node[box=0] (p-115-54) at (54,-115) {};
%\node[box=0] (p-115-55) at (55,-115) {};
%\node[box=0] (p-115-56) at (56,-115) {};
%\node[box=0] (p-115-57) at (57,-115) {};
%\node[box=0] (p-115-58) at (58,-115) {};
%\node[box=0] (p-115-59) at (59,-115) {};
%\node[box=0] (p-115-60) at (60,-115) {};
%\node[box=0] (p-115-61) at (61,-115) {};
%\node[box=0] (p-115-62) at (62,-115) {};
%\node[box=0] (p-115-63) at (63,-115) {};
%\node[box=1] (p-115-64) at (64,-115) {};
%\node[box=1] (p-115-65) at (65,-115) {};
%\node[box=1] (p-115-66) at (66,-115) {};
%\node[box=1] (p-115-67) at (67,-115) {};
%\node[box=0] (p-115-68) at (68,-115) {};
%\node[box=0] (p-115-69) at (69,-115) {};
%\node[box=0] (p-115-70) at (70,-115) {};
%\node[box=0] (p-115-71) at (71,-115) {};
%\node[box=0] (p-115-72) at (72,-115) {};
%\node[box=0] (p-115-73) at (73,-115) {};
%\node[box=0] (p-115-74) at (74,-115) {};
%\node[box=0] (p-115-75) at (75,-115) {};
%\node[box=0] (p-115-76) at (76,-115) {};
%\node[box=0] (p-115-77) at (77,-115) {};
%\node[box=0] (p-115-78) at (78,-115) {};
%\node[box=0] (p-115-79) at (79,-115) {};
%\node[box=1] (p-115-80) at (80,-115) {};
%\node[box=1] (p-115-81) at (81,-115) {};
%\node[box=1] (p-115-82) at (82,-115) {};
%\node[box=1] (p-115-83) at (83,-115) {};
%\node[box=0] (p-115-84) at (84,-115) {};
%\node[box=0] (p-115-85) at (85,-115) {};
%\node[box=0] (p-115-86) at (86,-115) {};
%\node[box=0] (p-115-87) at (87,-115) {};
%\node[box=0] (p-115-88) at (88,-115) {};
%\node[box=0] (p-115-89) at (89,-115) {};
%\node[box=0] (p-115-90) at (90,-115) {};
%\node[box=0] (p-115-91) at (91,-115) {};
%\node[box=0] (p-115-92) at (92,-115) {};
%\node[box=0] (p-115-93) at (93,-115) {};
%\node[box=0] (p-115-94) at (94,-115) {};
%\node[box=0] (p-115-95) at (95,-115) {};
%\node[box=1] (p-115-96) at (96,-115) {};
%\node[box=1] (p-115-97) at (97,-115) {};
%\node[box=1] (p-115-98) at (98,-115) {};
%\node[box=1] (p-115-99) at (99,-115) {};
%\node[box=0] (p-115-100) at (100,-115) {};
%\node[box=0] (p-115-101) at (101,-115) {};
%\node[box=0] (p-115-102) at (102,-115) {};
%\node[box=0] (p-115-103) at (103,-115) {};
%\node[box=0] (p-115-104) at (104,-115) {};
%\node[box=0] (p-115-105) at (105,-115) {};
%\node[box=0] (p-115-106) at (106,-115) {};
%\node[box=0] (p-115-107) at (107,-115) {};
%\node[box=0] (p-115-108) at (108,-115) {};
%\node[box=0] (p-115-109) at (109,-115) {};
%\node[box=0] (p-115-110) at (110,-115) {};
%\node[box=0] (p-115-111) at (111,-115) {};
%\node[box=1] (p-115-112) at (112,-115) {};
%\node[box=1] (p-115-113) at (113,-115) {};
%\node[box=1] (p-115-114) at (114,-115) {};
%\node[box=1] (p-115-115) at (115,-115) {};
%\node[box=1] (p-116-0) at (0,-116) {};
%\node[box=0] (p-116-1) at (1,-116) {};
%\node[box=0] (p-116-2) at (2,-116) {};
%\node[box=0] (p-116-3) at (3,-116) {};
%\node[box=1] (p-116-4) at (4,-116) {};
%\node[box=0] (p-116-5) at (5,-116) {};
%\node[box=0] (p-116-6) at (6,-116) {};
%\node[box=0] (p-116-7) at (7,-116) {};
%\node[box=0] (p-116-8) at (8,-116) {};
%\node[box=0] (p-116-9) at (9,-116) {};
%\node[box=0] (p-116-10) at (10,-116) {};
%\node[box=0] (p-116-11) at (11,-116) {};
%\node[box=0] (p-116-12) at (12,-116) {};
%\node[box=0] (p-116-13) at (13,-116) {};
%\node[box=0] (p-116-14) at (14,-116) {};
%\node[box=0] (p-116-15) at (15,-116) {};
%\node[box=1] (p-116-16) at (16,-116) {};
%\node[box=0] (p-116-17) at (17,-116) {};
%\node[box=0] (p-116-18) at (18,-116) {};
%\node[box=0] (p-116-19) at (19,-116) {};
%\node[box=1] (p-116-20) at (20,-116) {};
%\node[box=0] (p-116-21) at (21,-116) {};
%\node[box=0] (p-116-22) at (22,-116) {};
%\node[box=0] (p-116-23) at (23,-116) {};
%\node[box=0] (p-116-24) at (24,-116) {};
%\node[box=0] (p-116-25) at (25,-116) {};
%\node[box=0] (p-116-26) at (26,-116) {};
%\node[box=0] (p-116-27) at (27,-116) {};
%\node[box=0] (p-116-28) at (28,-116) {};
%\node[box=0] (p-116-29) at (29,-116) {};
%\node[box=0] (p-116-30) at (30,-116) {};
%\node[box=0] (p-116-31) at (31,-116) {};
%\node[box=1] (p-116-32) at (32,-116) {};
%\node[box=0] (p-116-33) at (33,-116) {};
%\node[box=0] (p-116-34) at (34,-116) {};
%\node[box=0] (p-116-35) at (35,-116) {};
%\node[box=1] (p-116-36) at (36,-116) {};
%\node[box=0] (p-116-37) at (37,-116) {};
%\node[box=0] (p-116-38) at (38,-116) {};
%\node[box=0] (p-116-39) at (39,-116) {};
%\node[box=0] (p-116-40) at (40,-116) {};
%\node[box=0] (p-116-41) at (41,-116) {};
%\node[box=0] (p-116-42) at (42,-116) {};
%\node[box=0] (p-116-43) at (43,-116) {};
%\node[box=0] (p-116-44) at (44,-116) {};
%\node[box=0] (p-116-45) at (45,-116) {};
%\node[box=0] (p-116-46) at (46,-116) {};
%\node[box=0] (p-116-47) at (47,-116) {};
%\node[box=1] (p-116-48) at (48,-116) {};
%\node[box=0] (p-116-49) at (49,-116) {};
%\node[box=0] (p-116-50) at (50,-116) {};
%\node[box=0] (p-116-51) at (51,-116) {};
%\node[box=1] (p-116-52) at (52,-116) {};
%\node[box=0] (p-116-53) at (53,-116) {};
%\node[box=0] (p-116-54) at (54,-116) {};
%\node[box=0] (p-116-55) at (55,-116) {};
%\node[box=0] (p-116-56) at (56,-116) {};
%\node[box=0] (p-116-57) at (57,-116) {};
%\node[box=0] (p-116-58) at (58,-116) {};
%\node[box=0] (p-116-59) at (59,-116) {};
%\node[box=0] (p-116-60) at (60,-116) {};
%\node[box=0] (p-116-61) at (61,-116) {};
%\node[box=0] (p-116-62) at (62,-116) {};
%\node[box=0] (p-116-63) at (63,-116) {};
%\node[box=1] (p-116-64) at (64,-116) {};
%\node[box=0] (p-116-65) at (65,-116) {};
%\node[box=0] (p-116-66) at (66,-116) {};
%\node[box=0] (p-116-67) at (67,-116) {};
%\node[box=1] (p-116-68) at (68,-116) {};
%\node[box=0] (p-116-69) at (69,-116) {};
%\node[box=0] (p-116-70) at (70,-116) {};
%\node[box=0] (p-116-71) at (71,-116) {};
%\node[box=0] (p-116-72) at (72,-116) {};
%\node[box=0] (p-116-73) at (73,-116) {};
%\node[box=0] (p-116-74) at (74,-116) {};
%\node[box=0] (p-116-75) at (75,-116) {};
%\node[box=0] (p-116-76) at (76,-116) {};
%\node[box=0] (p-116-77) at (77,-116) {};
%\node[box=0] (p-116-78) at (78,-116) {};
%\node[box=0] (p-116-79) at (79,-116) {};
%\node[box=1] (p-116-80) at (80,-116) {};
%\node[box=0] (p-116-81) at (81,-116) {};
%\node[box=0] (p-116-82) at (82,-116) {};
%\node[box=0] (p-116-83) at (83,-116) {};
%\node[box=1] (p-116-84) at (84,-116) {};
%\node[box=0] (p-116-85) at (85,-116) {};
%\node[box=0] (p-116-86) at (86,-116) {};
%\node[box=0] (p-116-87) at (87,-116) {};
%\node[box=0] (p-116-88) at (88,-116) {};
%\node[box=0] (p-116-89) at (89,-116) {};
%\node[box=0] (p-116-90) at (90,-116) {};
%\node[box=0] (p-116-91) at (91,-116) {};
%\node[box=0] (p-116-92) at (92,-116) {};
%\node[box=0] (p-116-93) at (93,-116) {};
%\node[box=0] (p-116-94) at (94,-116) {};
%\node[box=0] (p-116-95) at (95,-116) {};
%\node[box=1] (p-116-96) at (96,-116) {};
%\node[box=0] (p-116-97) at (97,-116) {};
%\node[box=0] (p-116-98) at (98,-116) {};
%\node[box=0] (p-116-99) at (99,-116) {};
%\node[box=1] (p-116-100) at (100,-116) {};
%\node[box=0] (p-116-101) at (101,-116) {};
%\node[box=0] (p-116-102) at (102,-116) {};
%\node[box=0] (p-116-103) at (103,-116) {};
%\node[box=0] (p-116-104) at (104,-116) {};
%\node[box=0] (p-116-105) at (105,-116) {};
%\node[box=0] (p-116-106) at (106,-116) {};
%\node[box=0] (p-116-107) at (107,-116) {};
%\node[box=0] (p-116-108) at (108,-116) {};
%\node[box=0] (p-116-109) at (109,-116) {};
%\node[box=0] (p-116-110) at (110,-116) {};
%\node[box=0] (p-116-111) at (111,-116) {};
%\node[box=1] (p-116-112) at (112,-116) {};
%\node[box=0] (p-116-113) at (113,-116) {};
%\node[box=0] (p-116-114) at (114,-116) {};
%\node[box=0] (p-116-115) at (115,-116) {};
%\node[box=1] (p-116-116) at (116,-116) {};
%\node[box=1] (p-117-0) at (0,-117) {};
%\node[box=1] (p-117-1) at (1,-117) {};
%\node[box=0] (p-117-2) at (2,-117) {};
%\node[box=0] (p-117-3) at (3,-117) {};
%\node[box=1] (p-117-4) at (4,-117) {};
%\node[box=1] (p-117-5) at (5,-117) {};
%\node[box=0] (p-117-6) at (6,-117) {};
%\node[box=0] (p-117-7) at (7,-117) {};
%\node[box=0] (p-117-8) at (8,-117) {};
%\node[box=0] (p-117-9) at (9,-117) {};
%\node[box=0] (p-117-10) at (10,-117) {};
%\node[box=0] (p-117-11) at (11,-117) {};
%\node[box=0] (p-117-12) at (12,-117) {};
%\node[box=0] (p-117-13) at (13,-117) {};
%\node[box=0] (p-117-14) at (14,-117) {};
%\node[box=0] (p-117-15) at (15,-117) {};
%\node[box=1] (p-117-16) at (16,-117) {};
%\node[box=1] (p-117-17) at (17,-117) {};
%\node[box=0] (p-117-18) at (18,-117) {};
%\node[box=0] (p-117-19) at (19,-117) {};
%\node[box=1] (p-117-20) at (20,-117) {};
%\node[box=1] (p-117-21) at (21,-117) {};
%\node[box=0] (p-117-22) at (22,-117) {};
%\node[box=0] (p-117-23) at (23,-117) {};
%\node[box=0] (p-117-24) at (24,-117) {};
%\node[box=0] (p-117-25) at (25,-117) {};
%\node[box=0] (p-117-26) at (26,-117) {};
%\node[box=0] (p-117-27) at (27,-117) {};
%\node[box=0] (p-117-28) at (28,-117) {};
%\node[box=0] (p-117-29) at (29,-117) {};
%\node[box=0] (p-117-30) at (30,-117) {};
%\node[box=0] (p-117-31) at (31,-117) {};
%\node[box=1] (p-117-32) at (32,-117) {};
%\node[box=1] (p-117-33) at (33,-117) {};
%\node[box=0] (p-117-34) at (34,-117) {};
%\node[box=0] (p-117-35) at (35,-117) {};
%\node[box=1] (p-117-36) at (36,-117) {};
%\node[box=1] (p-117-37) at (37,-117) {};
%\node[box=0] (p-117-38) at (38,-117) {};
%\node[box=0] (p-117-39) at (39,-117) {};
%\node[box=0] (p-117-40) at (40,-117) {};
%\node[box=0] (p-117-41) at (41,-117) {};
%\node[box=0] (p-117-42) at (42,-117) {};
%\node[box=0] (p-117-43) at (43,-117) {};
%\node[box=0] (p-117-44) at (44,-117) {};
%\node[box=0] (p-117-45) at (45,-117) {};
%\node[box=0] (p-117-46) at (46,-117) {};
%\node[box=0] (p-117-47) at (47,-117) {};
%\node[box=1] (p-117-48) at (48,-117) {};
%\node[box=1] (p-117-49) at (49,-117) {};
%\node[box=0] (p-117-50) at (50,-117) {};
%\node[box=0] (p-117-51) at (51,-117) {};
%\node[box=1] (p-117-52) at (52,-117) {};
%\node[box=1] (p-117-53) at (53,-117) {};
%\node[box=0] (p-117-54) at (54,-117) {};
%\node[box=0] (p-117-55) at (55,-117) {};
%\node[box=0] (p-117-56) at (56,-117) {};
%\node[box=0] (p-117-57) at (57,-117) {};
%\node[box=0] (p-117-58) at (58,-117) {};
%\node[box=0] (p-117-59) at (59,-117) {};
%\node[box=0] (p-117-60) at (60,-117) {};
%\node[box=0] (p-117-61) at (61,-117) {};
%\node[box=0] (p-117-62) at (62,-117) {};
%\node[box=0] (p-117-63) at (63,-117) {};
%\node[box=1] (p-117-64) at (64,-117) {};
%\node[box=1] (p-117-65) at (65,-117) {};
%\node[box=0] (p-117-66) at (66,-117) {};
%\node[box=0] (p-117-67) at (67,-117) {};
%\node[box=1] (p-117-68) at (68,-117) {};
%\node[box=1] (p-117-69) at (69,-117) {};
%\node[box=0] (p-117-70) at (70,-117) {};
%\node[box=0] (p-117-71) at (71,-117) {};
%\node[box=0] (p-117-72) at (72,-117) {};
%\node[box=0] (p-117-73) at (73,-117) {};
%\node[box=0] (p-117-74) at (74,-117) {};
%\node[box=0] (p-117-75) at (75,-117) {};
%\node[box=0] (p-117-76) at (76,-117) {};
%\node[box=0] (p-117-77) at (77,-117) {};
%\node[box=0] (p-117-78) at (78,-117) {};
%\node[box=0] (p-117-79) at (79,-117) {};
%\node[box=1] (p-117-80) at (80,-117) {};
%\node[box=1] (p-117-81) at (81,-117) {};
%\node[box=0] (p-117-82) at (82,-117) {};
%\node[box=0] (p-117-83) at (83,-117) {};
%\node[box=1] (p-117-84) at (84,-117) {};
%\node[box=1] (p-117-85) at (85,-117) {};
%\node[box=0] (p-117-86) at (86,-117) {};
%\node[box=0] (p-117-87) at (87,-117) {};
%\node[box=0] (p-117-88) at (88,-117) {};
%\node[box=0] (p-117-89) at (89,-117) {};
%\node[box=0] (p-117-90) at (90,-117) {};
%\node[box=0] (p-117-91) at (91,-117) {};
%\node[box=0] (p-117-92) at (92,-117) {};
%\node[box=0] (p-117-93) at (93,-117) {};
%\node[box=0] (p-117-94) at (94,-117) {};
%\node[box=0] (p-117-95) at (95,-117) {};
%\node[box=1] (p-117-96) at (96,-117) {};
%\node[box=1] (p-117-97) at (97,-117) {};
%\node[box=0] (p-117-98) at (98,-117) {};
%\node[box=0] (p-117-99) at (99,-117) {};
%\node[box=1] (p-117-100) at (100,-117) {};
%\node[box=1] (p-117-101) at (101,-117) {};
%\node[box=0] (p-117-102) at (102,-117) {};
%\node[box=0] (p-117-103) at (103,-117) {};
%\node[box=0] (p-117-104) at (104,-117) {};
%\node[box=0] (p-117-105) at (105,-117) {};
%\node[box=0] (p-117-106) at (106,-117) {};
%\node[box=0] (p-117-107) at (107,-117) {};
%\node[box=0] (p-117-108) at (108,-117) {};
%\node[box=0] (p-117-109) at (109,-117) {};
%\node[box=0] (p-117-110) at (110,-117) {};
%\node[box=0] (p-117-111) at (111,-117) {};
%\node[box=1] (p-117-112) at (112,-117) {};
%\node[box=1] (p-117-113) at (113,-117) {};
%\node[box=0] (p-117-114) at (114,-117) {};
%\node[box=0] (p-117-115) at (115,-117) {};
%\node[box=1] (p-117-116) at (116,-117) {};
%\node[box=1] (p-117-117) at (117,-117) {};
%\node[box=1] (p-118-0) at (0,-118) {};
%\node[box=0] (p-118-1) at (1,-118) {};
%\node[box=1] (p-118-2) at (2,-118) {};
%\node[box=0] (p-118-3) at (3,-118) {};
%\node[box=1] (p-118-4) at (4,-118) {};
%\node[box=0] (p-118-5) at (5,-118) {};
%\node[box=1] (p-118-6) at (6,-118) {};
%\node[box=0] (p-118-7) at (7,-118) {};
%\node[box=0] (p-118-8) at (8,-118) {};
%\node[box=0] (p-118-9) at (9,-118) {};
%\node[box=0] (p-118-10) at (10,-118) {};
%\node[box=0] (p-118-11) at (11,-118) {};
%\node[box=0] (p-118-12) at (12,-118) {};
%\node[box=0] (p-118-13) at (13,-118) {};
%\node[box=0] (p-118-14) at (14,-118) {};
%\node[box=0] (p-118-15) at (15,-118) {};
%\node[box=1] (p-118-16) at (16,-118) {};
%\node[box=0] (p-118-17) at (17,-118) {};
%\node[box=1] (p-118-18) at (18,-118) {};
%\node[box=0] (p-118-19) at (19,-118) {};
%\node[box=1] (p-118-20) at (20,-118) {};
%\node[box=0] (p-118-21) at (21,-118) {};
%\node[box=1] (p-118-22) at (22,-118) {};
%\node[box=0] (p-118-23) at (23,-118) {};
%\node[box=0] (p-118-24) at (24,-118) {};
%\node[box=0] (p-118-25) at (25,-118) {};
%\node[box=0] (p-118-26) at (26,-118) {};
%\node[box=0] (p-118-27) at (27,-118) {};
%\node[box=0] (p-118-28) at (28,-118) {};
%\node[box=0] (p-118-29) at (29,-118) {};
%\node[box=0] (p-118-30) at (30,-118) {};
%\node[box=0] (p-118-31) at (31,-118) {};
%\node[box=1] (p-118-32) at (32,-118) {};
%\node[box=0] (p-118-33) at (33,-118) {};
%\node[box=1] (p-118-34) at (34,-118) {};
%\node[box=0] (p-118-35) at (35,-118) {};
%\node[box=1] (p-118-36) at (36,-118) {};
%\node[box=0] (p-118-37) at (37,-118) {};
%\node[box=1] (p-118-38) at (38,-118) {};
%\node[box=0] (p-118-39) at (39,-118) {};
%\node[box=0] (p-118-40) at (40,-118) {};
%\node[box=0] (p-118-41) at (41,-118) {};
%\node[box=0] (p-118-42) at (42,-118) {};
%\node[box=0] (p-118-43) at (43,-118) {};
%\node[box=0] (p-118-44) at (44,-118) {};
%\node[box=0] (p-118-45) at (45,-118) {};
%\node[box=0] (p-118-46) at (46,-118) {};
%\node[box=0] (p-118-47) at (47,-118) {};
%\node[box=1] (p-118-48) at (48,-118) {};
%\node[box=0] (p-118-49) at (49,-118) {};
%\node[box=1] (p-118-50) at (50,-118) {};
%\node[box=0] (p-118-51) at (51,-118) {};
%\node[box=1] (p-118-52) at (52,-118) {};
%\node[box=0] (p-118-53) at (53,-118) {};
%\node[box=1] (p-118-54) at (54,-118) {};
%\node[box=0] (p-118-55) at (55,-118) {};
%\node[box=0] (p-118-56) at (56,-118) {};
%\node[box=0] (p-118-57) at (57,-118) {};
%\node[box=0] (p-118-58) at (58,-118) {};
%\node[box=0] (p-118-59) at (59,-118) {};
%\node[box=0] (p-118-60) at (60,-118) {};
%\node[box=0] (p-118-61) at (61,-118) {};
%\node[box=0] (p-118-62) at (62,-118) {};
%\node[box=0] (p-118-63) at (63,-118) {};
%\node[box=1] (p-118-64) at (64,-118) {};
%\node[box=0] (p-118-65) at (65,-118) {};
%\node[box=1] (p-118-66) at (66,-118) {};
%\node[box=0] (p-118-67) at (67,-118) {};
%\node[box=1] (p-118-68) at (68,-118) {};
%\node[box=0] (p-118-69) at (69,-118) {};
%\node[box=1] (p-118-70) at (70,-118) {};
%\node[box=0] (p-118-71) at (71,-118) {};
%\node[box=0] (p-118-72) at (72,-118) {};
%\node[box=0] (p-118-73) at (73,-118) {};
%\node[box=0] (p-118-74) at (74,-118) {};
%\node[box=0] (p-118-75) at (75,-118) {};
%\node[box=0] (p-118-76) at (76,-118) {};
%\node[box=0] (p-118-77) at (77,-118) {};
%\node[box=0] (p-118-78) at (78,-118) {};
%\node[box=0] (p-118-79) at (79,-118) {};
%\node[box=1] (p-118-80) at (80,-118) {};
%\node[box=0] (p-118-81) at (81,-118) {};
%\node[box=1] (p-118-82) at (82,-118) {};
%\node[box=0] (p-118-83) at (83,-118) {};
%\node[box=1] (p-118-84) at (84,-118) {};
%\node[box=0] (p-118-85) at (85,-118) {};
%\node[box=1] (p-118-86) at (86,-118) {};
%\node[box=0] (p-118-87) at (87,-118) {};
%\node[box=0] (p-118-88) at (88,-118) {};
%\node[box=0] (p-118-89) at (89,-118) {};
%\node[box=0] (p-118-90) at (90,-118) {};
%\node[box=0] (p-118-91) at (91,-118) {};
%\node[box=0] (p-118-92) at (92,-118) {};
%\node[box=0] (p-118-93) at (93,-118) {};
%\node[box=0] (p-118-94) at (94,-118) {};
%\node[box=0] (p-118-95) at (95,-118) {};
%\node[box=1] (p-118-96) at (96,-118) {};
%\node[box=0] (p-118-97) at (97,-118) {};
%\node[box=1] (p-118-98) at (98,-118) {};
%\node[box=0] (p-118-99) at (99,-118) {};
%\node[box=1] (p-118-100) at (100,-118) {};
%\node[box=0] (p-118-101) at (101,-118) {};
%\node[box=1] (p-118-102) at (102,-118) {};
%\node[box=0] (p-118-103) at (103,-118) {};
%\node[box=0] (p-118-104) at (104,-118) {};
%\node[box=0] (p-118-105) at (105,-118) {};
%\node[box=0] (p-118-106) at (106,-118) {};
%\node[box=0] (p-118-107) at (107,-118) {};
%\node[box=0] (p-118-108) at (108,-118) {};
%\node[box=0] (p-118-109) at (109,-118) {};
%\node[box=0] (p-118-110) at (110,-118) {};
%\node[box=0] (p-118-111) at (111,-118) {};
%\node[box=1] (p-118-112) at (112,-118) {};
%\node[box=0] (p-118-113) at (113,-118) {};
%\node[box=1] (p-118-114) at (114,-118) {};
%\node[box=0] (p-118-115) at (115,-118) {};
%\node[box=1] (p-118-116) at (116,-118) {};
%\node[box=0] (p-118-117) at (117,-118) {};
%\node[box=1] (p-118-118) at (118,-118) {};
%\node[box=1] (p-119-0) at (0,-119) {};
%\node[box=1] (p-119-1) at (1,-119) {};
%\node[box=1] (p-119-2) at (2,-119) {};
%\node[box=1] (p-119-3) at (3,-119) {};
%\node[box=1] (p-119-4) at (4,-119) {};
%\node[box=1] (p-119-5) at (5,-119) {};
%\node[box=1] (p-119-6) at (6,-119) {};
%\node[box=1] (p-119-7) at (7,-119) {};
%\node[box=0] (p-119-8) at (8,-119) {};
%\node[box=0] (p-119-9) at (9,-119) {};
%\node[box=0] (p-119-10) at (10,-119) {};
%\node[box=0] (p-119-11) at (11,-119) {};
%\node[box=0] (p-119-12) at (12,-119) {};
%\node[box=0] (p-119-13) at (13,-119) {};
%\node[box=0] (p-119-14) at (14,-119) {};
%\node[box=0] (p-119-15) at (15,-119) {};
%\node[box=1] (p-119-16) at (16,-119) {};
%\node[box=1] (p-119-17) at (17,-119) {};
%\node[box=1] (p-119-18) at (18,-119) {};
%\node[box=1] (p-119-19) at (19,-119) {};
%\node[box=1] (p-119-20) at (20,-119) {};
%\node[box=1] (p-119-21) at (21,-119) {};
%\node[box=1] (p-119-22) at (22,-119) {};
%\node[box=1] (p-119-23) at (23,-119) {};
%\node[box=0] (p-119-24) at (24,-119) {};
%\node[box=0] (p-119-25) at (25,-119) {};
%\node[box=0] (p-119-26) at (26,-119) {};
%\node[box=0] (p-119-27) at (27,-119) {};
%\node[box=0] (p-119-28) at (28,-119) {};
%\node[box=0] (p-119-29) at (29,-119) {};
%\node[box=0] (p-119-30) at (30,-119) {};
%\node[box=0] (p-119-31) at (31,-119) {};
%\node[box=1] (p-119-32) at (32,-119) {};
%\node[box=1] (p-119-33) at (33,-119) {};
%\node[box=1] (p-119-34) at (34,-119) {};
%\node[box=1] (p-119-35) at (35,-119) {};
%\node[box=1] (p-119-36) at (36,-119) {};
%\node[box=1] (p-119-37) at (37,-119) {};
%\node[box=1] (p-119-38) at (38,-119) {};
%\node[box=1] (p-119-39) at (39,-119) {};
%\node[box=0] (p-119-40) at (40,-119) {};
%\node[box=0] (p-119-41) at (41,-119) {};
%\node[box=0] (p-119-42) at (42,-119) {};
%\node[box=0] (p-119-43) at (43,-119) {};
%\node[box=0] (p-119-44) at (44,-119) {};
%\node[box=0] (p-119-45) at (45,-119) {};
%\node[box=0] (p-119-46) at (46,-119) {};
%\node[box=0] (p-119-47) at (47,-119) {};
%\node[box=1] (p-119-48) at (48,-119) {};
%\node[box=1] (p-119-49) at (49,-119) {};
%\node[box=1] (p-119-50) at (50,-119) {};
%\node[box=1] (p-119-51) at (51,-119) {};
%\node[box=1] (p-119-52) at (52,-119) {};
%\node[box=1] (p-119-53) at (53,-119) {};
%\node[box=1] (p-119-54) at (54,-119) {};
%\node[box=1] (p-119-55) at (55,-119) {};
%\node[box=0] (p-119-56) at (56,-119) {};
%\node[box=0] (p-119-57) at (57,-119) {};
%\node[box=0] (p-119-58) at (58,-119) {};
%\node[box=0] (p-119-59) at (59,-119) {};
%\node[box=0] (p-119-60) at (60,-119) {};
%\node[box=0] (p-119-61) at (61,-119) {};
%\node[box=0] (p-119-62) at (62,-119) {};
%\node[box=0] (p-119-63) at (63,-119) {};
%\node[box=1] (p-119-64) at (64,-119) {};
%\node[box=1] (p-119-65) at (65,-119) {};
%\node[box=1] (p-119-66) at (66,-119) {};
%\node[box=1] (p-119-67) at (67,-119) {};
%\node[box=1] (p-119-68) at (68,-119) {};
%\node[box=1] (p-119-69) at (69,-119) {};
%\node[box=1] (p-119-70) at (70,-119) {};
%\node[box=1] (p-119-71) at (71,-119) {};
%\node[box=0] (p-119-72) at (72,-119) {};
%\node[box=0] (p-119-73) at (73,-119) {};
%\node[box=0] (p-119-74) at (74,-119) {};
%\node[box=0] (p-119-75) at (75,-119) {};
%\node[box=0] (p-119-76) at (76,-119) {};
%\node[box=0] (p-119-77) at (77,-119) {};
%\node[box=0] (p-119-78) at (78,-119) {};
%\node[box=0] (p-119-79) at (79,-119) {};
%\node[box=1] (p-119-80) at (80,-119) {};
%\node[box=1] (p-119-81) at (81,-119) {};
%\node[box=1] (p-119-82) at (82,-119) {};
%\node[box=1] (p-119-83) at (83,-119) {};
%\node[box=1] (p-119-84) at (84,-119) {};
%\node[box=1] (p-119-85) at (85,-119) {};
%\node[box=1] (p-119-86) at (86,-119) {};
%\node[box=1] (p-119-87) at (87,-119) {};
%\node[box=0] (p-119-88) at (88,-119) {};
%\node[box=0] (p-119-89) at (89,-119) {};
%\node[box=0] (p-119-90) at (90,-119) {};
%\node[box=0] (p-119-91) at (91,-119) {};
%\node[box=0] (p-119-92) at (92,-119) {};
%\node[box=0] (p-119-93) at (93,-119) {};
%\node[box=0] (p-119-94) at (94,-119) {};
%\node[box=0] (p-119-95) at (95,-119) {};
%\node[box=1] (p-119-96) at (96,-119) {};
%\node[box=1] (p-119-97) at (97,-119) {};
%\node[box=1] (p-119-98) at (98,-119) {};
%\node[box=1] (p-119-99) at (99,-119) {};
%\node[box=1] (p-119-100) at (100,-119) {};
%\node[box=1] (p-119-101) at (101,-119) {};
%\node[box=1] (p-119-102) at (102,-119) {};
%\node[box=1] (p-119-103) at (103,-119) {};
%\node[box=0] (p-119-104) at (104,-119) {};
%\node[box=0] (p-119-105) at (105,-119) {};
%\node[box=0] (p-119-106) at (106,-119) {};
%\node[box=0] (p-119-107) at (107,-119) {};
%\node[box=0] (p-119-108) at (108,-119) {};
%\node[box=0] (p-119-109) at (109,-119) {};
%\node[box=0] (p-119-110) at (110,-119) {};
%\node[box=0] (p-119-111) at (111,-119) {};
%\node[box=1] (p-119-112) at (112,-119) {};
%\node[box=1] (p-119-113) at (113,-119) {};
%\node[box=1] (p-119-114) at (114,-119) {};
%\node[box=1] (p-119-115) at (115,-119) {};
%\node[box=1] (p-119-116) at (116,-119) {};
%\node[box=1] (p-119-117) at (117,-119) {};
%\node[box=1] (p-119-118) at (118,-119) {};
%\node[box=1] (p-119-119) at (119,-119) {};
%\node[box=1] (p-120-0) at (0,-120) {};
%\node[box=0] (p-120-1) at (1,-120) {};
%\node[box=0] (p-120-2) at (2,-120) {};
%\node[box=0] (p-120-3) at (3,-120) {};
%\node[box=0] (p-120-4) at (4,-120) {};
%\node[box=0] (p-120-5) at (5,-120) {};
%\node[box=0] (p-120-6) at (6,-120) {};
%\node[box=0] (p-120-7) at (7,-120) {};
%\node[box=1] (p-120-8) at (8,-120) {};
%\node[box=0] (p-120-9) at (9,-120) {};
%\node[box=0] (p-120-10) at (10,-120) {};
%\node[box=0] (p-120-11) at (11,-120) {};
%\node[box=0] (p-120-12) at (12,-120) {};
%\node[box=0] (p-120-13) at (13,-120) {};
%\node[box=0] (p-120-14) at (14,-120) {};
%\node[box=0] (p-120-15) at (15,-120) {};
%\node[box=1] (p-120-16) at (16,-120) {};
%\node[box=0] (p-120-17) at (17,-120) {};
%\node[box=0] (p-120-18) at (18,-120) {};
%\node[box=0] (p-120-19) at (19,-120) {};
%\node[box=0] (p-120-20) at (20,-120) {};
%\node[box=0] (p-120-21) at (21,-120) {};
%\node[box=0] (p-120-22) at (22,-120) {};
%\node[box=0] (p-120-23) at (23,-120) {};
%\node[box=1] (p-120-24) at (24,-120) {};
%\node[box=0] (p-120-25) at (25,-120) {};
%\node[box=0] (p-120-26) at (26,-120) {};
%\node[box=0] (p-120-27) at (27,-120) {};
%\node[box=0] (p-120-28) at (28,-120) {};
%\node[box=0] (p-120-29) at (29,-120) {};
%\node[box=0] (p-120-30) at (30,-120) {};
%\node[box=0] (p-120-31) at (31,-120) {};
%\node[box=1] (p-120-32) at (32,-120) {};
%\node[box=0] (p-120-33) at (33,-120) {};
%\node[box=0] (p-120-34) at (34,-120) {};
%\node[box=0] (p-120-35) at (35,-120) {};
%\node[box=0] (p-120-36) at (36,-120) {};
%\node[box=0] (p-120-37) at (37,-120) {};
%\node[box=0] (p-120-38) at (38,-120) {};
%\node[box=0] (p-120-39) at (39,-120) {};
%\node[box=1] (p-120-40) at (40,-120) {};
%\node[box=0] (p-120-41) at (41,-120) {};
%\node[box=0] (p-120-42) at (42,-120) {};
%\node[box=0] (p-120-43) at (43,-120) {};
%\node[box=0] (p-120-44) at (44,-120) {};
%\node[box=0] (p-120-45) at (45,-120) {};
%\node[box=0] (p-120-46) at (46,-120) {};
%\node[box=0] (p-120-47) at (47,-120) {};
%\node[box=1] (p-120-48) at (48,-120) {};
%\node[box=0] (p-120-49) at (49,-120) {};
%\node[box=0] (p-120-50) at (50,-120) {};
%\node[box=0] (p-120-51) at (51,-120) {};
%\node[box=0] (p-120-52) at (52,-120) {};
%\node[box=0] (p-120-53) at (53,-120) {};
%\node[box=0] (p-120-54) at (54,-120) {};
%\node[box=0] (p-120-55) at (55,-120) {};
%\node[box=1] (p-120-56) at (56,-120) {};
%\node[box=0] (p-120-57) at (57,-120) {};
%\node[box=0] (p-120-58) at (58,-120) {};
%\node[box=0] (p-120-59) at (59,-120) {};
%\node[box=0] (p-120-60) at (60,-120) {};
%\node[box=0] (p-120-61) at (61,-120) {};
%\node[box=0] (p-120-62) at (62,-120) {};
%\node[box=0] (p-120-63) at (63,-120) {};
%\node[box=1] (p-120-64) at (64,-120) {};
%\node[box=0] (p-120-65) at (65,-120) {};
%\node[box=0] (p-120-66) at (66,-120) {};
%\node[box=0] (p-120-67) at (67,-120) {};
%\node[box=0] (p-120-68) at (68,-120) {};
%\node[box=0] (p-120-69) at (69,-120) {};
%\node[box=0] (p-120-70) at (70,-120) {};
%\node[box=0] (p-120-71) at (71,-120) {};
%\node[box=1] (p-120-72) at (72,-120) {};
%\node[box=0] (p-120-73) at (73,-120) {};
%\node[box=0] (p-120-74) at (74,-120) {};
%\node[box=0] (p-120-75) at (75,-120) {};
%\node[box=0] (p-120-76) at (76,-120) {};
%\node[box=0] (p-120-77) at (77,-120) {};
%\node[box=0] (p-120-78) at (78,-120) {};
%\node[box=0] (p-120-79) at (79,-120) {};
%\node[box=1] (p-120-80) at (80,-120) {};
%\node[box=0] (p-120-81) at (81,-120) {};
%\node[box=0] (p-120-82) at (82,-120) {};
%\node[box=0] (p-120-83) at (83,-120) {};
%\node[box=0] (p-120-84) at (84,-120) {};
%\node[box=0] (p-120-85) at (85,-120) {};
%\node[box=0] (p-120-86) at (86,-120) {};
%\node[box=0] (p-120-87) at (87,-120) {};
%\node[box=1] (p-120-88) at (88,-120) {};
%\node[box=0] (p-120-89) at (89,-120) {};
%\node[box=0] (p-120-90) at (90,-120) {};
%\node[box=0] (p-120-91) at (91,-120) {};
%\node[box=0] (p-120-92) at (92,-120) {};
%\node[box=0] (p-120-93) at (93,-120) {};
%\node[box=0] (p-120-94) at (94,-120) {};
%\node[box=0] (p-120-95) at (95,-120) {};
%\node[box=1] (p-120-96) at (96,-120) {};
%\node[box=0] (p-120-97) at (97,-120) {};
%\node[box=0] (p-120-98) at (98,-120) {};
%\node[box=0] (p-120-99) at (99,-120) {};
%\node[box=0] (p-120-100) at (100,-120) {};
%\node[box=0] (p-120-101) at (101,-120) {};
%\node[box=0] (p-120-102) at (102,-120) {};
%\node[box=0] (p-120-103) at (103,-120) {};
%\node[box=1] (p-120-104) at (104,-120) {};
%\node[box=0] (p-120-105) at (105,-120) {};
%\node[box=0] (p-120-106) at (106,-120) {};
%\node[box=0] (p-120-107) at (107,-120) {};
%\node[box=0] (p-120-108) at (108,-120) {};
%\node[box=0] (p-120-109) at (109,-120) {};
%\node[box=0] (p-120-110) at (110,-120) {};
%\node[box=0] (p-120-111) at (111,-120) {};
%\node[box=1] (p-120-112) at (112,-120) {};
%\node[box=0] (p-120-113) at (113,-120) {};
%\node[box=0] (p-120-114) at (114,-120) {};
%\node[box=0] (p-120-115) at (115,-120) {};
%\node[box=0] (p-120-116) at (116,-120) {};
%\node[box=0] (p-120-117) at (117,-120) {};
%\node[box=0] (p-120-118) at (118,-120) {};
%\node[box=0] (p-120-119) at (119,-120) {};
%\node[box=1] (p-120-120) at (120,-120) {};
%\node[box=1] (p-121-0) at (0,-121) {};
%\node[box=1] (p-121-1) at (1,-121) {};
%\node[box=0] (p-121-2) at (2,-121) {};
%\node[box=0] (p-121-3) at (3,-121) {};
%\node[box=0] (p-121-4) at (4,-121) {};
%\node[box=0] (p-121-5) at (5,-121) {};
%\node[box=0] (p-121-6) at (6,-121) {};
%\node[box=0] (p-121-7) at (7,-121) {};
%\node[box=1] (p-121-8) at (8,-121) {};
%\node[box=1] (p-121-9) at (9,-121) {};
%\node[box=0] (p-121-10) at (10,-121) {};
%\node[box=0] (p-121-11) at (11,-121) {};
%\node[box=0] (p-121-12) at (12,-121) {};
%\node[box=0] (p-121-13) at (13,-121) {};
%\node[box=0] (p-121-14) at (14,-121) {};
%\node[box=0] (p-121-15) at (15,-121) {};
%\node[box=1] (p-121-16) at (16,-121) {};
%\node[box=1] (p-121-17) at (17,-121) {};
%\node[box=0] (p-121-18) at (18,-121) {};
%\node[box=0] (p-121-19) at (19,-121) {};
%\node[box=0] (p-121-20) at (20,-121) {};
%\node[box=0] (p-121-21) at (21,-121) {};
%\node[box=0] (p-121-22) at (22,-121) {};
%\node[box=0] (p-121-23) at (23,-121) {};
%\node[box=1] (p-121-24) at (24,-121) {};
%\node[box=1] (p-121-25) at (25,-121) {};
%\node[box=0] (p-121-26) at (26,-121) {};
%\node[box=0] (p-121-27) at (27,-121) {};
%\node[box=0] (p-121-28) at (28,-121) {};
%\node[box=0] (p-121-29) at (29,-121) {};
%\node[box=0] (p-121-30) at (30,-121) {};
%\node[box=0] (p-121-31) at (31,-121) {};
%\node[box=1] (p-121-32) at (32,-121) {};
%\node[box=1] (p-121-33) at (33,-121) {};
%\node[box=0] (p-121-34) at (34,-121) {};
%\node[box=0] (p-121-35) at (35,-121) {};
%\node[box=0] (p-121-36) at (36,-121) {};
%\node[box=0] (p-121-37) at (37,-121) {};
%\node[box=0] (p-121-38) at (38,-121) {};
%\node[box=0] (p-121-39) at (39,-121) {};
%\node[box=1] (p-121-40) at (40,-121) {};
%\node[box=1] (p-121-41) at (41,-121) {};
%\node[box=0] (p-121-42) at (42,-121) {};
%\node[box=0] (p-121-43) at (43,-121) {};
%\node[box=0] (p-121-44) at (44,-121) {};
%\node[box=0] (p-121-45) at (45,-121) {};
%\node[box=0] (p-121-46) at (46,-121) {};
%\node[box=0] (p-121-47) at (47,-121) {};
%\node[box=1] (p-121-48) at (48,-121) {};
%\node[box=1] (p-121-49) at (49,-121) {};
%\node[box=0] (p-121-50) at (50,-121) {};
%\node[box=0] (p-121-51) at (51,-121) {};
%\node[box=0] (p-121-52) at (52,-121) {};
%\node[box=0] (p-121-53) at (53,-121) {};
%\node[box=0] (p-121-54) at (54,-121) {};
%\node[box=0] (p-121-55) at (55,-121) {};
%\node[box=1] (p-121-56) at (56,-121) {};
%\node[box=1] (p-121-57) at (57,-121) {};
%\node[box=0] (p-121-58) at (58,-121) {};
%\node[box=0] (p-121-59) at (59,-121) {};
%\node[box=0] (p-121-60) at (60,-121) {};
%\node[box=0] (p-121-61) at (61,-121) {};
%\node[box=0] (p-121-62) at (62,-121) {};
%\node[box=0] (p-121-63) at (63,-121) {};
%\node[box=1] (p-121-64) at (64,-121) {};
%\node[box=1] (p-121-65) at (65,-121) {};
%\node[box=0] (p-121-66) at (66,-121) {};
%\node[box=0] (p-121-67) at (67,-121) {};
%\node[box=0] (p-121-68) at (68,-121) {};
%\node[box=0] (p-121-69) at (69,-121) {};
%\node[box=0] (p-121-70) at (70,-121) {};
%\node[box=0] (p-121-71) at (71,-121) {};
%\node[box=1] (p-121-72) at (72,-121) {};
%\node[box=1] (p-121-73) at (73,-121) {};
%\node[box=0] (p-121-74) at (74,-121) {};
%\node[box=0] (p-121-75) at (75,-121) {};
%\node[box=0] (p-121-76) at (76,-121) {};
%\node[box=0] (p-121-77) at (77,-121) {};
%\node[box=0] (p-121-78) at (78,-121) {};
%\node[box=0] (p-121-79) at (79,-121) {};
%\node[box=1] (p-121-80) at (80,-121) {};
%\node[box=1] (p-121-81) at (81,-121) {};
%\node[box=0] (p-121-82) at (82,-121) {};
%\node[box=0] (p-121-83) at (83,-121) {};
%\node[box=0] (p-121-84) at (84,-121) {};
%\node[box=0] (p-121-85) at (85,-121) {};
%\node[box=0] (p-121-86) at (86,-121) {};
%\node[box=0] (p-121-87) at (87,-121) {};
%\node[box=1] (p-121-88) at (88,-121) {};
%\node[box=1] (p-121-89) at (89,-121) {};
%\node[box=0] (p-121-90) at (90,-121) {};
%\node[box=0] (p-121-91) at (91,-121) {};
%\node[box=0] (p-121-92) at (92,-121) {};
%\node[box=0] (p-121-93) at (93,-121) {};
%\node[box=0] (p-121-94) at (94,-121) {};
%\node[box=0] (p-121-95) at (95,-121) {};
%\node[box=1] (p-121-96) at (96,-121) {};
%\node[box=1] (p-121-97) at (97,-121) {};
%\node[box=0] (p-121-98) at (98,-121) {};
%\node[box=0] (p-121-99) at (99,-121) {};
%\node[box=0] (p-121-100) at (100,-121) {};
%\node[box=0] (p-121-101) at (101,-121) {};
%\node[box=0] (p-121-102) at (102,-121) {};
%\node[box=0] (p-121-103) at (103,-121) {};
%\node[box=1] (p-121-104) at (104,-121) {};
%\node[box=1] (p-121-105) at (105,-121) {};
%\node[box=0] (p-121-106) at (106,-121) {};
%\node[box=0] (p-121-107) at (107,-121) {};
%\node[box=0] (p-121-108) at (108,-121) {};
%\node[box=0] (p-121-109) at (109,-121) {};
%\node[box=0] (p-121-110) at (110,-121) {};
%\node[box=0] (p-121-111) at (111,-121) {};
%\node[box=1] (p-121-112) at (112,-121) {};
%\node[box=1] (p-121-113) at (113,-121) {};
%\node[box=0] (p-121-114) at (114,-121) {};
%\node[box=0] (p-121-115) at (115,-121) {};
%\node[box=0] (p-121-116) at (116,-121) {};
%\node[box=0] (p-121-117) at (117,-121) {};
%\node[box=0] (p-121-118) at (118,-121) {};
%\node[box=0] (p-121-119) at (119,-121) {};
%\node[box=1] (p-121-120) at (120,-121) {};
%\node[box=1] (p-121-121) at (121,-121) {};
%\node[box=1] (p-122-0) at (0,-122) {};
%\node[box=0] (p-122-1) at (1,-122) {};
%\node[box=1] (p-122-2) at (2,-122) {};
%\node[box=0] (p-122-3) at (3,-122) {};
%\node[box=0] (p-122-4) at (4,-122) {};
%\node[box=0] (p-122-5) at (5,-122) {};
%\node[box=0] (p-122-6) at (6,-122) {};
%\node[box=0] (p-122-7) at (7,-122) {};
%\node[box=1] (p-122-8) at (8,-122) {};
%\node[box=0] (p-122-9) at (9,-122) {};
%\node[box=1] (p-122-10) at (10,-122) {};
%\node[box=0] (p-122-11) at (11,-122) {};
%\node[box=0] (p-122-12) at (12,-122) {};
%\node[box=0] (p-122-13) at (13,-122) {};
%\node[box=0] (p-122-14) at (14,-122) {};
%\node[box=0] (p-122-15) at (15,-122) {};
%\node[box=1] (p-122-16) at (16,-122) {};
%\node[box=0] (p-122-17) at (17,-122) {};
%\node[box=1] (p-122-18) at (18,-122) {};
%\node[box=0] (p-122-19) at (19,-122) {};
%\node[box=0] (p-122-20) at (20,-122) {};
%\node[box=0] (p-122-21) at (21,-122) {};
%\node[box=0] (p-122-22) at (22,-122) {};
%\node[box=0] (p-122-23) at (23,-122) {};
%\node[box=1] (p-122-24) at (24,-122) {};
%\node[box=0] (p-122-25) at (25,-122) {};
%\node[box=1] (p-122-26) at (26,-122) {};
%\node[box=0] (p-122-27) at (27,-122) {};
%\node[box=0] (p-122-28) at (28,-122) {};
%\node[box=0] (p-122-29) at (29,-122) {};
%\node[box=0] (p-122-30) at (30,-122) {};
%\node[box=0] (p-122-31) at (31,-122) {};
%\node[box=1] (p-122-32) at (32,-122) {};
%\node[box=0] (p-122-33) at (33,-122) {};
%\node[box=1] (p-122-34) at (34,-122) {};
%\node[box=0] (p-122-35) at (35,-122) {};
%\node[box=0] (p-122-36) at (36,-122) {};
%\node[box=0] (p-122-37) at (37,-122) {};
%\node[box=0] (p-122-38) at (38,-122) {};
%\node[box=0] (p-122-39) at (39,-122) {};
%\node[box=1] (p-122-40) at (40,-122) {};
%\node[box=0] (p-122-41) at (41,-122) {};
%\node[box=1] (p-122-42) at (42,-122) {};
%\node[box=0] (p-122-43) at (43,-122) {};
%\node[box=0] (p-122-44) at (44,-122) {};
%\node[box=0] (p-122-45) at (45,-122) {};
%\node[box=0] (p-122-46) at (46,-122) {};
%\node[box=0] (p-122-47) at (47,-122) {};
%\node[box=1] (p-122-48) at (48,-122) {};
%\node[box=0] (p-122-49) at (49,-122) {};
%\node[box=1] (p-122-50) at (50,-122) {};
%\node[box=0] (p-122-51) at (51,-122) {};
%\node[box=0] (p-122-52) at (52,-122) {};
%\node[box=0] (p-122-53) at (53,-122) {};
%\node[box=0] (p-122-54) at (54,-122) {};
%\node[box=0] (p-122-55) at (55,-122) {};
%\node[box=1] (p-122-56) at (56,-122) {};
%\node[box=0] (p-122-57) at (57,-122) {};
%\node[box=1] (p-122-58) at (58,-122) {};
%\node[box=0] (p-122-59) at (59,-122) {};
%\node[box=0] (p-122-60) at (60,-122) {};
%\node[box=0] (p-122-61) at (61,-122) {};
%\node[box=0] (p-122-62) at (62,-122) {};
%\node[box=0] (p-122-63) at (63,-122) {};
%\node[box=1] (p-122-64) at (64,-122) {};
%\node[box=0] (p-122-65) at (65,-122) {};
%\node[box=1] (p-122-66) at (66,-122) {};
%\node[box=0] (p-122-67) at (67,-122) {};
%\node[box=0] (p-122-68) at (68,-122) {};
%\node[box=0] (p-122-69) at (69,-122) {};
%\node[box=0] (p-122-70) at (70,-122) {};
%\node[box=0] (p-122-71) at (71,-122) {};
%\node[box=1] (p-122-72) at (72,-122) {};
%\node[box=0] (p-122-73) at (73,-122) {};
%\node[box=1] (p-122-74) at (74,-122) {};
%\node[box=0] (p-122-75) at (75,-122) {};
%\node[box=0] (p-122-76) at (76,-122) {};
%\node[box=0] (p-122-77) at (77,-122) {};
%\node[box=0] (p-122-78) at (78,-122) {};
%\node[box=0] (p-122-79) at (79,-122) {};
%\node[box=1] (p-122-80) at (80,-122) {};
%\node[box=0] (p-122-81) at (81,-122) {};
%\node[box=1] (p-122-82) at (82,-122) {};
%\node[box=0] (p-122-83) at (83,-122) {};
%\node[box=0] (p-122-84) at (84,-122) {};
%\node[box=0] (p-122-85) at (85,-122) {};
%\node[box=0] (p-122-86) at (86,-122) {};
%\node[box=0] (p-122-87) at (87,-122) {};
%\node[box=1] (p-122-88) at (88,-122) {};
%\node[box=0] (p-122-89) at (89,-122) {};
%\node[box=1] (p-122-90) at (90,-122) {};
%\node[box=0] (p-122-91) at (91,-122) {};
%\node[box=0] (p-122-92) at (92,-122) {};
%\node[box=0] (p-122-93) at (93,-122) {};
%\node[box=0] (p-122-94) at (94,-122) {};
%\node[box=0] (p-122-95) at (95,-122) {};
%\node[box=1] (p-122-96) at (96,-122) {};
%\node[box=0] (p-122-97) at (97,-122) {};
%\node[box=1] (p-122-98) at (98,-122) {};
%\node[box=0] (p-122-99) at (99,-122) {};
%\node[box=0] (p-122-100) at (100,-122) {};
%\node[box=0] (p-122-101) at (101,-122) {};
%\node[box=0] (p-122-102) at (102,-122) {};
%\node[box=0] (p-122-103) at (103,-122) {};
%\node[box=1] (p-122-104) at (104,-122) {};
%\node[box=0] (p-122-105) at (105,-122) {};
%\node[box=1] (p-122-106) at (106,-122) {};
%\node[box=0] (p-122-107) at (107,-122) {};
%\node[box=0] (p-122-108) at (108,-122) {};
%\node[box=0] (p-122-109) at (109,-122) {};
%\node[box=0] (p-122-110) at (110,-122) {};
%\node[box=0] (p-122-111) at (111,-122) {};
%\node[box=1] (p-122-112) at (112,-122) {};
%\node[box=0] (p-122-113) at (113,-122) {};
%\node[box=1] (p-122-114) at (114,-122) {};
%\node[box=0] (p-122-115) at (115,-122) {};
%\node[box=0] (p-122-116) at (116,-122) {};
%\node[box=0] (p-122-117) at (117,-122) {};
%\node[box=0] (p-122-118) at (118,-122) {};
%\node[box=0] (p-122-119) at (119,-122) {};
%\node[box=1] (p-122-120) at (120,-122) {};
%\node[box=0] (p-122-121) at (121,-122) {};
%\node[box=1] (p-122-122) at (122,-122) {};
%\node[box=1] (p-123-0) at (0,-123) {};
%\node[box=1] (p-123-1) at (1,-123) {};
%\node[box=1] (p-123-2) at (2,-123) {};
%\node[box=1] (p-123-3) at (3,-123) {};
%\node[box=0] (p-123-4) at (4,-123) {};
%\node[box=0] (p-123-5) at (5,-123) {};
%\node[box=0] (p-123-6) at (6,-123) {};
%\node[box=0] (p-123-7) at (7,-123) {};
%\node[box=1] (p-123-8) at (8,-123) {};
%\node[box=1] (p-123-9) at (9,-123) {};
%\node[box=1] (p-123-10) at (10,-123) {};
%\node[box=1] (p-123-11) at (11,-123) {};
%\node[box=0] (p-123-12) at (12,-123) {};
%\node[box=0] (p-123-13) at (13,-123) {};
%\node[box=0] (p-123-14) at (14,-123) {};
%\node[box=0] (p-123-15) at (15,-123) {};
%\node[box=1] (p-123-16) at (16,-123) {};
%\node[box=1] (p-123-17) at (17,-123) {};
%\node[box=1] (p-123-18) at (18,-123) {};
%\node[box=1] (p-123-19) at (19,-123) {};
%\node[box=0] (p-123-20) at (20,-123) {};
%\node[box=0] (p-123-21) at (21,-123) {};
%\node[box=0] (p-123-22) at (22,-123) {};
%\node[box=0] (p-123-23) at (23,-123) {};
%\node[box=1] (p-123-24) at (24,-123) {};
%\node[box=1] (p-123-25) at (25,-123) {};
%\node[box=1] (p-123-26) at (26,-123) {};
%\node[box=1] (p-123-27) at (27,-123) {};
%\node[box=0] (p-123-28) at (28,-123) {};
%\node[box=0] (p-123-29) at (29,-123) {};
%\node[box=0] (p-123-30) at (30,-123) {};
%\node[box=0] (p-123-31) at (31,-123) {};
%\node[box=1] (p-123-32) at (32,-123) {};
%\node[box=1] (p-123-33) at (33,-123) {};
%\node[box=1] (p-123-34) at (34,-123) {};
%\node[box=1] (p-123-35) at (35,-123) {};
%\node[box=0] (p-123-36) at (36,-123) {};
%\node[box=0] (p-123-37) at (37,-123) {};
%\node[box=0] (p-123-38) at (38,-123) {};
%\node[box=0] (p-123-39) at (39,-123) {};
%\node[box=1] (p-123-40) at (40,-123) {};
%\node[box=1] (p-123-41) at (41,-123) {};
%\node[box=1] (p-123-42) at (42,-123) {};
%\node[box=1] (p-123-43) at (43,-123) {};
%\node[box=0] (p-123-44) at (44,-123) {};
%\node[box=0] (p-123-45) at (45,-123) {};
%\node[box=0] (p-123-46) at (46,-123) {};
%\node[box=0] (p-123-47) at (47,-123) {};
%\node[box=1] (p-123-48) at (48,-123) {};
%\node[box=1] (p-123-49) at (49,-123) {};
%\node[box=1] (p-123-50) at (50,-123) {};
%\node[box=1] (p-123-51) at (51,-123) {};
%\node[box=0] (p-123-52) at (52,-123) {};
%\node[box=0] (p-123-53) at (53,-123) {};
%\node[box=0] (p-123-54) at (54,-123) {};
%\node[box=0] (p-123-55) at (55,-123) {};
%\node[box=1] (p-123-56) at (56,-123) {};
%\node[box=1] (p-123-57) at (57,-123) {};
%\node[box=1] (p-123-58) at (58,-123) {};
%\node[box=1] (p-123-59) at (59,-123) {};
%\node[box=0] (p-123-60) at (60,-123) {};
%\node[box=0] (p-123-61) at (61,-123) {};
%\node[box=0] (p-123-62) at (62,-123) {};
%\node[box=0] (p-123-63) at (63,-123) {};
%\node[box=1] (p-123-64) at (64,-123) {};
%\node[box=1] (p-123-65) at (65,-123) {};
%\node[box=1] (p-123-66) at (66,-123) {};
%\node[box=1] (p-123-67) at (67,-123) {};
%\node[box=0] (p-123-68) at (68,-123) {};
%\node[box=0] (p-123-69) at (69,-123) {};
%\node[box=0] (p-123-70) at (70,-123) {};
%\node[box=0] (p-123-71) at (71,-123) {};
%\node[box=1] (p-123-72) at (72,-123) {};
%\node[box=1] (p-123-73) at (73,-123) {};
%\node[box=1] (p-123-74) at (74,-123) {};
%\node[box=1] (p-123-75) at (75,-123) {};
%\node[box=0] (p-123-76) at (76,-123) {};
%\node[box=0] (p-123-77) at (77,-123) {};
%\node[box=0] (p-123-78) at (78,-123) {};
%\node[box=0] (p-123-79) at (79,-123) {};
%\node[box=1] (p-123-80) at (80,-123) {};
%\node[box=1] (p-123-81) at (81,-123) {};
%\node[box=1] (p-123-82) at (82,-123) {};
%\node[box=1] (p-123-83) at (83,-123) {};
%\node[box=0] (p-123-84) at (84,-123) {};
%\node[box=0] (p-123-85) at (85,-123) {};
%\node[box=0] (p-123-86) at (86,-123) {};
%\node[box=0] (p-123-87) at (87,-123) {};
%\node[box=1] (p-123-88) at (88,-123) {};
%\node[box=1] (p-123-89) at (89,-123) {};
%\node[box=1] (p-123-90) at (90,-123) {};
%\node[box=1] (p-123-91) at (91,-123) {};
%\node[box=0] (p-123-92) at (92,-123) {};
%\node[box=0] (p-123-93) at (93,-123) {};
%\node[box=0] (p-123-94) at (94,-123) {};
%\node[box=0] (p-123-95) at (95,-123) {};
%\node[box=1] (p-123-96) at (96,-123) {};
%\node[box=1] (p-123-97) at (97,-123) {};
%\node[box=1] (p-123-98) at (98,-123) {};
%\node[box=1] (p-123-99) at (99,-123) {};
%\node[box=0] (p-123-100) at (100,-123) {};
%\node[box=0] (p-123-101) at (101,-123) {};
%\node[box=0] (p-123-102) at (102,-123) {};
%\node[box=0] (p-123-103) at (103,-123) {};
%\node[box=1] (p-123-104) at (104,-123) {};
%\node[box=1] (p-123-105) at (105,-123) {};
%\node[box=1] (p-123-106) at (106,-123) {};
%\node[box=1] (p-123-107) at (107,-123) {};
%\node[box=0] (p-123-108) at (108,-123) {};
%\node[box=0] (p-123-109) at (109,-123) {};
%\node[box=0] (p-123-110) at (110,-123) {};
%\node[box=0] (p-123-111) at (111,-123) {};
%\node[box=1] (p-123-112) at (112,-123) {};
%\node[box=1] (p-123-113) at (113,-123) {};
%\node[box=1] (p-123-114) at (114,-123) {};
%\node[box=1] (p-123-115) at (115,-123) {};
%\node[box=0] (p-123-116) at (116,-123) {};
%\node[box=0] (p-123-117) at (117,-123) {};
%\node[box=0] (p-123-118) at (118,-123) {};
%\node[box=0] (p-123-119) at (119,-123) {};
%\node[box=1] (p-123-120) at (120,-123) {};
%\node[box=1] (p-123-121) at (121,-123) {};
%\node[box=1] (p-123-122) at (122,-123) {};
%\node[box=1] (p-123-123) at (123,-123) {};
%\node[box=1] (p-124-0) at (0,-124) {};
%\node[box=0] (p-124-1) at (1,-124) {};
%\node[box=0] (p-124-2) at (2,-124) {};
%\node[box=0] (p-124-3) at (3,-124) {};
%\node[box=1] (p-124-4) at (4,-124) {};
%\node[box=0] (p-124-5) at (5,-124) {};
%\node[box=0] (p-124-6) at (6,-124) {};
%\node[box=0] (p-124-7) at (7,-124) {};
%\node[box=1] (p-124-8) at (8,-124) {};
%\node[box=0] (p-124-9) at (9,-124) {};
%\node[box=0] (p-124-10) at (10,-124) {};
%\node[box=0] (p-124-11) at (11,-124) {};
%\node[box=1] (p-124-12) at (12,-124) {};
%\node[box=0] (p-124-13) at (13,-124) {};
%\node[box=0] (p-124-14) at (14,-124) {};
%\node[box=0] (p-124-15) at (15,-124) {};
%\node[box=1] (p-124-16) at (16,-124) {};
%\node[box=0] (p-124-17) at (17,-124) {};
%\node[box=0] (p-124-18) at (18,-124) {};
%\node[box=0] (p-124-19) at (19,-124) {};
%\node[box=1] (p-124-20) at (20,-124) {};
%\node[box=0] (p-124-21) at (21,-124) {};
%\node[box=0] (p-124-22) at (22,-124) {};
%\node[box=0] (p-124-23) at (23,-124) {};
%\node[box=1] (p-124-24) at (24,-124) {};
%\node[box=0] (p-124-25) at (25,-124) {};
%\node[box=0] (p-124-26) at (26,-124) {};
%\node[box=0] (p-124-27) at (27,-124) {};
%\node[box=1] (p-124-28) at (28,-124) {};
%\node[box=0] (p-124-29) at (29,-124) {};
%\node[box=0] (p-124-30) at (30,-124) {};
%\node[box=0] (p-124-31) at (31,-124) {};
%\node[box=1] (p-124-32) at (32,-124) {};
%\node[box=0] (p-124-33) at (33,-124) {};
%\node[box=0] (p-124-34) at (34,-124) {};
%\node[box=0] (p-124-35) at (35,-124) {};
%\node[box=1] (p-124-36) at (36,-124) {};
%\node[box=0] (p-124-37) at (37,-124) {};
%\node[box=0] (p-124-38) at (38,-124) {};
%\node[box=0] (p-124-39) at (39,-124) {};
%\node[box=1] (p-124-40) at (40,-124) {};
%\node[box=0] (p-124-41) at (41,-124) {};
%\node[box=0] (p-124-42) at (42,-124) {};
%\node[box=0] (p-124-43) at (43,-124) {};
%\node[box=1] (p-124-44) at (44,-124) {};
%\node[box=0] (p-124-45) at (45,-124) {};
%\node[box=0] (p-124-46) at (46,-124) {};
%\node[box=0] (p-124-47) at (47,-124) {};
%\node[box=1] (p-124-48) at (48,-124) {};
%\node[box=0] (p-124-49) at (49,-124) {};
%\node[box=0] (p-124-50) at (50,-124) {};
%\node[box=0] (p-124-51) at (51,-124) {};
%\node[box=1] (p-124-52) at (52,-124) {};
%\node[box=0] (p-124-53) at (53,-124) {};
%\node[box=0] (p-124-54) at (54,-124) {};
%\node[box=0] (p-124-55) at (55,-124) {};
%\node[box=1] (p-124-56) at (56,-124) {};
%\node[box=0] (p-124-57) at (57,-124) {};
%\node[box=0] (p-124-58) at (58,-124) {};
%\node[box=0] (p-124-59) at (59,-124) {};
%\node[box=1] (p-124-60) at (60,-124) {};
%\node[box=0] (p-124-61) at (61,-124) {};
%\node[box=0] (p-124-62) at (62,-124) {};
%\node[box=0] (p-124-63) at (63,-124) {};
%\node[box=1] (p-124-64) at (64,-124) {};
%\node[box=0] (p-124-65) at (65,-124) {};
%\node[box=0] (p-124-66) at (66,-124) {};
%\node[box=0] (p-124-67) at (67,-124) {};
%\node[box=1] (p-124-68) at (68,-124) {};
%\node[box=0] (p-124-69) at (69,-124) {};
%\node[box=0] (p-124-70) at (70,-124) {};
%\node[box=0] (p-124-71) at (71,-124) {};
%\node[box=1] (p-124-72) at (72,-124) {};
%\node[box=0] (p-124-73) at (73,-124) {};
%\node[box=0] (p-124-74) at (74,-124) {};
%\node[box=0] (p-124-75) at (75,-124) {};
%\node[box=1] (p-124-76) at (76,-124) {};
%\node[box=0] (p-124-77) at (77,-124) {};
%\node[box=0] (p-124-78) at (78,-124) {};
%\node[box=0] (p-124-79) at (79,-124) {};
%\node[box=1] (p-124-80) at (80,-124) {};
%\node[box=0] (p-124-81) at (81,-124) {};
%\node[box=0] (p-124-82) at (82,-124) {};
%\node[box=0] (p-124-83) at (83,-124) {};
%\node[box=1] (p-124-84) at (84,-124) {};
%\node[box=0] (p-124-85) at (85,-124) {};
%\node[box=0] (p-124-86) at (86,-124) {};
%\node[box=0] (p-124-87) at (87,-124) {};
%\node[box=1] (p-124-88) at (88,-124) {};
%\node[box=0] (p-124-89) at (89,-124) {};
%\node[box=0] (p-124-90) at (90,-124) {};
%\node[box=0] (p-124-91) at (91,-124) {};
%\node[box=1] (p-124-92) at (92,-124) {};
%\node[box=0] (p-124-93) at (93,-124) {};
%\node[box=0] (p-124-94) at (94,-124) {};
%\node[box=0] (p-124-95) at (95,-124) {};
%\node[box=1] (p-124-96) at (96,-124) {};
%\node[box=0] (p-124-97) at (97,-124) {};
%\node[box=0] (p-124-98) at (98,-124) {};
%\node[box=0] (p-124-99) at (99,-124) {};
%\node[box=1] (p-124-100) at (100,-124) {};
%\node[box=0] (p-124-101) at (101,-124) {};
%\node[box=0] (p-124-102) at (102,-124) {};
%\node[box=0] (p-124-103) at (103,-124) {};
%\node[box=1] (p-124-104) at (104,-124) {};
%\node[box=0] (p-124-105) at (105,-124) {};
%\node[box=0] (p-124-106) at (106,-124) {};
%\node[box=0] (p-124-107) at (107,-124) {};
%\node[box=1] (p-124-108) at (108,-124) {};
%\node[box=0] (p-124-109) at (109,-124) {};
%\node[box=0] (p-124-110) at (110,-124) {};
%\node[box=0] (p-124-111) at (111,-124) {};
%\node[box=1] (p-124-112) at (112,-124) {};
%\node[box=0] (p-124-113) at (113,-124) {};
%\node[box=0] (p-124-114) at (114,-124) {};
%\node[box=0] (p-124-115) at (115,-124) {};
%\node[box=1] (p-124-116) at (116,-124) {};
%\node[box=0] (p-124-117) at (117,-124) {};
%\node[box=0] (p-124-118) at (118,-124) {};
%\node[box=0] (p-124-119) at (119,-124) {};
%\node[box=1] (p-124-120) at (120,-124) {};
%\node[box=0] (p-124-121) at (121,-124) {};
%\node[box=0] (p-124-122) at (122,-124) {};
%\node[box=0] (p-124-123) at (123,-124) {};
%\node[box=1] (p-124-124) at (124,-124) {};
%\node[box=1] (p-125-0) at (0,-125) {};
%\node[box=1] (p-125-1) at (1,-125) {};
%\node[box=0] (p-125-2) at (2,-125) {};
%\node[box=0] (p-125-3) at (3,-125) {};
%\node[box=1] (p-125-4) at (4,-125) {};
%\node[box=1] (p-125-5) at (5,-125) {};
%\node[box=0] (p-125-6) at (6,-125) {};
%\node[box=0] (p-125-7) at (7,-125) {};
%\node[box=1] (p-125-8) at (8,-125) {};
%\node[box=1] (p-125-9) at (9,-125) {};
%\node[box=0] (p-125-10) at (10,-125) {};
%\node[box=0] (p-125-11) at (11,-125) {};
%\node[box=1] (p-125-12) at (12,-125) {};
%\node[box=1] (p-125-13) at (13,-125) {};
%\node[box=0] (p-125-14) at (14,-125) {};
%\node[box=0] (p-125-15) at (15,-125) {};
%\node[box=1] (p-125-16) at (16,-125) {};
%\node[box=1] (p-125-17) at (17,-125) {};
%\node[box=0] (p-125-18) at (18,-125) {};
%\node[box=0] (p-125-19) at (19,-125) {};
%\node[box=1] (p-125-20) at (20,-125) {};
%\node[box=1] (p-125-21) at (21,-125) {};
%\node[box=0] (p-125-22) at (22,-125) {};
%\node[box=0] (p-125-23) at (23,-125) {};
%\node[box=1] (p-125-24) at (24,-125) {};
%\node[box=1] (p-125-25) at (25,-125) {};
%\node[box=0] (p-125-26) at (26,-125) {};
%\node[box=0] (p-125-27) at (27,-125) {};
%\node[box=1] (p-125-28) at (28,-125) {};
%\node[box=1] (p-125-29) at (29,-125) {};
%\node[box=0] (p-125-30) at (30,-125) {};
%\node[box=0] (p-125-31) at (31,-125) {};
%\node[box=1] (p-125-32) at (32,-125) {};
%\node[box=1] (p-125-33) at (33,-125) {};
%\node[box=0] (p-125-34) at (34,-125) {};
%\node[box=0] (p-125-35) at (35,-125) {};
%\node[box=1] (p-125-36) at (36,-125) {};
%\node[box=1] (p-125-37) at (37,-125) {};
%\node[box=0] (p-125-38) at (38,-125) {};
%\node[box=0] (p-125-39) at (39,-125) {};
%\node[box=1] (p-125-40) at (40,-125) {};
%\node[box=1] (p-125-41) at (41,-125) {};
%\node[box=0] (p-125-42) at (42,-125) {};
%\node[box=0] (p-125-43) at (43,-125) {};
%\node[box=1] (p-125-44) at (44,-125) {};
%\node[box=1] (p-125-45) at (45,-125) {};
%\node[box=0] (p-125-46) at (46,-125) {};
%\node[box=0] (p-125-47) at (47,-125) {};
%\node[box=1] (p-125-48) at (48,-125) {};
%\node[box=1] (p-125-49) at (49,-125) {};
%\node[box=0] (p-125-50) at (50,-125) {};
%\node[box=0] (p-125-51) at (51,-125) {};
%\node[box=1] (p-125-52) at (52,-125) {};
%\node[box=1] (p-125-53) at (53,-125) {};
%\node[box=0] (p-125-54) at (54,-125) {};
%\node[box=0] (p-125-55) at (55,-125) {};
%\node[box=1] (p-125-56) at (56,-125) {};
%\node[box=1] (p-125-57) at (57,-125) {};
%\node[box=0] (p-125-58) at (58,-125) {};
%\node[box=0] (p-125-59) at (59,-125) {};
%\node[box=1] (p-125-60) at (60,-125) {};
%\node[box=1] (p-125-61) at (61,-125) {};
%\node[box=0] (p-125-62) at (62,-125) {};
%\node[box=0] (p-125-63) at (63,-125) {};
%\node[box=1] (p-125-64) at (64,-125) {};
%\node[box=1] (p-125-65) at (65,-125) {};
%\node[box=0] (p-125-66) at (66,-125) {};
%\node[box=0] (p-125-67) at (67,-125) {};
%\node[box=1] (p-125-68) at (68,-125) {};
%\node[box=1] (p-125-69) at (69,-125) {};
%\node[box=0] (p-125-70) at (70,-125) {};
%\node[box=0] (p-125-71) at (71,-125) {};
%\node[box=1] (p-125-72) at (72,-125) {};
%\node[box=1] (p-125-73) at (73,-125) {};
%\node[box=0] (p-125-74) at (74,-125) {};
%\node[box=0] (p-125-75) at (75,-125) {};
%\node[box=1] (p-125-76) at (76,-125) {};
%\node[box=1] (p-125-77) at (77,-125) {};
%\node[box=0] (p-125-78) at (78,-125) {};
%\node[box=0] (p-125-79) at (79,-125) {};
%\node[box=1] (p-125-80) at (80,-125) {};
%\node[box=1] (p-125-81) at (81,-125) {};
%\node[box=0] (p-125-82) at (82,-125) {};
%\node[box=0] (p-125-83) at (83,-125) {};
%\node[box=1] (p-125-84) at (84,-125) {};
%\node[box=1] (p-125-85) at (85,-125) {};
%\node[box=0] (p-125-86) at (86,-125) {};
%\node[box=0] (p-125-87) at (87,-125) {};
%\node[box=1] (p-125-88) at (88,-125) {};
%\node[box=1] (p-125-89) at (89,-125) {};
%\node[box=0] (p-125-90) at (90,-125) {};
%\node[box=0] (p-125-91) at (91,-125) {};
%\node[box=1] (p-125-92) at (92,-125) {};
%\node[box=1] (p-125-93) at (93,-125) {};
%\node[box=0] (p-125-94) at (94,-125) {};
%\node[box=0] (p-125-95) at (95,-125) {};
%\node[box=1] (p-125-96) at (96,-125) {};
%\node[box=1] (p-125-97) at (97,-125) {};
%\node[box=0] (p-125-98) at (98,-125) {};
%\node[box=0] (p-125-99) at (99,-125) {};
%\node[box=1] (p-125-100) at (100,-125) {};
%\node[box=1] (p-125-101) at (101,-125) {};
%\node[box=0] (p-125-102) at (102,-125) {};
%\node[box=0] (p-125-103) at (103,-125) {};
%\node[box=1] (p-125-104) at (104,-125) {};
%\node[box=1] (p-125-105) at (105,-125) {};
%\node[box=0] (p-125-106) at (106,-125) {};
%\node[box=0] (p-125-107) at (107,-125) {};
%\node[box=1] (p-125-108) at (108,-125) {};
%\node[box=1] (p-125-109) at (109,-125) {};
%\node[box=0] (p-125-110) at (110,-125) {};
%\node[box=0] (p-125-111) at (111,-125) {};
%\node[box=1] (p-125-112) at (112,-125) {};
%\node[box=1] (p-125-113) at (113,-125) {};
%\node[box=0] (p-125-114) at (114,-125) {};
%\node[box=0] (p-125-115) at (115,-125) {};
%\node[box=1] (p-125-116) at (116,-125) {};
%\node[box=1] (p-125-117) at (117,-125) {};
%\node[box=0] (p-125-118) at (118,-125) {};
%\node[box=0] (p-125-119) at (119,-125) {};
%\node[box=1] (p-125-120) at (120,-125) {};
%\node[box=1] (p-125-121) at (121,-125) {};
%\node[box=0] (p-125-122) at (122,-125) {};
%\node[box=0] (p-125-123) at (123,-125) {};
%\node[box=1] (p-125-124) at (124,-125) {};
%\node[box=1] (p-125-125) at (125,-125) {};
%\node[box=1] (p-126-0) at (0,-126) {};
%\node[box=0] (p-126-1) at (1,-126) {};
%\node[box=1] (p-126-2) at (2,-126) {};
%\node[box=0] (p-126-3) at (3,-126) {};
%\node[box=1] (p-126-4) at (4,-126) {};
%\node[box=0] (p-126-5) at (5,-126) {};
%\node[box=1] (p-126-6) at (6,-126) {};
%\node[box=0] (p-126-7) at (7,-126) {};
%\node[box=1] (p-126-8) at (8,-126) {};
%\node[box=0] (p-126-9) at (9,-126) {};
%\node[box=1] (p-126-10) at (10,-126) {};
%\node[box=0] (p-126-11) at (11,-126) {};
%\node[box=1] (p-126-12) at (12,-126) {};
%\node[box=0] (p-126-13) at (13,-126) {};
%\node[box=1] (p-126-14) at (14,-126) {};
%\node[box=0] (p-126-15) at (15,-126) {};
%\node[box=1] (p-126-16) at (16,-126) {};
%\node[box=0] (p-126-17) at (17,-126) {};
%\node[box=1] (p-126-18) at (18,-126) {};
%\node[box=0] (p-126-19) at (19,-126) {};
%\node[box=1] (p-126-20) at (20,-126) {};
%\node[box=0] (p-126-21) at (21,-126) {};
%\node[box=1] (p-126-22) at (22,-126) {};
%\node[box=0] (p-126-23) at (23,-126) {};
%\node[box=1] (p-126-24) at (24,-126) {};
%\node[box=0] (p-126-25) at (25,-126) {};
%\node[box=1] (p-126-26) at (26,-126) {};
%\node[box=0] (p-126-27) at (27,-126) {};
%\node[box=1] (p-126-28) at (28,-126) {};
%\node[box=0] (p-126-29) at (29,-126) {};
%\node[box=1] (p-126-30) at (30,-126) {};
%\node[box=0] (p-126-31) at (31,-126) {};
%\node[box=1] (p-126-32) at (32,-126) {};
%\node[box=0] (p-126-33) at (33,-126) {};
%\node[box=1] (p-126-34) at (34,-126) {};
%\node[box=0] (p-126-35) at (35,-126) {};
%\node[box=1] (p-126-36) at (36,-126) {};
%\node[box=0] (p-126-37) at (37,-126) {};
%\node[box=1] (p-126-38) at (38,-126) {};
%\node[box=0] (p-126-39) at (39,-126) {};
%\node[box=1] (p-126-40) at (40,-126) {};
%\node[box=0] (p-126-41) at (41,-126) {};
%\node[box=1] (p-126-42) at (42,-126) {};
%\node[box=0] (p-126-43) at (43,-126) {};
%\node[box=1] (p-126-44) at (44,-126) {};
%\node[box=0] (p-126-45) at (45,-126) {};
%\node[box=1] (p-126-46) at (46,-126) {};
%\node[box=0] (p-126-47) at (47,-126) {};
%\node[box=1] (p-126-48) at (48,-126) {};
%\node[box=0] (p-126-49) at (49,-126) {};
%\node[box=1] (p-126-50) at (50,-126) {};
%\node[box=0] (p-126-51) at (51,-126) {};
%\node[box=1] (p-126-52) at (52,-126) {};
%\node[box=0] (p-126-53) at (53,-126) {};
%\node[box=1] (p-126-54) at (54,-126) {};
%\node[box=0] (p-126-55) at (55,-126) {};
%\node[box=1] (p-126-56) at (56,-126) {};
%\node[box=0] (p-126-57) at (57,-126) {};
%\node[box=1] (p-126-58) at (58,-126) {};
%\node[box=0] (p-126-59) at (59,-126) {};
%\node[box=1] (p-126-60) at (60,-126) {};
%\node[box=0] (p-126-61) at (61,-126) {};
%\node[box=1] (p-126-62) at (62,-126) {};
%\node[box=0] (p-126-63) at (63,-126) {};
%\node[box=1] (p-126-64) at (64,-126) {};
%\node[box=0] (p-126-65) at (65,-126) {};
%\node[box=1] (p-126-66) at (66,-126) {};
%\node[box=0] (p-126-67) at (67,-126) {};
%\node[box=1] (p-126-68) at (68,-126) {};
%\node[box=0] (p-126-69) at (69,-126) {};
%\node[box=1] (p-126-70) at (70,-126) {};
%\node[box=0] (p-126-71) at (71,-126) {};
%\node[box=1] (p-126-72) at (72,-126) {};
%\node[box=0] (p-126-73) at (73,-126) {};
%\node[box=1] (p-126-74) at (74,-126) {};
%\node[box=0] (p-126-75) at (75,-126) {};
%\node[box=1] (p-126-76) at (76,-126) {};
%\node[box=0] (p-126-77) at (77,-126) {};
%\node[box=1] (p-126-78) at (78,-126) {};
%\node[box=0] (p-126-79) at (79,-126) {};
%\node[box=1] (p-126-80) at (80,-126) {};
%\node[box=0] (p-126-81) at (81,-126) {};
%\node[box=1] (p-126-82) at (82,-126) {};
%\node[box=0] (p-126-83) at (83,-126) {};
%\node[box=1] (p-126-84) at (84,-126) {};
%\node[box=0] (p-126-85) at (85,-126) {};
%\node[box=1] (p-126-86) at (86,-126) {};
%\node[box=0] (p-126-87) at (87,-126) {};
%\node[box=1] (p-126-88) at (88,-126) {};
%\node[box=0] (p-126-89) at (89,-126) {};
%\node[box=1] (p-126-90) at (90,-126) {};
%\node[box=0] (p-126-91) at (91,-126) {};
%\node[box=1] (p-126-92) at (92,-126) {};
%\node[box=0] (p-126-93) at (93,-126) {};
%\node[box=1] (p-126-94) at (94,-126) {};
%\node[box=0] (p-126-95) at (95,-126) {};
%\node[box=1] (p-126-96) at (96,-126) {};
%\node[box=0] (p-126-97) at (97,-126) {};
%\node[box=1] (p-126-98) at (98,-126) {};
%\node[box=0] (p-126-99) at (99,-126) {};
%\node[box=1] (p-126-100) at (100,-126) {};
%\node[box=0] (p-126-101) at (101,-126) {};
%\node[box=1] (p-126-102) at (102,-126) {};
%\node[box=0] (p-126-103) at (103,-126) {};
%\node[box=1] (p-126-104) at (104,-126) {};
%\node[box=0] (p-126-105) at (105,-126) {};
%\node[box=1] (p-126-106) at (106,-126) {};
%\node[box=0] (p-126-107) at (107,-126) {};
%\node[box=1] (p-126-108) at (108,-126) {};
%\node[box=0] (p-126-109) at (109,-126) {};
%\node[box=1] (p-126-110) at (110,-126) {};
%\node[box=0] (p-126-111) at (111,-126) {};
%\node[box=1] (p-126-112) at (112,-126) {};
%\node[box=0] (p-126-113) at (113,-126) {};
%\node[box=1] (p-126-114) at (114,-126) {};
%\node[box=0] (p-126-115) at (115,-126) {};
%\node[box=1] (p-126-116) at (116,-126) {};
%\node[box=0] (p-126-117) at (117,-126) {};
%\node[box=1] (p-126-118) at (118,-126) {};
%\node[box=0] (p-126-119) at (119,-126) {};
%\node[box=1] (p-126-120) at (120,-126) {};
%\node[box=0] (p-126-121) at (121,-126) {};
%\node[box=1] (p-126-122) at (122,-126) {};
%\node[box=0] (p-126-123) at (123,-126) {};
%\node[box=1] (p-126-124) at (124,-126) {};
%\node[box=0] (p-126-125) at (125,-126) {};
%\node[box=1] (p-126-126) at (126,-126) {};

\end{tikzpicture}

\end{document}
%%%%%%%%%%%%%%%%%%%%%%%%%%%%%%%%%
