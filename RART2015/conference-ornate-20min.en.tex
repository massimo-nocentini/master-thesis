% $Header: /Users/joseph/Documents/LaTeX/beamer/solutions/conference-talks/conference-ornate-20min.en.tex,v 90e850259b8b 2007/01/28 20:48:30 tantau $

\documentclass[serif, professionalfont]{beamer}

% This file is a solution template for:

% - Talk at a conference/colloquium.
% - Talk length is about 20min.
% - Style is ornate.



% Copyright 2004 by Till Tantau <tantau@users.sourceforge.net>.
%
% In principle, this file can be redistributed and/or modified under
% the terms of the GNU Public License, version 2.
%
% However, this file is supposed to be a template to be modified
% for your own needs. For this reason, if you use this file as a
% template and not specifically distribute it as part of a another
% package/program, I grant the extra permission to freely copy and
% modify this file as you see fit and even to delete this copyright
% notice. 


\mode<presentation>
{
  %\usetheme{Singapore}
  %\usetheme{Goettingen}
  \usetheme{Pittsburgh}
  % or ...

  \setbeamercovered{transparent}
  % or whatever (possibly just delete it)

  \usecolortheme{dove}
}


\usepackage[english]{babel}
% or whatever

%\usepackage[latin1]{inputenc}
% or whatever

%\usepackage{times}
\usepackage[T1]{fontenc}
% Or whatever. Note that the encoding and the font should match. If T1
% does not look nice, try deleting the line with the fontenc.

\usepackage{concrete}

\usepackage{graphicx}
\usepackage{float}
\usepackage{amsmath}
\usepackage{amsthm}
\usepackage{amssymb}
\usepackage{caption}


\title%[Short Paper Title] % (optional, use only with long paper titles)
{Patterns in Riordan arrays}

%\subtitle {Include Only If Paper Has a Subtitle}

\author[Merlini, Nocentini] % (optional, use only with lots of authors)
{Donatella Merlini\inst{1} \and Massimo Nocentini}
% - Give the names in the same order as the appear in the paper.
% - Use the \inst{?} command only if the authors have different
%   affiliation.

\institute[Universities of Somewhere and Elsewhere] % (optional, but mostly needed)
{
  \inst{1}%
  Dipartimento di Statistica, Informatica, Applicazioni\\
  Universit\`a di Firenze, Italy
  %\and
  %\inst{2}%
  %Department of Theoretical Philosophy\\
  %University of Elsewhere
    \begin{center}
        \includegraphics[width=3cm]{logo/unifi} \\ \medskip
    \end{center}
  }
% - Use the \inst command only if there are several affiliations.
% - Keep it simple, no one is interested in your street address.


\date[RART2015] % (optional, should be abbreviation of conference name)
{RART, 2015}
% - Either use conference name or its abbreviation.
% - Not really informative to the audience, more for people (including
%   yourself) who are reading the slides online

\subject{Theoretical Computer Science}
% This is only inserted into the PDF information catalog. Can be left
% out. 



% If you have a file called "university-logo-filename.xxx", where xxx
% is a graphic format that can be processed by latex or pdflatex,
% resp., then you can add a logo as follows:

\pgfdeclareimage[height=0.5cm]{university-logo}{logo/unifi}
\logo{\pgfuseimage{university-logo}}



% Delete this, if you do not want the table of contents to pop up at
% the beginning of each subsection:
\AtBeginSubsection[]
{
  \begin{frame}<beamer>{Outline}
    \tableofcontents[currentsection,currentsubsection]
  \end{frame}
}


% If you wish to uncover everything in a step-wise fashion, uncomment
% the following command: 

%\beamerdefaultoverlayspecification{<+->}


\begin{document}

\begin{frame}
  \titlepage
\end{frame}

\begin{frame}{Outline}
  \tableofcontents
  % You might wish to add the option [pausesections]
\end{frame}


% Structuring a talk is a difficult task and the following structure
% may not be suitable. Here are some rules that apply for this
% solution: 

% - Exactly two or three sections (other than the summary).
% - At *most* three subsections per section.
% - Talk about 30s to 2min per frame. So there should be between about
%   15 and 30 frames, all told.

% - A conference audience is likely to know very little of what you
%   are going to talk about. So *simplify*!
% - In a 20min talk, getting the main ideas across is hard
%   enough. Leave out details, even if it means being less precise than
%   you think necessary.
% - If you omit details that are vital to the proof/implementation,
%   just say so once. Everybody will be happy with that.

\section{Motivation}

\subsection{Some definitions}

\begin{frame}{Riordan arrays}
    We consider  Riordan arrays of the form $\mathcal{R}\left( d(t), h(t)\right)$
    where:
    \begin{displaymath}
        d_{nk}\in\mathcal{R} \leftrightarrow d_{nk} = \big[t^{n}\big]d(t)\,h(t)^{k}
    \end{displaymath}

    Let $\mathcal{R}$ be a Riordan array and $p$ be a prime.
    
    \begin{block}{$\mathcal{R}_{\equiv_{p}}$}
        Define $\mathcal{R}_{\equiv_{p}}$
        as the \emph{congruent of} $\mathcal{R}$, modulo $p$, such that:
        \begin{displaymath}
            d_{nk}\in\mathcal{R} \leftrightarrow a \in\mathcal{R}_{\equiv_{p}}
            \quad \text{where} \quad d_{nk}\equiv_{p} a
        \end{displaymath}
    \end{block}

    \begin{block}{$\mathcal{R}^{(\alpha)}$}
        Define $\mathcal{R}^{(\alpha)}$
        as the \emph{principal $\alpha$-cluster of} $\mathcal{R}$, such that:
        \begin{displaymath}
            d_{nk}\in\mathcal{R}^{(\alpha)}\rightarrow d_{nk}\in\mathcal{R}
            \quad \text{where}\quad n\in\lbrace0,\ldots,p^{\alpha}-1\rbrace
        \end{displaymath}
    \end{block}
    

\end{frame}

\begin{frame}{Riordan arrays}
    $\mathcal{P}$ascal and $\mathcal{F}$ibonacci arrays, respectively:
    \begin{displaymath}
            \mathcal{P}\left(\frac{1}{1-t},\frac{t}{1-t}\right) \quad\quad 
                \mathcal{F}\left(\frac{1}{1-t-t^2}, \frac{1-\sqrt{1-4t}}{2} \right)
    \end{displaymath}

    $\mathcal{D}$elannoy and $\mathcal{C}$atalan arrays, respectively:
    \begin{displaymath}
        \mathcal{D}\left(\frac{1}{1-t},\frac{t(1+t)}{1-t}\right) \quad\quad
            \mathcal{C}\left(\frac{1-\sqrt{1-4t}}{2t}, \frac{1-\sqrt{1-4t}}{2}\right)
    \end{displaymath}

    finally, $\mathcal{M}$otzkin array:
    \begin{displaymath}
        \mathcal{M}\left(\frac{1-t-\sqrt{1-2t-3t^2}}{2t^2}, \frac{1-t-\sqrt{1-2t-3t^2}}{2t}\right)
    \end{displaymath}
\end{frame}

\subsection{Previous Work}

\begin{frame}{Sved, McIlroy, Lucas theorem et al.}
Put here some thoughts on works by Marta Sved, McIlroy
and recall Lucas theorem and its presentation by Fine.
\end{frame}

\section{Some arrays and their congruents}

\subsection{Colouring for some primes}

%\subsection{Pascal}

\begin{frame}{$\mathcal{P}_{\equiv_{2}}$ where 
    $\textcolor{blue}{[0]_{2}},\textcolor{orange}{[1]_{2}}$}

    \input{../sympy/pascal/pascal-standard-ignore-negatives-centered-colouring-127-rows-mod2-partitioning-include-figure}
\end{frame}

\begin{frame}{$\mathcal{P}_{\equiv_{3}}$ where 
    $\textcolor{blue}{[0]_{3}},\textcolor{orange}{[1]_{3}},\textcolor{red}{[2]_{3}}$}

    \input{../sympy/pascal/pascal-standard-ignore-negatives-centered-colouring-127-rows-mod3-partitioning-include-figure}
\end{frame}
\begin{frame}{$\mathcal{P}_{\stackrel{\circ}{\equiv_{3}}}$ where 
    $\textcolor{blue}{[0]_{3}},\textcolor{orange}{\lbrace [1],[2] \rbrace_{3}}$}

    \input{../sympy/pascal/pascal-standard-ignore-negatives-centered-colouring-127-rows-multiples-of-3-partitioning-include-figure}
\end{frame}
\begin{frame}{$\mathcal{P}_{\stackrel{\circ}{\equiv_{5}}}$ where 
    $\textcolor{blue}{[0]_{5}},\textcolor{orange}{\lbrace [1],[2],[3],[4] \rbrace_{5}}$}

    \input{../sympy/pascal/pascal-standard-ignore-negatives-centered-colouring-127-rows-multiples-of-5-partitioning-include-figure}
\end{frame}
\begin{frame}{$\mathcal{P}_{\stackrel{\circ}{\equiv_{7}}}$ where 
    $\textcolor{blue}{[0]_{7}},\textcolor{orange}{\lbrace [1],[2],[3],[4],[5],[6] \rbrace_{7}}$}

    
\begin{figure}[H]

    \hspace{1cm}
    \noindent\makebox[\textwidth]{
        \centering
        %\includegraphics[width=0.8\textwidth]{../../sympy/catalan/coloured.pdf}

        % using *angle* property to rotate it is difficult to properly align it
        % in order to have a "real" matrix representation.
        \includegraphics[width=15cm, height=15cm, keepaspectratio=true]{../sympy/pascal/pascal-standard-ignore-negatives-centered-colouring-127-rows-multiples-of-7-partitioning-triangle.pdf}
    }

    % this 'particular' line is necessary to use `displaymath' environment
    % into the caption environment, togheter with the inclusion of 
    % `caption' package. See here for more explanation:
    % http://stackoverflow.com/questions/2716227/adding-an-equation-or-formula-to-a-figure-caption-in-latex
    \captionsetup{singlelinecheck=off}
    \caption[$\mathcal{P}_{\stackrel{\circ}{\equiv_{7}}}$]{ $\mathcal{P}_{\stackrel{\circ}{\equiv_{7}}}$ }

    \label{fig:pascal-standard-ignore-negatives-centered-colouring-127-rows-multiples-of-7-partitioning-triangle}

\end{figure}

\end{frame}

%\subsection{Fibonacci}

\begin{frame}{$\mathcal{F}_{\equiv_{2}}$ where 
    $\textcolor{blue}{[0]_{2}},\textcolor{orange}{[1]_{2}}$}

    
\begin{figure}[H]

    \hspace{-1.5cm}
    \noindent\makebox[\textwidth]{
        \centering
        %\includegraphics[width=0.8\textwidth]{../../sympy/catalan/coloured.pdf}

        % using *angle* property to rotate it is difficult to properly align it
        % in order to have a "real" matrix representation.
        \includegraphics[width=15cm, height=15cm, keepaspectratio=true]{../sympy/fibonacci/fibonacci-standard-handle-negatives-centered-colouring-127-rows-mod2-partitioning-triangle.pdf}
    }

    % this 'particular' line is necessary to use `displaymath' environment
    % into the caption environment, togheter with the inclusion of 
    % `caption' package. See here for more explanation:
    % http://stackoverflow.com/questions/2716227/adding-an-equation-or-formula-to-a-figure-caption-in-latex
    \captionsetup{singlelinecheck=off}
    \caption[$\mathcal{F}_{\stackrel{\circ}{\equiv_{2}}}$]{ $\mathcal{F}_{\stackrel{\circ}{\equiv_{2}}}$ }

    \label{fig:fibonacci-standard-handle-negatives-centered-colouring-127-rows-mod2-partitioning-triangle}

\end{figure}

\end{frame}
%\begin{frame}{$\mathcal{F}_{\equiv_{3}}$ where 
    %$\textcolor{blue}{[0]_{3}},\textcolor{orange}{[1]_{3}},\textcolor{red}{[2]_{3}}$}

    %\input{../sympy/fibonacci/fibonacci-standard-ignore-negatives-centered-colouring-127-rows-mod3-partitioning-include-figure}
%\end{frame}
\begin{frame}{$\mathcal{F}_{\stackrel{\circ}{\equiv_{3}}}$ where 
    $\textcolor{blue}{[0]_{3}},\textcolor{orange}{\lbrace [1],[2] \rbrace_{3}}$}

    \input{../sympy/fibonacci/fibonacci-standard-handle-negatives-centered-colouring-127-rows-multiples-of-3-partitioning-include-figure}
\end{frame}
\begin{frame}{$\mathcal{F}_{\stackrel{\circ}{\equiv_{5}}}$ where 
    $\textcolor{blue}{[0]_{5}},\textcolor{orange}{\lbrace [1],[2],[3],[4] \rbrace_{5}}$}

    \input{../sympy/fibonacci/fibonacci-standard-handle-negatives-centered-colouring-127-rows-multiples-of-5-partitioning-include-figure}
\end{frame}
\begin{frame}{$\mathcal{F}_{\stackrel{\circ}{\equiv_{7}}}$ where 
    $\textcolor{blue}{[0]_{7}},\textcolor{orange}{\lbrace [1],[2],[3],[4],[5],[6] \rbrace_{7}}$}

    \input{../sympy/fibonacci/fibonacci-standard-handle-negatives-centered-colouring-127-rows-multiples-of-7-partitioning-include-figure}
\end{frame}


%\subsection{Catalan}
\begin{frame}{$\mathcal{C}_{\equiv_{2}}$ where 
    $\textcolor{blue}{[0]_{2}},\textcolor{orange}{[1]_{2}}$}

    \input{../sympy/catalan/Catalan-traditional-standard-handle-negatives-centered-colouring-127-rows-mod2-partitioning-include-figure.tex}
\end{frame}
%\begin{frame}{$\mathcal{C}_{\equiv_{3}}$ where 
    %$\textcolor{blue}{[0]_{3}},\textcolor{orange}{[1]_{3}},\textcolor{red}{[2]_{3}}$}

    %\input{../sympy/catalan/catalan-traditional-standard-handle-negatives-centered-colouring-127-rows-mod3-partitioning-include-figure}
%\end{frame}
\begin{frame}{$\mathcal{C}_{\stackrel{\circ}{\equiv_{3}}}$ where 
    $\textcolor{blue}{[0]_{3}},\textcolor{orange}{\lbrace [1],[2] \rbrace_{3}}$}

    \input{../sympy/catalan/catalan-traditional-standard-handle-negatives-centered-colouring-127-rows-multiples-of-3-partitioning-include-figure}
\end{frame}
\begin{frame}{$\mathcal{C}_{\stackrel{\circ}{\equiv_{5}}}$ where 
    $\textcolor{blue}{[0]_{5}},\textcolor{orange}{\lbrace [1],[2],[3],[4] \rbrace_{5}}$}

    \input{../sympy/catalan/catalan-traditional-standard-handle-negatives-centered-colouring-127-rows-multiples-of-5-partitioning-include-figure}
\end{frame}
\begin{frame}{$\mathcal{C}_{\stackrel{\circ}{\equiv_{7}}}$ where 
    $\textcolor{blue}{[0]_{7}},\textcolor{orange}{\lbrace [1],[2],[3],[4],[5],[6] \rbrace_{7}}$}

    \input{../sympy/catalan/catalan-traditional-standard-handle-negatives-centered-colouring-127-rows-multiples-of-7-partitioning-include-figure}
\end{frame}

%\subsection{Motzkin}

\begin{frame}{$\mathcal{M}_{\equiv_{2}}$ where 
    $\textcolor{blue}{[0]_{2}},\textcolor{orange}{[1]_{2}}$}

    \input{../sympy/motzkin/motzkin-standard-ignore-negatives-centered-colouring-127-rows-mod2-partitioning-include-figure.tex}
\end{frame}
%\begin{frame}{$\mathcal{M}_{\equiv_{3}}$ where 
    %$\textcolor{blue}{[0]_{3}},\textcolor{orange}{[1]_{3}},\textcolor{red}{[2]_{3}}$}

    %\input{../sympy/motzkin/motzkin-standard-ignore-negatives-centered-colouring-127-rows-mod3-partitioning-include-figure}
%\end{frame}
\begin{frame}{$\mathcal{M}_{\stackrel{\circ}{\equiv_{3}}}$ where 
    $\textcolor{blue}{[0]_{3}},\textcolor{orange}{\lbrace [1],[2] \rbrace_{3}}$}

    
\begin{figure}[H]

    \noindent\makebox[\textwidth]{
        \centering
        %\includegraphics[width=0.8\textwidth]{../../sympy/catalan/coloured.pdf}

        % using *angle* property to rotate it is difficult to properly align it
        % in order to have a "real" matrix representation.
        \includegraphics[width=20cm, height=20cm, keepaspectratio=true]{../sympy/motzkin/motzkin-standard-ignore-negatives-centered-colouring-127-rows-multiples-of-3-partitioning-triangle.pdf}
    }

    % this 'particular' line is necessary to use `displaymath' environment
    % into the caption environment, togheter with the inclusion of 
    % `caption' package. See here for more explanation:
    % http://stackoverflow.com/questions/2716227/adding-an-equation-or-formula-to-a-figure-caption-in-latex
    \captionsetup{singlelinecheck=off}
    \caption[$\mathcal{M}_{\stackrel{\circ}{\equiv_{3}}}$]{ $\mathcal{M}_{\stackrel{\circ}{\equiv_{3}}}$ }

    \label{fig:motzkin-standard-ignore-negatives-centered-colouring-127-rows-multiples-of-3-partitioning-triangle}

\end{figure}

\end{frame}
\begin{frame}{$\mathcal{M}_{\stackrel{\circ}{\equiv_{5}}}$ where 
    $\textcolor{blue}{[0]_{5}},\textcolor{orange}{\lbrace [1],[2],[3],[4] \rbrace_{5}}$}

    \input{../sympy/motzkin/motzkin-standard-ignore-negatives-centered-colouring-127-rows-multiples-of-5-partitioning-include-figure}
\end{frame}
\begin{frame}{$\mathcal{M}_{\stackrel{\circ}{\equiv_{7}}}$ where 
    $\textcolor{blue}{[0]_{7}},\textcolor{orange}{\lbrace [1],[2],[3],[4],[5],[6] \rbrace_{7}}$}

    \input{../sympy/motzkin/motzkin-standard-ignore-negatives-centered-colouring-127-rows-multiples-of-7-partitioning-include-figure}
\end{frame}


%\subsection{Delannoy}

\begin{frame}{$\mathcal{D}_{\stackrel{\circ}{\equiv_{3}}}$ where 
    $\textcolor{blue}{[0]_{3}},\textcolor{orange}{\lbrace [1],[2] \rbrace_{3}}$}

    \input{../sympy/delannoy/delannoy-standard-handle-negatives-centered-colouring-127-rows-multiples-of-3-partitioning-include-figure}
\end{frame}
\begin{frame}{$\mathcal{D}_{\stackrel{\circ}{\equiv_{5}}}$ where 
    $\textcolor{blue}{[0]_{5}},\textcolor{orange}{\lbrace [1],[2],[3],[4] \rbrace_{5}}$}

    
\begin{figure}[H]

    \hspace{1cm}
    \noindent\makebox[\textwidth]{
        \centering
        %\includegraphics[width=0.8\textwidth]{../../sympy/catalan/coloured.pdf}

        % using *angle* property to rotate it is difficult to properly align it
        % in order to have a "real" matrix representation.
        \includegraphics[width=15cm, height=15cm, keepaspectratio=true]{../sympy/delannoy/delannoy-standard-handle-negatives-centered-colouring-127-rows-multiples-of-5-partitioning-triangle.pdf}
    }

    % this 'particular' line is necessary to use `displaymath' environment
    % into the caption environment, togheter with the inclusion of 
    % `caption' package. See here for more explanation:
    % http://stackoverflow.com/questions/2716227/adding-an-equation-or-formula-to-a-figure-caption-in-latex
    \captionsetup{singlelinecheck=off}
    \caption[$\mathcal{D}_{\stackrel{\circ}{\equiv_{5}}}$]{ $\mathcal{D}_{\stackrel{\circ}{\equiv_{5}}}$ }

    \label{fig:delannoy-standard-handle-negatives-centered-colouring-127-rows-multiples-of-5-partitioning-triangle}

\end{figure}

\end{frame}
\begin{frame}{$\mathcal{D}_{\stackrel{\circ}{\equiv_{7}}}$ where 
    $\textcolor{blue}{[0]_{7}},\textcolor{orange}{\lbrace [1],[2],[3],[4],[5],[6] \rbrace_{7}}$}

    \input{../sympy/delannoy/delannoy-standard-handle-negatives-centered-colouring-127-rows-multiples-of-7-partitioning-include-figure}
\end{frame}

\subsection{Observations}

\begin{frame}{Observations}
    For increasing values of prime $p$:
    \begin{itemize} 
        \item $\mathcal{P}_{\stackrel{\circ}{\equiv_{p}}}$ and
            $\mathcal{C}_{\stackrel{\circ}{\equiv_{p}}}$ have a \emph{recursive structure}, 
            like a fractal;
        \item $\mathcal{D}_{\stackrel{\circ}{\equiv_{p}}}$ has a common pattern but
            not completly recursive;
        \item $\mathcal{F}_{\stackrel{\circ}{\equiv_{p}}}$ and
            $\mathcal{M}_{\stackrel{\circ}{\equiv_{p}}}$ have a pattern but it seems
            to be augmented for each increase of $p$. 
    \end{itemize} 
    By \emph{recursive structure} we mean that $\mathcal{R}^{(\alpha+1)}$ can
    be built using combinations of  $\mathcal{R}^{(\alpha)}$ only.

    \pause
    If $p$ isn't a prime, colouring gets ``noise'' and it's difficult
    to recognize the underlying pattern.

    \pause
    $\mathcal{D}_{\stackrel{\circ}{\equiv_{2}}}$ isn't very interesting since
    $d_{nk}\in\mathcal{D}_{\stackrel{\circ}{\equiv_{2}}}\rightarrow d_{nk}\equiv_{2}1$

    \pause
    Considering the inverse array $\mathcal{R}^{-1}$ of $\mathcal{R}$ yield 
    interesting colourings: in order to distinguish negative entries, darker
    and lighter colour variants are introduced.
\end{frame}

\begin{frame}{Make Titles Informative.}

  You can create overlays\dots
  \begin{itemize}
  \item using the \texttt{pause} command:
    \begin{itemize}
    \item
      First item.
      \pause
    \item    
      Second item.
    \end{itemize}
  \item
    using overlay specifications:
    \begin{itemize}
    \item<3->
      First item.
    \item<4->
      Second item.
    \end{itemize}
  \item
    using the general \texttt{uncover} command:
    \begin{itemize}
      \uncover<5->{\item
        First item.}
      \uncover<6->{\item
        Second item.}
    \end{itemize}
  \end{itemize}
\end{frame}




\section{Our Contribution}

\subsection{Main Results}

\begin{frame}{Make Titles Informative.}
\end{frame}

\begin{frame}{Make Titles Informative.}
\end{frame}

\begin{frame}{Make Titles Informative.}
\end{frame}


\subsection{Basic Ideas for Proofs/Implementation}

\begin{frame}{Make Titles Informative.}
\end{frame}

\begin{frame}{Make Titles Informative.}
\end{frame}

\begin{frame}{Make Titles Informative.}
\end{frame}



\section*{Summary}

\begin{frame}{Summary}

  % Keep the summary *very short*.
  \begin{itemize}
  \item
    The \alert{first main message} of your talk in one or two lines.
  \item
    The \alert{second main message} of your talk in one or two lines.
  \item
    Perhaps a \alert{third message}, but not more than that.
  \end{itemize}
  
  % The following outlook is optional.
  \vskip0pt plus.5fill
  \begin{itemize}
  \item
    Outlook
    \begin{itemize}
    \item
      Something you haven't solved.
    \item
      Something else you haven't solved.
    \end{itemize}
  \end{itemize}
\end{frame}



% All of the following is optional and typically not needed. 
\appendix
\section<presentation>*{\appendixname}
\subsection<presentation>*{For Further Reading}

\begin{frame}[allowframebreaks]
  \frametitle<presentation>{For Further Reading}
    
  \begin{thebibliography}{10}
    
  \beamertemplatebookbibitems
  % Start with overview books.

  \bibitem{Author1990}
    A.~Author.
    \newblock {\em Handbook of Everything}.
    \newblock Some Press, 1990.
 
    
  \beamertemplatearticlebibitems
  % Followed by interesting articles. Keep the list short. 

  \bibitem{Someone2000}
    S.~Someone.
    \newblock On this and that.
    \newblock {\em Journal of This and That}, 2(1):50--100,
    2000.
  \end{thebibliography}
\end{frame}

\end{document}


